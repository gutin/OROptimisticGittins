\section{Computational Experiments} \label{sec:experiments}
Our goal is to benchmark Optimistic Gittins Indices (OGI) against state-of-the-art Bayesian algorithms. Specifically, we compare ourselves against Thomson Sampling, Bayes UCB and IDS. Each of these algorithms has in turn been shown to substantially dominate other extant schemes. Our experimental setup closely follows that of \cite{russo2014learning,kaufmann2012bayesian} and \cite{chapelle2011empirical}. 
%The only difference in our paper is we will randomize arm parameters rather than set them to specific constants. This is for consistency and so that we focus on evaluating algorithms purely on their Bayesian performance. 
The experiment from \cite{kaufmann2012bayesian} is deferred to Appendix~\ref{exp:bayes_ucb} because it is brief and sends a similar message to the rest of this section. We conclude with a novel experiment to test the problem with multiple simultaneous arm pulls.

For the majority of experiments, we configure the OGI algorithm with $K =1$ to keep the computational burden under control {(there we simply use the Newton-Raphson method to compute the index directly)}. In one experiment, included for completeness, we test OGI with $K = 3$ and $K=\infty$, where the latter is equivalent to using Gittins indices. {There we use direct dynamic programming for the $K=3$ case, and rely on an approximation due to \cite{powell2012optimal} for the $K=\infty$ case.} The purpose of those experiments is to show the (limited) value of a higher lookahead in the OGI algorithm. 

We use a common discount factor schedule in all experiments setting $\gamma_t = 1 - 1/(100 + t)$. The choice of $\alpha = 100$ is second order; our conclusions remain unchanged, and actually appear to improve in an absolute sense with other choices of $\alpha$. In addition, in one experiment we examine the regret of OGI relative to its competitors up to a horizon of $10^6$ epochs, so that this choice of $\alpha$ does not represent an attempt to tune the performance of OGI for a specific time horizon. 



\subsection{Smaller scale experiments with IDS}

This section considers a series of smaller scale experiments (10 arms, 1000 time periods) drawn from the paper introducing the IDS algorithm, \citep{russo2014learning}. A major consideration in running these experiments is that the CPU time required to execute IDS, the closest competitor, based on the current suggested implementation is orders of magnitudes greater than that of the index schemes or Thompson Sampling. The main bottleneck is that IDS uses numerical integration, requiring the calculation of a CDF over, at least, hundreds of iterations. By contrast, the version of OGI with $K=1$ uses 10 iterations of the Newton-Raphson method. 

\paragraph{Gaussian}We replicate the Gaussian experiments from \cite{russo2014learning}. In the first experiment (Table~\ref{table:gaussian_experiment1}), the arms generate Gaussian rewards  $X_{i,t} \sim \mathcal{N}(\theta_i, 1)$ where each $\theta_i$ is independently drawn from a standard Gaussian distribution. {That is, the prior $q$ on each arm's reward is a stand Gaussian prior.} We simulate 1,000 independent trials with 10 arms and 1,000 time periods. The implementation of OGI in this experiment uses $K = 1$. It is difficult to compute exact Gittins indices in this setting, but a classical approximation for Gaussian bandits does exist; see \cite{powell2012optimal}, Chapter 6.1.3. We term the use of that approximation `OGI($\infty$) Approx'. {It is shown in \citep{powell2012optimal} that the Gittins index here can be expressed in terms of a univariate function, which cannot be computed analytically but can be approximated reasonably well with a piecewise closed-form function. As mentioned in that book, the approximation is more accurate for smaller values of the posterior variance; that is, as we play an arm more times, we expect its Gittins index approximation to improve.}  

In addition to regret, we  show the average CPU time taken, in seconds, to execute each trial.
%We also evaluate a policy (labeled `OGI Approx' in the table) that computes a particular closed-form approximation to the Gittins Index given in Chapter 6.1.3 of Powell and Ryzhov \cite{powell2012optimal}. 

\begin{table}[h!]
	\centering
	\begin{tabular}{cccccc} \toprule
		\textbf{Algorithm}  & \textbf{OGI(1)} & \textbf{OGI($\infty$) Approx.} & \textbf{IDS} & \textbf{TS} & \textbf{Bayes UCB}\\ \midrule
		Mean   & 49.19 & 47.64  &  55.83 & 67.40 & 60.30  \\ 
		Standard error  & 1.61 & 1.6 & 2.08 & 1.5 & 1.43 \\ 
		25\%  & 17.49 & 16.88  & 18.61 & 37.46 & 31.41 \\
		50\%   & 41.72 & 40.99 & 40.79 & 63.06 & 57.71 \\ 
		75\%  & 73.24 & 72.26 & 78.76 & 94.52 & 86.40 \\ 
		CPU time (s) & 0.02 & 0.01 & 11.18 & 0.01 & 0.02 \\
		\bottomrule
	\end{tabular}
	\caption[Table caption text]{Gaussian experiment. OGI(1) denotes OGI with $K =1$, while OGI Approx. uses the approximation to the Gaussian Gittins Index from \cite{powell2012optimal}.}
	\label{table:gaussian_experiment1}
\end{table}

The key feature of the results here is that OGI offers an approximately 10\% improvement in regret over its nearest competitor IDS, and larger improvements (20 and 40 \% respectively) over Bayes UCB and Thompson Sampling. The best performing policy is OGI with the specialized Gaussian approximation since it gives a closer approximation to the Gittins Index. At the same time, OGI is essentially as fast as Thompson sampling, and three orders of magnitude faster than its nearest competitor (in terms of regret). 


\paragraph{Bernoulli}
We next replicate the Beta-Bernoulli experiments from \cite{russo2014learning}.
In this experiment regret is simulated over 1,000 periods, with 10 arms each having a uniformly distributed Bernoulli parameter. 
{
We compute approximations to the exact Gittins index, i.e. OGI($\infty$), via value iteration \footnote{
We use the termination condition in Proposition 2.2.1 of \cite{bertsekas2011dynamic}. That is, we iterate the value iteration operator $k$ times so that $\bar{c}_k - \underline{c}_k$ (as in Proposition 2.2.1) is at most some predetermined fixed tolerance.
}.
}
We simulate 1,000 independent trials and Table~\ref{table:bernoulli_experiment1} summarizes the results.

\begin{table}[h!]
	\centering
	\begin{tabular}{ccccccc} \toprule
		\textbf{Algorithm} & \textbf{OGI(1)} & \textbf{OGI(3)} &  \textbf{OGI($\infty$)} & \textbf{IDS} & \textbf{TS} & \textbf{Bayes UCB}  \\ \midrule
		Mean &  18.12 & 18.00 & 17.52 & 19.03 & 27.39 & 22.71 \\ 
		Standard error & 0.65 & 0.64 &  0.68 & 0.67 & 0.57 & 0.56 \\ 
		25\% & 6.26 & 5.60 & 4.45 & 5.85 & 14.62 & 10.09 \\
		50\% & 15.08 & 14.84 &12.06 & 14.06 & 23.53 & 18.52 \\
		75\% & 27.63 & 27.74 & 24.93 & 26.48 & 36.11 & 30.58 \\
		CPU time (s) & 0.19 & 0.89 & (?) hours & 8.11 & 0.01 & 0.05  \\ \bottomrule
	\end{tabular}
	\caption[Table caption text]{Bernoulli experiment. OGI($K$) denotes the OGI algorithm with a $K$ step approximation and tuning parameter $\alpha = 100$. OGI($\infty$) is the algorithm that uses Gittins Indices.}
	\label{table:bernoulli_experiment1}
\end{table}
Each version of OGI outperforms other algorithms and the one that uses exact Gittins Indices shows the lowest mean regret.
Perhaps, unsurprisingly, when OGI  looks ahead 3 steps (or when the lookahead is not limited), it performs better than with a single step. It is however apparent that in each of these cases the improvement over simply setting $K=1$ is marginal. Indeed, looking ahead 1 step is a reasonably close approximation to the Gittins Index in the Bernoulli problem. In the Appendix, we report the approximation error in approximating the Gittins index for various choice of $K$. When using an optimistic 1 step approximation, the error is around 15\% and if $K$ is increased to 3, the error drops to around 4\% (see Tables~\ref{table:ogi_table_for_gamma_9} and \ref{table:ogi_table_for_gamma_95} in the Appendix). 

As an aside, we note that the regret we computed for the IDS algorithm is slightly different from that reported by \cite{russo2014learning}. Specifically, we obtain slightly lower regret for IDS than they report in the Gaussian experiments, and slightly higher values for the Beta-Bernoulli case; we include a link to the code we used to implement the algorithms\footnote{\url{https://github.com/gutin/FastGittins}} as a reference.  

\subsection{Large scale experiment} \label{exp:ts_sampling_experiment}
This experiment replicates a large scale synthetic experiment in \cite{chapelle2011empirical}.
{Here the arms' rewards are Bernoulli and their means are independently sampled from a uniform prior. Every algorithm that we will test assumes this same prior over the arms' mean rewards.}
The key feature here is that we simulate a  longer horizon of $T = 10^6$ and include a large number of arms, particularly we let $A = 100$. This is an order of magnitude greater than in the majority of synthetic bandit experiments we are aware of.
Our goal is to see how the algorithms scale both computationally and in terms of performance.
Such a setup is practically relevant because in applications such as e-commerce or online advertising, the problems of interest are typically modeled with many arms relative to the horizon, where each arm could represent a product or ad.

Because all the methods we test in our numerical experiments are regret optimal, any relative difference in regret must shrink after a sufficiently large number of time periods. The length of time for this `burn in' period intuitively depends on the number of arms in the problem.
In particular, we can think of the horizon as giving us a rough budget on the number of trials per arm via the ratio $T/A$.
The idea is that with more trials per arm we should expect a smaller relative difference between the algorithms {(and indeed the theoretical guarantees for the algorithms require this to happen)}. 
We will see that even when the ratio $T/A$ and $A$ itself are large, there is a substantial difference between OGI and the competing benchmarks in both a relative and absolute sense.

As this experiment requires an order of magnitude more iterations than the earlier ones, we are only able to simulate the fastest algorithms, which are OGI with $K=1$, Thompson Sampling and Bayes UCB. 
It was not possible to include IDS because its performance is hindered by the fact that each arm pull decision requires time that is quadratic in the number of arms to compute. 

We show the algorithms' regret averaged over 5,000 trials in Figure~\ref{fig:chapelle_and_li} and Table~\ref{table:additional_cli_table}.
\begin{figure}[h!]
	\centering
	%% Creator: Matplotlib, PGF backend
%%
%% To include the figure in your LaTeX document, write
%%   \input{<filename>.pgf}
%%
%% Make sure the required packages are loaded in your preamble
%%   \usepackage{pgf}
%%
%% Figures using additional raster images can only be included by \input if
%% they are in the same directory as the main LaTeX file. For loading figures
%% from other directories you can use the `import` package
%%   \usepackage{import}
%% and then include the figures with
%%   \import{<path to file>}{<filename>.pgf}
%%
%% Matplotlib used the following preamble
%%   \usepackage[utf8x]{inputenc}
%%   \usepackage[T1]{fontenc}
%%
\begingroup%
\makeatletter%
\begin{pgfpicture}%
\pgfpathrectangle{\pgfpointorigin}{\pgfqpoint{5.082555in}{3.141192in}}%
\pgfusepath{use as bounding box, clip}%
\begin{pgfscope}%
\pgfsetbuttcap%
\pgfsetmiterjoin%
\definecolor{currentfill}{rgb}{1.000000,1.000000,1.000000}%
\pgfsetfillcolor{currentfill}%
\pgfsetlinewidth{0.000000pt}%
\definecolor{currentstroke}{rgb}{1.000000,1.000000,1.000000}%
\pgfsetstrokecolor{currentstroke}%
\pgfsetdash{}{0pt}%
\pgfpathmoveto{\pgfqpoint{0.000000in}{0.000000in}}%
\pgfpathlineto{\pgfqpoint{5.082555in}{0.000000in}}%
\pgfpathlineto{\pgfqpoint{5.082555in}{3.141192in}}%
\pgfpathlineto{\pgfqpoint{0.000000in}{3.141192in}}%
\pgfpathclose%
\pgfusepath{fill}%
\end{pgfscope}%
\begin{pgfscope}%
\pgfsetbuttcap%
\pgfsetmiterjoin%
\definecolor{currentfill}{rgb}{1.000000,1.000000,1.000000}%
\pgfsetfillcolor{currentfill}%
\pgfsetlinewidth{0.000000pt}%
\definecolor{currentstroke}{rgb}{0.000000,0.000000,0.000000}%
\pgfsetstrokecolor{currentstroke}%
\pgfsetstrokeopacity{0.000000}%
\pgfsetdash{}{0pt}%
\pgfpathmoveto{\pgfqpoint{0.635319in}{0.392649in}}%
\pgfpathlineto{\pgfqpoint{4.574300in}{0.392649in}}%
\pgfpathlineto{\pgfqpoint{4.574300in}{2.764249in}}%
\pgfpathlineto{\pgfqpoint{0.635319in}{2.764249in}}%
\pgfpathclose%
\pgfusepath{fill}%
\end{pgfscope}%
\begin{pgfscope}%
\pgfsetbuttcap%
\pgfsetroundjoin%
\definecolor{currentfill}{rgb}{0.000000,0.000000,0.000000}%
\pgfsetfillcolor{currentfill}%
\pgfsetlinewidth{0.803000pt}%
\definecolor{currentstroke}{rgb}{0.000000,0.000000,0.000000}%
\pgfsetstrokecolor{currentstroke}%
\pgfsetdash{}{0pt}%
\pgfsys@defobject{currentmarker}{\pgfqpoint{0.000000in}{-0.048611in}}{\pgfqpoint{0.000000in}{0.000000in}}{%
\pgfpathmoveto{\pgfqpoint{0.000000in}{0.000000in}}%
\pgfpathlineto{\pgfqpoint{0.000000in}{-0.048611in}}%
\pgfusepath{stroke,fill}%
}%
\begin{pgfscope}%
\pgfsys@transformshift{0.814364in}{0.392649in}%
\pgfsys@useobject{currentmarker}{}%
\end{pgfscope}%
\end{pgfscope}%
\begin{pgfscope}%
\pgftext[x=0.814364in,y=0.295427in,,top]{\rmfamily\fontsize{8.000000}{9.600000}\selectfont \(\displaystyle 10^{4}\)}%
\end{pgfscope}%
\begin{pgfscope}%
\pgfsetbuttcap%
\pgfsetroundjoin%
\definecolor{currentfill}{rgb}{0.000000,0.000000,0.000000}%
\pgfsetfillcolor{currentfill}%
\pgfsetlinewidth{0.803000pt}%
\definecolor{currentstroke}{rgb}{0.000000,0.000000,0.000000}%
\pgfsetstrokecolor{currentstroke}%
\pgfsetdash{}{0pt}%
\pgfsys@defobject{currentmarker}{\pgfqpoint{0.000000in}{-0.048611in}}{\pgfqpoint{0.000000in}{0.000000in}}{%
\pgfpathmoveto{\pgfqpoint{0.000000in}{0.000000in}}%
\pgfpathlineto{\pgfqpoint{0.000000in}{-0.048611in}}%
\pgfusepath{stroke,fill}%
}%
\begin{pgfscope}%
\pgfsys@transformshift{2.608725in}{0.392649in}%
\pgfsys@useobject{currentmarker}{}%
\end{pgfscope}%
\end{pgfscope}%
\begin{pgfscope}%
\pgftext[x=2.608725in,y=0.295427in,,top]{\rmfamily\fontsize{8.000000}{9.600000}\selectfont \(\displaystyle 10^{5}\)}%
\end{pgfscope}%
\begin{pgfscope}%
\pgfsetbuttcap%
\pgfsetroundjoin%
\definecolor{currentfill}{rgb}{0.000000,0.000000,0.000000}%
\pgfsetfillcolor{currentfill}%
\pgfsetlinewidth{0.803000pt}%
\definecolor{currentstroke}{rgb}{0.000000,0.000000,0.000000}%
\pgfsetstrokecolor{currentstroke}%
\pgfsetdash{}{0pt}%
\pgfsys@defobject{currentmarker}{\pgfqpoint{0.000000in}{-0.048611in}}{\pgfqpoint{0.000000in}{0.000000in}}{%
\pgfpathmoveto{\pgfqpoint{0.000000in}{0.000000in}}%
\pgfpathlineto{\pgfqpoint{0.000000in}{-0.048611in}}%
\pgfusepath{stroke,fill}%
}%
\begin{pgfscope}%
\pgfsys@transformshift{4.403087in}{0.392649in}%
\pgfsys@useobject{currentmarker}{}%
\end{pgfscope}%
\end{pgfscope}%
\begin{pgfscope}%
\pgftext[x=4.403087in,y=0.295427in,,top]{\rmfamily\fontsize{8.000000}{9.600000}\selectfont \(\displaystyle 10^{6}\)}%
\end{pgfscope}%
\begin{pgfscope}%
\pgfsetbuttcap%
\pgfsetroundjoin%
\definecolor{currentfill}{rgb}{0.000000,0.000000,0.000000}%
\pgfsetfillcolor{currentfill}%
\pgfsetlinewidth{0.602250pt}%
\definecolor{currentstroke}{rgb}{0.000000,0.000000,0.000000}%
\pgfsetstrokecolor{currentstroke}%
\pgfsetdash{}{0pt}%
\pgfsys@defobject{currentmarker}{\pgfqpoint{0.000000in}{-0.027778in}}{\pgfqpoint{0.000000in}{0.000000in}}{%
\pgfpathmoveto{\pgfqpoint{0.000000in}{0.000000in}}%
\pgfpathlineto{\pgfqpoint{0.000000in}{-0.027778in}}%
\pgfusepath{stroke,fill}%
}%
\begin{pgfscope}%
\pgfsys@transformshift{0.640472in}{0.392649in}%
\pgfsys@useobject{currentmarker}{}%
\end{pgfscope}%
\end{pgfscope}%
\begin{pgfscope}%
\pgfsetbuttcap%
\pgfsetroundjoin%
\definecolor{currentfill}{rgb}{0.000000,0.000000,0.000000}%
\pgfsetfillcolor{currentfill}%
\pgfsetlinewidth{0.602250pt}%
\definecolor{currentstroke}{rgb}{0.000000,0.000000,0.000000}%
\pgfsetstrokecolor{currentstroke}%
\pgfsetdash{}{0pt}%
\pgfsys@defobject{currentmarker}{\pgfqpoint{0.000000in}{-0.027778in}}{\pgfqpoint{0.000000in}{0.000000in}}{%
\pgfpathmoveto{\pgfqpoint{0.000000in}{0.000000in}}%
\pgfpathlineto{\pgfqpoint{0.000000in}{-0.027778in}}%
\pgfusepath{stroke,fill}%
}%
\begin{pgfscope}%
\pgfsys@transformshift{0.732258in}{0.392649in}%
\pgfsys@useobject{currentmarker}{}%
\end{pgfscope}%
\end{pgfscope}%
\begin{pgfscope}%
\pgfsetbuttcap%
\pgfsetroundjoin%
\definecolor{currentfill}{rgb}{0.000000,0.000000,0.000000}%
\pgfsetfillcolor{currentfill}%
\pgfsetlinewidth{0.602250pt}%
\definecolor{currentstroke}{rgb}{0.000000,0.000000,0.000000}%
\pgfsetstrokecolor{currentstroke}%
\pgfsetdash{}{0pt}%
\pgfsys@defobject{currentmarker}{\pgfqpoint{0.000000in}{-0.027778in}}{\pgfqpoint{0.000000in}{0.000000in}}{%
\pgfpathmoveto{\pgfqpoint{0.000000in}{0.000000in}}%
\pgfpathlineto{\pgfqpoint{0.000000in}{-0.027778in}}%
\pgfusepath{stroke,fill}%
}%
\begin{pgfscope}%
\pgfsys@transformshift{1.354521in}{0.392649in}%
\pgfsys@useobject{currentmarker}{}%
\end{pgfscope}%
\end{pgfscope}%
\begin{pgfscope}%
\pgfsetbuttcap%
\pgfsetroundjoin%
\definecolor{currentfill}{rgb}{0.000000,0.000000,0.000000}%
\pgfsetfillcolor{currentfill}%
\pgfsetlinewidth{0.602250pt}%
\definecolor{currentstroke}{rgb}{0.000000,0.000000,0.000000}%
\pgfsetstrokecolor{currentstroke}%
\pgfsetdash{}{0pt}%
\pgfsys@defobject{currentmarker}{\pgfqpoint{0.000000in}{-0.027778in}}{\pgfqpoint{0.000000in}{0.000000in}}{%
\pgfpathmoveto{\pgfqpoint{0.000000in}{0.000000in}}%
\pgfpathlineto{\pgfqpoint{0.000000in}{-0.027778in}}%
\pgfusepath{stroke,fill}%
}%
\begin{pgfscope}%
\pgfsys@transformshift{1.670492in}{0.392649in}%
\pgfsys@useobject{currentmarker}{}%
\end{pgfscope}%
\end{pgfscope}%
\begin{pgfscope}%
\pgfsetbuttcap%
\pgfsetroundjoin%
\definecolor{currentfill}{rgb}{0.000000,0.000000,0.000000}%
\pgfsetfillcolor{currentfill}%
\pgfsetlinewidth{0.602250pt}%
\definecolor{currentstroke}{rgb}{0.000000,0.000000,0.000000}%
\pgfsetstrokecolor{currentstroke}%
\pgfsetdash{}{0pt}%
\pgfsys@defobject{currentmarker}{\pgfqpoint{0.000000in}{-0.027778in}}{\pgfqpoint{0.000000in}{0.000000in}}{%
\pgfpathmoveto{\pgfqpoint{0.000000in}{0.000000in}}%
\pgfpathlineto{\pgfqpoint{0.000000in}{-0.027778in}}%
\pgfusepath{stroke,fill}%
}%
\begin{pgfscope}%
\pgfsys@transformshift{1.894677in}{0.392649in}%
\pgfsys@useobject{currentmarker}{}%
\end{pgfscope}%
\end{pgfscope}%
\begin{pgfscope}%
\pgfsetbuttcap%
\pgfsetroundjoin%
\definecolor{currentfill}{rgb}{0.000000,0.000000,0.000000}%
\pgfsetfillcolor{currentfill}%
\pgfsetlinewidth{0.602250pt}%
\definecolor{currentstroke}{rgb}{0.000000,0.000000,0.000000}%
\pgfsetstrokecolor{currentstroke}%
\pgfsetdash{}{0pt}%
\pgfsys@defobject{currentmarker}{\pgfqpoint{0.000000in}{-0.027778in}}{\pgfqpoint{0.000000in}{0.000000in}}{%
\pgfpathmoveto{\pgfqpoint{0.000000in}{0.000000in}}%
\pgfpathlineto{\pgfqpoint{0.000000in}{-0.027778in}}%
\pgfusepath{stroke,fill}%
}%
\begin{pgfscope}%
\pgfsys@transformshift{2.068569in}{0.392649in}%
\pgfsys@useobject{currentmarker}{}%
\end{pgfscope}%
\end{pgfscope}%
\begin{pgfscope}%
\pgfsetbuttcap%
\pgfsetroundjoin%
\definecolor{currentfill}{rgb}{0.000000,0.000000,0.000000}%
\pgfsetfillcolor{currentfill}%
\pgfsetlinewidth{0.602250pt}%
\definecolor{currentstroke}{rgb}{0.000000,0.000000,0.000000}%
\pgfsetstrokecolor{currentstroke}%
\pgfsetdash{}{0pt}%
\pgfsys@defobject{currentmarker}{\pgfqpoint{0.000000in}{-0.027778in}}{\pgfqpoint{0.000000in}{0.000000in}}{%
\pgfpathmoveto{\pgfqpoint{0.000000in}{0.000000in}}%
\pgfpathlineto{\pgfqpoint{0.000000in}{-0.027778in}}%
\pgfusepath{stroke,fill}%
}%
\begin{pgfscope}%
\pgfsys@transformshift{2.210649in}{0.392649in}%
\pgfsys@useobject{currentmarker}{}%
\end{pgfscope}%
\end{pgfscope}%
\begin{pgfscope}%
\pgfsetbuttcap%
\pgfsetroundjoin%
\definecolor{currentfill}{rgb}{0.000000,0.000000,0.000000}%
\pgfsetfillcolor{currentfill}%
\pgfsetlinewidth{0.602250pt}%
\definecolor{currentstroke}{rgb}{0.000000,0.000000,0.000000}%
\pgfsetstrokecolor{currentstroke}%
\pgfsetdash{}{0pt}%
\pgfsys@defobject{currentmarker}{\pgfqpoint{0.000000in}{-0.027778in}}{\pgfqpoint{0.000000in}{0.000000in}}{%
\pgfpathmoveto{\pgfqpoint{0.000000in}{0.000000in}}%
\pgfpathlineto{\pgfqpoint{0.000000in}{-0.027778in}}%
\pgfusepath{stroke,fill}%
}%
\begin{pgfscope}%
\pgfsys@transformshift{2.330775in}{0.392649in}%
\pgfsys@useobject{currentmarker}{}%
\end{pgfscope}%
\end{pgfscope}%
\begin{pgfscope}%
\pgfsetbuttcap%
\pgfsetroundjoin%
\definecolor{currentfill}{rgb}{0.000000,0.000000,0.000000}%
\pgfsetfillcolor{currentfill}%
\pgfsetlinewidth{0.602250pt}%
\definecolor{currentstroke}{rgb}{0.000000,0.000000,0.000000}%
\pgfsetstrokecolor{currentstroke}%
\pgfsetdash{}{0pt}%
\pgfsys@defobject{currentmarker}{\pgfqpoint{0.000000in}{-0.027778in}}{\pgfqpoint{0.000000in}{0.000000in}}{%
\pgfpathmoveto{\pgfqpoint{0.000000in}{0.000000in}}%
\pgfpathlineto{\pgfqpoint{0.000000in}{-0.027778in}}%
\pgfusepath{stroke,fill}%
}%
\begin{pgfscope}%
\pgfsys@transformshift{2.434834in}{0.392649in}%
\pgfsys@useobject{currentmarker}{}%
\end{pgfscope}%
\end{pgfscope}%
\begin{pgfscope}%
\pgfsetbuttcap%
\pgfsetroundjoin%
\definecolor{currentfill}{rgb}{0.000000,0.000000,0.000000}%
\pgfsetfillcolor{currentfill}%
\pgfsetlinewidth{0.602250pt}%
\definecolor{currentstroke}{rgb}{0.000000,0.000000,0.000000}%
\pgfsetstrokecolor{currentstroke}%
\pgfsetdash{}{0pt}%
\pgfsys@defobject{currentmarker}{\pgfqpoint{0.000000in}{-0.027778in}}{\pgfqpoint{0.000000in}{0.000000in}}{%
\pgfpathmoveto{\pgfqpoint{0.000000in}{0.000000in}}%
\pgfpathlineto{\pgfqpoint{0.000000in}{-0.027778in}}%
\pgfusepath{stroke,fill}%
}%
\begin{pgfscope}%
\pgfsys@transformshift{2.526620in}{0.392649in}%
\pgfsys@useobject{currentmarker}{}%
\end{pgfscope}%
\end{pgfscope}%
\begin{pgfscope}%
\pgfsetbuttcap%
\pgfsetroundjoin%
\definecolor{currentfill}{rgb}{0.000000,0.000000,0.000000}%
\pgfsetfillcolor{currentfill}%
\pgfsetlinewidth{0.602250pt}%
\definecolor{currentstroke}{rgb}{0.000000,0.000000,0.000000}%
\pgfsetstrokecolor{currentstroke}%
\pgfsetdash{}{0pt}%
\pgfsys@defobject{currentmarker}{\pgfqpoint{0.000000in}{-0.027778in}}{\pgfqpoint{0.000000in}{0.000000in}}{%
\pgfpathmoveto{\pgfqpoint{0.000000in}{0.000000in}}%
\pgfpathlineto{\pgfqpoint{0.000000in}{-0.027778in}}%
\pgfusepath{stroke,fill}%
}%
\begin{pgfscope}%
\pgfsys@transformshift{3.148882in}{0.392649in}%
\pgfsys@useobject{currentmarker}{}%
\end{pgfscope}%
\end{pgfscope}%
\begin{pgfscope}%
\pgfsetbuttcap%
\pgfsetroundjoin%
\definecolor{currentfill}{rgb}{0.000000,0.000000,0.000000}%
\pgfsetfillcolor{currentfill}%
\pgfsetlinewidth{0.602250pt}%
\definecolor{currentstroke}{rgb}{0.000000,0.000000,0.000000}%
\pgfsetstrokecolor{currentstroke}%
\pgfsetdash{}{0pt}%
\pgfsys@defobject{currentmarker}{\pgfqpoint{0.000000in}{-0.027778in}}{\pgfqpoint{0.000000in}{0.000000in}}{%
\pgfpathmoveto{\pgfqpoint{0.000000in}{0.000000in}}%
\pgfpathlineto{\pgfqpoint{0.000000in}{-0.027778in}}%
\pgfusepath{stroke,fill}%
}%
\begin{pgfscope}%
\pgfsys@transformshift{3.464853in}{0.392649in}%
\pgfsys@useobject{currentmarker}{}%
\end{pgfscope}%
\end{pgfscope}%
\begin{pgfscope}%
\pgfsetbuttcap%
\pgfsetroundjoin%
\definecolor{currentfill}{rgb}{0.000000,0.000000,0.000000}%
\pgfsetfillcolor{currentfill}%
\pgfsetlinewidth{0.602250pt}%
\definecolor{currentstroke}{rgb}{0.000000,0.000000,0.000000}%
\pgfsetstrokecolor{currentstroke}%
\pgfsetdash{}{0pt}%
\pgfsys@defobject{currentmarker}{\pgfqpoint{0.000000in}{-0.027778in}}{\pgfqpoint{0.000000in}{0.000000in}}{%
\pgfpathmoveto{\pgfqpoint{0.000000in}{0.000000in}}%
\pgfpathlineto{\pgfqpoint{0.000000in}{-0.027778in}}%
\pgfusepath{stroke,fill}%
}%
\begin{pgfscope}%
\pgfsys@transformshift{3.689039in}{0.392649in}%
\pgfsys@useobject{currentmarker}{}%
\end{pgfscope}%
\end{pgfscope}%
\begin{pgfscope}%
\pgfsetbuttcap%
\pgfsetroundjoin%
\definecolor{currentfill}{rgb}{0.000000,0.000000,0.000000}%
\pgfsetfillcolor{currentfill}%
\pgfsetlinewidth{0.602250pt}%
\definecolor{currentstroke}{rgb}{0.000000,0.000000,0.000000}%
\pgfsetstrokecolor{currentstroke}%
\pgfsetdash{}{0pt}%
\pgfsys@defobject{currentmarker}{\pgfqpoint{0.000000in}{-0.027778in}}{\pgfqpoint{0.000000in}{0.000000in}}{%
\pgfpathmoveto{\pgfqpoint{0.000000in}{0.000000in}}%
\pgfpathlineto{\pgfqpoint{0.000000in}{-0.027778in}}%
\pgfusepath{stroke,fill}%
}%
\begin{pgfscope}%
\pgfsys@transformshift{3.862930in}{0.392649in}%
\pgfsys@useobject{currentmarker}{}%
\end{pgfscope}%
\end{pgfscope}%
\begin{pgfscope}%
\pgfsetbuttcap%
\pgfsetroundjoin%
\definecolor{currentfill}{rgb}{0.000000,0.000000,0.000000}%
\pgfsetfillcolor{currentfill}%
\pgfsetlinewidth{0.602250pt}%
\definecolor{currentstroke}{rgb}{0.000000,0.000000,0.000000}%
\pgfsetstrokecolor{currentstroke}%
\pgfsetdash{}{0pt}%
\pgfsys@defobject{currentmarker}{\pgfqpoint{0.000000in}{-0.027778in}}{\pgfqpoint{0.000000in}{0.000000in}}{%
\pgfpathmoveto{\pgfqpoint{0.000000in}{0.000000in}}%
\pgfpathlineto{\pgfqpoint{0.000000in}{-0.027778in}}%
\pgfusepath{stroke,fill}%
}%
\begin{pgfscope}%
\pgfsys@transformshift{4.005010in}{0.392649in}%
\pgfsys@useobject{currentmarker}{}%
\end{pgfscope}%
\end{pgfscope}%
\begin{pgfscope}%
\pgfsetbuttcap%
\pgfsetroundjoin%
\definecolor{currentfill}{rgb}{0.000000,0.000000,0.000000}%
\pgfsetfillcolor{currentfill}%
\pgfsetlinewidth{0.602250pt}%
\definecolor{currentstroke}{rgb}{0.000000,0.000000,0.000000}%
\pgfsetstrokecolor{currentstroke}%
\pgfsetdash{}{0pt}%
\pgfsys@defobject{currentmarker}{\pgfqpoint{0.000000in}{-0.027778in}}{\pgfqpoint{0.000000in}{0.000000in}}{%
\pgfpathmoveto{\pgfqpoint{0.000000in}{0.000000in}}%
\pgfpathlineto{\pgfqpoint{0.000000in}{-0.027778in}}%
\pgfusepath{stroke,fill}%
}%
\begin{pgfscope}%
\pgfsys@transformshift{4.125137in}{0.392649in}%
\pgfsys@useobject{currentmarker}{}%
\end{pgfscope}%
\end{pgfscope}%
\begin{pgfscope}%
\pgfsetbuttcap%
\pgfsetroundjoin%
\definecolor{currentfill}{rgb}{0.000000,0.000000,0.000000}%
\pgfsetfillcolor{currentfill}%
\pgfsetlinewidth{0.602250pt}%
\definecolor{currentstroke}{rgb}{0.000000,0.000000,0.000000}%
\pgfsetstrokecolor{currentstroke}%
\pgfsetdash{}{0pt}%
\pgfsys@defobject{currentmarker}{\pgfqpoint{0.000000in}{-0.027778in}}{\pgfqpoint{0.000000in}{0.000000in}}{%
\pgfpathmoveto{\pgfqpoint{0.000000in}{0.000000in}}%
\pgfpathlineto{\pgfqpoint{0.000000in}{-0.027778in}}%
\pgfusepath{stroke,fill}%
}%
\begin{pgfscope}%
\pgfsys@transformshift{4.229195in}{0.392649in}%
\pgfsys@useobject{currentmarker}{}%
\end{pgfscope}%
\end{pgfscope}%
\begin{pgfscope}%
\pgfsetbuttcap%
\pgfsetroundjoin%
\definecolor{currentfill}{rgb}{0.000000,0.000000,0.000000}%
\pgfsetfillcolor{currentfill}%
\pgfsetlinewidth{0.602250pt}%
\definecolor{currentstroke}{rgb}{0.000000,0.000000,0.000000}%
\pgfsetstrokecolor{currentstroke}%
\pgfsetdash{}{0pt}%
\pgfsys@defobject{currentmarker}{\pgfqpoint{0.000000in}{-0.027778in}}{\pgfqpoint{0.000000in}{0.000000in}}{%
\pgfpathmoveto{\pgfqpoint{0.000000in}{0.000000in}}%
\pgfpathlineto{\pgfqpoint{0.000000in}{-0.027778in}}%
\pgfusepath{stroke,fill}%
}%
\begin{pgfscope}%
\pgfsys@transformshift{4.320982in}{0.392649in}%
\pgfsys@useobject{currentmarker}{}%
\end{pgfscope}%
\end{pgfscope}%
\begin{pgfscope}%
\pgftext[x=2.604809in,y=0.140010in,,top]{\rmfamily\fontsize{10.000000}{12.000000}\selectfont Period}%
\end{pgfscope}%
\begin{pgfscope}%
\pgfsetbuttcap%
\pgfsetroundjoin%
\definecolor{currentfill}{rgb}{0.000000,0.000000,0.000000}%
\pgfsetfillcolor{currentfill}%
\pgfsetlinewidth{0.803000pt}%
\definecolor{currentstroke}{rgb}{0.000000,0.000000,0.000000}%
\pgfsetstrokecolor{currentstroke}%
\pgfsetdash{}{0pt}%
\pgfsys@defobject{currentmarker}{\pgfqpoint{-0.048611in}{0.000000in}}{\pgfqpoint{0.000000in}{0.000000in}}{%
\pgfpathmoveto{\pgfqpoint{0.000000in}{0.000000in}}%
\pgfpathlineto{\pgfqpoint{-0.048611in}{0.000000in}}%
\pgfusepath{stroke,fill}%
}%
\begin{pgfscope}%
\pgfsys@transformshift{0.635319in}{0.747708in}%
\pgfsys@useobject{currentmarker}{}%
\end{pgfscope}%
\end{pgfscope}%
\begin{pgfscope}%
\pgftext[x=0.361011in,y=0.709446in,left,base]{\rmfamily\fontsize{8.000000}{9.600000}\selectfont \(\displaystyle 150\)}%
\end{pgfscope}%
\begin{pgfscope}%
\pgfsetbuttcap%
\pgfsetroundjoin%
\definecolor{currentfill}{rgb}{0.000000,0.000000,0.000000}%
\pgfsetfillcolor{currentfill}%
\pgfsetlinewidth{0.803000pt}%
\definecolor{currentstroke}{rgb}{0.000000,0.000000,0.000000}%
\pgfsetstrokecolor{currentstroke}%
\pgfsetdash{}{0pt}%
\pgfsys@defobject{currentmarker}{\pgfqpoint{-0.048611in}{0.000000in}}{\pgfqpoint{0.000000in}{0.000000in}}{%
\pgfpathmoveto{\pgfqpoint{0.000000in}{0.000000in}}%
\pgfpathlineto{\pgfqpoint{-0.048611in}{0.000000in}}%
\pgfusepath{stroke,fill}%
}%
\begin{pgfscope}%
\pgfsys@transformshift{0.635319in}{1.161341in}%
\pgfsys@useobject{currentmarker}{}%
\end{pgfscope}%
\end{pgfscope}%
\begin{pgfscope}%
\pgftext[x=0.361011in,y=1.123079in,left,base]{\rmfamily\fontsize{8.000000}{9.600000}\selectfont \(\displaystyle 200\)}%
\end{pgfscope}%
\begin{pgfscope}%
\pgfsetbuttcap%
\pgfsetroundjoin%
\definecolor{currentfill}{rgb}{0.000000,0.000000,0.000000}%
\pgfsetfillcolor{currentfill}%
\pgfsetlinewidth{0.803000pt}%
\definecolor{currentstroke}{rgb}{0.000000,0.000000,0.000000}%
\pgfsetstrokecolor{currentstroke}%
\pgfsetdash{}{0pt}%
\pgfsys@defobject{currentmarker}{\pgfqpoint{-0.048611in}{0.000000in}}{\pgfqpoint{0.000000in}{0.000000in}}{%
\pgfpathmoveto{\pgfqpoint{0.000000in}{0.000000in}}%
\pgfpathlineto{\pgfqpoint{-0.048611in}{0.000000in}}%
\pgfusepath{stroke,fill}%
}%
\begin{pgfscope}%
\pgfsys@transformshift{0.635319in}{1.574973in}%
\pgfsys@useobject{currentmarker}{}%
\end{pgfscope}%
\end{pgfscope}%
\begin{pgfscope}%
\pgftext[x=0.361011in,y=1.536711in,left,base]{\rmfamily\fontsize{8.000000}{9.600000}\selectfont \(\displaystyle 250\)}%
\end{pgfscope}%
\begin{pgfscope}%
\pgfsetbuttcap%
\pgfsetroundjoin%
\definecolor{currentfill}{rgb}{0.000000,0.000000,0.000000}%
\pgfsetfillcolor{currentfill}%
\pgfsetlinewidth{0.803000pt}%
\definecolor{currentstroke}{rgb}{0.000000,0.000000,0.000000}%
\pgfsetstrokecolor{currentstroke}%
\pgfsetdash{}{0pt}%
\pgfsys@defobject{currentmarker}{\pgfqpoint{-0.048611in}{0.000000in}}{\pgfqpoint{0.000000in}{0.000000in}}{%
\pgfpathmoveto{\pgfqpoint{0.000000in}{0.000000in}}%
\pgfpathlineto{\pgfqpoint{-0.048611in}{0.000000in}}%
\pgfusepath{stroke,fill}%
}%
\begin{pgfscope}%
\pgfsys@transformshift{0.635319in}{1.988606in}%
\pgfsys@useobject{currentmarker}{}%
\end{pgfscope}%
\end{pgfscope}%
\begin{pgfscope}%
\pgftext[x=0.361011in,y=1.950344in,left,base]{\rmfamily\fontsize{8.000000}{9.600000}\selectfont \(\displaystyle 300\)}%
\end{pgfscope}%
\begin{pgfscope}%
\pgfsetbuttcap%
\pgfsetroundjoin%
\definecolor{currentfill}{rgb}{0.000000,0.000000,0.000000}%
\pgfsetfillcolor{currentfill}%
\pgfsetlinewidth{0.803000pt}%
\definecolor{currentstroke}{rgb}{0.000000,0.000000,0.000000}%
\pgfsetstrokecolor{currentstroke}%
\pgfsetdash{}{0pt}%
\pgfsys@defobject{currentmarker}{\pgfqpoint{-0.048611in}{0.000000in}}{\pgfqpoint{0.000000in}{0.000000in}}{%
\pgfpathmoveto{\pgfqpoint{0.000000in}{0.000000in}}%
\pgfpathlineto{\pgfqpoint{-0.048611in}{0.000000in}}%
\pgfusepath{stroke,fill}%
}%
\begin{pgfscope}%
\pgfsys@transformshift{0.635319in}{2.402239in}%
\pgfsys@useobject{currentmarker}{}%
\end{pgfscope}%
\end{pgfscope}%
\begin{pgfscope}%
\pgftext[x=0.361011in,y=2.363976in,left,base]{\rmfamily\fontsize{8.000000}{9.600000}\selectfont \(\displaystyle 350\)}%
\end{pgfscope}%
\begin{pgfscope}%
\pgftext[x=0.305456in,y=1.578449in,,bottom,rotate=90.000000]{\rmfamily\fontsize{10.000000}{12.000000}\selectfont Avg. Bayesian Regret}%
\end{pgfscope}%
\begin{pgfscope}%
\pgfpathrectangle{\pgfqpoint{0.635319in}{0.392649in}}{\pgfqpoint{3.938980in}{2.371600in}} %
\pgfusepath{clip}%
\pgfsetrectcap%
\pgfsetroundjoin%
\pgfsetlinewidth{1.505625pt}%
\definecolor{currentstroke}{rgb}{0.121569,0.466667,0.705882}%
\pgfsetstrokecolor{currentstroke}%
\pgfsetdash{}{0pt}%
\pgfpathmoveto{\pgfqpoint{0.814364in}{0.500449in}}%
\pgfpathlineto{\pgfqpoint{1.354521in}{0.715479in}}%
\pgfpathlineto{\pgfqpoint{1.670492in}{0.841740in}}%
\pgfpathlineto{\pgfqpoint{1.894677in}{0.928736in}}%
\pgfpathlineto{\pgfqpoint{2.068569in}{1.000307in}}%
\pgfpathlineto{\pgfqpoint{2.210649in}{1.056527in}}%
\pgfpathlineto{\pgfqpoint{2.330775in}{1.106157in}}%
\pgfpathlineto{\pgfqpoint{2.434834in}{1.147815in}}%
\pgfpathlineto{\pgfqpoint{2.526620in}{1.187179in}}%
\pgfpathlineto{\pgfqpoint{2.608725in}{1.220844in}}%
\pgfpathlineto{\pgfqpoint{2.682999in}{1.248470in}}%
\pgfpathlineto{\pgfqpoint{2.750805in}{1.274705in}}%
\pgfpathlineto{\pgfqpoint{2.813181in}{1.300029in}}%
\pgfpathlineto{\pgfqpoint{2.870932in}{1.321061in}}%
\pgfpathlineto{\pgfqpoint{2.924697in}{1.343735in}}%
\pgfpathlineto{\pgfqpoint{2.974991in}{1.361653in}}%
\pgfpathlineto{\pgfqpoint{3.022234in}{1.379430in}}%
\pgfpathlineto{\pgfqpoint{3.066777in}{1.397384in}}%
\pgfpathlineto{\pgfqpoint{3.108910in}{1.414039in}}%
\pgfpathlineto{\pgfqpoint{3.148882in}{1.430125in}}%
\pgfpathlineto{\pgfqpoint{3.186903in}{1.445713in}}%
\pgfpathlineto{\pgfqpoint{3.223156in}{1.460101in}}%
\pgfpathlineto{\pgfqpoint{3.257796in}{1.473218in}}%
\pgfpathlineto{\pgfqpoint{3.290962in}{1.485836in}}%
\pgfpathlineto{\pgfqpoint{3.322774in}{1.497936in}}%
\pgfpathlineto{\pgfqpoint{3.353338in}{1.509726in}}%
\pgfpathlineto{\pgfqpoint{3.382748in}{1.519871in}}%
\pgfpathlineto{\pgfqpoint{3.411089in}{1.526866in}}%
\pgfpathlineto{\pgfqpoint{3.438435in}{1.536513in}}%
\pgfpathlineto{\pgfqpoint{3.464853in}{1.545701in}}%
\pgfpathlineto{\pgfqpoint{3.490406in}{1.553216in}}%
\pgfpathlineto{\pgfqpoint{3.515147in}{1.561802in}}%
\pgfpathlineto{\pgfqpoint{3.539127in}{1.569294in}}%
\pgfpathlineto{\pgfqpoint{3.562391in}{1.576416in}}%
\pgfpathlineto{\pgfqpoint{3.584980in}{1.583653in}}%
\pgfpathlineto{\pgfqpoint{3.606933in}{1.589801in}}%
\pgfpathlineto{\pgfqpoint{3.628285in}{1.597837in}}%
\pgfpathlineto{\pgfqpoint{3.649067in}{1.604878in}}%
\pgfpathlineto{\pgfqpoint{3.669309in}{1.613653in}}%
\pgfpathlineto{\pgfqpoint{3.689039in}{1.620770in}}%
\pgfpathlineto{\pgfqpoint{3.708281in}{1.628255in}}%
\pgfpathlineto{\pgfqpoint{3.727060in}{1.634785in}}%
\pgfpathlineto{\pgfqpoint{3.745397in}{1.641761in}}%
\pgfpathlineto{\pgfqpoint{3.763312in}{1.648257in}}%
\pgfpathlineto{\pgfqpoint{3.780825in}{1.654619in}}%
\pgfpathlineto{\pgfqpoint{3.797953in}{1.660629in}}%
\pgfpathlineto{\pgfqpoint{3.814712in}{1.666475in}}%
\pgfpathlineto{\pgfqpoint{3.831119in}{1.672411in}}%
\pgfpathlineto{\pgfqpoint{3.847187in}{1.678254in}}%
\pgfpathlineto{\pgfqpoint{3.862930in}{1.683031in}}%
\pgfpathlineto{\pgfqpoint{3.878362in}{1.688524in}}%
\pgfpathlineto{\pgfqpoint{3.893494in}{1.697076in}}%
\pgfpathlineto{\pgfqpoint{3.908338in}{1.702216in}}%
\pgfpathlineto{\pgfqpoint{3.922905in}{1.705424in}}%
\pgfpathlineto{\pgfqpoint{3.937204in}{1.709711in}}%
\pgfpathlineto{\pgfqpoint{3.951245in}{1.714631in}}%
\pgfpathlineto{\pgfqpoint{3.965038in}{1.718597in}}%
\pgfpathlineto{\pgfqpoint{3.978591in}{1.722965in}}%
\pgfpathlineto{\pgfqpoint{3.991913in}{1.728868in}}%
\pgfpathlineto{\pgfqpoint{4.005010in}{1.733140in}}%
\pgfpathlineto{\pgfqpoint{4.017891in}{1.739352in}}%
\pgfpathlineto{\pgfqpoint{4.030563in}{1.744911in}}%
\pgfpathlineto{\pgfqpoint{4.043031in}{1.749155in}}%
\pgfpathlineto{\pgfqpoint{4.055304in}{1.753718in}}%
\pgfpathlineto{\pgfqpoint{4.067386in}{1.758115in}}%
\pgfpathlineto{\pgfqpoint{4.079284in}{1.762915in}}%
\pgfpathlineto{\pgfqpoint{4.091002in}{1.767658in}}%
\pgfpathlineto{\pgfqpoint{4.102547in}{1.773223in}}%
\pgfpathlineto{\pgfqpoint{4.113924in}{1.778905in}}%
\pgfpathlineto{\pgfqpoint{4.125137in}{1.782900in}}%
\pgfpathlineto{\pgfqpoint{4.136191in}{1.787289in}}%
\pgfpathlineto{\pgfqpoint{4.147090in}{1.791509in}}%
\pgfpathlineto{\pgfqpoint{4.157839in}{1.795367in}}%
\pgfpathlineto{\pgfqpoint{4.168441in}{1.800169in}}%
\pgfpathlineto{\pgfqpoint{4.178902in}{1.803909in}}%
\pgfpathlineto{\pgfqpoint{4.189223in}{1.806668in}}%
\pgfpathlineto{\pgfqpoint{4.199410in}{1.808377in}}%
\pgfpathlineto{\pgfqpoint{4.209466in}{1.812116in}}%
\pgfpathlineto{\pgfqpoint{4.219393in}{1.815443in}}%
\pgfpathlineto{\pgfqpoint{4.229195in}{1.819638in}}%
\pgfpathlineto{\pgfqpoint{4.238876in}{1.824777in}}%
\pgfpathlineto{\pgfqpoint{4.248438in}{1.828979in}}%
\pgfpathlineto{\pgfqpoint{4.257884in}{1.833107in}}%
\pgfpathlineto{\pgfqpoint{4.267217in}{1.836095in}}%
\pgfpathlineto{\pgfqpoint{4.276439in}{1.839207in}}%
\pgfpathlineto{\pgfqpoint{4.285554in}{1.842095in}}%
\pgfpathlineto{\pgfqpoint{4.294563in}{1.845947in}}%
\pgfpathlineto{\pgfqpoint{4.303469in}{1.848644in}}%
\pgfpathlineto{\pgfqpoint{4.312274in}{1.854361in}}%
\pgfpathlineto{\pgfqpoint{4.320982in}{1.856366in}}%
\pgfpathlineto{\pgfqpoint{4.329592in}{1.860029in}}%
\pgfpathlineto{\pgfqpoint{4.338109in}{1.863009in}}%
\pgfpathlineto{\pgfqpoint{4.346534in}{1.864282in}}%
\pgfpathlineto{\pgfqpoint{4.354869in}{1.868059in}}%
\pgfpathlineto{\pgfqpoint{4.363115in}{1.870339in}}%
\pgfpathlineto{\pgfqpoint{4.371275in}{1.872054in}}%
\pgfpathlineto{\pgfqpoint{4.379351in}{1.875687in}}%
\pgfpathlineto{\pgfqpoint{4.387343in}{1.878561in}}%
\pgfpathlineto{\pgfqpoint{4.395255in}{1.881747in}}%
\pgfusepath{stroke}%
\end{pgfscope}%
\begin{pgfscope}%
\pgfpathrectangle{\pgfqpoint{0.635319in}{0.392649in}}{\pgfqpoint{3.938980in}{2.371600in}} %
\pgfusepath{clip}%
\pgfsetbuttcap%
\pgfsetroundjoin%
\pgfsetlinewidth{1.505625pt}%
\definecolor{currentstroke}{rgb}{1.000000,0.498039,0.054902}%
\pgfsetstrokecolor{currentstroke}%
\pgfsetdash{{1.500000pt}{2.475000pt}}{0.000000pt}%
\pgfpathmoveto{\pgfqpoint{0.814364in}{1.003474in}}%
\pgfpathlineto{\pgfqpoint{1.354521in}{1.196221in}}%
\pgfpathlineto{\pgfqpoint{1.670492in}{1.309744in}}%
\pgfpathlineto{\pgfqpoint{1.894677in}{1.390068in}}%
\pgfpathlineto{\pgfqpoint{2.068569in}{1.455483in}}%
\pgfpathlineto{\pgfqpoint{2.210649in}{1.506574in}}%
\pgfpathlineto{\pgfqpoint{2.330775in}{1.552407in}}%
\pgfpathlineto{\pgfqpoint{2.434834in}{1.592716in}}%
\pgfpathlineto{\pgfqpoint{2.526620in}{1.629018in}}%
\pgfpathlineto{\pgfqpoint{2.608725in}{1.661311in}}%
\pgfpathlineto{\pgfqpoint{2.682999in}{1.689570in}}%
\pgfpathlineto{\pgfqpoint{2.750805in}{1.715643in}}%
\pgfpathlineto{\pgfqpoint{2.813181in}{1.741107in}}%
\pgfpathlineto{\pgfqpoint{2.870932in}{1.763406in}}%
\pgfpathlineto{\pgfqpoint{2.924697in}{1.785478in}}%
\pgfpathlineto{\pgfqpoint{2.974991in}{1.804414in}}%
\pgfpathlineto{\pgfqpoint{3.022234in}{1.823359in}}%
\pgfpathlineto{\pgfqpoint{3.066777in}{1.841400in}}%
\pgfpathlineto{\pgfqpoint{3.108910in}{1.859756in}}%
\pgfpathlineto{\pgfqpoint{3.148882in}{1.875784in}}%
\pgfpathlineto{\pgfqpoint{3.186903in}{1.891288in}}%
\pgfpathlineto{\pgfqpoint{3.223156in}{1.905248in}}%
\pgfpathlineto{\pgfqpoint{3.257796in}{1.920377in}}%
\pgfpathlineto{\pgfqpoint{3.290962in}{1.933953in}}%
\pgfpathlineto{\pgfqpoint{3.322774in}{1.947309in}}%
\pgfpathlineto{\pgfqpoint{3.353338in}{1.959146in}}%
\pgfpathlineto{\pgfqpoint{3.382748in}{1.969810in}}%
\pgfpathlineto{\pgfqpoint{3.411089in}{1.979606in}}%
\pgfpathlineto{\pgfqpoint{3.438435in}{1.991017in}}%
\pgfpathlineto{\pgfqpoint{3.464853in}{2.002531in}}%
\pgfpathlineto{\pgfqpoint{3.490406in}{2.012209in}}%
\pgfpathlineto{\pgfqpoint{3.515147in}{2.022064in}}%
\pgfpathlineto{\pgfqpoint{3.539127in}{2.031476in}}%
\pgfpathlineto{\pgfqpoint{3.562391in}{2.040342in}}%
\pgfpathlineto{\pgfqpoint{3.584980in}{2.049184in}}%
\pgfpathlineto{\pgfqpoint{3.606933in}{2.056970in}}%
\pgfpathlineto{\pgfqpoint{3.628285in}{2.065490in}}%
\pgfpathlineto{\pgfqpoint{3.649067in}{2.073511in}}%
\pgfpathlineto{\pgfqpoint{3.669309in}{2.084896in}}%
\pgfpathlineto{\pgfqpoint{3.689039in}{2.093434in}}%
\pgfpathlineto{\pgfqpoint{3.708281in}{2.101422in}}%
\pgfpathlineto{\pgfqpoint{3.727060in}{2.108578in}}%
\pgfpathlineto{\pgfqpoint{3.745397in}{2.115723in}}%
\pgfpathlineto{\pgfqpoint{3.763312in}{2.123233in}}%
\pgfpathlineto{\pgfqpoint{3.780825in}{2.130028in}}%
\pgfpathlineto{\pgfqpoint{3.797953in}{2.136992in}}%
\pgfpathlineto{\pgfqpoint{3.814712in}{2.143901in}}%
\pgfpathlineto{\pgfqpoint{3.831119in}{2.150234in}}%
\pgfpathlineto{\pgfqpoint{3.847187in}{2.157015in}}%
\pgfpathlineto{\pgfqpoint{3.862930in}{2.161761in}}%
\pgfpathlineto{\pgfqpoint{3.878362in}{2.168700in}}%
\pgfpathlineto{\pgfqpoint{3.893494in}{2.178080in}}%
\pgfpathlineto{\pgfqpoint{3.908338in}{2.184067in}}%
\pgfpathlineto{\pgfqpoint{3.922905in}{2.187702in}}%
\pgfpathlineto{\pgfqpoint{3.937204in}{2.192834in}}%
\pgfpathlineto{\pgfqpoint{3.951245in}{2.198174in}}%
\pgfpathlineto{\pgfqpoint{3.965038in}{2.202972in}}%
\pgfpathlineto{\pgfqpoint{3.978591in}{2.208445in}}%
\pgfpathlineto{\pgfqpoint{3.991913in}{2.215417in}}%
\pgfpathlineto{\pgfqpoint{4.005010in}{2.220333in}}%
\pgfpathlineto{\pgfqpoint{4.017891in}{2.226917in}}%
\pgfpathlineto{\pgfqpoint{4.030563in}{2.233226in}}%
\pgfpathlineto{\pgfqpoint{4.043031in}{2.237948in}}%
\pgfpathlineto{\pgfqpoint{4.055304in}{2.242623in}}%
\pgfpathlineto{\pgfqpoint{4.067386in}{2.247292in}}%
\pgfpathlineto{\pgfqpoint{4.079284in}{2.252642in}}%
\pgfpathlineto{\pgfqpoint{4.091002in}{2.257858in}}%
\pgfpathlineto{\pgfqpoint{4.102547in}{2.264001in}}%
\pgfpathlineto{\pgfqpoint{4.113924in}{2.270044in}}%
\pgfpathlineto{\pgfqpoint{4.125137in}{2.274909in}}%
\pgfpathlineto{\pgfqpoint{4.136191in}{2.280591in}}%
\pgfpathlineto{\pgfqpoint{4.147090in}{2.285460in}}%
\pgfpathlineto{\pgfqpoint{4.157839in}{2.289815in}}%
\pgfpathlineto{\pgfqpoint{4.168441in}{2.295739in}}%
\pgfpathlineto{\pgfqpoint{4.178902in}{2.299778in}}%
\pgfpathlineto{\pgfqpoint{4.189223in}{2.302617in}}%
\pgfpathlineto{\pgfqpoint{4.199410in}{2.304590in}}%
\pgfpathlineto{\pgfqpoint{4.209466in}{2.308584in}}%
\pgfpathlineto{\pgfqpoint{4.219393in}{2.312689in}}%
\pgfpathlineto{\pgfqpoint{4.229195in}{2.317751in}}%
\pgfpathlineto{\pgfqpoint{4.238876in}{2.322997in}}%
\pgfpathlineto{\pgfqpoint{4.248438in}{2.327701in}}%
\pgfpathlineto{\pgfqpoint{4.257884in}{2.332124in}}%
\pgfpathlineto{\pgfqpoint{4.267217in}{2.334321in}}%
\pgfpathlineto{\pgfqpoint{4.276439in}{2.337761in}}%
\pgfpathlineto{\pgfqpoint{4.285554in}{2.340588in}}%
\pgfpathlineto{\pgfqpoint{4.294563in}{2.344687in}}%
\pgfpathlineto{\pgfqpoint{4.303469in}{2.347663in}}%
\pgfpathlineto{\pgfqpoint{4.312274in}{2.353814in}}%
\pgfpathlineto{\pgfqpoint{4.320982in}{2.356267in}}%
\pgfpathlineto{\pgfqpoint{4.329592in}{2.360495in}}%
\pgfpathlineto{\pgfqpoint{4.338109in}{2.363054in}}%
\pgfpathlineto{\pgfqpoint{4.346534in}{2.364311in}}%
\pgfpathlineto{\pgfqpoint{4.354869in}{2.367734in}}%
\pgfpathlineto{\pgfqpoint{4.363115in}{2.370450in}}%
\pgfpathlineto{\pgfqpoint{4.371275in}{2.372731in}}%
\pgfpathlineto{\pgfqpoint{4.379351in}{2.376451in}}%
\pgfpathlineto{\pgfqpoint{4.387343in}{2.379439in}}%
\pgfpathlineto{\pgfqpoint{4.395255in}{2.383075in}}%
\pgfusepath{stroke}%
\end{pgfscope}%
\begin{pgfscope}%
\pgfpathrectangle{\pgfqpoint{0.635319in}{0.392649in}}{\pgfqpoint{3.938980in}{2.371600in}} %
\pgfusepath{clip}%
\pgfsetbuttcap%
\pgfsetroundjoin%
\pgfsetlinewidth{1.505625pt}%
\definecolor{currentstroke}{rgb}{0.172549,0.627451,0.172549}%
\pgfsetstrokecolor{currentstroke}%
\pgfsetdash{{5.550000pt}{2.400000pt}}{0.000000pt}%
\pgfpathmoveto{\pgfqpoint{0.814364in}{0.837183in}}%
\pgfpathlineto{\pgfqpoint{1.354521in}{1.090555in}}%
\pgfpathlineto{\pgfqpoint{1.670492in}{1.242222in}}%
\pgfpathlineto{\pgfqpoint{1.894677in}{1.351065in}}%
\pgfpathlineto{\pgfqpoint{2.068569in}{1.438175in}}%
\pgfpathlineto{\pgfqpoint{2.210649in}{1.506619in}}%
\pgfpathlineto{\pgfqpoint{2.330775in}{1.567917in}}%
\pgfpathlineto{\pgfqpoint{2.434834in}{1.618646in}}%
\pgfpathlineto{\pgfqpoint{2.526620in}{1.666624in}}%
\pgfpathlineto{\pgfqpoint{2.608725in}{1.709570in}}%
\pgfpathlineto{\pgfqpoint{2.682999in}{1.748134in}}%
\pgfpathlineto{\pgfqpoint{2.750805in}{1.781747in}}%
\pgfpathlineto{\pgfqpoint{2.813181in}{1.814745in}}%
\pgfpathlineto{\pgfqpoint{2.870932in}{1.843133in}}%
\pgfpathlineto{\pgfqpoint{2.924697in}{1.872735in}}%
\pgfpathlineto{\pgfqpoint{2.974991in}{1.898641in}}%
\pgfpathlineto{\pgfqpoint{3.022234in}{1.923464in}}%
\pgfpathlineto{\pgfqpoint{3.066777in}{1.946915in}}%
\pgfpathlineto{\pgfqpoint{3.108910in}{1.971414in}}%
\pgfpathlineto{\pgfqpoint{3.148882in}{1.992657in}}%
\pgfpathlineto{\pgfqpoint{3.186903in}{2.012624in}}%
\pgfpathlineto{\pgfqpoint{3.223156in}{2.032417in}}%
\pgfpathlineto{\pgfqpoint{3.257796in}{2.051183in}}%
\pgfpathlineto{\pgfqpoint{3.290962in}{2.068025in}}%
\pgfpathlineto{\pgfqpoint{3.322774in}{2.085468in}}%
\pgfpathlineto{\pgfqpoint{3.353338in}{2.102088in}}%
\pgfpathlineto{\pgfqpoint{3.382748in}{2.117061in}}%
\pgfpathlineto{\pgfqpoint{3.411089in}{2.129250in}}%
\pgfpathlineto{\pgfqpoint{3.438435in}{2.143050in}}%
\pgfpathlineto{\pgfqpoint{3.464853in}{2.158238in}}%
\pgfpathlineto{\pgfqpoint{3.490406in}{2.172056in}}%
\pgfpathlineto{\pgfqpoint{3.515147in}{2.184406in}}%
\pgfpathlineto{\pgfqpoint{3.539127in}{2.195420in}}%
\pgfpathlineto{\pgfqpoint{3.562391in}{2.206607in}}%
\pgfpathlineto{\pgfqpoint{3.584980in}{2.218755in}}%
\pgfpathlineto{\pgfqpoint{3.606933in}{2.230208in}}%
\pgfpathlineto{\pgfqpoint{3.628285in}{2.243229in}}%
\pgfpathlineto{\pgfqpoint{3.649067in}{2.252934in}}%
\pgfpathlineto{\pgfqpoint{3.669309in}{2.265537in}}%
\pgfpathlineto{\pgfqpoint{3.689039in}{2.276261in}}%
\pgfpathlineto{\pgfqpoint{3.708281in}{2.287495in}}%
\pgfpathlineto{\pgfqpoint{3.727060in}{2.297385in}}%
\pgfpathlineto{\pgfqpoint{3.745397in}{2.308348in}}%
\pgfpathlineto{\pgfqpoint{3.763312in}{2.318392in}}%
\pgfpathlineto{\pgfqpoint{3.780825in}{2.327559in}}%
\pgfpathlineto{\pgfqpoint{3.797953in}{2.336694in}}%
\pgfpathlineto{\pgfqpoint{3.814712in}{2.345877in}}%
\pgfpathlineto{\pgfqpoint{3.831119in}{2.354245in}}%
\pgfpathlineto{\pgfqpoint{3.847187in}{2.362627in}}%
\pgfpathlineto{\pgfqpoint{3.862930in}{2.369352in}}%
\pgfpathlineto{\pgfqpoint{3.878362in}{2.378289in}}%
\pgfpathlineto{\pgfqpoint{3.893494in}{2.389425in}}%
\pgfpathlineto{\pgfqpoint{3.908338in}{2.397394in}}%
\pgfpathlineto{\pgfqpoint{3.922905in}{2.403846in}}%
\pgfpathlineto{\pgfqpoint{3.937204in}{2.410794in}}%
\pgfpathlineto{\pgfqpoint{3.951245in}{2.418281in}}%
\pgfpathlineto{\pgfqpoint{3.965038in}{2.425240in}}%
\pgfpathlineto{\pgfqpoint{3.978591in}{2.431820in}}%
\pgfpathlineto{\pgfqpoint{3.991913in}{2.439634in}}%
\pgfpathlineto{\pgfqpoint{4.005010in}{2.446790in}}%
\pgfpathlineto{\pgfqpoint{4.017891in}{2.454998in}}%
\pgfpathlineto{\pgfqpoint{4.030563in}{2.462291in}}%
\pgfpathlineto{\pgfqpoint{4.043031in}{2.468460in}}%
\pgfpathlineto{\pgfqpoint{4.055304in}{2.473969in}}%
\pgfpathlineto{\pgfqpoint{4.067386in}{2.479773in}}%
\pgfpathlineto{\pgfqpoint{4.079284in}{2.486567in}}%
\pgfpathlineto{\pgfqpoint{4.091002in}{2.493682in}}%
\pgfpathlineto{\pgfqpoint{4.102547in}{2.501542in}}%
\pgfpathlineto{\pgfqpoint{4.113924in}{2.509296in}}%
\pgfpathlineto{\pgfqpoint{4.125137in}{2.515337in}}%
\pgfpathlineto{\pgfqpoint{4.136191in}{2.522132in}}%
\pgfpathlineto{\pgfqpoint{4.147090in}{2.528520in}}%
\pgfpathlineto{\pgfqpoint{4.157839in}{2.534868in}}%
\pgfpathlineto{\pgfqpoint{4.168441in}{2.541797in}}%
\pgfpathlineto{\pgfqpoint{4.178902in}{2.546513in}}%
\pgfpathlineto{\pgfqpoint{4.189223in}{2.550899in}}%
\pgfpathlineto{\pgfqpoint{4.199410in}{2.554557in}}%
\pgfpathlineto{\pgfqpoint{4.209466in}{2.559616in}}%
\pgfpathlineto{\pgfqpoint{4.219393in}{2.564166in}}%
\pgfpathlineto{\pgfqpoint{4.229195in}{2.570064in}}%
\pgfpathlineto{\pgfqpoint{4.238876in}{2.576488in}}%
\pgfpathlineto{\pgfqpoint{4.248438in}{2.582776in}}%
\pgfpathlineto{\pgfqpoint{4.257884in}{2.587968in}}%
\pgfpathlineto{\pgfqpoint{4.267217in}{2.592218in}}%
\pgfpathlineto{\pgfqpoint{4.276439in}{2.597270in}}%
\pgfpathlineto{\pgfqpoint{4.285554in}{2.601011in}}%
\pgfpathlineto{\pgfqpoint{4.294563in}{2.606224in}}%
\pgfpathlineto{\pgfqpoint{4.303469in}{2.610413in}}%
\pgfpathlineto{\pgfqpoint{4.312274in}{2.617038in}}%
\pgfpathlineto{\pgfqpoint{4.320982in}{2.620232in}}%
\pgfpathlineto{\pgfqpoint{4.329592in}{2.625375in}}%
\pgfpathlineto{\pgfqpoint{4.338109in}{2.629538in}}%
\pgfpathlineto{\pgfqpoint{4.346534in}{2.631647in}}%
\pgfpathlineto{\pgfqpoint{4.354869in}{2.636117in}}%
\pgfpathlineto{\pgfqpoint{4.363115in}{2.640288in}}%
\pgfpathlineto{\pgfqpoint{4.371275in}{2.643941in}}%
\pgfpathlineto{\pgfqpoint{4.379351in}{2.648927in}}%
\pgfpathlineto{\pgfqpoint{4.387343in}{2.652628in}}%
\pgfpathlineto{\pgfqpoint{4.395255in}{2.656449in}}%
\pgfusepath{stroke}%
\end{pgfscope}%
\begin{pgfscope}%
\pgfsetrectcap%
\pgfsetmiterjoin%
\pgfsetlinewidth{0.803000pt}%
\definecolor{currentstroke}{rgb}{0.000000,0.000000,0.000000}%
\pgfsetstrokecolor{currentstroke}%
\pgfsetdash{}{0pt}%
\pgfpathmoveto{\pgfqpoint{0.635319in}{0.392649in}}%
\pgfpathlineto{\pgfqpoint{0.635319in}{2.764249in}}%
\pgfusepath{stroke}%
\end{pgfscope}%
\begin{pgfscope}%
\pgfsetrectcap%
\pgfsetmiterjoin%
\pgfsetlinewidth{0.803000pt}%
\definecolor{currentstroke}{rgb}{0.000000,0.000000,0.000000}%
\pgfsetstrokecolor{currentstroke}%
\pgfsetdash{}{0pt}%
\pgfpathmoveto{\pgfqpoint{4.574300in}{0.392649in}}%
\pgfpathlineto{\pgfqpoint{4.574300in}{2.764249in}}%
\pgfusepath{stroke}%
\end{pgfscope}%
\begin{pgfscope}%
\pgfsetrectcap%
\pgfsetmiterjoin%
\pgfsetlinewidth{0.803000pt}%
\definecolor{currentstroke}{rgb}{0.000000,0.000000,0.000000}%
\pgfsetstrokecolor{currentstroke}%
\pgfsetdash{}{0pt}%
\pgfpathmoveto{\pgfqpoint{0.635319in}{0.392649in}}%
\pgfpathlineto{\pgfqpoint{4.574300in}{0.392649in}}%
\pgfusepath{stroke}%
\end{pgfscope}%
\begin{pgfscope}%
\pgfsetrectcap%
\pgfsetmiterjoin%
\pgfsetlinewidth{0.803000pt}%
\definecolor{currentstroke}{rgb}{0.000000,0.000000,0.000000}%
\pgfsetstrokecolor{currentstroke}%
\pgfsetdash{}{0pt}%
\pgfpathmoveto{\pgfqpoint{0.635319in}{2.764249in}}%
\pgfpathlineto{\pgfqpoint{4.574300in}{2.764249in}}%
\pgfusepath{stroke}%
\end{pgfscope}%
\begin{pgfscope}%
\pgfsetbuttcap%
\pgfsetmiterjoin%
\definecolor{currentfill}{rgb}{1.000000,1.000000,1.000000}%
\pgfsetfillcolor{currentfill}%
\pgfsetfillopacity{0.800000}%
\pgfsetlinewidth{1.003750pt}%
\definecolor{currentstroke}{rgb}{0.800000,0.800000,0.800000}%
\pgfsetstrokecolor{currentstroke}%
\pgfsetstrokeopacity{0.800000}%
\pgfsetdash{}{0pt}%
\pgfpathmoveto{\pgfqpoint{0.713097in}{2.210561in}}%
\pgfpathlineto{\pgfqpoint{1.662489in}{2.210561in}}%
\pgfpathquadraticcurveto{\pgfqpoint{1.684711in}{2.210561in}}{\pgfqpoint{1.684711in}{2.232783in}}%
\pgfpathlineto{\pgfqpoint{1.684711in}{2.686471in}}%
\pgfpathquadraticcurveto{\pgfqpoint{1.684711in}{2.708693in}}{\pgfqpoint{1.662489in}{2.708693in}}%
\pgfpathlineto{\pgfqpoint{0.713097in}{2.708693in}}%
\pgfpathquadraticcurveto{\pgfqpoint{0.690875in}{2.708693in}}{\pgfqpoint{0.690875in}{2.686471in}}%
\pgfpathlineto{\pgfqpoint{0.690875in}{2.232783in}}%
\pgfpathquadraticcurveto{\pgfqpoint{0.690875in}{2.210561in}}{\pgfqpoint{0.713097in}{2.210561in}}%
\pgfpathclose%
\pgfusepath{stroke,fill}%
\end{pgfscope}%
\begin{pgfscope}%
\pgfsetrectcap%
\pgfsetroundjoin%
\pgfsetlinewidth{1.505625pt}%
\definecolor{currentstroke}{rgb}{0.121569,0.466667,0.705882}%
\pgfsetstrokecolor{currentstroke}%
\pgfsetdash{}{0pt}%
\pgfpathmoveto{\pgfqpoint{0.735319in}{2.625360in}}%
\pgfpathlineto{\pgfqpoint{0.957542in}{2.625360in}}%
\pgfusepath{stroke}%
\end{pgfscope}%
\begin{pgfscope}%
\pgftext[x=1.046430in,y=2.586471in,left,base]{\rmfamily\fontsize{8.000000}{9.600000}\selectfont OGI}%
\end{pgfscope}%
\begin{pgfscope}%
\pgfsetbuttcap%
\pgfsetroundjoin%
\pgfsetlinewidth{1.505625pt}%
\definecolor{currentstroke}{rgb}{1.000000,0.498039,0.054902}%
\pgfsetstrokecolor{currentstroke}%
\pgfsetdash{{1.500000pt}{2.475000pt}}{0.000000pt}%
\pgfpathmoveto{\pgfqpoint{0.735319in}{2.470427in}}%
\pgfpathlineto{\pgfqpoint{0.957542in}{2.470427in}}%
\pgfusepath{stroke}%
\end{pgfscope}%
\begin{pgfscope}%
\pgftext[x=1.046430in,y=2.431538in,left,base]{\rmfamily\fontsize{8.000000}{9.600000}\selectfont Thompson}%
\end{pgfscope}%
\begin{pgfscope}%
\pgfsetbuttcap%
\pgfsetroundjoin%
\pgfsetlinewidth{1.505625pt}%
\definecolor{currentstroke}{rgb}{0.172549,0.627451,0.172549}%
\pgfsetstrokecolor{currentstroke}%
\pgfsetdash{{5.550000pt}{2.400000pt}}{0.000000pt}%
\pgfpathmoveto{\pgfqpoint{0.735319in}{2.315494in}}%
\pgfpathlineto{\pgfqpoint{0.957542in}{2.315494in}}%
\pgfusepath{stroke}%
\end{pgfscope}%
\begin{pgfscope}%
\pgftext[x=1.046430in,y=2.276605in,left,base]{\rmfamily\fontsize{8.000000}{9.600000}\selectfont Bayes UCB}%
\end{pgfscope}%
\end{pgfpicture}%
\makeatother%
\endgroup%

	\caption{Cumulative regret in the large-scale problem of this section averaged over 5,000 independent trials.
		We plot the number of periods, $T$ on a logarithmic scale.}
	\label{fig:chapelle_and_li}
\end{figure}

\begin{table}[h!]
	\centering
	\begin{tabular}{c|ccccc}
		\toprule
		$T/A$ &   OGI &  Thompson &   Bayes UCB &  Rel. improvement (\%) &  Abs. improvement \\
		\midrule
		2,000 & 230.5 &     284.4 & 297.9 &                            18.9 &                  53.9 \\
		4,000 & 254.7 &     311.6 & 333.5 &                            18.3 &                  57.0 \\
		6,000 & 268.6 &     327.4 & 354.5 &                            18.0 &                  58.8 \\
		8,000 & 279.1 &     339.2 & 369.6 &                            17.7 &                  60.1 \\
		10,000 & 287.1 &     347.7 & 380.7 &                            17.4 &                  60.6 \\
		\bottomrule
	\end{tabular}
	
	\caption{Regret in the large scale experiment from OGI, Thompson Sampling and Bayes UCB. The last two columns show the relative and absolute difference from Thompson Sampling, which is the closest competitor to OGI.}
	\label{table:additional_cli_table}
\end{table}

As before, the OGI scheme consistently dominates the other two. What is particularly interesting is that despite going out to a horizon of $10^6$ time periods, the relative improvement in regret over these algorithms remains substantial. For instance, going from a horizon length of $2 \times 10^5$ (corresponding to a heuristic budget of {$T/A = 2,000$} pulls per arm) 
to a horizon length of $10^6$ (corresponding to a heuristic budget of {$T/A = 10,000$ }pulls per arm) resulted in the relative improvement offered by OGI shrining only marginally, from $18.9\%$ to $17.4\%$. 

% and the relative margin by which OGI outperforms the other algorithms appears to decline, albeit steadily with the horizon.
%In particular, we notice that the improvement in regret, over the nearest competitor Thompson Sampling, is 18.9\% when there are $2\times 10^5$ periods but this improvement decreases to almost 17.4\% with $10^6$ time periods.
%On the other hand, the \emph{absolute} difference in regret increases monotonically with $T$.
%Therefore our method appears to consistently retain its advantage over Thompson Sampling when $T$ is large even if, in a relative sense, the performance gap shrinks (as we would expect from the asymptotic bound).



\subsection{Bandits with multiple arm pulls}

In this section, we consider a somewhat exploratory experiment; we seek to adapt OGI to a more complex bandit problem (here, we allow for multiple simultaneous arm pulls). Again, in the discounted infinite horizon setting, a number of heuristic approaches have been proposed to adapt the Gittins index to more complex settings; a good example is the so-called Whittle relaxation for restless bandits. One might consider applying those same heuristic strategies to the optimistic gittins index. 

For this experiment, we consider a more general MAB problem, where the agent is able to pull up to a certain number (say $m < A$) of the arms simultaneously. In order to describe the problem, we recall that $A$ is the total number of arms and define  $\mathcal{D}_m := \{d \in \{0,1\}^A : \sum_i d_i \le m\}$ to be the set of all $A$-dimensional binary vectors with up to $m$ ones in them, which we take to be the action space. Let $X_t = (X_{1,t}, \ldots, {X_{A,t})}$ be a tuple of (potential) rewards from the $A$ arms at time $t$, where the definition of $X_{i,t}$ for any arm $i$ is the same as in Section~\ref{sec:model_and_prelim}. Given a decision $d \in \mathcal D_m$, which encodes the subset of arms pulled, the reward $d^\top X_t$ is earned and an arm $j$'s reward $X_{j,t}$ is observed if and only if that arm is pulled, i.e. $d_{j} = 1$. We can then define a policy $(\pi_t, t \in \mathbb{N})$ to be a $\mathcal{D}_m$-valued stochastic process adapted to an information set generated by past actions and observed feedback. 
%where $\pi_{t+1}$ is measurable with respect to $\sigma\left( (\pi_s, (\pi_{i,s} X_{i,s}, \;i =1,\ldots,A)), \; s=1,\ldots,t\right)$. That is a policy's information set comes from its previous decisions and observed feedback.
 A policy $\pi$'s regret is given by the equation 
\[
\Regret{\pi, T} = T \cdot  \Ee{\max_{d \in \mathcal D_m}   \sum_{i=1}^A d_i \mu(\theta_i)} - \sum_{t=1}^T \E[\pi_t^\top X_t ]
\]
where the expectation is over both the randomness in the rewards, the prior and the policy's actions.

We propose a heuristic to this problem using our scheme, which is to compute the Optimistic Gittins Index of every arm, at time $t$, using a discount factor of $1-1/t$ (just as before). However, for this problem, we pick $m$ arms with the largest indices. This is essentially Whittle's heuristic \citep{whittle1988restless}, which was originally given for the restless bandit problem but can be described as picking several arms with the largest Gittins indices. {We break ties randomly and it would be interesting to explore smarter tiebreaking schemes that could improve performance \citep{brown2017index}}.

To test our policy, we simulate $A = 6$ binary arms with uniformly distributed biases and fix $m=3$. 
We benchmark our heuristic against Thompson Sampling and IDS. Because the arms give independent Bernoulli rewards, we will use a flat Beta prior for all of the algorithms. We implement the version of IDS designed for the linear bandit problem because this experiment is a special case of a linear bandit. Our implementation of IDS also uses {10,000 Monte Carlo} samples per iteration.

The results, produced from 1,000 independent trials, are summarized in Figure~\ref{fig:restless1} and Table~\ref{table:restless1_summary}. We notice a significant spread in the performance between OGI and both Thompson Sampling and IDS. Just like for our main algorithm, the primary computational bottleneck in using OGI comes from solving the stopping problem and this can be onerous if $K$ is large. However, as the results suggest, the policy works well even for low to moderate look-ahead parameters. The experiment here sets the stage for an exploration of the appropriate extensions to the OGI algorithm for more complex bandit problems (such as contextual bandits) which we leave for future work. 

The results, produced from 2,000 independent trials, are summarized in Figure~\ref{fig:restless1} and Table~\ref{table:restless1_summary}. The horizon is limited to 250 time periods because of the increased computational effort required to execute a single trial of both the IDS algorithm and Whittle's heuristic, when $K > 1$. This extra time is on the order of minutes for these algorithms.
For the sake of simplicity, we dub this algorithm as exactly `Whittle's heuristic' for the remainder of this section.

{We notice a significant spread in performance between Whittle's heuristic and Thompson Sampling. Meanwhile, IDS and Whittle's heuristic show similar performance, however the computational cost of running the latter algorithm is up to 25 times lower (IDS requires generating 10,000 Monte Carlo samples on each iteration). Just like for our main algorithm, the primary computational bottleneck in using Whittle's heuristic comes from solving the stopping problem and this can be onerous if $K$ is large. However, as the results suggest, the policy works well even for low to moderate look-ahead parameters but nonetheless improves slightly when $K$ increases.% By contrast, IDS is one of the slowest algorithm because it needs to generate a 10,000 Monte Carlo samples in every iteration.}
}

\begin{figure}
	\centering
	%% Creator: Matplotlib, PGF backend
%%
%% To include the figure in your LaTeX document, write
%%   \input{<filename>.pgf}
%%
%% Make sure the required packages are loaded in your preamble
%%   \usepackage{pgf}
%%
%% Figures using additional raster images can only be included by \input if
%% they are in the same directory as the main LaTeX file. For loading figures
%% from other directories you can use the `import` package
%%   \usepackage{import}
%% and then include the figures with
%%   \import{<path to file>}{<filename>.pgf}
%%
%% Matplotlib used the following preamble
%%   \usepackage[utf8x]{inputenc}
%%   \usepackage[T1]{fontenc}
%%
\begingroup%
\makeatletter%
\begin{pgfpicture}%
\pgfpathrectangle{\pgfpointorigin}{\pgfqpoint{4.875000in}{3.012916in}}%
\pgfusepath{use as bounding box, clip}%
\begin{pgfscope}%
\pgfsetbuttcap%
\pgfsetmiterjoin%
\definecolor{currentfill}{rgb}{1.000000,1.000000,1.000000}%
\pgfsetfillcolor{currentfill}%
\pgfsetlinewidth{0.000000pt}%
\definecolor{currentstroke}{rgb}{1.000000,1.000000,1.000000}%
\pgfsetstrokecolor{currentstroke}%
\pgfsetdash{}{0pt}%
\pgfpathmoveto{\pgfqpoint{0.000000in}{0.000000in}}%
\pgfpathlineto{\pgfqpoint{4.875000in}{0.000000in}}%
\pgfpathlineto{\pgfqpoint{4.875000in}{3.012916in}}%
\pgfpathlineto{\pgfqpoint{0.000000in}{3.012916in}}%
\pgfpathclose%
\pgfusepath{fill}%
\end{pgfscope}%
\begin{pgfscope}%
\pgfsetbuttcap%
\pgfsetmiterjoin%
\definecolor{currentfill}{rgb}{1.000000,1.000000,1.000000}%
\pgfsetfillcolor{currentfill}%
\pgfsetlinewidth{0.000000pt}%
\definecolor{currentstroke}{rgb}{0.000000,0.000000,0.000000}%
\pgfsetstrokecolor{currentstroke}%
\pgfsetstrokeopacity{0.000000}%
\pgfsetdash{}{0pt}%
\pgfpathmoveto{\pgfqpoint{0.609375in}{0.376614in}}%
\pgfpathlineto{\pgfqpoint{4.387500in}{0.376614in}}%
\pgfpathlineto{\pgfqpoint{4.387500in}{2.651366in}}%
\pgfpathlineto{\pgfqpoint{0.609375in}{2.651366in}}%
\pgfpathclose%
\pgfusepath{fill}%
\end{pgfscope}%
\begin{pgfscope}%
\pgfsetbuttcap%
\pgfsetroundjoin%
\definecolor{currentfill}{rgb}{0.000000,0.000000,0.000000}%
\pgfsetfillcolor{currentfill}%
\pgfsetlinewidth{0.803000pt}%
\definecolor{currentstroke}{rgb}{0.000000,0.000000,0.000000}%
\pgfsetstrokecolor{currentstroke}%
\pgfsetdash{}{0pt}%
\pgfsys@defobject{currentmarker}{\pgfqpoint{0.000000in}{-0.048611in}}{\pgfqpoint{0.000000in}{0.000000in}}{%
\pgfpathmoveto{\pgfqpoint{0.000000in}{0.000000in}}%
\pgfpathlineto{\pgfqpoint{0.000000in}{-0.048611in}}%
\pgfusepath{stroke,fill}%
}%
\begin{pgfscope}%
\pgfsys@transformshift{0.781108in}{0.376614in}%
\pgfsys@useobject{currentmarker}{}%
\end{pgfscope}%
\end{pgfscope}%
\begin{pgfscope}%
\pgftext[x=0.781108in,y=0.279392in,,top]{\rmfamily\fontsize{8.000000}{9.600000}\selectfont \(\displaystyle 0\)}%
\end{pgfscope}%
\begin{pgfscope}%
\pgfsetbuttcap%
\pgfsetroundjoin%
\definecolor{currentfill}{rgb}{0.000000,0.000000,0.000000}%
\pgfsetfillcolor{currentfill}%
\pgfsetlinewidth{0.803000pt}%
\definecolor{currentstroke}{rgb}{0.000000,0.000000,0.000000}%
\pgfsetstrokecolor{currentstroke}%
\pgfsetdash{}{0pt}%
\pgfsys@defobject{currentmarker}{\pgfqpoint{0.000000in}{-0.048611in}}{\pgfqpoint{0.000000in}{0.000000in}}{%
\pgfpathmoveto{\pgfqpoint{0.000000in}{0.000000in}}%
\pgfpathlineto{\pgfqpoint{0.000000in}{-0.048611in}}%
\pgfusepath{stroke,fill}%
}%
\begin{pgfscope}%
\pgfsys@transformshift{1.470799in}{0.376614in}%
\pgfsys@useobject{currentmarker}{}%
\end{pgfscope}%
\end{pgfscope}%
\begin{pgfscope}%
\pgftext[x=1.470799in,y=0.279392in,,top]{\rmfamily\fontsize{8.000000}{9.600000}\selectfont \(\displaystyle 50\)}%
\end{pgfscope}%
\begin{pgfscope}%
\pgfsetbuttcap%
\pgfsetroundjoin%
\definecolor{currentfill}{rgb}{0.000000,0.000000,0.000000}%
\pgfsetfillcolor{currentfill}%
\pgfsetlinewidth{0.803000pt}%
\definecolor{currentstroke}{rgb}{0.000000,0.000000,0.000000}%
\pgfsetstrokecolor{currentstroke}%
\pgfsetdash{}{0pt}%
\pgfsys@defobject{currentmarker}{\pgfqpoint{0.000000in}{-0.048611in}}{\pgfqpoint{0.000000in}{0.000000in}}{%
\pgfpathmoveto{\pgfqpoint{0.000000in}{0.000000in}}%
\pgfpathlineto{\pgfqpoint{0.000000in}{-0.048611in}}%
\pgfusepath{stroke,fill}%
}%
\begin{pgfscope}%
\pgfsys@transformshift{2.160489in}{0.376614in}%
\pgfsys@useobject{currentmarker}{}%
\end{pgfscope}%
\end{pgfscope}%
\begin{pgfscope}%
\pgftext[x=2.160489in,y=0.279392in,,top]{\rmfamily\fontsize{8.000000}{9.600000}\selectfont \(\displaystyle 100\)}%
\end{pgfscope}%
\begin{pgfscope}%
\pgfsetbuttcap%
\pgfsetroundjoin%
\definecolor{currentfill}{rgb}{0.000000,0.000000,0.000000}%
\pgfsetfillcolor{currentfill}%
\pgfsetlinewidth{0.803000pt}%
\definecolor{currentstroke}{rgb}{0.000000,0.000000,0.000000}%
\pgfsetstrokecolor{currentstroke}%
\pgfsetdash{}{0pt}%
\pgfsys@defobject{currentmarker}{\pgfqpoint{0.000000in}{-0.048611in}}{\pgfqpoint{0.000000in}{0.000000in}}{%
\pgfpathmoveto{\pgfqpoint{0.000000in}{0.000000in}}%
\pgfpathlineto{\pgfqpoint{0.000000in}{-0.048611in}}%
\pgfusepath{stroke,fill}%
}%
\begin{pgfscope}%
\pgfsys@transformshift{2.850180in}{0.376614in}%
\pgfsys@useobject{currentmarker}{}%
\end{pgfscope}%
\end{pgfscope}%
\begin{pgfscope}%
\pgftext[x=2.850180in,y=0.279392in,,top]{\rmfamily\fontsize{8.000000}{9.600000}\selectfont \(\displaystyle 150\)}%
\end{pgfscope}%
\begin{pgfscope}%
\pgfsetbuttcap%
\pgfsetroundjoin%
\definecolor{currentfill}{rgb}{0.000000,0.000000,0.000000}%
\pgfsetfillcolor{currentfill}%
\pgfsetlinewidth{0.803000pt}%
\definecolor{currentstroke}{rgb}{0.000000,0.000000,0.000000}%
\pgfsetstrokecolor{currentstroke}%
\pgfsetdash{}{0pt}%
\pgfsys@defobject{currentmarker}{\pgfqpoint{0.000000in}{-0.048611in}}{\pgfqpoint{0.000000in}{0.000000in}}{%
\pgfpathmoveto{\pgfqpoint{0.000000in}{0.000000in}}%
\pgfpathlineto{\pgfqpoint{0.000000in}{-0.048611in}}%
\pgfusepath{stroke,fill}%
}%
\begin{pgfscope}%
\pgfsys@transformshift{3.539870in}{0.376614in}%
\pgfsys@useobject{currentmarker}{}%
\end{pgfscope}%
\end{pgfscope}%
\begin{pgfscope}%
\pgftext[x=3.539870in,y=0.279392in,,top]{\rmfamily\fontsize{8.000000}{9.600000}\selectfont \(\displaystyle 200\)}%
\end{pgfscope}%
\begin{pgfscope}%
\pgfsetbuttcap%
\pgfsetroundjoin%
\definecolor{currentfill}{rgb}{0.000000,0.000000,0.000000}%
\pgfsetfillcolor{currentfill}%
\pgfsetlinewidth{0.803000pt}%
\definecolor{currentstroke}{rgb}{0.000000,0.000000,0.000000}%
\pgfsetstrokecolor{currentstroke}%
\pgfsetdash{}{0pt}%
\pgfsys@defobject{currentmarker}{\pgfqpoint{0.000000in}{-0.048611in}}{\pgfqpoint{0.000000in}{0.000000in}}{%
\pgfpathmoveto{\pgfqpoint{0.000000in}{0.000000in}}%
\pgfpathlineto{\pgfqpoint{0.000000in}{-0.048611in}}%
\pgfusepath{stroke,fill}%
}%
\begin{pgfscope}%
\pgfsys@transformshift{4.229561in}{0.376614in}%
\pgfsys@useobject{currentmarker}{}%
\end{pgfscope}%
\end{pgfscope}%
\begin{pgfscope}%
\pgftext[x=4.229561in,y=0.279392in,,top]{\rmfamily\fontsize{8.000000}{9.600000}\selectfont \(\displaystyle 250\)}%
\end{pgfscope}%
\begin{pgfscope}%
\pgftext[x=2.498438in,y=0.125712in,,top]{\rmfamily\fontsize{10.000000}{12.000000}\selectfont \(\displaystyle T\)}%
\end{pgfscope}%
\begin{pgfscope}%
\pgfsetbuttcap%
\pgfsetroundjoin%
\definecolor{currentfill}{rgb}{0.000000,0.000000,0.000000}%
\pgfsetfillcolor{currentfill}%
\pgfsetlinewidth{0.803000pt}%
\definecolor{currentstroke}{rgb}{0.000000,0.000000,0.000000}%
\pgfsetstrokecolor{currentstroke}%
\pgfsetdash{}{0pt}%
\pgfsys@defobject{currentmarker}{\pgfqpoint{-0.048611in}{0.000000in}}{\pgfqpoint{0.000000in}{0.000000in}}{%
\pgfpathmoveto{\pgfqpoint{0.000000in}{0.000000in}}%
\pgfpathlineto{\pgfqpoint{-0.048611in}{0.000000in}}%
\pgfusepath{stroke,fill}%
}%
\begin{pgfscope}%
\pgfsys@transformshift{0.609375in}{0.389825in}%
\pgfsys@useobject{currentmarker}{}%
\end{pgfscope}%
\end{pgfscope}%
\begin{pgfscope}%
\pgftext[x=0.453124in,y=0.351563in,left,base]{\rmfamily\fontsize{8.000000}{9.600000}\selectfont \(\displaystyle 0\)}%
\end{pgfscope}%
\begin{pgfscope}%
\pgfsetbuttcap%
\pgfsetroundjoin%
\definecolor{currentfill}{rgb}{0.000000,0.000000,0.000000}%
\pgfsetfillcolor{currentfill}%
\pgfsetlinewidth{0.803000pt}%
\definecolor{currentstroke}{rgb}{0.000000,0.000000,0.000000}%
\pgfsetstrokecolor{currentstroke}%
\pgfsetdash{}{0pt}%
\pgfsys@defobject{currentmarker}{\pgfqpoint{-0.048611in}{0.000000in}}{\pgfqpoint{0.000000in}{0.000000in}}{%
\pgfpathmoveto{\pgfqpoint{0.000000in}{0.000000in}}%
\pgfpathlineto{\pgfqpoint{-0.048611in}{0.000000in}}%
\pgfusepath{stroke,fill}%
}%
\begin{pgfscope}%
\pgfsys@transformshift{0.609375in}{0.674173in}%
\pgfsys@useobject{currentmarker}{}%
\end{pgfscope}%
\end{pgfscope}%
\begin{pgfscope}%
\pgftext[x=0.453124in,y=0.635911in,left,base]{\rmfamily\fontsize{8.000000}{9.600000}\selectfont \(\displaystyle 2\)}%
\end{pgfscope}%
\begin{pgfscope}%
\pgfsetbuttcap%
\pgfsetroundjoin%
\definecolor{currentfill}{rgb}{0.000000,0.000000,0.000000}%
\pgfsetfillcolor{currentfill}%
\pgfsetlinewidth{0.803000pt}%
\definecolor{currentstroke}{rgb}{0.000000,0.000000,0.000000}%
\pgfsetstrokecolor{currentstroke}%
\pgfsetdash{}{0pt}%
\pgfsys@defobject{currentmarker}{\pgfqpoint{-0.048611in}{0.000000in}}{\pgfqpoint{0.000000in}{0.000000in}}{%
\pgfpathmoveto{\pgfqpoint{0.000000in}{0.000000in}}%
\pgfpathlineto{\pgfqpoint{-0.048611in}{0.000000in}}%
\pgfusepath{stroke,fill}%
}%
\begin{pgfscope}%
\pgfsys@transformshift{0.609375in}{0.958522in}%
\pgfsys@useobject{currentmarker}{}%
\end{pgfscope}%
\end{pgfscope}%
\begin{pgfscope}%
\pgftext[x=0.453124in,y=0.920260in,left,base]{\rmfamily\fontsize{8.000000}{9.600000}\selectfont \(\displaystyle 4\)}%
\end{pgfscope}%
\begin{pgfscope}%
\pgfsetbuttcap%
\pgfsetroundjoin%
\definecolor{currentfill}{rgb}{0.000000,0.000000,0.000000}%
\pgfsetfillcolor{currentfill}%
\pgfsetlinewidth{0.803000pt}%
\definecolor{currentstroke}{rgb}{0.000000,0.000000,0.000000}%
\pgfsetstrokecolor{currentstroke}%
\pgfsetdash{}{0pt}%
\pgfsys@defobject{currentmarker}{\pgfqpoint{-0.048611in}{0.000000in}}{\pgfqpoint{0.000000in}{0.000000in}}{%
\pgfpathmoveto{\pgfqpoint{0.000000in}{0.000000in}}%
\pgfpathlineto{\pgfqpoint{-0.048611in}{0.000000in}}%
\pgfusepath{stroke,fill}%
}%
\begin{pgfscope}%
\pgfsys@transformshift{0.609375in}{1.242871in}%
\pgfsys@useobject{currentmarker}{}%
\end{pgfscope}%
\end{pgfscope}%
\begin{pgfscope}%
\pgftext[x=0.453124in,y=1.204608in,left,base]{\rmfamily\fontsize{8.000000}{9.600000}\selectfont \(\displaystyle 6\)}%
\end{pgfscope}%
\begin{pgfscope}%
\pgfsetbuttcap%
\pgfsetroundjoin%
\definecolor{currentfill}{rgb}{0.000000,0.000000,0.000000}%
\pgfsetfillcolor{currentfill}%
\pgfsetlinewidth{0.803000pt}%
\definecolor{currentstroke}{rgb}{0.000000,0.000000,0.000000}%
\pgfsetstrokecolor{currentstroke}%
\pgfsetdash{}{0pt}%
\pgfsys@defobject{currentmarker}{\pgfqpoint{-0.048611in}{0.000000in}}{\pgfqpoint{0.000000in}{0.000000in}}{%
\pgfpathmoveto{\pgfqpoint{0.000000in}{0.000000in}}%
\pgfpathlineto{\pgfqpoint{-0.048611in}{0.000000in}}%
\pgfusepath{stroke,fill}%
}%
\begin{pgfscope}%
\pgfsys@transformshift{0.609375in}{1.527219in}%
\pgfsys@useobject{currentmarker}{}%
\end{pgfscope}%
\end{pgfscope}%
\begin{pgfscope}%
\pgftext[x=0.453124in,y=1.488957in,left,base]{\rmfamily\fontsize{8.000000}{9.600000}\selectfont \(\displaystyle 8\)}%
\end{pgfscope}%
\begin{pgfscope}%
\pgfsetbuttcap%
\pgfsetroundjoin%
\definecolor{currentfill}{rgb}{0.000000,0.000000,0.000000}%
\pgfsetfillcolor{currentfill}%
\pgfsetlinewidth{0.803000pt}%
\definecolor{currentstroke}{rgb}{0.000000,0.000000,0.000000}%
\pgfsetstrokecolor{currentstroke}%
\pgfsetdash{}{0pt}%
\pgfsys@defobject{currentmarker}{\pgfqpoint{-0.048611in}{0.000000in}}{\pgfqpoint{0.000000in}{0.000000in}}{%
\pgfpathmoveto{\pgfqpoint{0.000000in}{0.000000in}}%
\pgfpathlineto{\pgfqpoint{-0.048611in}{0.000000in}}%
\pgfusepath{stroke,fill}%
}%
\begin{pgfscope}%
\pgfsys@transformshift{0.609375in}{1.811568in}%
\pgfsys@useobject{currentmarker}{}%
\end{pgfscope}%
\end{pgfscope}%
\begin{pgfscope}%
\pgftext[x=0.394096in,y=1.773305in,left,base]{\rmfamily\fontsize{8.000000}{9.600000}\selectfont \(\displaystyle 10\)}%
\end{pgfscope}%
\begin{pgfscope}%
\pgfsetbuttcap%
\pgfsetroundjoin%
\definecolor{currentfill}{rgb}{0.000000,0.000000,0.000000}%
\pgfsetfillcolor{currentfill}%
\pgfsetlinewidth{0.803000pt}%
\definecolor{currentstroke}{rgb}{0.000000,0.000000,0.000000}%
\pgfsetstrokecolor{currentstroke}%
\pgfsetdash{}{0pt}%
\pgfsys@defobject{currentmarker}{\pgfqpoint{-0.048611in}{0.000000in}}{\pgfqpoint{0.000000in}{0.000000in}}{%
\pgfpathmoveto{\pgfqpoint{0.000000in}{0.000000in}}%
\pgfpathlineto{\pgfqpoint{-0.048611in}{0.000000in}}%
\pgfusepath{stroke,fill}%
}%
\begin{pgfscope}%
\pgfsys@transformshift{0.609375in}{2.095916in}%
\pgfsys@useobject{currentmarker}{}%
\end{pgfscope}%
\end{pgfscope}%
\begin{pgfscope}%
\pgftext[x=0.394096in,y=2.057654in,left,base]{\rmfamily\fontsize{8.000000}{9.600000}\selectfont \(\displaystyle 12\)}%
\end{pgfscope}%
\begin{pgfscope}%
\pgfsetbuttcap%
\pgfsetroundjoin%
\definecolor{currentfill}{rgb}{0.000000,0.000000,0.000000}%
\pgfsetfillcolor{currentfill}%
\pgfsetlinewidth{0.803000pt}%
\definecolor{currentstroke}{rgb}{0.000000,0.000000,0.000000}%
\pgfsetstrokecolor{currentstroke}%
\pgfsetdash{}{0pt}%
\pgfsys@defobject{currentmarker}{\pgfqpoint{-0.048611in}{0.000000in}}{\pgfqpoint{0.000000in}{0.000000in}}{%
\pgfpathmoveto{\pgfqpoint{0.000000in}{0.000000in}}%
\pgfpathlineto{\pgfqpoint{-0.048611in}{0.000000in}}%
\pgfusepath{stroke,fill}%
}%
\begin{pgfscope}%
\pgfsys@transformshift{0.609375in}{2.380265in}%
\pgfsys@useobject{currentmarker}{}%
\end{pgfscope}%
\end{pgfscope}%
\begin{pgfscope}%
\pgftext[x=0.394096in,y=2.342003in,left,base]{\rmfamily\fontsize{8.000000}{9.600000}\selectfont \(\displaystyle 14\)}%
\end{pgfscope}%
\begin{pgfscope}%
\pgftext[x=0.338540in,y=1.513990in,,bottom,rotate=90.000000]{\rmfamily\fontsize{10.000000}{12.000000}\selectfont Regret}%
\end{pgfscope}%
\begin{pgfscope}%
\pgfpathrectangle{\pgfqpoint{0.609375in}{0.376614in}}{\pgfqpoint{3.778125in}{2.274751in}} %
\pgfusepath{clip}%
\pgfsetrectcap%
\pgfsetroundjoin%
\pgfsetlinewidth{1.003750pt}%
\definecolor{currentstroke}{rgb}{0.121569,0.466667,0.705882}%
\pgfsetstrokecolor{currentstroke}%
\pgfsetdash{}{0pt}%
\pgfpathmoveto{\pgfqpoint{0.781108in}{0.483197in}}%
\pgfpathlineto{\pgfqpoint{0.794902in}{0.559792in}}%
\pgfpathlineto{\pgfqpoint{0.808696in}{0.620653in}}%
\pgfpathlineto{\pgfqpoint{0.836283in}{0.726832in}}%
\pgfpathlineto{\pgfqpoint{0.863871in}{0.811020in}}%
\pgfpathlineto{\pgfqpoint{0.891458in}{0.884214in}}%
\pgfpathlineto{\pgfqpoint{0.919046in}{0.946698in}}%
\pgfpathlineto{\pgfqpoint{0.974221in}{1.064745in}}%
\pgfpathlineto{\pgfqpoint{1.015603in}{1.134443in}}%
\pgfpathlineto{\pgfqpoint{1.043190in}{1.163563in}}%
\pgfpathlineto{\pgfqpoint{1.070778in}{1.196474in}}%
\pgfpathlineto{\pgfqpoint{1.098366in}{1.231091in}}%
\pgfpathlineto{\pgfqpoint{1.112159in}{1.243424in}}%
\pgfpathlineto{\pgfqpoint{1.153541in}{1.290942in}}%
\pgfpathlineto{\pgfqpoint{1.181128in}{1.325749in}}%
\pgfpathlineto{\pgfqpoint{1.194922in}{1.339503in}}%
\pgfpathlineto{\pgfqpoint{1.208716in}{1.355911in}}%
\pgfpathlineto{\pgfqpoint{1.222510in}{1.368149in}}%
\pgfpathlineto{\pgfqpoint{1.236304in}{1.377543in}}%
\pgfpathlineto{\pgfqpoint{1.277685in}{1.419564in}}%
\pgfpathlineto{\pgfqpoint{1.305273in}{1.441386in}}%
\pgfpathlineto{\pgfqpoint{1.319067in}{1.455235in}}%
\pgfpathlineto{\pgfqpoint{1.332860in}{1.463776in}}%
\pgfpathlineto{\pgfqpoint{1.374242in}{1.499447in}}%
\pgfpathlineto{\pgfqpoint{1.388036in}{1.506850in}}%
\pgfpathlineto{\pgfqpoint{1.401829in}{1.517003in}}%
\pgfpathlineto{\pgfqpoint{1.429417in}{1.546407in}}%
\pgfpathlineto{\pgfqpoint{1.457005in}{1.567376in}}%
\pgfpathlineto{\pgfqpoint{1.470799in}{1.574685in}}%
\pgfpathlineto{\pgfqpoint{1.484592in}{1.586543in}}%
\pgfpathlineto{\pgfqpoint{1.512180in}{1.602204in}}%
\pgfpathlineto{\pgfqpoint{1.567355in}{1.644804in}}%
\pgfpathlineto{\pgfqpoint{1.594943in}{1.655252in}}%
\pgfpathlineto{\pgfqpoint{1.608737in}{1.668532in}}%
\pgfpathlineto{\pgfqpoint{1.622530in}{1.676221in}}%
\pgfpathlineto{\pgfqpoint{1.636324in}{1.680591in}}%
\pgfpathlineto{\pgfqpoint{1.650118in}{1.692071in}}%
\pgfpathlineto{\pgfqpoint{1.663912in}{1.699664in}}%
\pgfpathlineto{\pgfqpoint{1.677706in}{1.701002in}}%
\pgfpathlineto{\pgfqpoint{1.691500in}{1.710396in}}%
\pgfpathlineto{\pgfqpoint{1.705293in}{1.717610in}}%
\pgfpathlineto{\pgfqpoint{1.719087in}{1.727099in}}%
\pgfpathlineto{\pgfqpoint{1.732881in}{1.731375in}}%
\pgfpathlineto{\pgfqpoint{1.746675in}{1.742950in}}%
\pgfpathlineto{\pgfqpoint{1.788056in}{1.769900in}}%
\pgfpathlineto{\pgfqpoint{1.815644in}{1.787077in}}%
\pgfpathlineto{\pgfqpoint{1.829438in}{1.793912in}}%
\pgfpathlineto{\pgfqpoint{1.843231in}{1.797146in}}%
\pgfpathlineto{\pgfqpoint{1.870819in}{1.811006in}}%
\pgfpathlineto{\pgfqpoint{1.884613in}{1.817367in}}%
\pgfpathlineto{\pgfqpoint{1.912201in}{1.836914in}}%
\pgfpathlineto{\pgfqpoint{1.925994in}{1.840716in}}%
\pgfpathlineto{\pgfqpoint{1.939788in}{1.849067in}}%
\pgfpathlineto{\pgfqpoint{1.953582in}{1.852775in}}%
\pgfpathlineto{\pgfqpoint{1.994963in}{1.871195in}}%
\pgfpathlineto{\pgfqpoint{2.008757in}{1.875850in}}%
\pgfpathlineto{\pgfqpoint{2.050139in}{1.897208in}}%
\pgfpathlineto{\pgfqpoint{2.063932in}{1.899115in}}%
\pgfpathlineto{\pgfqpoint{2.077726in}{1.905665in}}%
\pgfpathlineto{\pgfqpoint{2.091520in}{1.909088in}}%
\pgfpathlineto{\pgfqpoint{2.105314in}{1.917819in}}%
\pgfpathlineto{\pgfqpoint{2.119108in}{1.921431in}}%
\pgfpathlineto{\pgfqpoint{2.132901in}{1.929783in}}%
\pgfpathlineto{\pgfqpoint{2.146695in}{1.935481in}}%
\pgfpathlineto{\pgfqpoint{2.160489in}{1.938430in}}%
\pgfpathlineto{\pgfqpoint{2.174283in}{1.951995in}}%
\pgfpathlineto{\pgfqpoint{2.201871in}{1.965191in}}%
\pgfpathlineto{\pgfqpoint{2.215664in}{1.969846in}}%
\pgfpathlineto{\pgfqpoint{2.243252in}{1.984843in}}%
\pgfpathlineto{\pgfqpoint{2.270840in}{1.990172in}}%
\pgfpathlineto{\pgfqpoint{2.284633in}{1.998145in}}%
\pgfpathlineto{\pgfqpoint{2.298427in}{2.002895in}}%
\pgfpathlineto{\pgfqpoint{2.312221in}{2.011436in}}%
\pgfpathlineto{\pgfqpoint{2.339809in}{2.023305in}}%
\pgfpathlineto{\pgfqpoint{2.353602in}{2.027676in}}%
\pgfpathlineto{\pgfqpoint{2.367396in}{2.034985in}}%
\pgfpathlineto{\pgfqpoint{2.422572in}{2.055312in}}%
\pgfpathlineto{\pgfqpoint{2.436365in}{2.061199in}}%
\pgfpathlineto{\pgfqpoint{2.450159in}{2.065096in}}%
\pgfpathlineto{\pgfqpoint{2.463953in}{2.065391in}}%
\pgfpathlineto{\pgfqpoint{2.491541in}{2.072616in}}%
\pgfpathlineto{\pgfqpoint{2.505334in}{2.077366in}}%
\pgfpathlineto{\pgfqpoint{2.519128in}{2.080315in}}%
\pgfpathlineto{\pgfqpoint{2.546716in}{2.094459in}}%
\pgfpathlineto{\pgfqpoint{2.560510in}{2.095038in}}%
\pgfpathlineto{\pgfqpoint{2.574303in}{2.101494in}}%
\pgfpathlineto{\pgfqpoint{2.588097in}{2.103685in}}%
\pgfpathlineto{\pgfqpoint{2.601891in}{2.113838in}}%
\pgfpathlineto{\pgfqpoint{2.615685in}{2.116597in}}%
\pgfpathlineto{\pgfqpoint{2.629479in}{2.128930in}}%
\pgfpathlineto{\pgfqpoint{2.643273in}{2.131499in}}%
\pgfpathlineto{\pgfqpoint{2.657066in}{2.135681in}}%
\pgfpathlineto{\pgfqpoint{2.670860in}{2.137208in}}%
\pgfpathlineto{\pgfqpoint{2.684654in}{2.143853in}}%
\pgfpathlineto{\pgfqpoint{2.698448in}{2.154196in}}%
\pgfpathlineto{\pgfqpoint{2.712242in}{2.160178in}}%
\pgfpathlineto{\pgfqpoint{2.739829in}{2.165412in}}%
\pgfpathlineto{\pgfqpoint{2.753623in}{2.170352in}}%
\pgfpathlineto{\pgfqpoint{2.767417in}{2.179935in}}%
\pgfpathlineto{\pgfqpoint{2.781211in}{2.184022in}}%
\pgfpathlineto{\pgfqpoint{2.795004in}{2.186307in}}%
\pgfpathlineto{\pgfqpoint{2.808798in}{2.194469in}}%
\pgfpathlineto{\pgfqpoint{2.836386in}{2.206244in}}%
\pgfpathlineto{\pgfqpoint{2.850180in}{2.207108in}}%
\pgfpathlineto{\pgfqpoint{2.863974in}{2.213185in}}%
\pgfpathlineto{\pgfqpoint{2.877767in}{2.215660in}}%
\pgfpathlineto{\pgfqpoint{2.891561in}{2.223727in}}%
\pgfpathlineto{\pgfqpoint{2.905355in}{2.226392in}}%
\pgfpathlineto{\pgfqpoint{2.919149in}{2.236639in}}%
\pgfpathlineto{\pgfqpoint{2.946736in}{2.244054in}}%
\pgfpathlineto{\pgfqpoint{2.960530in}{2.254301in}}%
\pgfpathlineto{\pgfqpoint{2.974324in}{2.259999in}}%
\pgfpathlineto{\pgfqpoint{2.988118in}{2.263421in}}%
\pgfpathlineto{\pgfqpoint{3.001912in}{2.263432in}}%
\pgfpathlineto{\pgfqpoint{3.015705in}{2.270552in}}%
\pgfpathlineto{\pgfqpoint{3.029499in}{2.274828in}}%
\pgfpathlineto{\pgfqpoint{3.043293in}{2.274270in}}%
\pgfpathlineto{\pgfqpoint{3.057087in}{2.280820in}}%
\pgfpathlineto{\pgfqpoint{3.070881in}{2.279315in}}%
\pgfpathlineto{\pgfqpoint{3.098468in}{2.290047in}}%
\pgfpathlineto{\pgfqpoint{3.112262in}{2.296503in}}%
\pgfpathlineto{\pgfqpoint{3.139850in}{2.304201in}}%
\pgfpathlineto{\pgfqpoint{3.153644in}{2.307340in}}%
\pgfpathlineto{\pgfqpoint{3.167437in}{2.308773in}}%
\pgfpathlineto{\pgfqpoint{3.195025in}{2.316282in}}%
\pgfpathlineto{\pgfqpoint{3.263994in}{2.332543in}}%
\pgfpathlineto{\pgfqpoint{3.277788in}{2.335208in}}%
\pgfpathlineto{\pgfqpoint{3.305375in}{2.337884in}}%
\pgfpathlineto{\pgfqpoint{3.346757in}{2.347299in}}%
\pgfpathlineto{\pgfqpoint{3.360551in}{2.348542in}}%
\pgfpathlineto{\pgfqpoint{3.374345in}{2.354050in}}%
\pgfpathlineto{\pgfqpoint{3.388138in}{2.351786in}}%
\pgfpathlineto{\pgfqpoint{3.415726in}{2.358158in}}%
\pgfpathlineto{\pgfqpoint{3.429520in}{2.355420in}}%
\pgfpathlineto{\pgfqpoint{3.443314in}{2.363393in}}%
\pgfpathlineto{\pgfqpoint{3.457107in}{2.367669in}}%
\pgfpathlineto{\pgfqpoint{3.484695in}{2.382950in}}%
\pgfpathlineto{\pgfqpoint{3.512283in}{2.388469in}}%
\pgfpathlineto{\pgfqpoint{3.526076in}{2.391039in}}%
\pgfpathlineto{\pgfqpoint{3.539870in}{2.398064in}}%
\pgfpathlineto{\pgfqpoint{3.553664in}{2.401581in}}%
\pgfpathlineto{\pgfqpoint{3.567458in}{2.399697in}}%
\pgfpathlineto{\pgfqpoint{3.581252in}{2.400750in}}%
\pgfpathlineto{\pgfqpoint{3.608839in}{2.406932in}}%
\pgfpathlineto{\pgfqpoint{3.622633in}{2.414241in}}%
\pgfpathlineto{\pgfqpoint{3.636427in}{2.418138in}}%
\pgfpathlineto{\pgfqpoint{3.650221in}{2.419950in}}%
\pgfpathlineto{\pgfqpoint{3.664015in}{2.418065in}}%
\pgfpathlineto{\pgfqpoint{3.677808in}{2.423004in}}%
\pgfpathlineto{\pgfqpoint{3.691602in}{2.422920in}}%
\pgfpathlineto{\pgfqpoint{3.705396in}{2.425775in}}%
\pgfpathlineto{\pgfqpoint{3.719190in}{2.432230in}}%
\pgfpathlineto{\pgfqpoint{3.732984in}{2.434326in}}%
\pgfpathlineto{\pgfqpoint{3.746777in}{2.433674in}}%
\pgfpathlineto{\pgfqpoint{3.760571in}{2.437002in}}%
\pgfpathlineto{\pgfqpoint{3.774365in}{2.442700in}}%
\pgfpathlineto{\pgfqpoint{3.788159in}{2.450577in}}%
\pgfpathlineto{\pgfqpoint{3.829540in}{2.458666in}}%
\pgfpathlineto{\pgfqpoint{3.843334in}{2.459340in}}%
\pgfpathlineto{\pgfqpoint{3.857128in}{2.463332in}}%
\pgfpathlineto{\pgfqpoint{3.884716in}{2.468093in}}%
\pgfpathlineto{\pgfqpoint{3.898509in}{2.467155in}}%
\pgfpathlineto{\pgfqpoint{3.912303in}{2.474654in}}%
\pgfpathlineto{\pgfqpoint{3.926097in}{2.470684in}}%
\pgfpathlineto{\pgfqpoint{3.939891in}{2.476761in}}%
\pgfpathlineto{\pgfqpoint{3.953685in}{2.473833in}}%
\pgfpathlineto{\pgfqpoint{3.967478in}{2.475266in}}%
\pgfpathlineto{\pgfqpoint{3.995066in}{2.485524in}}%
\pgfpathlineto{\pgfqpoint{4.008860in}{2.489041in}}%
\pgfpathlineto{\pgfqpoint{4.022654in}{2.487630in}}%
\pgfpathlineto{\pgfqpoint{4.036447in}{2.492665in}}%
\pgfpathlineto{\pgfqpoint{4.050241in}{2.502343in}}%
\pgfpathlineto{\pgfqpoint{4.091623in}{2.518394in}}%
\pgfpathlineto{\pgfqpoint{4.105417in}{2.525324in}}%
\pgfpathlineto{\pgfqpoint{4.133004in}{2.533781in}}%
\pgfpathlineto{\pgfqpoint{4.160592in}{2.535319in}}%
\pgfpathlineto{\pgfqpoint{4.174386in}{2.538552in}}%
\pgfpathlineto{\pgfqpoint{4.188179in}{2.543776in}}%
\pgfpathlineto{\pgfqpoint{4.201973in}{2.546820in}}%
\pgfpathlineto{\pgfqpoint{4.215767in}{2.547968in}}%
\pgfpathlineto{\pgfqpoint{4.215767in}{2.547968in}}%
\pgfusepath{stroke}%
\end{pgfscope}%
\begin{pgfscope}%
\pgfpathrectangle{\pgfqpoint{0.609375in}{0.376614in}}{\pgfqpoint{3.778125in}{2.274751in}} %
\pgfusepath{clip}%
\pgfsetbuttcap%
\pgfsetroundjoin%
\pgfsetlinewidth{1.003750pt}%
\definecolor{currentstroke}{rgb}{1.000000,0.498039,0.054902}%
\pgfsetstrokecolor{currentstroke}%
\pgfsetdash{{1.000000pt}{1.650000pt}}{0.000000pt}%
\pgfpathmoveto{\pgfqpoint{0.781108in}{0.480012in}}%
\pgfpathlineto{\pgfqpoint{0.794902in}{0.535509in}}%
\pgfpathlineto{\pgfqpoint{0.808696in}{0.575154in}}%
\pgfpathlineto{\pgfqpoint{0.836283in}{0.641149in}}%
\pgfpathlineto{\pgfqpoint{0.850077in}{0.671623in}}%
\pgfpathlineto{\pgfqpoint{0.891458in}{0.749966in}}%
\pgfpathlineto{\pgfqpoint{0.919046in}{0.785963in}}%
\pgfpathlineto{\pgfqpoint{0.946634in}{0.829353in}}%
\pgfpathlineto{\pgfqpoint{0.974221in}{0.862009in}}%
\pgfpathlineto{\pgfqpoint{0.988015in}{0.874853in}}%
\pgfpathlineto{\pgfqpoint{1.015603in}{0.908078in}}%
\pgfpathlineto{\pgfqpoint{1.056984in}{0.945047in}}%
\pgfpathlineto{\pgfqpoint{1.070778in}{0.959669in}}%
\pgfpathlineto{\pgfqpoint{1.112159in}{0.997066in}}%
\pgfpathlineto{\pgfqpoint{1.139747in}{1.014438in}}%
\pgfpathlineto{\pgfqpoint{1.153541in}{1.027638in}}%
\pgfpathlineto{\pgfqpoint{1.167335in}{1.034795in}}%
\pgfpathlineto{\pgfqpoint{1.181128in}{1.043943in}}%
\pgfpathlineto{\pgfqpoint{1.208716in}{1.059041in}}%
\pgfpathlineto{\pgfqpoint{1.250098in}{1.085490in}}%
\pgfpathlineto{\pgfqpoint{1.277685in}{1.107056in}}%
\pgfpathlineto{\pgfqpoint{1.305273in}{1.125707in}}%
\pgfpathlineto{\pgfqpoint{1.319067in}{1.132367in}}%
\pgfpathlineto{\pgfqpoint{1.388036in}{1.179316in}}%
\pgfpathlineto{\pgfqpoint{1.429417in}{1.199225in}}%
\pgfpathlineto{\pgfqpoint{1.443211in}{1.211785in}}%
\pgfpathlineto{\pgfqpoint{1.457005in}{1.220009in}}%
\pgfpathlineto{\pgfqpoint{1.470799in}{1.221906in}}%
\pgfpathlineto{\pgfqpoint{1.512180in}{1.244161in}}%
\pgfpathlineto{\pgfqpoint{1.525974in}{1.250537in}}%
\pgfpathlineto{\pgfqpoint{1.539768in}{1.260822in}}%
\pgfpathlineto{\pgfqpoint{1.553561in}{1.267838in}}%
\pgfpathlineto{\pgfqpoint{1.567355in}{1.277412in}}%
\pgfpathlineto{\pgfqpoint{1.581149in}{1.284073in}}%
\pgfpathlineto{\pgfqpoint{1.594943in}{1.293647in}}%
\pgfpathlineto{\pgfqpoint{1.608737in}{1.300449in}}%
\pgfpathlineto{\pgfqpoint{1.622530in}{1.305332in}}%
\pgfpathlineto{\pgfqpoint{1.636324in}{1.312561in}}%
\pgfpathlineto{\pgfqpoint{1.650118in}{1.314245in}}%
\pgfpathlineto{\pgfqpoint{1.677706in}{1.326570in}}%
\pgfpathlineto{\pgfqpoint{1.691500in}{1.327400in}}%
\pgfpathlineto{\pgfqpoint{1.719087in}{1.341360in}}%
\pgfpathlineto{\pgfqpoint{1.732881in}{1.342902in}}%
\pgfpathlineto{\pgfqpoint{1.746675in}{1.346150in}}%
\pgfpathlineto{\pgfqpoint{1.760469in}{1.351104in}}%
\pgfpathlineto{\pgfqpoint{1.774262in}{1.353499in}}%
\pgfpathlineto{\pgfqpoint{1.788056in}{1.358950in}}%
\pgfpathlineto{\pgfqpoint{1.815644in}{1.359830in}}%
\pgfpathlineto{\pgfqpoint{1.829438in}{1.359808in}}%
\pgfpathlineto{\pgfqpoint{1.843231in}{1.366610in}}%
\pgfpathlineto{\pgfqpoint{1.857025in}{1.375758in}}%
\pgfpathlineto{\pgfqpoint{1.870819in}{1.380925in}}%
\pgfpathlineto{\pgfqpoint{1.884613in}{1.388509in}}%
\pgfpathlineto{\pgfqpoint{1.898407in}{1.387137in}}%
\pgfpathlineto{\pgfqpoint{1.912201in}{1.395432in}}%
\pgfpathlineto{\pgfqpoint{1.939788in}{1.404131in}}%
\pgfpathlineto{\pgfqpoint{1.953582in}{1.407165in}}%
\pgfpathlineto{\pgfqpoint{1.967376in}{1.412048in}}%
\pgfpathlineto{\pgfqpoint{1.981170in}{1.414159in}}%
\pgfpathlineto{\pgfqpoint{1.994963in}{1.417833in}}%
\pgfpathlineto{\pgfqpoint{2.008757in}{1.423853in}}%
\pgfpathlineto{\pgfqpoint{2.036345in}{1.432979in}}%
\pgfpathlineto{\pgfqpoint{2.050139in}{1.439710in}}%
\pgfpathlineto{\pgfqpoint{2.063932in}{1.444167in}}%
\pgfpathlineto{\pgfqpoint{2.091520in}{1.457700in}}%
\pgfpathlineto{\pgfqpoint{2.105314in}{1.458886in}}%
\pgfpathlineto{\pgfqpoint{2.119108in}{1.463911in}}%
\pgfpathlineto{\pgfqpoint{2.132901in}{1.465453in}}%
\pgfpathlineto{\pgfqpoint{2.146695in}{1.471758in}}%
\pgfpathlineto{\pgfqpoint{2.188077in}{1.483847in}}%
\pgfpathlineto{\pgfqpoint{2.201871in}{1.490152in}}%
\pgfpathlineto{\pgfqpoint{2.243252in}{1.500464in}}%
\pgfpathlineto{\pgfqpoint{2.257046in}{1.504778in}}%
\pgfpathlineto{\pgfqpoint{2.284633in}{1.518667in}}%
\pgfpathlineto{\pgfqpoint{2.298427in}{1.523550in}}%
\pgfpathlineto{\pgfqpoint{2.312221in}{1.525660in}}%
\pgfpathlineto{\pgfqpoint{2.326015in}{1.530756in}}%
\pgfpathlineto{\pgfqpoint{2.339809in}{1.531160in}}%
\pgfpathlineto{\pgfqpoint{2.353602in}{1.538673in}}%
\pgfpathlineto{\pgfqpoint{2.367396in}{1.543983in}}%
\pgfpathlineto{\pgfqpoint{2.394984in}{1.548844in}}%
\pgfpathlineto{\pgfqpoint{2.408778in}{1.550243in}}%
\pgfpathlineto{\pgfqpoint{2.450159in}{1.568233in}}%
\pgfpathlineto{\pgfqpoint{2.477747in}{1.576577in}}%
\pgfpathlineto{\pgfqpoint{2.491541in}{1.581317in}}%
\pgfpathlineto{\pgfqpoint{2.505334in}{1.588048in}}%
\pgfpathlineto{\pgfqpoint{2.519128in}{1.592149in}}%
\pgfpathlineto{\pgfqpoint{2.532922in}{1.594473in}}%
\pgfpathlineto{\pgfqpoint{2.546716in}{1.601062in}}%
\pgfpathlineto{\pgfqpoint{2.560510in}{1.603030in}}%
\pgfpathlineto{\pgfqpoint{2.615685in}{1.618438in}}%
\pgfpathlineto{\pgfqpoint{2.629479in}{1.620265in}}%
\pgfpathlineto{\pgfqpoint{2.643273in}{1.626711in}}%
\pgfpathlineto{\pgfqpoint{2.670860in}{1.635126in}}%
\pgfpathlineto{\pgfqpoint{2.698448in}{1.648660in}}%
\pgfpathlineto{\pgfqpoint{2.712242in}{1.649348in}}%
\pgfpathlineto{\pgfqpoint{2.726035in}{1.656151in}}%
\pgfpathlineto{\pgfqpoint{2.739829in}{1.661104in}}%
\pgfpathlineto{\pgfqpoint{2.767417in}{1.663406in}}%
\pgfpathlineto{\pgfqpoint{2.781211in}{1.666725in}}%
\pgfpathlineto{\pgfqpoint{2.808798in}{1.663482in}}%
\pgfpathlineto{\pgfqpoint{2.822592in}{1.664739in}}%
\pgfpathlineto{\pgfqpoint{2.877767in}{1.681640in}}%
\pgfpathlineto{\pgfqpoint{2.891561in}{1.681973in}}%
\pgfpathlineto{\pgfqpoint{2.932943in}{1.693494in}}%
\pgfpathlineto{\pgfqpoint{2.974324in}{1.703806in}}%
\pgfpathlineto{\pgfqpoint{3.001912in}{1.715562in}}%
\pgfpathlineto{\pgfqpoint{3.015705in}{1.716180in}}%
\pgfpathlineto{\pgfqpoint{3.029499in}{1.719215in}}%
\pgfpathlineto{\pgfqpoint{3.043293in}{1.718126in}}%
\pgfpathlineto{\pgfqpoint{3.057087in}{1.722085in}}%
\pgfpathlineto{\pgfqpoint{3.070881in}{1.723627in}}%
\pgfpathlineto{\pgfqpoint{3.098468in}{1.734174in}}%
\pgfpathlineto{\pgfqpoint{3.112262in}{1.737920in}}%
\pgfpathlineto{\pgfqpoint{3.126056in}{1.743442in}}%
\pgfpathlineto{\pgfqpoint{3.153644in}{1.749654in}}%
\pgfpathlineto{\pgfqpoint{3.167437in}{1.756100in}}%
\pgfpathlineto{\pgfqpoint{3.181231in}{1.757642in}}%
\pgfpathlineto{\pgfqpoint{3.195025in}{1.764515in}}%
\pgfpathlineto{\pgfqpoint{3.222613in}{1.760775in}}%
\pgfpathlineto{\pgfqpoint{3.236406in}{1.765729in}}%
\pgfpathlineto{\pgfqpoint{3.250200in}{1.765564in}}%
\pgfpathlineto{\pgfqpoint{3.263994in}{1.767675in}}%
\pgfpathlineto{\pgfqpoint{3.277788in}{1.766444in}}%
\pgfpathlineto{\pgfqpoint{3.291582in}{1.767346in}}%
\pgfpathlineto{\pgfqpoint{3.305375in}{1.773722in}}%
\pgfpathlineto{\pgfqpoint{3.332963in}{1.776307in}}%
\pgfpathlineto{\pgfqpoint{3.346757in}{1.780977in}}%
\pgfpathlineto{\pgfqpoint{3.360551in}{1.787708in}}%
\pgfpathlineto{\pgfqpoint{3.374345in}{1.791738in}}%
\pgfpathlineto{\pgfqpoint{3.388138in}{1.800957in}}%
\pgfpathlineto{\pgfqpoint{3.401932in}{1.801717in}}%
\pgfpathlineto{\pgfqpoint{3.415726in}{1.807524in}}%
\pgfpathlineto{\pgfqpoint{3.484695in}{1.824829in}}%
\pgfpathlineto{\pgfqpoint{3.498489in}{1.824949in}}%
\pgfpathlineto{\pgfqpoint{3.512283in}{1.829477in}}%
\pgfpathlineto{\pgfqpoint{3.526076in}{1.829739in}}%
\pgfpathlineto{\pgfqpoint{3.553664in}{1.841140in}}%
\pgfpathlineto{\pgfqpoint{3.581252in}{1.845787in}}%
\pgfpathlineto{\pgfqpoint{3.608839in}{1.854984in}}%
\pgfpathlineto{\pgfqpoint{3.622633in}{1.853824in}}%
\pgfpathlineto{\pgfqpoint{3.677808in}{1.867953in}}%
\pgfpathlineto{\pgfqpoint{3.705396in}{1.871392in}}%
\pgfpathlineto{\pgfqpoint{3.719190in}{1.872578in}}%
\pgfpathlineto{\pgfqpoint{3.746777in}{1.880567in}}%
\pgfpathlineto{\pgfqpoint{3.760571in}{1.886161in}}%
\pgfpathlineto{\pgfqpoint{3.774365in}{1.886778in}}%
\pgfpathlineto{\pgfqpoint{3.788159in}{1.891732in}}%
\pgfpathlineto{\pgfqpoint{3.815747in}{1.894318in}}%
\pgfpathlineto{\pgfqpoint{3.829540in}{1.893727in}}%
\pgfpathlineto{\pgfqpoint{3.857128in}{1.904133in}}%
\pgfpathlineto{\pgfqpoint{3.870922in}{1.905532in}}%
\pgfpathlineto{\pgfqpoint{3.884716in}{1.911410in}}%
\pgfpathlineto{\pgfqpoint{3.898509in}{1.909611in}}%
\pgfpathlineto{\pgfqpoint{3.912303in}{1.905963in}}%
\pgfpathlineto{\pgfqpoint{3.926097in}{1.907221in}}%
\pgfpathlineto{\pgfqpoint{3.939891in}{1.905279in}}%
\pgfpathlineto{\pgfqpoint{3.967478in}{1.910495in}}%
\pgfpathlineto{\pgfqpoint{3.981272in}{1.912961in}}%
\pgfpathlineto{\pgfqpoint{4.008860in}{1.913130in}}%
\pgfpathlineto{\pgfqpoint{4.050241in}{1.923158in}}%
\pgfpathlineto{\pgfqpoint{4.064035in}{1.924060in}}%
\pgfpathlineto{\pgfqpoint{4.077829in}{1.922687in}}%
\pgfpathlineto{\pgfqpoint{4.133004in}{1.928925in}}%
\pgfpathlineto{\pgfqpoint{4.160592in}{1.943169in}}%
\pgfpathlineto{\pgfqpoint{4.201973in}{1.945164in}}%
\pgfpathlineto{\pgfqpoint{4.215767in}{1.950829in}}%
\pgfpathlineto{\pgfqpoint{4.215767in}{1.950829in}}%
\pgfusepath{stroke}%
\end{pgfscope}%
\begin{pgfscope}%
\pgfpathrectangle{\pgfqpoint{0.609375in}{0.376614in}}{\pgfqpoint{3.778125in}{2.274751in}} %
\pgfusepath{clip}%
\pgfsetbuttcap%
\pgfsetroundjoin%
\pgfsetlinewidth{1.003750pt}%
\definecolor{currentstroke}{rgb}{0.172549,0.627451,0.172549}%
\pgfsetstrokecolor{currentstroke}%
\pgfsetdash{{3.700000pt}{1.600000pt}}{0.000000pt}%
\pgfpathmoveto{\pgfqpoint{0.781108in}{0.485851in}}%
\pgfpathlineto{\pgfqpoint{0.794902in}{0.543015in}}%
\pgfpathlineto{\pgfqpoint{0.822489in}{0.620854in}}%
\pgfpathlineto{\pgfqpoint{0.850077in}{0.680967in}}%
\pgfpathlineto{\pgfqpoint{0.863871in}{0.708276in}}%
\pgfpathlineto{\pgfqpoint{0.932840in}{0.817898in}}%
\pgfpathlineto{\pgfqpoint{0.960428in}{0.858582in}}%
\pgfpathlineto{\pgfqpoint{1.001809in}{0.905626in}}%
\pgfpathlineto{\pgfqpoint{1.015603in}{0.919191in}}%
\pgfpathlineto{\pgfqpoint{1.029397in}{0.929343in}}%
\pgfpathlineto{\pgfqpoint{1.056984in}{0.946615in}}%
\pgfpathlineto{\pgfqpoint{1.098366in}{0.978021in}}%
\pgfpathlineto{\pgfqpoint{1.112159in}{0.985330in}}%
\pgfpathlineto{\pgfqpoint{1.153541in}{1.018252in}}%
\pgfpathlineto{\pgfqpoint{1.167335in}{1.032385in}}%
\pgfpathlineto{\pgfqpoint{1.181128in}{1.044244in}}%
\pgfpathlineto{\pgfqpoint{1.194922in}{1.051079in}}%
\pgfpathlineto{\pgfqpoint{1.208716in}{1.065212in}}%
\pgfpathlineto{\pgfqpoint{1.222510in}{1.074606in}}%
\pgfpathlineto{\pgfqpoint{1.236304in}{1.080778in}}%
\pgfpathlineto{\pgfqpoint{1.263891in}{1.101936in}}%
\pgfpathlineto{\pgfqpoint{1.277685in}{1.111046in}}%
\pgfpathlineto{\pgfqpoint{1.305273in}{1.125380in}}%
\pgfpathlineto{\pgfqpoint{1.319067in}{1.135248in}}%
\pgfpathlineto{\pgfqpoint{1.332860in}{1.139713in}}%
\pgfpathlineto{\pgfqpoint{1.346654in}{1.146643in}}%
\pgfpathlineto{\pgfqpoint{1.374242in}{1.166285in}}%
\pgfpathlineto{\pgfqpoint{1.388036in}{1.173025in}}%
\pgfpathlineto{\pgfqpoint{1.401829in}{1.176164in}}%
\pgfpathlineto{\pgfqpoint{1.443211in}{1.205010in}}%
\pgfpathlineto{\pgfqpoint{1.484592in}{1.226274in}}%
\pgfpathlineto{\pgfqpoint{1.498386in}{1.234720in}}%
\pgfpathlineto{\pgfqpoint{1.512180in}{1.238996in}}%
\pgfpathlineto{\pgfqpoint{1.553561in}{1.264335in}}%
\pgfpathlineto{\pgfqpoint{1.594943in}{1.281997in}}%
\pgfpathlineto{\pgfqpoint{1.608737in}{1.291865in}}%
\pgfpathlineto{\pgfqpoint{1.663912in}{1.318163in}}%
\pgfpathlineto{\pgfqpoint{1.677706in}{1.318932in}}%
\pgfpathlineto{\pgfqpoint{1.691500in}{1.323398in}}%
\pgfpathlineto{\pgfqpoint{1.705293in}{1.325304in}}%
\pgfpathlineto{\pgfqpoint{1.719087in}{1.331855in}}%
\pgfpathlineto{\pgfqpoint{1.732881in}{1.334046in}}%
\pgfpathlineto{\pgfqpoint{1.774262in}{1.359195in}}%
\pgfpathlineto{\pgfqpoint{1.788056in}{1.363471in}}%
\pgfpathlineto{\pgfqpoint{1.815644in}{1.377047in}}%
\pgfpathlineto{\pgfqpoint{1.898407in}{1.404219in}}%
\pgfpathlineto{\pgfqpoint{1.912201in}{1.412760in}}%
\pgfpathlineto{\pgfqpoint{1.925994in}{1.415899in}}%
\pgfpathlineto{\pgfqpoint{1.953582in}{1.425304in}}%
\pgfpathlineto{\pgfqpoint{1.967376in}{1.430054in}}%
\pgfpathlineto{\pgfqpoint{1.981170in}{1.437457in}}%
\pgfpathlineto{\pgfqpoint{2.036345in}{1.458068in}}%
\pgfpathlineto{\pgfqpoint{2.050139in}{1.462344in}}%
\pgfpathlineto{\pgfqpoint{2.063932in}{1.462166in}}%
\pgfpathlineto{\pgfqpoint{2.105314in}{1.477079in}}%
\pgfpathlineto{\pgfqpoint{2.119108in}{1.481829in}}%
\pgfpathlineto{\pgfqpoint{2.132901in}{1.489043in}}%
\pgfpathlineto{\pgfqpoint{2.146695in}{1.492655in}}%
\pgfpathlineto{\pgfqpoint{2.160489in}{1.493709in}}%
\pgfpathlineto{\pgfqpoint{2.174283in}{1.504051in}}%
\pgfpathlineto{\pgfqpoint{2.188077in}{1.511170in}}%
\pgfpathlineto{\pgfqpoint{2.215664in}{1.520480in}}%
\pgfpathlineto{\pgfqpoint{2.229458in}{1.529116in}}%
\pgfpathlineto{\pgfqpoint{2.243252in}{1.535098in}}%
\pgfpathlineto{\pgfqpoint{2.270840in}{1.537205in}}%
\pgfpathlineto{\pgfqpoint{2.284633in}{1.543756in}}%
\pgfpathlineto{\pgfqpoint{2.298427in}{1.546231in}}%
\pgfpathlineto{\pgfqpoint{2.339809in}{1.563040in}}%
\pgfpathlineto{\pgfqpoint{2.353602in}{1.563714in}}%
\pgfpathlineto{\pgfqpoint{2.367396in}{1.568843in}}%
\pgfpathlineto{\pgfqpoint{2.422572in}{1.582535in}}%
\pgfpathlineto{\pgfqpoint{2.450159in}{1.590518in}}%
\pgfpathlineto{\pgfqpoint{2.463953in}{1.587970in}}%
\pgfpathlineto{\pgfqpoint{2.519128in}{1.598344in}}%
\pgfpathlineto{\pgfqpoint{2.546716in}{1.609360in}}%
\pgfpathlineto{\pgfqpoint{2.560510in}{1.610129in}}%
\pgfpathlineto{\pgfqpoint{2.574303in}{1.616585in}}%
\pgfpathlineto{\pgfqpoint{2.588097in}{1.618492in}}%
\pgfpathlineto{\pgfqpoint{2.601891in}{1.626275in}}%
\pgfpathlineto{\pgfqpoint{2.615685in}{1.629603in}}%
\pgfpathlineto{\pgfqpoint{2.629479in}{1.639092in}}%
\pgfpathlineto{\pgfqpoint{2.643273in}{1.640145in}}%
\pgfpathlineto{\pgfqpoint{2.670860in}{1.648223in}}%
\pgfpathlineto{\pgfqpoint{2.684654in}{1.654016in}}%
\pgfpathlineto{\pgfqpoint{2.698448in}{1.662557in}}%
\pgfpathlineto{\pgfqpoint{2.726035in}{1.670635in}}%
\pgfpathlineto{\pgfqpoint{2.739829in}{1.672920in}}%
\pgfpathlineto{\pgfqpoint{2.753623in}{1.678618in}}%
\pgfpathlineto{\pgfqpoint{2.767417in}{1.689339in}}%
\pgfpathlineto{\pgfqpoint{2.781211in}{1.691625in}}%
\pgfpathlineto{\pgfqpoint{2.795004in}{1.691446in}}%
\pgfpathlineto{\pgfqpoint{2.808798in}{1.698281in}}%
\pgfpathlineto{\pgfqpoint{2.822592in}{1.702178in}}%
\pgfpathlineto{\pgfqpoint{2.836386in}{1.709108in}}%
\pgfpathlineto{\pgfqpoint{2.850180in}{1.710351in}}%
\pgfpathlineto{\pgfqpoint{2.863974in}{1.714342in}}%
\pgfpathlineto{\pgfqpoint{2.877767in}{1.715301in}}%
\pgfpathlineto{\pgfqpoint{2.891561in}{1.720525in}}%
\pgfpathlineto{\pgfqpoint{2.905355in}{1.721862in}}%
\pgfpathlineto{\pgfqpoint{2.919149in}{1.732205in}}%
\pgfpathlineto{\pgfqpoint{2.932943in}{1.734869in}}%
\pgfpathlineto{\pgfqpoint{2.946736in}{1.739524in}}%
\pgfpathlineto{\pgfqpoint{2.960530in}{1.749203in}}%
\pgfpathlineto{\pgfqpoint{2.988118in}{1.755006in}}%
\pgfpathlineto{\pgfqpoint{3.001912in}{1.755491in}}%
\pgfpathlineto{\pgfqpoint{3.015705in}{1.762042in}}%
\pgfpathlineto{\pgfqpoint{3.029499in}{1.765180in}}%
\pgfpathlineto{\pgfqpoint{3.043293in}{1.764243in}}%
\pgfpathlineto{\pgfqpoint{3.057087in}{1.771552in}}%
\pgfpathlineto{\pgfqpoint{3.070881in}{1.770710in}}%
\pgfpathlineto{\pgfqpoint{3.084674in}{1.772711in}}%
\pgfpathlineto{\pgfqpoint{3.112262in}{1.783822in}}%
\pgfpathlineto{\pgfqpoint{3.126056in}{1.785349in}}%
\pgfpathlineto{\pgfqpoint{3.153644in}{1.792006in}}%
\pgfpathlineto{\pgfqpoint{3.167437in}{1.792301in}}%
\pgfpathlineto{\pgfqpoint{3.181231in}{1.797904in}}%
\pgfpathlineto{\pgfqpoint{3.208819in}{1.799916in}}%
\pgfpathlineto{\pgfqpoint{3.236406in}{1.806477in}}%
\pgfpathlineto{\pgfqpoint{3.250200in}{1.805445in}}%
\pgfpathlineto{\pgfqpoint{3.277788in}{1.809163in}}%
\pgfpathlineto{\pgfqpoint{3.346757in}{1.815852in}}%
\pgfpathlineto{\pgfqpoint{3.360551in}{1.816336in}}%
\pgfpathlineto{\pgfqpoint{3.374345in}{1.818527in}}%
\pgfpathlineto{\pgfqpoint{3.388138in}{1.815126in}}%
\pgfpathlineto{\pgfqpoint{3.415726in}{1.820266in}}%
\pgfpathlineto{\pgfqpoint{3.429520in}{1.819802in}}%
\pgfpathlineto{\pgfqpoint{3.443314in}{1.826353in}}%
\pgfpathlineto{\pgfqpoint{3.457107in}{1.829113in}}%
\pgfpathlineto{\pgfqpoint{3.470901in}{1.837275in}}%
\pgfpathlineto{\pgfqpoint{3.484695in}{1.841835in}}%
\pgfpathlineto{\pgfqpoint{3.512283in}{1.846501in}}%
\pgfpathlineto{\pgfqpoint{3.526076in}{1.848312in}}%
\pgfpathlineto{\pgfqpoint{3.539870in}{1.854200in}}%
\pgfpathlineto{\pgfqpoint{3.553664in}{1.858381in}}%
\pgfpathlineto{\pgfqpoint{3.567458in}{1.856875in}}%
\pgfpathlineto{\pgfqpoint{3.608839in}{1.860793in}}%
\pgfpathlineto{\pgfqpoint{3.622633in}{1.867913in}}%
\pgfpathlineto{\pgfqpoint{3.636427in}{1.871620in}}%
\pgfpathlineto{\pgfqpoint{3.650221in}{1.872579in}}%
\pgfpathlineto{\pgfqpoint{3.664015in}{1.871263in}}%
\pgfpathlineto{\pgfqpoint{3.677808in}{1.874591in}}%
\pgfpathlineto{\pgfqpoint{3.691602in}{1.873654in}}%
\pgfpathlineto{\pgfqpoint{3.705396in}{1.876413in}}%
\pgfpathlineto{\pgfqpoint{3.732984in}{1.887429in}}%
\pgfpathlineto{\pgfqpoint{3.760571in}{1.889726in}}%
\pgfpathlineto{\pgfqpoint{3.774365in}{1.891348in}}%
\pgfpathlineto{\pgfqpoint{3.788159in}{1.898278in}}%
\pgfpathlineto{\pgfqpoint{3.801953in}{1.900184in}}%
\pgfpathlineto{\pgfqpoint{3.815747in}{1.903796in}}%
\pgfpathlineto{\pgfqpoint{3.829540in}{1.904186in}}%
\pgfpathlineto{\pgfqpoint{3.843334in}{1.902491in}}%
\pgfpathlineto{\pgfqpoint{3.857128in}{1.907715in}}%
\pgfpathlineto{\pgfqpoint{3.870922in}{1.907441in}}%
\pgfpathlineto{\pgfqpoint{3.884716in}{1.909442in}}%
\pgfpathlineto{\pgfqpoint{3.898509in}{1.908790in}}%
\pgfpathlineto{\pgfqpoint{3.912303in}{1.915625in}}%
\pgfpathlineto{\pgfqpoint{3.926097in}{1.911749in}}%
\pgfpathlineto{\pgfqpoint{3.939891in}{1.915078in}}%
\pgfpathlineto{\pgfqpoint{3.953685in}{1.913287in}}%
\pgfpathlineto{\pgfqpoint{3.967478in}{1.916331in}}%
\pgfpathlineto{\pgfqpoint{3.981272in}{1.922124in}}%
\pgfpathlineto{\pgfqpoint{3.995066in}{1.922703in}}%
\pgfpathlineto{\pgfqpoint{4.008860in}{1.927737in}}%
\pgfpathlineto{\pgfqpoint{4.022654in}{1.924431in}}%
\pgfpathlineto{\pgfqpoint{4.036447in}{1.929086in}}%
\pgfpathlineto{\pgfqpoint{4.064035in}{1.941050in}}%
\pgfpathlineto{\pgfqpoint{4.105417in}{1.955868in}}%
\pgfpathlineto{\pgfqpoint{4.133004in}{1.962430in}}%
\pgfpathlineto{\pgfqpoint{4.160592in}{1.963399in}}%
\pgfpathlineto{\pgfqpoint{4.174386in}{1.964358in}}%
\pgfpathlineto{\pgfqpoint{4.188179in}{1.969297in}}%
\pgfpathlineto{\pgfqpoint{4.215767in}{1.973205in}}%
\pgfpathlineto{\pgfqpoint{4.215767in}{1.973205in}}%
\pgfusepath{stroke}%
\end{pgfscope}%
\begin{pgfscope}%
\pgfpathrectangle{\pgfqpoint{0.609375in}{0.376614in}}{\pgfqpoint{3.778125in}{2.274751in}} %
\pgfusepath{clip}%
\pgfsetbuttcap%
\pgfsetroundjoin%
\pgfsetlinewidth{1.003750pt}%
\definecolor{currentstroke}{rgb}{0.839216,0.152941,0.156863}%
\pgfsetstrokecolor{currentstroke}%
\pgfsetdash{{6.400000pt}{1.600000pt}{1.000000pt}{1.600000pt}}{0.000000pt}%
\pgfpathmoveto{\pgfqpoint{0.781108in}{0.485851in}}%
\pgfpathlineto{\pgfqpoint{0.794902in}{0.543015in}}%
\pgfpathlineto{\pgfqpoint{0.822489in}{0.621801in}}%
\pgfpathlineto{\pgfqpoint{0.850077in}{0.682579in}}%
\pgfpathlineto{\pgfqpoint{0.863871in}{0.710740in}}%
\pgfpathlineto{\pgfqpoint{0.919046in}{0.800163in}}%
\pgfpathlineto{\pgfqpoint{0.974221in}{0.875274in}}%
\pgfpathlineto{\pgfqpoint{1.015603in}{0.920423in}}%
\pgfpathlineto{\pgfqpoint{1.029397in}{0.930670in}}%
\pgfpathlineto{\pgfqpoint{1.084572in}{0.962087in}}%
\pgfpathlineto{\pgfqpoint{1.098366in}{0.972808in}}%
\pgfpathlineto{\pgfqpoint{1.112159in}{0.980306in}}%
\pgfpathlineto{\pgfqpoint{1.153541in}{1.015503in}}%
\pgfpathlineto{\pgfqpoint{1.167335in}{1.028973in}}%
\pgfpathlineto{\pgfqpoint{1.181128in}{1.039220in}}%
\pgfpathlineto{\pgfqpoint{1.194922in}{1.045961in}}%
\pgfpathlineto{\pgfqpoint{1.208716in}{1.059051in}}%
\pgfpathlineto{\pgfqpoint{1.222510in}{1.067213in}}%
\pgfpathlineto{\pgfqpoint{1.236304in}{1.072532in}}%
\pgfpathlineto{\pgfqpoint{1.263891in}{1.094448in}}%
\pgfpathlineto{\pgfqpoint{1.291479in}{1.109540in}}%
\pgfpathlineto{\pgfqpoint{1.305273in}{1.118650in}}%
\pgfpathlineto{\pgfqpoint{1.319067in}{1.129940in}}%
\pgfpathlineto{\pgfqpoint{1.332860in}{1.134311in}}%
\pgfpathlineto{\pgfqpoint{1.346654in}{1.140956in}}%
\pgfpathlineto{\pgfqpoint{1.374242in}{1.158513in}}%
\pgfpathlineto{\pgfqpoint{1.388036in}{1.165063in}}%
\pgfpathlineto{\pgfqpoint{1.401829in}{1.168866in}}%
\pgfpathlineto{\pgfqpoint{1.415623in}{1.177691in}}%
\pgfpathlineto{\pgfqpoint{1.429417in}{1.188507in}}%
\pgfpathlineto{\pgfqpoint{1.457005in}{1.205210in}}%
\pgfpathlineto{\pgfqpoint{1.470799in}{1.210339in}}%
\pgfpathlineto{\pgfqpoint{1.484592in}{1.222956in}}%
\pgfpathlineto{\pgfqpoint{1.512180in}{1.234541in}}%
\pgfpathlineto{\pgfqpoint{1.553561in}{1.263482in}}%
\pgfpathlineto{\pgfqpoint{1.594943in}{1.281334in}}%
\pgfpathlineto{\pgfqpoint{1.608737in}{1.291486in}}%
\pgfpathlineto{\pgfqpoint{1.622530in}{1.298321in}}%
\pgfpathlineto{\pgfqpoint{1.636324in}{1.301839in}}%
\pgfpathlineto{\pgfqpoint{1.663912in}{1.317215in}}%
\pgfpathlineto{\pgfqpoint{1.677706in}{1.318363in}}%
\pgfpathlineto{\pgfqpoint{1.691500in}{1.323682in}}%
\pgfpathlineto{\pgfqpoint{1.705293in}{1.325968in}}%
\pgfpathlineto{\pgfqpoint{1.719087in}{1.332518in}}%
\pgfpathlineto{\pgfqpoint{1.732881in}{1.334425in}}%
\pgfpathlineto{\pgfqpoint{1.760469in}{1.352929in}}%
\pgfpathlineto{\pgfqpoint{1.829438in}{1.387957in}}%
\pgfpathlineto{\pgfqpoint{1.843231in}{1.391475in}}%
\pgfpathlineto{\pgfqpoint{1.857025in}{1.397078in}}%
\pgfpathlineto{\pgfqpoint{1.870819in}{1.399458in}}%
\pgfpathlineto{\pgfqpoint{1.898407in}{1.409337in}}%
\pgfpathlineto{\pgfqpoint{1.912201in}{1.416931in}}%
\pgfpathlineto{\pgfqpoint{1.925994in}{1.419406in}}%
\pgfpathlineto{\pgfqpoint{1.967376in}{1.432992in}}%
\pgfpathlineto{\pgfqpoint{1.981170in}{1.438974in}}%
\pgfpathlineto{\pgfqpoint{2.022551in}{1.451518in}}%
\pgfpathlineto{\pgfqpoint{2.036345in}{1.459680in}}%
\pgfpathlineto{\pgfqpoint{2.050139in}{1.464145in}}%
\pgfpathlineto{\pgfqpoint{2.063932in}{1.463872in}}%
\pgfpathlineto{\pgfqpoint{2.077726in}{1.470233in}}%
\pgfpathlineto{\pgfqpoint{2.091520in}{1.473656in}}%
\pgfpathlineto{\pgfqpoint{2.160489in}{1.497879in}}%
\pgfpathlineto{\pgfqpoint{2.174283in}{1.508126in}}%
\pgfpathlineto{\pgfqpoint{2.188077in}{1.514108in}}%
\pgfpathlineto{\pgfqpoint{2.201871in}{1.518005in}}%
\pgfpathlineto{\pgfqpoint{2.215664in}{1.523798in}}%
\pgfpathlineto{\pgfqpoint{2.229458in}{1.531675in}}%
\pgfpathlineto{\pgfqpoint{2.243252in}{1.536899in}}%
\pgfpathlineto{\pgfqpoint{2.257046in}{1.539564in}}%
\pgfpathlineto{\pgfqpoint{2.270840in}{1.539101in}}%
\pgfpathlineto{\pgfqpoint{2.298427in}{1.551349in}}%
\pgfpathlineto{\pgfqpoint{2.339809in}{1.568253in}}%
\pgfpathlineto{\pgfqpoint{2.353602in}{1.570538in}}%
\pgfpathlineto{\pgfqpoint{2.408778in}{1.587358in}}%
\pgfpathlineto{\pgfqpoint{2.422572in}{1.593814in}}%
\pgfpathlineto{\pgfqpoint{2.450159in}{1.601134in}}%
\pgfpathlineto{\pgfqpoint{2.463953in}{1.599059in}}%
\pgfpathlineto{\pgfqpoint{2.477747in}{1.601819in}}%
\pgfpathlineto{\pgfqpoint{2.491541in}{1.607042in}}%
\pgfpathlineto{\pgfqpoint{2.505334in}{1.608001in}}%
\pgfpathlineto{\pgfqpoint{2.588097in}{1.632993in}}%
\pgfpathlineto{\pgfqpoint{2.601891in}{1.641061in}}%
\pgfpathlineto{\pgfqpoint{2.615685in}{1.642778in}}%
\pgfpathlineto{\pgfqpoint{2.629479in}{1.651319in}}%
\pgfpathlineto{\pgfqpoint{2.657066in}{1.657691in}}%
\pgfpathlineto{\pgfqpoint{2.684654in}{1.668328in}}%
\pgfpathlineto{\pgfqpoint{2.698448in}{1.676679in}}%
\pgfpathlineto{\pgfqpoint{2.712242in}{1.682282in}}%
\pgfpathlineto{\pgfqpoint{2.739829in}{1.687233in}}%
\pgfpathlineto{\pgfqpoint{2.753623in}{1.693973in}}%
\pgfpathlineto{\pgfqpoint{2.767417in}{1.702798in}}%
\pgfpathlineto{\pgfqpoint{2.781211in}{1.704894in}}%
\pgfpathlineto{\pgfqpoint{2.795004in}{1.705095in}}%
\pgfpathlineto{\pgfqpoint{2.808798in}{1.713067in}}%
\pgfpathlineto{\pgfqpoint{2.836386in}{1.723704in}}%
\pgfpathlineto{\pgfqpoint{2.850180in}{1.722957in}}%
\pgfpathlineto{\pgfqpoint{2.877767in}{1.727338in}}%
\pgfpathlineto{\pgfqpoint{2.905355in}{1.737028in}}%
\pgfpathlineto{\pgfqpoint{2.919149in}{1.746896in}}%
\pgfpathlineto{\pgfqpoint{2.946736in}{1.755258in}}%
\pgfpathlineto{\pgfqpoint{2.960530in}{1.766738in}}%
\pgfpathlineto{\pgfqpoint{3.001912in}{1.772836in}}%
\pgfpathlineto{\pgfqpoint{3.015705in}{1.780050in}}%
\pgfpathlineto{\pgfqpoint{3.029499in}{1.783473in}}%
\pgfpathlineto{\pgfqpoint{3.043293in}{1.783484in}}%
\pgfpathlineto{\pgfqpoint{3.057087in}{1.787855in}}%
\pgfpathlineto{\pgfqpoint{3.070881in}{1.786633in}}%
\pgfpathlineto{\pgfqpoint{3.084674in}{1.788824in}}%
\pgfpathlineto{\pgfqpoint{3.098468in}{1.793384in}}%
\pgfpathlineto{\pgfqpoint{3.112262in}{1.799840in}}%
\pgfpathlineto{\pgfqpoint{3.126056in}{1.801747in}}%
\pgfpathlineto{\pgfqpoint{3.153644in}{1.809351in}}%
\pgfpathlineto{\pgfqpoint{3.167437in}{1.810783in}}%
\pgfpathlineto{\pgfqpoint{3.181231in}{1.816197in}}%
\pgfpathlineto{\pgfqpoint{3.195025in}{1.816776in}}%
\pgfpathlineto{\pgfqpoint{3.208819in}{1.815744in}}%
\pgfpathlineto{\pgfqpoint{3.236406in}{1.822685in}}%
\pgfpathlineto{\pgfqpoint{3.250200in}{1.822032in}}%
\pgfpathlineto{\pgfqpoint{3.263994in}{1.825455in}}%
\pgfpathlineto{\pgfqpoint{3.291582in}{1.827941in}}%
\pgfpathlineto{\pgfqpoint{3.305375in}{1.827762in}}%
\pgfpathlineto{\pgfqpoint{3.319169in}{1.830142in}}%
\pgfpathlineto{\pgfqpoint{3.332963in}{1.830248in}}%
\pgfpathlineto{\pgfqpoint{3.374345in}{1.839285in}}%
\pgfpathlineto{\pgfqpoint{3.388138in}{1.836831in}}%
\pgfpathlineto{\pgfqpoint{3.401932in}{1.838169in}}%
\pgfpathlineto{\pgfqpoint{3.415726in}{1.841213in}}%
\pgfpathlineto{\pgfqpoint{3.429520in}{1.841602in}}%
\pgfpathlineto{\pgfqpoint{3.443314in}{1.849670in}}%
\pgfpathlineto{\pgfqpoint{3.457107in}{1.852050in}}%
\pgfpathlineto{\pgfqpoint{3.470901in}{1.860876in}}%
\pgfpathlineto{\pgfqpoint{3.484695in}{1.865152in}}%
\pgfpathlineto{\pgfqpoint{3.498489in}{1.865447in}}%
\pgfpathlineto{\pgfqpoint{3.512283in}{1.869343in}}%
\pgfpathlineto{\pgfqpoint{3.526076in}{1.871629in}}%
\pgfpathlineto{\pgfqpoint{3.553664in}{1.880655in}}%
\pgfpathlineto{\pgfqpoint{3.567458in}{1.877727in}}%
\pgfpathlineto{\pgfqpoint{3.608839in}{1.884679in}}%
\pgfpathlineto{\pgfqpoint{3.622633in}{1.891893in}}%
\pgfpathlineto{\pgfqpoint{3.636427in}{1.895221in}}%
\pgfpathlineto{\pgfqpoint{3.650221in}{1.896938in}}%
\pgfpathlineto{\pgfqpoint{3.664015in}{1.894579in}}%
\pgfpathlineto{\pgfqpoint{3.677808in}{1.898097in}}%
\pgfpathlineto{\pgfqpoint{3.691602in}{1.896781in}}%
\pgfpathlineto{\pgfqpoint{3.705396in}{1.897834in}}%
\pgfpathlineto{\pgfqpoint{3.719190in}{1.904100in}}%
\pgfpathlineto{\pgfqpoint{3.732984in}{1.906291in}}%
\pgfpathlineto{\pgfqpoint{3.760571in}{1.906502in}}%
\pgfpathlineto{\pgfqpoint{3.774365in}{1.908598in}}%
\pgfpathlineto{\pgfqpoint{3.788159in}{1.916665in}}%
\pgfpathlineto{\pgfqpoint{3.829540in}{1.921626in}}%
\pgfpathlineto{\pgfqpoint{3.843334in}{1.920784in}}%
\pgfpathlineto{\pgfqpoint{3.857128in}{1.928093in}}%
\pgfpathlineto{\pgfqpoint{3.884716in}{1.928968in}}%
\pgfpathlineto{\pgfqpoint{3.898509in}{1.926988in}}%
\pgfpathlineto{\pgfqpoint{3.912303in}{1.935150in}}%
\pgfpathlineto{\pgfqpoint{3.926097in}{1.932128in}}%
\pgfpathlineto{\pgfqpoint{3.939891in}{1.936214in}}%
\pgfpathlineto{\pgfqpoint{3.953685in}{1.934140in}}%
\pgfpathlineto{\pgfqpoint{3.981272in}{1.943071in}}%
\pgfpathlineto{\pgfqpoint{3.995066in}{1.943745in}}%
\pgfpathlineto{\pgfqpoint{4.008860in}{1.949253in}}%
\pgfpathlineto{\pgfqpoint{4.022654in}{1.948695in}}%
\pgfpathlineto{\pgfqpoint{4.036447in}{1.952023in}}%
\pgfpathlineto{\pgfqpoint{4.050241in}{1.958479in}}%
\pgfpathlineto{\pgfqpoint{4.133004in}{1.981766in}}%
\pgfpathlineto{\pgfqpoint{4.146798in}{1.981681in}}%
\pgfpathlineto{\pgfqpoint{4.174386in}{1.987580in}}%
\pgfpathlineto{\pgfqpoint{4.188179in}{1.992235in}}%
\pgfpathlineto{\pgfqpoint{4.201973in}{1.994804in}}%
\pgfpathlineto{\pgfqpoint{4.215767in}{1.995289in}}%
\pgfpathlineto{\pgfqpoint{4.215767in}{1.995289in}}%
\pgfusepath{stroke}%
\end{pgfscope}%
\begin{pgfscope}%
\pgfpathrectangle{\pgfqpoint{0.609375in}{0.376614in}}{\pgfqpoint{3.778125in}{2.274751in}} %
\pgfusepath{clip}%
\pgfsetrectcap%
\pgfsetroundjoin%
\pgfsetlinewidth{1.003750pt}%
\definecolor{currentstroke}{rgb}{0.580392,0.403922,0.741176}%
\pgfsetstrokecolor{currentstroke}%
\pgfsetdash{}{0pt}%
\pgfpathmoveto{\pgfqpoint{0.781108in}{0.485851in}}%
\pgfpathlineto{\pgfqpoint{0.794902in}{0.543015in}}%
\pgfpathlineto{\pgfqpoint{0.822489in}{0.622181in}}%
\pgfpathlineto{\pgfqpoint{0.850077in}{0.683906in}}%
\pgfpathlineto{\pgfqpoint{0.863871in}{0.711783in}}%
\pgfpathlineto{\pgfqpoint{0.919046in}{0.802628in}}%
\pgfpathlineto{\pgfqpoint{0.932840in}{0.820931in}}%
\pgfpathlineto{\pgfqpoint{0.960428in}{0.861330in}}%
\pgfpathlineto{\pgfqpoint{0.988015in}{0.892915in}}%
\pgfpathlineto{\pgfqpoint{1.001809in}{0.908944in}}%
\pgfpathlineto{\pgfqpoint{1.015603in}{0.922603in}}%
\pgfpathlineto{\pgfqpoint{1.029397in}{0.932282in}}%
\pgfpathlineto{\pgfqpoint{1.056984in}{0.946995in}}%
\pgfpathlineto{\pgfqpoint{1.084572in}{0.963793in}}%
\pgfpathlineto{\pgfqpoint{1.139747in}{1.003360in}}%
\pgfpathlineto{\pgfqpoint{1.167335in}{1.029257in}}%
\pgfpathlineto{\pgfqpoint{1.181128in}{1.039505in}}%
\pgfpathlineto{\pgfqpoint{1.194922in}{1.046814in}}%
\pgfpathlineto{\pgfqpoint{1.208716in}{1.059430in}}%
\pgfpathlineto{\pgfqpoint{1.236304in}{1.073290in}}%
\pgfpathlineto{\pgfqpoint{1.250098in}{1.086002in}}%
\pgfpathlineto{\pgfqpoint{1.263891in}{1.096249in}}%
\pgfpathlineto{\pgfqpoint{1.277685in}{1.104411in}}%
\pgfpathlineto{\pgfqpoint{1.305273in}{1.117892in}}%
\pgfpathlineto{\pgfqpoint{1.319067in}{1.128897in}}%
\pgfpathlineto{\pgfqpoint{1.332860in}{1.133268in}}%
\pgfpathlineto{\pgfqpoint{1.374242in}{1.158513in}}%
\pgfpathlineto{\pgfqpoint{1.388036in}{1.163547in}}%
\pgfpathlineto{\pgfqpoint{1.401829in}{1.166401in}}%
\pgfpathlineto{\pgfqpoint{1.415623in}{1.175416in}}%
\pgfpathlineto{\pgfqpoint{1.429417in}{1.187180in}}%
\pgfpathlineto{\pgfqpoint{1.443211in}{1.193352in}}%
\pgfpathlineto{\pgfqpoint{1.457005in}{1.201893in}}%
\pgfpathlineto{\pgfqpoint{1.470799in}{1.208538in}}%
\pgfpathlineto{\pgfqpoint{1.484592in}{1.218217in}}%
\pgfpathlineto{\pgfqpoint{1.512180in}{1.228854in}}%
\pgfpathlineto{\pgfqpoint{1.539768in}{1.248780in}}%
\pgfpathlineto{\pgfqpoint{1.567355in}{1.262261in}}%
\pgfpathlineto{\pgfqpoint{1.581149in}{1.267011in}}%
\pgfpathlineto{\pgfqpoint{1.594943in}{1.269201in}}%
\pgfpathlineto{\pgfqpoint{1.608737in}{1.277743in}}%
\pgfpathlineto{\pgfqpoint{1.636324in}{1.286768in}}%
\pgfpathlineto{\pgfqpoint{1.663912in}{1.299680in}}%
\pgfpathlineto{\pgfqpoint{1.677706in}{1.299596in}}%
\pgfpathlineto{\pgfqpoint{1.691500in}{1.304157in}}%
\pgfpathlineto{\pgfqpoint{1.705293in}{1.306442in}}%
\pgfpathlineto{\pgfqpoint{1.719087in}{1.312330in}}%
\pgfpathlineto{\pgfqpoint{1.732881in}{1.314331in}}%
\pgfpathlineto{\pgfqpoint{1.746675in}{1.325431in}}%
\pgfpathlineto{\pgfqpoint{1.774262in}{1.341376in}}%
\pgfpathlineto{\pgfqpoint{1.788056in}{1.346031in}}%
\pgfpathlineto{\pgfqpoint{1.815644in}{1.358848in}}%
\pgfpathlineto{\pgfqpoint{1.829438in}{1.364262in}}%
\pgfpathlineto{\pgfqpoint{1.843231in}{1.367685in}}%
\pgfpathlineto{\pgfqpoint{1.857025in}{1.374520in}}%
\pgfpathlineto{\pgfqpoint{1.884613in}{1.382598in}}%
\pgfpathlineto{\pgfqpoint{1.898407in}{1.388106in}}%
\pgfpathlineto{\pgfqpoint{1.912201in}{1.397595in}}%
\pgfpathlineto{\pgfqpoint{1.925994in}{1.400070in}}%
\pgfpathlineto{\pgfqpoint{2.022551in}{1.433414in}}%
\pgfpathlineto{\pgfqpoint{2.036345in}{1.440154in}}%
\pgfpathlineto{\pgfqpoint{2.050139in}{1.443577in}}%
\pgfpathlineto{\pgfqpoint{2.063932in}{1.443683in}}%
\pgfpathlineto{\pgfqpoint{2.077726in}{1.450897in}}%
\pgfpathlineto{\pgfqpoint{2.119108in}{1.457754in}}%
\pgfpathlineto{\pgfqpoint{2.132901in}{1.464399in}}%
\pgfpathlineto{\pgfqpoint{2.160489in}{1.470771in}}%
\pgfpathlineto{\pgfqpoint{2.174283in}{1.482725in}}%
\pgfpathlineto{\pgfqpoint{2.188077in}{1.490602in}}%
\pgfpathlineto{\pgfqpoint{2.201871in}{1.494215in}}%
\pgfpathlineto{\pgfqpoint{2.243252in}{1.513962in}}%
\pgfpathlineto{\pgfqpoint{2.270840in}{1.519291in}}%
\pgfpathlineto{\pgfqpoint{2.284633in}{1.525652in}}%
\pgfpathlineto{\pgfqpoint{2.298427in}{1.528317in}}%
\pgfpathlineto{\pgfqpoint{2.339809in}{1.543420in}}%
\pgfpathlineto{\pgfqpoint{2.353602in}{1.544284in}}%
\pgfpathlineto{\pgfqpoint{2.408778in}{1.557596in}}%
\pgfpathlineto{\pgfqpoint{2.450159in}{1.572035in}}%
\pgfpathlineto{\pgfqpoint{2.463953in}{1.570814in}}%
\pgfpathlineto{\pgfqpoint{2.477747in}{1.573952in}}%
\pgfpathlineto{\pgfqpoint{2.491541in}{1.579271in}}%
\pgfpathlineto{\pgfqpoint{2.519128in}{1.583842in}}%
\pgfpathlineto{\pgfqpoint{2.546716in}{1.595427in}}%
\pgfpathlineto{\pgfqpoint{2.560510in}{1.596196in}}%
\pgfpathlineto{\pgfqpoint{2.574303in}{1.603316in}}%
\pgfpathlineto{\pgfqpoint{2.588097in}{1.605601in}}%
\pgfpathlineto{\pgfqpoint{2.601891in}{1.614048in}}%
\pgfpathlineto{\pgfqpoint{2.615685in}{1.617186in}}%
\pgfpathlineto{\pgfqpoint{2.629479in}{1.626296in}}%
\pgfpathlineto{\pgfqpoint{2.657066in}{1.632289in}}%
\pgfpathlineto{\pgfqpoint{2.684654in}{1.642357in}}%
\pgfpathlineto{\pgfqpoint{2.698448in}{1.649761in}}%
\pgfpathlineto{\pgfqpoint{2.712242in}{1.654321in}}%
\pgfpathlineto{\pgfqpoint{2.739829in}{1.657850in}}%
\pgfpathlineto{\pgfqpoint{2.753623in}{1.662600in}}%
\pgfpathlineto{\pgfqpoint{2.767417in}{1.671615in}}%
\pgfpathlineto{\pgfqpoint{2.781211in}{1.674848in}}%
\pgfpathlineto{\pgfqpoint{2.795004in}{1.675428in}}%
\pgfpathlineto{\pgfqpoint{2.808798in}{1.682926in}}%
\pgfpathlineto{\pgfqpoint{2.836386in}{1.693563in}}%
\pgfpathlineto{\pgfqpoint{2.850180in}{1.694143in}}%
\pgfpathlineto{\pgfqpoint{2.905355in}{1.704802in}}%
\pgfpathlineto{\pgfqpoint{2.919149in}{1.713627in}}%
\pgfpathlineto{\pgfqpoint{2.946736in}{1.721326in}}%
\pgfpathlineto{\pgfqpoint{2.960530in}{1.731005in}}%
\pgfpathlineto{\pgfqpoint{2.988118in}{1.738230in}}%
\pgfpathlineto{\pgfqpoint{3.001912in}{1.738430in}}%
\pgfpathlineto{\pgfqpoint{3.015705in}{1.745265in}}%
\pgfpathlineto{\pgfqpoint{3.029499in}{1.748024in}}%
\pgfpathlineto{\pgfqpoint{3.043293in}{1.746898in}}%
\pgfpathlineto{\pgfqpoint{3.057087in}{1.754586in}}%
\pgfpathlineto{\pgfqpoint{3.070881in}{1.754312in}}%
\pgfpathlineto{\pgfqpoint{3.084674in}{1.755650in}}%
\pgfpathlineto{\pgfqpoint{3.098468in}{1.759642in}}%
\pgfpathlineto{\pgfqpoint{3.112262in}{1.766572in}}%
\pgfpathlineto{\pgfqpoint{3.153644in}{1.774566in}}%
\pgfpathlineto{\pgfqpoint{3.167437in}{1.775714in}}%
\pgfpathlineto{\pgfqpoint{3.181231in}{1.780748in}}%
\pgfpathlineto{\pgfqpoint{3.222613in}{1.785898in}}%
\pgfpathlineto{\pgfqpoint{3.236406in}{1.789227in}}%
\pgfpathlineto{\pgfqpoint{3.250200in}{1.788479in}}%
\pgfpathlineto{\pgfqpoint{3.263994in}{1.791333in}}%
\pgfpathlineto{\pgfqpoint{3.291582in}{1.792871in}}%
\pgfpathlineto{\pgfqpoint{3.332963in}{1.793662in}}%
\pgfpathlineto{\pgfqpoint{3.360551in}{1.797380in}}%
\pgfpathlineto{\pgfqpoint{3.374345in}{1.800898in}}%
\pgfpathlineto{\pgfqpoint{3.388138in}{1.797117in}}%
\pgfpathlineto{\pgfqpoint{3.415726in}{1.799887in}}%
\pgfpathlineto{\pgfqpoint{3.429520in}{1.797813in}}%
\pgfpathlineto{\pgfqpoint{3.443314in}{1.805690in}}%
\pgfpathlineto{\pgfqpoint{3.457107in}{1.807976in}}%
\pgfpathlineto{\pgfqpoint{3.470901in}{1.815759in}}%
\pgfpathlineto{\pgfqpoint{3.484695in}{1.820983in}}%
\pgfpathlineto{\pgfqpoint{3.498489in}{1.821278in}}%
\pgfpathlineto{\pgfqpoint{3.526076in}{1.827555in}}%
\pgfpathlineto{\pgfqpoint{3.539870in}{1.831073in}}%
\pgfpathlineto{\pgfqpoint{3.553664in}{1.832979in}}%
\pgfpathlineto{\pgfqpoint{3.567458in}{1.830431in}}%
\pgfpathlineto{\pgfqpoint{3.608839in}{1.834349in}}%
\pgfpathlineto{\pgfqpoint{3.622633in}{1.841468in}}%
\pgfpathlineto{\pgfqpoint{3.650221in}{1.844428in}}%
\pgfpathlineto{\pgfqpoint{3.664015in}{1.842354in}}%
\pgfpathlineto{\pgfqpoint{3.677808in}{1.846440in}}%
\pgfpathlineto{\pgfqpoint{3.691602in}{1.844745in}}%
\pgfpathlineto{\pgfqpoint{3.705396in}{1.846272in}}%
\pgfpathlineto{\pgfqpoint{3.732984in}{1.854350in}}%
\pgfpathlineto{\pgfqpoint{3.760571in}{1.856836in}}%
\pgfpathlineto{\pgfqpoint{3.774365in}{1.860164in}}%
\pgfpathlineto{\pgfqpoint{3.788159in}{1.865957in}}%
\pgfpathlineto{\pgfqpoint{3.829540in}{1.871771in}}%
\pgfpathlineto{\pgfqpoint{3.843334in}{1.871781in}}%
\pgfpathlineto{\pgfqpoint{3.857128in}{1.879659in}}%
\pgfpathlineto{\pgfqpoint{3.870922in}{1.879006in}}%
\pgfpathlineto{\pgfqpoint{3.884716in}{1.881102in}}%
\pgfpathlineto{\pgfqpoint{3.898509in}{1.880260in}}%
\pgfpathlineto{\pgfqpoint{3.912303in}{1.886716in}}%
\pgfpathlineto{\pgfqpoint{3.926097in}{1.883409in}}%
\pgfpathlineto{\pgfqpoint{3.939891in}{1.884937in}}%
\pgfpathlineto{\pgfqpoint{3.953685in}{1.882673in}}%
\pgfpathlineto{\pgfqpoint{3.967478in}{1.885242in}}%
\pgfpathlineto{\pgfqpoint{3.981272in}{1.890087in}}%
\pgfpathlineto{\pgfqpoint{3.995066in}{1.891804in}}%
\pgfpathlineto{\pgfqpoint{4.008860in}{1.898070in}}%
\pgfpathlineto{\pgfqpoint{4.022654in}{1.896944in}}%
\pgfpathlineto{\pgfqpoint{4.077829in}{1.916701in}}%
\pgfpathlineto{\pgfqpoint{4.091623in}{1.919651in}}%
\pgfpathlineto{\pgfqpoint{4.105417in}{1.925633in}}%
\pgfpathlineto{\pgfqpoint{4.133004in}{1.931341in}}%
\pgfpathlineto{\pgfqpoint{4.146798in}{1.931257in}}%
\pgfpathlineto{\pgfqpoint{4.174386in}{1.936776in}}%
\pgfpathlineto{\pgfqpoint{4.188179in}{1.942000in}}%
\pgfpathlineto{\pgfqpoint{4.201973in}{1.944190in}}%
\pgfpathlineto{\pgfqpoint{4.215767in}{1.944580in}}%
\pgfpathlineto{\pgfqpoint{4.215767in}{1.944580in}}%
\pgfusepath{stroke}%
\end{pgfscope}%
\begin{pgfscope}%
\pgfpathrectangle{\pgfqpoint{0.609375in}{0.376614in}}{\pgfqpoint{3.778125in}{2.274751in}} %
\pgfusepath{clip}%
\pgfsetbuttcap%
\pgfsetroundjoin%
\pgfsetlinewidth{1.003750pt}%
\definecolor{currentstroke}{rgb}{0.549020,0.337255,0.294118}%
\pgfsetstrokecolor{currentstroke}%
\pgfsetdash{{1.000000pt}{1.650000pt}}{0.000000pt}%
\pgfpathmoveto{\pgfqpoint{0.781108in}{0.485851in}}%
\pgfpathlineto{\pgfqpoint{0.794902in}{0.543015in}}%
\pgfpathlineto{\pgfqpoint{0.822489in}{0.621043in}}%
\pgfpathlineto{\pgfqpoint{0.850077in}{0.680304in}}%
\pgfpathlineto{\pgfqpoint{0.863871in}{0.707802in}}%
\pgfpathlineto{\pgfqpoint{0.919046in}{0.797320in}}%
\pgfpathlineto{\pgfqpoint{0.974221in}{0.872526in}}%
\pgfpathlineto{\pgfqpoint{1.015603in}{0.917390in}}%
\pgfpathlineto{\pgfqpoint{1.043190in}{0.935420in}}%
\pgfpathlineto{\pgfqpoint{1.112159in}{0.979264in}}%
\pgfpathlineto{\pgfqpoint{1.181128in}{1.037040in}}%
\pgfpathlineto{\pgfqpoint{1.194922in}{1.045202in}}%
\pgfpathlineto{\pgfqpoint{1.208716in}{1.058767in}}%
\pgfpathlineto{\pgfqpoint{1.222510in}{1.068066in}}%
\pgfpathlineto{\pgfqpoint{1.236304in}{1.073954in}}%
\pgfpathlineto{\pgfqpoint{1.263891in}{1.095775in}}%
\pgfpathlineto{\pgfqpoint{1.305273in}{1.118935in}}%
\pgfpathlineto{\pgfqpoint{1.319067in}{1.129561in}}%
\pgfpathlineto{\pgfqpoint{1.332860in}{1.134216in}}%
\pgfpathlineto{\pgfqpoint{1.346654in}{1.141430in}}%
\pgfpathlineto{\pgfqpoint{1.374242in}{1.160977in}}%
\pgfpathlineto{\pgfqpoint{1.388036in}{1.167338in}}%
\pgfpathlineto{\pgfqpoint{1.401829in}{1.170098in}}%
\pgfpathlineto{\pgfqpoint{1.415623in}{1.177975in}}%
\pgfpathlineto{\pgfqpoint{1.429417in}{1.188507in}}%
\pgfpathlineto{\pgfqpoint{1.470799in}{1.210908in}}%
\pgfpathlineto{\pgfqpoint{1.484592in}{1.221819in}}%
\pgfpathlineto{\pgfqpoint{1.498386in}{1.229317in}}%
\pgfpathlineto{\pgfqpoint{1.512180in}{1.234636in}}%
\pgfpathlineto{\pgfqpoint{1.553561in}{1.259880in}}%
\pgfpathlineto{\pgfqpoint{1.594943in}{1.275647in}}%
\pgfpathlineto{\pgfqpoint{1.608737in}{1.284757in}}%
\pgfpathlineto{\pgfqpoint{1.622530in}{1.290739in}}%
\pgfpathlineto{\pgfqpoint{1.636324in}{1.294730in}}%
\pgfpathlineto{\pgfqpoint{1.663912in}{1.307547in}}%
\pgfpathlineto{\pgfqpoint{1.677706in}{1.307558in}}%
\pgfpathlineto{\pgfqpoint{1.705293in}{1.314120in}}%
\pgfpathlineto{\pgfqpoint{1.719087in}{1.321903in}}%
\pgfpathlineto{\pgfqpoint{1.732881in}{1.323146in}}%
\pgfpathlineto{\pgfqpoint{1.760469in}{1.340607in}}%
\pgfpathlineto{\pgfqpoint{1.774262in}{1.347727in}}%
\pgfpathlineto{\pgfqpoint{1.788056in}{1.352856in}}%
\pgfpathlineto{\pgfqpoint{1.815644in}{1.367284in}}%
\pgfpathlineto{\pgfqpoint{1.829438in}{1.373361in}}%
\pgfpathlineto{\pgfqpoint{1.843231in}{1.377068in}}%
\pgfpathlineto{\pgfqpoint{1.857025in}{1.383714in}}%
\pgfpathlineto{\pgfqpoint{1.884613in}{1.393119in}}%
\pgfpathlineto{\pgfqpoint{1.925994in}{1.411918in}}%
\pgfpathlineto{\pgfqpoint{1.953582in}{1.421039in}}%
\pgfpathlineto{\pgfqpoint{1.967376in}{1.427021in}}%
\pgfpathlineto{\pgfqpoint{1.994963in}{1.441828in}}%
\pgfpathlineto{\pgfqpoint{2.050139in}{1.460354in}}%
\pgfpathlineto{\pgfqpoint{2.063932in}{1.460270in}}%
\pgfpathlineto{\pgfqpoint{2.077726in}{1.466631in}}%
\pgfpathlineto{\pgfqpoint{2.119108in}{1.477658in}}%
\pgfpathlineto{\pgfqpoint{2.132901in}{1.484778in}}%
\pgfpathlineto{\pgfqpoint{2.160489in}{1.490865in}}%
\pgfpathlineto{\pgfqpoint{2.174283in}{1.501871in}}%
\pgfpathlineto{\pgfqpoint{2.188077in}{1.508801in}}%
\pgfpathlineto{\pgfqpoint{2.201871in}{1.511276in}}%
\pgfpathlineto{\pgfqpoint{2.215664in}{1.518111in}}%
\pgfpathlineto{\pgfqpoint{2.229458in}{1.522861in}}%
\pgfpathlineto{\pgfqpoint{2.243252in}{1.529317in}}%
\pgfpathlineto{\pgfqpoint{2.270840in}{1.533603in}}%
\pgfpathlineto{\pgfqpoint{2.284633in}{1.540059in}}%
\pgfpathlineto{\pgfqpoint{2.298427in}{1.543861in}}%
\pgfpathlineto{\pgfqpoint{2.326015in}{1.554783in}}%
\pgfpathlineto{\pgfqpoint{2.339809in}{1.560196in}}%
\pgfpathlineto{\pgfqpoint{2.367396in}{1.564767in}}%
\pgfpathlineto{\pgfqpoint{2.422572in}{1.581492in}}%
\pgfpathlineto{\pgfqpoint{2.436365in}{1.586811in}}%
\pgfpathlineto{\pgfqpoint{2.450159in}{1.590423in}}%
\pgfpathlineto{\pgfqpoint{2.463953in}{1.587875in}}%
\pgfpathlineto{\pgfqpoint{2.477747in}{1.591013in}}%
\pgfpathlineto{\pgfqpoint{2.491541in}{1.595763in}}%
\pgfpathlineto{\pgfqpoint{2.505334in}{1.596532in}}%
\pgfpathlineto{\pgfqpoint{2.519128in}{1.599671in}}%
\pgfpathlineto{\pgfqpoint{2.546716in}{1.609550in}}%
\pgfpathlineto{\pgfqpoint{2.560510in}{1.610508in}}%
\pgfpathlineto{\pgfqpoint{2.574303in}{1.616680in}}%
\pgfpathlineto{\pgfqpoint{2.588097in}{1.618207in}}%
\pgfpathlineto{\pgfqpoint{2.601891in}{1.626938in}}%
\pgfpathlineto{\pgfqpoint{2.615685in}{1.627991in}}%
\pgfpathlineto{\pgfqpoint{2.629479in}{1.636817in}}%
\pgfpathlineto{\pgfqpoint{2.643273in}{1.637776in}}%
\pgfpathlineto{\pgfqpoint{2.670860in}{1.644906in}}%
\pgfpathlineto{\pgfqpoint{2.684654in}{1.650509in}}%
\pgfpathlineto{\pgfqpoint{2.712242in}{1.664558in}}%
\pgfpathlineto{\pgfqpoint{2.726035in}{1.667317in}}%
\pgfpathlineto{\pgfqpoint{2.739829in}{1.668181in}}%
\pgfpathlineto{\pgfqpoint{2.753623in}{1.672457in}}%
\pgfpathlineto{\pgfqpoint{2.767417in}{1.682515in}}%
\pgfpathlineto{\pgfqpoint{2.781211in}{1.685654in}}%
\pgfpathlineto{\pgfqpoint{2.795004in}{1.685096in}}%
\pgfpathlineto{\pgfqpoint{2.808798in}{1.693068in}}%
\pgfpathlineto{\pgfqpoint{2.836386in}{1.703800in}}%
\pgfpathlineto{\pgfqpoint{2.850180in}{1.702958in}}%
\pgfpathlineto{\pgfqpoint{2.863974in}{1.706002in}}%
\pgfpathlineto{\pgfqpoint{2.877767in}{1.706297in}}%
\pgfpathlineto{\pgfqpoint{2.891561in}{1.711900in}}%
\pgfpathlineto{\pgfqpoint{2.905355in}{1.715512in}}%
\pgfpathlineto{\pgfqpoint{2.919149in}{1.726423in}}%
\pgfpathlineto{\pgfqpoint{2.946736in}{1.732795in}}%
\pgfpathlineto{\pgfqpoint{2.960530in}{1.742473in}}%
\pgfpathlineto{\pgfqpoint{2.988118in}{1.747139in}}%
\pgfpathlineto{\pgfqpoint{3.001912in}{1.746960in}}%
\pgfpathlineto{\pgfqpoint{3.015705in}{1.752089in}}%
\pgfpathlineto{\pgfqpoint{3.029499in}{1.755038in}}%
\pgfpathlineto{\pgfqpoint{3.043293in}{1.755144in}}%
\pgfpathlineto{\pgfqpoint{3.057087in}{1.762358in}}%
\pgfpathlineto{\pgfqpoint{3.070881in}{1.761137in}}%
\pgfpathlineto{\pgfqpoint{3.084674in}{1.763422in}}%
\pgfpathlineto{\pgfqpoint{3.112262in}{1.772732in}}%
\pgfpathlineto{\pgfqpoint{3.126056in}{1.773407in}}%
\pgfpathlineto{\pgfqpoint{3.167437in}{1.782538in}}%
\pgfpathlineto{\pgfqpoint{3.181231in}{1.788994in}}%
\pgfpathlineto{\pgfqpoint{3.222613in}{1.794429in}}%
\pgfpathlineto{\pgfqpoint{3.236406in}{1.799274in}}%
\pgfpathlineto{\pgfqpoint{3.250200in}{1.797673in}}%
\pgfpathlineto{\pgfqpoint{3.277788in}{1.801012in}}%
\pgfpathlineto{\pgfqpoint{3.360551in}{1.808185in}}%
\pgfpathlineto{\pgfqpoint{3.374345in}{1.812366in}}%
\pgfpathlineto{\pgfqpoint{3.388138in}{1.810671in}}%
\pgfpathlineto{\pgfqpoint{3.415726in}{1.815811in}}%
\pgfpathlineto{\pgfqpoint{3.429520in}{1.815063in}}%
\pgfpathlineto{\pgfqpoint{3.443314in}{1.823889in}}%
\pgfpathlineto{\pgfqpoint{3.457107in}{1.826079in}}%
\pgfpathlineto{\pgfqpoint{3.470901in}{1.834147in}}%
\pgfpathlineto{\pgfqpoint{3.484695in}{1.839655in}}%
\pgfpathlineto{\pgfqpoint{3.498489in}{1.839571in}}%
\pgfpathlineto{\pgfqpoint{3.512283in}{1.843089in}}%
\pgfpathlineto{\pgfqpoint{3.526076in}{1.844616in}}%
\pgfpathlineto{\pgfqpoint{3.539870in}{1.850408in}}%
\pgfpathlineto{\pgfqpoint{3.553664in}{1.853357in}}%
\pgfpathlineto{\pgfqpoint{3.567458in}{1.851567in}}%
\pgfpathlineto{\pgfqpoint{3.608839in}{1.858613in}}%
\pgfpathlineto{\pgfqpoint{3.622633in}{1.866302in}}%
\pgfpathlineto{\pgfqpoint{3.636427in}{1.869251in}}%
\pgfpathlineto{\pgfqpoint{3.650221in}{1.869830in}}%
\pgfpathlineto{\pgfqpoint{3.664015in}{1.867850in}}%
\pgfpathlineto{\pgfqpoint{3.677808in}{1.870705in}}%
\pgfpathlineto{\pgfqpoint{3.691602in}{1.869767in}}%
\pgfpathlineto{\pgfqpoint{3.705396in}{1.871295in}}%
\pgfpathlineto{\pgfqpoint{3.719190in}{1.878130in}}%
\pgfpathlineto{\pgfqpoint{3.732984in}{1.880889in}}%
\pgfpathlineto{\pgfqpoint{3.760571in}{1.881290in}}%
\pgfpathlineto{\pgfqpoint{3.774365in}{1.884523in}}%
\pgfpathlineto{\pgfqpoint{3.788159in}{1.890790in}}%
\pgfpathlineto{\pgfqpoint{3.801953in}{1.892696in}}%
\pgfpathlineto{\pgfqpoint{3.815747in}{1.897256in}}%
\pgfpathlineto{\pgfqpoint{3.843334in}{1.896709in}}%
\pgfpathlineto{\pgfqpoint{3.857128in}{1.903734in}}%
\pgfpathlineto{\pgfqpoint{3.870922in}{1.903650in}}%
\pgfpathlineto{\pgfqpoint{3.884716in}{1.905746in}}%
\pgfpathlineto{\pgfqpoint{3.898509in}{1.904050in}}%
\pgfpathlineto{\pgfqpoint{3.912303in}{1.909938in}}%
\pgfpathlineto{\pgfqpoint{3.926097in}{1.907484in}}%
\pgfpathlineto{\pgfqpoint{3.939891in}{1.910718in}}%
\pgfpathlineto{\pgfqpoint{3.953685in}{1.908264in}}%
\pgfpathlineto{\pgfqpoint{3.967478in}{1.911023in}}%
\pgfpathlineto{\pgfqpoint{3.981272in}{1.916626in}}%
\pgfpathlineto{\pgfqpoint{3.995066in}{1.918059in}}%
\pgfpathlineto{\pgfqpoint{4.008860in}{1.922998in}}%
\pgfpathlineto{\pgfqpoint{4.022654in}{1.922156in}}%
\pgfpathlineto{\pgfqpoint{4.036447in}{1.924726in}}%
\pgfpathlineto{\pgfqpoint{4.050241in}{1.931371in}}%
\pgfpathlineto{\pgfqpoint{4.077829in}{1.939829in}}%
\pgfpathlineto{\pgfqpoint{4.119210in}{1.951045in}}%
\pgfpathlineto{\pgfqpoint{4.133004in}{1.953710in}}%
\pgfpathlineto{\pgfqpoint{4.174386in}{1.957249in}}%
\pgfpathlineto{\pgfqpoint{4.201973in}{1.965043in}}%
\pgfpathlineto{\pgfqpoint{4.215767in}{1.964674in}}%
\pgfpathlineto{\pgfqpoint{4.215767in}{1.964674in}}%
\pgfusepath{stroke}%
\end{pgfscope}%
\begin{pgfscope}%
\pgfsetrectcap%
\pgfsetmiterjoin%
\pgfsetlinewidth{0.803000pt}%
\definecolor{currentstroke}{rgb}{0.000000,0.000000,0.000000}%
\pgfsetstrokecolor{currentstroke}%
\pgfsetdash{}{0pt}%
\pgfpathmoveto{\pgfqpoint{0.609375in}{0.376614in}}%
\pgfpathlineto{\pgfqpoint{0.609375in}{2.651366in}}%
\pgfusepath{stroke}%
\end{pgfscope}%
\begin{pgfscope}%
\pgfsetrectcap%
\pgfsetmiterjoin%
\pgfsetlinewidth{0.803000pt}%
\definecolor{currentstroke}{rgb}{0.000000,0.000000,0.000000}%
\pgfsetstrokecolor{currentstroke}%
\pgfsetdash{}{0pt}%
\pgfpathmoveto{\pgfqpoint{4.387500in}{0.376614in}}%
\pgfpathlineto{\pgfqpoint{4.387500in}{2.651366in}}%
\pgfusepath{stroke}%
\end{pgfscope}%
\begin{pgfscope}%
\pgfsetrectcap%
\pgfsetmiterjoin%
\pgfsetlinewidth{0.803000pt}%
\definecolor{currentstroke}{rgb}{0.000000,0.000000,0.000000}%
\pgfsetstrokecolor{currentstroke}%
\pgfsetdash{}{0pt}%
\pgfpathmoveto{\pgfqpoint{0.609375in}{0.376614in}}%
\pgfpathlineto{\pgfqpoint{4.387500in}{0.376614in}}%
\pgfusepath{stroke}%
\end{pgfscope}%
\begin{pgfscope}%
\pgfsetrectcap%
\pgfsetmiterjoin%
\pgfsetlinewidth{0.803000pt}%
\definecolor{currentstroke}{rgb}{0.000000,0.000000,0.000000}%
\pgfsetstrokecolor{currentstroke}%
\pgfsetdash{}{0pt}%
\pgfpathmoveto{\pgfqpoint{0.609375in}{2.651366in}}%
\pgfpathlineto{\pgfqpoint{4.387500in}{2.651366in}}%
\pgfusepath{stroke}%
\end{pgfscope}%
\begin{pgfscope}%
\pgfsetbuttcap%
\pgfsetmiterjoin%
\definecolor{currentfill}{rgb}{1.000000,1.000000,1.000000}%
\pgfsetfillcolor{currentfill}%
\pgfsetfillopacity{0.800000}%
\pgfsetlinewidth{1.003750pt}%
\definecolor{currentstroke}{rgb}{0.800000,0.800000,0.800000}%
\pgfsetstrokecolor{currentstroke}%
\pgfsetstrokeopacity{0.800000}%
\pgfsetdash{}{0pt}%
\pgfpathmoveto{\pgfqpoint{3.406769in}{0.432170in}}%
\pgfpathlineto{\pgfqpoint{4.309722in}{0.432170in}}%
\pgfpathquadraticcurveto{\pgfqpoint{4.331944in}{0.432170in}}{\pgfqpoint{4.331944in}{0.454392in}}%
\pgfpathlineto{\pgfqpoint{4.331944in}{1.419706in}}%
\pgfpathquadraticcurveto{\pgfqpoint{4.331944in}{1.441928in}}{\pgfqpoint{4.309722in}{1.441928in}}%
\pgfpathlineto{\pgfqpoint{3.406769in}{1.441928in}}%
\pgfpathquadraticcurveto{\pgfqpoint{3.384547in}{1.441928in}}{\pgfqpoint{3.384547in}{1.419706in}}%
\pgfpathlineto{\pgfqpoint{3.384547in}{0.454392in}}%
\pgfpathquadraticcurveto{\pgfqpoint{3.384547in}{0.432170in}}{\pgfqpoint{3.406769in}{0.432170in}}%
\pgfpathclose%
\pgfusepath{stroke,fill}%
\end{pgfscope}%
\begin{pgfscope}%
\pgfsetrectcap%
\pgfsetroundjoin%
\pgfsetlinewidth{1.003750pt}%
\definecolor{currentstroke}{rgb}{0.121569,0.466667,0.705882}%
\pgfsetstrokecolor{currentstroke}%
\pgfsetdash{}{0pt}%
\pgfpathmoveto{\pgfqpoint{3.428992in}{1.358594in}}%
\pgfpathlineto{\pgfqpoint{3.651214in}{1.358594in}}%
\pgfusepath{stroke}%
\end{pgfscope}%
\begin{pgfscope}%
\pgftext[x=3.740103in,y=1.319706in,left,base]{\rmfamily\fontsize{8.000000}{9.600000}\selectfont Thompson}%
\end{pgfscope}%
\begin{pgfscope}%
\pgfsetbuttcap%
\pgfsetroundjoin%
\pgfsetlinewidth{1.003750pt}%
\definecolor{currentstroke}{rgb}{1.000000,0.498039,0.054902}%
\pgfsetstrokecolor{currentstroke}%
\pgfsetdash{{1.000000pt}{1.650000pt}}{0.000000pt}%
\pgfpathmoveto{\pgfqpoint{3.428992in}{1.203661in}}%
\pgfpathlineto{\pgfqpoint{3.651214in}{1.203661in}}%
\pgfusepath{stroke}%
\end{pgfscope}%
\begin{pgfscope}%
\pgftext[x=3.740103in,y=1.164773in,left,base]{\rmfamily\fontsize{8.000000}{9.600000}\selectfont IDS}%
\end{pgfscope}%
\begin{pgfscope}%
\pgfsetbuttcap%
\pgfsetroundjoin%
\pgfsetlinewidth{1.003750pt}%
\definecolor{currentstroke}{rgb}{0.172549,0.627451,0.172549}%
\pgfsetstrokecolor{currentstroke}%
\pgfsetdash{{3.700000pt}{1.600000pt}}{0.000000pt}%
\pgfpathmoveto{\pgfqpoint{3.428992in}{1.043193in}}%
\pgfpathlineto{\pgfqpoint{3.651214in}{1.043193in}}%
\pgfusepath{stroke}%
\end{pgfscope}%
\begin{pgfscope}%
\pgftext[x=3.740103in,y=1.004304in,left,base]{\rmfamily\fontsize{8.000000}{9.600000}\selectfont Whittle(1)}%
\end{pgfscope}%
\begin{pgfscope}%
\pgfsetbuttcap%
\pgfsetroundjoin%
\pgfsetlinewidth{1.003750pt}%
\definecolor{currentstroke}{rgb}{0.839216,0.152941,0.156863}%
\pgfsetstrokecolor{currentstroke}%
\pgfsetdash{{6.400000pt}{1.600000pt}{1.000000pt}{1.600000pt}}{0.000000pt}%
\pgfpathmoveto{\pgfqpoint{3.428992in}{0.876553in}}%
\pgfpathlineto{\pgfqpoint{3.651214in}{0.876553in}}%
\pgfusepath{stroke}%
\end{pgfscope}%
\begin{pgfscope}%
\pgftext[x=3.740103in,y=0.837665in,left,base]{\rmfamily\fontsize{8.000000}{9.600000}\selectfont Whittle(2)}%
\end{pgfscope}%
\begin{pgfscope}%
\pgfsetrectcap%
\pgfsetroundjoin%
\pgfsetlinewidth{1.003750pt}%
\definecolor{currentstroke}{rgb}{0.580392,0.403922,0.741176}%
\pgfsetstrokecolor{currentstroke}%
\pgfsetdash{}{0pt}%
\pgfpathmoveto{\pgfqpoint{3.428992in}{0.709914in}}%
\pgfpathlineto{\pgfqpoint{3.651214in}{0.709914in}}%
\pgfusepath{stroke}%
\end{pgfscope}%
\begin{pgfscope}%
\pgftext[x=3.740103in,y=0.671025in,left,base]{\rmfamily\fontsize{8.000000}{9.600000}\selectfont Whittle(3)}%
\end{pgfscope}%
\begin{pgfscope}%
\pgfsetbuttcap%
\pgfsetroundjoin%
\pgfsetlinewidth{1.003750pt}%
\definecolor{currentstroke}{rgb}{0.549020,0.337255,0.294118}%
\pgfsetstrokecolor{currentstroke}%
\pgfsetdash{{1.000000pt}{1.650000pt}}{0.000000pt}%
\pgfpathmoveto{\pgfqpoint{3.428992in}{0.543274in}}%
\pgfpathlineto{\pgfqpoint{3.651214in}{0.543274in}}%
\pgfusepath{stroke}%
\end{pgfscope}%
\begin{pgfscope}%
\pgftext[x=3.740103in,y=0.504385in,left,base]{\rmfamily\fontsize{8.000000}{9.600000}\selectfont Whittle(4)}%
\end{pgfscope}%
\end{pgfpicture}%
\makeatother%
\endgroup%

	\caption{Regret for bandits with multiple simultaneous arm pulls}
	\label{fig:restless1}
\end{figure}

\begin{table}
	\centering
	
	{
	\begin{tabular}{lrrrrrr}
		\toprule
		{} &    IDS &  Thompson &  Whittle(1) &  Whittle(2) &  Whittle(3) &  Whittle(4) \\
		\midrule
		Mean           &  10.97 &     15.23 &       11.13 &       11.29 &       10.93 &       11.07 \\
		Standard error &   0.21 &      0.13 &        0.14 &        0.15 &        0.15 &        0.15 \\
		25\%            &   1.18 &      6.60 &        1.66 &        1.57 &        1.20 &        1.39 \\
		50\%            &  10.84 &     14.75 &       10.34 &        9.96 &        9.91 &        9.74 \\
		75\%            &  24.60 &     23.52 &       19.62 &       19.41 &       19.27 &       19.13 \\
		CPU time (s)   & 54.45 &      2.07 &       14.20 &     1122.12 &     2196.83 &     4106.89 \\
		\bottomrule
	\end{tabular}
	}
	\caption{Regret from the multiple arm pulls experiment. ``Whittle($K$)" refers to the Whittle heuristic policy, where $K$ look-ahead steps are used in computing the Optimistic Gittins index.}
	\label{table:restless1_summary}

	
\end{table}

\section{Model and Preliminaries} \label{sec:model_and_prelim}

The multi-armed bandit problem is described via a handful of primitives. These include the notion of an `arm', the concept of an arm selection rule or policy and the notion of regret. This section seeks to formalize each of these notions.

\noindent\textbf{\textsf{Arms:}}
We consider a multi-armed bandit problem with $A > 1$ arms. We index arms by $i$ and denote by $\Ascr$ the set of all arm indices, $\{1,\ldots,A\}$. At each point in time, $t \in \N$, we are permitted to select or `pull' a single arm. We denote by $N_i(t)$ the cumulative number of pulls of arm $i$ up to and including time $t$. If arm $i$ were pulled at time $t$, we collect a reward  $X_{i,N_i(t)} \in \R$. 

All random variables are generated on a common probability space $\left(\Omega, \Fscr, \mathbb{P}\right)$. For a given arm $i$, $(X_{i,s}, s \in \N)$ is assumed to be an i.i.d. sequence of random variables, each distributed according to a distribution $p_{\theta_i}(\cdot)$. Denote by $\mu(\theta_i)$ the mean of this distribution. Thus, $\theta_i$ is a parameter specifying the reward distribution for arm $i$ and we denote by $\Theta$ the set of all possible values of $\theta_i$. 
We let 
\[
\theta \triangleq \left(\theta_1,\theta_2, \dots, \theta_A\right)
\]
denote a tuple of the parameters defining the reward distributions for all of the arms. $(X_{i,s}, i \in \Ascr, s \in \mathbb{N})$ is itself assumed to be an independent sequence of random variables so that the arms are independent. 


\noindent\textbf{\textsf{Policies:}} At every point in time, we choose an arm to pull according to some history dependent policy $\pi$. Formally, any policy $\pi$ is specified by an $\Ascr$-valued stochastic process $(\pi_t, t \in \mathbb{N})$. Denote by $\Fscr_t$ the filtration generated by the sequence of indices of the first $t-1$ arms pulled, as well as their corresponding rewards
\[
\mathcal{F}_t
\triangleq
\sigma\left(
\left(
\pi_s, X_{\pi_s,N_{\pi_s}(s)}
\right)
, s=1,2,\dots,t-1
\right)
\]
We require that the process $\pi_t$ be $\mathcal{F}_t$-adapted,\footnote{in order to capture the possibility of a randomized policy $\mathcal{F}_t$ must also contain the realization of a random variable describing the randomization, but we ignore this here for notational brevity} and denote by $\Pi$ the space of all such policies. 

\noindent\textbf{\textsf{Frequentist Regret and Regret Optimality:}} Over time, the agent accumulates rewards, and we denote by 
\[
V(\pi, T, \theta) := 
\E \left[
	\sum_{t=1}^T
	X_{
		\pi_t,
		N_{\pi_t}(t)
	}
\Bigg |
\theta
\right]
\] 
the reward accumulated up to time $T$ when using policy $\pi$. Denote by $\mu^*(\theta)$ the maximum expected reward across arms for a given $\theta$:  $\mu^*(\theta) \triangleq \max_i \mu(\theta_i)$. The frequentist regret of a policy over $T$ time periods, for a given $\theta \in \Theta^A$, is the expected shortfall against always pulling the optimal arm for that $\theta$, namely
\[
\Regret{\pi, T, \theta} := 
T \mu^*(\theta) -
V(\pi, T, \theta)
\]
In a seminal paper, \cite{lai1985asymptotically} established a lower bound on achievable regret. They considered the class of policies under which for {\em any} choice of $\theta$ and positive constant $a$, any policy in the class achieves $o(n^a)$ regret. They showed that for any policy $\pi \in \Pi$, and any $\theta$ such that the set of arms with expected reward $\mu^*(\theta)$ is a singleton, we must have
\begin{equation}
\label{eq:lai_robbins_lb}
\liminf_T
\frac{
	\Regret{\pi, T, \theta}
}
{
	\log T
}
\geq
\sum_{i}
\frac{\mu^*(\theta) - \mu(\theta_i)}{d_{\rm KL}\left(p_{\theta_i},p_{\theta_{i^*}} \right)}
\end{equation}
where $d_{\rm KL}$ is the Kullback-Liebler divergence. A policy $\pi'$ that achieves this lower bound is considered {\em regret optimal}. Specifically, $\pi'$ is regret optimal iff 
\[
\limsup_T
\frac{
	\Regret{\pi', T, \theta}
}
{
	\log T
}
\leq
\sum_{i}
\frac{\mu^*(\theta) - \mu(\theta_i)}{d_{\rm KL}\left(p_{\theta_i},p_{\theta_{i^*}} \right)}
\]

\noindent\textbf{\textsf{Bayesian Bandits: }}A {\em Bayesian} MAB problem is endowed with additional structure: we are given a prior on $\theta$. Specifically, we suppose that each $\theta_i$ is, in fact, an independent draw according to some prior distribution $q$ that is supported on $\Theta$. We assume that $q$ is conjugate to $p_{\theta_i}$ and with a minor abuse of notation, denote by $y$ the sufficient statistic specifying $q$. We denote by $\Yscr$ the set of all possible values of $y$. 

An algorithm that leverages knowledge of $q$ will frequently maintain a posterior distribution on $\theta_i$ given observations from that arm. To that end, denote by $q_{i,s}$ the posterior distribution on $\theta_i$ given the first $s$ rewards from that arm, $X_{i,1},X_{i,2},\dots,X_{i,s}$. Denote by $y_{i,s}$ the corresponding values of the sufficient statistic describing the posterior. Of course, $q_{i,1} \triangleq q$. 

Now, one can define a notion of regret that depends on the prior $q$. Specifically, the Bayes' risk (or Bayesian regret) for any policy $\pi$ is simply the expected regret over draws of $\theta$ according to the prior $q$:
\[
\Regret{\pi, T} := \int_\Theta \Regret{\pi, T, \theta} q^A\left(d\theta\right)
\]
In yet another landmark paper, \cite{lai1985asymptotically} showed that for a restricted class of priors $q$ a similar class of algorithms to those found to be regret optimal in \cite{lai1985asymptotically} were also Bayes optimal. Interestingly, however, this class of algorithms ignores information about the prior altogether -- i.e. they do not require knowledge of $q$. A number of algorithms that {\em do} exploit prior information have in recent years received a good deal of attention; these include Thompson sampling \cite{thompson1933likelihood}, Bayes-UCB \cite{kaufmann2012thompson}, KL-UCB \cite{garivier2011kl}, and Information Directed Sampling \cite{russo2014learning}. All of these algorithms maintain a posterior on the mean of an arm, but leverage this posterior in different ways. It has been empirically observed that these approaches offer excellent performance, even in a frequentist sense. In fact, Thompson sampling \cite{thompson1933likelihood}, Bayes-UCB \cite{kaufmann2012thompson} and KL-UCB \cite{garivier2011kl} have each been shown to be regret optimal in the sense of meeting the lower bound  \eqref{eq:lai_robbins_lb}. 
 
\noindent\textbf{\textsf{The Discounted Infinite Horizon Objective: }}Assuming the structure afforded by the Bayesian setting, i.e. the prior $q$, one may consider a distinct objective to Bayesian regret. Specifically, given some fixed discount factor $\gamma < 1$, one could consider the problem of maximizing discounted infinite horizon rewards. Assume we start with a prior q on the mean of any arm; and as before denote by $y$ the sufficient statistic corresponding to this prior. In the parlance of Markov Decisions Processes, we might refer to this as starting with ever arm in state $y$. For a given policy $\pi$, we define the expected discounted infinite horizon reward under that policy according to
\[
V^\pi_\gamma(\mathbf{y}) 
=
\E_{\mathbf{y}}
\left[
	\sum_{t=1}^\infty \gamma^{t-1} X_{\pi_t,N_{\pi_t}(t)}
\right]
\]
where $\mathbf{y}$ is an $A$-tuple all of with every entry equal to $y$. The subscript on the expectation indicates that $\theta_i$ is drawn according to a prior with sufficient statistic $y$ for each arm $i$. An optimal such policy must solve the problem
\[
V^*_\gamma(\mathbf{y}) 
\triangleq 
\max_{\pi \in \Pi} V^\pi_\gamma(\mathbf{y}) 
\]
This is, of course a challenging MDP in that it has a high dimensional state space ($\Yscr^A$). The celebrated Gittins index theorem (which we present in the next section) provides an approach to computing an optimal policy by instead simply solving a dynamic program on the state space $\Yscr$. 


%We write $V(\pi,T) := \E{V(\pi,T,\theta)}$. 
%
%The regret of a policy over $T$ time periods, for a given $\theta \in \Theta^A$, is the expected shortfall against always pulling the optimal arm, namely
%\[
%\Regret{\pi, T, \theta} := 
%T \mu^*(\theta) -
%V(\pi, T, \theta)
%\]
%
%
%
%We denote by $\mu^*(\theta)$ the maximum expected reward across arms;  $\mu^*(\theta) := \max_i \mu_i(\theta_i)$ and let $i^*$ be an optimal arm. The present paper will focus on the Bayesian setting, and so we suppose that each $\theta_i$ is an independent draw from some prior distribution $q$ over $\Theta$. 
%
%We define a policy, $\pi := (\pi_t, t \in \mathbb{N})$, to be a stochastic process taking values in $\mathcal{A}$. We require that $\pi$ be adapted to the filtration $\mathcal{F}_t$ generated by the history of arm pulls and their corresponding rewards up to and including time $t-1$.
%
%
%
%The Bayesian setting endows us with the structure of a (high dimensional) Markov Decision process. An alternative objective to minimizing Bayes risk, is the maximization of the cumulative reward discounted over an infinite horizon. Specifically, for any positive discount factor $\gamma < 1$, define
%\[
%V_\gamma(\pi) := 
%\E_q \left[
%	\sum_{t=1}^\infty \gamma^{t-1} X_{\pi_t,N_{\pi_t}(t)}
%\right].
%\]
%The celebrated Gittin's index theorem provides an {\em optimal, efficient} solution to this problem that we will describe in greater detail shortly; unfortunately as alluded to earlier even a minor `tweak' to the objective above -- such as maximizing cumulative expected reward over a finite horizon renders the Gittins index sub-optimal \cite{nino2011computing}. 
%
%As a final point of notation, every scheme we consider will maintain a posterior on the mean of an arm at every point in time. We denote by $q_{i,s}$ the posterior on the mean of the $i$th arm after $s-1$ pulls of that arm; $q_{i,1} := q$. Since our prior on $\theta_i$ will frequently be conjugate to the distribution of the reward $X_i$, $q_{i,s}$ will permit a succinct description via a sufficient statistic we will denote by $y_{i,s}$; denote the set of all such sufficient statistics $\mathcal{Y}$. We will thus use $q_{i,s}$ and $y_{i,s}$ interchangeably and refer to the latter as the `state' of the $i$th arm after $s-1$ pulls.  
%

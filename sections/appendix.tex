\appendix
\section{Proof of Lemma~\ref{le:linearregret}} \label{proof:linearregret}
\begin{myproof}[Proof.]
	Consider an instance of the MAB with $A=2$ arms and Bernoulli rewards. We assume that the prior on arm $1$ is degenerate with mean $\lambda = 1/2$ while arm $2$ has a {\rm Beta}$(\alpha,\alpha)$ prior where $\alpha$ is a parameter we set later. Then, it is simple to check that $\pi^{G,\gamma}$ must pull arm $2$ at the first time period. With probability $1/2$, we receive a reward of $0$, so that the posterior on arm $2$ is given by a {\rm Beta}$(\alpha,\alpha+1)$ prior. Now, the continuation value from pulling arm $1$ at this stage is lower bounded by $\frac{1/2}{1-\gamma}$, while the continuation value from pulling arm $2$ is upper bounded by
	\[
	\frac{\alpha}{1+\alpha}
	+
	\frac{
		\gamma
		\E
		\left[
		\max\left(
		R(y_{2,0}),1/2
		\right)
		\Big| y_{2,0} = \left(\alpha, \alpha+1 \right)
		\right]
	}
	{
		1-\gamma
	}.
	\]
	It follows that any optimal policy must pull arm $1$ if
	\[
	\frac{1/2}{1-\gamma}
	>
	\frac{\alpha}{1+\alpha}
	+
	\frac{
		\gamma
		\E
		\left[
		\max\left(
		R(y_{2,0}),1/2
		\right)
		\Big| y_{2,0} = \left(\alpha, \alpha+1 \right)
		\right]
	}
	{
		1-\gamma
	},
	\]
	an inequality which in turn is satisfied if
	\[
	1/2
	>
	\frac{\alpha}{1+\alpha}
	+
	\frac{
		\gamma
		\P{
			R(y_{2,0})>1/2 \
			\Big| y_{2,0} = \left(\alpha, \alpha+1 \right)
		}
	}
	{
		1-\gamma
	},
	\]
	But the right hand side of the above expression goes to $0$ as $\alpha \tends 0$. Consequently, we can choose an $\alpha$ such that any optimal policy (for the discounted infinite horizon problem) chooses to pull the first arm; let $\alpha^*$ be the largest such $\alpha$. Since the state of the first arm does not change (the prior on that arm was assumed degenerate), the same condition must hold at subsequent iterations. Consequently, $\pi^{G,\gamma}$ must incur $T$-period regret lower bounded by 
	$\frac{T}{2}
	\E
	\left[
	(
	R((\alpha^*,\alpha^*+1)) - 1/2
	)^+
	\right].
	$
	The result follows. 
\end{myproof}
\section{Proof of Proposition~\ref{prop:gittins_log3T}} \label{proof:prop_log3T}

We begin by establishing a simple Lemma concerning the allocation rule proposed in \cite{lai1987adaptive}; we note that the allocation rule there is specified for a fixed time horizon, and the Lemma below extends it to a policy as defined in this paper. 

\begin{lemma} 
\label{lemma:laianytime}
Let the prior with sufficient statistic $\mathbf y$ satisfy the requirements of Theorem~$3$ in \cite{lai1987adaptive}. Then, there exists a policy $\tilde \pi_{\mathbf y}$, and a constant $C_{\mathbf y}$ for which
\[
{\rm Regret}\left(
\tilde \pi_{\mathbf y}, T
\right)
\triangleq 
\E_{\mathbf y}
\left[
{\rm Regret}\left(
\tilde \pi_{\mathbf y}, T,\theta
\right)
\right]
\leq
C_{\mathbf y} \log^3 T
\] 
for all $T$. 
\end{lemma}
\begin{myproof}[Proof.]
By Theorem $3$ in \cite{lai1987adaptive}, we know that there exists a constant $\bar C_{\mathbf y}$ such that for any $T$, there exists a policy, $\tilde \pi_{\mathbf y, T}$  (that depends on $T$ and is specified only up to time $T$), for which
\[
\E
_{\mathbf y}
\left[
{\rm Regret}\left(
\tilde \pi_{\mathbf y, T}, T,\theta
\right)
\right]
\leq
\bar C_{\mathbf y} \log^2 T
\]
Now consider the `doubling' policy $\tilde \pi_{\mathbf y}$ which at time $t$ selects arms according to the policy $\tilde \pi_{\mathbf y,2^k}$ applied at the state $g\left(\mathbf y, h_{2^{k(t)}-1}^t\right)$ where $k(t) = \lfloor \log_2(t+1) \rfloor$. In words, this is the policy obtained wherein (a) time is divided into epochs such that the $k$th epoch extends from time $2^{k}-1$ to $2^{k+1}-2$, and (b) at the start of the $k$th epoch we `forget' everything learned up until that time and subsequently use the policy $\tilde \pi_{\mathbf y,2^k}$ over the course of that epoch. Thus,
\[
\begin{split}
\E
_{\mathbf y}
\left[
{\rm Regret}\left(
\tilde \pi_{\mathbf y}, T,\theta
\right)
\right]
&
\leq
\sum_{k=1}^{\lfloor \log (T+2) \rfloor}
\E_{\mathbf y}
\left[
\E_{\mathbf y_{2^k -2}}
\left[
{\rm Regret}\left(
\tilde \pi_{\mathbf y,2^k}, 2^k,\theta
\right)
\right]
\right]
\\
&
=
\sum_{k=1}^{\lfloor \log (T+2) \rfloor}
\E
\left[
{\rm Regret}\left(
\tilde \pi_{\mathbf y,2^k}, 2^k,\theta
\right)
\right]
\\
&
\leq
\sum_{k=1}^{\lfloor \log (T+2) \rfloor}
C_y \log^2 2^k
\\
&
\leq
\bar C_y C \log^3 T
\end{split}
\]
where the equality follows from the tower property and where $C$ is some absolute constant. 
\end{myproof}

Next, we establish a simple result related to the Gittins index policy that relates the finite horizon performance of the policy to the (discounted) infinite horizon performance. 

\begin{lemma}
\label{lemma:gittinsfiniteinfinite}
For any $\mathbf{\hat y} \in \Yscr$, and horizon $T' \geq 2$, we have
\[
\E
_{\hat{\mathbf y}}
\left[
{\rm Regret}\left(
\pi^{G,1-1/{T'}}, T',\theta
\right)
\right]
\leq
4
\E
_{\hat{\mathbf y}}
\left[
{\rm Regret}\left(
\pi^{G,1-1/{T'}}, H_{T'},\theta
\right)
\right]
\]
where $H_{T'}$ is an independent Geometrically distributed random variable with mean $T'$. 
\end{lemma}
\begin{myproof}[Proof.]
We have
\[
\begin{split}
\E_{\hat{\mathbf y}}\left[\Regret{\pi^{G, 1-1/{T'}}, T', \theta}\right]  
& = 
\E_{\hat{\mathbf y}}\left[\Regret{\pi^{G, 1-1/{T'}}, T', \theta}\right]\frac{\P{H_{T'} > T'}}{(1 - 1/{T'})^{T'}}
\\
& \leq
\E_{\hat{\mathbf y}}\left[\Regret{\pi^{G, 1-1/{T'}}, H_{T'}, \theta} \given H_{T'} > T'\right]\frac{\P{H_{T'} > T'} }{(1 - 1/{T'})^{T'}}
\\ 
& \leq
{(1 - 1/{T'})^{-T'}} \E_{\hat{\mathbf y}}\left[\Regret{\pi^{G, 1-1/{T'}}, H_{T'}, \theta}\right]
\\
& \leq
4 \E_{\hat{\mathbf y}}\left[\Regret{\pi^{G, 1-1/{T'}}, H_{T'}, \theta}\right]
\end{split}
\]
where the first inequality follows from the fact that $\E_{\hat{\mathbf y}}\left[\Regret{\pi^{G, 1-1/{T'}}, n, \theta}\right]$ is non-decreasing in $n$, the second inequality follows from the fact that regret is non-negative. 
\end{myproof}

We can now proceed with the proof of the proposition. First, we note that since the Gittins policy with discount factor $1-1/{T'}$ is optimal for a geometrically distributed horizon with mean $T'$, we must have for any $\mathbf{\hat y} \in \Yscr$, 
\begin{equation}
\label{eq:gittinsopt}
\E_{\hat{\mathbf y}}\left[\Regret{\pi^{G, 1-1/{T'}}, H_{T'}, \theta}\right]
\leq
\E_{\hat{\mathbf y}}\left[\Regret{\tilde \pi_{\mathbf{y}}, H_{T'}, \theta}\right]
\end{equation}
But we have
\[
\begin{split}
\E_{{\mathbf y}}\left[\Regret{\pi^{D}, T, \theta}\right]
&\leq
\sum_{k=1}^{\lfloor \log (T+2) \rfloor}
\E_{{\mathbf y}}\left[
\E_{\hat{\mathbf y}_{2^k-2}}\left[
\Regret{\pi^{G,1-1/2^k}, 2^k, \theta}\right]
\right]
\\
&\leq
\sum_{k=1}^{\lfloor \log (T+2) \rfloor}
\E_{{\mathbf y}}\left[
4
\E_{\hat{\mathbf y}_{2^k-2}}\left[
\Regret{\pi^{G,1-1/2^k}, H_{2^k}, \theta}\right]
\right]
\\
&\leq
\sum_{k=1}^{\lfloor \log (T+2) \rfloor}
\E_{{\mathbf y}}\left[
4
\E_{\hat{\mathbf y}_{2^k-2}}\left[
\Regret{\tilde \pi_{\mathbf{y}}, H_{2^k}, \theta}
\right]
\right]
\\
&
=
4\sum_{k=1}^{\lfloor \log (T+2) \rfloor}
\E_{{\mathbf y}}\left[
\Regret{\tilde \pi_{\mathbf{y}}, H_{2^k}, \theta}
\right]
\\
&
\leq 
4 C_{\mathbf y}\sum_{k=1}^{\lfloor \log (T+2) \rfloor}
\E \left[ \log^3 H_{2^k} \right]
\\
&
\leq 
4 C_{\mathbf y}\sum_{k=1}^{\lfloor \log (T+2) \rfloor}
k^3
\end{split}
\]
where the first inequality follows simply from the definition of $\pi^D$, the second inequality follows from Lemma~\ref{lemma:gittinsfiniteinfinite}, the third inequality follows from the aforementioned optimality of the Gittins policy (namely, \eqref{eq:gittinsopt}), the first equality follows from the tower property, the fourth inequality follows from Lemma~\ref{lemma:laianytime}, and the fifth and final inequality is simply Jensen's inequality. 




%\begin{myproof}[Proof.]
%	%First, letting $\gamma_n = 1 - 1/n$, we show that
%	%\begin{equation} \label{eq:basic_finite_time_gittins}
%	%\Regret{\pi^{G, \gamma_n}, n} = O\left( \log^2(n) \right).
%	%\end{equation}
%	We start by proving a regret bound for the Gittins index policy over a finite horizon.
%	Since we are dealing with Bayesian regret, we will prove our bound by comparing against the Information Directed Sampling (IDS) policy, denoted by $\pi^{IDS}$. It was shown in \citep{russo2014learning} (Proposition 2) that for any horizon $T \in \mathbb N$ and any prior, described by a vector $\hat{\mathbf y}$, that
%	\[
%	\E_{\hat{\mathbf y}}\left[ \Regret{\pi^{IDS}, T, \theta} \right] 
%	 \le  \sqrt{\frac{1}{2} (A \log A) T}.
%	\]
%	whenever the reward distribution is uniformly bounded.
%	We will leverage this useful result in the next step of the proof.
%	
%	Fix any $n \ge 2$ and sufficient statistic $\hat{\mathbf y}$. Define $H$ to be a geometric random variable with mean $n$, then	
%	\begin{align}
%	\E_{\hat{\mathbf y}}\left[\Regret{\pi^{G, \gamma_n}, n, \theta}\right]  & = \E_{\hat{\mathbf y}}\left[\Regret{\pi^{G, \gamma_n}, n, \theta}\right]\frac{\P{H > n}}{(1 - 1/n)^n} \nonumber \\
%	&  \le  \E_{\hat{\mathbf y}}\left[\Regret{\pi^{G, \gamma_n}, H, \theta} \given H > n\right]\frac{\P{H > n} }{(1 - 1/n)^n} \nonumber\\
%	& \le  (1 - 1/n)^{-n}  \E_{\hat{\mathbf y}}\left[ \Regret{\pi^{G,\gamma_n}, H, \theta} \right] \nonumber   \\
%	& \le  4 \E_{\hat{\mathbf y}}\left[ \Regret{\pi^{G,\gamma_n}, H, \theta} \right]   \nonumber \\
%	& \le 4\E_{\hat{\mathbf y}}\left[ \Regret{\pi^{IDS}, H, \theta} \right] \label{eqn:opt_gittins_over_geo_horizon} \\
%	& \le 4 \E\left[  \sqrt{\frac{1}{2} (A \log A) H} \right] \nonumber\\
%	& \le  \E\left[  \sqrt{8 (A \log A) H} \right] \nonumber \\
%	& \le  \sqrt{8 (A \log A) \E[H]}  \label{eqn:jensens_ids} \\
%	& =  \sqrt{8 (A \log A) n} \label{eqn:regret_bound_for_ids}   
%	\end{align}
%	where \eqref{eqn:opt_gittins_over_geo_horizon} follows from the Gittins index policy being optimal over the random \& geometrically distributed horizon of $H$ (because $\E[H] = n$ and the discount factor is assumed to be $1 - 1/n$ here). Equation~\eqref{eqn:jensens_ids} follows from Jensen's inequality.
%	
%	Back to the doubling trick policy. Recall that this entails pulling the arms according to $\pi^{G,1 - 1/2^{k-1}}$ during periods $\{2^{k-1},\ldots,2^{k}-1\}$ for each $k \in \mathbb{N}$.
%	For convenience, we define the function $\text{Regret}(\pi, t_1, t_2, \theta)$ to be the total regret accumulated from periods $t_1$ up to $t_2$ under policy $\pi$ and where the arms' parameters are set to $\theta$. We will also denote the tuple of sufficient statistics for the arms in period $t$ as $\mathbf y_t$, to reflect all the Bayesian updates up to and inlcuding that time.
%	
%	Now we can bound the Bayes risk up to time $T$ as follows:
%	\begin{align}
%		\E_{\mathbf y}\left[\Regret{\pi^D, T, \theta} \right] &
%		\leq 1 + \sum_{k=2}^{\ceil{\log_2 T}} \E_{\mathbf y}\left[\Regret{\pi^D, 2^{k-1}, 2^k - 1, \theta} \right] \nonumber \\
%		& = 1+ \sum_{k=2}^{\ceil{\log_2 T}} \E_{\mathbf y}\left[ \E_{\mathbf y_{2^k-1}}\left[\Regret{\pi^{G, 1 - 1/2^{k-1}}, 2^{k-1}, \theta} \right] \right] \nonumber \\
%		& \le 1 + \sum_{k=2}^{\ceil{\log_2 T}}  \sqrt{8(A \log A) 2^{k-1}} \label{eqn:application_of_ids}\\
%		& = \mathcal O\left( \sqrt{T} \right), \nonumber
%	\end{align}
%	where inequality \eqref{eqn:application_of_ids} is implied by  \eqref{eqn:regret_bound_for_ids}.
%\end{myproof}

\section{Properties of the Optimistic Gittins index}\label{sec:appendix_properties_of_ogi}
This section gives proofs for a few properties of the Optimistic Gittins index that are used throughout the paper and particularly in the proof of Theorem~\ref{thm:frequentist_optimal_bound}.  
It shall be useful, in what follows, to define the continuation value for the Whittle's retirement problem (\cite{whittle1980multi}) as
\[
V_\gamma(y, \lambda)  \defeq \sup_{\tau > 0} \E_y\left[\sum_{t=1}^{\tau} \gamma^{t-1} X_{i,t} + \gamma^{\tau} \frac{\lambda}{1-\gamma}\right],
\]
so that the Gittins index is then the solution in $\lambda$ to $\lambda/(1-\gamma) = V_\gamma(y, \lambda)$. In an analogous fashion, we define the optimistic continuation value, for parameters $K$ and $\lambda$, to be
\[
V^K_\gamma(y, \lambda) \defeq \sup_{1 \le \tau \le K} \E_y\left[\sum_{t=1}^{\tau} \gamma^{t-1}  X_{i,t} + \gamma^{\tau} \frac{R_{\lambda, K}(\tau, y_{i,\tau-1})}{1-\gamma}\right].
\]
From this definition, it follows that the solution for $\lambda$ to the equation $\lambda/(1-\gamma) = V^K_\gamma(y, \lambda)$ is the Optimistic Gittins index.

Throughout this section, we will sometimes discuss the value of the index at some particular time $t$ during the execution of the algorithm, which depends on the statistic gathered about the arm using information up to but strictly \emph{not including} time $t$. As such, we will define the number of pulls of arm $i$ up to time $t-1$ as
\[
\Ntg{i}{t} \defeq N_i(t-1)
\]
where we recall $N_i(t)$ is the counter for the number of total number of pulls up to and including $t$. From the $\Ntg{i}{t}$ pulls of the arm, the total reward accumulated is defined as
\[
S_i(t) \defeq \sum_{s=1}^{\Ntg{i}{t}} X_{i,s}.
\]

We begin by investigating the effect of the parameter $\lambda$, which gives the deterministic payoff in \eqref{eqn:ogi_general}, on the continuation value $V^K_\gamma(y, \lambda)$ and use that to find out how close an approximation $v^K_\gamma(y)$ is to the Gittins index.
\begin{fact}\label{fact:v_is_convex}
	For any state $y \in \mathcal{Y}$, discount factor $\gamma$ and parameter $K$, the function $V^K_\gamma(y,\lambda)$ is convex in $\lambda$.% Moreover, if $R(y)$ is a continuous random variable, the function is also differentiable.
\end{fact}
\begin{myproof}[Proof.]
	Fix an arbitrary state $y$ and discount factor $\gamma \in (0,1)$. We prove convexity by induction on the parameter $K$. For $K = 1$, recall from Section~\ref{sec:gittins_and_approx} that
	\begin{align*}
	V^1_\gamma(y, \lambda) = \E_y\left[X_{i,1} \right] +  \gamma \E_y\left[\max(\lambda/(1-\gamma), R(y_{i,0}))\right].
	\end{align*}
	Thus the function is convex because it is an expectation over a convex piecewise linear function of random variables $X_{i,1}$ and $R(y_{i,0})$. 
	
	%Now assume that $R(y)$ is a continuous random variable. We will verify through the bounded convergence theorem that $V^1_\gamma(y, \lambda)$ is differentiable. Indeed, this holds because the event $\{ R(y)= \lambda/(1-\gamma)\}$, at which the random variable inside the expectation is not differentiable, has measure zero, precisely because $R(y)$ is a continuous random variable.
	Now we show the inductive step. For any $K > 1$, assume that $V^{K-1}_\gamma(y, \lambda)$ is convex. By writing the Bellman equation,
	\begin{align*}
	V^K_\gamma(y, \lambda) & = \E_y\left[X_{i,1}\right] + \gamma \E_y\left[\max\left(\lambda/(1-\gamma), V^K_\gamma(y_{i,1}, \lambda)\right) \right],
	\end{align*}
	we again notice an expectation over a maximum of convex functions in $\lambda$. This form for $V^K_\gamma(y, \lambda)$ implies that  it is also convex in $\lambda$.% If $V^{K-1}$ is differentiable and $R(y)$ is a continuous random variable, then the form of $V^{K-1}$ also implies that it is differentiable in $\lambda$.
\end{myproof}

\begin{lemma} \label{cor:equivalent_event}
	Suppose that arm rewards are bounded. That is, there exists a constant $B \in \Re_+$ such that $X_{i,t} \in [0, B]$ for every arm $i$ and time $t$. 
	
	Let $v^K_{i,t}$ be the Optimistic Gittins Index of arm $i$ at time $t$ and let $\eta$ be a scalar, then the following equivalence holds
	\[
	\{v^K_{i,t} < \eta \} = \{(1-\gamma_t)V^K_{\gamma_t}(y_{i,\Ntg{i}{t}}, \eta) < \eta\}\]
	where $y_{i,\Ntg{1}{t}}$ is the sufficient statistic for estimating the $i$th arm's parameter $\theta_i$ at time $t$.
\end{lemma}
\begin{myproof}[Proof.]
	First of all note that for any state $y$ and discount factor $\gamma$, the function
	\begin{align*}
	(1-\gamma)V^K_\gamma(y, \lambda) - \lambda&  = \sup_{1 \le \tau \le K}\E_y\left[\sum_{s=1}^\tau\gamma^{s-1}(1-\gamma)X_{i,s} + \ind{\tau = K}\gamma^\tau(R(y_{i,\tau-1}) - \lambda)^+\right],
	\end{align*}
	is convex (from Fact~\ref{fact:v_is_convex}) yet decreasing in $\lambda$. Also notice that at $\lambda = 0$
	\[
		V^K_{\gamma}(y, 0) = \frac{\E_y\left[X_{i,1}\right]}{(1-\gamma)} \ge 0
	\]
	because it is never optimal to retire in the stopping problem. Also, in the other extreme case when $\lambda = B$, the function in question evaluates to
	\[
		V^K_{\gamma}(y, B) = \E_y\left[X_{i,1}\right] + \frac{\gamma B}{(1-\gamma)} \le \frac{B}{(1-\gamma)}.
	\]
	Thus, consider again the above function of $\lambda$, namely,  $(1-\gamma)V^K_\gamma(y, \lambda) - \lambda$. Such a function is non-negative for any $\lambda \le v^K_\gamma(y)$ (since $v^K_\gamma(y)$ is the root of the function) and is also negative for $\lambda > v^K_\gamma(y)$.
	\begin{figure}
		\centering
		%% Creator: Matplotlib, PGF backend
%%
%% To include the figure in your LaTeX document, write
%%   \input{<filename>.pgf}
%%
%% Make sure the required packages are loaded in your preamble
%%   \usepackage{pgf}
%%
%% Figures using additional raster images can only be included by \input if
%% they are in the same directory as the main LaTeX file. For loading figures
%% from other directories you can use the `import` package
%%   \usepackage{import}
%% and then include the figures with
%%   \import{<path to file>}{<filename>.pgf}
%%
%% Matplotlib used the following preamble
%%   \usepackage[utf8x]{inputenc}
%%   \usepackage[T1]{fontenc}
%%
\begingroup%
\makeatletter%
\begin{pgfpicture}%
\pgfpathrectangle{\pgfpointorigin}{\pgfqpoint{3.900000in}{2.410333in}}%
\pgfusepath{use as bounding box, clip}%
\begin{pgfscope}%
\pgfsetbuttcap%
\pgfsetmiterjoin%
\definecolor{currentfill}{rgb}{1.000000,1.000000,1.000000}%
\pgfsetfillcolor{currentfill}%
\pgfsetlinewidth{0.000000pt}%
\definecolor{currentstroke}{rgb}{1.000000,1.000000,1.000000}%
\pgfsetstrokecolor{currentstroke}%
\pgfsetdash{}{0pt}%
\pgfpathmoveto{\pgfqpoint{0.000000in}{0.000000in}}%
\pgfpathlineto{\pgfqpoint{3.900000in}{0.000000in}}%
\pgfpathlineto{\pgfqpoint{3.900000in}{2.410333in}}%
\pgfpathlineto{\pgfqpoint{0.000000in}{2.410333in}}%
\pgfpathclose%
\pgfusepath{fill}%
\end{pgfscope}%
\begin{pgfscope}%
\pgfsetbuttcap%
\pgfsetmiterjoin%
\definecolor{currentfill}{rgb}{1.000000,1.000000,1.000000}%
\pgfsetfillcolor{currentfill}%
\pgfsetlinewidth{0.000000pt}%
\definecolor{currentstroke}{rgb}{0.000000,0.000000,0.000000}%
\pgfsetstrokecolor{currentstroke}%
\pgfsetstrokeopacity{0.000000}%
\pgfsetdash{}{0pt}%
\pgfpathmoveto{\pgfqpoint{0.487500in}{0.301292in}}%
\pgfpathlineto{\pgfqpoint{3.510000in}{0.301292in}}%
\pgfpathlineto{\pgfqpoint{3.510000in}{2.169299in}}%
\pgfpathlineto{\pgfqpoint{0.487500in}{2.169299in}}%
\pgfpathclose%
\pgfusepath{fill}%
\end{pgfscope}%
\begin{pgfscope}%
\pgfpathrectangle{\pgfqpoint{0.487500in}{0.301292in}}{\pgfqpoint{3.022500in}{1.868008in}} %
\pgfusepath{clip}%
\pgfsetbuttcap%
\pgfsetroundjoin%
\pgfsetlinewidth{1.003750pt}%
\definecolor{currentstroke}{rgb}{1.000000,0.000000,0.000000}%
\pgfsetstrokecolor{currentstroke}%
\pgfsetdash{{6.000000pt}{6.000000pt}}{0.000000pt}%
\pgfpathmoveto{\pgfqpoint{0.487500in}{0.301292in}}%
\pgfpathlineto{\pgfqpoint{0.591724in}{0.365706in}}%
\pgfpathlineto{\pgfqpoint{0.695948in}{0.430120in}}%
\pgfpathlineto{\pgfqpoint{0.800172in}{0.494534in}}%
\pgfpathlineto{\pgfqpoint{0.904397in}{0.558948in}}%
\pgfpathlineto{\pgfqpoint{1.008621in}{0.623362in}}%
\pgfpathlineto{\pgfqpoint{1.112845in}{0.687776in}}%
\pgfpathlineto{\pgfqpoint{1.217069in}{0.752190in}}%
\pgfpathlineto{\pgfqpoint{1.321293in}{0.816604in}}%
\pgfpathlineto{\pgfqpoint{1.425517in}{0.881018in}}%
\pgfpathlineto{\pgfqpoint{1.529741in}{0.945432in}}%
\pgfpathlineto{\pgfqpoint{1.633966in}{1.009846in}}%
\pgfpathlineto{\pgfqpoint{1.738190in}{1.074260in}}%
\pgfpathlineto{\pgfqpoint{1.842414in}{1.138674in}}%
\pgfpathlineto{\pgfqpoint{1.946638in}{1.203088in}}%
\pgfpathlineto{\pgfqpoint{2.050862in}{1.267502in}}%
\pgfpathlineto{\pgfqpoint{2.155086in}{1.331917in}}%
\pgfpathlineto{\pgfqpoint{2.259310in}{1.396331in}}%
\pgfpathlineto{\pgfqpoint{2.363534in}{1.460745in}}%
\pgfpathlineto{\pgfqpoint{2.467759in}{1.525159in}}%
\pgfpathlineto{\pgfqpoint{2.571983in}{1.589573in}}%
\pgfpathlineto{\pgfqpoint{2.676207in}{1.653987in}}%
\pgfpathlineto{\pgfqpoint{2.780431in}{1.718401in}}%
\pgfpathlineto{\pgfqpoint{2.884655in}{1.782815in}}%
\pgfpathlineto{\pgfqpoint{2.988879in}{1.847229in}}%
\pgfpathlineto{\pgfqpoint{3.093103in}{1.911643in}}%
\pgfpathlineto{\pgfqpoint{3.197328in}{1.976057in}}%
\pgfpathlineto{\pgfqpoint{3.301552in}{2.040471in}}%
\pgfpathlineto{\pgfqpoint{3.405776in}{2.104885in}}%
\pgfpathlineto{\pgfqpoint{3.510000in}{2.169299in}}%
\pgfusepath{stroke}%
\end{pgfscope}%
\begin{pgfscope}%
\pgfpathrectangle{\pgfqpoint{0.487500in}{0.301292in}}{\pgfqpoint{3.022500in}{1.868008in}} %
\pgfusepath{clip}%
\pgfsetrectcap%
\pgfsetroundjoin%
\pgfsetlinewidth{1.003750pt}%
\definecolor{currentstroke}{rgb}{0.000000,0.000000,1.000000}%
\pgfsetstrokecolor{currentstroke}%
\pgfsetdash{}{0pt}%
\pgfpathmoveto{\pgfqpoint{0.487500in}{0.835008in}}%
\pgfpathlineto{\pgfqpoint{0.591724in}{0.835131in}}%
\pgfpathlineto{\pgfqpoint{0.695948in}{0.835923in}}%
\pgfpathlineto{\pgfqpoint{0.800172in}{0.837886in}}%
\pgfpathlineto{\pgfqpoint{0.904397in}{0.841366in}}%
\pgfpathlineto{\pgfqpoint{1.008621in}{0.846579in}}%
\pgfpathlineto{\pgfqpoint{1.112845in}{0.853633in}}%
\pgfpathlineto{\pgfqpoint{1.217069in}{0.862552in}}%
\pgfpathlineto{\pgfqpoint{1.321293in}{0.873295in}}%
\pgfpathlineto{\pgfqpoint{1.425517in}{0.898119in}}%
\pgfpathlineto{\pgfqpoint{1.529741in}{0.932953in}}%
\pgfpathlineto{\pgfqpoint{1.633966in}{0.968286in}}%
\pgfpathlineto{\pgfqpoint{1.738190in}{1.004555in}}%
\pgfpathlineto{\pgfqpoint{1.842414in}{1.047574in}}%
\pgfpathlineto{\pgfqpoint{1.946638in}{1.092664in}}%
\pgfpathlineto{\pgfqpoint{2.050862in}{1.137754in}}%
\pgfpathlineto{\pgfqpoint{2.155086in}{1.182844in}}%
\pgfpathlineto{\pgfqpoint{2.259310in}{1.227934in}}%
\pgfpathlineto{\pgfqpoint{2.363534in}{1.273024in}}%
\pgfpathlineto{\pgfqpoint{2.467759in}{1.318114in}}%
\pgfpathlineto{\pgfqpoint{2.571983in}{1.363203in}}%
\pgfpathlineto{\pgfqpoint{2.676207in}{1.408293in}}%
\pgfpathlineto{\pgfqpoint{2.780431in}{1.453383in}}%
\pgfpathlineto{\pgfqpoint{2.884655in}{1.498473in}}%
\pgfpathlineto{\pgfqpoint{2.988879in}{1.543563in}}%
\pgfpathlineto{\pgfqpoint{3.093103in}{1.588653in}}%
\pgfpathlineto{\pgfqpoint{3.197328in}{1.633742in}}%
\pgfpathlineto{\pgfqpoint{3.301552in}{1.678832in}}%
\pgfpathlineto{\pgfqpoint{3.405776in}{1.723922in}}%
\pgfpathlineto{\pgfqpoint{3.510000in}{1.769012in}}%
\pgfusepath{stroke}%
\end{pgfscope}%
\begin{pgfscope}%
\pgfsetrectcap%
\pgfsetmiterjoin%
\pgfsetlinewidth{1.003750pt}%
\definecolor{currentstroke}{rgb}{0.000000,0.000000,0.000000}%
\pgfsetstrokecolor{currentstroke}%
\pgfsetdash{}{0pt}%
\pgfpathmoveto{\pgfqpoint{0.487500in}{2.169299in}}%
\pgfpathlineto{\pgfqpoint{3.510000in}{2.169299in}}%
\pgfusepath{stroke}%
\end{pgfscope}%
\begin{pgfscope}%
\pgfsetrectcap%
\pgfsetmiterjoin%
\pgfsetlinewidth{1.003750pt}%
\definecolor{currentstroke}{rgb}{0.000000,0.000000,0.000000}%
\pgfsetstrokecolor{currentstroke}%
\pgfsetdash{}{0pt}%
\pgfpathmoveto{\pgfqpoint{3.510000in}{0.301292in}}%
\pgfpathlineto{\pgfqpoint{3.510000in}{2.169299in}}%
\pgfusepath{stroke}%
\end{pgfscope}%
\begin{pgfscope}%
\pgfsetrectcap%
\pgfsetmiterjoin%
\pgfsetlinewidth{1.003750pt}%
\definecolor{currentstroke}{rgb}{0.000000,0.000000,0.000000}%
\pgfsetstrokecolor{currentstroke}%
\pgfsetdash{}{0pt}%
\pgfpathmoveto{\pgfqpoint{0.487500in}{0.301292in}}%
\pgfpathlineto{\pgfqpoint{3.510000in}{0.301292in}}%
\pgfusepath{stroke}%
\end{pgfscope}%
\begin{pgfscope}%
\pgfsetrectcap%
\pgfsetmiterjoin%
\pgfsetlinewidth{1.003750pt}%
\definecolor{currentstroke}{rgb}{0.000000,0.000000,0.000000}%
\pgfsetstrokecolor{currentstroke}%
\pgfsetdash{}{0pt}%
\pgfpathmoveto{\pgfqpoint{0.487500in}{0.301292in}}%
\pgfpathlineto{\pgfqpoint{0.487500in}{2.169299in}}%
\pgfusepath{stroke}%
\end{pgfscope}%
\begin{pgfscope}%
\pgfsetbuttcap%
\pgfsetroundjoin%
\definecolor{currentfill}{rgb}{0.000000,0.000000,0.000000}%
\pgfsetfillcolor{currentfill}%
\pgfsetlinewidth{0.501875pt}%
\definecolor{currentstroke}{rgb}{0.000000,0.000000,0.000000}%
\pgfsetstrokecolor{currentstroke}%
\pgfsetdash{}{0pt}%
\pgfsys@defobject{currentmarker}{\pgfqpoint{0.000000in}{0.000000in}}{\pgfqpoint{0.000000in}{0.055556in}}{%
\pgfpathmoveto{\pgfqpoint{0.000000in}{0.000000in}}%
\pgfpathlineto{\pgfqpoint{0.000000in}{0.055556in}}%
\pgfusepath{stroke,fill}%
}%
\begin{pgfscope}%
\pgfsys@transformshift{0.487500in}{0.301292in}%
\pgfsys@useobject{currentmarker}{}%
\end{pgfscope}%
\end{pgfscope}%
\begin{pgfscope}%
\pgfsetbuttcap%
\pgfsetroundjoin%
\definecolor{currentfill}{rgb}{0.000000,0.000000,0.000000}%
\pgfsetfillcolor{currentfill}%
\pgfsetlinewidth{0.501875pt}%
\definecolor{currentstroke}{rgb}{0.000000,0.000000,0.000000}%
\pgfsetstrokecolor{currentstroke}%
\pgfsetdash{}{0pt}%
\pgfsys@defobject{currentmarker}{\pgfqpoint{0.000000in}{-0.055556in}}{\pgfqpoint{0.000000in}{0.000000in}}{%
\pgfpathmoveto{\pgfqpoint{0.000000in}{0.000000in}}%
\pgfpathlineto{\pgfqpoint{0.000000in}{-0.055556in}}%
\pgfusepath{stroke,fill}%
}%
\begin{pgfscope}%
\pgfsys@transformshift{0.487500in}{2.169299in}%
\pgfsys@useobject{currentmarker}{}%
\end{pgfscope}%
\end{pgfscope}%
\begin{pgfscope}%
\pgftext[x=0.487500in,y=0.245736in,,top]{\rmfamily\fontsize{8.000000}{9.600000}\selectfont \(\displaystyle 0.0\)}%
\end{pgfscope}%
\begin{pgfscope}%
\pgfsetbuttcap%
\pgfsetroundjoin%
\definecolor{currentfill}{rgb}{0.000000,0.000000,0.000000}%
\pgfsetfillcolor{currentfill}%
\pgfsetlinewidth{0.501875pt}%
\definecolor{currentstroke}{rgb}{0.000000,0.000000,0.000000}%
\pgfsetstrokecolor{currentstroke}%
\pgfsetdash{}{0pt}%
\pgfsys@defobject{currentmarker}{\pgfqpoint{0.000000in}{0.000000in}}{\pgfqpoint{0.000000in}{0.055556in}}{%
\pgfpathmoveto{\pgfqpoint{0.000000in}{0.000000in}}%
\pgfpathlineto{\pgfqpoint{0.000000in}{0.055556in}}%
\pgfusepath{stroke,fill}%
}%
\begin{pgfscope}%
\pgfsys@transformshift{1.092000in}{0.301292in}%
\pgfsys@useobject{currentmarker}{}%
\end{pgfscope}%
\end{pgfscope}%
\begin{pgfscope}%
\pgfsetbuttcap%
\pgfsetroundjoin%
\definecolor{currentfill}{rgb}{0.000000,0.000000,0.000000}%
\pgfsetfillcolor{currentfill}%
\pgfsetlinewidth{0.501875pt}%
\definecolor{currentstroke}{rgb}{0.000000,0.000000,0.000000}%
\pgfsetstrokecolor{currentstroke}%
\pgfsetdash{}{0pt}%
\pgfsys@defobject{currentmarker}{\pgfqpoint{0.000000in}{-0.055556in}}{\pgfqpoint{0.000000in}{0.000000in}}{%
\pgfpathmoveto{\pgfqpoint{0.000000in}{0.000000in}}%
\pgfpathlineto{\pgfqpoint{0.000000in}{-0.055556in}}%
\pgfusepath{stroke,fill}%
}%
\begin{pgfscope}%
\pgfsys@transformshift{1.092000in}{2.169299in}%
\pgfsys@useobject{currentmarker}{}%
\end{pgfscope}%
\end{pgfscope}%
\begin{pgfscope}%
\pgftext[x=1.092000in,y=0.245736in,,top]{\rmfamily\fontsize{8.000000}{9.600000}\selectfont \(\displaystyle 0.2\)}%
\end{pgfscope}%
\begin{pgfscope}%
\pgfsetbuttcap%
\pgfsetroundjoin%
\definecolor{currentfill}{rgb}{0.000000,0.000000,0.000000}%
\pgfsetfillcolor{currentfill}%
\pgfsetlinewidth{0.501875pt}%
\definecolor{currentstroke}{rgb}{0.000000,0.000000,0.000000}%
\pgfsetstrokecolor{currentstroke}%
\pgfsetdash{}{0pt}%
\pgfsys@defobject{currentmarker}{\pgfqpoint{0.000000in}{0.000000in}}{\pgfqpoint{0.000000in}{0.055556in}}{%
\pgfpathmoveto{\pgfqpoint{0.000000in}{0.000000in}}%
\pgfpathlineto{\pgfqpoint{0.000000in}{0.055556in}}%
\pgfusepath{stroke,fill}%
}%
\begin{pgfscope}%
\pgfsys@transformshift{1.696500in}{0.301292in}%
\pgfsys@useobject{currentmarker}{}%
\end{pgfscope}%
\end{pgfscope}%
\begin{pgfscope}%
\pgfsetbuttcap%
\pgfsetroundjoin%
\definecolor{currentfill}{rgb}{0.000000,0.000000,0.000000}%
\pgfsetfillcolor{currentfill}%
\pgfsetlinewidth{0.501875pt}%
\definecolor{currentstroke}{rgb}{0.000000,0.000000,0.000000}%
\pgfsetstrokecolor{currentstroke}%
\pgfsetdash{}{0pt}%
\pgfsys@defobject{currentmarker}{\pgfqpoint{0.000000in}{-0.055556in}}{\pgfqpoint{0.000000in}{0.000000in}}{%
\pgfpathmoveto{\pgfqpoint{0.000000in}{0.000000in}}%
\pgfpathlineto{\pgfqpoint{0.000000in}{-0.055556in}}%
\pgfusepath{stroke,fill}%
}%
\begin{pgfscope}%
\pgfsys@transformshift{1.696500in}{2.169299in}%
\pgfsys@useobject{currentmarker}{}%
\end{pgfscope}%
\end{pgfscope}%
\begin{pgfscope}%
\pgftext[x=1.696500in,y=0.245736in,,top]{\rmfamily\fontsize{8.000000}{9.600000}\selectfont \(\displaystyle 0.4\)}%
\end{pgfscope}%
\begin{pgfscope}%
\pgfsetbuttcap%
\pgfsetroundjoin%
\definecolor{currentfill}{rgb}{0.000000,0.000000,0.000000}%
\pgfsetfillcolor{currentfill}%
\pgfsetlinewidth{0.501875pt}%
\definecolor{currentstroke}{rgb}{0.000000,0.000000,0.000000}%
\pgfsetstrokecolor{currentstroke}%
\pgfsetdash{}{0pt}%
\pgfsys@defobject{currentmarker}{\pgfqpoint{0.000000in}{0.000000in}}{\pgfqpoint{0.000000in}{0.055556in}}{%
\pgfpathmoveto{\pgfqpoint{0.000000in}{0.000000in}}%
\pgfpathlineto{\pgfqpoint{0.000000in}{0.055556in}}%
\pgfusepath{stroke,fill}%
}%
\begin{pgfscope}%
\pgfsys@transformshift{2.301000in}{0.301292in}%
\pgfsys@useobject{currentmarker}{}%
\end{pgfscope}%
\end{pgfscope}%
\begin{pgfscope}%
\pgfsetbuttcap%
\pgfsetroundjoin%
\definecolor{currentfill}{rgb}{0.000000,0.000000,0.000000}%
\pgfsetfillcolor{currentfill}%
\pgfsetlinewidth{0.501875pt}%
\definecolor{currentstroke}{rgb}{0.000000,0.000000,0.000000}%
\pgfsetstrokecolor{currentstroke}%
\pgfsetdash{}{0pt}%
\pgfsys@defobject{currentmarker}{\pgfqpoint{0.000000in}{-0.055556in}}{\pgfqpoint{0.000000in}{0.000000in}}{%
\pgfpathmoveto{\pgfqpoint{0.000000in}{0.000000in}}%
\pgfpathlineto{\pgfqpoint{0.000000in}{-0.055556in}}%
\pgfusepath{stroke,fill}%
}%
\begin{pgfscope}%
\pgfsys@transformshift{2.301000in}{2.169299in}%
\pgfsys@useobject{currentmarker}{}%
\end{pgfscope}%
\end{pgfscope}%
\begin{pgfscope}%
\pgftext[x=2.301000in,y=0.245736in,,top]{\rmfamily\fontsize{8.000000}{9.600000}\selectfont \(\displaystyle 0.6\)}%
\end{pgfscope}%
\begin{pgfscope}%
\pgfsetbuttcap%
\pgfsetroundjoin%
\definecolor{currentfill}{rgb}{0.000000,0.000000,0.000000}%
\pgfsetfillcolor{currentfill}%
\pgfsetlinewidth{0.501875pt}%
\definecolor{currentstroke}{rgb}{0.000000,0.000000,0.000000}%
\pgfsetstrokecolor{currentstroke}%
\pgfsetdash{}{0pt}%
\pgfsys@defobject{currentmarker}{\pgfqpoint{0.000000in}{0.000000in}}{\pgfqpoint{0.000000in}{0.055556in}}{%
\pgfpathmoveto{\pgfqpoint{0.000000in}{0.000000in}}%
\pgfpathlineto{\pgfqpoint{0.000000in}{0.055556in}}%
\pgfusepath{stroke,fill}%
}%
\begin{pgfscope}%
\pgfsys@transformshift{2.905500in}{0.301292in}%
\pgfsys@useobject{currentmarker}{}%
\end{pgfscope}%
\end{pgfscope}%
\begin{pgfscope}%
\pgfsetbuttcap%
\pgfsetroundjoin%
\definecolor{currentfill}{rgb}{0.000000,0.000000,0.000000}%
\pgfsetfillcolor{currentfill}%
\pgfsetlinewidth{0.501875pt}%
\definecolor{currentstroke}{rgb}{0.000000,0.000000,0.000000}%
\pgfsetstrokecolor{currentstroke}%
\pgfsetdash{}{0pt}%
\pgfsys@defobject{currentmarker}{\pgfqpoint{0.000000in}{-0.055556in}}{\pgfqpoint{0.000000in}{0.000000in}}{%
\pgfpathmoveto{\pgfqpoint{0.000000in}{0.000000in}}%
\pgfpathlineto{\pgfqpoint{0.000000in}{-0.055556in}}%
\pgfusepath{stroke,fill}%
}%
\begin{pgfscope}%
\pgfsys@transformshift{2.905500in}{2.169299in}%
\pgfsys@useobject{currentmarker}{}%
\end{pgfscope}%
\end{pgfscope}%
\begin{pgfscope}%
\pgftext[x=2.905500in,y=0.245736in,,top]{\rmfamily\fontsize{8.000000}{9.600000}\selectfont \(\displaystyle 0.8\)}%
\end{pgfscope}%
\begin{pgfscope}%
\pgfsetbuttcap%
\pgfsetroundjoin%
\definecolor{currentfill}{rgb}{0.000000,0.000000,0.000000}%
\pgfsetfillcolor{currentfill}%
\pgfsetlinewidth{0.501875pt}%
\definecolor{currentstroke}{rgb}{0.000000,0.000000,0.000000}%
\pgfsetstrokecolor{currentstroke}%
\pgfsetdash{}{0pt}%
\pgfsys@defobject{currentmarker}{\pgfqpoint{0.000000in}{0.000000in}}{\pgfqpoint{0.000000in}{0.055556in}}{%
\pgfpathmoveto{\pgfqpoint{0.000000in}{0.000000in}}%
\pgfpathlineto{\pgfqpoint{0.000000in}{0.055556in}}%
\pgfusepath{stroke,fill}%
}%
\begin{pgfscope}%
\pgfsys@transformshift{3.510000in}{0.301292in}%
\pgfsys@useobject{currentmarker}{}%
\end{pgfscope}%
\end{pgfscope}%
\begin{pgfscope}%
\pgfsetbuttcap%
\pgfsetroundjoin%
\definecolor{currentfill}{rgb}{0.000000,0.000000,0.000000}%
\pgfsetfillcolor{currentfill}%
\pgfsetlinewidth{0.501875pt}%
\definecolor{currentstroke}{rgb}{0.000000,0.000000,0.000000}%
\pgfsetstrokecolor{currentstroke}%
\pgfsetdash{}{0pt}%
\pgfsys@defobject{currentmarker}{\pgfqpoint{0.000000in}{-0.055556in}}{\pgfqpoint{0.000000in}{0.000000in}}{%
\pgfpathmoveto{\pgfqpoint{0.000000in}{0.000000in}}%
\pgfpathlineto{\pgfqpoint{0.000000in}{-0.055556in}}%
\pgfusepath{stroke,fill}%
}%
\begin{pgfscope}%
\pgfsys@transformshift{3.510000in}{2.169299in}%
\pgfsys@useobject{currentmarker}{}%
\end{pgfscope}%
\end{pgfscope}%
\begin{pgfscope}%
\pgftext[x=3.510000in,y=0.245736in,,top]{\rmfamily\fontsize{8.000000}{9.600000}\selectfont \(\displaystyle 1.0\)}%
\end{pgfscope}%
\begin{pgfscope}%
\pgftext[x=1.998750in,y=0.078167in,,top]{\rmfamily\fontsize{10.000000}{12.000000}\selectfont \(\displaystyle x\)}%
\end{pgfscope}%
\begin{pgfscope}%
\pgfsetbuttcap%
\pgfsetroundjoin%
\definecolor{currentfill}{rgb}{0.000000,0.000000,0.000000}%
\pgfsetfillcolor{currentfill}%
\pgfsetlinewidth{0.501875pt}%
\definecolor{currentstroke}{rgb}{0.000000,0.000000,0.000000}%
\pgfsetstrokecolor{currentstroke}%
\pgfsetdash{}{0pt}%
\pgfsys@defobject{currentmarker}{\pgfqpoint{0.000000in}{0.000000in}}{\pgfqpoint{0.055556in}{0.000000in}}{%
\pgfpathmoveto{\pgfqpoint{0.000000in}{0.000000in}}%
\pgfpathlineto{\pgfqpoint{0.055556in}{0.000000in}}%
\pgfusepath{stroke,fill}%
}%
\begin{pgfscope}%
\pgfsys@transformshift{0.487500in}{0.301292in}%
\pgfsys@useobject{currentmarker}{}%
\end{pgfscope}%
\end{pgfscope}%
\begin{pgfscope}%
\pgfsetbuttcap%
\pgfsetroundjoin%
\definecolor{currentfill}{rgb}{0.000000,0.000000,0.000000}%
\pgfsetfillcolor{currentfill}%
\pgfsetlinewidth{0.501875pt}%
\definecolor{currentstroke}{rgb}{0.000000,0.000000,0.000000}%
\pgfsetstrokecolor{currentstroke}%
\pgfsetdash{}{0pt}%
\pgfsys@defobject{currentmarker}{\pgfqpoint{-0.055556in}{0.000000in}}{\pgfqpoint{0.000000in}{0.000000in}}{%
\pgfpathmoveto{\pgfqpoint{0.000000in}{0.000000in}}%
\pgfpathlineto{\pgfqpoint{-0.055556in}{0.000000in}}%
\pgfusepath{stroke,fill}%
}%
\begin{pgfscope}%
\pgfsys@transformshift{3.510000in}{0.301292in}%
\pgfsys@useobject{currentmarker}{}%
\end{pgfscope}%
\end{pgfscope}%
\begin{pgfscope}%
\pgftext[x=0.431944in,y=0.301292in,right,]{\rmfamily\fontsize{8.000000}{9.600000}\selectfont \(\displaystyle 0.0\)}%
\end{pgfscope}%
\begin{pgfscope}%
\pgfsetbuttcap%
\pgfsetroundjoin%
\definecolor{currentfill}{rgb}{0.000000,0.000000,0.000000}%
\pgfsetfillcolor{currentfill}%
\pgfsetlinewidth{0.501875pt}%
\definecolor{currentstroke}{rgb}{0.000000,0.000000,0.000000}%
\pgfsetstrokecolor{currentstroke}%
\pgfsetdash{}{0pt}%
\pgfsys@defobject{currentmarker}{\pgfqpoint{0.000000in}{0.000000in}}{\pgfqpoint{0.055556in}{0.000000in}}{%
\pgfpathmoveto{\pgfqpoint{0.000000in}{0.000000in}}%
\pgfpathlineto{\pgfqpoint{0.055556in}{0.000000in}}%
\pgfusepath{stroke,fill}%
}%
\begin{pgfscope}%
\pgfsys@transformshift{0.487500in}{0.674893in}%
\pgfsys@useobject{currentmarker}{}%
\end{pgfscope}%
\end{pgfscope}%
\begin{pgfscope}%
\pgfsetbuttcap%
\pgfsetroundjoin%
\definecolor{currentfill}{rgb}{0.000000,0.000000,0.000000}%
\pgfsetfillcolor{currentfill}%
\pgfsetlinewidth{0.501875pt}%
\definecolor{currentstroke}{rgb}{0.000000,0.000000,0.000000}%
\pgfsetstrokecolor{currentstroke}%
\pgfsetdash{}{0pt}%
\pgfsys@defobject{currentmarker}{\pgfqpoint{-0.055556in}{0.000000in}}{\pgfqpoint{0.000000in}{0.000000in}}{%
\pgfpathmoveto{\pgfqpoint{0.000000in}{0.000000in}}%
\pgfpathlineto{\pgfqpoint{-0.055556in}{0.000000in}}%
\pgfusepath{stroke,fill}%
}%
\begin{pgfscope}%
\pgfsys@transformshift{3.510000in}{0.674893in}%
\pgfsys@useobject{currentmarker}{}%
\end{pgfscope}%
\end{pgfscope}%
\begin{pgfscope}%
\pgftext[x=0.431944in,y=0.674893in,right,]{\rmfamily\fontsize{8.000000}{9.600000}\selectfont \(\displaystyle 0.2\)}%
\end{pgfscope}%
\begin{pgfscope}%
\pgfsetbuttcap%
\pgfsetroundjoin%
\definecolor{currentfill}{rgb}{0.000000,0.000000,0.000000}%
\pgfsetfillcolor{currentfill}%
\pgfsetlinewidth{0.501875pt}%
\definecolor{currentstroke}{rgb}{0.000000,0.000000,0.000000}%
\pgfsetstrokecolor{currentstroke}%
\pgfsetdash{}{0pt}%
\pgfsys@defobject{currentmarker}{\pgfqpoint{0.000000in}{0.000000in}}{\pgfqpoint{0.055556in}{0.000000in}}{%
\pgfpathmoveto{\pgfqpoint{0.000000in}{0.000000in}}%
\pgfpathlineto{\pgfqpoint{0.055556in}{0.000000in}}%
\pgfusepath{stroke,fill}%
}%
\begin{pgfscope}%
\pgfsys@transformshift{0.487500in}{1.048495in}%
\pgfsys@useobject{currentmarker}{}%
\end{pgfscope}%
\end{pgfscope}%
\begin{pgfscope}%
\pgfsetbuttcap%
\pgfsetroundjoin%
\definecolor{currentfill}{rgb}{0.000000,0.000000,0.000000}%
\pgfsetfillcolor{currentfill}%
\pgfsetlinewidth{0.501875pt}%
\definecolor{currentstroke}{rgb}{0.000000,0.000000,0.000000}%
\pgfsetstrokecolor{currentstroke}%
\pgfsetdash{}{0pt}%
\pgfsys@defobject{currentmarker}{\pgfqpoint{-0.055556in}{0.000000in}}{\pgfqpoint{0.000000in}{0.000000in}}{%
\pgfpathmoveto{\pgfqpoint{0.000000in}{0.000000in}}%
\pgfpathlineto{\pgfqpoint{-0.055556in}{0.000000in}}%
\pgfusepath{stroke,fill}%
}%
\begin{pgfscope}%
\pgfsys@transformshift{3.510000in}{1.048495in}%
\pgfsys@useobject{currentmarker}{}%
\end{pgfscope}%
\end{pgfscope}%
\begin{pgfscope}%
\pgftext[x=0.431944in,y=1.048495in,right,]{\rmfamily\fontsize{8.000000}{9.600000}\selectfont \(\displaystyle 0.4\)}%
\end{pgfscope}%
\begin{pgfscope}%
\pgfsetbuttcap%
\pgfsetroundjoin%
\definecolor{currentfill}{rgb}{0.000000,0.000000,0.000000}%
\pgfsetfillcolor{currentfill}%
\pgfsetlinewidth{0.501875pt}%
\definecolor{currentstroke}{rgb}{0.000000,0.000000,0.000000}%
\pgfsetstrokecolor{currentstroke}%
\pgfsetdash{}{0pt}%
\pgfsys@defobject{currentmarker}{\pgfqpoint{0.000000in}{0.000000in}}{\pgfqpoint{0.055556in}{0.000000in}}{%
\pgfpathmoveto{\pgfqpoint{0.000000in}{0.000000in}}%
\pgfpathlineto{\pgfqpoint{0.055556in}{0.000000in}}%
\pgfusepath{stroke,fill}%
}%
\begin{pgfscope}%
\pgfsys@transformshift{0.487500in}{1.422096in}%
\pgfsys@useobject{currentmarker}{}%
\end{pgfscope}%
\end{pgfscope}%
\begin{pgfscope}%
\pgfsetbuttcap%
\pgfsetroundjoin%
\definecolor{currentfill}{rgb}{0.000000,0.000000,0.000000}%
\pgfsetfillcolor{currentfill}%
\pgfsetlinewidth{0.501875pt}%
\definecolor{currentstroke}{rgb}{0.000000,0.000000,0.000000}%
\pgfsetstrokecolor{currentstroke}%
\pgfsetdash{}{0pt}%
\pgfsys@defobject{currentmarker}{\pgfqpoint{-0.055556in}{0.000000in}}{\pgfqpoint{0.000000in}{0.000000in}}{%
\pgfpathmoveto{\pgfqpoint{0.000000in}{0.000000in}}%
\pgfpathlineto{\pgfqpoint{-0.055556in}{0.000000in}}%
\pgfusepath{stroke,fill}%
}%
\begin{pgfscope}%
\pgfsys@transformshift{3.510000in}{1.422096in}%
\pgfsys@useobject{currentmarker}{}%
\end{pgfscope}%
\end{pgfscope}%
\begin{pgfscope}%
\pgftext[x=0.431944in,y=1.422096in,right,]{\rmfamily\fontsize{8.000000}{9.600000}\selectfont \(\displaystyle 0.6\)}%
\end{pgfscope}%
\begin{pgfscope}%
\pgfsetbuttcap%
\pgfsetroundjoin%
\definecolor{currentfill}{rgb}{0.000000,0.000000,0.000000}%
\pgfsetfillcolor{currentfill}%
\pgfsetlinewidth{0.501875pt}%
\definecolor{currentstroke}{rgb}{0.000000,0.000000,0.000000}%
\pgfsetstrokecolor{currentstroke}%
\pgfsetdash{}{0pt}%
\pgfsys@defobject{currentmarker}{\pgfqpoint{0.000000in}{0.000000in}}{\pgfqpoint{0.055556in}{0.000000in}}{%
\pgfpathmoveto{\pgfqpoint{0.000000in}{0.000000in}}%
\pgfpathlineto{\pgfqpoint{0.055556in}{0.000000in}}%
\pgfusepath{stroke,fill}%
}%
\begin{pgfscope}%
\pgfsys@transformshift{0.487500in}{1.795698in}%
\pgfsys@useobject{currentmarker}{}%
\end{pgfscope}%
\end{pgfscope}%
\begin{pgfscope}%
\pgfsetbuttcap%
\pgfsetroundjoin%
\definecolor{currentfill}{rgb}{0.000000,0.000000,0.000000}%
\pgfsetfillcolor{currentfill}%
\pgfsetlinewidth{0.501875pt}%
\definecolor{currentstroke}{rgb}{0.000000,0.000000,0.000000}%
\pgfsetstrokecolor{currentstroke}%
\pgfsetdash{}{0pt}%
\pgfsys@defobject{currentmarker}{\pgfqpoint{-0.055556in}{0.000000in}}{\pgfqpoint{0.000000in}{0.000000in}}{%
\pgfpathmoveto{\pgfqpoint{0.000000in}{0.000000in}}%
\pgfpathlineto{\pgfqpoint{-0.055556in}{0.000000in}}%
\pgfusepath{stroke,fill}%
}%
\begin{pgfscope}%
\pgfsys@transformshift{3.510000in}{1.795698in}%
\pgfsys@useobject{currentmarker}{}%
\end{pgfscope}%
\end{pgfscope}%
\begin{pgfscope}%
\pgftext[x=0.431944in,y=1.795698in,right,]{\rmfamily\fontsize{8.000000}{9.600000}\selectfont \(\displaystyle 0.8\)}%
\end{pgfscope}%
\begin{pgfscope}%
\pgfsetbuttcap%
\pgfsetroundjoin%
\definecolor{currentfill}{rgb}{0.000000,0.000000,0.000000}%
\pgfsetfillcolor{currentfill}%
\pgfsetlinewidth{0.501875pt}%
\definecolor{currentstroke}{rgb}{0.000000,0.000000,0.000000}%
\pgfsetstrokecolor{currentstroke}%
\pgfsetdash{}{0pt}%
\pgfsys@defobject{currentmarker}{\pgfqpoint{0.000000in}{0.000000in}}{\pgfqpoint{0.055556in}{0.000000in}}{%
\pgfpathmoveto{\pgfqpoint{0.000000in}{0.000000in}}%
\pgfpathlineto{\pgfqpoint{0.055556in}{0.000000in}}%
\pgfusepath{stroke,fill}%
}%
\begin{pgfscope}%
\pgfsys@transformshift{0.487500in}{2.169299in}%
\pgfsys@useobject{currentmarker}{}%
\end{pgfscope}%
\end{pgfscope}%
\begin{pgfscope}%
\pgfsetbuttcap%
\pgfsetroundjoin%
\definecolor{currentfill}{rgb}{0.000000,0.000000,0.000000}%
\pgfsetfillcolor{currentfill}%
\pgfsetlinewidth{0.501875pt}%
\definecolor{currentstroke}{rgb}{0.000000,0.000000,0.000000}%
\pgfsetstrokecolor{currentstroke}%
\pgfsetdash{}{0pt}%
\pgfsys@defobject{currentmarker}{\pgfqpoint{-0.055556in}{0.000000in}}{\pgfqpoint{0.000000in}{0.000000in}}{%
\pgfpathmoveto{\pgfqpoint{0.000000in}{0.000000in}}%
\pgfpathlineto{\pgfqpoint{-0.055556in}{0.000000in}}%
\pgfusepath{stroke,fill}%
}%
\begin{pgfscope}%
\pgfsys@transformshift{3.510000in}{2.169299in}%
\pgfsys@useobject{currentmarker}{}%
\end{pgfscope}%
\end{pgfscope}%
\begin{pgfscope}%
\pgftext[x=0.431944in,y=2.169299in,right,]{\rmfamily\fontsize{8.000000}{9.600000}\selectfont \(\displaystyle 1.0\)}%
\end{pgfscope}%
\begin{pgfscope}%
\pgfsetbuttcap%
\pgfsetmiterjoin%
\definecolor{currentfill}{rgb}{1.000000,1.000000,1.000000}%
\pgfsetfillcolor{currentfill}%
\pgfsetlinewidth{1.003750pt}%
\definecolor{currentstroke}{rgb}{0.000000,0.000000,0.000000}%
\pgfsetstrokecolor{currentstroke}%
\pgfsetdash{}{0pt}%
\pgfpathmoveto{\pgfqpoint{0.543056in}{1.758811in}}%
\pgfpathlineto{\pgfqpoint{1.196319in}{1.758811in}}%
\pgfpathlineto{\pgfqpoint{1.196319in}{2.113744in}}%
\pgfpathlineto{\pgfqpoint{0.543056in}{2.113744in}}%
\pgfpathclose%
\pgfusepath{stroke,fill}%
\end{pgfscope}%
\begin{pgfscope}%
\pgfsetbuttcap%
\pgfsetroundjoin%
\pgfsetlinewidth{1.003750pt}%
\definecolor{currentstroke}{rgb}{1.000000,0.000000,0.000000}%
\pgfsetstrokecolor{currentstroke}%
\pgfsetdash{{6.000000pt}{6.000000pt}}{0.000000pt}%
\pgfpathmoveto{\pgfqpoint{0.620833in}{2.030410in}}%
\pgfpathlineto{\pgfqpoint{0.776389in}{2.030410in}}%
\pgfusepath{stroke}%
\end{pgfscope}%
\begin{pgfscope}%
\pgftext[x=0.898611in,y=1.991522in,left,base]{\rmfamily\fontsize{8.000000}{9.600000}\selectfont \(\displaystyle x\)}%
\end{pgfscope}%
\begin{pgfscope}%
\pgfsetrectcap%
\pgfsetroundjoin%
\pgfsetlinewidth{1.003750pt}%
\definecolor{currentstroke}{rgb}{0.000000,0.000000,1.000000}%
\pgfsetstrokecolor{currentstroke}%
\pgfsetdash{}{0pt}%
\pgfpathmoveto{\pgfqpoint{0.620833in}{1.869922in}}%
\pgfpathlineto{\pgfqpoint{0.776389in}{1.869922in}}%
\pgfusepath{stroke}%
\end{pgfscope}%
\begin{pgfscope}%
\pgftext[x=0.898611in,y=1.831033in,left,base]{\rmfamily\fontsize{8.000000}{9.600000}\selectfont \(\displaystyle V(x)\)}%
\end{pgfscope}%
\end{pgfpicture}%
\makeatother%
\endgroup%
 
		\caption{Visualization of Lemma~\ref{cor:equivalent_event}'s proof for a instance of the problem with a Beta prior corresponding to the pair $y = (4,5)$, a discount factor of $\gamma=0.95$ and $K = 2$. The intersection of the two lines represents the Optimistic Gittins index.}
		\label{fig:visaulize_gx_proof}
	\end{figure}
	From these observations, and the fact that $y$ and $\gamma$ were arbitrary, the result follows. Figure~\ref{fig:visaulize_gx_proof} provides a visualization of this proof.
\end{myproof}


\subsection{Proof of Lemma~\ref{lemma:approx_bound}} \label{prf:approx_bound}
\begin{myproof}[Proof.]
	Let $K < M$ be two look-ahead parameters used in the definition of OGI. We will show that $V^K_\gamma(y, \lambda) \le V^M_\gamma(y, \lambda)$ where we recall the definitions of these functions from the beginning of Section~\ref{sec:appendix_properties_of_ogi}.
	
	We begin with a fundamental step. Let $\tau_1$ and $\tau_2$ be any predictable stopping times (i.e. $\mathcal F_{t-1}$-measurable random times) such that $\tau_1$ precedes $\tau_2$ almost surely, that is $\tau_1 < \tau_2$. Recall that the expected reward of the $i$th arm satisfies $\Ee{X_{i,t} \given \theta_i} = \mu(\theta_i)$ for all $t$. Let $\hat \theta_i \in \Theta$ denote a realization of the random variable $\theta_i$ and let $\zeta(\hat \theta_i)$ be a real-valued, measurable function of $\hat \theta_i$. In this case, we have that
	\begin{align*}
	\Ee{\sum_{t=\tau_1+1}^{\tau_2} \gamma^{t-1} X_{i,t} + \gamma^{\tau_2}\frac{\zeta(\hat \theta_i)}{1 - \gamma} \given[\Bigg] \theta_i = \hat \theta_i} & =\mu(\hat \theta_i) \Ee{ \sum_{t=\tau_1+1}^{\tau_2} \gamma^{t-1}  \given[\Bigg] \theta_i = \hat \theta_i} + \Ee{\frac{\gamma^{\tau_2}}{1-\gamma}\given[\Bigg] \theta_i = \hat \theta_i}\zeta(\hat \theta_i) \\
	%& =\mu(\hat \theta_i) \Ee{ \sum_{t=\tau_1+1}^{\tau_2} \gamma^{t-1} \given[\Bigg] \theta_i = \hat \theta_i } + \Ee{\sum_{t=\tau_2+1}^{\infty} \gamma^{t-1} \given[\Bigg] \theta_i = \hat \theta_i}\frac{\zeta(\hat \theta_i)}{1 - \gamma} \\
	& \le \Ee{\gamma^{\tau_1} \given \theta_i = \hat \theta_i}  \frac{\max(\zeta(\hat \theta_i), \mu(\hat \theta_i))}{1-\gamma}.
	\end{align*}
	Thus we conclude, because $\hat \theta_i$ was arbitrary, that almost surely,
	\begin{equation} \label{ineq:fundamntal_bound_for_lemma_2}
	\Ee{\sum_{t=\tau_1+1}^{\tau_2} \gamma^{t-1} X_{i,t} + \gamma^{\tau_2}\frac{\zeta(\hat \theta_i)}{1 - \gamma} \given[\Bigg] \theta_i}  \le \Ee{\gamma^{\tau_1} \given \theta_i }  \frac{\max(\zeta(\hat \theta_i), \mu(\theta_i))}{1-\gamma}.
	\end{equation}
	Let $\tau^\star$ be a stopping time that achieves the supremum in  $V_\gamma^M(y, \lambda)$ and define the predictable stopping time $\tau^\star_K \defeq K \wedge \tau^\star$. Consider the (conditional) cumulative rewards in the definition of $V^M_\gamma(y)$, from time $\tau^\star_K+1$ onwards, given the sufficient statistic observed at time $\tau_K^\star$. That is, 
	\[
		\E\left[\sum_{t=\tau_K^\star+1}^{\tau^\star}  \gamma^{t-1} X_{i,t} + \gamma^{\tau^\star} R_{\lambda,M}(\tau^\star, y_{i,\tau^\star-1})/(1-\gamma)
	\given[\Bigg] y_{i,\tau_K^\star-1} \right].
	\]
	We upper bound this random variable as follows. First, we note that, at any time $s$ and for any statistic $\hat y \in \mathcal{Y}$, the following statement holds
	\begin{equation}\label{eqn:dist_equal_theta_i}
	\P{R(\hat y) \le r} = \P{\mu(\theta_i) \le r \given y_{i,s} = \hat y}, \qquad \forall r \in \Re
	\end{equation}
	meaning that the posterior distribution of the arm's expected reward $R(y_{i,s})$ is the same as $\mu(\theta_i)$ \emph{conditioned} on having observed statistic $\hat y$ about the arm. This holds by definition of the random variable $R(y)$. Because of this observation, we have that the following inequality  holds almost surely,
	\begin{align*}
		& \gamma^{\tau^\star} \frac{R_{\lambda,M}(\tau^\star, y_{i,\tau^\star-1})}{1-\gamma}
		 \nonumber \\
		&\quad =  \gamma^{\tau^\star}\left( \ind{\tau^\star =M}\frac{\max(\lambda, R(y_{i,\tau^\star-1}))}{1-\gamma} + \ind{\tau^\star < M}\frac{\lambda}{1-\gamma}\right)
		 \nonumber\\
		&\quad= \ind{\tau^\star =M} \gamma^{M}\frac{\max(\lambda, R(y_{i,M}))}{1-\gamma} + \ind{\tau^\star < M} \gamma^{\tau^\star}\frac{\lambda}{1-\gamma}
	\nonumber\\
		& \quad\overset{(*)}{=} \ind{\tau^\star =M} \gamma^{M}\frac{\Ee{\max(\lambda, R(y_{i,M})) \given y_{i,M} }}{1-\gamma} + \ind{\tau^\star < M} \gamma^{\tau^\star}\frac{\lambda}{1-\gamma}
		\\
		& \quad\overset{(\dagger)}{=} \ind{\tau^\star =M} \gamma^{M}\frac{\Ee{\max(\lambda, \mu(\theta_i)) \given y_{i,M} }}{1-\gamma} + \ind{\tau^\star < M} \gamma^{\tau^\star}\frac{\lambda}{1-\gamma}
		 \\
		&\quad \overset{(**)}{\le} \frac{  \Ee{\gamma^{\tau^\star}\max(\lambda, \mu(\theta_i)) \given y_{i,\tau^\star-1} }}{1-\gamma}
	\end{align*}
	where $(*)$ and $(**)$ both use the fact that for any $t$, $\tau^\star \le t$ is measurable with respect to the $\sigma$-algebra generated by $y_{i,t-1}$, namely $\mathcal F_{t-1}$. Equation ($\dagger$) follows from \eqref{eqn:dist_equal_theta_i}. Therefore, immediately using the above inequality and conditioning on the event $\tau^\star > K$, we have that
	\begin{align} 
	&\E\left[\sum_{t=\tau_K^\star+1}^{\tau^\star} \gamma^{t-1} X_{i,t} + \gamma^{\tau^\star} \frac{R_{\lambda,M}(\tau^\star, y_{i,\tau^\star-1})}{1-\gamma}
	\given[\Bigg] \tau^\star > K, \;y_{i,\tau^\star_K-1} \right] \nonumber \\
	&\qquad \le  \E\left[\sum_{t=\tau_K^\star+1}^{\tau^\star} \gamma^{t-1} X_{i,t} +  \Ee{ \gamma^{\tau^\star}\frac{\max(\lambda,\mu(\theta_i)))}{1-\gamma}\given[\Bigg] y_{i,\tau^\star-1}}
	\given[\Bigg] \tau^\star > K, \;y_{i,\tau^\star_K-1} \right] \nonumber\\
	&\qquad = \E\left[\sum_{t=\tau_K^\star+1}^{\tau^\star} \gamma^{t-1} X_{i,t} + \gamma^{\tau^\star} \frac{\max(\lambda, \mu(\theta_i))}{1-\gamma}
	\given[\Bigg] \tau^\star > K, \;y_{i,\tau^\star_K-1} \right] \label{eqn:another_toer_prop_use}\\
	&\qquad = \E\left[ \Ee{\sum_{t=K+1}^{\tau^\star} \gamma^{t-1} X_{i,t} + \gamma^{\tau^\star}\frac{\max(\lambda,\mu(\theta_i))}{1 - \gamma} \given[\Bigg] \theta_i} 
	\given[\Bigg]\tau^\star > K, \; y_{i,\tau^\star_K-1} \right] \label{eqn:proof_lemma2_toer_prop} \\
	&\qquad \le \E\left[ \gamma^{\tau^\star_K} \frac{\max(\mu(\theta_i), \lambda)}{1 - \gamma}\given[\Bigg] \tau^\star > K, \; y_{i,\tau^\star_K-1}  \right]  \label{ineq:proof_lem_2_use_of_first_bound} \\
	&\qquad = \E\left[ \gamma^{\tau^\star_K} \frac{\max(R(y_{i,\tau^\star_K-1}), \lambda)}{1 - \gamma}\given[\Bigg] \tau^\star > K, \; y_{i,\tau^\star_K-1}  \right]  \label{ineq:obvious_step} \\
	&\qquad = \Ee{\frac{\gamma^{\tau^\star_K}R_{\lambda,K}(\tau^\star_K, y_{i,\tau^\star_K-1})}{1-\gamma} \given[\Bigg]  \tau^\star > K, \; y_{i,\tau^\star_K-1}} \label{eqn:use_of_def_of_R}
	\end{align}
	where \eqref{eqn:another_toer_prop_use}, \eqref{eqn:proof_lemma2_toer_prop} use the tower property and \eqref{ineq:proof_lem_2_use_of_first_bound} follows from the bound in \eqref{ineq:fundamntal_bound_for_lemma_2} because $\tau^\star_K < \tau^\star$, almost surely. Equation \eqref{ineq:obvious_step} follows from statement \eqref{eqn:dist_equal_theta_i} and that the event $\tau^\star > K$ is $\mathcal{F}_{K-1}$-measurable (we can decide whether to pull arm $i$ or retire based on information up to and including time $K-1$). Finally equation \eqref{eqn:use_of_def_of_R} is derived by substituting in the definition of $R_{\lambda,K}$ (as given in Section~\ref{sec:gittins_and_approx}) and noting that $\tau^\star_K = K$ under the above conditioning.
	
	We now condition on the complement of the previous event we considered, namely, $\tau^\star \le K$. Under that event, $\tau^\star$ occurred early enough before time $K+1$ and thus $\tau^\star_K = \tau^\star$. Therefore, it follows from this observation that
	\begin{align} 
	&\E\left[\sum_{t=\tau_K^\star+1}^{\tau^\star}\gamma^{t-1} X_{i,t} + \gamma^{\tau^\star} \frac{R_{\lambda,M}(\tau^\star, y_{i,\tau^\star-1})}{1-\gamma}
	\given[\Bigg] \tau^\star \le K, \;y_{i,\tau^\star_K-1} \right] \nonumber  \\
	&\qquad = \E\left[\gamma^{\tau^\star}  \frac{\lambda}{1-\gamma}
	\given[\Bigg]\tau^\star \le K, \; y_{i,\tau^\star_K-1} \right] \nonumber \\
	&\qquad \le \E\left[\gamma^{\tau^\star_K}  \frac{R_{\lambda,K}(\tau^\star_K, y_{i,\tau^\star_K-1})}{1-\gamma}
	\given[\Bigg]\tau^\star \le K, \; y_{i,\tau^\star_K-1} \right] \label{eqn:proof_lem_2_second_use_of_R_def}
	\end{align}
	where \eqref{eqn:proof_lem_2_second_use_of_R_def} is obtained by noting that $R_{\lambda,K}(\tau, y) \ge \lambda$ for any choice of $\tau, K$ and $y$. Thus, by the law of total expectation and \eqref{eqn:use_of_def_of_R}, \eqref{eqn:proof_lem_2_second_use_of_R_def}, we establish that
	\begin{equation} \label{ineq:proof_lem_2_main_bound_in_proof}
	\E\left[\sum_{t=\tau_K^\star+1}^{\tau^\star}\gamma^{t-1} X_{i,t} + \gamma^{\tau^\star} \frac{R_{\lambda,M}(\tau^\star, y_{i,\tau^\star-1})}{1-\gamma}
	\given[\Bigg] \;y_{i,\tau^\star_K-1} \right] \le \Ee{\gamma^{\tau^\star_K}  \frac{R_{\lambda,K}(\tau^\star_K, y_{i,\tau^\star_K-1})}{1-\gamma} \given[\Bigg] y_{i,\tau^\star_K-1}}.
	\end{equation}
	We are ready to complete our main argument in this proof by using the above bound and `breaking up' the $V_\gamma^M(y, \lambda)$ into rewards from times before $\tau_K^\star$ and after (and bounding the latter terms). More precisely, we obtain that
	\begin{align}
	V_\gamma^M(y,\lambda) & = \E_{y}\left[\sum_{t=1}^{\tau^\star}\gamma^{t-1} X_{i,t} + \gamma^{\tau^\star}\frac{R_{\lambda,M}(\tau^\star, y_{i,\tau^\star-1})}{1-\gamma}\right] \\
	& = \E_{y}\left[\sum_{t=1}^{\tau^\star_K}\gamma^{t-1} X_{i,t} + \sum_{t'=\tau^\star_K+1}^{\tau^\star}\gamma^{t'-1} X_{i,t'} +  \gamma^{\tau^\star}\frac{R_{\lambda,M}(\tau^\star, y_{i,\tau^\star-1})}{1-\gamma}\right] \nonumber \\
	& = \E_{y}\left[\sum_{t=1}^{\tau^\star_K}\gamma^{t-1} X_{i,t} + \Ee{\sum_{t'=\tau^\star_K+1}^{\tau^\star}\gamma^{t'-1} X_{i,t'} +  \gamma^{\tau^\star}\frac{R_{\lambda,M}(\tau^\star, y_{i,\tau^\star-1})}{1-\gamma} \given[\Bigg] y_{i,\tau^\star_K-1} }\right] \label{eqn:proof_lem_2_tower_prop_again} \\
	& \le  \E_{y}\left[\sum_{t=1}^{\tau^\star_K}\gamma^{t-1} X_{i,t} + \Ee{\gamma^{\tau^\star_K}  \frac{R_{\lambda,K}(\tau^\star_K, y_{i,\tau^\star_K-1})}{1-\gamma} \given[\Bigg] y_{i,\tau^\star_K-1}}\right] \label{eqn:proof_lem2_using_the_main_argument} \\
	& =  \E_{y}\left[\sum_{t=1}^{\tau^\star_K}\gamma^{t-1} X_{i,t} +  \gamma^{\tau^\star_K}  \frac{R_{\lambda,K}(\tau^\star_K, y_{i,\tau^\star_K-1})}{1-\gamma}\right] \label{eqn:proof_lem_2_tower_prop_yet_again} \\
	& \le \sup_{1 \le \tau \le K}  \E_{y}\left[\sum_{t=1}^{\tau}\gamma^{t-1} X_{i,t} +  \gamma^{\tau}  \frac{R_{\lambda,K}(\tau, y_{i,\tau-1})}{1-\gamma}\right] \nonumber \\
	& = V^K_{\gamma}(y, \lambda)
	\end{align}
	where Equations \eqref{eqn:proof_lem_2_tower_prop_again}, \eqref{eqn:proof_lem_2_tower_prop_yet_again} use the tower property and \eqref{eqn:proof_lem2_using_the_main_argument} is immediately derived by using the bound of \eqref{ineq:proof_lem_2_main_bound_in_proof}. Finally, we note that an almost identical proof can be given to show that $V^K_\gamma(y, \lambda) \ge V_\gamma(y, \lambda)$ where the lower bound is the continuation value used to compute the Gittins index.
	
	We have shown that for any $\lambda$ and $y$,  that $V^K_\gamma(y, \lambda)$ is non-increasing in $K$, and that $V_\gamma(y, \lambda)$ is a lower bound to this sequence. We make use of these facts to now prove that $v^K_\gamma(y)$ is also non-increasing in $K$. To this end, let us suppose for contradiction that there exist two integers $K_1 \le K_2$ and $v^{K_1}_\gamma(y) < v^{K_2}_\gamma(y)$. From Lemma~\ref{cor:equivalent_event} we know that
	\begin{equation}
		V^{K_2}_\gamma(y, v^K_\gamma(y)) > v^K_\gamma(y)/(1-\gamma) = V^K_\gamma(y, v^K_\gamma(y)),
	\end{equation}
	which contradicts the claim just shown. Therefore, $v^K_\gamma(y)$ must also be a  non-increasing sequence in $K$. The same argument can be used to further show that $v^K_\gamma(y) \ge v_\gamma(y)$.
	
	We now turn our attention to proving the convergence property stated in the Lemma. The first step will be to prove that for all $y \in \mathcal{Y}$ and $\lambda \in \Re_+$, that 
	\begin{equation} \label{eqn:proof_vk_bound_convergence_of_continuation_value}
	\lim_{K \to \infty}V^K_\gamma(y, \lambda) = V_\gamma(y, \lambda).
	\end{equation}
	Indeed, we upper bound the optimistic continuation value for a fixed parameter $M$ as follows:
	\begin{align}
		V^M_\gamma(y, \lambda) & = \sup_{1 \le \tau \le M} \E_y\left[\sum_{t=1}^{\tau} \gamma^{t-1} X_{i,t} + \frac{\gamma^{\tau}R_{\lambda,M}(\tau, y_{i,\tau-1})}{1-\gamma}\right] \nonumber\\
		& = \sup_{1 \le \tau \le M} \E_y\left[\sum_{t=1}^{\tau} \gamma^{t-1} X_{i,t} + \frac{\gamma^{\tau}\lambda}{1-\gamma} + \frac{\gamma^{\tau}R_{\lambda,M}(\tau, y_{i,\tau-1})}{1-\gamma} - \frac{\gamma^{\tau}\lambda}{1-\gamma}\right] \nonumber \\
		& \le \sup_{\tau \ge 1} \E_y\left[\sum_{t=1}^{\tau } \gamma^{t-1} X_{i,t} + \frac{\gamma^{\tau}\lambda}{1-\gamma}\right] + \sup_{1 \le \tau \le M}\E_y\left[\frac{\gamma^{\tau}R_{\lambda,M}(\tau, y_{i,\tau-1})}{1-\gamma} - \frac{\gamma^{\tau -1}\lambda}{1-\gamma}\right] \nonumber \\
		& =V_{\gamma}(y, \lambda) + \sup_{1 \le \tau \le M}\E_y\left[\frac{\gamma^{\tau}[R_{\lambda,M}(\tau, y_{i,\tau-1})-\lambda]}{1-\gamma}\right] \nonumber \\
		& \le V_{\gamma}(y, \lambda)  + \gamma^{M}\E_y\left[\frac{R_{\lambda,M}(M, y_{i,M-1})-\lambda}{1-\gamma}\right] \nonumber \\
		& = V_{\gamma}(y, \lambda)  + \gamma^{M}\E_y\left[\frac{( R(y_{i,M-1}) - \lambda)^+}{1-\gamma}\right] \nonumber \\
		& \le V_{\gamma}(y, \lambda)  + \gamma^{M}\E_y\left[\frac{|R(y_{i,M-1})|}{1-\gamma}\right] \nonumber  \\
		& = V_{\gamma}(y, \lambda)  + \gamma^M \E_y\left[\frac{|\mu(\theta_i)|}{1-\gamma}\right] \label{eqn:proof_lem_vk_bound_iter_exp},
	\end{align}
	where equation \eqref{eqn:proof_lem_vk_bound_iter_exp} follows from the definition of the random variable $R(.)$ and the law of iterated expectation. Now because $0 < \gamma < 1$ and $\E_y\left[|\mu(\theta_i)|\right] < \infty$, the right hand side above converges to $V_{\gamma}(y, \lambda)$. Finally, notice that $V^M_\gamma(y, \lambda) \ge V_\gamma(y, \lambda)$, and from this equation~\eqref{eqn:proof_vk_bound_convergence_of_continuation_value} follows. To finish the proof, we note that $V^K_\gamma(y, \lambda)$ is continuous in $\lambda$. Therefore, if we fix $\epsilon$, there is an integer $K = K(\epsilon)$ that is large enough so that
	\begin{align*}
	\left| v^K_\gamma(y) - v_\gamma(y) \right| & = \left| V^K_\gamma(y, v^K_\gamma(y)) -  V_\gamma(y, v_\gamma(y)) \right| \\
	 &  \le \left| V^K_\gamma(y, v_\gamma(y)) -  V_\gamma(y, v_\gamma(y)) + \eps\right| \\
	 &  \le \left| V^K_\gamma(y, v_\gamma(y)) -  V_\gamma(y, v_\gamma(y))\right| + \eps \\
	 & \le 2\epsilon.
	\end{align*}
	Then, we take the limit as $\epsilon \downarrow 0$ and the Lemma is shown.
\end{myproof}


The next Lemma will be the final property of the function $V^K_\gamma$ that we prove. This will subsequently be used in the proof of Lemma~\ref{lemma:underestimation}.
\begin{lemma} \label{lemma:vk_bound}
	Let $i$ be any arm. For any look-ahead parameter $K \in \mathbb{Z}_+$, discount factor $\gamma$ and any constant $\eta$, we have
	\begin{equation*}
	\E_y\left[V^K_\gamma(y_{i,1}, \eta) \right] \ge V^K_\gamma(y, \eta)
	\end{equation*}
	where we recall that $y_{i,1}$ is the summary statistic corresponding to the posterior obtained from pulling arm $i$ once.
\end{lemma}
\begin{myproof}[Proof.]
	For any $\hat y \in \mathcal{Y}$, let $\tau^\star(\hat y)$ be the (predictable) optimal stopping time for the problem (involving computing $V^K_\gamma$) whose initial state is $y_{i,0} = \hat y$. With this notation in hand, we conclude that
	\begin{align}
	\E_y \left[V^K_\gamma(y_{i,1}, \eta) \right] & = \E_y\left[\E_{y_{i,1}}\left[\sum_{s=1}^{\tau^\star(y_{i,1})} \gamma^{s-1} X_{i,s} + \frac{\gamma^{\tau^\star(y_{i,1})} R_{\eta, K}(\tau, y_{i,\tau^\star(y_{i,1})-1})}{1-\gamma} \right]\right]\label{eqn:pf_vk_bound_tower_prop1}  \\
	& \ge  \E_y\left[\E_{y_{i,2}}\left[\sum_{s=1}^{\tau^\star(y)} \gamma^{s-1} X_{i,s} + \frac{\gamma^{\tau^\star(y)} R_{\eta, K}(\tau, y_{i,\tau^\star(y)-1})}{1-\gamma}\right] \right] \label{ineq:subopt_of_y}\\
	& = \E_y\left[\sum_{s=1}^{\tau^\star(y)} \gamma^{s-1} X_{i,s} + \frac{\gamma^{\tau^\star(y)} R_{\eta, K}(\tau, y_{i,\tau^\star(y)-1})}{1-\gamma}\right] \label{eqn:pf_vk_bound_tower_prop2} \\
	& = V^K_\gamma(y, \eta) \nonumber
	\end{align}
	where \eqref{eqn:pf_vk_bound_tower_prop1}, \eqref{eqn:pf_vk_bound_tower_prop2} both follow from the tower property and \eqref{ineq:subopt_of_y} is due to the sub-optimality of the stopping rule $\tau^\star(y)$ when the actual starting state is $y_{i,1}$. Intuitively, we lose out revenue by throwing away information about the arm.
\end{myproof}


\section{Results for the frequentist regret bound}
This section contains proofs of results required to show Theorem~\ref{thm:frequentist_optimal_bound}. It is helpful to go over the definitions and some general properties of the Optimistic Gittins index given in Section~\ref{sec:appendix_properties_of_ogi} when reading this.
\subsection{Definitions and properties of Binomial distributions.}
We list notation and facts related to Beta and Binomial distributions, which are used through this section.
\begin{definition}
	$F^B_{n,p}(.)$ is the CDF of the Binomial distribution with parameters $n$ and $p$, and $F^\beta_{a,b}(.)$ is the CDF of the Beta distribution with parameters $a$ and $b$.
\end{definition}

\begin{lemma} \label{fact:equation_for_beta_binomial_cdfs}
	Let $a$ and $b$ be positive integers and $y \in [0,1]$, 
	\[
	F^\beta_{a,b}(y) = 1 - F^B_{a+b-1,y}(a-1)
	\]
\end{lemma}
\begin{myproof}[Proof.]
	Proof is found in \cite{agrawalanalysis}.
\end{myproof}
\begin{lemma} \label{fact:median_of_binomial_dist}
	The median of a Binomial$(n,p)$ distribution is either $\ceil{np}$ or $\floor{np}$.
\end{lemma}
\begin{myproof}[Proof]
	A proof of this fact can be found in \cite{jogdeo1968monotone}.
\end{myproof}

\begin{corollary}[Corollary of Fact~\ref{fact:median_of_binomial_dist}] \label{cor:corollarly_of_binomial_median_property}
	Let $n$ be a positive integer and $p \in (0,1)$. For any non-negative integer $s < np$
	\[
	F^B_{n,p}(s) \le 1/2
	\]
\end{corollary}

\begin{lemma} \label{fact:relationship_with_binom_cdfs}
	Let $n$ be a positive integer and $p \in [0,1]$. Then for any $k \in \{0,\ldots,n\}$,
	\[
	(1-p)F^B_{n-1,p}(k)\le F^B_{n,p}(k) \le F^B_{n-1,p}(k)
	\] 
\end{lemma}
\begin{myproof}[Proof]
	To prove $F^B_{n,p}(k) \le F^B_{n-1,p}(k)$, we let $X_1,\ldots,X_{n}$ be i.i.d samples from a Bernoulli($p$) distribution. We then have
	\begin{align*}
	F^B_{n,p}(k)  = \P{\sum_{i=1}^{n} X_i \le k}  \le  \P{\sum_{i=1}^{n-1} X_i \le k}  = F^B_{n-1,p}(k)
	\end{align*}
	Now to prove $(1-p)F^B_{n-1,p}(k)\le F^B_{n,p}(k)$, it's enough to observe that $F^B_{n,p}(k) = p F^B_{n-1,p}(k-1) + (1-p) F^B_{n-1,p}(k)$.
\end{myproof}

\subsubsection{Ratio of Binomial CDFs.} \label{sec:ratio_of_bin_cdfs}
\begin{lemma} \label{lemma:ratio_of_cdfs}
	Let $0< q < p < 1$. Let $n$ be a positive integer such that $e^{\frac{n}{2} d(q,p)} \ge (n+1)^4$ and let $k$ be a non-negative integer such that $k < nq$. It then follows that
	\[
	F^B_{n,q}(k)/F^B_{n,p}(k) >  e^{\frac{n}{2} d(q,p)}.
	\]
\end{lemma}
\begin{proof}[Proof.]
	From the method of types  (see \cite{cover2012elements}), we have for any $r \in (0,1)$ and $j < nr$
	\begin{equation} \label{eqn:appl_of_sanovs}
	\frac{e^{-nd(j/n, r)}}{(1+n)^2}\le F^B_{n,r}(j) \le (n+1)^2 e^{- n d(j/n, r)}.
	\end{equation}
	Because $k < nq < np$, by applying \eqref{eqn:appl_of_sanovs} to both the numerator and denominator, we get
	\begin{align*}
	\frac{F^B_{n,q}(k)}{F^B_{n,p}(k)} & \ge  \frac{e^{-nd(k/n, q)}}{(n+1)^4 e^{- n d(k/n, p)}} = \frac{e^{n(d(k/n,p) - d(k/n,q))}}{(n+1)^4}.
	\end{align*}
	Examining the exponent, we find
	\begin{align*}
	d(k/n, p) - d(k/n,q) & = \frac{k}{n} \log \frac{q}{p} + \left(1-\frac{k}{n}\right)\log \frac{1-q}{1-p} \\
	& > q \log \frac{q}{p} + (1-q)\log \frac{1-q}{1-p} \\
	& = d(q,p)
	\end{align*}
	where the bound holds because the expression is decreasing in $k$, and $k < nq$. Therefore,
	\begin{align}
	\frac{F^B_{n,q}(k)}{F^B_{n,p}(k)} & > \frac{e^{n  d(q,p)}}{(n+1)^4} = \frac{e^{\frac{n}{2}d(q,p)}}{(n+1)^4} e^{\frac{n}{2}d(q,p)} \ge e^{\frac{n}{2}d(q,p)} \label{bound:log_1minusq_etc}.
	\end{align}
	The final lower bound in \eqref{bound:log_1minusq_etc} follows from the assumption on $n$.
\end{proof}

\subsection{Proof of Lemma~\ref{lemma:underestimation}} \label{proof:underestimation_proof}
\begin{myproof}[Proof.]
	The proof hinges on showing that for any $K$, which is the number of look-ahead steps used to compute the Optimistic Gittins index, that
	\begin{equation} \label{eqn:big_oh_result_for_ogi}
	\P{v^K_{1,t} < \eta} = O\left(\frac{1}{t^{1 + h_\eta}}\right)
	\end{equation}
	where $h_\eta > 0$ is some constant that depends on $\eta$. After showing the above statement, the result would follow due to convergence of the series $\sum_{t=1}^\infty \P{v^K_{1,t} < \eta}$. The first step will be to show that for any $K \ge 1$ and any $\zeta \ge 0$, that there exists $h'_\eta > 0$, such that
	\begin{equation} \label{eqn:big_oh_result_for_vk}
		\P{(1-\gamma_t)V^K_{\gamma_t}(y_{1,\Nt{1}}, \eta) < \eta + \zeta/t} = O_{\eta,\zeta}\left(\frac{1}{t^{1 + h'_\eta}}\right),
	\end{equation}
	where $V^K_{\gamma_t}$ is the continuation value defined in Section~\ref{sec:appendix_properties_of_ogi} and $O_{\eta,\zeta}$ means that the constant in front the big-Oh depends on both $\zeta$ and $\eta$. After showing the above claim, Lemma~\ref{cor:equivalent_event} would imply Equation~\eqref{eqn:big_oh_result_for_ogi} because we know from that result that,
	\begin{align*}
		\P{v^K_{1,t} < \eta} & = \P{(1-\gamma_t)V^K_{\gamma_t}(y_{1,\Nt{1}}, \eta) < \eta} \\
		& = O\left(\frac{1}{t^{1 + h_\eta}}\right)
	\end{align*}
	for some $h_\eta > 0$. The second equation above is just a special case of \eqref{eqn:big_oh_result_for_vk} when $\zeta = 0$.
	
	Ultimately, showing equation \eqref{eqn:big_oh_result_for_vk}, and thus proving the Lemma, is an induction over the parameter $K$ and we begin with the base case, which requires some work using properties of the Beta and Binomial distributions.
	\subsubsection*{Proof of the base case}
	Let us fix $\zeta \ge 0$. We prove that when the algorithm uses a look-ahead parameter of $K = 1$, that there exists a positive constant $h_\eta$ such that
	 \begin{equation} \label{eqn:base_case_lemma_underestimation}
	 \P{(1-\gamma_t)V^1_{\gamma_t}(y_{1,\Nt{1}}, \eta) < \eta + \zeta/t} = O_{\eta,\zeta}\left(\frac{1}{t^{1 + h_\eta}}\right).
	 \end{equation}
	 %To simplify notation, let us abbreviate $v^1_{1,t}$ as $v_{1,t}$. 
	 First, we define $\delta := (\theta_1 - \eta)/2$ and  $\eta' :=  \eta + \delta$. In other words, $\delta$ is half the distance between $\eta$ and $\theta_1$; $\eta'$ is the point half-way. Recall that $\Nt{i}$ refers to the counting process for the number of pulls of arm $i$ up to \emph{but not including} time $t$ and that $S_i(t)$ is the corresponding total reward (or number of successes from all the Bernoulli trials). Showing this base case consists of showing two claims:
	\subsubsection*{Claim 1: $\{(1-\gamma_t)V^1_{\gamma_t}(y_{1,\Nt{1}}, \eta) < \eta + \zeta/t\} \subseteq \left\{F^B_{\Nt{1}+1, \eta'}(S_1(t)) < \frac{\zeta + 1}{\delta t}\right\}$}
	Let $V_t \sim $Beta$(S_1(t)+1,\Nt{1} - S_1(t) + 1)$ be the agent's posterior on the expected reward from the optimal arm (notice that $y_{1,\Nt{1}} = (S_1(t)+1,\Nt{1} - S_1(t) + 1)$ in this case). Using the simplified equation for the continuation value when $K =1$, namely $V^1_{\gamma_t}$ (see Equation~\eqref{eqn:ogi_k1}), 
	\[
		(1-\gamma_t)V^1_{\gamma_t}\left((S_1(t)+1, \Nt{1} - S_1(t) + 1), \eta\right) = \Ee{V_t} + \gamma_t\Ee{(\eta - V_t)^+},
	\] 
	we find that
	\begin{align}
	\left\{(1-\gamma_t)V^1_{\gamma_t}(y_{1,N_{1}(t)}, \eta) < \eta + \frac{\zeta}{t}\right\} & = \left\{ \Ee{V_t } + \gamma_t\Ee{(\eta - V_t)^+} < \eta + \frac{\zeta}{t}\right\} \nonumber\\
	& =  \left\{ (1-1/t)\Ee{(\eta - V_t)^+} < \Ee{\eta - V_t} +\frac{\zeta}{t} \right\} \label{eq:def_gamma_t} \\
	& =  \left\{ \Ee{(\eta - V_t)^+} - \Ee{\eta - V_t} <  \frac{1}{t}\Ee{(\eta - V_t)^+} +\frac{\zeta}{t}\right\} \nonumber\\
	& =  \left\{ \Ee{(V_t - \eta)^+}<  \frac{1}{t}\Ee{(\eta - V_t)^+} +\frac{\zeta}{t}\right\} \nonumber\\
	& \subseteq \left\{ \Ee{ (V_t - \eta)^+}< \frac{\zeta + 1}{t}  \right\} \label{eq:intermediate_event}
	\end{align}
	where \eqref{eq:def_gamma_t} follows from the definition of $\gamma_t$ and \eqref{eq:intermediate_event} is due to $V_t, \eta$ both lying in the interval $[0,1]$. We approximate the conditional expectation in \eqref{eq:intermediate_event} with the following bound:
	\begin{align}
	\Ee{(V_t - \eta)^+ } & = \Ee{(V_t - \eta) \ind{V_t \ge \eta} }\nonumber \\
	& = \Ee{(V_t - \eta) \ind{\eta + \delta > V_t \ge \eta} }  \nonumber \\
	& \qquad + \Ee{(V_t - \eta) \ind{ V_t \ge \eta + \delta} } \nonumber \\
	& > \Ee{(V_t - \eta) \ind{ V_t \ge \eta + \delta} } \nonumber \\
	& \ge \delta\P{ V_t \ge \eta' } \nonumber \\
	& = \delta (1 - F^\beta_{S_1(t)+1,\Nt{1}-S_1(t)+1}(\eta'))  \nonumber\\ 
	& = \delta F^B_{\Nt{1}+1,\eta'}(S_1(t)) \label{ineq:lower_bound_on_ppart_term}
	\end{align}
	where the final equality is due to Fact~\ref{fact:equation_for_beta_binomial_cdfs}. The claim then follows from the above bound and \eqref{eq:intermediate_event}. We proceed with the second part of the base case's proof:
	\subsubsection*{Claim 2: $\mathbb{P}\left(F^B_{\Nt{1}+1, \eta'}(S_1(t)) < \frac{\zeta + 1}{\delta t}\right) = O\left(\frac{1}{t^{1 + h_\eta}}\right)$ for some $h_\eta > 0$}
	Let us fix the sequence $f_t \defeq -\frac{\log (\delta t/(\zeta+1)) }{\log (1-\eta')}-1 = O(\log t)$. We then have by a straightforward decomposition that
	\begin{align}
	\P{F^B_{\Nt{1}+1, \eta'}(S_1(t)) < \frac{\zeta + 1}{\delta t}} & = \P{F^B_{\Nt{1}+1, \eta'}(S_1(t)) < \frac{\zeta + 1}{\delta t}, \; \Nt{1} > f_t}  \nonumber \\
	& \qquad + \P{F^B_{\Nt{1}+1, \eta'}(S_1(t)) < \frac{\zeta + 1}{\delta t}, \; \Nt{1} \le f_t} \label{eq:decomp2}.
	\end{align}
	Then notice that for the second term in the RHS of \eqref{eq:decomp2} we have the following bound,
	\begin{align}
	\P{F^B_{\Nt{1}+1, \eta'}(S_1(t)) < \frac{\zeta + 1}{\delta t}, \; \Nt{1} \le f_t}  &  \le \P{F^B_{\Nt{1}+1,\eta'}(0) < \frac{\zeta + 1}{\delta  t}, \; \Nt{1} \le f_t} \nonumber \\
	& = \P{(1-\eta')^{\Nt{1}+1} <  \frac{\zeta + 1}{\delta  t}, \; \Nt{1} \le f_t} \nonumber \\
	& \le \P{(1-\eta')^{f_t+1} <  \frac{\zeta + 1}{\delta  t}} \nonumber \\
	& = 0. \label{bound:bdd_by_zero}
	\end{align}
	Now we use the following fact to correspondingly bound the left term on the RHS of \eqref{eq:decomp2}. Define the function
	\[
	F^{-B}_{n,p}(u) := \inf\{x : F^B_{n,p}(x) \ge u\}
	\]
	which is the inverse CDF. Then it is known that if $U \sim \text{Uniform}(0,1)$, then $F^{-B}_{n,p}(U) \sim \text{Binomial}(n,p)$. Furthermore, the event $F^B_{n,p}(F^{-B}_{n,p}(U)) \ge U$ occurs with probability 1 due to the definition of the inverse CDF.
	
	Now let us only consider large $t$, in particular $t > M = M(\theta_1, \eta')$ where:
	\begin{enumerate}
		\item $M$ is such that $e^{d(\eta', \theta_1)f_{M}/2} > (f_M + 1)^4$ (we need this condition when we use Lemma~\ref{lemma:ratio_of_cdfs})
		\item $M > \frac{4(\zeta + 1)}{(1-\eta')\delta }$
		\item $\ceil{f_M} > 0$ and $F^B_{t',\eta'}(t' \eta') > 1/4$ for all $t' > \ceil{f_M}$. Note that there is a large enough integer for this because $F^B_{\ceil{f_t},\eta'}(f_t \eta') \to \frac{1}{2}$ as $t \to \infty$.
	\end{enumerate} 
	Suppose that $t > M$. It then follows that the event \[\left\{F^B_{\Nt{1}, \eta'}(S_1(t)) < \frac{\zeta + 1}{(1-\eta')\delta t},\; S_1(t) \ge \Nt{1} \eta', \; \Nt{1} > f_t\right\}\] has measure zero because of the assumptions made on $M$. Therefore if $t > M$, we have
	\begin{align}
	\mathbb{P}\bigg(F^B_{\Nt{1}+1,  \eta'}(S_1(t)) &< \frac{\zeta + 1}{\delta t }  , \; \Nt{1} > f_t \bigg) \nonumber \\
	& \le \P{F^B_{\Nt{1},  \eta'}(S_1(t)) < \frac{\zeta + 1}{(1-\eta')\delta t}, \; \Nt{1} > f_t} \label{eqn:part1_decomp_the_cdf_of_y} \\ 
	& = \P{F^B_{\Nt{1},  \eta'}(S_1(t)) < \frac{\zeta + 1}{(1-\eta')\delta t}, \; S_1(t) < \Nt{1} \eta', \; \Nt{1} > f_t} \nonumber \\ 
	& =  \P{F^B_{\Nt{1},\theta_1}(S_1(t)) \frac{F^B_{\Nt{1},\eta'}(S_1(t))}{F^B_{\Nt{1},\theta_1}(S_1(t))} < \frac{\zeta + 1}{(1-\eta')\delta  t}, \;S_1(t) < \Nt{1} \eta', \; \Nt{1} > f_t} \nonumber \\
	& \le  \P{F_{\Nt{1},\theta_1}^B(S_1(t))  e^{\Nt{1} D} < \frac{\zeta + 1}{(1-\eta')\delta  t} , \; \Nt{1} > f_t} \label{eqn:part1_app_of_lemma2} \\
	& \le  \P{F_{\Nt{1},\theta_1}^B(S_1(t)) e^{f_t D} < \frac{\zeta + 1}{(1-\eta')\delta  t}} \nonumber \\
	& =  \P{F_{\Nt{1},\theta_1}^B(F^{-B}_{\Nt{1},\theta_1}(U)) < \frac{e^{-f_t D}(\zeta + 1)}{(1-\eta')\delta  t} } \label{eqn:part1_propert_of_inverse_sampling}\\
	& \le  \P{U < \frac{e^{-f_t D}(\zeta + 1)}{(1-\eta')\delta  t} } \nonumber \\  
	& =  \frac{e^{-f_t D}(\zeta + 1)}{(1-\eta')\delta  t} \nonumber  \nonumber\\
	& = \mathcal O_{\eta, \zeta}\left( \frac{1}{t^{1+Dc_{\eta'}}} \right)  \label{bound:one_over_t_plus_eps} 
	\end{align}
	where $D = d(\eta',\theta_1) > 0$ and $c_{\eta'} = -\log^{-1}(1-\eta') > 0$ are constant. The bound \eqref{eqn:part1_decomp_the_cdf_of_y} holds due to Fact~\eqref{fact:relationship_with_binom_cdfs}. Bound \eqref{eqn:part1_app_of_lemma2} follows from an application of Lemma~\ref{lemma:ratio_of_cdfs} and the fact that $t > M$. Equation \eqref{eqn:part1_propert_of_inverse_sampling} follows from $S_1(t) \sim \text{Binomial}(\Nt{1}, \theta_1)$ and the inverse sampling technique. By combining bounds \eqref{bound:one_over_t_plus_eps}, \eqref{bound:bdd_by_zero} and \eqref{eq:decomp2}, we finally obtain the result for the base case by taking $h_\eta = Dc_{\eta'}$.

	\subsubsection*{Proof of the inductive step}
	 Now, suppose that for $K-1 \ge 1$ and any $\zeta \ge 0$,  the following induction hypothesis holds
	\[
	\P{(1-\gamma_t)V^{K-1}_{\gamma_t}(y_{1,\Nt{1}}, \eta) < \eta + \frac{\zeta}{t}} = O_{\eta,\zeta}\left(\frac{1}{t^{1 + h_\eta}}\right)
	\]
	for some $h_\eta > 0$. We prove the same result for the next integer $K$. Observe that when $t$ is large enough, using the Bellman equation for $V^K_\gamma$, we have
	\begin{align}
	&\P{(1-\gamma_t)V^K_{\gamma_t}(y_{1,\Nt{1}}, \eta) < \eta + \frac{\zeta}{t}} \nonumber  \\
	&\qquad = \mathbb{P}\left((1-\gamma_t)\Ee{X_{1,t} \given y_{1,\Nt{1}}}  \right. \nonumber\\
	&\hspace{6em} + \gamma_t \Ee{\max(\eta, (1-\gamma_t)V^{K-1}_{\gamma_t}(y_{1,\Nt{1}+1}, \eta)) \given y_{1,\Nt{1}}} < \eta + \frac{\zeta}{t}\bigg) \label{eq:appl_of_lemma_9} \\
	& \qquad \le \P{ \left(1-\frac{1}{t}\right) \Ee{(1-\gamma_t)V^{K-1}_{\gamma_{t}}(y_{1,\Nt{1}+1}, \eta) \given y_{1,\Nt{1}} }< \eta + \frac{\zeta}{t}} \nonumber \\
	& \qquad \le \P{\left(1-\frac{1}{t}\right)(1-\gamma_t) V^{K-1}_{\gamma_{t}}(y_{1,\Nt{1}}, \eta)< \eta  + \frac{\zeta}{t}} \label{ineq:missing_step} \\
	& \qquad \le \P{ (1-\gamma_t)V^{K-1}_{\gamma_{t}}(y_{1,\Nt{1}}, \eta)< \frac{t}{t-1}\left(\eta +  \frac{\zeta}{t}}\right) \nonumber \\
	& \qquad \le \P{ (1-\gamma_t)V^{K-1}_{\gamma_{t}}(y_{1,\Nt{1}}, \eta)< \eta + \frac{\eta}{t-1} +  \frac{\zeta}{t-1}} \nonumber \\
	& \qquad \le \P{ (1-\gamma_t)V^{K-1}_{\gamma_{t}}(y_{1,\Nt{1}}, \eta)< \eta +   \frac{\zeta + 1}{t}} \nonumber \\
	&\qquad =  O_{\eta,\zeta}\left(\frac{1}{t^{1 + h_\eta}}\right) \label{eqn:ind_hyp}
	\end{align}
	where the final inequality holds when $t$ is large enough because $\eta < 1$, equation \eqref{eq:appl_of_lemma_9} results from an expansion of Bellman's equation and bound \eqref{ineq:missing_step} follows from Lemma~\ref{lemma:vk_bound}. %\eqref{eqn:another_appl_of_lemma_9} follows from both \eqref{bnd:conc_result} and Lemma~\ref{cor:equivalent_event}. 
	Finally, equation \eqref{eqn:ind_hyp} follows from the induction hypothesis.
	%We finish the proof by using the following asymptotic argument. Take $M$ to be a large enough integer, then we have, using the result of the preceding induction proof, that
	%\begin{align*}
	%\sum_{t=1}^\infty \P{v^K_{1,t} < \eta} & \le M +  \sum_{t=M+1}^\infty \frac{C_1}{t^{1 + h(\eta)}} \le M +  C_2
	%\end{align*}
	%where $C_2 = C_2(\eta)$ is the limit of the series and $C_1$ is a constant used in the definition of the big-Oh.
\end{myproof}

\subsection{Proof of Lemma~\ref{lemma:overestimation}} \label{proof:overestimation_proof}

\begin{proof}[Proof.]
	See the main proof of Theorem~\ref{thm:frequentist_optimal_bound} to recall the definition of constants $\eta_1$, $\eta_3$ and their relationship with $\theta_2$ and $\theta_1$. As an abbreviation we let $L = L(T)$. Moreover, because for any arm $i$ $v^K_{i,t} \le v^{K-1}_{i,t} \le \ldots \le v^1_{i,t}$ (Lemma~\ref{lemma:approx_bound}), it will be sufficient to consider this proof only for $v^1_{2,t}$, which we also will abbreviate as $v_{2,t} \defeq v^1_{2,t}$. Similarly, we will abbreviate the notation for the OGI policy as $\pi^{OG}$ and suppress the parameter $K$.
	
	Firstly, by the law of total probability and the definition of $P_i(t)$ in Section~\ref{sec:appendix_properties_of_ogi}, we find that
	\begin{align} 
	\sum_{t=1}^T \mathbb{P}(v_{2,t} & \ge \eta_3 ,\; N_{2}(t-1) \ge L,\; \pi^{\rm OG}_t = 2) \nonumber \\
	& = \sum_{t=1}^T \P{v_{2,t} \ge \eta_3 ,\; \Nt{2} \ge L, \; S_2(t) < \floor{\Nt{2} \eta_1}, \; \pi^{\rm OG}_t = 2} \nonumber \\
	& \qquad + \sum_{t=1}^T \P{v_{2,t} \ge \eta_3 ,\; \Nt{2} \ge L, \; S_2(t) \ge \floor{\Nt{2} \eta_1},\; \pi^{\rm OG}_t = 2} \nonumber \\
	& \le \sum_{t=1}^T \P{v_{2,t} \ge \eta_3 ,\; \Nt{2} \ge L, \; S_2(t) < \floor{\Nt{2} \eta_1}} + \sum_{t=1}^T \P{\pi^{\rm OG}_t = 2,\; S_2(t) \ge \floor{\Nt{2} \eta_1}} \label{eqn:splitting_not_underestimate},
	\end{align}
	where $S_2(t)$ is also defined in Section~\ref{sec:appendix_properties_of_ogi} as the total reward from the second arm observed up to time $t-1$. Let $V_t \sim \text{Beta}(S_2(t) + 1, \Nt{2}- S_2(t) + 1)$ denote the agent's posterior on the second arm at time $t$, then
	\begin{align}
	\sum_{t=1}^T \mathbb{P}(v_{2,t} \ge \eta_3 ,\; & \; \Nt{2} \ge L,\; S_2(t) < \floor{\Nt{2} \eta_1})  \nonumber\\
	& = \sum_{t=1}^T \P{\Ee{V_t} + \gamma_t \Ee{(\eta_3 - V_t)^+} \ge \eta_3, \; \Nt{2} \ge L,\; S_2(t) < \floor{\Nt{2} \eta_1}} \nonumber \\
	& = \sum_{t=1}^T \P{\frac{\Ee{(\eta_3-V_t)^+ }}{  \Ee{(V_t - \eta_3)^+ }} \le t , \; \Nt{2} \ge L,\; S_2(t) < \floor{\Nt{2} \eta_1}} \label{eq:complicated_rv_in_part2}
	\end{align}
	where the first equality follows from Lemma~\ref{cor:equivalent_event} and the simplified form of the continuation value (defined in Section~\ref{sec:appendix_properties_of_ogi}) when $K = 1$. The following result lets us bound \eqref{eq:complicated_rv_in_part2},
	\begin{lemma} \label{lem:lb_rv2}
		Let $0 < x < y < 1$. For any non-negative integers $s$ and $k$ with $s < \floor{kx}$, it holds that
		\begin{equation*}
		\frac{\Ee{(y-V)^+ }}{  \Ee{(V - y)^+ } } \ge \frac{(y-x) \exp(k d(x,y))}{2}
		\end{equation*}
		where $V \sim \text{Beta}(s+1,k-s+1)$.
	\end{lemma}
	\begin{myproof}[Proof.]
		See Appendix~\ref{prf:proof_of_lb_rv2}.
	\end{myproof}
	Therefore, from equation \eqref{eq:complicated_rv_in_part2} and Lemma~\ref{lem:lb_rv2}, we find that whenever $T > \left(\frac{2}{\eta_3-\eta_1}\right)^{1/\eps} =: T^*(\eps, \theta)$,
	\begin{align}
	\sum_{t=1}^T \mathbb{P}(v_{2,t} \ge \eta_3 ,\; & \; \Nt{2} \ge L,\; S_2(t) < \floor{\Nt{2} \eta_1}) \nonumber \\
	& \le  \sum_{t=1}^T\P{  (\eta_3-\eta_1) \exp\{\Nt{2} d(\eta_1,\eta_3) \} \le 2t,\; \Nt{2} \ge L} \nonumber \\
	& \le  \sum_{t=1}^T\P{  (\eta_3-\eta_1) \exp\{L d(\eta_1,\eta_3) \} \le 2t} \nonumber \\
	& =   \sum_{t=1}^T\P{  (\eta_3-\eta_1) T^{1+\eps} \le 2t} \nonumber \\
	& = 0. \label{bound:equal_to_zero}
	\end{align}
	All that is left is to bound the second term in \eqref{eqn:splitting_not_underestimate}, and to do so we apply the following Lemma whose proof is in Appendix~\ref{prf:proof_of_acc_sub_means}
	\begin{lemma} \label{lem:accurate_suboptimal_mean}
		There exist positive constants $C = C(\theta_2,\eta_1)$ and $x' > \theta_2$ such that
		\begin{equation*}
		\sum_{t=1}^T \P{S_2(t) \ge \floor{\Nt{2} \eta_1}, \; \pi^{\rm OG}_t = 2} \le  K + \frac{1}{1 - e^{-d(x',\theta_2)}} 
		\end{equation*}
	\end{lemma}
	Combining Lemma~\ref{lem:accurate_suboptimal_mean}, \eqref{bound:equal_to_zero}, \eqref{eqn:splitting_not_underestimate} and \eqref{eq:complicated_rv_in_part2} shows the claim.
\end{proof}

\subsubsection{Proof of Lemma~\ref{lem:lb_rv2}.} \label{prf:proof_of_lb_rv2}
\begin{myproof}[Proof.]
	We upper bound the denominator as follows. Given that $s < \floor{k x}$, we have $s \le kx - 1$. Let $B(a,b)$ denote the Beta function for parameters $a, b > 0$, that is
	\[
	B(a, b) \defeq \int_0^1 t^{a-1}(1-t)^{b-1} \;dt,
	\]
	which is used in the definition of the Beta CDF. We can derive an upper bound on the denominator in the following way. Namely, we have
	\begin{align}
	\Ee{(V - y)^+ } & = \frac{1}{B(s+1,k-s+1)}\int_{y}^1 (t-y) t^s (1-t)^{k-s} \; dt \nonumber \\
	& = \frac{1}{B(s+1,k-s+1)}\int_{y}^1 t^{s+1} (1-t)^{k-s} \; dt - y \P{V \ge y} \nonumber \\
	& = \frac{B(s+2,k-s+1)}{B(s+1,j-s+1)}\left( \frac{1}{B(s+2,k-s+1)} \right)\int_{y}^1 t^{s+1} (1-t)^{k-s}\; dt - y \P{V \ge y} \nonumber \\
	& = \frac{s+1}{k+2} F^B_{k+2,y}(s+1)  - y \P{V \ge y} \label{eq:part2_use_of_equiv_between_beta_and_binom} \\
	& \le \frac{s+1}{k+2} F^B_{k+2,y}(s+1) \nonumber \\
	& \le  F^B_{k,y}(k x)  \nonumber \\
	& \le \exp\left\{- k d(x,y) \label{ineq:chernoff_app} \right\}
	\end{align}
	where we use Fact~\ref{fact:equation_for_beta_binomial_cdfs} and the definition of the Beta CDF to establish equation \eqref{eq:part2_use_of_equiv_between_beta_and_binom}. The final bound in \eqref{ineq:chernoff_app} is the result of the Chernoff-Hoeffding theorem and Fact~\ref{fact:relationship_with_binom_cdfs}. Let $\delta:=y-x$, and note that $s < kx \Longrightarrow s \le \floor{(k+1)x}$ due to $s$ being integer, whence
	\begin{align}
	\Ee{(y - V)^+ } & =  \Ee{(y - V) \ind{V \le y}} \nonumber \\
	& = \Ee{(y - V) \ind{y - \delta \le V \le y} } +  \Ee{(y - V) \ind{V < y - \delta} } \nonumber\\
	& > \Ee{(y - V) \ind{V < y - \delta} }\nonumber \\
	& \ge \delta\Ee{\ind{V < y-\delta} }\\
	& = \delta \P{V < x } \nonumber \\
	& = \delta\left(1 - F^B_{k+1,x}(s) \right) \label{eq:use_of_bin_beta_identity}  \\
	& \ge \delta/2  \label{eq:use_of_median_prop}
	\end{align}
	where equation \eqref{eq:use_of_bin_beta_identity} relies on Fact~\ref{fact:equation_for_beta_binomial_cdfs}. The bound \eqref{eq:use_of_median_prop} is justified from Fact~\ref{fact:median_of_binomial_dist} and $s \le \floor{(k+1) x}$. Thus using the inequalities for both the numerator and denominator, we obtain the desired bound.
\end{myproof}
\subsubsection{Proof of Lemma~\ref{lem:accurate_suboptimal_mean}.} \label{prf:proof_of_acc_sub_means}
\begin{proof}[Proof.]
	The steps in this proof follow a similar one in \cite{agrawal2013further} but we show them for completeness. We bound the number of times the sub-optimal arm's mean is overestimated. Let $\tau_\ell$ be the time step in which the  sub-optimal arm is sampled for the $\ell$\textsuperscript{th} time. Because for any $x$, $\lim_{n\to\infty}\frac{\floor{nx}}{nx} = 1$, we can let $N$ be a large enough integer so that if $\ell \ge N$, then $\eta_1 \frac{\floor{\ell \eta_1}}{\ell \eta_1} > x' := (\theta_2 + \eta_1)/2 > \theta_2$. In that case,
	\begin{align}
	\sum_{t=1}^T\P{S_2(t) \ge \floor{\Nt{2} \eta_1}, \; \pi^{\rm OG}_t = 2} & \le \Ee{\sum_{\ell=1}^T \sum_{t=\tau_\ell}^{\tau_{\ell+1}-1}\ind{S_2(\ell) \ge \floor{\Ntg{2}{\ell} \eta_1}} \ind{\pi^{\rm OG}_t = 2}} \nonumber \\
	& = \Ee{\sum_{\ell=1}^T \ind{S_2(\tau_{\ell}) \ge \floor{(\ell-1) \eta_1}} \sum_{t=\tau_\ell}^{\tau_{\ell+1}-1} \ind{\pi^{\rm OG}_t = 2}} \nonumber\\
	& = \Ee{\sum_{\ell=0}^{T-1} \ind{S_2(\tau_{\ell+1}) \ge \floor{\ell \eta_1}}} \nonumber\\
	& \le  N + \sum_{\ell=N+1}^{T-1} \P{ S_2(\tau_{\ell+1}) \ge \ell \eta_1 \frac{\floor{\ell \eta_1}}{\ell \eta_1}} \nonumber \\
	& \le N + \sum_{\ell=N+1}^{T-1} \P{ S_2(\tau_{\ell+1}) \ge \ell x'} \nonumber \\
	& \le  N + \sum_{\ell=1}^{\infty} \exp(-\ell d(x', \theta_2)) \label{bound:cf_thm} \\
	& = N + \frac{1}{1 - e^{-d(x',\theta_2)}} \nonumber
	\end{align}
	where \eqref{bound:cf_thm} follows from the Chernoff-Hoeffding theorem and the fact that $S_2(\tau_{\ell+1})$ is drawn from a $\text{Binomial}(\Ntg{2}{\ell+1}, \theta_2) \equiv \text{Binomial}(\ell, \theta_2)$ distribution.
\end{proof}

\section{Further experiment results} \label{sec:further_exp}
\subsection{Bayes UCB experiment} \label{exp:bayes_ucb}
This experiment is motivated by \cite{kaufmann2012thompson} and in it we simulate the Bernoulli bandit problem with a $T = 500$ and two arms. Since we are interested in measuring expected regret over the prior, we draw the arms' mean rewards at random from the uniform distribution. There are 5,000 independent trials and we show the results in Figures~\ref{fig:kaufmann_regret}. OGI offers notable performance improvements over both Thompson Sampling and IDS for this modest horizon.
\begin{figure}[h!]
	\centering
	%% Creator: Matplotlib, PGF backend
%%
%% To include the figure in your LaTeX document, write
%%   \input{<filename>.pgf}
%%
%% Make sure the required packages are loaded in your preamble
%%   \usepackage{pgf}
%%
%% Figures using additional raster images can only be included by \input if
%% they are in the same directory as the main LaTeX file. For loading figures
%% from other directories you can use the `import` package
%%   \usepackage{import}
%% and then include the figures with
%%   \import{<path to file>}{<filename>.pgf}
%%
%% Matplotlib used the following preamble
%%   \usepackage[utf8x]{inputenc}
%%   \usepackage[T1]{fontenc}
%%
\begingroup%
\makeatletter%
\begin{pgfpicture}%
\pgfpathrectangle{\pgfpointorigin}{\pgfqpoint{6.099066in}{3.769430in}}%
\pgfusepath{use as bounding box, clip}%
\begin{pgfscope}%
\pgfsetbuttcap%
\pgfsetmiterjoin%
\definecolor{currentfill}{rgb}{1.000000,1.000000,1.000000}%
\pgfsetfillcolor{currentfill}%
\pgfsetlinewidth{0.000000pt}%
\definecolor{currentstroke}{rgb}{1.000000,1.000000,1.000000}%
\pgfsetstrokecolor{currentstroke}%
\pgfsetdash{}{0pt}%
\pgfpathmoveto{\pgfqpoint{0.000000in}{0.000000in}}%
\pgfpathlineto{\pgfqpoint{6.099066in}{0.000000in}}%
\pgfpathlineto{\pgfqpoint{6.099066in}{3.769430in}}%
\pgfpathlineto{\pgfqpoint{0.000000in}{3.769430in}}%
\pgfpathclose%
\pgfusepath{fill}%
\end{pgfscope}%
\begin{pgfscope}%
\pgfsetbuttcap%
\pgfsetmiterjoin%
\definecolor{currentfill}{rgb}{1.000000,1.000000,1.000000}%
\pgfsetfillcolor{currentfill}%
\pgfsetlinewidth{0.000000pt}%
\definecolor{currentstroke}{rgb}{0.000000,0.000000,0.000000}%
\pgfsetstrokecolor{currentstroke}%
\pgfsetstrokeopacity{0.000000}%
\pgfsetdash{}{0pt}%
\pgfpathmoveto{\pgfqpoint{0.762383in}{0.471179in}}%
\pgfpathlineto{\pgfqpoint{5.489159in}{0.471179in}}%
\pgfpathlineto{\pgfqpoint{5.489159in}{3.317098in}}%
\pgfpathlineto{\pgfqpoint{0.762383in}{3.317098in}}%
\pgfpathclose%
\pgfusepath{fill}%
\end{pgfscope}%
\begin{pgfscope}%
\pgfsetbuttcap%
\pgfsetroundjoin%
\definecolor{currentfill}{rgb}{0.000000,0.000000,0.000000}%
\pgfsetfillcolor{currentfill}%
\pgfsetlinewidth{0.803000pt}%
\definecolor{currentstroke}{rgb}{0.000000,0.000000,0.000000}%
\pgfsetstrokecolor{currentstroke}%
\pgfsetdash{}{0pt}%
\pgfsys@defobject{currentmarker}{\pgfqpoint{0.000000in}{-0.048611in}}{\pgfqpoint{0.000000in}{0.000000in}}{%
\pgfpathmoveto{\pgfqpoint{0.000000in}{0.000000in}}%
\pgfpathlineto{\pgfqpoint{0.000000in}{-0.048611in}}%
\pgfusepath{stroke,fill}%
}%
\begin{pgfscope}%
\pgfsys@transformshift{0.762383in}{0.471179in}%
\pgfsys@useobject{currentmarker}{}%
\end{pgfscope}%
\end{pgfscope}%
\begin{pgfscope}%
\pgftext[x=0.762383in,y=0.373957in,,top]{\rmfamily\fontsize{8.000000}{9.600000}\selectfont \(\displaystyle 0\)}%
\end{pgfscope}%
\begin{pgfscope}%
\pgfsetbuttcap%
\pgfsetroundjoin%
\definecolor{currentfill}{rgb}{0.000000,0.000000,0.000000}%
\pgfsetfillcolor{currentfill}%
\pgfsetlinewidth{0.803000pt}%
\definecolor{currentstroke}{rgb}{0.000000,0.000000,0.000000}%
\pgfsetstrokecolor{currentstroke}%
\pgfsetdash{}{0pt}%
\pgfsys@defobject{currentmarker}{\pgfqpoint{0.000000in}{-0.048611in}}{\pgfqpoint{0.000000in}{0.000000in}}{%
\pgfpathmoveto{\pgfqpoint{0.000000in}{0.000000in}}%
\pgfpathlineto{\pgfqpoint{0.000000in}{-0.048611in}}%
\pgfusepath{stroke,fill}%
}%
\begin{pgfscope}%
\pgfsys@transformshift{1.709633in}{0.471179in}%
\pgfsys@useobject{currentmarker}{}%
\end{pgfscope}%
\end{pgfscope}%
\begin{pgfscope}%
\pgftext[x=1.709633in,y=0.373957in,,top]{\rmfamily\fontsize{8.000000}{9.600000}\selectfont \(\displaystyle 100\)}%
\end{pgfscope}%
\begin{pgfscope}%
\pgfsetbuttcap%
\pgfsetroundjoin%
\definecolor{currentfill}{rgb}{0.000000,0.000000,0.000000}%
\pgfsetfillcolor{currentfill}%
\pgfsetlinewidth{0.803000pt}%
\definecolor{currentstroke}{rgb}{0.000000,0.000000,0.000000}%
\pgfsetstrokecolor{currentstroke}%
\pgfsetdash{}{0pt}%
\pgfsys@defobject{currentmarker}{\pgfqpoint{0.000000in}{-0.048611in}}{\pgfqpoint{0.000000in}{0.000000in}}{%
\pgfpathmoveto{\pgfqpoint{0.000000in}{0.000000in}}%
\pgfpathlineto{\pgfqpoint{0.000000in}{-0.048611in}}%
\pgfusepath{stroke,fill}%
}%
\begin{pgfscope}%
\pgfsys@transformshift{2.656883in}{0.471179in}%
\pgfsys@useobject{currentmarker}{}%
\end{pgfscope}%
\end{pgfscope}%
\begin{pgfscope}%
\pgftext[x=2.656883in,y=0.373957in,,top]{\rmfamily\fontsize{8.000000}{9.600000}\selectfont \(\displaystyle 200\)}%
\end{pgfscope}%
\begin{pgfscope}%
\pgfsetbuttcap%
\pgfsetroundjoin%
\definecolor{currentfill}{rgb}{0.000000,0.000000,0.000000}%
\pgfsetfillcolor{currentfill}%
\pgfsetlinewidth{0.803000pt}%
\definecolor{currentstroke}{rgb}{0.000000,0.000000,0.000000}%
\pgfsetstrokecolor{currentstroke}%
\pgfsetdash{}{0pt}%
\pgfsys@defobject{currentmarker}{\pgfqpoint{0.000000in}{-0.048611in}}{\pgfqpoint{0.000000in}{0.000000in}}{%
\pgfpathmoveto{\pgfqpoint{0.000000in}{0.000000in}}%
\pgfpathlineto{\pgfqpoint{0.000000in}{-0.048611in}}%
\pgfusepath{stroke,fill}%
}%
\begin{pgfscope}%
\pgfsys@transformshift{3.604132in}{0.471179in}%
\pgfsys@useobject{currentmarker}{}%
\end{pgfscope}%
\end{pgfscope}%
\begin{pgfscope}%
\pgftext[x=3.604132in,y=0.373957in,,top]{\rmfamily\fontsize{8.000000}{9.600000}\selectfont \(\displaystyle 300\)}%
\end{pgfscope}%
\begin{pgfscope}%
\pgfsetbuttcap%
\pgfsetroundjoin%
\definecolor{currentfill}{rgb}{0.000000,0.000000,0.000000}%
\pgfsetfillcolor{currentfill}%
\pgfsetlinewidth{0.803000pt}%
\definecolor{currentstroke}{rgb}{0.000000,0.000000,0.000000}%
\pgfsetstrokecolor{currentstroke}%
\pgfsetdash{}{0pt}%
\pgfsys@defobject{currentmarker}{\pgfqpoint{0.000000in}{-0.048611in}}{\pgfqpoint{0.000000in}{0.000000in}}{%
\pgfpathmoveto{\pgfqpoint{0.000000in}{0.000000in}}%
\pgfpathlineto{\pgfqpoint{0.000000in}{-0.048611in}}%
\pgfusepath{stroke,fill}%
}%
\begin{pgfscope}%
\pgfsys@transformshift{4.551382in}{0.471179in}%
\pgfsys@useobject{currentmarker}{}%
\end{pgfscope}%
\end{pgfscope}%
\begin{pgfscope}%
\pgftext[x=4.551382in,y=0.373957in,,top]{\rmfamily\fontsize{8.000000}{9.600000}\selectfont \(\displaystyle 400\)}%
\end{pgfscope}%
\begin{pgfscope}%
\pgfsetbuttcap%
\pgfsetroundjoin%
\definecolor{currentfill}{rgb}{0.000000,0.000000,0.000000}%
\pgfsetfillcolor{currentfill}%
\pgfsetlinewidth{0.803000pt}%
\definecolor{currentstroke}{rgb}{0.000000,0.000000,0.000000}%
\pgfsetstrokecolor{currentstroke}%
\pgfsetdash{}{0pt}%
\pgfsys@defobject{currentmarker}{\pgfqpoint{-0.048611in}{0.000000in}}{\pgfqpoint{0.000000in}{0.000000in}}{%
\pgfpathmoveto{\pgfqpoint{0.000000in}{0.000000in}}%
\pgfpathlineto{\pgfqpoint{-0.048611in}{0.000000in}}%
\pgfusepath{stroke,fill}%
}%
\begin{pgfscope}%
\pgfsys@transformshift{0.762383in}{0.515648in}%
\pgfsys@useobject{currentmarker}{}%
\end{pgfscope}%
\end{pgfscope}%
\begin{pgfscope}%
\pgftext[x=0.606132in,y=0.477386in,left,base]{\rmfamily\fontsize{8.000000}{9.600000}\selectfont \(\displaystyle 0\)}%
\end{pgfscope}%
\begin{pgfscope}%
\pgfsetbuttcap%
\pgfsetroundjoin%
\definecolor{currentfill}{rgb}{0.000000,0.000000,0.000000}%
\pgfsetfillcolor{currentfill}%
\pgfsetlinewidth{0.803000pt}%
\definecolor{currentstroke}{rgb}{0.000000,0.000000,0.000000}%
\pgfsetstrokecolor{currentstroke}%
\pgfsetdash{}{0pt}%
\pgfsys@defobject{currentmarker}{\pgfqpoint{-0.048611in}{0.000000in}}{\pgfqpoint{0.000000in}{0.000000in}}{%
\pgfpathmoveto{\pgfqpoint{0.000000in}{0.000000in}}%
\pgfpathlineto{\pgfqpoint{-0.048611in}{0.000000in}}%
\pgfusepath{stroke,fill}%
}%
\begin{pgfscope}%
\pgfsys@transformshift{0.762383in}{1.005062in}%
\pgfsys@useobject{currentmarker}{}%
\end{pgfscope}%
\end{pgfscope}%
\begin{pgfscope}%
\pgftext[x=0.606132in,y=0.966800in,left,base]{\rmfamily\fontsize{8.000000}{9.600000}\selectfont \(\displaystyle 1\)}%
\end{pgfscope}%
\begin{pgfscope}%
\pgfsetbuttcap%
\pgfsetroundjoin%
\definecolor{currentfill}{rgb}{0.000000,0.000000,0.000000}%
\pgfsetfillcolor{currentfill}%
\pgfsetlinewidth{0.803000pt}%
\definecolor{currentstroke}{rgb}{0.000000,0.000000,0.000000}%
\pgfsetstrokecolor{currentstroke}%
\pgfsetdash{}{0pt}%
\pgfsys@defobject{currentmarker}{\pgfqpoint{-0.048611in}{0.000000in}}{\pgfqpoint{0.000000in}{0.000000in}}{%
\pgfpathmoveto{\pgfqpoint{0.000000in}{0.000000in}}%
\pgfpathlineto{\pgfqpoint{-0.048611in}{0.000000in}}%
\pgfusepath{stroke,fill}%
}%
\begin{pgfscope}%
\pgfsys@transformshift{0.762383in}{1.494477in}%
\pgfsys@useobject{currentmarker}{}%
\end{pgfscope}%
\end{pgfscope}%
\begin{pgfscope}%
\pgftext[x=0.606132in,y=1.456215in,left,base]{\rmfamily\fontsize{8.000000}{9.600000}\selectfont \(\displaystyle 2\)}%
\end{pgfscope}%
\begin{pgfscope}%
\pgfsetbuttcap%
\pgfsetroundjoin%
\definecolor{currentfill}{rgb}{0.000000,0.000000,0.000000}%
\pgfsetfillcolor{currentfill}%
\pgfsetlinewidth{0.803000pt}%
\definecolor{currentstroke}{rgb}{0.000000,0.000000,0.000000}%
\pgfsetstrokecolor{currentstroke}%
\pgfsetdash{}{0pt}%
\pgfsys@defobject{currentmarker}{\pgfqpoint{-0.048611in}{0.000000in}}{\pgfqpoint{0.000000in}{0.000000in}}{%
\pgfpathmoveto{\pgfqpoint{0.000000in}{0.000000in}}%
\pgfpathlineto{\pgfqpoint{-0.048611in}{0.000000in}}%
\pgfusepath{stroke,fill}%
}%
\begin{pgfscope}%
\pgfsys@transformshift{0.762383in}{1.983891in}%
\pgfsys@useobject{currentmarker}{}%
\end{pgfscope}%
\end{pgfscope}%
\begin{pgfscope}%
\pgftext[x=0.606132in,y=1.945629in,left,base]{\rmfamily\fontsize{8.000000}{9.600000}\selectfont \(\displaystyle 3\)}%
\end{pgfscope}%
\begin{pgfscope}%
\pgfsetbuttcap%
\pgfsetroundjoin%
\definecolor{currentfill}{rgb}{0.000000,0.000000,0.000000}%
\pgfsetfillcolor{currentfill}%
\pgfsetlinewidth{0.803000pt}%
\definecolor{currentstroke}{rgb}{0.000000,0.000000,0.000000}%
\pgfsetstrokecolor{currentstroke}%
\pgfsetdash{}{0pt}%
\pgfsys@defobject{currentmarker}{\pgfqpoint{-0.048611in}{0.000000in}}{\pgfqpoint{0.000000in}{0.000000in}}{%
\pgfpathmoveto{\pgfqpoint{0.000000in}{0.000000in}}%
\pgfpathlineto{\pgfqpoint{-0.048611in}{0.000000in}}%
\pgfusepath{stroke,fill}%
}%
\begin{pgfscope}%
\pgfsys@transformshift{0.762383in}{2.473306in}%
\pgfsys@useobject{currentmarker}{}%
\end{pgfscope}%
\end{pgfscope}%
\begin{pgfscope}%
\pgftext[x=0.606132in,y=2.435044in,left,base]{\rmfamily\fontsize{8.000000}{9.600000}\selectfont \(\displaystyle 4\)}%
\end{pgfscope}%
\begin{pgfscope}%
\pgfsetbuttcap%
\pgfsetroundjoin%
\definecolor{currentfill}{rgb}{0.000000,0.000000,0.000000}%
\pgfsetfillcolor{currentfill}%
\pgfsetlinewidth{0.803000pt}%
\definecolor{currentstroke}{rgb}{0.000000,0.000000,0.000000}%
\pgfsetstrokecolor{currentstroke}%
\pgfsetdash{}{0pt}%
\pgfsys@defobject{currentmarker}{\pgfqpoint{-0.048611in}{0.000000in}}{\pgfqpoint{0.000000in}{0.000000in}}{%
\pgfpathmoveto{\pgfqpoint{0.000000in}{0.000000in}}%
\pgfpathlineto{\pgfqpoint{-0.048611in}{0.000000in}}%
\pgfusepath{stroke,fill}%
}%
\begin{pgfscope}%
\pgfsys@transformshift{0.762383in}{2.962720in}%
\pgfsys@useobject{currentmarker}{}%
\end{pgfscope}%
\end{pgfscope}%
\begin{pgfscope}%
\pgftext[x=0.606132in,y=2.924458in,left,base]{\rmfamily\fontsize{8.000000}{9.600000}\selectfont \(\displaystyle 5\)}%
\end{pgfscope}%
\begin{pgfscope}%
\pgfpathrectangle{\pgfqpoint{0.762383in}{0.471179in}}{\pgfqpoint{4.726776in}{2.845920in}} %
\pgfusepath{clip}%
\pgfsetrectcap%
\pgfsetroundjoin%
\pgfsetlinewidth{1.505625pt}%
\definecolor{currentstroke}{rgb}{0.121569,0.466667,0.705882}%
\pgfsetstrokecolor{currentstroke}%
\pgfsetdash{}{0pt}%
\pgfpathmoveto{\pgfqpoint{0.762383in}{0.600539in}}%
\pgfpathlineto{\pgfqpoint{0.781328in}{0.681933in}}%
\pgfpathlineto{\pgfqpoint{0.800273in}{0.745512in}}%
\pgfpathlineto{\pgfqpoint{0.838163in}{0.820597in}}%
\pgfpathlineto{\pgfqpoint{0.847636in}{0.841864in}}%
\pgfpathlineto{\pgfqpoint{0.857108in}{0.854517in}}%
\pgfpathlineto{\pgfqpoint{0.885526in}{0.907454in}}%
\pgfpathlineto{\pgfqpoint{0.904471in}{0.934131in}}%
\pgfpathlineto{\pgfqpoint{0.923416in}{0.954348in}}%
\pgfpathlineto{\pgfqpoint{0.942361in}{0.970748in}}%
\pgfpathlineto{\pgfqpoint{0.951833in}{0.985751in}}%
\pgfpathlineto{\pgfqpoint{0.961306in}{0.993412in}}%
\pgfpathlineto{\pgfqpoint{0.970778in}{1.007142in}}%
\pgfpathlineto{\pgfqpoint{0.980251in}{1.015095in}}%
\pgfpathlineto{\pgfqpoint{0.989723in}{1.028241in}}%
\pgfpathlineto{\pgfqpoint{0.999196in}{1.034236in}}%
\pgfpathlineto{\pgfqpoint{1.008668in}{1.042879in}}%
\pgfpathlineto{\pgfqpoint{1.018141in}{1.046819in}}%
\pgfpathlineto{\pgfqpoint{1.027613in}{1.054875in}}%
\pgfpathlineto{\pgfqpoint{1.046558in}{1.064810in}}%
\pgfpathlineto{\pgfqpoint{1.056031in}{1.071104in}}%
\pgfpathlineto{\pgfqpoint{1.065503in}{1.084245in}}%
\pgfpathlineto{\pgfqpoint{1.074976in}{1.089266in}}%
\pgfpathlineto{\pgfqpoint{1.084448in}{1.100841in}}%
\pgfpathlineto{\pgfqpoint{1.112866in}{1.119810in}}%
\pgfpathlineto{\pgfqpoint{1.122338in}{1.124636in}}%
\pgfpathlineto{\pgfqpoint{1.131811in}{1.127401in}}%
\pgfpathlineto{\pgfqpoint{1.141283in}{1.136827in}}%
\pgfpathlineto{\pgfqpoint{1.160228in}{1.145103in}}%
\pgfpathlineto{\pgfqpoint{1.169701in}{1.154231in}}%
\pgfpathlineto{\pgfqpoint{1.207591in}{1.171958in}}%
\pgfpathlineto{\pgfqpoint{1.217063in}{1.175310in}}%
\pgfpathlineto{\pgfqpoint{1.236008in}{1.192885in}}%
\pgfpathlineto{\pgfqpoint{1.245481in}{1.195557in}}%
\pgfpathlineto{\pgfqpoint{1.254953in}{1.201944in}}%
\pgfpathlineto{\pgfqpoint{1.264426in}{1.205982in}}%
\pgfpathlineto{\pgfqpoint{1.302316in}{1.229679in}}%
\pgfpathlineto{\pgfqpoint{1.321261in}{1.234236in}}%
\pgfpathlineto{\pgfqpoint{1.340206in}{1.244567in}}%
\pgfpathlineto{\pgfqpoint{1.349678in}{1.251933in}}%
\pgfpathlineto{\pgfqpoint{1.387568in}{1.268485in}}%
\pgfpathlineto{\pgfqpoint{1.397041in}{1.275067in}}%
\pgfpathlineto{\pgfqpoint{1.415986in}{1.279037in}}%
\pgfpathlineto{\pgfqpoint{1.425458in}{1.282688in}}%
\pgfpathlineto{\pgfqpoint{1.444403in}{1.294390in}}%
\pgfpathlineto{\pgfqpoint{1.463348in}{1.297869in}}%
\pgfpathlineto{\pgfqpoint{1.472821in}{1.301907in}}%
\pgfpathlineto{\pgfqpoint{1.482293in}{1.303693in}}%
\pgfpathlineto{\pgfqpoint{1.491766in}{1.310379in}}%
\pgfpathlineto{\pgfqpoint{1.501238in}{1.321268in}}%
\pgfpathlineto{\pgfqpoint{1.510711in}{1.323842in}}%
\pgfpathlineto{\pgfqpoint{1.520183in}{1.331502in}}%
\pgfpathlineto{\pgfqpoint{1.529656in}{1.333195in}}%
\pgfpathlineto{\pgfqpoint{1.539128in}{1.341540in}}%
\pgfpathlineto{\pgfqpoint{1.558073in}{1.346194in}}%
\pgfpathlineto{\pgfqpoint{1.577018in}{1.358581in}}%
\pgfpathlineto{\pgfqpoint{1.586491in}{1.360172in}}%
\pgfpathlineto{\pgfqpoint{1.595963in}{1.366857in}}%
\pgfpathlineto{\pgfqpoint{1.605436in}{1.369231in}}%
\pgfpathlineto{\pgfqpoint{1.614908in}{1.376596in}}%
\pgfpathlineto{\pgfqpoint{1.624381in}{1.376528in}}%
\pgfpathlineto{\pgfqpoint{1.662270in}{1.392101in}}%
\pgfpathlineto{\pgfqpoint{1.671743in}{1.392321in}}%
\pgfpathlineto{\pgfqpoint{1.681215in}{1.395380in}}%
\pgfpathlineto{\pgfqpoint{1.690688in}{1.402359in}}%
\pgfpathlineto{\pgfqpoint{1.700160in}{1.405614in}}%
\pgfpathlineto{\pgfqpoint{1.719105in}{1.407038in}}%
\pgfpathlineto{\pgfqpoint{1.728578in}{1.411864in}}%
\pgfpathlineto{\pgfqpoint{1.738050in}{1.412280in}}%
\pgfpathlineto{\pgfqpoint{1.756995in}{1.416347in}}%
\pgfpathlineto{\pgfqpoint{1.766468in}{1.423712in}}%
\pgfpathlineto{\pgfqpoint{1.775940in}{1.425308in}}%
\pgfpathlineto{\pgfqpoint{1.785413in}{1.430618in}}%
\pgfpathlineto{\pgfqpoint{1.804358in}{1.432434in}}%
\pgfpathlineto{\pgfqpoint{1.823303in}{1.441591in}}%
\pgfpathlineto{\pgfqpoint{1.832775in}{1.437798in}}%
\pgfpathlineto{\pgfqpoint{1.842248in}{1.447028in}}%
\pgfpathlineto{\pgfqpoint{1.851720in}{1.449402in}}%
\pgfpathlineto{\pgfqpoint{1.861193in}{1.455402in}}%
\pgfpathlineto{\pgfqpoint{1.870665in}{1.455720in}}%
\pgfpathlineto{\pgfqpoint{1.889610in}{1.462626in}}%
\pgfpathlineto{\pgfqpoint{1.908555in}{1.466203in}}%
\pgfpathlineto{\pgfqpoint{1.918028in}{1.467598in}}%
\pgfpathlineto{\pgfqpoint{1.927500in}{1.467236in}}%
\pgfpathlineto{\pgfqpoint{1.936973in}{1.473036in}}%
\pgfpathlineto{\pgfqpoint{1.946445in}{1.474533in}}%
\pgfpathlineto{\pgfqpoint{1.955918in}{1.485129in}}%
\pgfpathlineto{\pgfqpoint{1.965390in}{1.488188in}}%
\pgfpathlineto{\pgfqpoint{1.974863in}{1.493405in}}%
\pgfpathlineto{\pgfqpoint{1.984335in}{1.494800in}}%
\pgfpathlineto{\pgfqpoint{1.993808in}{1.493263in}}%
\pgfpathlineto{\pgfqpoint{2.003280in}{1.496616in}}%
\pgfpathlineto{\pgfqpoint{2.012753in}{1.502322in}}%
\pgfpathlineto{\pgfqpoint{2.031698in}{1.505900in}}%
\pgfpathlineto{\pgfqpoint{2.050643in}{1.508689in}}%
\pgfpathlineto{\pgfqpoint{2.069588in}{1.511876in}}%
\pgfpathlineto{\pgfqpoint{2.107478in}{1.513158in}}%
\pgfpathlineto{\pgfqpoint{2.126423in}{1.520063in}}%
\pgfpathlineto{\pgfqpoint{2.135895in}{1.528408in}}%
\pgfpathlineto{\pgfqpoint{2.145368in}{1.531570in}}%
\pgfpathlineto{\pgfqpoint{2.154840in}{1.532573in}}%
\pgfpathlineto{\pgfqpoint{2.173785in}{1.542904in}}%
\pgfpathlineto{\pgfqpoint{2.183258in}{1.543320in}}%
\pgfpathlineto{\pgfqpoint{2.192730in}{1.551376in}}%
\pgfpathlineto{\pgfqpoint{2.211675in}{1.560337in}}%
\pgfpathlineto{\pgfqpoint{2.221148in}{1.559481in}}%
\pgfpathlineto{\pgfqpoint{2.240093in}{1.567757in}}%
\pgfpathlineto{\pgfqpoint{2.249565in}{1.574242in}}%
\pgfpathlineto{\pgfqpoint{2.287455in}{1.579243in}}%
\pgfpathlineto{\pgfqpoint{2.296928in}{1.579958in}}%
\pgfpathlineto{\pgfqpoint{2.315873in}{1.576880in}}%
\pgfpathlineto{\pgfqpoint{2.334818in}{1.584074in}}%
\pgfpathlineto{\pgfqpoint{2.344290in}{1.589781in}}%
\pgfpathlineto{\pgfqpoint{2.353763in}{1.590882in}}%
\pgfpathlineto{\pgfqpoint{2.363235in}{1.589639in}}%
\pgfpathlineto{\pgfqpoint{2.382180in}{1.593999in}}%
\pgfpathlineto{\pgfqpoint{2.391653in}{1.597841in}}%
\pgfpathlineto{\pgfqpoint{2.401125in}{1.599530in}}%
\pgfpathlineto{\pgfqpoint{2.410598in}{1.603279in}}%
\pgfpathlineto{\pgfqpoint{2.420070in}{1.601443in}}%
\pgfpathlineto{\pgfqpoint{2.429543in}{1.606269in}}%
\pgfpathlineto{\pgfqpoint{2.439015in}{1.608447in}}%
\pgfpathlineto{\pgfqpoint{2.457960in}{1.608892in}}%
\pgfpathlineto{\pgfqpoint{2.486378in}{1.616997in}}%
\pgfpathlineto{\pgfqpoint{2.514795in}{1.622659in}}%
\pgfpathlineto{\pgfqpoint{2.524268in}{1.629829in}}%
\pgfpathlineto{\pgfqpoint{2.533740in}{1.627700in}}%
\pgfpathlineto{\pgfqpoint{2.543213in}{1.628513in}}%
\pgfpathlineto{\pgfqpoint{2.562158in}{1.637572in}}%
\pgfpathlineto{\pgfqpoint{2.571630in}{1.641414in}}%
\pgfpathlineto{\pgfqpoint{2.600048in}{1.641203in}}%
\pgfpathlineto{\pgfqpoint{2.618993in}{1.638609in}}%
\pgfpathlineto{\pgfqpoint{2.637938in}{1.635531in}}%
\pgfpathlineto{\pgfqpoint{2.647410in}{1.637224in}}%
\pgfpathlineto{\pgfqpoint{2.656883in}{1.641164in}}%
\pgfpathlineto{\pgfqpoint{2.685300in}{1.647018in}}%
\pgfpathlineto{\pgfqpoint{2.694773in}{1.646166in}}%
\pgfpathlineto{\pgfqpoint{2.704245in}{1.654315in}}%
\pgfpathlineto{\pgfqpoint{2.713718in}{1.659042in}}%
\pgfpathlineto{\pgfqpoint{2.723190in}{1.655641in}}%
\pgfpathlineto{\pgfqpoint{2.732663in}{1.664088in}}%
\pgfpathlineto{\pgfqpoint{2.751608in}{1.666780in}}%
\pgfpathlineto{\pgfqpoint{2.789498in}{1.680689in}}%
\pgfpathlineto{\pgfqpoint{2.798970in}{1.680719in}}%
\pgfpathlineto{\pgfqpoint{2.817915in}{1.684884in}}%
\pgfpathlineto{\pgfqpoint{2.827388in}{1.682069in}}%
\pgfpathlineto{\pgfqpoint{2.836860in}{1.681017in}}%
\pgfpathlineto{\pgfqpoint{2.855805in}{1.689195in}}%
\pgfpathlineto{\pgfqpoint{2.865278in}{1.694119in}}%
\pgfpathlineto{\pgfqpoint{2.893695in}{1.696351in}}%
\pgfpathlineto{\pgfqpoint{2.903168in}{1.699899in}}%
\pgfpathlineto{\pgfqpoint{2.912640in}{1.701299in}}%
\pgfpathlineto{\pgfqpoint{2.922113in}{1.700246in}}%
\pgfpathlineto{\pgfqpoint{2.931585in}{1.703408in}}%
\pgfpathlineto{\pgfqpoint{2.941058in}{1.710088in}}%
\pgfpathlineto{\pgfqpoint{2.950530in}{1.711978in}}%
\pgfpathlineto{\pgfqpoint{2.960003in}{1.720420in}}%
\pgfpathlineto{\pgfqpoint{2.978948in}{1.727717in}}%
\pgfpathlineto{\pgfqpoint{3.016838in}{1.730468in}}%
\pgfpathlineto{\pgfqpoint{3.026310in}{1.734701in}}%
\pgfpathlineto{\pgfqpoint{3.035783in}{1.735704in}}%
\pgfpathlineto{\pgfqpoint{3.045255in}{1.735049in}}%
\pgfpathlineto{\pgfqpoint{3.073673in}{1.740510in}}%
\pgfpathlineto{\pgfqpoint{3.083145in}{1.746511in}}%
\pgfpathlineto{\pgfqpoint{3.102090in}{1.754004in}}%
\pgfpathlineto{\pgfqpoint{3.121035in}{1.755814in}}%
\pgfpathlineto{\pgfqpoint{3.130508in}{1.760934in}}%
\pgfpathlineto{\pgfqpoint{3.139980in}{1.760860in}}%
\pgfpathlineto{\pgfqpoint{3.149453in}{1.767154in}}%
\pgfpathlineto{\pgfqpoint{3.168398in}{1.772592in}}%
\pgfpathlineto{\pgfqpoint{3.177870in}{1.770267in}}%
\pgfpathlineto{\pgfqpoint{3.196815in}{1.777075in}}%
\pgfpathlineto{\pgfqpoint{3.206288in}{1.780525in}}%
\pgfpathlineto{\pgfqpoint{3.215760in}{1.780946in}}%
\pgfpathlineto{\pgfqpoint{3.225233in}{1.784005in}}%
\pgfpathlineto{\pgfqpoint{3.253650in}{1.781440in}}%
\pgfpathlineto{\pgfqpoint{3.263123in}{1.783623in}}%
\pgfpathlineto{\pgfqpoint{3.272595in}{1.787759in}}%
\pgfpathlineto{\pgfqpoint{3.282068in}{1.788669in}}%
\pgfpathlineto{\pgfqpoint{3.291540in}{1.787910in}}%
\pgfpathlineto{\pgfqpoint{3.301013in}{1.792246in}}%
\pgfpathlineto{\pgfqpoint{3.310485in}{1.790411in}}%
\pgfpathlineto{\pgfqpoint{3.319958in}{1.790729in}}%
\pgfpathlineto{\pgfqpoint{3.329430in}{1.792814in}}%
\pgfpathlineto{\pgfqpoint{3.338903in}{1.799690in}}%
\pgfpathlineto{\pgfqpoint{3.367320in}{1.802514in}}%
\pgfpathlineto{\pgfqpoint{3.376793in}{1.802049in}}%
\pgfpathlineto{\pgfqpoint{3.395737in}{1.809347in}}%
\pgfpathlineto{\pgfqpoint{3.405210in}{1.811622in}}%
\pgfpathlineto{\pgfqpoint{3.414682in}{1.810379in}}%
\pgfpathlineto{\pgfqpoint{3.433627in}{1.812293in}}%
\pgfpathlineto{\pgfqpoint{3.443100in}{1.810947in}}%
\pgfpathlineto{\pgfqpoint{3.452572in}{1.812542in}}%
\pgfpathlineto{\pgfqpoint{3.490462in}{1.825174in}}%
\pgfpathlineto{\pgfqpoint{3.499935in}{1.823833in}}%
\pgfpathlineto{\pgfqpoint{3.509407in}{1.824837in}}%
\pgfpathlineto{\pgfqpoint{3.518880in}{1.823985in}}%
\pgfpathlineto{\pgfqpoint{3.528352in}{1.827435in}}%
\pgfpathlineto{\pgfqpoint{3.537825in}{1.827656in}}%
\pgfpathlineto{\pgfqpoint{3.547297in}{1.830034in}}%
\pgfpathlineto{\pgfqpoint{3.556770in}{1.829961in}}%
\pgfpathlineto{\pgfqpoint{3.566242in}{1.831654in}}%
\pgfpathlineto{\pgfqpoint{3.575715in}{1.830602in}}%
\pgfpathlineto{\pgfqpoint{3.623077in}{1.836998in}}%
\pgfpathlineto{\pgfqpoint{3.642022in}{1.843415in}}%
\pgfpathlineto{\pgfqpoint{3.651495in}{1.843836in}}%
\pgfpathlineto{\pgfqpoint{3.660967in}{1.849733in}}%
\pgfpathlineto{\pgfqpoint{3.679912in}{1.848319in}}%
\pgfpathlineto{\pgfqpoint{3.689385in}{1.853139in}}%
\pgfpathlineto{\pgfqpoint{3.698857in}{1.855616in}}%
\pgfpathlineto{\pgfqpoint{3.708330in}{1.853095in}}%
\pgfpathlineto{\pgfqpoint{3.746220in}{1.862796in}}%
\pgfpathlineto{\pgfqpoint{3.755692in}{1.862037in}}%
\pgfpathlineto{\pgfqpoint{3.765165in}{1.864122in}}%
\pgfpathlineto{\pgfqpoint{3.774637in}{1.864538in}}%
\pgfpathlineto{\pgfqpoint{3.784110in}{1.862316in}}%
\pgfpathlineto{\pgfqpoint{3.793582in}{1.861851in}}%
\pgfpathlineto{\pgfqpoint{3.803055in}{1.866187in}}%
\pgfpathlineto{\pgfqpoint{3.812527in}{1.873553in}}%
\pgfpathlineto{\pgfqpoint{3.822000in}{1.876710in}}%
\pgfpathlineto{\pgfqpoint{3.850417in}{1.876793in}}%
\pgfpathlineto{\pgfqpoint{3.859890in}{1.878775in}}%
\pgfpathlineto{\pgfqpoint{3.869362in}{1.875672in}}%
\pgfpathlineto{\pgfqpoint{3.878835in}{1.879514in}}%
\pgfpathlineto{\pgfqpoint{3.888307in}{1.876896in}}%
\pgfpathlineto{\pgfqpoint{3.897780in}{1.876827in}}%
\pgfpathlineto{\pgfqpoint{3.916725in}{1.873063in}}%
\pgfpathlineto{\pgfqpoint{3.926197in}{1.874752in}}%
\pgfpathlineto{\pgfqpoint{3.935670in}{1.872726in}}%
\pgfpathlineto{\pgfqpoint{3.945142in}{1.868639in}}%
\pgfpathlineto{\pgfqpoint{3.954615in}{1.871703in}}%
\pgfpathlineto{\pgfqpoint{3.964087in}{1.872119in}}%
\pgfpathlineto{\pgfqpoint{3.973560in}{1.877527in}}%
\pgfpathlineto{\pgfqpoint{3.983032in}{1.875794in}}%
\pgfpathlineto{\pgfqpoint{3.992505in}{1.879343in}}%
\pgfpathlineto{\pgfqpoint{4.020922in}{1.881971in}}%
\pgfpathlineto{\pgfqpoint{4.049340in}{1.881560in}}%
\pgfpathlineto{\pgfqpoint{4.058812in}{1.884716in}}%
\pgfpathlineto{\pgfqpoint{4.068285in}{1.891695in}}%
\pgfpathlineto{\pgfqpoint{4.077757in}{1.896516in}}%
\pgfpathlineto{\pgfqpoint{4.096702in}{1.896864in}}%
\pgfpathlineto{\pgfqpoint{4.134592in}{1.905781in}}%
\pgfpathlineto{\pgfqpoint{4.172482in}{1.911071in}}%
\pgfpathlineto{\pgfqpoint{4.181955in}{1.909045in}}%
\pgfpathlineto{\pgfqpoint{4.191427in}{1.910440in}}%
\pgfpathlineto{\pgfqpoint{4.200900in}{1.908218in}}%
\pgfpathlineto{\pgfqpoint{4.219845in}{1.907391in}}%
\pgfpathlineto{\pgfqpoint{4.229317in}{1.906926in}}%
\pgfpathlineto{\pgfqpoint{4.238790in}{1.903236in}}%
\pgfpathlineto{\pgfqpoint{4.248262in}{1.908252in}}%
\pgfpathlineto{\pgfqpoint{4.257735in}{1.908179in}}%
\pgfpathlineto{\pgfqpoint{4.267207in}{1.903999in}}%
\pgfpathlineto{\pgfqpoint{4.276680in}{1.901870in}}%
\pgfpathlineto{\pgfqpoint{4.286152in}{1.902683in}}%
\pgfpathlineto{\pgfqpoint{4.295625in}{1.907993in}}%
\pgfpathlineto{\pgfqpoint{4.305097in}{1.903226in}}%
\pgfpathlineto{\pgfqpoint{4.314570in}{1.903544in}}%
\pgfpathlineto{\pgfqpoint{4.324042in}{1.908267in}}%
\pgfpathlineto{\pgfqpoint{4.333515in}{1.911135in}}%
\pgfpathlineto{\pgfqpoint{4.342987in}{1.910670in}}%
\pgfpathlineto{\pgfqpoint{4.352460in}{1.915006in}}%
\pgfpathlineto{\pgfqpoint{4.361932in}{1.916401in}}%
\pgfpathlineto{\pgfqpoint{4.380877in}{1.914987in}}%
\pgfpathlineto{\pgfqpoint{4.390350in}{1.915696in}}%
\pgfpathlineto{\pgfqpoint{4.409295in}{1.920253in}}%
\pgfpathlineto{\pgfqpoint{4.418767in}{1.925861in}}%
\pgfpathlineto{\pgfqpoint{4.447185in}{1.929561in}}%
\pgfpathlineto{\pgfqpoint{4.456657in}{1.927335in}}%
\pgfpathlineto{\pgfqpoint{4.466130in}{1.927462in}}%
\pgfpathlineto{\pgfqpoint{4.513492in}{1.917610in}}%
\pgfpathlineto{\pgfqpoint{4.522965in}{1.916758in}}%
\pgfpathlineto{\pgfqpoint{4.532437in}{1.917664in}}%
\pgfpathlineto{\pgfqpoint{4.551382in}{1.925059in}}%
\pgfpathlineto{\pgfqpoint{4.560855in}{1.923811in}}%
\pgfpathlineto{\pgfqpoint{4.579800in}{1.927193in}}%
\pgfpathlineto{\pgfqpoint{4.589272in}{1.928103in}}%
\pgfpathlineto{\pgfqpoint{4.617690in}{1.923483in}}%
\pgfpathlineto{\pgfqpoint{4.636635in}{1.924711in}}%
\pgfpathlineto{\pgfqpoint{4.646107in}{1.926596in}}%
\pgfpathlineto{\pgfqpoint{4.674525in}{1.940084in}}%
\pgfpathlineto{\pgfqpoint{4.693470in}{1.940823in}}%
\pgfpathlineto{\pgfqpoint{4.750305in}{1.955080in}}%
\pgfpathlineto{\pgfqpoint{4.759777in}{1.953636in}}%
\pgfpathlineto{\pgfqpoint{4.778722in}{1.966512in}}%
\pgfpathlineto{\pgfqpoint{4.788195in}{1.968108in}}%
\pgfpathlineto{\pgfqpoint{4.797667in}{1.960791in}}%
\pgfpathlineto{\pgfqpoint{4.807140in}{1.967183in}}%
\pgfpathlineto{\pgfqpoint{4.826085in}{1.965861in}}%
\pgfpathlineto{\pgfqpoint{4.835557in}{1.963150in}}%
\pgfpathlineto{\pgfqpoint{4.845030in}{1.966209in}}%
\pgfpathlineto{\pgfqpoint{4.854502in}{1.965553in}}%
\pgfpathlineto{\pgfqpoint{4.882920in}{1.976301in}}%
\pgfpathlineto{\pgfqpoint{4.892392in}{1.980436in}}%
\pgfpathlineto{\pgfqpoint{4.901865in}{1.978312in}}%
\pgfpathlineto{\pgfqpoint{4.920810in}{1.977583in}}%
\pgfpathlineto{\pgfqpoint{4.939755in}{1.983803in}}%
\pgfpathlineto{\pgfqpoint{4.949227in}{1.980206in}}%
\pgfpathlineto{\pgfqpoint{4.958700in}{1.985223in}}%
\pgfpathlineto{\pgfqpoint{4.977645in}{1.985179in}}%
\pgfpathlineto{\pgfqpoint{4.987117in}{1.988340in}}%
\pgfpathlineto{\pgfqpoint{4.996590in}{1.986505in}}%
\pgfpathlineto{\pgfqpoint{5.015535in}{1.998403in}}%
\pgfpathlineto{\pgfqpoint{5.034480in}{1.998554in}}%
\pgfpathlineto{\pgfqpoint{5.062897in}{2.006757in}}%
\pgfpathlineto{\pgfqpoint{5.072370in}{2.006493in}}%
\pgfpathlineto{\pgfqpoint{5.081842in}{2.009943in}}%
\pgfpathlineto{\pgfqpoint{5.091315in}{2.010755in}}%
\pgfpathlineto{\pgfqpoint{5.100787in}{2.009703in}}%
\pgfpathlineto{\pgfqpoint{5.110260in}{2.013349in}}%
\pgfpathlineto{\pgfqpoint{5.119732in}{2.014553in}}%
\pgfpathlineto{\pgfqpoint{5.148149in}{2.006311in}}%
\pgfpathlineto{\pgfqpoint{5.157622in}{2.008494in}}%
\pgfpathlineto{\pgfqpoint{5.167094in}{2.007834in}}%
\pgfpathlineto{\pgfqpoint{5.176567in}{2.010403in}}%
\pgfpathlineto{\pgfqpoint{5.195512in}{2.007814in}}%
\pgfpathlineto{\pgfqpoint{5.204984in}{2.007843in}}%
\pgfpathlineto{\pgfqpoint{5.214457in}{2.012468in}}%
\pgfpathlineto{\pgfqpoint{5.233402in}{2.017123in}}%
\pgfpathlineto{\pgfqpoint{5.242874in}{2.017832in}}%
\pgfpathlineto{\pgfqpoint{5.252347in}{2.016393in}}%
\pgfpathlineto{\pgfqpoint{5.280764in}{2.020974in}}%
\pgfpathlineto{\pgfqpoint{5.290237in}{2.025996in}}%
\pgfpathlineto{\pgfqpoint{5.299709in}{2.020637in}}%
\pgfpathlineto{\pgfqpoint{5.318654in}{2.020984in}}%
\pgfpathlineto{\pgfqpoint{5.328127in}{2.025315in}}%
\pgfpathlineto{\pgfqpoint{5.356544in}{2.020994in}}%
\pgfpathlineto{\pgfqpoint{5.375489in}{2.024963in}}%
\pgfpathlineto{\pgfqpoint{5.384962in}{2.024204in}}%
\pgfpathlineto{\pgfqpoint{5.403907in}{2.027586in}}%
\pgfpathlineto{\pgfqpoint{5.413379in}{2.025066in}}%
\pgfpathlineto{\pgfqpoint{5.432324in}{2.022379in}}%
\pgfpathlineto{\pgfqpoint{5.441797in}{2.021527in}}%
\pgfpathlineto{\pgfqpoint{5.451269in}{2.024782in}}%
\pgfpathlineto{\pgfqpoint{5.460742in}{2.033126in}}%
\pgfpathlineto{\pgfqpoint{5.470214in}{2.029632in}}%
\pgfpathlineto{\pgfqpoint{5.489159in}{2.026162in}}%
\pgfpathlineto{\pgfqpoint{5.489159in}{2.026162in}}%
\pgfusepath{stroke}%
\end{pgfscope}%
\begin{pgfscope}%
\pgfpathrectangle{\pgfqpoint{0.762383in}{0.471179in}}{\pgfqpoint{4.726776in}{2.845920in}} %
\pgfusepath{clip}%
\pgfsetbuttcap%
\pgfsetroundjoin%
\pgfsetlinewidth{1.505625pt}%
\definecolor{currentstroke}{rgb}{1.000000,0.498039,0.054902}%
\pgfsetstrokecolor{currentstroke}%
\pgfsetdash{{5.550000pt}{2.400000pt}}{0.000000pt}%
\pgfpathmoveto{\pgfqpoint{0.762383in}{0.603279in}}%
\pgfpathlineto{\pgfqpoint{0.781328in}{0.687414in}}%
\pgfpathlineto{\pgfqpoint{0.800273in}{0.743554in}}%
\pgfpathlineto{\pgfqpoint{0.819218in}{0.788927in}}%
\pgfpathlineto{\pgfqpoint{0.828691in}{0.806377in}}%
\pgfpathlineto{\pgfqpoint{0.838163in}{0.828134in}}%
\pgfpathlineto{\pgfqpoint{0.857108in}{0.855986in}}%
\pgfpathlineto{\pgfqpoint{0.894998in}{0.922261in}}%
\pgfpathlineto{\pgfqpoint{0.904471in}{0.934523in}}%
\pgfpathlineto{\pgfqpoint{0.913943in}{0.942086in}}%
\pgfpathlineto{\pgfqpoint{0.923416in}{0.956208in}}%
\pgfpathlineto{\pgfqpoint{0.942361in}{0.973685in}}%
\pgfpathlineto{\pgfqpoint{0.970778in}{1.008708in}}%
\pgfpathlineto{\pgfqpoint{0.980251in}{1.013529in}}%
\pgfpathlineto{\pgfqpoint{0.989723in}{1.023053in}}%
\pgfpathlineto{\pgfqpoint{1.027613in}{1.052624in}}%
\pgfpathlineto{\pgfqpoint{1.037086in}{1.056172in}}%
\pgfpathlineto{\pgfqpoint{1.065503in}{1.085811in}}%
\pgfpathlineto{\pgfqpoint{1.103393in}{1.119982in}}%
\pgfpathlineto{\pgfqpoint{1.122338in}{1.127475in}}%
\pgfpathlineto{\pgfqpoint{1.141283in}{1.141722in}}%
\pgfpathlineto{\pgfqpoint{1.150756in}{1.141942in}}%
\pgfpathlineto{\pgfqpoint{1.160228in}{1.149410in}}%
\pgfpathlineto{\pgfqpoint{1.179173in}{1.171879in}}%
\pgfpathlineto{\pgfqpoint{1.188646in}{1.176406in}}%
\pgfpathlineto{\pgfqpoint{1.198118in}{1.178780in}}%
\pgfpathlineto{\pgfqpoint{1.207591in}{1.183410in}}%
\pgfpathlineto{\pgfqpoint{1.217063in}{1.184022in}}%
\pgfpathlineto{\pgfqpoint{1.245481in}{1.201039in}}%
\pgfpathlineto{\pgfqpoint{1.264426in}{1.217042in}}%
\pgfpathlineto{\pgfqpoint{1.273898in}{1.223826in}}%
\pgfpathlineto{\pgfqpoint{1.292843in}{1.241205in}}%
\pgfpathlineto{\pgfqpoint{1.302316in}{1.245732in}}%
\pgfpathlineto{\pgfqpoint{1.311788in}{1.245370in}}%
\pgfpathlineto{\pgfqpoint{1.321261in}{1.247156in}}%
\pgfpathlineto{\pgfqpoint{1.330733in}{1.250998in}}%
\pgfpathlineto{\pgfqpoint{1.340206in}{1.261599in}}%
\pgfpathlineto{\pgfqpoint{1.359151in}{1.274965in}}%
\pgfpathlineto{\pgfqpoint{1.387568in}{1.289530in}}%
\pgfpathlineto{\pgfqpoint{1.463348in}{1.320872in}}%
\pgfpathlineto{\pgfqpoint{1.482293in}{1.331786in}}%
\pgfpathlineto{\pgfqpoint{1.491766in}{1.338765in}}%
\pgfpathlineto{\pgfqpoint{1.501238in}{1.342411in}}%
\pgfpathlineto{\pgfqpoint{1.520183in}{1.360671in}}%
\pgfpathlineto{\pgfqpoint{1.529656in}{1.362462in}}%
\pgfpathlineto{\pgfqpoint{1.539128in}{1.369436in}}%
\pgfpathlineto{\pgfqpoint{1.548601in}{1.370244in}}%
\pgfpathlineto{\pgfqpoint{1.577018in}{1.381877in}}%
\pgfpathlineto{\pgfqpoint{1.586491in}{1.382978in}}%
\pgfpathlineto{\pgfqpoint{1.595963in}{1.393286in}}%
\pgfpathlineto{\pgfqpoint{1.614908in}{1.403710in}}%
\pgfpathlineto{\pgfqpoint{1.624381in}{1.403642in}}%
\pgfpathlineto{\pgfqpoint{1.643325in}{1.416224in}}%
\pgfpathlineto{\pgfqpoint{1.652798in}{1.418598in}}%
\pgfpathlineto{\pgfqpoint{1.690688in}{1.434465in}}%
\pgfpathlineto{\pgfqpoint{1.700160in}{1.437524in}}%
\pgfpathlineto{\pgfqpoint{1.728578in}{1.450821in}}%
\pgfpathlineto{\pgfqpoint{1.766468in}{1.459538in}}%
\pgfpathlineto{\pgfqpoint{1.775940in}{1.459469in}}%
\pgfpathlineto{\pgfqpoint{1.785413in}{1.467520in}}%
\pgfpathlineto{\pgfqpoint{1.804358in}{1.469923in}}%
\pgfpathlineto{\pgfqpoint{1.813830in}{1.478860in}}%
\pgfpathlineto{\pgfqpoint{1.832775in}{1.477930in}}%
\pgfpathlineto{\pgfqpoint{1.842248in}{1.484224in}}%
\pgfpathlineto{\pgfqpoint{1.851720in}{1.487674in}}%
\pgfpathlineto{\pgfqpoint{1.861193in}{1.495828in}}%
\pgfpathlineto{\pgfqpoint{1.899083in}{1.498475in}}%
\pgfpathlineto{\pgfqpoint{1.918028in}{1.505479in}}%
\pgfpathlineto{\pgfqpoint{1.927500in}{1.509326in}}%
\pgfpathlineto{\pgfqpoint{1.955918in}{1.531428in}}%
\pgfpathlineto{\pgfqpoint{1.993808in}{1.542009in}}%
\pgfpathlineto{\pgfqpoint{2.003280in}{1.539586in}}%
\pgfpathlineto{\pgfqpoint{2.012753in}{1.546369in}}%
\pgfpathlineto{\pgfqpoint{2.022225in}{1.544926in}}%
\pgfpathlineto{\pgfqpoint{2.031698in}{1.547304in}}%
\pgfpathlineto{\pgfqpoint{2.041170in}{1.547524in}}%
\pgfpathlineto{\pgfqpoint{2.050643in}{1.554694in}}%
\pgfpathlineto{\pgfqpoint{2.060115in}{1.555703in}}%
\pgfpathlineto{\pgfqpoint{2.069588in}{1.555140in}}%
\pgfpathlineto{\pgfqpoint{2.107478in}{1.563372in}}%
\pgfpathlineto{\pgfqpoint{2.126423in}{1.569103in}}%
\pgfpathlineto{\pgfqpoint{2.135895in}{1.575881in}}%
\pgfpathlineto{\pgfqpoint{2.154840in}{1.585625in}}%
\pgfpathlineto{\pgfqpoint{2.164313in}{1.587319in}}%
\pgfpathlineto{\pgfqpoint{2.202203in}{1.605236in}}%
\pgfpathlineto{\pgfqpoint{2.211675in}{1.610943in}}%
\pgfpathlineto{\pgfqpoint{2.221148in}{1.609891in}}%
\pgfpathlineto{\pgfqpoint{2.230620in}{1.613248in}}%
\pgfpathlineto{\pgfqpoint{2.249565in}{1.624162in}}%
\pgfpathlineto{\pgfqpoint{2.268510in}{1.624412in}}%
\pgfpathlineto{\pgfqpoint{2.277983in}{1.628454in}}%
\pgfpathlineto{\pgfqpoint{2.287455in}{1.625542in}}%
\pgfpathlineto{\pgfqpoint{2.296928in}{1.628116in}}%
\pgfpathlineto{\pgfqpoint{2.306400in}{1.626477in}}%
\pgfpathlineto{\pgfqpoint{2.315873in}{1.627583in}}%
\pgfpathlineto{\pgfqpoint{2.344290in}{1.645867in}}%
\pgfpathlineto{\pgfqpoint{2.353763in}{1.646479in}}%
\pgfpathlineto{\pgfqpoint{2.363235in}{1.642691in}}%
\pgfpathlineto{\pgfqpoint{2.372708in}{1.646044in}}%
\pgfpathlineto{\pgfqpoint{2.382180in}{1.652729in}}%
\pgfpathlineto{\pgfqpoint{2.391653in}{1.652166in}}%
\pgfpathlineto{\pgfqpoint{2.401125in}{1.656889in}}%
\pgfpathlineto{\pgfqpoint{2.420070in}{1.659977in}}%
\pgfpathlineto{\pgfqpoint{2.429543in}{1.671459in}}%
\pgfpathlineto{\pgfqpoint{2.448488in}{1.668381in}}%
\pgfpathlineto{\pgfqpoint{2.457960in}{1.673397in}}%
\pgfpathlineto{\pgfqpoint{2.467433in}{1.674792in}}%
\pgfpathlineto{\pgfqpoint{2.486378in}{1.680523in}}%
\pgfpathlineto{\pgfqpoint{2.495850in}{1.678790in}}%
\pgfpathlineto{\pgfqpoint{2.524268in}{1.694041in}}%
\pgfpathlineto{\pgfqpoint{2.533740in}{1.693869in}}%
\pgfpathlineto{\pgfqpoint{2.543213in}{1.697227in}}%
\pgfpathlineto{\pgfqpoint{2.562158in}{1.708243in}}%
\pgfpathlineto{\pgfqpoint{2.590575in}{1.716054in}}%
\pgfpathlineto{\pgfqpoint{2.600048in}{1.717063in}}%
\pgfpathlineto{\pgfqpoint{2.628465in}{1.714400in}}%
\pgfpathlineto{\pgfqpoint{2.656883in}{1.718883in}}%
\pgfpathlineto{\pgfqpoint{2.666355in}{1.716172in}}%
\pgfpathlineto{\pgfqpoint{2.675828in}{1.719818in}}%
\pgfpathlineto{\pgfqpoint{2.694773in}{1.723885in}}%
\pgfpathlineto{\pgfqpoint{2.704245in}{1.731838in}}%
\pgfpathlineto{\pgfqpoint{2.713718in}{1.735195in}}%
\pgfpathlineto{\pgfqpoint{2.723190in}{1.736394in}}%
\pgfpathlineto{\pgfqpoint{2.751608in}{1.744010in}}%
\pgfpathlineto{\pgfqpoint{2.761080in}{1.744822in}}%
\pgfpathlineto{\pgfqpoint{2.789498in}{1.755863in}}%
\pgfpathlineto{\pgfqpoint{2.817915in}{1.758589in}}%
\pgfpathlineto{\pgfqpoint{2.827388in}{1.759299in}}%
\pgfpathlineto{\pgfqpoint{2.836860in}{1.762847in}}%
\pgfpathlineto{\pgfqpoint{2.846333in}{1.761017in}}%
\pgfpathlineto{\pgfqpoint{2.855805in}{1.761139in}}%
\pgfpathlineto{\pgfqpoint{2.931585in}{1.776820in}}%
\pgfpathlineto{\pgfqpoint{2.941058in}{1.783011in}}%
\pgfpathlineto{\pgfqpoint{2.960003in}{1.787568in}}%
\pgfpathlineto{\pgfqpoint{2.978948in}{1.797116in}}%
\pgfpathlineto{\pgfqpoint{2.988420in}{1.797924in}}%
\pgfpathlineto{\pgfqpoint{2.997893in}{1.795408in}}%
\pgfpathlineto{\pgfqpoint{3.007365in}{1.800229in}}%
\pgfpathlineto{\pgfqpoint{3.016838in}{1.800454in}}%
\pgfpathlineto{\pgfqpoint{3.035783in}{1.809704in}}%
\pgfpathlineto{\pgfqpoint{3.045255in}{1.810614in}}%
\pgfpathlineto{\pgfqpoint{3.064200in}{1.817422in}}%
\pgfpathlineto{\pgfqpoint{3.073673in}{1.815782in}}%
\pgfpathlineto{\pgfqpoint{3.083145in}{1.820706in}}%
\pgfpathlineto{\pgfqpoint{3.092618in}{1.822884in}}%
\pgfpathlineto{\pgfqpoint{3.111563in}{1.829398in}}%
\pgfpathlineto{\pgfqpoint{3.121035in}{1.836568in}}%
\pgfpathlineto{\pgfqpoint{3.130508in}{1.837968in}}%
\pgfpathlineto{\pgfqpoint{3.139980in}{1.843376in}}%
\pgfpathlineto{\pgfqpoint{3.149453in}{1.843307in}}%
\pgfpathlineto{\pgfqpoint{3.158925in}{1.851162in}}%
\pgfpathlineto{\pgfqpoint{3.168398in}{1.848842in}}%
\pgfpathlineto{\pgfqpoint{3.177870in}{1.849258in}}%
\pgfpathlineto{\pgfqpoint{3.187343in}{1.852807in}}%
\pgfpathlineto{\pgfqpoint{3.196815in}{1.851759in}}%
\pgfpathlineto{\pgfqpoint{3.206288in}{1.852665in}}%
\pgfpathlineto{\pgfqpoint{3.215760in}{1.851911in}}%
\pgfpathlineto{\pgfqpoint{3.244178in}{1.855905in}}%
\pgfpathlineto{\pgfqpoint{3.253650in}{1.853874in}}%
\pgfpathlineto{\pgfqpoint{3.263123in}{1.858503in}}%
\pgfpathlineto{\pgfqpoint{3.272595in}{1.860192in}}%
\pgfpathlineto{\pgfqpoint{3.301013in}{1.872217in}}%
\pgfpathlineto{\pgfqpoint{3.310485in}{1.871850in}}%
\pgfpathlineto{\pgfqpoint{3.319958in}{1.874125in}}%
\pgfpathlineto{\pgfqpoint{3.329430in}{1.872001in}}%
\pgfpathlineto{\pgfqpoint{3.338903in}{1.879465in}}%
\pgfpathlineto{\pgfqpoint{3.357848in}{1.886469in}}%
\pgfpathlineto{\pgfqpoint{3.367320in}{1.885421in}}%
\pgfpathlineto{\pgfqpoint{3.376793in}{1.889557in}}%
\pgfpathlineto{\pgfqpoint{3.386265in}{1.889684in}}%
\pgfpathlineto{\pgfqpoint{3.395737in}{1.894407in}}%
\pgfpathlineto{\pgfqpoint{3.405210in}{1.896193in}}%
\pgfpathlineto{\pgfqpoint{3.414682in}{1.900725in}}%
\pgfpathlineto{\pgfqpoint{3.424155in}{1.902805in}}%
\pgfpathlineto{\pgfqpoint{3.433627in}{1.901464in}}%
\pgfpathlineto{\pgfqpoint{3.452572in}{1.906706in}}%
\pgfpathlineto{\pgfqpoint{3.462045in}{1.909863in}}%
\pgfpathlineto{\pgfqpoint{3.471517in}{1.911257in}}%
\pgfpathlineto{\pgfqpoint{3.480990in}{1.910602in}}%
\pgfpathlineto{\pgfqpoint{3.490462in}{1.911996in}}%
\pgfpathlineto{\pgfqpoint{3.499935in}{1.910460in}}%
\pgfpathlineto{\pgfqpoint{3.509407in}{1.914106in}}%
\pgfpathlineto{\pgfqpoint{3.518880in}{1.915603in}}%
\pgfpathlineto{\pgfqpoint{3.528352in}{1.915530in}}%
\pgfpathlineto{\pgfqpoint{3.537825in}{1.920057in}}%
\pgfpathlineto{\pgfqpoint{3.556770in}{1.918153in}}%
\pgfpathlineto{\pgfqpoint{3.566242in}{1.919553in}}%
\pgfpathlineto{\pgfqpoint{3.594660in}{1.929517in}}%
\pgfpathlineto{\pgfqpoint{3.604132in}{1.929346in}}%
\pgfpathlineto{\pgfqpoint{3.632550in}{1.933736in}}%
\pgfpathlineto{\pgfqpoint{3.651495in}{1.927917in}}%
\pgfpathlineto{\pgfqpoint{3.660967in}{1.930584in}}%
\pgfpathlineto{\pgfqpoint{3.679912in}{1.933085in}}%
\pgfpathlineto{\pgfqpoint{3.727275in}{1.940069in}}%
\pgfpathlineto{\pgfqpoint{3.736747in}{1.945384in}}%
\pgfpathlineto{\pgfqpoint{3.746220in}{1.945017in}}%
\pgfpathlineto{\pgfqpoint{3.755692in}{1.942301in}}%
\pgfpathlineto{\pgfqpoint{3.765165in}{1.951238in}}%
\pgfpathlineto{\pgfqpoint{3.774637in}{1.954101in}}%
\pgfpathlineto{\pgfqpoint{3.793582in}{1.950827in}}%
\pgfpathlineto{\pgfqpoint{3.803055in}{1.945864in}}%
\pgfpathlineto{\pgfqpoint{3.812527in}{1.953719in}}%
\pgfpathlineto{\pgfqpoint{3.831472in}{1.963365in}}%
\pgfpathlineto{\pgfqpoint{3.840945in}{1.965250in}}%
\pgfpathlineto{\pgfqpoint{3.850417in}{1.964692in}}%
\pgfpathlineto{\pgfqpoint{3.859890in}{1.967848in}}%
\pgfpathlineto{\pgfqpoint{3.869362in}{1.963179in}}%
\pgfpathlineto{\pgfqpoint{3.878835in}{1.966923in}}%
\pgfpathlineto{\pgfqpoint{3.888307in}{1.966067in}}%
\pgfpathlineto{\pgfqpoint{3.897780in}{1.968739in}}%
\pgfpathlineto{\pgfqpoint{3.907252in}{1.969253in}}%
\pgfpathlineto{\pgfqpoint{3.916725in}{1.963703in}}%
\pgfpathlineto{\pgfqpoint{3.926197in}{1.966077in}}%
\pgfpathlineto{\pgfqpoint{3.935670in}{1.966400in}}%
\pgfpathlineto{\pgfqpoint{3.945142in}{1.964271in}}%
\pgfpathlineto{\pgfqpoint{3.954615in}{1.969488in}}%
\pgfpathlineto{\pgfqpoint{3.964087in}{1.968925in}}%
\pgfpathlineto{\pgfqpoint{3.973560in}{1.974529in}}%
\pgfpathlineto{\pgfqpoint{3.983032in}{1.974362in}}%
\pgfpathlineto{\pgfqpoint{3.992505in}{1.978009in}}%
\pgfpathlineto{\pgfqpoint{4.001977in}{1.974906in}}%
\pgfpathlineto{\pgfqpoint{4.020922in}{1.981028in}}%
\pgfpathlineto{\pgfqpoint{4.030395in}{1.978214in}}%
\pgfpathlineto{\pgfqpoint{4.039867in}{1.981567in}}%
\pgfpathlineto{\pgfqpoint{4.058812in}{1.981914in}}%
\pgfpathlineto{\pgfqpoint{4.068285in}{1.988012in}}%
\pgfpathlineto{\pgfqpoint{4.077757in}{1.991658in}}%
\pgfpathlineto{\pgfqpoint{4.096702in}{1.990244in}}%
\pgfpathlineto{\pgfqpoint{4.115647in}{1.991766in}}%
\pgfpathlineto{\pgfqpoint{4.125120in}{1.998544in}}%
\pgfpathlineto{\pgfqpoint{4.144065in}{1.999186in}}%
\pgfpathlineto{\pgfqpoint{4.163010in}{2.005993in}}%
\pgfpathlineto{\pgfqpoint{4.181955in}{2.001740in}}%
\pgfpathlineto{\pgfqpoint{4.200900in}{2.007569in}}%
\pgfpathlineto{\pgfqpoint{4.210372in}{2.007398in}}%
\pgfpathlineto{\pgfqpoint{4.219845in}{2.011147in}}%
\pgfpathlineto{\pgfqpoint{4.229317in}{2.006081in}}%
\pgfpathlineto{\pgfqpoint{4.238790in}{2.004740in}}%
\pgfpathlineto{\pgfqpoint{4.257735in}{2.009292in}}%
\pgfpathlineto{\pgfqpoint{4.267207in}{2.008734in}}%
\pgfpathlineto{\pgfqpoint{4.286152in}{2.001838in}}%
\pgfpathlineto{\pgfqpoint{4.305097in}{2.007080in}}%
\pgfpathlineto{\pgfqpoint{4.314570in}{2.005147in}}%
\pgfpathlineto{\pgfqpoint{4.342987in}{2.012762in}}%
\pgfpathlineto{\pgfqpoint{4.352460in}{2.011812in}}%
\pgfpathlineto{\pgfqpoint{4.361932in}{2.013403in}}%
\pgfpathlineto{\pgfqpoint{4.371405in}{2.010985in}}%
\pgfpathlineto{\pgfqpoint{4.380877in}{2.012087in}}%
\pgfpathlineto{\pgfqpoint{4.399822in}{2.018013in}}%
\pgfpathlineto{\pgfqpoint{4.409295in}{2.019310in}}%
\pgfpathlineto{\pgfqpoint{4.418767in}{2.025213in}}%
\pgfpathlineto{\pgfqpoint{4.428240in}{2.028663in}}%
\pgfpathlineto{\pgfqpoint{4.437712in}{2.029378in}}%
\pgfpathlineto{\pgfqpoint{4.466130in}{2.022213in}}%
\pgfpathlineto{\pgfqpoint{4.494547in}{2.024248in}}%
\pgfpathlineto{\pgfqpoint{4.513492in}{2.020093in}}%
\pgfpathlineto{\pgfqpoint{4.522965in}{2.024430in}}%
\pgfpathlineto{\pgfqpoint{4.532437in}{2.025531in}}%
\pgfpathlineto{\pgfqpoint{4.541910in}{2.028981in}}%
\pgfpathlineto{\pgfqpoint{4.551382in}{2.028717in}}%
\pgfpathlineto{\pgfqpoint{4.570327in}{2.036993in}}%
\pgfpathlineto{\pgfqpoint{4.579800in}{2.040541in}}%
\pgfpathlineto{\pgfqpoint{4.589272in}{2.046150in}}%
\pgfpathlineto{\pgfqpoint{4.598745in}{2.044119in}}%
\pgfpathlineto{\pgfqpoint{4.608217in}{2.037781in}}%
\pgfpathlineto{\pgfqpoint{4.617690in}{2.034776in}}%
\pgfpathlineto{\pgfqpoint{4.627162in}{2.037149in}}%
\pgfpathlineto{\pgfqpoint{4.636635in}{2.044129in}}%
\pgfpathlineto{\pgfqpoint{4.646107in}{2.044349in}}%
\pgfpathlineto{\pgfqpoint{4.655580in}{2.046140in}}%
\pgfpathlineto{\pgfqpoint{4.665052in}{2.046262in}}%
\pgfpathlineto{\pgfqpoint{4.693470in}{2.050452in}}%
\pgfpathlineto{\pgfqpoint{4.712415in}{2.050897in}}%
\pgfpathlineto{\pgfqpoint{4.721887in}{2.053569in}}%
\pgfpathlineto{\pgfqpoint{4.740832in}{2.051372in}}%
\pgfpathlineto{\pgfqpoint{4.750305in}{2.053746in}}%
\pgfpathlineto{\pgfqpoint{4.759777in}{2.065222in}}%
\pgfpathlineto{\pgfqpoint{4.769250in}{2.063490in}}%
\pgfpathlineto{\pgfqpoint{4.778722in}{2.067821in}}%
\pgfpathlineto{\pgfqpoint{4.816612in}{2.075074in}}%
\pgfpathlineto{\pgfqpoint{4.826085in}{2.077937in}}%
\pgfpathlineto{\pgfqpoint{4.845030in}{2.074565in}}%
\pgfpathlineto{\pgfqpoint{4.854502in}{2.075182in}}%
\pgfpathlineto{\pgfqpoint{4.863975in}{2.073934in}}%
\pgfpathlineto{\pgfqpoint{4.873447in}{2.075236in}}%
\pgfpathlineto{\pgfqpoint{4.882920in}{2.080839in}}%
\pgfpathlineto{\pgfqpoint{4.892392in}{2.088597in}}%
\pgfpathlineto{\pgfqpoint{4.911337in}{2.089140in}}%
\pgfpathlineto{\pgfqpoint{4.920810in}{2.090540in}}%
\pgfpathlineto{\pgfqpoint{4.930282in}{2.094773in}}%
\pgfpathlineto{\pgfqpoint{4.939755in}{2.097152in}}%
\pgfpathlineto{\pgfqpoint{4.949227in}{2.095316in}}%
\pgfpathlineto{\pgfqpoint{4.968172in}{2.103984in}}%
\pgfpathlineto{\pgfqpoint{4.977645in}{2.103715in}}%
\pgfpathlineto{\pgfqpoint{4.987117in}{2.101689in}}%
\pgfpathlineto{\pgfqpoint{4.996590in}{2.107488in}}%
\pgfpathlineto{\pgfqpoint{5.006062in}{2.106637in}}%
\pgfpathlineto{\pgfqpoint{5.015535in}{2.111261in}}%
\pgfpathlineto{\pgfqpoint{5.025007in}{2.110606in}}%
\pgfpathlineto{\pgfqpoint{5.043952in}{2.112612in}}%
\pgfpathlineto{\pgfqpoint{5.053425in}{2.119493in}}%
\pgfpathlineto{\pgfqpoint{5.062897in}{2.120595in}}%
\pgfpathlineto{\pgfqpoint{5.072370in}{2.123658in}}%
\pgfpathlineto{\pgfqpoint{5.081842in}{2.124662in}}%
\pgfpathlineto{\pgfqpoint{5.091315in}{2.123223in}}%
\pgfpathlineto{\pgfqpoint{5.100787in}{2.125205in}}%
\pgfpathlineto{\pgfqpoint{5.110260in}{2.122782in}}%
\pgfpathlineto{\pgfqpoint{5.119732in}{2.124671in}}%
\pgfpathlineto{\pgfqpoint{5.129205in}{2.124207in}}%
\pgfpathlineto{\pgfqpoint{5.138677in}{2.118852in}}%
\pgfpathlineto{\pgfqpoint{5.148149in}{2.117409in}}%
\pgfpathlineto{\pgfqpoint{5.157622in}{2.121647in}}%
\pgfpathlineto{\pgfqpoint{5.167094in}{2.118148in}}%
\pgfpathlineto{\pgfqpoint{5.176567in}{2.121011in}}%
\pgfpathlineto{\pgfqpoint{5.204984in}{2.124618in}}%
\pgfpathlineto{\pgfqpoint{5.223929in}{2.121148in}}%
\pgfpathlineto{\pgfqpoint{5.233402in}{2.126066in}}%
\pgfpathlineto{\pgfqpoint{5.242874in}{2.126678in}}%
\pgfpathlineto{\pgfqpoint{5.252347in}{2.125141in}}%
\pgfpathlineto{\pgfqpoint{5.280764in}{2.127569in}}%
\pgfpathlineto{\pgfqpoint{5.290237in}{2.125347in}}%
\pgfpathlineto{\pgfqpoint{5.299709in}{2.128308in}}%
\pgfpathlineto{\pgfqpoint{5.309182in}{2.126184in}}%
\pgfpathlineto{\pgfqpoint{5.328127in}{2.129169in}}%
\pgfpathlineto{\pgfqpoint{5.337599in}{2.125283in}}%
\pgfpathlineto{\pgfqpoint{5.347072in}{2.128440in}}%
\pgfpathlineto{\pgfqpoint{5.356544in}{2.127099in}}%
\pgfpathlineto{\pgfqpoint{5.366017in}{2.134563in}}%
\pgfpathlineto{\pgfqpoint{5.375489in}{2.134788in}}%
\pgfpathlineto{\pgfqpoint{5.384962in}{2.138630in}}%
\pgfpathlineto{\pgfqpoint{5.394434in}{2.135718in}}%
\pgfpathlineto{\pgfqpoint{5.403907in}{2.140054in}}%
\pgfpathlineto{\pgfqpoint{5.422852in}{2.138737in}}%
\pgfpathlineto{\pgfqpoint{5.432324in}{2.135629in}}%
\pgfpathlineto{\pgfqpoint{5.441797in}{2.130177in}}%
\pgfpathlineto{\pgfqpoint{5.451269in}{2.136564in}}%
\pgfpathlineto{\pgfqpoint{5.460742in}{2.135904in}}%
\pgfpathlineto{\pgfqpoint{5.470214in}{2.131332in}}%
\pgfpathlineto{\pgfqpoint{5.489159in}{2.131093in}}%
\pgfpathlineto{\pgfqpoint{5.489159in}{2.131093in}}%
\pgfusepath{stroke}%
\end{pgfscope}%
\begin{pgfscope}%
\pgfpathrectangle{\pgfqpoint{0.762383in}{0.471179in}}{\pgfqpoint{4.726776in}{2.845920in}} %
\pgfusepath{clip}%
\pgfsetbuttcap%
\pgfsetroundjoin%
\pgfsetlinewidth{1.505625pt}%
\definecolor{currentstroke}{rgb}{0.172549,0.627451,0.172549}%
\pgfsetstrokecolor{currentstroke}%
\pgfsetdash{{9.600000pt}{2.400000pt}{1.500000pt}{2.400000pt}}{0.000000pt}%
\pgfpathmoveto{\pgfqpoint{0.762383in}{0.600539in}}%
\pgfpathlineto{\pgfqpoint{0.800273in}{0.763718in}}%
\pgfpathlineto{\pgfqpoint{0.828691in}{0.860017in}}%
\pgfpathlineto{\pgfqpoint{0.838163in}{0.886178in}}%
\pgfpathlineto{\pgfqpoint{0.866581in}{0.974940in}}%
\pgfpathlineto{\pgfqpoint{0.885526in}{1.021585in}}%
\pgfpathlineto{\pgfqpoint{0.904471in}{1.060400in}}%
\pgfpathlineto{\pgfqpoint{0.913943in}{1.084406in}}%
\pgfpathlineto{\pgfqpoint{0.932888in}{1.116468in}}%
\pgfpathlineto{\pgfqpoint{0.951833in}{1.155675in}}%
\pgfpathlineto{\pgfqpoint{0.989723in}{1.218525in}}%
\pgfpathlineto{\pgfqpoint{0.999196in}{1.233917in}}%
\pgfpathlineto{\pgfqpoint{1.065503in}{1.312312in}}%
\pgfpathlineto{\pgfqpoint{1.074976in}{1.326339in}}%
\pgfpathlineto{\pgfqpoint{1.084448in}{1.344080in}}%
\pgfpathlineto{\pgfqpoint{1.093921in}{1.354974in}}%
\pgfpathlineto{\pgfqpoint{1.103393in}{1.369681in}}%
\pgfpathlineto{\pgfqpoint{1.122338in}{1.391269in}}%
\pgfpathlineto{\pgfqpoint{1.131811in}{1.397069in}}%
\pgfpathlineto{\pgfqpoint{1.179173in}{1.440568in}}%
\pgfpathlineto{\pgfqpoint{1.188646in}{1.454002in}}%
\pgfpathlineto{\pgfqpoint{1.198118in}{1.464598in}}%
\pgfpathlineto{\pgfqpoint{1.207591in}{1.471871in}}%
\pgfpathlineto{\pgfqpoint{1.217063in}{1.482565in}}%
\pgfpathlineto{\pgfqpoint{1.226536in}{1.489152in}}%
\pgfpathlineto{\pgfqpoint{1.254953in}{1.516246in}}%
\pgfpathlineto{\pgfqpoint{1.283371in}{1.546276in}}%
\pgfpathlineto{\pgfqpoint{1.311788in}{1.572397in}}%
\pgfpathlineto{\pgfqpoint{1.330733in}{1.582234in}}%
\pgfpathlineto{\pgfqpoint{1.368623in}{1.614056in}}%
\pgfpathlineto{\pgfqpoint{1.378096in}{1.625048in}}%
\pgfpathlineto{\pgfqpoint{1.387568in}{1.628694in}}%
\pgfpathlineto{\pgfqpoint{1.397041in}{1.630480in}}%
\pgfpathlineto{\pgfqpoint{1.406513in}{1.636578in}}%
\pgfpathlineto{\pgfqpoint{1.415986in}{1.645021in}}%
\pgfpathlineto{\pgfqpoint{1.425458in}{1.649357in}}%
\pgfpathlineto{\pgfqpoint{1.434931in}{1.657408in}}%
\pgfpathlineto{\pgfqpoint{1.463348in}{1.673446in}}%
\pgfpathlineto{\pgfqpoint{1.482293in}{1.690820in}}%
\pgfpathlineto{\pgfqpoint{1.510711in}{1.709991in}}%
\pgfpathlineto{\pgfqpoint{1.548601in}{1.745233in}}%
\pgfpathlineto{\pgfqpoint{1.558073in}{1.746046in}}%
\pgfpathlineto{\pgfqpoint{1.567546in}{1.749007in}}%
\pgfpathlineto{\pgfqpoint{1.614908in}{1.774490in}}%
\pgfpathlineto{\pgfqpoint{1.624381in}{1.781274in}}%
\pgfpathlineto{\pgfqpoint{1.633853in}{1.785116in}}%
\pgfpathlineto{\pgfqpoint{1.643325in}{1.794150in}}%
\pgfpathlineto{\pgfqpoint{1.652798in}{1.800439in}}%
\pgfpathlineto{\pgfqpoint{1.671743in}{1.807541in}}%
\pgfpathlineto{\pgfqpoint{1.709633in}{1.829574in}}%
\pgfpathlineto{\pgfqpoint{1.719105in}{1.834395in}}%
\pgfpathlineto{\pgfqpoint{1.747523in}{1.852978in}}%
\pgfpathlineto{\pgfqpoint{1.756995in}{1.852121in}}%
\pgfpathlineto{\pgfqpoint{1.775940in}{1.863530in}}%
\pgfpathlineto{\pgfqpoint{1.785413in}{1.863163in}}%
\pgfpathlineto{\pgfqpoint{1.804358in}{1.873396in}}%
\pgfpathlineto{\pgfqpoint{1.813830in}{1.875090in}}%
\pgfpathlineto{\pgfqpoint{1.823303in}{1.881379in}}%
\pgfpathlineto{\pgfqpoint{1.842248in}{1.882901in}}%
\pgfpathlineto{\pgfqpoint{1.851720in}{1.888211in}}%
\pgfpathlineto{\pgfqpoint{1.861193in}{1.888240in}}%
\pgfpathlineto{\pgfqpoint{1.870665in}{1.894138in}}%
\pgfpathlineto{\pgfqpoint{1.880138in}{1.905130in}}%
\pgfpathlineto{\pgfqpoint{1.899083in}{1.911150in}}%
\pgfpathlineto{\pgfqpoint{1.908555in}{1.913039in}}%
\pgfpathlineto{\pgfqpoint{1.918028in}{1.919426in}}%
\pgfpathlineto{\pgfqpoint{1.936973in}{1.922710in}}%
\pgfpathlineto{\pgfqpoint{1.955918in}{1.933726in}}%
\pgfpathlineto{\pgfqpoint{1.965390in}{1.936492in}}%
\pgfpathlineto{\pgfqpoint{1.984335in}{1.945355in}}%
\pgfpathlineto{\pgfqpoint{2.012753in}{1.956695in}}%
\pgfpathlineto{\pgfqpoint{2.022225in}{1.959264in}}%
\pgfpathlineto{\pgfqpoint{2.031698in}{1.965558in}}%
\pgfpathlineto{\pgfqpoint{2.041170in}{1.969987in}}%
\pgfpathlineto{\pgfqpoint{2.069588in}{1.978973in}}%
\pgfpathlineto{\pgfqpoint{2.107478in}{1.999636in}}%
\pgfpathlineto{\pgfqpoint{2.116950in}{1.999171in}}%
\pgfpathlineto{\pgfqpoint{2.126423in}{2.000277in}}%
\pgfpathlineto{\pgfqpoint{2.135895in}{2.005391in}}%
\pgfpathlineto{\pgfqpoint{2.154840in}{2.007794in}}%
\pgfpathlineto{\pgfqpoint{2.164313in}{2.015459in}}%
\pgfpathlineto{\pgfqpoint{2.173785in}{2.015385in}}%
\pgfpathlineto{\pgfqpoint{2.183258in}{2.020597in}}%
\pgfpathlineto{\pgfqpoint{2.202203in}{2.038074in}}%
\pgfpathlineto{\pgfqpoint{2.230620in}{2.051568in}}%
\pgfpathlineto{\pgfqpoint{2.240093in}{2.052571in}}%
\pgfpathlineto{\pgfqpoint{2.249565in}{2.058370in}}%
\pgfpathlineto{\pgfqpoint{2.259038in}{2.066328in}}%
\pgfpathlineto{\pgfqpoint{2.268510in}{2.064493in}}%
\pgfpathlineto{\pgfqpoint{2.277983in}{2.064327in}}%
\pgfpathlineto{\pgfqpoint{2.296928in}{2.070645in}}%
\pgfpathlineto{\pgfqpoint{2.306400in}{2.077423in}}%
\pgfpathlineto{\pgfqpoint{2.315873in}{2.076082in}}%
\pgfpathlineto{\pgfqpoint{2.325345in}{2.081295in}}%
\pgfpathlineto{\pgfqpoint{2.334818in}{2.083962in}}%
\pgfpathlineto{\pgfqpoint{2.344290in}{2.082523in}}%
\pgfpathlineto{\pgfqpoint{2.353763in}{2.083233in}}%
\pgfpathlineto{\pgfqpoint{2.363235in}{2.082185in}}%
\pgfpathlineto{\pgfqpoint{2.372708in}{2.087104in}}%
\pgfpathlineto{\pgfqpoint{2.382180in}{2.085763in}}%
\pgfpathlineto{\pgfqpoint{2.391653in}{2.086962in}}%
\pgfpathlineto{\pgfqpoint{2.401125in}{2.094719in}}%
\pgfpathlineto{\pgfqpoint{2.410598in}{2.098566in}}%
\pgfpathlineto{\pgfqpoint{2.420070in}{2.098884in}}%
\pgfpathlineto{\pgfqpoint{2.429543in}{2.105374in}}%
\pgfpathlineto{\pgfqpoint{2.439015in}{2.108531in}}%
\pgfpathlineto{\pgfqpoint{2.457960in}{2.107410in}}%
\pgfpathlineto{\pgfqpoint{2.467433in}{2.112328in}}%
\pgfpathlineto{\pgfqpoint{2.495850in}{2.119263in}}%
\pgfpathlineto{\pgfqpoint{2.514795in}{2.133510in}}%
\pgfpathlineto{\pgfqpoint{2.524268in}{2.132164in}}%
\pgfpathlineto{\pgfqpoint{2.543213in}{2.137504in}}%
\pgfpathlineto{\pgfqpoint{2.562158in}{2.140494in}}%
\pgfpathlineto{\pgfqpoint{2.571630in}{2.145413in}}%
\pgfpathlineto{\pgfqpoint{2.581103in}{2.146029in}}%
\pgfpathlineto{\pgfqpoint{2.600048in}{2.154893in}}%
\pgfpathlineto{\pgfqpoint{2.609520in}{2.162944in}}%
\pgfpathlineto{\pgfqpoint{2.618993in}{2.165905in}}%
\pgfpathlineto{\pgfqpoint{2.647410in}{2.161877in}}%
\pgfpathlineto{\pgfqpoint{2.656883in}{2.168361in}}%
\pgfpathlineto{\pgfqpoint{2.675828in}{2.169492in}}%
\pgfpathlineto{\pgfqpoint{2.685300in}{2.176075in}}%
\pgfpathlineto{\pgfqpoint{2.694773in}{2.184620in}}%
\pgfpathlineto{\pgfqpoint{2.713718in}{2.189470in}}%
\pgfpathlineto{\pgfqpoint{2.742135in}{2.190233in}}%
\pgfpathlineto{\pgfqpoint{2.751608in}{2.198676in}}%
\pgfpathlineto{\pgfqpoint{2.770553in}{2.205777in}}%
\pgfpathlineto{\pgfqpoint{2.780025in}{2.206198in}}%
\pgfpathlineto{\pgfqpoint{2.798970in}{2.216921in}}%
\pgfpathlineto{\pgfqpoint{2.817915in}{2.221184in}}%
\pgfpathlineto{\pgfqpoint{2.827388in}{2.221404in}}%
\pgfpathlineto{\pgfqpoint{2.836860in}{2.226616in}}%
\pgfpathlineto{\pgfqpoint{2.865278in}{2.228070in}}%
\pgfpathlineto{\pgfqpoint{2.874750in}{2.230933in}}%
\pgfpathlineto{\pgfqpoint{2.884223in}{2.229298in}}%
\pgfpathlineto{\pgfqpoint{2.912640in}{2.233096in}}%
\pgfpathlineto{\pgfqpoint{2.922113in}{2.239190in}}%
\pgfpathlineto{\pgfqpoint{2.950530in}{2.249942in}}%
\pgfpathlineto{\pgfqpoint{2.960003in}{2.256720in}}%
\pgfpathlineto{\pgfqpoint{2.969475in}{2.260171in}}%
\pgfpathlineto{\pgfqpoint{2.978948in}{2.260885in}}%
\pgfpathlineto{\pgfqpoint{2.988420in}{2.266979in}}%
\pgfpathlineto{\pgfqpoint{3.007365in}{2.272905in}}%
\pgfpathlineto{\pgfqpoint{3.016838in}{2.279199in}}%
\pgfpathlineto{\pgfqpoint{3.026310in}{2.276091in}}%
\pgfpathlineto{\pgfqpoint{3.035783in}{2.276214in}}%
\pgfpathlineto{\pgfqpoint{3.054728in}{2.281749in}}%
\pgfpathlineto{\pgfqpoint{3.064200in}{2.280212in}}%
\pgfpathlineto{\pgfqpoint{3.083145in}{2.280560in}}%
\pgfpathlineto{\pgfqpoint{3.102090in}{2.285801in}}%
\pgfpathlineto{\pgfqpoint{3.111563in}{2.294537in}}%
\pgfpathlineto{\pgfqpoint{3.130508in}{2.303303in}}%
\pgfpathlineto{\pgfqpoint{3.139980in}{2.303034in}}%
\pgfpathlineto{\pgfqpoint{3.149453in}{2.305216in}}%
\pgfpathlineto{\pgfqpoint{3.158925in}{2.303577in}}%
\pgfpathlineto{\pgfqpoint{3.177870in}{2.307252in}}%
\pgfpathlineto{\pgfqpoint{3.187343in}{2.305711in}}%
\pgfpathlineto{\pgfqpoint{3.196815in}{2.313864in}}%
\pgfpathlineto{\pgfqpoint{3.206288in}{2.317413in}}%
\pgfpathlineto{\pgfqpoint{3.215760in}{2.318323in}}%
\pgfpathlineto{\pgfqpoint{3.225233in}{2.321088in}}%
\pgfpathlineto{\pgfqpoint{3.234705in}{2.321313in}}%
\pgfpathlineto{\pgfqpoint{3.244178in}{2.319772in}}%
\pgfpathlineto{\pgfqpoint{3.263123in}{2.324524in}}%
\pgfpathlineto{\pgfqpoint{3.282068in}{2.332702in}}%
\pgfpathlineto{\pgfqpoint{3.291540in}{2.336740in}}%
\pgfpathlineto{\pgfqpoint{3.310485in}{2.341883in}}%
\pgfpathlineto{\pgfqpoint{3.319958in}{2.345823in}}%
\pgfpathlineto{\pgfqpoint{3.329430in}{2.345755in}}%
\pgfpathlineto{\pgfqpoint{3.338903in}{2.348618in}}%
\pgfpathlineto{\pgfqpoint{3.348375in}{2.354520in}}%
\pgfpathlineto{\pgfqpoint{3.357848in}{2.355621in}}%
\pgfpathlineto{\pgfqpoint{3.367320in}{2.358685in}}%
\pgfpathlineto{\pgfqpoint{3.376793in}{2.357241in}}%
\pgfpathlineto{\pgfqpoint{3.395737in}{2.360427in}}%
\pgfpathlineto{\pgfqpoint{3.405210in}{2.363193in}}%
\pgfpathlineto{\pgfqpoint{3.414682in}{2.363222in}}%
\pgfpathlineto{\pgfqpoint{3.424155in}{2.368434in}}%
\pgfpathlineto{\pgfqpoint{3.443100in}{2.365943in}}%
\pgfpathlineto{\pgfqpoint{3.452572in}{2.372922in}}%
\pgfpathlineto{\pgfqpoint{3.462045in}{2.369912in}}%
\pgfpathlineto{\pgfqpoint{3.499935in}{2.386073in}}%
\pgfpathlineto{\pgfqpoint{3.509407in}{2.381399in}}%
\pgfpathlineto{\pgfqpoint{3.518880in}{2.381526in}}%
\pgfpathlineto{\pgfqpoint{3.556770in}{2.399248in}}%
\pgfpathlineto{\pgfqpoint{3.585187in}{2.400114in}}%
\pgfpathlineto{\pgfqpoint{3.594660in}{2.408850in}}%
\pgfpathlineto{\pgfqpoint{3.604132in}{2.411909in}}%
\pgfpathlineto{\pgfqpoint{3.642022in}{2.411527in}}%
\pgfpathlineto{\pgfqpoint{3.651495in}{2.409697in}}%
\pgfpathlineto{\pgfqpoint{3.660967in}{2.413930in}}%
\pgfpathlineto{\pgfqpoint{3.679912in}{2.417997in}}%
\pgfpathlineto{\pgfqpoint{3.708330in}{2.431975in}}%
\pgfpathlineto{\pgfqpoint{3.727275in}{2.434378in}}%
\pgfpathlineto{\pgfqpoint{3.746220in}{2.430516in}}%
\pgfpathlineto{\pgfqpoint{3.755692in}{2.431911in}}%
\pgfpathlineto{\pgfqpoint{3.774637in}{2.438719in}}%
\pgfpathlineto{\pgfqpoint{3.784110in}{2.437280in}}%
\pgfpathlineto{\pgfqpoint{3.793582in}{2.441807in}}%
\pgfpathlineto{\pgfqpoint{3.812527in}{2.436575in}}%
\pgfpathlineto{\pgfqpoint{3.831472in}{2.444362in}}%
\pgfpathlineto{\pgfqpoint{3.850417in}{2.447744in}}%
\pgfpathlineto{\pgfqpoint{3.878835in}{2.450367in}}%
\pgfpathlineto{\pgfqpoint{3.888307in}{2.454698in}}%
\pgfpathlineto{\pgfqpoint{3.897780in}{2.452868in}}%
\pgfpathlineto{\pgfqpoint{3.907252in}{2.456220in}}%
\pgfpathlineto{\pgfqpoint{3.916725in}{2.461242in}}%
\pgfpathlineto{\pgfqpoint{3.926197in}{2.458134in}}%
\pgfpathlineto{\pgfqpoint{3.935670in}{2.457478in}}%
\pgfpathlineto{\pgfqpoint{3.945142in}{2.454077in}}%
\pgfpathlineto{\pgfqpoint{3.954615in}{2.454987in}}%
\pgfpathlineto{\pgfqpoint{3.964087in}{2.453348in}}%
\pgfpathlineto{\pgfqpoint{3.973560in}{2.454057in}}%
\pgfpathlineto{\pgfqpoint{3.983032in}{2.456534in}}%
\pgfpathlineto{\pgfqpoint{3.992505in}{2.461061in}}%
\pgfpathlineto{\pgfqpoint{4.011450in}{2.466302in}}%
\pgfpathlineto{\pgfqpoint{4.030395in}{2.467335in}}%
\pgfpathlineto{\pgfqpoint{4.049340in}{2.472577in}}%
\pgfpathlineto{\pgfqpoint{4.058812in}{2.471622in}}%
\pgfpathlineto{\pgfqpoint{4.077757in}{2.477941in}}%
\pgfpathlineto{\pgfqpoint{4.087230in}{2.483451in}}%
\pgfpathlineto{\pgfqpoint{4.096702in}{2.484455in}}%
\pgfpathlineto{\pgfqpoint{4.106175in}{2.487905in}}%
\pgfpathlineto{\pgfqpoint{4.125120in}{2.497649in}}%
\pgfpathlineto{\pgfqpoint{4.144065in}{2.502108in}}%
\pgfpathlineto{\pgfqpoint{4.153537in}{2.496558in}}%
\pgfpathlineto{\pgfqpoint{4.163010in}{2.499323in}}%
\pgfpathlineto{\pgfqpoint{4.172482in}{2.497194in}}%
\pgfpathlineto{\pgfqpoint{4.181955in}{2.499181in}}%
\pgfpathlineto{\pgfqpoint{4.200900in}{2.497571in}}%
\pgfpathlineto{\pgfqpoint{4.219845in}{2.500953in}}%
\pgfpathlineto{\pgfqpoint{4.248262in}{2.504457in}}%
\pgfpathlineto{\pgfqpoint{4.267207in}{2.500596in}}%
\pgfpathlineto{\pgfqpoint{4.276680in}{2.501501in}}%
\pgfpathlineto{\pgfqpoint{4.286152in}{2.499279in}}%
\pgfpathlineto{\pgfqpoint{4.295625in}{2.502240in}}%
\pgfpathlineto{\pgfqpoint{4.305097in}{2.499431in}}%
\pgfpathlineto{\pgfqpoint{4.314570in}{2.503566in}}%
\pgfpathlineto{\pgfqpoint{4.333515in}{2.507927in}}%
\pgfpathlineto{\pgfqpoint{4.342987in}{2.511573in}}%
\pgfpathlineto{\pgfqpoint{4.352460in}{2.508960in}}%
\pgfpathlineto{\pgfqpoint{4.361932in}{2.503796in}}%
\pgfpathlineto{\pgfqpoint{4.371405in}{2.510188in}}%
\pgfpathlineto{\pgfqpoint{4.418767in}{2.527939in}}%
\pgfpathlineto{\pgfqpoint{4.428240in}{2.523657in}}%
\pgfpathlineto{\pgfqpoint{4.447185in}{2.531835in}}%
\pgfpathlineto{\pgfqpoint{4.456657in}{2.534013in}}%
\pgfpathlineto{\pgfqpoint{4.466130in}{2.539622in}}%
\pgfpathlineto{\pgfqpoint{4.494547in}{2.542049in}}%
\pgfpathlineto{\pgfqpoint{4.504020in}{2.545994in}}%
\pgfpathlineto{\pgfqpoint{4.541910in}{2.532687in}}%
\pgfpathlineto{\pgfqpoint{4.560855in}{2.540963in}}%
\pgfpathlineto{\pgfqpoint{4.570327in}{2.536391in}}%
\pgfpathlineto{\pgfqpoint{4.579800in}{2.536514in}}%
\pgfpathlineto{\pgfqpoint{4.589272in}{2.539871in}}%
\pgfpathlineto{\pgfqpoint{4.598745in}{2.539406in}}%
\pgfpathlineto{\pgfqpoint{4.617690in}{2.546605in}}%
\pgfpathlineto{\pgfqpoint{4.627162in}{2.545553in}}%
\pgfpathlineto{\pgfqpoint{4.636635in}{2.546757in}}%
\pgfpathlineto{\pgfqpoint{4.646107in}{2.545901in}}%
\pgfpathlineto{\pgfqpoint{4.655580in}{2.548964in}}%
\pgfpathlineto{\pgfqpoint{4.665052in}{2.548010in}}%
\pgfpathlineto{\pgfqpoint{4.674525in}{2.551754in}}%
\pgfpathlineto{\pgfqpoint{4.683997in}{2.547477in}}%
\pgfpathlineto{\pgfqpoint{4.702942in}{2.556046in}}%
\pgfpathlineto{\pgfqpoint{4.759777in}{2.566481in}}%
\pgfpathlineto{\pgfqpoint{4.769250in}{2.573068in}}%
\pgfpathlineto{\pgfqpoint{4.778722in}{2.572407in}}%
\pgfpathlineto{\pgfqpoint{4.797667in}{2.581173in}}%
\pgfpathlineto{\pgfqpoint{4.816612in}{2.582303in}}%
\pgfpathlineto{\pgfqpoint{4.835557in}{2.584021in}}%
\pgfpathlineto{\pgfqpoint{4.845030in}{2.584535in}}%
\pgfpathlineto{\pgfqpoint{4.854502in}{2.587403in}}%
\pgfpathlineto{\pgfqpoint{4.863975in}{2.586938in}}%
\pgfpathlineto{\pgfqpoint{4.873447in}{2.588240in}}%
\pgfpathlineto{\pgfqpoint{4.882920in}{2.592473in}}%
\pgfpathlineto{\pgfqpoint{4.892392in}{2.591323in}}%
\pgfpathlineto{\pgfqpoint{4.901865in}{2.594681in}}%
\pgfpathlineto{\pgfqpoint{4.911337in}{2.594411in}}%
\pgfpathlineto{\pgfqpoint{4.920810in}{2.589547in}}%
\pgfpathlineto{\pgfqpoint{4.930282in}{2.587907in}}%
\pgfpathlineto{\pgfqpoint{4.977645in}{2.602722in}}%
\pgfpathlineto{\pgfqpoint{4.987117in}{2.596584in}}%
\pgfpathlineto{\pgfqpoint{4.996590in}{2.600916in}}%
\pgfpathlineto{\pgfqpoint{5.006062in}{2.607014in}}%
\pgfpathlineto{\pgfqpoint{5.015535in}{2.611247in}}%
\pgfpathlineto{\pgfqpoint{5.025007in}{2.609613in}}%
\pgfpathlineto{\pgfqpoint{5.034480in}{2.605526in}}%
\pgfpathlineto{\pgfqpoint{5.043952in}{2.613185in}}%
\pgfpathlineto{\pgfqpoint{5.053425in}{2.611649in}}%
\pgfpathlineto{\pgfqpoint{5.072370in}{2.618065in}}%
\pgfpathlineto{\pgfqpoint{5.100787in}{2.618045in}}%
\pgfpathlineto{\pgfqpoint{5.110260in}{2.625998in}}%
\pgfpathlineto{\pgfqpoint{5.119732in}{2.628377in}}%
\pgfpathlineto{\pgfqpoint{5.129205in}{2.633295in}}%
\pgfpathlineto{\pgfqpoint{5.157622in}{2.632302in}}%
\pgfpathlineto{\pgfqpoint{5.167094in}{2.630369in}}%
\pgfpathlineto{\pgfqpoint{5.176567in}{2.630785in}}%
\pgfpathlineto{\pgfqpoint{5.186039in}{2.628465in}}%
\pgfpathlineto{\pgfqpoint{5.195512in}{2.621638in}}%
\pgfpathlineto{\pgfqpoint{5.204984in}{2.628715in}}%
\pgfpathlineto{\pgfqpoint{5.223929in}{2.632586in}}%
\pgfpathlineto{\pgfqpoint{5.233402in}{2.632121in}}%
\pgfpathlineto{\pgfqpoint{5.252347in}{2.639222in}}%
\pgfpathlineto{\pgfqpoint{5.261819in}{2.639247in}}%
\pgfpathlineto{\pgfqpoint{5.299709in}{2.646108in}}%
\pgfpathlineto{\pgfqpoint{5.309182in}{2.649466in}}%
\pgfpathlineto{\pgfqpoint{5.318654in}{2.645673in}}%
\pgfpathlineto{\pgfqpoint{5.328127in}{2.648927in}}%
\pgfpathlineto{\pgfqpoint{5.337599in}{2.648761in}}%
\pgfpathlineto{\pgfqpoint{5.347072in}{2.646436in}}%
\pgfpathlineto{\pgfqpoint{5.366017in}{2.648644in}}%
\pgfpathlineto{\pgfqpoint{5.375489in}{2.647694in}}%
\pgfpathlineto{\pgfqpoint{5.384962in}{2.649676in}}%
\pgfpathlineto{\pgfqpoint{5.394434in}{2.645100in}}%
\pgfpathlineto{\pgfqpoint{5.403907in}{2.646891in}}%
\pgfpathlineto{\pgfqpoint{5.413379in}{2.653376in}}%
\pgfpathlineto{\pgfqpoint{5.432324in}{2.652060in}}%
\pgfpathlineto{\pgfqpoint{5.441797in}{2.654634in}}%
\pgfpathlineto{\pgfqpoint{5.451269in}{2.649569in}}%
\pgfpathlineto{\pgfqpoint{5.470214in}{2.658530in}}%
\pgfpathlineto{\pgfqpoint{5.489159in}{2.660835in}}%
\pgfpathlineto{\pgfqpoint{5.489159in}{2.660835in}}%
\pgfusepath{stroke}%
\end{pgfscope}%
\begin{pgfscope}%
\pgfpathrectangle{\pgfqpoint{0.762383in}{0.471179in}}{\pgfqpoint{4.726776in}{2.845920in}} %
\pgfusepath{clip}%
\pgfsetbuttcap%
\pgfsetroundjoin%
\pgfsetlinewidth{1.505625pt}%
\definecolor{currentstroke}{rgb}{0.839216,0.152941,0.156863}%
\pgfsetstrokecolor{currentstroke}%
\pgfsetdash{{1.500000pt}{2.475000pt}}{0.000000pt}%
\pgfpathmoveto{\pgfqpoint{0.762383in}{0.600539in}}%
\pgfpathlineto{\pgfqpoint{0.771856in}{0.688562in}}%
\pgfpathlineto{\pgfqpoint{0.781328in}{0.724022in}}%
\pgfpathlineto{\pgfqpoint{0.800273in}{0.776541in}}%
\pgfpathlineto{\pgfqpoint{0.819218in}{0.817607in}}%
\pgfpathlineto{\pgfqpoint{0.828691in}{0.832610in}}%
\pgfpathlineto{\pgfqpoint{0.838163in}{0.855345in}}%
\pgfpathlineto{\pgfqpoint{0.847636in}{0.865943in}}%
\pgfpathlineto{\pgfqpoint{0.857108in}{0.883882in}}%
\pgfpathlineto{\pgfqpoint{0.866581in}{0.897221in}}%
\pgfpathlineto{\pgfqpoint{0.885526in}{0.935644in}}%
\pgfpathlineto{\pgfqpoint{0.904471in}{0.962517in}}%
\pgfpathlineto{\pgfqpoint{0.913943in}{0.970179in}}%
\pgfpathlineto{\pgfqpoint{0.923416in}{0.980581in}}%
\pgfpathlineto{\pgfqpoint{0.942361in}{0.998155in}}%
\pgfpathlineto{\pgfqpoint{0.970778in}{1.036116in}}%
\pgfpathlineto{\pgfqpoint{0.980251in}{1.043579in}}%
\pgfpathlineto{\pgfqpoint{0.989723in}{1.057116in}}%
\pgfpathlineto{\pgfqpoint{1.027613in}{1.087764in}}%
\pgfpathlineto{\pgfqpoint{1.037086in}{1.091801in}}%
\pgfpathlineto{\pgfqpoint{1.056031in}{1.105852in}}%
\pgfpathlineto{\pgfqpoint{1.065503in}{1.118112in}}%
\pgfpathlineto{\pgfqpoint{1.074976in}{1.126560in}}%
\pgfpathlineto{\pgfqpoint{1.084448in}{1.138624in}}%
\pgfpathlineto{\pgfqpoint{1.093921in}{1.147560in}}%
\pgfpathlineto{\pgfqpoint{1.112866in}{1.159649in}}%
\pgfpathlineto{\pgfqpoint{1.122338in}{1.165062in}}%
\pgfpathlineto{\pgfqpoint{1.131811in}{1.167827in}}%
\pgfpathlineto{\pgfqpoint{1.141283in}{1.175589in}}%
\pgfpathlineto{\pgfqpoint{1.150756in}{1.186087in}}%
\pgfpathlineto{\pgfqpoint{1.160228in}{1.191206in}}%
\pgfpathlineto{\pgfqpoint{1.198118in}{1.220282in}}%
\pgfpathlineto{\pgfqpoint{1.207591in}{1.223444in}}%
\pgfpathlineto{\pgfqpoint{1.217063in}{1.228558in}}%
\pgfpathlineto{\pgfqpoint{1.236008in}{1.245546in}}%
\pgfpathlineto{\pgfqpoint{1.254953in}{1.253822in}}%
\pgfpathlineto{\pgfqpoint{1.264426in}{1.259132in}}%
\pgfpathlineto{\pgfqpoint{1.311788in}{1.298422in}}%
\pgfpathlineto{\pgfqpoint{1.330733in}{1.307966in}}%
\pgfpathlineto{\pgfqpoint{1.349678in}{1.322506in}}%
\pgfpathlineto{\pgfqpoint{1.368623in}{1.331272in}}%
\pgfpathlineto{\pgfqpoint{1.378096in}{1.339523in}}%
\pgfpathlineto{\pgfqpoint{1.387568in}{1.343659in}}%
\pgfpathlineto{\pgfqpoint{1.406513in}{1.356927in}}%
\pgfpathlineto{\pgfqpoint{1.425458in}{1.363343in}}%
\pgfpathlineto{\pgfqpoint{1.434931in}{1.372471in}}%
\pgfpathlineto{\pgfqpoint{1.444403in}{1.374458in}}%
\pgfpathlineto{\pgfqpoint{1.482293in}{1.395410in}}%
\pgfpathlineto{\pgfqpoint{1.491766in}{1.397592in}}%
\pgfpathlineto{\pgfqpoint{1.501238in}{1.407405in}}%
\pgfpathlineto{\pgfqpoint{1.510711in}{1.411252in}}%
\pgfpathlineto{\pgfqpoint{1.529656in}{1.424031in}}%
\pgfpathlineto{\pgfqpoint{1.539128in}{1.434528in}}%
\pgfpathlineto{\pgfqpoint{1.548601in}{1.435825in}}%
\pgfpathlineto{\pgfqpoint{1.558073in}{1.438889in}}%
\pgfpathlineto{\pgfqpoint{1.567546in}{1.447234in}}%
\pgfpathlineto{\pgfqpoint{1.577018in}{1.452255in}}%
\pgfpathlineto{\pgfqpoint{1.586491in}{1.455412in}}%
\pgfpathlineto{\pgfqpoint{1.595963in}{1.460629in}}%
\pgfpathlineto{\pgfqpoint{1.605436in}{1.468876in}}%
\pgfpathlineto{\pgfqpoint{1.614908in}{1.474088in}}%
\pgfpathlineto{\pgfqpoint{1.633853in}{1.476295in}}%
\pgfpathlineto{\pgfqpoint{1.652798in}{1.488291in}}%
\pgfpathlineto{\pgfqpoint{1.662270in}{1.493508in}}%
\pgfpathlineto{\pgfqpoint{1.681215in}{1.501094in}}%
\pgfpathlineto{\pgfqpoint{1.700160in}{1.512013in}}%
\pgfpathlineto{\pgfqpoint{1.709633in}{1.513119in}}%
\pgfpathlineto{\pgfqpoint{1.738050in}{1.526803in}}%
\pgfpathlineto{\pgfqpoint{1.747523in}{1.526734in}}%
\pgfpathlineto{\pgfqpoint{1.766468in}{1.539997in}}%
\pgfpathlineto{\pgfqpoint{1.775940in}{1.543550in}}%
\pgfpathlineto{\pgfqpoint{1.794885in}{1.547128in}}%
\pgfpathlineto{\pgfqpoint{1.804358in}{1.552928in}}%
\pgfpathlineto{\pgfqpoint{1.813830in}{1.552076in}}%
\pgfpathlineto{\pgfqpoint{1.823303in}{1.556603in}}%
\pgfpathlineto{\pgfqpoint{1.832775in}{1.557802in}}%
\pgfpathlineto{\pgfqpoint{1.842248in}{1.568109in}}%
\pgfpathlineto{\pgfqpoint{1.851720in}{1.572930in}}%
\pgfpathlineto{\pgfqpoint{1.861193in}{1.572372in}}%
\pgfpathlineto{\pgfqpoint{1.870665in}{1.578759in}}%
\pgfpathlineto{\pgfqpoint{1.880138in}{1.581725in}}%
\pgfpathlineto{\pgfqpoint{1.889610in}{1.588210in}}%
\pgfpathlineto{\pgfqpoint{1.908555in}{1.591298in}}%
\pgfpathlineto{\pgfqpoint{1.918028in}{1.597293in}}%
\pgfpathlineto{\pgfqpoint{1.927500in}{1.598986in}}%
\pgfpathlineto{\pgfqpoint{1.936973in}{1.597347in}}%
\pgfpathlineto{\pgfqpoint{1.955918in}{1.612377in}}%
\pgfpathlineto{\pgfqpoint{1.965390in}{1.623071in}}%
\pgfpathlineto{\pgfqpoint{1.974863in}{1.626134in}}%
\pgfpathlineto{\pgfqpoint{1.984335in}{1.627431in}}%
\pgfpathlineto{\pgfqpoint{1.993808in}{1.632648in}}%
\pgfpathlineto{\pgfqpoint{2.003280in}{1.629834in}}%
\pgfpathlineto{\pgfqpoint{2.012753in}{1.634366in}}%
\pgfpathlineto{\pgfqpoint{2.022225in}{1.642515in}}%
\pgfpathlineto{\pgfqpoint{2.031698in}{1.643034in}}%
\pgfpathlineto{\pgfqpoint{2.041170in}{1.646680in}}%
\pgfpathlineto{\pgfqpoint{2.050643in}{1.646313in}}%
\pgfpathlineto{\pgfqpoint{2.060115in}{1.647615in}}%
\pgfpathlineto{\pgfqpoint{2.069588in}{1.654882in}}%
\pgfpathlineto{\pgfqpoint{2.079060in}{1.652367in}}%
\pgfpathlineto{\pgfqpoint{2.088533in}{1.660614in}}%
\pgfpathlineto{\pgfqpoint{2.098005in}{1.658098in}}%
\pgfpathlineto{\pgfqpoint{2.107478in}{1.661157in}}%
\pgfpathlineto{\pgfqpoint{2.116950in}{1.660594in}}%
\pgfpathlineto{\pgfqpoint{2.135895in}{1.665542in}}%
\pgfpathlineto{\pgfqpoint{2.145368in}{1.673696in}}%
\pgfpathlineto{\pgfqpoint{2.183258in}{1.695822in}}%
\pgfpathlineto{\pgfqpoint{2.192730in}{1.702997in}}%
\pgfpathlineto{\pgfqpoint{2.202203in}{1.706545in}}%
\pgfpathlineto{\pgfqpoint{2.211675in}{1.712839in}}%
\pgfpathlineto{\pgfqpoint{2.221148in}{1.714527in}}%
\pgfpathlineto{\pgfqpoint{2.230620in}{1.714361in}}%
\pgfpathlineto{\pgfqpoint{2.240093in}{1.724761in}}%
\pgfpathlineto{\pgfqpoint{2.259038in}{1.736169in}}%
\pgfpathlineto{\pgfqpoint{2.277983in}{1.737985in}}%
\pgfpathlineto{\pgfqpoint{2.306400in}{1.733071in}}%
\pgfpathlineto{\pgfqpoint{2.315873in}{1.735646in}}%
\pgfpathlineto{\pgfqpoint{2.325345in}{1.734593in}}%
\pgfpathlineto{\pgfqpoint{2.344290in}{1.741793in}}%
\pgfpathlineto{\pgfqpoint{2.353763in}{1.747788in}}%
\pgfpathlineto{\pgfqpoint{2.363235in}{1.745860in}}%
\pgfpathlineto{\pgfqpoint{2.391653in}{1.755237in}}%
\pgfpathlineto{\pgfqpoint{2.420070in}{1.766768in}}%
\pgfpathlineto{\pgfqpoint{2.429543in}{1.767384in}}%
\pgfpathlineto{\pgfqpoint{2.448488in}{1.773800in}}%
\pgfpathlineto{\pgfqpoint{2.467433in}{1.774339in}}%
\pgfpathlineto{\pgfqpoint{2.486378in}{1.783006in}}%
\pgfpathlineto{\pgfqpoint{2.505323in}{1.792163in}}%
\pgfpathlineto{\pgfqpoint{2.514795in}{1.796499in}}%
\pgfpathlineto{\pgfqpoint{2.524268in}{1.795154in}}%
\pgfpathlineto{\pgfqpoint{2.533740in}{1.796059in}}%
\pgfpathlineto{\pgfqpoint{2.581103in}{1.810878in}}%
\pgfpathlineto{\pgfqpoint{2.609520in}{1.821430in}}%
\pgfpathlineto{\pgfqpoint{2.618993in}{1.814897in}}%
\pgfpathlineto{\pgfqpoint{2.628465in}{1.815317in}}%
\pgfpathlineto{\pgfqpoint{2.637938in}{1.813972in}}%
\pgfpathlineto{\pgfqpoint{2.647410in}{1.815665in}}%
\pgfpathlineto{\pgfqpoint{2.656883in}{1.819703in}}%
\pgfpathlineto{\pgfqpoint{2.666355in}{1.820417in}}%
\pgfpathlineto{\pgfqpoint{2.685300in}{1.829080in}}%
\pgfpathlineto{\pgfqpoint{2.694773in}{1.833612in}}%
\pgfpathlineto{\pgfqpoint{2.704245in}{1.834517in}}%
\pgfpathlineto{\pgfqpoint{2.713718in}{1.841203in}}%
\pgfpathlineto{\pgfqpoint{2.723190in}{1.843283in}}%
\pgfpathlineto{\pgfqpoint{2.732663in}{1.840767in}}%
\pgfpathlineto{\pgfqpoint{2.742135in}{1.846958in}}%
\pgfpathlineto{\pgfqpoint{2.751608in}{1.849723in}}%
\pgfpathlineto{\pgfqpoint{2.761080in}{1.855038in}}%
\pgfpathlineto{\pgfqpoint{2.770553in}{1.854671in}}%
\pgfpathlineto{\pgfqpoint{2.780025in}{1.859105in}}%
\pgfpathlineto{\pgfqpoint{2.798970in}{1.864641in}}%
\pgfpathlineto{\pgfqpoint{2.808443in}{1.869266in}}%
\pgfpathlineto{\pgfqpoint{2.817915in}{1.871155in}}%
\pgfpathlineto{\pgfqpoint{2.836860in}{1.877077in}}%
\pgfpathlineto{\pgfqpoint{2.846333in}{1.877106in}}%
\pgfpathlineto{\pgfqpoint{2.855805in}{1.879675in}}%
\pgfpathlineto{\pgfqpoint{2.865278in}{1.884305in}}%
\pgfpathlineto{\pgfqpoint{2.884223in}{1.882793in}}%
\pgfpathlineto{\pgfqpoint{2.893695in}{1.886341in}}%
\pgfpathlineto{\pgfqpoint{2.903168in}{1.886953in}}%
\pgfpathlineto{\pgfqpoint{2.912640in}{1.892464in}}%
\pgfpathlineto{\pgfqpoint{2.922113in}{1.894250in}}%
\pgfpathlineto{\pgfqpoint{2.931585in}{1.898586in}}%
\pgfpathlineto{\pgfqpoint{2.941058in}{1.901156in}}%
\pgfpathlineto{\pgfqpoint{2.960003in}{1.911977in}}%
\pgfpathlineto{\pgfqpoint{2.969475in}{1.923062in}}%
\pgfpathlineto{\pgfqpoint{2.978948in}{1.927496in}}%
\pgfpathlineto{\pgfqpoint{2.988420in}{1.929478in}}%
\pgfpathlineto{\pgfqpoint{3.007365in}{1.926987in}}%
\pgfpathlineto{\pgfqpoint{3.026310in}{1.932131in}}%
\pgfpathlineto{\pgfqpoint{3.035783in}{1.935777in}}%
\pgfpathlineto{\pgfqpoint{3.045255in}{1.936100in}}%
\pgfpathlineto{\pgfqpoint{3.054728in}{1.934167in}}%
\pgfpathlineto{\pgfqpoint{3.064200in}{1.936350in}}%
\pgfpathlineto{\pgfqpoint{3.073673in}{1.940289in}}%
\pgfpathlineto{\pgfqpoint{3.092618in}{1.942888in}}%
\pgfpathlineto{\pgfqpoint{3.102090in}{1.950944in}}%
\pgfpathlineto{\pgfqpoint{3.121035in}{1.956572in}}%
\pgfpathlineto{\pgfqpoint{3.139980in}{1.957213in}}%
\pgfpathlineto{\pgfqpoint{3.149453in}{1.959396in}}%
\pgfpathlineto{\pgfqpoint{3.158925in}{1.964902in}}%
\pgfpathlineto{\pgfqpoint{3.168398in}{1.968162in}}%
\pgfpathlineto{\pgfqpoint{3.177870in}{1.968480in}}%
\pgfpathlineto{\pgfqpoint{3.225233in}{1.979183in}}%
\pgfpathlineto{\pgfqpoint{3.234705in}{1.983911in}}%
\pgfpathlineto{\pgfqpoint{3.244178in}{1.983544in}}%
\pgfpathlineto{\pgfqpoint{3.253650in}{1.989148in}}%
\pgfpathlineto{\pgfqpoint{3.272595in}{1.989593in}}%
\pgfpathlineto{\pgfqpoint{3.291540in}{1.998358in}}%
\pgfpathlineto{\pgfqpoint{3.301013in}{1.998779in}}%
\pgfpathlineto{\pgfqpoint{3.310485in}{1.997140in}}%
\pgfpathlineto{\pgfqpoint{3.329430in}{2.000326in}}%
\pgfpathlineto{\pgfqpoint{3.338903in}{2.005734in}}%
\pgfpathlineto{\pgfqpoint{3.367320in}{2.014724in}}%
\pgfpathlineto{\pgfqpoint{3.376793in}{2.015043in}}%
\pgfpathlineto{\pgfqpoint{3.386265in}{2.012233in}}%
\pgfpathlineto{\pgfqpoint{3.395737in}{2.013237in}}%
\pgfpathlineto{\pgfqpoint{3.405210in}{2.021190in}}%
\pgfpathlineto{\pgfqpoint{3.414682in}{2.027092in}}%
\pgfpathlineto{\pgfqpoint{3.424155in}{2.026823in}}%
\pgfpathlineto{\pgfqpoint{3.433627in}{2.029691in}}%
\pgfpathlineto{\pgfqpoint{3.443100in}{2.027464in}}%
\pgfpathlineto{\pgfqpoint{3.452572in}{2.033073in}}%
\pgfpathlineto{\pgfqpoint{3.462045in}{2.034370in}}%
\pgfpathlineto{\pgfqpoint{3.471517in}{2.037624in}}%
\pgfpathlineto{\pgfqpoint{3.480990in}{2.045191in}}%
\pgfpathlineto{\pgfqpoint{3.490462in}{2.046977in}}%
\pgfpathlineto{\pgfqpoint{3.499935in}{2.045832in}}%
\pgfpathlineto{\pgfqpoint{3.509407in}{2.048695in}}%
\pgfpathlineto{\pgfqpoint{3.518880in}{2.049703in}}%
\pgfpathlineto{\pgfqpoint{3.537825in}{2.047501in}}%
\pgfpathlineto{\pgfqpoint{3.556770in}{2.055385in}}%
\pgfpathlineto{\pgfqpoint{3.585187in}{2.060362in}}%
\pgfpathlineto{\pgfqpoint{3.594660in}{2.064302in}}%
\pgfpathlineto{\pgfqpoint{3.604132in}{2.064816in}}%
\pgfpathlineto{\pgfqpoint{3.613605in}{2.069348in}}%
\pgfpathlineto{\pgfqpoint{3.623077in}{2.066730in}}%
\pgfpathlineto{\pgfqpoint{3.632550in}{2.071555in}}%
\pgfpathlineto{\pgfqpoint{3.651495in}{2.072490in}}%
\pgfpathlineto{\pgfqpoint{3.679912in}{2.086468in}}%
\pgfpathlineto{\pgfqpoint{3.689385in}{2.088058in}}%
\pgfpathlineto{\pgfqpoint{3.717802in}{2.097440in}}%
\pgfpathlineto{\pgfqpoint{3.727275in}{2.095116in}}%
\pgfpathlineto{\pgfqpoint{3.736747in}{2.097103in}}%
\pgfpathlineto{\pgfqpoint{3.746220in}{2.101434in}}%
\pgfpathlineto{\pgfqpoint{3.755692in}{2.100284in}}%
\pgfpathlineto{\pgfqpoint{3.765165in}{2.101586in}}%
\pgfpathlineto{\pgfqpoint{3.774637in}{2.106015in}}%
\pgfpathlineto{\pgfqpoint{3.784110in}{2.108589in}}%
\pgfpathlineto{\pgfqpoint{3.803055in}{2.107860in}}%
\pgfpathlineto{\pgfqpoint{3.822000in}{2.113684in}}%
\pgfpathlineto{\pgfqpoint{3.831472in}{2.119097in}}%
\pgfpathlineto{\pgfqpoint{3.840945in}{2.120590in}}%
\pgfpathlineto{\pgfqpoint{3.850417in}{2.124437in}}%
\pgfpathlineto{\pgfqpoint{3.859890in}{2.124950in}}%
\pgfpathlineto{\pgfqpoint{3.869362in}{2.123218in}}%
\pgfpathlineto{\pgfqpoint{3.888307in}{2.127182in}}%
\pgfpathlineto{\pgfqpoint{3.897780in}{2.124960in}}%
\pgfpathlineto{\pgfqpoint{3.907252in}{2.126355in}}%
\pgfpathlineto{\pgfqpoint{3.916725in}{2.126091in}}%
\pgfpathlineto{\pgfqpoint{3.926197in}{2.129541in}}%
\pgfpathlineto{\pgfqpoint{3.935670in}{2.125753in}}%
\pgfpathlineto{\pgfqpoint{3.954615in}{2.124632in}}%
\pgfpathlineto{\pgfqpoint{3.964087in}{2.127495in}}%
\pgfpathlineto{\pgfqpoint{3.973560in}{2.122821in}}%
\pgfpathlineto{\pgfqpoint{3.983032in}{2.128920in}}%
\pgfpathlineto{\pgfqpoint{3.992505in}{2.125910in}}%
\pgfpathlineto{\pgfqpoint{4.011450in}{2.128508in}}%
\pgfpathlineto{\pgfqpoint{4.020922in}{2.132649in}}%
\pgfpathlineto{\pgfqpoint{4.039867in}{2.132111in}}%
\pgfpathlineto{\pgfqpoint{4.049340in}{2.133412in}}%
\pgfpathlineto{\pgfqpoint{4.058812in}{2.137646in}}%
\pgfpathlineto{\pgfqpoint{4.068285in}{2.139046in}}%
\pgfpathlineto{\pgfqpoint{4.087230in}{2.153195in}}%
\pgfpathlineto{\pgfqpoint{4.096702in}{2.151555in}}%
\pgfpathlineto{\pgfqpoint{4.106175in}{2.157550in}}%
\pgfpathlineto{\pgfqpoint{4.144065in}{2.164706in}}%
\pgfpathlineto{\pgfqpoint{4.153537in}{2.164148in}}%
\pgfpathlineto{\pgfqpoint{4.172482in}{2.170461in}}%
\pgfpathlineto{\pgfqpoint{4.181955in}{2.167945in}}%
\pgfpathlineto{\pgfqpoint{4.191427in}{2.171690in}}%
\pgfpathlineto{\pgfqpoint{4.200900in}{2.171523in}}%
\pgfpathlineto{\pgfqpoint{4.210372in}{2.175267in}}%
\pgfpathlineto{\pgfqpoint{4.229317in}{2.178845in}}%
\pgfpathlineto{\pgfqpoint{4.257735in}{2.171484in}}%
\pgfpathlineto{\pgfqpoint{4.276680in}{2.173300in}}%
\pgfpathlineto{\pgfqpoint{4.286152in}{2.170295in}}%
\pgfpathlineto{\pgfqpoint{4.295625in}{2.173549in}}%
\pgfpathlineto{\pgfqpoint{4.305097in}{2.179256in}}%
\pgfpathlineto{\pgfqpoint{4.314570in}{2.182021in}}%
\pgfpathlineto{\pgfqpoint{4.324042in}{2.179990in}}%
\pgfpathlineto{\pgfqpoint{4.352460in}{2.187708in}}%
\pgfpathlineto{\pgfqpoint{4.361932in}{2.187145in}}%
\pgfpathlineto{\pgfqpoint{4.380877in}{2.193268in}}%
\pgfpathlineto{\pgfqpoint{4.390350in}{2.197012in}}%
\pgfpathlineto{\pgfqpoint{4.399822in}{2.193615in}}%
\pgfpathlineto{\pgfqpoint{4.418767in}{2.204338in}}%
\pgfpathlineto{\pgfqpoint{4.428240in}{2.203286in}}%
\pgfpathlineto{\pgfqpoint{4.447185in}{2.212247in}}%
\pgfpathlineto{\pgfqpoint{4.456657in}{2.220983in}}%
\pgfpathlineto{\pgfqpoint{4.466130in}{2.219153in}}%
\pgfpathlineto{\pgfqpoint{4.475602in}{2.215458in}}%
\pgfpathlineto{\pgfqpoint{4.485075in}{2.217249in}}%
\pgfpathlineto{\pgfqpoint{4.494547in}{2.214435in}}%
\pgfpathlineto{\pgfqpoint{4.504020in}{2.214464in}}%
\pgfpathlineto{\pgfqpoint{4.532437in}{2.207593in}}%
\pgfpathlineto{\pgfqpoint{4.541910in}{2.208890in}}%
\pgfpathlineto{\pgfqpoint{4.551382in}{2.208528in}}%
\pgfpathlineto{\pgfqpoint{4.560855in}{2.211782in}}%
\pgfpathlineto{\pgfqpoint{4.570327in}{2.208777in}}%
\pgfpathlineto{\pgfqpoint{4.579800in}{2.212130in}}%
\pgfpathlineto{\pgfqpoint{4.589272in}{2.210691in}}%
\pgfpathlineto{\pgfqpoint{4.598745in}{2.211890in}}%
\pgfpathlineto{\pgfqpoint{4.608217in}{2.214753in}}%
\pgfpathlineto{\pgfqpoint{4.617690in}{2.220753in}}%
\pgfpathlineto{\pgfqpoint{4.627162in}{2.219701in}}%
\pgfpathlineto{\pgfqpoint{4.636635in}{2.215228in}}%
\pgfpathlineto{\pgfqpoint{4.646107in}{2.212609in}}%
\pgfpathlineto{\pgfqpoint{4.674525in}{2.213569in}}%
\pgfpathlineto{\pgfqpoint{4.683997in}{2.219960in}}%
\pgfpathlineto{\pgfqpoint{4.693470in}{2.222236in}}%
\pgfpathlineto{\pgfqpoint{4.702942in}{2.218546in}}%
\pgfpathlineto{\pgfqpoint{4.721887in}{2.223592in}}%
\pgfpathlineto{\pgfqpoint{4.740832in}{2.231868in}}%
\pgfpathlineto{\pgfqpoint{4.750305in}{2.232773in}}%
\pgfpathlineto{\pgfqpoint{4.759777in}{2.236713in}}%
\pgfpathlineto{\pgfqpoint{4.769250in}{2.238015in}}%
\pgfpathlineto{\pgfqpoint{4.778722in}{2.242542in}}%
\pgfpathlineto{\pgfqpoint{4.788195in}{2.243354in}}%
\pgfpathlineto{\pgfqpoint{4.797667in}{2.248077in}}%
\pgfpathlineto{\pgfqpoint{4.807140in}{2.248205in}}%
\pgfpathlineto{\pgfqpoint{4.816612in}{2.251166in}}%
\pgfpathlineto{\pgfqpoint{4.826085in}{2.251092in}}%
\pgfpathlineto{\pgfqpoint{4.835557in}{2.257973in}}%
\pgfpathlineto{\pgfqpoint{4.845030in}{2.254963in}}%
\pgfpathlineto{\pgfqpoint{4.854502in}{2.254308in}}%
\pgfpathlineto{\pgfqpoint{4.863975in}{2.262163in}}%
\pgfpathlineto{\pgfqpoint{4.873447in}{2.263073in}}%
\pgfpathlineto{\pgfqpoint{4.892392in}{2.271246in}}%
\pgfpathlineto{\pgfqpoint{4.901865in}{2.273037in}}%
\pgfpathlineto{\pgfqpoint{4.911337in}{2.269049in}}%
\pgfpathlineto{\pgfqpoint{4.939755in}{2.274418in}}%
\pgfpathlineto{\pgfqpoint{4.949227in}{2.280021in}}%
\pgfpathlineto{\pgfqpoint{4.968172in}{2.282131in}}%
\pgfpathlineto{\pgfqpoint{4.977645in}{2.285288in}}%
\pgfpathlineto{\pgfqpoint{4.996590in}{2.288571in}}%
\pgfpathlineto{\pgfqpoint{5.015535in}{2.285199in}}%
\pgfpathlineto{\pgfqpoint{5.025007in}{2.289242in}}%
\pgfpathlineto{\pgfqpoint{5.043952in}{2.299569in}}%
\pgfpathlineto{\pgfqpoint{5.053425in}{2.296466in}}%
\pgfpathlineto{\pgfqpoint{5.081842in}{2.302809in}}%
\pgfpathlineto{\pgfqpoint{5.091315in}{2.302055in}}%
\pgfpathlineto{\pgfqpoint{5.100787in}{2.306484in}}%
\pgfpathlineto{\pgfqpoint{5.119732in}{2.317697in}}%
\pgfpathlineto{\pgfqpoint{5.148149in}{2.319341in}}%
\pgfpathlineto{\pgfqpoint{5.157622in}{2.318881in}}%
\pgfpathlineto{\pgfqpoint{5.167094in}{2.320178in}}%
\pgfpathlineto{\pgfqpoint{5.176567in}{2.319909in}}%
\pgfpathlineto{\pgfqpoint{5.186039in}{2.314750in}}%
\pgfpathlineto{\pgfqpoint{5.195512in}{2.319963in}}%
\pgfpathlineto{\pgfqpoint{5.204984in}{2.318622in}}%
\pgfpathlineto{\pgfqpoint{5.223929in}{2.314075in}}%
\pgfpathlineto{\pgfqpoint{5.233402in}{2.313121in}}%
\pgfpathlineto{\pgfqpoint{5.242874in}{2.319507in}}%
\pgfpathlineto{\pgfqpoint{5.252347in}{2.321397in}}%
\pgfpathlineto{\pgfqpoint{5.271292in}{2.321548in}}%
\pgfpathlineto{\pgfqpoint{5.280764in}{2.323726in}}%
\pgfpathlineto{\pgfqpoint{5.290237in}{2.323853in}}%
\pgfpathlineto{\pgfqpoint{5.299709in}{2.330925in}}%
\pgfpathlineto{\pgfqpoint{5.309182in}{2.332423in}}%
\pgfpathlineto{\pgfqpoint{5.318654in}{2.339006in}}%
\pgfpathlineto{\pgfqpoint{5.328127in}{2.338639in}}%
\pgfpathlineto{\pgfqpoint{5.337599in}{2.335829in}}%
\pgfpathlineto{\pgfqpoint{5.356544in}{2.337058in}}%
\pgfpathlineto{\pgfqpoint{5.366017in}{2.339431in}}%
\pgfpathlineto{\pgfqpoint{5.375489in}{2.337503in}}%
\pgfpathlineto{\pgfqpoint{5.384962in}{2.332731in}}%
\pgfpathlineto{\pgfqpoint{5.394434in}{2.336182in}}%
\pgfpathlineto{\pgfqpoint{5.403907in}{2.335330in}}%
\pgfpathlineto{\pgfqpoint{5.413379in}{2.336627in}}%
\pgfpathlineto{\pgfqpoint{5.432324in}{2.340890in}}%
\pgfpathlineto{\pgfqpoint{5.441797in}{2.340919in}}%
\pgfpathlineto{\pgfqpoint{5.479687in}{2.349636in}}%
\pgfpathlineto{\pgfqpoint{5.489159in}{2.351231in}}%
\pgfpathlineto{\pgfqpoint{5.489159in}{2.351231in}}%
\pgfusepath{stroke}%
\end{pgfscope}%
\begin{pgfscope}%
\pgfpathrectangle{\pgfqpoint{0.762383in}{0.471179in}}{\pgfqpoint{4.726776in}{2.845920in}} %
\pgfusepath{clip}%
\pgfsetrectcap%
\pgfsetroundjoin%
\pgfsetlinewidth{1.505625pt}%
\definecolor{currentstroke}{rgb}{0.580392,0.403922,0.741176}%
\pgfsetstrokecolor{currentstroke}%
\pgfsetdash{}{0pt}%
\pgfpathmoveto{\pgfqpoint{0.762383in}{0.600539in}}%
\pgfpathlineto{\pgfqpoint{0.771856in}{0.688562in}}%
\pgfpathlineto{\pgfqpoint{0.781328in}{0.724022in}}%
\pgfpathlineto{\pgfqpoint{0.800273in}{0.776541in}}%
\pgfpathlineto{\pgfqpoint{0.828691in}{0.840538in}}%
\pgfpathlineto{\pgfqpoint{0.838163in}{0.865623in}}%
\pgfpathlineto{\pgfqpoint{0.866581in}{0.916112in}}%
\pgfpathlineto{\pgfqpoint{0.894998in}{0.976292in}}%
\pgfpathlineto{\pgfqpoint{0.923416in}{1.015232in}}%
\pgfpathlineto{\pgfqpoint{0.942361in}{1.042008in}}%
\pgfpathlineto{\pgfqpoint{0.951833in}{1.057988in}}%
\pgfpathlineto{\pgfqpoint{0.961306in}{1.069469in}}%
\pgfpathlineto{\pgfqpoint{0.989723in}{1.112322in}}%
\pgfpathlineto{\pgfqpoint{1.008668in}{1.132442in}}%
\pgfpathlineto{\pgfqpoint{1.018141in}{1.148128in}}%
\pgfpathlineto{\pgfqpoint{1.046558in}{1.167783in}}%
\pgfpathlineto{\pgfqpoint{1.074976in}{1.205943in}}%
\pgfpathlineto{\pgfqpoint{1.103393in}{1.240182in}}%
\pgfpathlineto{\pgfqpoint{1.112866in}{1.248331in}}%
\pgfpathlineto{\pgfqpoint{1.131811in}{1.260033in}}%
\pgfpathlineto{\pgfqpoint{1.188646in}{1.319707in}}%
\pgfpathlineto{\pgfqpoint{1.198118in}{1.324919in}}%
\pgfpathlineto{\pgfqpoint{1.245481in}{1.380849in}}%
\pgfpathlineto{\pgfqpoint{1.254953in}{1.385964in}}%
\pgfpathlineto{\pgfqpoint{1.292843in}{1.427770in}}%
\pgfpathlineto{\pgfqpoint{1.302316in}{1.435331in}}%
\pgfpathlineto{\pgfqpoint{1.311788in}{1.446128in}}%
\pgfpathlineto{\pgfqpoint{1.330733in}{1.455769in}}%
\pgfpathlineto{\pgfqpoint{1.349678in}{1.472855in}}%
\pgfpathlineto{\pgfqpoint{1.387568in}{1.501740in}}%
\pgfpathlineto{\pgfqpoint{1.397041in}{1.507931in}}%
\pgfpathlineto{\pgfqpoint{1.415986in}{1.525310in}}%
\pgfpathlineto{\pgfqpoint{1.444403in}{1.540565in}}%
\pgfpathlineto{\pgfqpoint{1.453876in}{1.543330in}}%
\pgfpathlineto{\pgfqpoint{1.463348in}{1.549918in}}%
\pgfpathlineto{\pgfqpoint{1.472821in}{1.561395in}}%
\pgfpathlineto{\pgfqpoint{1.482293in}{1.568173in}}%
\pgfpathlineto{\pgfqpoint{1.510711in}{1.600557in}}%
\pgfpathlineto{\pgfqpoint{1.520183in}{1.605280in}}%
\pgfpathlineto{\pgfqpoint{1.539128in}{1.624030in}}%
\pgfpathlineto{\pgfqpoint{1.548601in}{1.634038in}}%
\pgfpathlineto{\pgfqpoint{1.595963in}{1.670000in}}%
\pgfpathlineto{\pgfqpoint{1.605436in}{1.670123in}}%
\pgfpathlineto{\pgfqpoint{1.624381in}{1.683489in}}%
\pgfpathlineto{\pgfqpoint{1.643325in}{1.697246in}}%
\pgfpathlineto{\pgfqpoint{1.652798in}{1.707059in}}%
\pgfpathlineto{\pgfqpoint{1.662270in}{1.709633in}}%
\pgfpathlineto{\pgfqpoint{1.671743in}{1.717586in}}%
\pgfpathlineto{\pgfqpoint{1.681215in}{1.722505in}}%
\pgfpathlineto{\pgfqpoint{1.700160in}{1.738611in}}%
\pgfpathlineto{\pgfqpoint{1.719105in}{1.746692in}}%
\pgfpathlineto{\pgfqpoint{1.738050in}{1.762211in}}%
\pgfpathlineto{\pgfqpoint{1.747523in}{1.771148in}}%
\pgfpathlineto{\pgfqpoint{1.766468in}{1.783041in}}%
\pgfpathlineto{\pgfqpoint{1.775940in}{1.789922in}}%
\pgfpathlineto{\pgfqpoint{1.794885in}{1.794968in}}%
\pgfpathlineto{\pgfqpoint{1.804358in}{1.803116in}}%
\pgfpathlineto{\pgfqpoint{1.813830in}{1.807746in}}%
\pgfpathlineto{\pgfqpoint{1.823303in}{1.814133in}}%
\pgfpathlineto{\pgfqpoint{1.842248in}{1.824465in}}%
\pgfpathlineto{\pgfqpoint{1.851720in}{1.834082in}}%
\pgfpathlineto{\pgfqpoint{1.861193in}{1.834111in}}%
\pgfpathlineto{\pgfqpoint{1.880138in}{1.845421in}}%
\pgfpathlineto{\pgfqpoint{1.889610in}{1.852298in}}%
\pgfpathlineto{\pgfqpoint{1.899083in}{1.854280in}}%
\pgfpathlineto{\pgfqpoint{1.918028in}{1.867646in}}%
\pgfpathlineto{\pgfqpoint{1.936973in}{1.870538in}}%
\pgfpathlineto{\pgfqpoint{1.955918in}{1.881946in}}%
\pgfpathlineto{\pgfqpoint{1.965390in}{1.893717in}}%
\pgfpathlineto{\pgfqpoint{1.993808in}{1.907014in}}%
\pgfpathlineto{\pgfqpoint{2.003280in}{1.910954in}}%
\pgfpathlineto{\pgfqpoint{2.012753in}{1.917737in}}%
\pgfpathlineto{\pgfqpoint{2.022225in}{1.926669in}}%
\pgfpathlineto{\pgfqpoint{2.031698in}{1.932865in}}%
\pgfpathlineto{\pgfqpoint{2.050643in}{1.949456in}}%
\pgfpathlineto{\pgfqpoint{2.069588in}{1.954796in}}%
\pgfpathlineto{\pgfqpoint{2.079060in}{1.962460in}}%
\pgfpathlineto{\pgfqpoint{2.107478in}{1.975067in}}%
\pgfpathlineto{\pgfqpoint{2.116950in}{1.973819in}}%
\pgfpathlineto{\pgfqpoint{2.135895in}{1.982976in}}%
\pgfpathlineto{\pgfqpoint{2.145368in}{1.987704in}}%
\pgfpathlineto{\pgfqpoint{2.154840in}{1.995657in}}%
\pgfpathlineto{\pgfqpoint{2.173785in}{2.003150in}}%
\pgfpathlineto{\pgfqpoint{2.183258in}{2.006894in}}%
\pgfpathlineto{\pgfqpoint{2.192730in}{2.018082in}}%
\pgfpathlineto{\pgfqpoint{2.221148in}{2.040869in}}%
\pgfpathlineto{\pgfqpoint{2.230620in}{2.040507in}}%
\pgfpathlineto{\pgfqpoint{2.240093in}{2.044838in}}%
\pgfpathlineto{\pgfqpoint{2.249565in}{2.054455in}}%
\pgfpathlineto{\pgfqpoint{2.287455in}{2.082655in}}%
\pgfpathlineto{\pgfqpoint{2.306400in}{2.084471in}}%
\pgfpathlineto{\pgfqpoint{2.315873in}{2.083228in}}%
\pgfpathlineto{\pgfqpoint{2.325345in}{2.087951in}}%
\pgfpathlineto{\pgfqpoint{2.334818in}{2.090912in}}%
\pgfpathlineto{\pgfqpoint{2.344290in}{2.096422in}}%
\pgfpathlineto{\pgfqpoint{2.363235in}{2.112040in}}%
\pgfpathlineto{\pgfqpoint{2.382180in}{2.113855in}}%
\pgfpathlineto{\pgfqpoint{2.391653in}{2.119068in}}%
\pgfpathlineto{\pgfqpoint{2.401125in}{2.127412in}}%
\pgfpathlineto{\pgfqpoint{2.420070in}{2.133828in}}%
\pgfpathlineto{\pgfqpoint{2.429543in}{2.142471in}}%
\pgfpathlineto{\pgfqpoint{2.448488in}{2.145755in}}%
\pgfpathlineto{\pgfqpoint{2.467433in}{2.151775in}}%
\pgfpathlineto{\pgfqpoint{2.476905in}{2.152490in}}%
\pgfpathlineto{\pgfqpoint{2.486378in}{2.156234in}}%
\pgfpathlineto{\pgfqpoint{2.495850in}{2.156948in}}%
\pgfpathlineto{\pgfqpoint{2.505323in}{2.160594in}}%
\pgfpathlineto{\pgfqpoint{2.524268in}{2.170339in}}%
\pgfpathlineto{\pgfqpoint{2.533740in}{2.173789in}}%
\pgfpathlineto{\pgfqpoint{2.543213in}{2.179202in}}%
\pgfpathlineto{\pgfqpoint{2.552685in}{2.178835in}}%
\pgfpathlineto{\pgfqpoint{2.581103in}{2.194775in}}%
\pgfpathlineto{\pgfqpoint{2.590575in}{2.205371in}}%
\pgfpathlineto{\pgfqpoint{2.600048in}{2.206477in}}%
\pgfpathlineto{\pgfqpoint{2.647410in}{2.225403in}}%
\pgfpathlineto{\pgfqpoint{2.656883in}{2.225623in}}%
\pgfpathlineto{\pgfqpoint{2.666355in}{2.227414in}}%
\pgfpathlineto{\pgfqpoint{2.675828in}{2.225579in}}%
\pgfpathlineto{\pgfqpoint{2.694773in}{2.242273in}}%
\pgfpathlineto{\pgfqpoint{2.704245in}{2.247877in}}%
\pgfpathlineto{\pgfqpoint{2.713718in}{2.246340in}}%
\pgfpathlineto{\pgfqpoint{2.723190in}{2.252727in}}%
\pgfpathlineto{\pgfqpoint{2.751608in}{2.264551in}}%
\pgfpathlineto{\pgfqpoint{2.761080in}{2.272509in}}%
\pgfpathlineto{\pgfqpoint{2.770553in}{2.274393in}}%
\pgfpathlineto{\pgfqpoint{2.780025in}{2.280883in}}%
\pgfpathlineto{\pgfqpoint{2.789498in}{2.298233in}}%
\pgfpathlineto{\pgfqpoint{2.817915in}{2.315837in}}%
\pgfpathlineto{\pgfqpoint{2.846333in}{2.322865in}}%
\pgfpathlineto{\pgfqpoint{2.855805in}{2.321715in}}%
\pgfpathlineto{\pgfqpoint{2.865278in}{2.328008in}}%
\pgfpathlineto{\pgfqpoint{2.874750in}{2.331361in}}%
\pgfpathlineto{\pgfqpoint{2.884223in}{2.331782in}}%
\pgfpathlineto{\pgfqpoint{2.903168in}{2.335746in}}%
\pgfpathlineto{\pgfqpoint{2.912640in}{2.333720in}}%
\pgfpathlineto{\pgfqpoint{2.922113in}{2.338051in}}%
\pgfpathlineto{\pgfqpoint{2.941058in}{2.354647in}}%
\pgfpathlineto{\pgfqpoint{2.960003in}{2.362336in}}%
\pgfpathlineto{\pgfqpoint{2.988420in}{2.386102in}}%
\pgfpathlineto{\pgfqpoint{3.016838in}{2.396071in}}%
\pgfpathlineto{\pgfqpoint{3.026310in}{2.393942in}}%
\pgfpathlineto{\pgfqpoint{3.045255in}{2.406329in}}%
\pgfpathlineto{\pgfqpoint{3.054728in}{2.408410in}}%
\pgfpathlineto{\pgfqpoint{3.073673in}{2.406408in}}%
\pgfpathlineto{\pgfqpoint{3.083145in}{2.416519in}}%
\pgfpathlineto{\pgfqpoint{3.102090in}{2.419412in}}%
\pgfpathlineto{\pgfqpoint{3.111563in}{2.425211in}}%
\pgfpathlineto{\pgfqpoint{3.121035in}{2.428074in}}%
\pgfpathlineto{\pgfqpoint{3.139980in}{2.440657in}}%
\pgfpathlineto{\pgfqpoint{3.149453in}{2.440295in}}%
\pgfpathlineto{\pgfqpoint{3.158925in}{2.444920in}}%
\pgfpathlineto{\pgfqpoint{3.168398in}{2.455227in}}%
\pgfpathlineto{\pgfqpoint{3.177870in}{2.457992in}}%
\pgfpathlineto{\pgfqpoint{3.196815in}{2.469498in}}%
\pgfpathlineto{\pgfqpoint{3.225233in}{2.477016in}}%
\pgfpathlineto{\pgfqpoint{3.244178in}{2.487054in}}%
\pgfpathlineto{\pgfqpoint{3.263123in}{2.492002in}}%
\pgfpathlineto{\pgfqpoint{3.272595in}{2.491830in}}%
\pgfpathlineto{\pgfqpoint{3.282068in}{2.493328in}}%
\pgfpathlineto{\pgfqpoint{3.291540in}{2.498638in}}%
\pgfpathlineto{\pgfqpoint{3.301013in}{2.500429in}}%
\pgfpathlineto{\pgfqpoint{3.310485in}{2.508774in}}%
\pgfpathlineto{\pgfqpoint{3.319958in}{2.509288in}}%
\pgfpathlineto{\pgfqpoint{3.329430in}{2.513722in}}%
\pgfpathlineto{\pgfqpoint{3.338903in}{2.515704in}}%
\pgfpathlineto{\pgfqpoint{3.376793in}{2.534311in}}%
\pgfpathlineto{\pgfqpoint{3.386265in}{2.530915in}}%
\pgfpathlineto{\pgfqpoint{3.405210in}{2.534096in}}%
\pgfpathlineto{\pgfqpoint{3.414682in}{2.542641in}}%
\pgfpathlineto{\pgfqpoint{3.433627in}{2.549253in}}%
\pgfpathlineto{\pgfqpoint{3.443100in}{2.548103in}}%
\pgfpathlineto{\pgfqpoint{3.452572in}{2.550090in}}%
\pgfpathlineto{\pgfqpoint{3.462045in}{2.549429in}}%
\pgfpathlineto{\pgfqpoint{3.480990in}{2.561523in}}%
\pgfpathlineto{\pgfqpoint{3.490462in}{2.567225in}}%
\pgfpathlineto{\pgfqpoint{3.499935in}{2.562751in}}%
\pgfpathlineto{\pgfqpoint{3.547297in}{2.577566in}}%
\pgfpathlineto{\pgfqpoint{3.556770in}{2.582876in}}%
\pgfpathlineto{\pgfqpoint{3.566242in}{2.579382in}}%
\pgfpathlineto{\pgfqpoint{3.585187in}{2.586581in}}%
\pgfpathlineto{\pgfqpoint{3.604132in}{2.593188in}}%
\pgfpathlineto{\pgfqpoint{3.613605in}{2.593315in}}%
\pgfpathlineto{\pgfqpoint{3.623077in}{2.588935in}}%
\pgfpathlineto{\pgfqpoint{3.632550in}{2.590824in}}%
\pgfpathlineto{\pgfqpoint{3.642022in}{2.598386in}}%
\pgfpathlineto{\pgfqpoint{3.651495in}{2.602526in}}%
\pgfpathlineto{\pgfqpoint{3.660967in}{2.602746in}}%
\pgfpathlineto{\pgfqpoint{3.670440in}{2.606887in}}%
\pgfpathlineto{\pgfqpoint{3.679912in}{2.616112in}}%
\pgfpathlineto{\pgfqpoint{3.689385in}{2.618584in}}%
\pgfpathlineto{\pgfqpoint{3.698857in}{2.626542in}}%
\pgfpathlineto{\pgfqpoint{3.708330in}{2.626958in}}%
\pgfpathlineto{\pgfqpoint{3.717802in}{2.634818in}}%
\pgfpathlineto{\pgfqpoint{3.727275in}{2.638464in}}%
\pgfpathlineto{\pgfqpoint{3.736747in}{2.639276in}}%
\pgfpathlineto{\pgfqpoint{3.755692in}{2.648820in}}%
\pgfpathlineto{\pgfqpoint{3.774637in}{2.659249in}}%
\pgfpathlineto{\pgfqpoint{3.784110in}{2.665445in}}%
\pgfpathlineto{\pgfqpoint{3.803055in}{2.673819in}}%
\pgfpathlineto{\pgfqpoint{3.812527in}{2.674627in}}%
\pgfpathlineto{\pgfqpoint{3.831472in}{2.687209in}}%
\pgfpathlineto{\pgfqpoint{3.840945in}{2.688800in}}%
\pgfpathlineto{\pgfqpoint{3.859890in}{2.697565in}}%
\pgfpathlineto{\pgfqpoint{3.878835in}{2.698696in}}%
\pgfpathlineto{\pgfqpoint{3.888307in}{2.703615in}}%
\pgfpathlineto{\pgfqpoint{3.897780in}{2.711475in}}%
\pgfpathlineto{\pgfqpoint{3.907252in}{2.712282in}}%
\pgfpathlineto{\pgfqpoint{3.935670in}{2.719511in}}%
\pgfpathlineto{\pgfqpoint{3.945142in}{2.716403in}}%
\pgfpathlineto{\pgfqpoint{3.964087in}{2.721742in}}%
\pgfpathlineto{\pgfqpoint{4.001977in}{2.733498in}}%
\pgfpathlineto{\pgfqpoint{4.011450in}{2.730586in}}%
\pgfpathlineto{\pgfqpoint{4.020922in}{2.735314in}}%
\pgfpathlineto{\pgfqpoint{4.030395in}{2.736905in}}%
\pgfpathlineto{\pgfqpoint{4.039867in}{2.733503in}}%
\pgfpathlineto{\pgfqpoint{4.049340in}{2.734609in}}%
\pgfpathlineto{\pgfqpoint{4.068285in}{2.745430in}}%
\pgfpathlineto{\pgfqpoint{4.077757in}{2.745846in}}%
\pgfpathlineto{\pgfqpoint{4.096702in}{2.757744in}}%
\pgfpathlineto{\pgfqpoint{4.106175in}{2.760411in}}%
\pgfpathlineto{\pgfqpoint{4.115647in}{2.768663in}}%
\pgfpathlineto{\pgfqpoint{4.134592in}{2.778798in}}%
\pgfpathlineto{\pgfqpoint{4.144065in}{2.782738in}}%
\pgfpathlineto{\pgfqpoint{4.153537in}{2.780125in}}%
\pgfpathlineto{\pgfqpoint{4.163010in}{2.780345in}}%
\pgfpathlineto{\pgfqpoint{4.172482in}{2.787417in}}%
\pgfpathlineto{\pgfqpoint{4.191427in}{2.791093in}}%
\pgfpathlineto{\pgfqpoint{4.200900in}{2.796603in}}%
\pgfpathlineto{\pgfqpoint{4.210372in}{2.795943in}}%
\pgfpathlineto{\pgfqpoint{4.219845in}{2.800279in}}%
\pgfpathlineto{\pgfqpoint{4.229317in}{2.801478in}}%
\pgfpathlineto{\pgfqpoint{4.238790in}{2.806206in}}%
\pgfpathlineto{\pgfqpoint{4.248262in}{2.808188in}}%
\pgfpathlineto{\pgfqpoint{4.267207in}{2.807850in}}%
\pgfpathlineto{\pgfqpoint{4.276680in}{2.810322in}}%
\pgfpathlineto{\pgfqpoint{4.286152in}{2.818182in}}%
\pgfpathlineto{\pgfqpoint{4.295625in}{2.812333in}}%
\pgfpathlineto{\pgfqpoint{4.305097in}{2.811677in}}%
\pgfpathlineto{\pgfqpoint{4.333515in}{2.819097in}}%
\pgfpathlineto{\pgfqpoint{4.342987in}{2.821177in}}%
\pgfpathlineto{\pgfqpoint{4.361932in}{2.834151in}}%
\pgfpathlineto{\pgfqpoint{4.371405in}{2.835551in}}%
\pgfpathlineto{\pgfqpoint{4.380877in}{2.839491in}}%
\pgfpathlineto{\pgfqpoint{4.390350in}{2.839907in}}%
\pgfpathlineto{\pgfqpoint{4.399822in}{2.842481in}}%
\pgfpathlineto{\pgfqpoint{4.428240in}{2.859493in}}%
\pgfpathlineto{\pgfqpoint{4.437712in}{2.866472in}}%
\pgfpathlineto{\pgfqpoint{4.447185in}{2.870118in}}%
\pgfpathlineto{\pgfqpoint{4.456657in}{2.876701in}}%
\pgfpathlineto{\pgfqpoint{4.466130in}{2.879079in}}%
\pgfpathlineto{\pgfqpoint{4.475602in}{2.887130in}}%
\pgfpathlineto{\pgfqpoint{4.485075in}{2.889215in}}%
\pgfpathlineto{\pgfqpoint{4.504020in}{2.896415in}}%
\pgfpathlineto{\pgfqpoint{4.513492in}{2.895656in}}%
\pgfpathlineto{\pgfqpoint{4.541910in}{2.906697in}}%
\pgfpathlineto{\pgfqpoint{4.560855in}{2.906555in}}%
\pgfpathlineto{\pgfqpoint{4.579800in}{2.915223in}}%
\pgfpathlineto{\pgfqpoint{4.589272in}{2.913490in}}%
\pgfpathlineto{\pgfqpoint{4.598745in}{2.918507in}}%
\pgfpathlineto{\pgfqpoint{4.608217in}{2.917944in}}%
\pgfpathlineto{\pgfqpoint{4.636635in}{2.931633in}}%
\pgfpathlineto{\pgfqpoint{4.655580in}{2.933644in}}%
\pgfpathlineto{\pgfqpoint{4.665052in}{2.932200in}}%
\pgfpathlineto{\pgfqpoint{4.674525in}{2.928995in}}%
\pgfpathlineto{\pgfqpoint{4.712415in}{2.948288in}}%
\pgfpathlineto{\pgfqpoint{4.731360in}{2.953138in}}%
\pgfpathlineto{\pgfqpoint{4.740832in}{2.958355in}}%
\pgfpathlineto{\pgfqpoint{4.750305in}{2.961512in}}%
\pgfpathlineto{\pgfqpoint{4.759777in}{2.969073in}}%
\pgfpathlineto{\pgfqpoint{4.769250in}{2.972332in}}%
\pgfpathlineto{\pgfqpoint{4.788195in}{2.974638in}}%
\pgfpathlineto{\pgfqpoint{4.797667in}{2.982003in}}%
\pgfpathlineto{\pgfqpoint{4.807140in}{2.987123in}}%
\pgfpathlineto{\pgfqpoint{4.816612in}{2.985777in}}%
\pgfpathlineto{\pgfqpoint{4.845030in}{3.004257in}}%
\pgfpathlineto{\pgfqpoint{4.854502in}{3.003210in}}%
\pgfpathlineto{\pgfqpoint{4.901865in}{3.017535in}}%
\pgfpathlineto{\pgfqpoint{4.911337in}{3.023334in}}%
\pgfpathlineto{\pgfqpoint{4.939755in}{3.025473in}}%
\pgfpathlineto{\pgfqpoint{4.958700in}{3.031199in}}%
\pgfpathlineto{\pgfqpoint{4.977645in}{3.033407in}}%
\pgfpathlineto{\pgfqpoint{4.987117in}{3.041169in}}%
\pgfpathlineto{\pgfqpoint{5.006062in}{3.043767in}}%
\pgfpathlineto{\pgfqpoint{5.053425in}{3.061127in}}%
\pgfpathlineto{\pgfqpoint{5.081842in}{3.071483in}}%
\pgfpathlineto{\pgfqpoint{5.091315in}{3.077092in}}%
\pgfpathlineto{\pgfqpoint{5.110260in}{3.082720in}}%
\pgfpathlineto{\pgfqpoint{5.119732in}{3.087350in}}%
\pgfpathlineto{\pgfqpoint{5.138677in}{3.102184in}}%
\pgfpathlineto{\pgfqpoint{5.157622in}{3.107621in}}%
\pgfpathlineto{\pgfqpoint{5.167094in}{3.108429in}}%
\pgfpathlineto{\pgfqpoint{5.176567in}{3.105321in}}%
\pgfpathlineto{\pgfqpoint{5.186039in}{3.110636in}}%
\pgfpathlineto{\pgfqpoint{5.195512in}{3.111150in}}%
\pgfpathlineto{\pgfqpoint{5.214457in}{3.116979in}}%
\pgfpathlineto{\pgfqpoint{5.223929in}{3.114561in}}%
\pgfpathlineto{\pgfqpoint{5.233402in}{3.115075in}}%
\pgfpathlineto{\pgfqpoint{5.242874in}{3.110401in}}%
\pgfpathlineto{\pgfqpoint{5.252347in}{3.111312in}}%
\pgfpathlineto{\pgfqpoint{5.261819in}{3.117013in}}%
\pgfpathlineto{\pgfqpoint{5.271292in}{3.119196in}}%
\pgfpathlineto{\pgfqpoint{5.280764in}{3.124212in}}%
\pgfpathlineto{\pgfqpoint{5.290237in}{3.130996in}}%
\pgfpathlineto{\pgfqpoint{5.309182in}{3.135650in}}%
\pgfpathlineto{\pgfqpoint{5.328127in}{3.141474in}}%
\pgfpathlineto{\pgfqpoint{5.337599in}{3.143168in}}%
\pgfpathlineto{\pgfqpoint{5.347072in}{3.148184in}}%
\pgfpathlineto{\pgfqpoint{5.366017in}{3.161256in}}%
\pgfpathlineto{\pgfqpoint{5.375489in}{3.164712in}}%
\pgfpathlineto{\pgfqpoint{5.384962in}{3.170903in}}%
\pgfpathlineto{\pgfqpoint{5.394434in}{3.171514in}}%
\pgfpathlineto{\pgfqpoint{5.403907in}{3.169390in}}%
\pgfpathlineto{\pgfqpoint{5.413379in}{3.173428in}}%
\pgfpathlineto{\pgfqpoint{5.432324in}{3.175048in}}%
\pgfpathlineto{\pgfqpoint{5.441797in}{3.174196in}}%
\pgfpathlineto{\pgfqpoint{5.451269in}{3.175493in}}%
\pgfpathlineto{\pgfqpoint{5.460742in}{3.174539in}}%
\pgfpathlineto{\pgfqpoint{5.470214in}{3.177113in}}%
\pgfpathlineto{\pgfqpoint{5.479687in}{3.184577in}}%
\pgfpathlineto{\pgfqpoint{5.489159in}{3.187738in}}%
\pgfpathlineto{\pgfqpoint{5.489159in}{3.187738in}}%
\pgfusepath{stroke}%
\end{pgfscope}%
\begin{pgfscope}%
\pgfpathrectangle{\pgfqpoint{0.762383in}{0.471179in}}{\pgfqpoint{4.726776in}{2.845920in}} %
\pgfusepath{clip}%
\pgfsetbuttcap%
\pgfsetroundjoin%
\definecolor{currentfill}{rgb}{0.580392,0.403922,0.741176}%
\pgfsetfillcolor{currentfill}%
\pgfsetlinewidth{1.003750pt}%
\definecolor{currentstroke}{rgb}{0.580392,0.403922,0.741176}%
\pgfsetstrokecolor{currentstroke}%
\pgfsetdash{}{0pt}%
\pgfsys@defobject{currentmarker}{\pgfqpoint{-0.020833in}{-0.020833in}}{\pgfqpoint{0.020833in}{0.020833in}}{%
\pgfpathmoveto{\pgfqpoint{0.000000in}{-0.020833in}}%
\pgfpathcurveto{\pgfqpoint{0.005525in}{-0.020833in}}{\pgfqpoint{0.010825in}{-0.018638in}}{\pgfqpoint{0.014731in}{-0.014731in}}%
\pgfpathcurveto{\pgfqpoint{0.018638in}{-0.010825in}}{\pgfqpoint{0.020833in}{-0.005525in}}{\pgfqpoint{0.020833in}{0.000000in}}%
\pgfpathcurveto{\pgfqpoint{0.020833in}{0.005525in}}{\pgfqpoint{0.018638in}{0.010825in}}{\pgfqpoint{0.014731in}{0.014731in}}%
\pgfpathcurveto{\pgfqpoint{0.010825in}{0.018638in}}{\pgfqpoint{0.005525in}{0.020833in}}{\pgfqpoint{0.000000in}{0.020833in}}%
\pgfpathcurveto{\pgfqpoint{-0.005525in}{0.020833in}}{\pgfqpoint{-0.010825in}{0.018638in}}{\pgfqpoint{-0.014731in}{0.014731in}}%
\pgfpathcurveto{\pgfqpoint{-0.018638in}{0.010825in}}{\pgfqpoint{-0.020833in}{0.005525in}}{\pgfqpoint{-0.020833in}{0.000000in}}%
\pgfpathcurveto{\pgfqpoint{-0.020833in}{-0.005525in}}{\pgfqpoint{-0.018638in}{-0.010825in}}{\pgfqpoint{-0.014731in}{-0.014731in}}%
\pgfpathcurveto{\pgfqpoint{-0.010825in}{-0.018638in}}{\pgfqpoint{-0.005525in}{-0.020833in}}{\pgfqpoint{0.000000in}{-0.020833in}}%
\pgfpathclose%
\pgfusepath{stroke,fill}%
}%
\begin{pgfscope}%
\pgfsys@transformshift{0.762383in}{0.600539in}%
\pgfsys@useobject{currentmarker}{}%
\end{pgfscope}%
\begin{pgfscope}%
\pgfsys@transformshift{0.771856in}{0.688562in}%
\pgfsys@useobject{currentmarker}{}%
\end{pgfscope}%
\begin{pgfscope}%
\pgfsys@transformshift{0.781328in}{0.724022in}%
\pgfsys@useobject{currentmarker}{}%
\end{pgfscope}%
\begin{pgfscope}%
\pgfsys@transformshift{0.790801in}{0.751945in}%
\pgfsys@useobject{currentmarker}{}%
\end{pgfscope}%
\begin{pgfscope}%
\pgfsys@transformshift{0.800273in}{0.776541in}%
\pgfsys@useobject{currentmarker}{}%
\end{pgfscope}%
\begin{pgfscope}%
\pgfsys@transformshift{0.809746in}{0.797025in}%
\pgfsys@useobject{currentmarker}{}%
\end{pgfscope}%
\begin{pgfscope}%
\pgfsys@transformshift{0.819218in}{0.819956in}%
\pgfsys@useobject{currentmarker}{}%
\end{pgfscope}%
\begin{pgfscope}%
\pgfsys@transformshift{0.828691in}{0.840538in}%
\pgfsys@useobject{currentmarker}{}%
\end{pgfscope}%
\begin{pgfscope}%
\pgfsys@transformshift{0.838163in}{0.865623in}%
\pgfsys@useobject{currentmarker}{}%
\end{pgfscope}%
\begin{pgfscope}%
\pgfsys@transformshift{0.847636in}{0.881506in}%
\pgfsys@useobject{currentmarker}{}%
\end{pgfscope}%
\begin{pgfscope}%
\pgfsys@transformshift{0.857108in}{0.897782in}%
\pgfsys@useobject{currentmarker}{}%
\end{pgfscope}%
\begin{pgfscope}%
\pgfsys@transformshift{0.866581in}{0.916112in}%
\pgfsys@useobject{currentmarker}{}%
\end{pgfscope}%
\begin{pgfscope}%
\pgfsys@transformshift{0.876053in}{0.936890in}%
\pgfsys@useobject{currentmarker}{}%
\end{pgfscope}%
\begin{pgfscope}%
\pgfsys@transformshift{0.885526in}{0.960898in}%
\pgfsys@useobject{currentmarker}{}%
\end{pgfscope}%
\begin{pgfscope}%
\pgfsys@transformshift{0.894998in}{0.976292in}%
\pgfsys@useobject{currentmarker}{}%
\end{pgfscope}%
\begin{pgfscope}%
\pgfsys@transformshift{0.904471in}{0.990022in}%
\pgfsys@useobject{currentmarker}{}%
\end{pgfscope}%
\begin{pgfscope}%
\pgfsys@transformshift{0.913943in}{1.002480in}%
\pgfsys@useobject{currentmarker}{}%
\end{pgfscope}%
\begin{pgfscope}%
\pgfsys@transformshift{0.923416in}{1.015232in}%
\pgfsys@useobject{currentmarker}{}%
\end{pgfscope}%
\begin{pgfscope}%
\pgfsys@transformshift{0.932888in}{1.027688in}%
\pgfsys@useobject{currentmarker}{}%
\end{pgfscope}%
\begin{pgfscope}%
\pgfsys@transformshift{0.942361in}{1.042008in}%
\pgfsys@useobject{currentmarker}{}%
\end{pgfscope}%
\begin{pgfscope}%
\pgfsys@transformshift{0.951833in}{1.057988in}%
\pgfsys@useobject{currentmarker}{}%
\end{pgfscope}%
\begin{pgfscope}%
\pgfsys@transformshift{0.961306in}{1.069469in}%
\pgfsys@useobject{currentmarker}{}%
\end{pgfscope}%
\begin{pgfscope}%
\pgfsys@transformshift{0.970778in}{1.083687in}%
\pgfsys@useobject{currentmarker}{}%
\end{pgfscope}%
\begin{pgfscope}%
\pgfsys@transformshift{0.980251in}{1.098394in}%
\pgfsys@useobject{currentmarker}{}%
\end{pgfscope}%
\begin{pgfscope}%
\pgfsys@transformshift{0.989723in}{1.112322in}%
\pgfsys@useobject{currentmarker}{}%
\end{pgfscope}%
\begin{pgfscope}%
\pgfsys@transformshift{0.999196in}{1.122233in}%
\pgfsys@useobject{currentmarker}{}%
\end{pgfscope}%
\begin{pgfscope}%
\pgfsys@transformshift{1.008668in}{1.132442in}%
\pgfsys@useobject{currentmarker}{}%
\end{pgfscope}%
\begin{pgfscope}%
\pgfsys@transformshift{1.018141in}{1.148128in}%
\pgfsys@useobject{currentmarker}{}%
\end{pgfscope}%
\begin{pgfscope}%
\pgfsys@transformshift{1.027613in}{1.154618in}%
\pgfsys@useobject{currentmarker}{}%
\end{pgfscope}%
\begin{pgfscope}%
\pgfsys@transformshift{1.037086in}{1.161004in}%
\pgfsys@useobject{currentmarker}{}%
\end{pgfscope}%
\begin{pgfscope}%
\pgfsys@transformshift{1.046558in}{1.167783in}%
\pgfsys@useobject{currentmarker}{}%
\end{pgfscope}%
\begin{pgfscope}%
\pgfsys@transformshift{1.056031in}{1.181222in}%
\pgfsys@useobject{currentmarker}{}%
\end{pgfscope}%
\begin{pgfscope}%
\pgfsys@transformshift{1.065503in}{1.195244in}%
\pgfsys@useobject{currentmarker}{}%
\end{pgfscope}%
\begin{pgfscope}%
\pgfsys@transformshift{1.074976in}{1.205943in}%
\pgfsys@useobject{currentmarker}{}%
\end{pgfscope}%
\begin{pgfscope}%
\pgfsys@transformshift{1.084448in}{1.218007in}%
\pgfsys@useobject{currentmarker}{}%
\end{pgfscope}%
\begin{pgfscope}%
\pgfsys@transformshift{1.093921in}{1.231446in}%
\pgfsys@useobject{currentmarker}{}%
\end{pgfscope}%
\begin{pgfscope}%
\pgfsys@transformshift{1.103393in}{1.240182in}%
\pgfsys@useobject{currentmarker}{}%
\end{pgfscope}%
\begin{pgfscope}%
\pgfsys@transformshift{1.112866in}{1.248331in}%
\pgfsys@useobject{currentmarker}{}%
\end{pgfscope}%
\begin{pgfscope}%
\pgfsys@transformshift{1.122338in}{1.254429in}%
\pgfsys@useobject{currentmarker}{}%
\end{pgfscope}%
\begin{pgfscope}%
\pgfsys@transformshift{1.131811in}{1.260033in}%
\pgfsys@useobject{currentmarker}{}%
\end{pgfscope}%
\begin{pgfscope}%
\pgfsys@transformshift{1.141283in}{1.270340in}%
\pgfsys@useobject{currentmarker}{}%
\end{pgfscope}%
\begin{pgfscope}%
\pgfsys@transformshift{1.150756in}{1.278684in}%
\pgfsys@useobject{currentmarker}{}%
\end{pgfscope}%
\begin{pgfscope}%
\pgfsys@transformshift{1.160228in}{1.292711in}%
\pgfsys@useobject{currentmarker}{}%
\end{pgfscope}%
\begin{pgfscope}%
\pgfsys@transformshift{1.169701in}{1.301545in}%
\pgfsys@useobject{currentmarker}{}%
\end{pgfscope}%
\begin{pgfscope}%
\pgfsys@transformshift{1.179173in}{1.310090in}%
\pgfsys@useobject{currentmarker}{}%
\end{pgfscope}%
\begin{pgfscope}%
\pgfsys@transformshift{1.188646in}{1.319707in}%
\pgfsys@useobject{currentmarker}{}%
\end{pgfscope}%
\begin{pgfscope}%
\pgfsys@transformshift{1.198118in}{1.324919in}%
\pgfsys@useobject{currentmarker}{}%
\end{pgfscope}%
\begin{pgfscope}%
\pgfsys@transformshift{1.207591in}{1.336401in}%
\pgfsys@useobject{currentmarker}{}%
\end{pgfscope}%
\begin{pgfscope}%
\pgfsys@transformshift{1.217063in}{1.345822in}%
\pgfsys@useobject{currentmarker}{}%
\end{pgfscope}%
\begin{pgfscope}%
\pgfsys@transformshift{1.226536in}{1.357793in}%
\pgfsys@useobject{currentmarker}{}%
\end{pgfscope}%
\begin{pgfscope}%
\pgfsys@transformshift{1.236008in}{1.370640in}%
\pgfsys@useobject{currentmarker}{}%
\end{pgfscope}%
\begin{pgfscope}%
\pgfsys@transformshift{1.245481in}{1.380849in}%
\pgfsys@useobject{currentmarker}{}%
\end{pgfscope}%
\begin{pgfscope}%
\pgfsys@transformshift{1.254953in}{1.385964in}%
\pgfsys@useobject{currentmarker}{}%
\end{pgfscope}%
\begin{pgfscope}%
\pgfsys@transformshift{1.264426in}{1.396853in}%
\pgfsys@useobject{currentmarker}{}%
\end{pgfscope}%
\begin{pgfscope}%
\pgfsys@transformshift{1.273898in}{1.409020in}%
\pgfsys@useobject{currentmarker}{}%
\end{pgfscope}%
\begin{pgfscope}%
\pgfsys@transformshift{1.283371in}{1.419224in}%
\pgfsys@useobject{currentmarker}{}%
\end{pgfscope}%
\begin{pgfscope}%
\pgfsys@transformshift{1.292843in}{1.427770in}%
\pgfsys@useobject{currentmarker}{}%
\end{pgfscope}%
\begin{pgfscope}%
\pgfsys@transformshift{1.302316in}{1.435331in}%
\pgfsys@useobject{currentmarker}{}%
\end{pgfscope}%
\begin{pgfscope}%
\pgfsys@transformshift{1.311788in}{1.446128in}%
\pgfsys@useobject{currentmarker}{}%
\end{pgfscope}%
\begin{pgfscope}%
\pgfsys@transformshift{1.321261in}{1.451438in}%
\pgfsys@useobject{currentmarker}{}%
\end{pgfscope}%
\begin{pgfscope}%
\pgfsys@transformshift{1.330733in}{1.455769in}%
\pgfsys@useobject{currentmarker}{}%
\end{pgfscope}%
\begin{pgfscope}%
\pgfsys@transformshift{1.340206in}{1.465195in}%
\pgfsys@useobject{currentmarker}{}%
\end{pgfscope}%
\begin{pgfscope}%
\pgfsys@transformshift{1.349678in}{1.472855in}%
\pgfsys@useobject{currentmarker}{}%
\end{pgfscope}%
\begin{pgfscope}%
\pgfsys@transformshift{1.359151in}{1.480323in}%
\pgfsys@useobject{currentmarker}{}%
\end{pgfscope}%
\begin{pgfscope}%
\pgfsys@transformshift{1.368623in}{1.485829in}%
\pgfsys@useobject{currentmarker}{}%
\end{pgfscope}%
\begin{pgfscope}%
\pgfsys@transformshift{1.378096in}{1.493885in}%
\pgfsys@useobject{currentmarker}{}%
\end{pgfscope}%
\begin{pgfscope}%
\pgfsys@transformshift{1.387568in}{1.501740in}%
\pgfsys@useobject{currentmarker}{}%
\end{pgfscope}%
\begin{pgfscope}%
\pgfsys@transformshift{1.397041in}{1.507931in}%
\pgfsys@useobject{currentmarker}{}%
\end{pgfscope}%
\begin{pgfscope}%
\pgfsys@transformshift{1.406513in}{1.517161in}%
\pgfsys@useobject{currentmarker}{}%
\end{pgfscope}%
\begin{pgfscope}%
\pgfsys@transformshift{1.415986in}{1.525310in}%
\pgfsys@useobject{currentmarker}{}%
\end{pgfscope}%
\begin{pgfscope}%
\pgfsys@transformshift{1.425458in}{1.530331in}%
\pgfsys@useobject{currentmarker}{}%
\end{pgfscope}%
\begin{pgfscope}%
\pgfsys@transformshift{1.434931in}{1.536327in}%
\pgfsys@useobject{currentmarker}{}%
\end{pgfscope}%
\begin{pgfscope}%
\pgfsys@transformshift{1.444403in}{1.540565in}%
\pgfsys@useobject{currentmarker}{}%
\end{pgfscope}%
\begin{pgfscope}%
\pgfsys@transformshift{1.453876in}{1.543330in}%
\pgfsys@useobject{currentmarker}{}%
\end{pgfscope}%
\begin{pgfscope}%
\pgfsys@transformshift{1.463348in}{1.549918in}%
\pgfsys@useobject{currentmarker}{}%
\end{pgfscope}%
\begin{pgfscope}%
\pgfsys@transformshift{1.472821in}{1.561395in}%
\pgfsys@useobject{currentmarker}{}%
\end{pgfscope}%
\begin{pgfscope}%
\pgfsys@transformshift{1.482293in}{1.568173in}%
\pgfsys@useobject{currentmarker}{}%
\end{pgfscope}%
\begin{pgfscope}%
\pgfsys@transformshift{1.491766in}{1.578284in}%
\pgfsys@useobject{currentmarker}{}%
\end{pgfscope}%
\begin{pgfscope}%
\pgfsys@transformshift{1.501238in}{1.588684in}%
\pgfsys@useobject{currentmarker}{}%
\end{pgfscope}%
\begin{pgfscope}%
\pgfsys@transformshift{1.510711in}{1.600557in}%
\pgfsys@useobject{currentmarker}{}%
\end{pgfscope}%
\begin{pgfscope}%
\pgfsys@transformshift{1.520183in}{1.605280in}%
\pgfsys@useobject{currentmarker}{}%
\end{pgfscope}%
\begin{pgfscope}%
\pgfsys@transformshift{1.529656in}{1.613825in}%
\pgfsys@useobject{currentmarker}{}%
\end{pgfscope}%
\begin{pgfscope}%
\pgfsys@transformshift{1.539128in}{1.624030in}%
\pgfsys@useobject{currentmarker}{}%
\end{pgfscope}%
\begin{pgfscope}%
\pgfsys@transformshift{1.548601in}{1.634038in}%
\pgfsys@useobject{currentmarker}{}%
\end{pgfscope}%
\begin{pgfscope}%
\pgfsys@transformshift{1.558073in}{1.641409in}%
\pgfsys@useobject{currentmarker}{}%
\end{pgfscope}%
\begin{pgfscope}%
\pgfsys@transformshift{1.567546in}{1.648872in}%
\pgfsys@useobject{currentmarker}{}%
\end{pgfscope}%
\begin{pgfscope}%
\pgfsys@transformshift{1.577018in}{1.656439in}%
\pgfsys@useobject{currentmarker}{}%
\end{pgfscope}%
\begin{pgfscope}%
\pgfsys@transformshift{1.586491in}{1.663119in}%
\pgfsys@useobject{currentmarker}{}%
\end{pgfscope}%
\begin{pgfscope}%
\pgfsys@transformshift{1.595963in}{1.670000in}%
\pgfsys@useobject{currentmarker}{}%
\end{pgfscope}%
\begin{pgfscope}%
\pgfsys@transformshift{1.605436in}{1.670123in}%
\pgfsys@useobject{currentmarker}{}%
\end{pgfscope}%
\begin{pgfscope}%
\pgfsys@transformshift{1.614908in}{1.677782in}%
\pgfsys@useobject{currentmarker}{}%
\end{pgfscope}%
\begin{pgfscope}%
\pgfsys@transformshift{1.624381in}{1.683489in}%
\pgfsys@useobject{currentmarker}{}%
\end{pgfscope}%
\begin{pgfscope}%
\pgfsys@transformshift{1.633853in}{1.690169in}%
\pgfsys@useobject{currentmarker}{}%
\end{pgfscope}%
\begin{pgfscope}%
\pgfsys@transformshift{1.643325in}{1.697246in}%
\pgfsys@useobject{currentmarker}{}%
\end{pgfscope}%
\begin{pgfscope}%
\pgfsys@transformshift{1.652798in}{1.707059in}%
\pgfsys@useobject{currentmarker}{}%
\end{pgfscope}%
\begin{pgfscope}%
\pgfsys@transformshift{1.662270in}{1.709633in}%
\pgfsys@useobject{currentmarker}{}%
\end{pgfscope}%
\begin{pgfscope}%
\pgfsys@transformshift{1.671743in}{1.717586in}%
\pgfsys@useobject{currentmarker}{}%
\end{pgfscope}%
\begin{pgfscope}%
\pgfsys@transformshift{1.681215in}{1.722505in}%
\pgfsys@useobject{currentmarker}{}%
\end{pgfscope}%
\begin{pgfscope}%
\pgfsys@transformshift{1.690688in}{1.731442in}%
\pgfsys@useobject{currentmarker}{}%
\end{pgfscope}%
\begin{pgfscope}%
\pgfsys@transformshift{1.700160in}{1.738611in}%
\pgfsys@useobject{currentmarker}{}%
\end{pgfscope}%
\begin{pgfscope}%
\pgfsys@transformshift{1.709633in}{1.742556in}%
\pgfsys@useobject{currentmarker}{}%
\end{pgfscope}%
\begin{pgfscope}%
\pgfsys@transformshift{1.719105in}{1.746692in}%
\pgfsys@useobject{currentmarker}{}%
\end{pgfscope}%
\begin{pgfscope}%
\pgfsys@transformshift{1.728578in}{1.753964in}%
\pgfsys@useobject{currentmarker}{}%
\end{pgfscope}%
\begin{pgfscope}%
\pgfsys@transformshift{1.738050in}{1.762211in}%
\pgfsys@useobject{currentmarker}{}%
\end{pgfscope}%
\begin{pgfscope}%
\pgfsys@transformshift{1.747523in}{1.771148in}%
\pgfsys@useobject{currentmarker}{}%
\end{pgfscope}%
\begin{pgfscope}%
\pgfsys@transformshift{1.756995in}{1.776458in}%
\pgfsys@useobject{currentmarker}{}%
\end{pgfscope}%
\begin{pgfscope}%
\pgfsys@transformshift{1.766468in}{1.783041in}%
\pgfsys@useobject{currentmarker}{}%
\end{pgfscope}%
\begin{pgfscope}%
\pgfsys@transformshift{1.775940in}{1.789922in}%
\pgfsys@useobject{currentmarker}{}%
\end{pgfscope}%
\begin{pgfscope}%
\pgfsys@transformshift{1.785413in}{1.792100in}%
\pgfsys@useobject{currentmarker}{}%
\end{pgfscope}%
\begin{pgfscope}%
\pgfsys@transformshift{1.794885in}{1.794968in}%
\pgfsys@useobject{currentmarker}{}%
\end{pgfscope}%
\begin{pgfscope}%
\pgfsys@transformshift{1.804358in}{1.803116in}%
\pgfsys@useobject{currentmarker}{}%
\end{pgfscope}%
\begin{pgfscope}%
\pgfsys@transformshift{1.813830in}{1.807746in}%
\pgfsys@useobject{currentmarker}{}%
\end{pgfscope}%
\begin{pgfscope}%
\pgfsys@transformshift{1.823303in}{1.814133in}%
\pgfsys@useobject{currentmarker}{}%
\end{pgfscope}%
\begin{pgfscope}%
\pgfsys@transformshift{1.832775in}{1.818562in}%
\pgfsys@useobject{currentmarker}{}%
\end{pgfscope}%
\begin{pgfscope}%
\pgfsys@transformshift{1.842248in}{1.824465in}%
\pgfsys@useobject{currentmarker}{}%
\end{pgfscope}%
\begin{pgfscope}%
\pgfsys@transformshift{1.851720in}{1.834082in}%
\pgfsys@useobject{currentmarker}{}%
\end{pgfscope}%
\begin{pgfscope}%
\pgfsys@transformshift{1.861193in}{1.834111in}%
\pgfsys@useobject{currentmarker}{}%
\end{pgfscope}%
\begin{pgfscope}%
\pgfsys@transformshift{1.870665in}{1.839519in}%
\pgfsys@useobject{currentmarker}{}%
\end{pgfscope}%
\begin{pgfscope}%
\pgfsys@transformshift{1.880138in}{1.845421in}%
\pgfsys@useobject{currentmarker}{}%
\end{pgfscope}%
\begin{pgfscope}%
\pgfsys@transformshift{1.889610in}{1.852298in}%
\pgfsys@useobject{currentmarker}{}%
\end{pgfscope}%
\begin{pgfscope}%
\pgfsys@transformshift{1.899083in}{1.854280in}%
\pgfsys@useobject{currentmarker}{}%
\end{pgfscope}%
\begin{pgfscope}%
\pgfsys@transformshift{1.908555in}{1.861063in}%
\pgfsys@useobject{currentmarker}{}%
\end{pgfscope}%
\begin{pgfscope}%
\pgfsys@transformshift{1.918028in}{1.867646in}%
\pgfsys@useobject{currentmarker}{}%
\end{pgfscope}%
\begin{pgfscope}%
\pgfsys@transformshift{1.927500in}{1.868556in}%
\pgfsys@useobject{currentmarker}{}%
\end{pgfscope}%
\begin{pgfscope}%
\pgfsys@transformshift{1.936973in}{1.870538in}%
\pgfsys@useobject{currentmarker}{}%
\end{pgfscope}%
\begin{pgfscope}%
\pgfsys@transformshift{1.946445in}{1.876636in}%
\pgfsys@useobject{currentmarker}{}%
\end{pgfscope}%
\begin{pgfscope}%
\pgfsys@transformshift{1.955918in}{1.881946in}%
\pgfsys@useobject{currentmarker}{}%
\end{pgfscope}%
\begin{pgfscope}%
\pgfsys@transformshift{1.965390in}{1.893717in}%
\pgfsys@useobject{currentmarker}{}%
\end{pgfscope}%
\begin{pgfscope}%
\pgfsys@transformshift{1.974863in}{1.898640in}%
\pgfsys@useobject{currentmarker}{}%
\end{pgfscope}%
\begin{pgfscope}%
\pgfsys@transformshift{1.984335in}{1.901895in}%
\pgfsys@useobject{currentmarker}{}%
\end{pgfscope}%
\begin{pgfscope}%
\pgfsys@transformshift{1.993808in}{1.907014in}%
\pgfsys@useobject{currentmarker}{}%
\end{pgfscope}%
\begin{pgfscope}%
\pgfsys@transformshift{2.003280in}{1.910954in}%
\pgfsys@useobject{currentmarker}{}%
\end{pgfscope}%
\begin{pgfscope}%
\pgfsys@transformshift{2.012753in}{1.917737in}%
\pgfsys@useobject{currentmarker}{}%
\end{pgfscope}%
\begin{pgfscope}%
\pgfsys@transformshift{2.022225in}{1.926669in}%
\pgfsys@useobject{currentmarker}{}%
\end{pgfscope}%
\begin{pgfscope}%
\pgfsys@transformshift{2.031698in}{1.932865in}%
\pgfsys@useobject{currentmarker}{}%
\end{pgfscope}%
\begin{pgfscope}%
\pgfsys@transformshift{2.041170in}{1.940916in}%
\pgfsys@useobject{currentmarker}{}%
\end{pgfscope}%
\begin{pgfscope}%
\pgfsys@transformshift{2.050643in}{1.949456in}%
\pgfsys@useobject{currentmarker}{}%
\end{pgfscope}%
\begin{pgfscope}%
\pgfsys@transformshift{2.060115in}{1.952520in}%
\pgfsys@useobject{currentmarker}{}%
\end{pgfscope}%
\begin{pgfscope}%
\pgfsys@transformshift{2.069588in}{1.954796in}%
\pgfsys@useobject{currentmarker}{}%
\end{pgfscope}%
\begin{pgfscope}%
\pgfsys@transformshift{2.079060in}{1.962460in}%
\pgfsys@useobject{currentmarker}{}%
\end{pgfscope}%
\begin{pgfscope}%
\pgfsys@transformshift{2.088533in}{1.966791in}%
\pgfsys@useobject{currentmarker}{}%
\end{pgfscope}%
\begin{pgfscope}%
\pgfsys@transformshift{2.098005in}{1.970442in}%
\pgfsys@useobject{currentmarker}{}%
\end{pgfscope}%
\begin{pgfscope}%
\pgfsys@transformshift{2.107478in}{1.975067in}%
\pgfsys@useobject{currentmarker}{}%
\end{pgfscope}%
\begin{pgfscope}%
\pgfsys@transformshift{2.116950in}{1.973819in}%
\pgfsys@useobject{currentmarker}{}%
\end{pgfscope}%
\begin{pgfscope}%
\pgfsys@transformshift{2.126423in}{1.979232in}%
\pgfsys@useobject{currentmarker}{}%
\end{pgfscope}%
\begin{pgfscope}%
\pgfsys@transformshift{2.135895in}{1.982976in}%
\pgfsys@useobject{currentmarker}{}%
\end{pgfscope}%
\begin{pgfscope}%
\pgfsys@transformshift{2.145368in}{1.987704in}%
\pgfsys@useobject{currentmarker}{}%
\end{pgfscope}%
\begin{pgfscope}%
\pgfsys@transformshift{2.154840in}{1.995657in}%
\pgfsys@useobject{currentmarker}{}%
\end{pgfscope}%
\begin{pgfscope}%
\pgfsys@transformshift{2.164313in}{2.000091in}%
\pgfsys@useobject{currentmarker}{}%
\end{pgfscope}%
\begin{pgfscope}%
\pgfsys@transformshift{2.173785in}{2.003150in}%
\pgfsys@useobject{currentmarker}{}%
\end{pgfscope}%
\begin{pgfscope}%
\pgfsys@transformshift{2.183258in}{2.006894in}%
\pgfsys@useobject{currentmarker}{}%
\end{pgfscope}%
\begin{pgfscope}%
\pgfsys@transformshift{2.192730in}{2.018082in}%
\pgfsys@useobject{currentmarker}{}%
\end{pgfscope}%
\begin{pgfscope}%
\pgfsys@transformshift{2.202203in}{2.026328in}%
\pgfsys@useobject{currentmarker}{}%
\end{pgfscope}%
\begin{pgfscope}%
\pgfsys@transformshift{2.211675in}{2.033210in}%
\pgfsys@useobject{currentmarker}{}%
\end{pgfscope}%
\begin{pgfscope}%
\pgfsys@transformshift{2.221148in}{2.040869in}%
\pgfsys@useobject{currentmarker}{}%
\end{pgfscope}%
\begin{pgfscope}%
\pgfsys@transformshift{2.230620in}{2.040507in}%
\pgfsys@useobject{currentmarker}{}%
\end{pgfscope}%
\begin{pgfscope}%
\pgfsys@transformshift{2.240093in}{2.044838in}%
\pgfsys@useobject{currentmarker}{}%
\end{pgfscope}%
\begin{pgfscope}%
\pgfsys@transformshift{2.249565in}{2.054455in}%
\pgfsys@useobject{currentmarker}{}%
\end{pgfscope}%
\begin{pgfscope}%
\pgfsys@transformshift{2.259038in}{2.061532in}%
\pgfsys@useobject{currentmarker}{}%
\end{pgfscope}%
\begin{pgfscope}%
\pgfsys@transformshift{2.268510in}{2.068408in}%
\pgfsys@useobject{currentmarker}{}%
\end{pgfscope}%
\begin{pgfscope}%
\pgfsys@transformshift{2.277983in}{2.076856in}%
\pgfsys@useobject{currentmarker}{}%
\end{pgfscope}%
\begin{pgfscope}%
\pgfsys@transformshift{2.287455in}{2.082655in}%
\pgfsys@useobject{currentmarker}{}%
\end{pgfscope}%
\begin{pgfscope}%
\pgfsys@transformshift{2.296928in}{2.084153in}%
\pgfsys@useobject{currentmarker}{}%
\end{pgfscope}%
\begin{pgfscope}%
\pgfsys@transformshift{2.306400in}{2.084471in}%
\pgfsys@useobject{currentmarker}{}%
\end{pgfscope}%
\begin{pgfscope}%
\pgfsys@transformshift{2.315873in}{2.083228in}%
\pgfsys@useobject{currentmarker}{}%
\end{pgfscope}%
\begin{pgfscope}%
\pgfsys@transformshift{2.325345in}{2.087951in}%
\pgfsys@useobject{currentmarker}{}%
\end{pgfscope}%
\begin{pgfscope}%
\pgfsys@transformshift{2.334818in}{2.090912in}%
\pgfsys@useobject{currentmarker}{}%
\end{pgfscope}%
\begin{pgfscope}%
\pgfsys@transformshift{2.344290in}{2.096422in}%
\pgfsys@useobject{currentmarker}{}%
\end{pgfscope}%
\begin{pgfscope}%
\pgfsys@transformshift{2.353763in}{2.104865in}%
\pgfsys@useobject{currentmarker}{}%
\end{pgfscope}%
\begin{pgfscope}%
\pgfsys@transformshift{2.363235in}{2.112040in}%
\pgfsys@useobject{currentmarker}{}%
\end{pgfscope}%
\begin{pgfscope}%
\pgfsys@transformshift{2.372708in}{2.112456in}%
\pgfsys@useobject{currentmarker}{}%
\end{pgfscope}%
\begin{pgfscope}%
\pgfsys@transformshift{2.382180in}{2.113855in}%
\pgfsys@useobject{currentmarker}{}%
\end{pgfscope}%
\begin{pgfscope}%
\pgfsys@transformshift{2.391653in}{2.119068in}%
\pgfsys@useobject{currentmarker}{}%
\end{pgfscope}%
\begin{pgfscope}%
\pgfsys@transformshift{2.401125in}{2.127412in}%
\pgfsys@useobject{currentmarker}{}%
\end{pgfscope}%
\begin{pgfscope}%
\pgfsys@transformshift{2.410598in}{2.130770in}%
\pgfsys@useobject{currentmarker}{}%
\end{pgfscope}%
\begin{pgfscope}%
\pgfsys@transformshift{2.420070in}{2.133828in}%
\pgfsys@useobject{currentmarker}{}%
\end{pgfscope}%
\begin{pgfscope}%
\pgfsys@transformshift{2.429543in}{2.142471in}%
\pgfsys@useobject{currentmarker}{}%
\end{pgfscope}%
\begin{pgfscope}%
\pgfsys@transformshift{2.439015in}{2.144258in}%
\pgfsys@useobject{currentmarker}{}%
\end{pgfscope}%
\begin{pgfscope}%
\pgfsys@transformshift{2.448488in}{2.145755in}%
\pgfsys@useobject{currentmarker}{}%
\end{pgfscope}%
\begin{pgfscope}%
\pgfsys@transformshift{2.457960in}{2.149597in}%
\pgfsys@useobject{currentmarker}{}%
\end{pgfscope}%
\begin{pgfscope}%
\pgfsys@transformshift{2.467433in}{2.151775in}%
\pgfsys@useobject{currentmarker}{}%
\end{pgfscope}%
\begin{pgfscope}%
\pgfsys@transformshift{2.476905in}{2.152490in}%
\pgfsys@useobject{currentmarker}{}%
\end{pgfscope}%
\begin{pgfscope}%
\pgfsys@transformshift{2.486378in}{2.156234in}%
\pgfsys@useobject{currentmarker}{}%
\end{pgfscope}%
\begin{pgfscope}%
\pgfsys@transformshift{2.495850in}{2.156948in}%
\pgfsys@useobject{currentmarker}{}%
\end{pgfscope}%
\begin{pgfscope}%
\pgfsys@transformshift{2.505323in}{2.160594in}%
\pgfsys@useobject{currentmarker}{}%
\end{pgfscope}%
\begin{pgfscope}%
\pgfsys@transformshift{2.514795in}{2.165910in}%
\pgfsys@useobject{currentmarker}{}%
\end{pgfscope}%
\begin{pgfscope}%
\pgfsys@transformshift{2.524268in}{2.170339in}%
\pgfsys@useobject{currentmarker}{}%
\end{pgfscope}%
\begin{pgfscope}%
\pgfsys@transformshift{2.533740in}{2.173789in}%
\pgfsys@useobject{currentmarker}{}%
\end{pgfscope}%
\begin{pgfscope}%
\pgfsys@transformshift{2.543213in}{2.179202in}%
\pgfsys@useobject{currentmarker}{}%
\end{pgfscope}%
\begin{pgfscope}%
\pgfsys@transformshift{2.552685in}{2.178835in}%
\pgfsys@useobject{currentmarker}{}%
\end{pgfscope}%
\begin{pgfscope}%
\pgfsys@transformshift{2.562158in}{2.183954in}%
\pgfsys@useobject{currentmarker}{}%
\end{pgfscope}%
\begin{pgfscope}%
\pgfsys@transformshift{2.571630in}{2.190439in}%
\pgfsys@useobject{currentmarker}{}%
\end{pgfscope}%
\begin{pgfscope}%
\pgfsys@transformshift{2.581103in}{2.194775in}%
\pgfsys@useobject{currentmarker}{}%
\end{pgfscope}%
\begin{pgfscope}%
\pgfsys@transformshift{2.590575in}{2.205371in}%
\pgfsys@useobject{currentmarker}{}%
\end{pgfscope}%
\begin{pgfscope}%
\pgfsys@transformshift{2.600048in}{2.206477in}%
\pgfsys@useobject{currentmarker}{}%
\end{pgfscope}%
\begin{pgfscope}%
\pgfsys@transformshift{2.609520in}{2.210025in}%
\pgfsys@useobject{currentmarker}{}%
\end{pgfscope}%
\begin{pgfscope}%
\pgfsys@transformshift{2.618993in}{2.214259in}%
\pgfsys@useobject{currentmarker}{}%
\end{pgfscope}%
\begin{pgfscope}%
\pgfsys@transformshift{2.628465in}{2.217420in}%
\pgfsys@useobject{currentmarker}{}%
\end{pgfscope}%
\begin{pgfscope}%
\pgfsys@transformshift{2.637938in}{2.219990in}%
\pgfsys@useobject{currentmarker}{}%
\end{pgfscope}%
\begin{pgfscope}%
\pgfsys@transformshift{2.647410in}{2.225403in}%
\pgfsys@useobject{currentmarker}{}%
\end{pgfscope}%
\begin{pgfscope}%
\pgfsys@transformshift{2.656883in}{2.225623in}%
\pgfsys@useobject{currentmarker}{}%
\end{pgfscope}%
\begin{pgfscope}%
\pgfsys@transformshift{2.666355in}{2.227414in}%
\pgfsys@useobject{currentmarker}{}%
\end{pgfscope}%
\begin{pgfscope}%
\pgfsys@transformshift{2.675828in}{2.225579in}%
\pgfsys@useobject{currentmarker}{}%
\end{pgfscope}%
\begin{pgfscope}%
\pgfsys@transformshift{2.685300in}{2.234413in}%
\pgfsys@useobject{currentmarker}{}%
\end{pgfscope}%
\begin{pgfscope}%
\pgfsys@transformshift{2.694773in}{2.242273in}%
\pgfsys@useobject{currentmarker}{}%
\end{pgfscope}%
\begin{pgfscope}%
\pgfsys@transformshift{2.704245in}{2.247877in}%
\pgfsys@useobject{currentmarker}{}%
\end{pgfscope}%
\begin{pgfscope}%
\pgfsys@transformshift{2.713718in}{2.246340in}%
\pgfsys@useobject{currentmarker}{}%
\end{pgfscope}%
\begin{pgfscope}%
\pgfsys@transformshift{2.723190in}{2.252727in}%
\pgfsys@useobject{currentmarker}{}%
\end{pgfscope}%
\begin{pgfscope}%
\pgfsys@transformshift{2.732663in}{2.256476in}%
\pgfsys@useobject{currentmarker}{}%
\end{pgfscope}%
\begin{pgfscope}%
\pgfsys@transformshift{2.742135in}{2.260905in}%
\pgfsys@useobject{currentmarker}{}%
\end{pgfscope}%
\begin{pgfscope}%
\pgfsys@transformshift{2.751608in}{2.264551in}%
\pgfsys@useobject{currentmarker}{}%
\end{pgfscope}%
\begin{pgfscope}%
\pgfsys@transformshift{2.761080in}{2.272509in}%
\pgfsys@useobject{currentmarker}{}%
\end{pgfscope}%
\begin{pgfscope}%
\pgfsys@transformshift{2.770553in}{2.274393in}%
\pgfsys@useobject{currentmarker}{}%
\end{pgfscope}%
\begin{pgfscope}%
\pgfsys@transformshift{2.780025in}{2.280883in}%
\pgfsys@useobject{currentmarker}{}%
\end{pgfscope}%
\begin{pgfscope}%
\pgfsys@transformshift{2.789498in}{2.298233in}%
\pgfsys@useobject{currentmarker}{}%
\end{pgfscope}%
\begin{pgfscope}%
\pgfsys@transformshift{2.798970in}{2.304429in}%
\pgfsys@useobject{currentmarker}{}%
\end{pgfscope}%
\begin{pgfscope}%
\pgfsys@transformshift{2.808443in}{2.308858in}%
\pgfsys@useobject{currentmarker}{}%
\end{pgfscope}%
\begin{pgfscope}%
\pgfsys@transformshift{2.817915in}{2.315837in}%
\pgfsys@useobject{currentmarker}{}%
\end{pgfscope}%
\begin{pgfscope}%
\pgfsys@transformshift{2.827388in}{2.318700in}%
\pgfsys@useobject{currentmarker}{}%
\end{pgfscope}%
\begin{pgfscope}%
\pgfsys@transformshift{2.836860in}{2.320584in}%
\pgfsys@useobject{currentmarker}{}%
\end{pgfscope}%
\begin{pgfscope}%
\pgfsys@transformshift{2.846333in}{2.322865in}%
\pgfsys@useobject{currentmarker}{}%
\end{pgfscope}%
\begin{pgfscope}%
\pgfsys@transformshift{2.855805in}{2.321715in}%
\pgfsys@useobject{currentmarker}{}%
\end{pgfscope}%
\begin{pgfscope}%
\pgfsys@transformshift{2.865278in}{2.328008in}%
\pgfsys@useobject{currentmarker}{}%
\end{pgfscope}%
\begin{pgfscope}%
\pgfsys@transformshift{2.874750in}{2.331361in}%
\pgfsys@useobject{currentmarker}{}%
\end{pgfscope}%
\begin{pgfscope}%
\pgfsys@transformshift{2.884223in}{2.331782in}%
\pgfsys@useobject{currentmarker}{}%
\end{pgfscope}%
\begin{pgfscope}%
\pgfsys@transformshift{2.893695in}{2.333862in}%
\pgfsys@useobject{currentmarker}{}%
\end{pgfscope}%
\begin{pgfscope}%
\pgfsys@transformshift{2.903168in}{2.335746in}%
\pgfsys@useobject{currentmarker}{}%
\end{pgfscope}%
\begin{pgfscope}%
\pgfsys@transformshift{2.912640in}{2.333720in}%
\pgfsys@useobject{currentmarker}{}%
\end{pgfscope}%
\begin{pgfscope}%
\pgfsys@transformshift{2.922113in}{2.338051in}%
\pgfsys@useobject{currentmarker}{}%
\end{pgfscope}%
\begin{pgfscope}%
\pgfsys@transformshift{2.931585in}{2.345911in}%
\pgfsys@useobject{currentmarker}{}%
\end{pgfscope}%
\begin{pgfscope}%
\pgfsys@transformshift{2.941058in}{2.354647in}%
\pgfsys@useobject{currentmarker}{}%
\end{pgfscope}%
\begin{pgfscope}%
\pgfsys@transformshift{2.950530in}{2.358396in}%
\pgfsys@useobject{currentmarker}{}%
\end{pgfscope}%
\begin{pgfscope}%
\pgfsys@transformshift{2.960003in}{2.362336in}%
\pgfsys@useobject{currentmarker}{}%
\end{pgfscope}%
\begin{pgfscope}%
\pgfsys@transformshift{2.969475in}{2.369800in}%
\pgfsys@useobject{currentmarker}{}%
\end{pgfscope}%
\begin{pgfscope}%
\pgfsys@transformshift{2.978948in}{2.377660in}%
\pgfsys@useobject{currentmarker}{}%
\end{pgfscope}%
\begin{pgfscope}%
\pgfsys@transformshift{2.988420in}{2.386102in}%
\pgfsys@useobject{currentmarker}{}%
\end{pgfscope}%
\begin{pgfscope}%
\pgfsys@transformshift{2.997893in}{2.388970in}%
\pgfsys@useobject{currentmarker}{}%
\end{pgfscope}%
\begin{pgfscope}%
\pgfsys@transformshift{3.007365in}{2.391735in}%
\pgfsys@useobject{currentmarker}{}%
\end{pgfscope}%
\begin{pgfscope}%
\pgfsys@transformshift{3.016838in}{2.396071in}%
\pgfsys@useobject{currentmarker}{}%
\end{pgfscope}%
\begin{pgfscope}%
\pgfsys@transformshift{3.026310in}{2.393942in}%
\pgfsys@useobject{currentmarker}{}%
\end{pgfscope}%
\begin{pgfscope}%
\pgfsys@transformshift{3.035783in}{2.399742in}%
\pgfsys@useobject{currentmarker}{}%
\end{pgfscope}%
\begin{pgfscope}%
\pgfsys@transformshift{3.045255in}{2.406329in}%
\pgfsys@useobject{currentmarker}{}%
\end{pgfscope}%
\begin{pgfscope}%
\pgfsys@transformshift{3.054728in}{2.408410in}%
\pgfsys@useobject{currentmarker}{}%
\end{pgfscope}%
\begin{pgfscope}%
\pgfsys@transformshift{3.064200in}{2.407852in}%
\pgfsys@useobject{currentmarker}{}%
\end{pgfscope}%
\begin{pgfscope}%
\pgfsys@transformshift{3.073673in}{2.406408in}%
\pgfsys@useobject{currentmarker}{}%
\end{pgfscope}%
\begin{pgfscope}%
\pgfsys@transformshift{3.083145in}{2.416519in}%
\pgfsys@useobject{currentmarker}{}%
\end{pgfscope}%
\begin{pgfscope}%
\pgfsys@transformshift{3.092618in}{2.418305in}%
\pgfsys@useobject{currentmarker}{}%
\end{pgfscope}%
\begin{pgfscope}%
\pgfsys@transformshift{3.102090in}{2.419412in}%
\pgfsys@useobject{currentmarker}{}%
\end{pgfscope}%
\begin{pgfscope}%
\pgfsys@transformshift{3.111563in}{2.425211in}%
\pgfsys@useobject{currentmarker}{}%
\end{pgfscope}%
\begin{pgfscope}%
\pgfsys@transformshift{3.121035in}{2.428074in}%
\pgfsys@useobject{currentmarker}{}%
\end{pgfscope}%
\begin{pgfscope}%
\pgfsys@transformshift{3.130508in}{2.433781in}%
\pgfsys@useobject{currentmarker}{}%
\end{pgfscope}%
\begin{pgfscope}%
\pgfsys@transformshift{3.139980in}{2.440657in}%
\pgfsys@useobject{currentmarker}{}%
\end{pgfscope}%
\begin{pgfscope}%
\pgfsys@transformshift{3.149453in}{2.440295in}%
\pgfsys@useobject{currentmarker}{}%
\end{pgfscope}%
\begin{pgfscope}%
\pgfsys@transformshift{3.158925in}{2.444920in}%
\pgfsys@useobject{currentmarker}{}%
\end{pgfscope}%
\begin{pgfscope}%
\pgfsys@transformshift{3.168398in}{2.455227in}%
\pgfsys@useobject{currentmarker}{}%
\end{pgfscope}%
\begin{pgfscope}%
\pgfsys@transformshift{3.177870in}{2.457992in}%
\pgfsys@useobject{currentmarker}{}%
\end{pgfscope}%
\begin{pgfscope}%
\pgfsys@transformshift{3.187343in}{2.463204in}%
\pgfsys@useobject{currentmarker}{}%
\end{pgfscope}%
\begin{pgfscope}%
\pgfsys@transformshift{3.196815in}{2.469498in}%
\pgfsys@useobject{currentmarker}{}%
\end{pgfscope}%
\begin{pgfscope}%
\pgfsys@transformshift{3.206288in}{2.471676in}%
\pgfsys@useobject{currentmarker}{}%
\end{pgfscope}%
\begin{pgfscope}%
\pgfsys@transformshift{3.215760in}{2.474153in}%
\pgfsys@useobject{currentmarker}{}%
\end{pgfscope}%
\begin{pgfscope}%
\pgfsys@transformshift{3.225233in}{2.477016in}%
\pgfsys@useobject{currentmarker}{}%
\end{pgfscope}%
\begin{pgfscope}%
\pgfsys@transformshift{3.234705in}{2.482918in}%
\pgfsys@useobject{currentmarker}{}%
\end{pgfscope}%
\begin{pgfscope}%
\pgfsys@transformshift{3.244178in}{2.487054in}%
\pgfsys@useobject{currentmarker}{}%
\end{pgfscope}%
\begin{pgfscope}%
\pgfsys@transformshift{3.253650in}{2.489623in}%
\pgfsys@useobject{currentmarker}{}%
\end{pgfscope}%
\begin{pgfscope}%
\pgfsys@transformshift{3.263123in}{2.492002in}%
\pgfsys@useobject{currentmarker}{}%
\end{pgfscope}%
\begin{pgfscope}%
\pgfsys@transformshift{3.272595in}{2.491830in}%
\pgfsys@useobject{currentmarker}{}%
\end{pgfscope}%
\begin{pgfscope}%
\pgfsys@transformshift{3.282068in}{2.493328in}%
\pgfsys@useobject{currentmarker}{}%
\end{pgfscope}%
\begin{pgfscope}%
\pgfsys@transformshift{3.291540in}{2.498638in}%
\pgfsys@useobject{currentmarker}{}%
\end{pgfscope}%
\begin{pgfscope}%
\pgfsys@transformshift{3.301013in}{2.500429in}%
\pgfsys@useobject{currentmarker}{}%
\end{pgfscope}%
\begin{pgfscope}%
\pgfsys@transformshift{3.310485in}{2.508774in}%
\pgfsys@useobject{currentmarker}{}%
\end{pgfscope}%
\begin{pgfscope}%
\pgfsys@transformshift{3.319958in}{2.509288in}%
\pgfsys@useobject{currentmarker}{}%
\end{pgfscope}%
\begin{pgfscope}%
\pgfsys@transformshift{3.329430in}{2.513722in}%
\pgfsys@useobject{currentmarker}{}%
\end{pgfscope}%
\begin{pgfscope}%
\pgfsys@transformshift{3.338903in}{2.515704in}%
\pgfsys@useobject{currentmarker}{}%
\end{pgfscope}%
\begin{pgfscope}%
\pgfsys@transformshift{3.348375in}{2.519942in}%
\pgfsys@useobject{currentmarker}{}%
\end{pgfscope}%
\begin{pgfscope}%
\pgfsys@transformshift{3.357848in}{2.525154in}%
\pgfsys@useobject{currentmarker}{}%
\end{pgfscope}%
\begin{pgfscope}%
\pgfsys@transformshift{3.367320in}{2.527925in}%
\pgfsys@useobject{currentmarker}{}%
\end{pgfscope}%
\begin{pgfscope}%
\pgfsys@transformshift{3.376793in}{2.534311in}%
\pgfsys@useobject{currentmarker}{}%
\end{pgfscope}%
\begin{pgfscope}%
\pgfsys@transformshift{3.386265in}{2.530915in}%
\pgfsys@useobject{currentmarker}{}%
\end{pgfscope}%
\begin{pgfscope}%
\pgfsys@transformshift{3.395737in}{2.533093in}%
\pgfsys@useobject{currentmarker}{}%
\end{pgfscope}%
\begin{pgfscope}%
\pgfsys@transformshift{3.405210in}{2.534096in}%
\pgfsys@useobject{currentmarker}{}%
\end{pgfscope}%
\begin{pgfscope}%
\pgfsys@transformshift{3.414682in}{2.542641in}%
\pgfsys@useobject{currentmarker}{}%
\end{pgfscope}%
\begin{pgfscope}%
\pgfsys@transformshift{3.424155in}{2.545309in}%
\pgfsys@useobject{currentmarker}{}%
\end{pgfscope}%
\begin{pgfscope}%
\pgfsys@transformshift{3.433627in}{2.549253in}%
\pgfsys@useobject{currentmarker}{}%
\end{pgfscope}%
\begin{pgfscope}%
\pgfsys@transformshift{3.443100in}{2.548103in}%
\pgfsys@useobject{currentmarker}{}%
\end{pgfscope}%
\begin{pgfscope}%
\pgfsys@transformshift{3.452572in}{2.550090in}%
\pgfsys@useobject{currentmarker}{}%
\end{pgfscope}%
\begin{pgfscope}%
\pgfsys@transformshift{3.462045in}{2.549429in}%
\pgfsys@useobject{currentmarker}{}%
\end{pgfscope}%
\begin{pgfscope}%
\pgfsys@transformshift{3.471517in}{2.556208in}%
\pgfsys@useobject{currentmarker}{}%
\end{pgfscope}%
\begin{pgfscope}%
\pgfsys@transformshift{3.480990in}{2.561523in}%
\pgfsys@useobject{currentmarker}{}%
\end{pgfscope}%
\begin{pgfscope}%
\pgfsys@transformshift{3.490462in}{2.567225in}%
\pgfsys@useobject{currentmarker}{}%
\end{pgfscope}%
\begin{pgfscope}%
\pgfsys@transformshift{3.499935in}{2.562751in}%
\pgfsys@useobject{currentmarker}{}%
\end{pgfscope}%
\begin{pgfscope}%
\pgfsys@transformshift{3.509407in}{2.565712in}%
\pgfsys@useobject{currentmarker}{}%
\end{pgfscope}%
\begin{pgfscope}%
\pgfsys@transformshift{3.518880in}{2.567112in}%
\pgfsys@useobject{currentmarker}{}%
\end{pgfscope}%
\begin{pgfscope}%
\pgfsys@transformshift{3.528352in}{2.571345in}%
\pgfsys@useobject{currentmarker}{}%
\end{pgfscope}%
\begin{pgfscope}%
\pgfsys@transformshift{3.537825in}{2.573328in}%
\pgfsys@useobject{currentmarker}{}%
\end{pgfscope}%
\begin{pgfscope}%
\pgfsys@transformshift{3.547297in}{2.577566in}%
\pgfsys@useobject{currentmarker}{}%
\end{pgfscope}%
\begin{pgfscope}%
\pgfsys@transformshift{3.556770in}{2.582876in}%
\pgfsys@useobject{currentmarker}{}%
\end{pgfscope}%
\begin{pgfscope}%
\pgfsys@transformshift{3.566242in}{2.579382in}%
\pgfsys@useobject{currentmarker}{}%
\end{pgfscope}%
\begin{pgfscope}%
\pgfsys@transformshift{3.575715in}{2.583713in}%
\pgfsys@useobject{currentmarker}{}%
\end{pgfscope}%
\begin{pgfscope}%
\pgfsys@transformshift{3.585187in}{2.586581in}%
\pgfsys@useobject{currentmarker}{}%
\end{pgfscope}%
\begin{pgfscope}%
\pgfsys@transformshift{3.594660in}{2.589738in}%
\pgfsys@useobject{currentmarker}{}%
\end{pgfscope}%
\begin{pgfscope}%
\pgfsys@transformshift{3.604132in}{2.593188in}%
\pgfsys@useobject{currentmarker}{}%
\end{pgfscope}%
\begin{pgfscope}%
\pgfsys@transformshift{3.613605in}{2.593315in}%
\pgfsys@useobject{currentmarker}{}%
\end{pgfscope}%
\begin{pgfscope}%
\pgfsys@transformshift{3.623077in}{2.588935in}%
\pgfsys@useobject{currentmarker}{}%
\end{pgfscope}%
\begin{pgfscope}%
\pgfsys@transformshift{3.632550in}{2.590824in}%
\pgfsys@useobject{currentmarker}{}%
\end{pgfscope}%
\begin{pgfscope}%
\pgfsys@transformshift{3.642022in}{2.598386in}%
\pgfsys@useobject{currentmarker}{}%
\end{pgfscope}%
\begin{pgfscope}%
\pgfsys@transformshift{3.651495in}{2.602526in}%
\pgfsys@useobject{currentmarker}{}%
\end{pgfscope}%
\begin{pgfscope}%
\pgfsys@transformshift{3.660967in}{2.602746in}%
\pgfsys@useobject{currentmarker}{}%
\end{pgfscope}%
\begin{pgfscope}%
\pgfsys@transformshift{3.670440in}{2.606887in}%
\pgfsys@useobject{currentmarker}{}%
\end{pgfscope}%
\begin{pgfscope}%
\pgfsys@transformshift{3.679912in}{2.616112in}%
\pgfsys@useobject{currentmarker}{}%
\end{pgfscope}%
\begin{pgfscope}%
\pgfsys@transformshift{3.689385in}{2.618584in}%
\pgfsys@useobject{currentmarker}{}%
\end{pgfscope}%
\begin{pgfscope}%
\pgfsys@transformshift{3.698857in}{2.626542in}%
\pgfsys@useobject{currentmarker}{}%
\end{pgfscope}%
\begin{pgfscope}%
\pgfsys@transformshift{3.708330in}{2.626958in}%
\pgfsys@useobject{currentmarker}{}%
\end{pgfscope}%
\begin{pgfscope}%
\pgfsys@transformshift{3.717802in}{2.634818in}%
\pgfsys@useobject{currentmarker}{}%
\end{pgfscope}%
\begin{pgfscope}%
\pgfsys@transformshift{3.727275in}{2.638464in}%
\pgfsys@useobject{currentmarker}{}%
\end{pgfscope}%
\begin{pgfscope}%
\pgfsys@transformshift{3.736747in}{2.639276in}%
\pgfsys@useobject{currentmarker}{}%
\end{pgfscope}%
\begin{pgfscope}%
\pgfsys@transformshift{3.746220in}{2.644782in}%
\pgfsys@useobject{currentmarker}{}%
\end{pgfscope}%
\begin{pgfscope}%
\pgfsys@transformshift{3.755692in}{2.648820in}%
\pgfsys@useobject{currentmarker}{}%
\end{pgfscope}%
\begin{pgfscope}%
\pgfsys@transformshift{3.765165in}{2.653254in}%
\pgfsys@useobject{currentmarker}{}%
\end{pgfscope}%
\begin{pgfscope}%
\pgfsys@transformshift{3.774637in}{2.659249in}%
\pgfsys@useobject{currentmarker}{}%
\end{pgfscope}%
\begin{pgfscope}%
\pgfsys@transformshift{3.784110in}{2.665445in}%
\pgfsys@useobject{currentmarker}{}%
\end{pgfscope}%
\begin{pgfscope}%
\pgfsys@transformshift{3.793582in}{2.669483in}%
\pgfsys@useobject{currentmarker}{}%
\end{pgfscope}%
\begin{pgfscope}%
\pgfsys@transformshift{3.803055in}{2.673819in}%
\pgfsys@useobject{currentmarker}{}%
\end{pgfscope}%
\begin{pgfscope}%
\pgfsys@transformshift{3.812527in}{2.674627in}%
\pgfsys@useobject{currentmarker}{}%
\end{pgfscope}%
\begin{pgfscope}%
\pgfsys@transformshift{3.822000in}{2.680622in}%
\pgfsys@useobject{currentmarker}{}%
\end{pgfscope}%
\begin{pgfscope}%
\pgfsys@transformshift{3.831472in}{2.687209in}%
\pgfsys@useobject{currentmarker}{}%
\end{pgfscope}%
\begin{pgfscope}%
\pgfsys@transformshift{3.840945in}{2.688800in}%
\pgfsys@useobject{currentmarker}{}%
\end{pgfscope}%
\begin{pgfscope}%
\pgfsys@transformshift{3.850417in}{2.693332in}%
\pgfsys@useobject{currentmarker}{}%
\end{pgfscope}%
\begin{pgfscope}%
\pgfsys@transformshift{3.859890in}{2.697565in}%
\pgfsys@useobject{currentmarker}{}%
\end{pgfscope}%
\begin{pgfscope}%
\pgfsys@transformshift{3.869362in}{2.697399in}%
\pgfsys@useobject{currentmarker}{}%
\end{pgfscope}%
\begin{pgfscope}%
\pgfsys@transformshift{3.878835in}{2.698696in}%
\pgfsys@useobject{currentmarker}{}%
\end{pgfscope}%
\begin{pgfscope}%
\pgfsys@transformshift{3.888307in}{2.703615in}%
\pgfsys@useobject{currentmarker}{}%
\end{pgfscope}%
\begin{pgfscope}%
\pgfsys@transformshift{3.897780in}{2.711475in}%
\pgfsys@useobject{currentmarker}{}%
\end{pgfscope}%
\begin{pgfscope}%
\pgfsys@transformshift{3.907252in}{2.712282in}%
\pgfsys@useobject{currentmarker}{}%
\end{pgfscope}%
\begin{pgfscope}%
\pgfsys@transformshift{3.916725in}{2.714954in}%
\pgfsys@useobject{currentmarker}{}%
\end{pgfscope}%
\begin{pgfscope}%
\pgfsys@transformshift{3.926197in}{2.717132in}%
\pgfsys@useobject{currentmarker}{}%
\end{pgfscope}%
\begin{pgfscope}%
\pgfsys@transformshift{3.935670in}{2.719511in}%
\pgfsys@useobject{currentmarker}{}%
\end{pgfscope}%
\begin{pgfscope}%
\pgfsys@transformshift{3.945142in}{2.716403in}%
\pgfsys@useobject{currentmarker}{}%
\end{pgfscope}%
\begin{pgfscope}%
\pgfsys@transformshift{3.954615in}{2.718684in}%
\pgfsys@useobject{currentmarker}{}%
\end{pgfscope}%
\begin{pgfscope}%
\pgfsys@transformshift{3.964087in}{2.721742in}%
\pgfsys@useobject{currentmarker}{}%
\end{pgfscope}%
\begin{pgfscope}%
\pgfsys@transformshift{3.973560in}{2.724997in}%
\pgfsys@useobject{currentmarker}{}%
\end{pgfscope}%
\begin{pgfscope}%
\pgfsys@transformshift{3.983032in}{2.728844in}%
\pgfsys@useobject{currentmarker}{}%
\end{pgfscope}%
\begin{pgfscope}%
\pgfsys@transformshift{3.992505in}{2.730337in}%
\pgfsys@useobject{currentmarker}{}%
\end{pgfscope}%
\begin{pgfscope}%
\pgfsys@transformshift{4.001977in}{2.733498in}%
\pgfsys@useobject{currentmarker}{}%
\end{pgfscope}%
\begin{pgfscope}%
\pgfsys@transformshift{4.011450in}{2.730586in}%
\pgfsys@useobject{currentmarker}{}%
\end{pgfscope}%
\begin{pgfscope}%
\pgfsys@transformshift{4.020922in}{2.735314in}%
\pgfsys@useobject{currentmarker}{}%
\end{pgfscope}%
\begin{pgfscope}%
\pgfsys@transformshift{4.030395in}{2.736905in}%
\pgfsys@useobject{currentmarker}{}%
\end{pgfscope}%
\begin{pgfscope}%
\pgfsys@transformshift{4.039867in}{2.733503in}%
\pgfsys@useobject{currentmarker}{}%
\end{pgfscope}%
\begin{pgfscope}%
\pgfsys@transformshift{4.049340in}{2.734609in}%
\pgfsys@useobject{currentmarker}{}%
\end{pgfscope}%
\begin{pgfscope}%
\pgfsys@transformshift{4.058812in}{2.739234in}%
\pgfsys@useobject{currentmarker}{}%
\end{pgfscope}%
\begin{pgfscope}%
\pgfsys@transformshift{4.068285in}{2.745430in}%
\pgfsys@useobject{currentmarker}{}%
\end{pgfscope}%
\begin{pgfscope}%
\pgfsys@transformshift{4.077757in}{2.745846in}%
\pgfsys@useobject{currentmarker}{}%
\end{pgfscope}%
\begin{pgfscope}%
\pgfsys@transformshift{4.087230in}{2.751651in}%
\pgfsys@useobject{currentmarker}{}%
\end{pgfscope}%
\begin{pgfscope}%
\pgfsys@transformshift{4.096702in}{2.757744in}%
\pgfsys@useobject{currentmarker}{}%
\end{pgfscope}%
\begin{pgfscope}%
\pgfsys@transformshift{4.106175in}{2.760411in}%
\pgfsys@useobject{currentmarker}{}%
\end{pgfscope}%
\begin{pgfscope}%
\pgfsys@transformshift{4.115647in}{2.768663in}%
\pgfsys@useobject{currentmarker}{}%
\end{pgfscope}%
\begin{pgfscope}%
\pgfsys@transformshift{4.125120in}{2.774266in}%
\pgfsys@useobject{currentmarker}{}%
\end{pgfscope}%
\begin{pgfscope}%
\pgfsys@transformshift{4.134592in}{2.778798in}%
\pgfsys@useobject{currentmarker}{}%
\end{pgfscope}%
\begin{pgfscope}%
\pgfsys@transformshift{4.144065in}{2.782738in}%
\pgfsys@useobject{currentmarker}{}%
\end{pgfscope}%
\begin{pgfscope}%
\pgfsys@transformshift{4.153537in}{2.780125in}%
\pgfsys@useobject{currentmarker}{}%
\end{pgfscope}%
\begin{pgfscope}%
\pgfsys@transformshift{4.163010in}{2.780345in}%
\pgfsys@useobject{currentmarker}{}%
\end{pgfscope}%
\begin{pgfscope}%
\pgfsys@transformshift{4.172482in}{2.787417in}%
\pgfsys@useobject{currentmarker}{}%
\end{pgfscope}%
\begin{pgfscope}%
\pgfsys@transformshift{4.181955in}{2.789110in}%
\pgfsys@useobject{currentmarker}{}%
\end{pgfscope}%
\begin{pgfscope}%
\pgfsys@transformshift{4.191427in}{2.791093in}%
\pgfsys@useobject{currentmarker}{}%
\end{pgfscope}%
\begin{pgfscope}%
\pgfsys@transformshift{4.200900in}{2.796603in}%
\pgfsys@useobject{currentmarker}{}%
\end{pgfscope}%
\begin{pgfscope}%
\pgfsys@transformshift{4.210372in}{2.795943in}%
\pgfsys@useobject{currentmarker}{}%
\end{pgfscope}%
\begin{pgfscope}%
\pgfsys@transformshift{4.219845in}{2.800279in}%
\pgfsys@useobject{currentmarker}{}%
\end{pgfscope}%
\begin{pgfscope}%
\pgfsys@transformshift{4.229317in}{2.801478in}%
\pgfsys@useobject{currentmarker}{}%
\end{pgfscope}%
\begin{pgfscope}%
\pgfsys@transformshift{4.238790in}{2.806206in}%
\pgfsys@useobject{currentmarker}{}%
\end{pgfscope}%
\begin{pgfscope}%
\pgfsys@transformshift{4.248262in}{2.808188in}%
\pgfsys@useobject{currentmarker}{}%
\end{pgfscope}%
\begin{pgfscope}%
\pgfsys@transformshift{4.257735in}{2.807723in}%
\pgfsys@useobject{currentmarker}{}%
\end{pgfscope}%
\begin{pgfscope}%
\pgfsys@transformshift{4.267207in}{2.807850in}%
\pgfsys@useobject{currentmarker}{}%
\end{pgfscope}%
\begin{pgfscope}%
\pgfsys@transformshift{4.276680in}{2.810322in}%
\pgfsys@useobject{currentmarker}{}%
\end{pgfscope}%
\begin{pgfscope}%
\pgfsys@transformshift{4.286152in}{2.818182in}%
\pgfsys@useobject{currentmarker}{}%
\end{pgfscope}%
\begin{pgfscope}%
\pgfsys@transformshift{4.295625in}{2.812333in}%
\pgfsys@useobject{currentmarker}{}%
\end{pgfscope}%
\begin{pgfscope}%
\pgfsys@transformshift{4.305097in}{2.811677in}%
\pgfsys@useobject{currentmarker}{}%
\end{pgfscope}%
\begin{pgfscope}%
\pgfsys@transformshift{4.314570in}{2.814540in}%
\pgfsys@useobject{currentmarker}{}%
\end{pgfscope}%
\begin{pgfscope}%
\pgfsys@transformshift{4.324042in}{2.817991in}%
\pgfsys@useobject{currentmarker}{}%
\end{pgfscope}%
\begin{pgfscope}%
\pgfsys@transformshift{4.333515in}{2.819097in}%
\pgfsys@useobject{currentmarker}{}%
\end{pgfscope}%
\begin{pgfscope}%
\pgfsys@transformshift{4.342987in}{2.821177in}%
\pgfsys@useobject{currentmarker}{}%
\end{pgfscope}%
\begin{pgfscope}%
\pgfsys@transformshift{4.352460in}{2.826981in}%
\pgfsys@useobject{currentmarker}{}%
\end{pgfscope}%
\begin{pgfscope}%
\pgfsys@transformshift{4.361932in}{2.834151in}%
\pgfsys@useobject{currentmarker}{}%
\end{pgfscope}%
\begin{pgfscope}%
\pgfsys@transformshift{4.371405in}{2.835551in}%
\pgfsys@useobject{currentmarker}{}%
\end{pgfscope}%
\begin{pgfscope}%
\pgfsys@transformshift{4.380877in}{2.839491in}%
\pgfsys@useobject{currentmarker}{}%
\end{pgfscope}%
\begin{pgfscope}%
\pgfsys@transformshift{4.390350in}{2.839907in}%
\pgfsys@useobject{currentmarker}{}%
\end{pgfscope}%
\begin{pgfscope}%
\pgfsys@transformshift{4.399822in}{2.842481in}%
\pgfsys@useobject{currentmarker}{}%
\end{pgfscope}%
\begin{pgfscope}%
\pgfsys@transformshift{4.409295in}{2.847987in}%
\pgfsys@useobject{currentmarker}{}%
\end{pgfscope}%
\begin{pgfscope}%
\pgfsys@transformshift{4.418767in}{2.852617in}%
\pgfsys@useobject{currentmarker}{}%
\end{pgfscope}%
\begin{pgfscope}%
\pgfsys@transformshift{4.428240in}{2.859493in}%
\pgfsys@useobject{currentmarker}{}%
\end{pgfscope}%
\begin{pgfscope}%
\pgfsys@transformshift{4.437712in}{2.866472in}%
\pgfsys@useobject{currentmarker}{}%
\end{pgfscope}%
\begin{pgfscope}%
\pgfsys@transformshift{4.447185in}{2.870118in}%
\pgfsys@useobject{currentmarker}{}%
\end{pgfscope}%
\begin{pgfscope}%
\pgfsys@transformshift{4.456657in}{2.876701in}%
\pgfsys@useobject{currentmarker}{}%
\end{pgfscope}%
\begin{pgfscope}%
\pgfsys@transformshift{4.466130in}{2.879079in}%
\pgfsys@useobject{currentmarker}{}%
\end{pgfscope}%
\begin{pgfscope}%
\pgfsys@transformshift{4.475602in}{2.887130in}%
\pgfsys@useobject{currentmarker}{}%
\end{pgfscope}%
\begin{pgfscope}%
\pgfsys@transformshift{4.485075in}{2.889215in}%
\pgfsys@useobject{currentmarker}{}%
\end{pgfscope}%
\begin{pgfscope}%
\pgfsys@transformshift{4.494547in}{2.893547in}%
\pgfsys@useobject{currentmarker}{}%
\end{pgfscope}%
\begin{pgfscope}%
\pgfsys@transformshift{4.504020in}{2.896415in}%
\pgfsys@useobject{currentmarker}{}%
\end{pgfscope}%
\begin{pgfscope}%
\pgfsys@transformshift{4.513492in}{2.895656in}%
\pgfsys@useobject{currentmarker}{}%
\end{pgfscope}%
\begin{pgfscope}%
\pgfsys@transformshift{4.522965in}{2.899405in}%
\pgfsys@useobject{currentmarker}{}%
\end{pgfscope}%
\begin{pgfscope}%
\pgfsys@transformshift{4.532437in}{2.903149in}%
\pgfsys@useobject{currentmarker}{}%
\end{pgfscope}%
\begin{pgfscope}%
\pgfsys@transformshift{4.541910in}{2.906697in}%
\pgfsys@useobject{currentmarker}{}%
\end{pgfscope}%
\begin{pgfscope}%
\pgfsys@transformshift{4.551382in}{2.906531in}%
\pgfsys@useobject{currentmarker}{}%
\end{pgfscope}%
\begin{pgfscope}%
\pgfsys@transformshift{4.560855in}{2.906555in}%
\pgfsys@useobject{currentmarker}{}%
\end{pgfscope}%
\begin{pgfscope}%
\pgfsys@transformshift{4.570327in}{2.911577in}%
\pgfsys@useobject{currentmarker}{}%
\end{pgfscope}%
\begin{pgfscope}%
\pgfsys@transformshift{4.579800in}{2.915223in}%
\pgfsys@useobject{currentmarker}{}%
\end{pgfscope}%
\begin{pgfscope}%
\pgfsys@transformshift{4.589272in}{2.913490in}%
\pgfsys@useobject{currentmarker}{}%
\end{pgfscope}%
\begin{pgfscope}%
\pgfsys@transformshift{4.598745in}{2.918507in}%
\pgfsys@useobject{currentmarker}{}%
\end{pgfscope}%
\begin{pgfscope}%
\pgfsys@transformshift{4.608217in}{2.917944in}%
\pgfsys@useobject{currentmarker}{}%
\end{pgfscope}%
\begin{pgfscope}%
\pgfsys@transformshift{4.617690in}{2.922574in}%
\pgfsys@useobject{currentmarker}{}%
\end{pgfscope}%
\begin{pgfscope}%
\pgfsys@transformshift{4.627162in}{2.926318in}%
\pgfsys@useobject{currentmarker}{}%
\end{pgfscope}%
\begin{pgfscope}%
\pgfsys@transformshift{4.636635in}{2.931633in}%
\pgfsys@useobject{currentmarker}{}%
\end{pgfscope}%
\begin{pgfscope}%
\pgfsys@transformshift{4.646107in}{2.933125in}%
\pgfsys@useobject{currentmarker}{}%
\end{pgfscope}%
\begin{pgfscope}%
\pgfsys@transformshift{4.655580in}{2.933644in}%
\pgfsys@useobject{currentmarker}{}%
\end{pgfscope}%
\begin{pgfscope}%
\pgfsys@transformshift{4.665052in}{2.932200in}%
\pgfsys@useobject{currentmarker}{}%
\end{pgfscope}%
\begin{pgfscope}%
\pgfsys@transformshift{4.674525in}{2.928995in}%
\pgfsys@useobject{currentmarker}{}%
\end{pgfscope}%
\begin{pgfscope}%
\pgfsys@transformshift{4.683997in}{2.933723in}%
\pgfsys@useobject{currentmarker}{}%
\end{pgfscope}%
\begin{pgfscope}%
\pgfsys@transformshift{4.693470in}{2.940012in}%
\pgfsys@useobject{currentmarker}{}%
\end{pgfscope}%
\begin{pgfscope}%
\pgfsys@transformshift{4.702942in}{2.943075in}%
\pgfsys@useobject{currentmarker}{}%
\end{pgfscope}%
\begin{pgfscope}%
\pgfsys@transformshift{4.712415in}{2.948288in}%
\pgfsys@useobject{currentmarker}{}%
\end{pgfscope}%
\begin{pgfscope}%
\pgfsys@transformshift{4.721887in}{2.950666in}%
\pgfsys@useobject{currentmarker}{}%
\end{pgfscope}%
\begin{pgfscope}%
\pgfsys@transformshift{4.731360in}{2.953138in}%
\pgfsys@useobject{currentmarker}{}%
\end{pgfscope}%
\begin{pgfscope}%
\pgfsys@transformshift{4.740832in}{2.958355in}%
\pgfsys@useobject{currentmarker}{}%
\end{pgfscope}%
\begin{pgfscope}%
\pgfsys@transformshift{4.750305in}{2.961512in}%
\pgfsys@useobject{currentmarker}{}%
\end{pgfscope}%
\begin{pgfscope}%
\pgfsys@transformshift{4.759777in}{2.969073in}%
\pgfsys@useobject{currentmarker}{}%
\end{pgfscope}%
\begin{pgfscope}%
\pgfsys@transformshift{4.769250in}{2.972332in}%
\pgfsys@useobject{currentmarker}{}%
\end{pgfscope}%
\begin{pgfscope}%
\pgfsys@transformshift{4.778722in}{2.973434in}%
\pgfsys@useobject{currentmarker}{}%
\end{pgfscope}%
\begin{pgfscope}%
\pgfsys@transformshift{4.788195in}{2.974638in}%
\pgfsys@useobject{currentmarker}{}%
\end{pgfscope}%
\begin{pgfscope}%
\pgfsys@transformshift{4.797667in}{2.982003in}%
\pgfsys@useobject{currentmarker}{}%
\end{pgfscope}%
\begin{pgfscope}%
\pgfsys@transformshift{4.807140in}{2.987123in}%
\pgfsys@useobject{currentmarker}{}%
\end{pgfscope}%
\begin{pgfscope}%
\pgfsys@transformshift{4.816612in}{2.985777in}%
\pgfsys@useobject{currentmarker}{}%
\end{pgfscope}%
\begin{pgfscope}%
\pgfsys@transformshift{4.826085in}{2.991674in}%
\pgfsys@useobject{currentmarker}{}%
\end{pgfscope}%
\begin{pgfscope}%
\pgfsys@transformshift{4.835557in}{2.997968in}%
\pgfsys@useobject{currentmarker}{}%
\end{pgfscope}%
\begin{pgfscope}%
\pgfsys@transformshift{4.845030in}{3.004257in}%
\pgfsys@useobject{currentmarker}{}%
\end{pgfscope}%
\begin{pgfscope}%
\pgfsys@transformshift{4.854502in}{3.003210in}%
\pgfsys@useobject{currentmarker}{}%
\end{pgfscope}%
\begin{pgfscope}%
\pgfsys@transformshift{4.863975in}{3.006171in}%
\pgfsys@useobject{currentmarker}{}%
\end{pgfscope}%
\begin{pgfscope}%
\pgfsys@transformshift{4.873447in}{3.009724in}%
\pgfsys@useobject{currentmarker}{}%
\end{pgfscope}%
\begin{pgfscope}%
\pgfsys@transformshift{4.882920in}{3.010727in}%
\pgfsys@useobject{currentmarker}{}%
\end{pgfscope}%
\begin{pgfscope}%
\pgfsys@transformshift{4.892392in}{3.015156in}%
\pgfsys@useobject{currentmarker}{}%
\end{pgfscope}%
\begin{pgfscope}%
\pgfsys@transformshift{4.901865in}{3.017535in}%
\pgfsys@useobject{currentmarker}{}%
\end{pgfscope}%
\begin{pgfscope}%
\pgfsys@transformshift{4.911337in}{3.023334in}%
\pgfsys@useobject{currentmarker}{}%
\end{pgfscope}%
\begin{pgfscope}%
\pgfsys@transformshift{4.920810in}{3.023657in}%
\pgfsys@useobject{currentmarker}{}%
\end{pgfscope}%
\begin{pgfscope}%
\pgfsys@transformshift{4.930282in}{3.024465in}%
\pgfsys@useobject{currentmarker}{}%
\end{pgfscope}%
\begin{pgfscope}%
\pgfsys@transformshift{4.939755in}{3.025473in}%
\pgfsys@useobject{currentmarker}{}%
\end{pgfscope}%
\begin{pgfscope}%
\pgfsys@transformshift{4.949227in}{3.029119in}%
\pgfsys@useobject{currentmarker}{}%
\end{pgfscope}%
\begin{pgfscope}%
\pgfsys@transformshift{4.958700in}{3.031199in}%
\pgfsys@useobject{currentmarker}{}%
\end{pgfscope}%
\begin{pgfscope}%
\pgfsys@transformshift{4.968172in}{3.032501in}%
\pgfsys@useobject{currentmarker}{}%
\end{pgfscope}%
\begin{pgfscope}%
\pgfsys@transformshift{4.977645in}{3.033407in}%
\pgfsys@useobject{currentmarker}{}%
\end{pgfscope}%
\begin{pgfscope}%
\pgfsys@transformshift{4.987117in}{3.041169in}%
\pgfsys@useobject{currentmarker}{}%
\end{pgfscope}%
\begin{pgfscope}%
\pgfsys@transformshift{4.996590in}{3.042172in}%
\pgfsys@useobject{currentmarker}{}%
\end{pgfscope}%
\begin{pgfscope}%
\pgfsys@transformshift{5.006062in}{3.043767in}%
\pgfsys@useobject{currentmarker}{}%
\end{pgfscope}%
\begin{pgfscope}%
\pgfsys@transformshift{5.015535in}{3.047511in}%
\pgfsys@useobject{currentmarker}{}%
\end{pgfscope}%
\begin{pgfscope}%
\pgfsys@transformshift{5.025007in}{3.050184in}%
\pgfsys@useobject{currentmarker}{}%
\end{pgfscope}%
\begin{pgfscope}%
\pgfsys@transformshift{5.034480in}{3.056571in}%
\pgfsys@useobject{currentmarker}{}%
\end{pgfscope}%
\begin{pgfscope}%
\pgfsys@transformshift{5.043952in}{3.059434in}%
\pgfsys@useobject{currentmarker}{}%
\end{pgfscope}%
\begin{pgfscope}%
\pgfsys@transformshift{5.053425in}{3.061127in}%
\pgfsys@useobject{currentmarker}{}%
\end{pgfscope}%
\begin{pgfscope}%
\pgfsys@transformshift{5.062897in}{3.064479in}%
\pgfsys@useobject{currentmarker}{}%
\end{pgfscope}%
\begin{pgfscope}%
\pgfsys@transformshift{5.072370in}{3.067250in}%
\pgfsys@useobject{currentmarker}{}%
\end{pgfscope}%
\begin{pgfscope}%
\pgfsys@transformshift{5.081842in}{3.071483in}%
\pgfsys@useobject{currentmarker}{}%
\end{pgfscope}%
\begin{pgfscope}%
\pgfsys@transformshift{5.091315in}{3.077092in}%
\pgfsys@useobject{currentmarker}{}%
\end{pgfscope}%
\begin{pgfscope}%
\pgfsys@transformshift{5.100787in}{3.079955in}%
\pgfsys@useobject{currentmarker}{}%
\end{pgfscope}%
\begin{pgfscope}%
\pgfsys@transformshift{5.110260in}{3.082720in}%
\pgfsys@useobject{currentmarker}{}%
\end{pgfscope}%
\begin{pgfscope}%
\pgfsys@transformshift{5.119732in}{3.087350in}%
\pgfsys@useobject{currentmarker}{}%
\end{pgfscope}%
\begin{pgfscope}%
\pgfsys@transformshift{5.129205in}{3.095303in}%
\pgfsys@useobject{currentmarker}{}%
\end{pgfscope}%
\begin{pgfscope}%
\pgfsys@transformshift{5.138677in}{3.102184in}%
\pgfsys@useobject{currentmarker}{}%
\end{pgfscope}%
\begin{pgfscope}%
\pgfsys@transformshift{5.148149in}{3.105439in}%
\pgfsys@useobject{currentmarker}{}%
\end{pgfscope}%
\begin{pgfscope}%
\pgfsys@transformshift{5.157622in}{3.107621in}%
\pgfsys@useobject{currentmarker}{}%
\end{pgfscope}%
\begin{pgfscope}%
\pgfsys@transformshift{5.167094in}{3.108429in}%
\pgfsys@useobject{currentmarker}{}%
\end{pgfscope}%
\begin{pgfscope}%
\pgfsys@transformshift{5.176567in}{3.105321in}%
\pgfsys@useobject{currentmarker}{}%
\end{pgfscope}%
\begin{pgfscope}%
\pgfsys@transformshift{5.186039in}{3.110636in}%
\pgfsys@useobject{currentmarker}{}%
\end{pgfscope}%
\begin{pgfscope}%
\pgfsys@transformshift{5.195512in}{3.111150in}%
\pgfsys@useobject{currentmarker}{}%
\end{pgfscope}%
\begin{pgfscope}%
\pgfsys@transformshift{5.204984in}{3.113431in}%
\pgfsys@useobject{currentmarker}{}%
\end{pgfscope}%
\begin{pgfscope}%
\pgfsys@transformshift{5.214457in}{3.116979in}%
\pgfsys@useobject{currentmarker}{}%
\end{pgfscope}%
\begin{pgfscope}%
\pgfsys@transformshift{5.223929in}{3.114561in}%
\pgfsys@useobject{currentmarker}{}%
\end{pgfscope}%
\begin{pgfscope}%
\pgfsys@transformshift{5.233402in}{3.115075in}%
\pgfsys@useobject{currentmarker}{}%
\end{pgfscope}%
\begin{pgfscope}%
\pgfsys@transformshift{5.242874in}{3.110401in}%
\pgfsys@useobject{currentmarker}{}%
\end{pgfscope}%
\begin{pgfscope}%
\pgfsys@transformshift{5.252347in}{3.111312in}%
\pgfsys@useobject{currentmarker}{}%
\end{pgfscope}%
\begin{pgfscope}%
\pgfsys@transformshift{5.261819in}{3.117013in}%
\pgfsys@useobject{currentmarker}{}%
\end{pgfscope}%
\begin{pgfscope}%
\pgfsys@transformshift{5.271292in}{3.119196in}%
\pgfsys@useobject{currentmarker}{}%
\end{pgfscope}%
\begin{pgfscope}%
\pgfsys@transformshift{5.280764in}{3.124212in}%
\pgfsys@useobject{currentmarker}{}%
\end{pgfscope}%
\begin{pgfscope}%
\pgfsys@transformshift{5.290237in}{3.130996in}%
\pgfsys@useobject{currentmarker}{}%
\end{pgfscope}%
\begin{pgfscope}%
\pgfsys@transformshift{5.299709in}{3.132978in}%
\pgfsys@useobject{currentmarker}{}%
\end{pgfscope}%
\begin{pgfscope}%
\pgfsys@transformshift{5.309182in}{3.135650in}%
\pgfsys@useobject{currentmarker}{}%
\end{pgfscope}%
\begin{pgfscope}%
\pgfsys@transformshift{5.318654in}{3.139003in}%
\pgfsys@useobject{currentmarker}{}%
\end{pgfscope}%
\begin{pgfscope}%
\pgfsys@transformshift{5.328127in}{3.141474in}%
\pgfsys@useobject{currentmarker}{}%
\end{pgfscope}%
\begin{pgfscope}%
\pgfsys@transformshift{5.337599in}{3.143168in}%
\pgfsys@useobject{currentmarker}{}%
\end{pgfscope}%
\begin{pgfscope}%
\pgfsys@transformshift{5.347072in}{3.148184in}%
\pgfsys@useobject{currentmarker}{}%
\end{pgfscope}%
\begin{pgfscope}%
\pgfsys@transformshift{5.356544in}{3.155555in}%
\pgfsys@useobject{currentmarker}{}%
\end{pgfscope}%
\begin{pgfscope}%
\pgfsys@transformshift{5.366017in}{3.161256in}%
\pgfsys@useobject{currentmarker}{}%
\end{pgfscope}%
\begin{pgfscope}%
\pgfsys@transformshift{5.375489in}{3.164712in}%
\pgfsys@useobject{currentmarker}{}%
\end{pgfscope}%
\begin{pgfscope}%
\pgfsys@transformshift{5.384962in}{3.170903in}%
\pgfsys@useobject{currentmarker}{}%
\end{pgfscope}%
\begin{pgfscope}%
\pgfsys@transformshift{5.394434in}{3.171514in}%
\pgfsys@useobject{currentmarker}{}%
\end{pgfscope}%
\begin{pgfscope}%
\pgfsys@transformshift{5.403907in}{3.169390in}%
\pgfsys@useobject{currentmarker}{}%
\end{pgfscope}%
\begin{pgfscope}%
\pgfsys@transformshift{5.413379in}{3.173428in}%
\pgfsys@useobject{currentmarker}{}%
\end{pgfscope}%
\begin{pgfscope}%
\pgfsys@transformshift{5.422852in}{3.174632in}%
\pgfsys@useobject{currentmarker}{}%
\end{pgfscope}%
\begin{pgfscope}%
\pgfsys@transformshift{5.432324in}{3.175048in}%
\pgfsys@useobject{currentmarker}{}%
\end{pgfscope}%
\begin{pgfscope}%
\pgfsys@transformshift{5.441797in}{3.174196in}%
\pgfsys@useobject{currentmarker}{}%
\end{pgfscope}%
\begin{pgfscope}%
\pgfsys@transformshift{5.451269in}{3.175493in}%
\pgfsys@useobject{currentmarker}{}%
\end{pgfscope}%
\begin{pgfscope}%
\pgfsys@transformshift{5.460742in}{3.174539in}%
\pgfsys@useobject{currentmarker}{}%
\end{pgfscope}%
\begin{pgfscope}%
\pgfsys@transformshift{5.470214in}{3.177113in}%
\pgfsys@useobject{currentmarker}{}%
\end{pgfscope}%
\begin{pgfscope}%
\pgfsys@transformshift{5.479687in}{3.184577in}%
\pgfsys@useobject{currentmarker}{}%
\end{pgfscope}%
\begin{pgfscope}%
\pgfsys@transformshift{5.489159in}{3.187738in}%
\pgfsys@useobject{currentmarker}{}%
\end{pgfscope}%
\end{pgfscope}%
\begin{pgfscope}%
\pgfsetrectcap%
\pgfsetmiterjoin%
\pgfsetlinewidth{0.803000pt}%
\definecolor{currentstroke}{rgb}{0.000000,0.000000,0.000000}%
\pgfsetstrokecolor{currentstroke}%
\pgfsetdash{}{0pt}%
\pgfpathmoveto{\pgfqpoint{0.762383in}{0.471179in}}%
\pgfpathlineto{\pgfqpoint{0.762383in}{3.317098in}}%
\pgfusepath{stroke}%
\end{pgfscope}%
\begin{pgfscope}%
\pgfsetrectcap%
\pgfsetmiterjoin%
\pgfsetlinewidth{0.803000pt}%
\definecolor{currentstroke}{rgb}{0.000000,0.000000,0.000000}%
\pgfsetstrokecolor{currentstroke}%
\pgfsetdash{}{0pt}%
\pgfpathmoveto{\pgfqpoint{5.489159in}{0.471179in}}%
\pgfpathlineto{\pgfqpoint{5.489159in}{3.317098in}}%
\pgfusepath{stroke}%
\end{pgfscope}%
\begin{pgfscope}%
\pgfsetrectcap%
\pgfsetmiterjoin%
\pgfsetlinewidth{0.803000pt}%
\definecolor{currentstroke}{rgb}{0.000000,0.000000,0.000000}%
\pgfsetstrokecolor{currentstroke}%
\pgfsetdash{}{0pt}%
\pgfpathmoveto{\pgfqpoint{0.762383in}{0.471179in}}%
\pgfpathlineto{\pgfqpoint{5.489159in}{0.471179in}}%
\pgfusepath{stroke}%
\end{pgfscope}%
\begin{pgfscope}%
\pgfsetrectcap%
\pgfsetmiterjoin%
\pgfsetlinewidth{0.803000pt}%
\definecolor{currentstroke}{rgb}{0.000000,0.000000,0.000000}%
\pgfsetstrokecolor{currentstroke}%
\pgfsetdash{}{0pt}%
\pgfpathmoveto{\pgfqpoint{0.762383in}{3.317098in}}%
\pgfpathlineto{\pgfqpoint{5.489159in}{3.317098in}}%
\pgfusepath{stroke}%
\end{pgfscope}%
\begin{pgfscope}%
\pgfsetbuttcap%
\pgfsetmiterjoin%
\definecolor{currentfill}{rgb}{1.000000,1.000000,1.000000}%
\pgfsetfillcolor{currentfill}%
\pgfsetfillopacity{0.800000}%
\pgfsetlinewidth{1.003750pt}%
\definecolor{currentstroke}{rgb}{0.800000,0.800000,0.800000}%
\pgfsetstrokecolor{currentstroke}%
\pgfsetstrokeopacity{0.800000}%
\pgfsetdash{}{0pt}%
\pgfpathmoveto{\pgfqpoint{0.840161in}{2.453544in}}%
\pgfpathlineto{\pgfqpoint{1.789553in}{2.453544in}}%
\pgfpathquadraticcurveto{\pgfqpoint{1.811775in}{2.453544in}}{\pgfqpoint{1.811775in}{2.475767in}}%
\pgfpathlineto{\pgfqpoint{1.811775in}{3.239321in}}%
\pgfpathquadraticcurveto{\pgfqpoint{1.811775in}{3.261543in}}{\pgfqpoint{1.789553in}{3.261543in}}%
\pgfpathlineto{\pgfqpoint{0.840161in}{3.261543in}}%
\pgfpathquadraticcurveto{\pgfqpoint{0.817939in}{3.261543in}}{\pgfqpoint{0.817939in}{3.239321in}}%
\pgfpathlineto{\pgfqpoint{0.817939in}{2.475767in}}%
\pgfpathquadraticcurveto{\pgfqpoint{0.817939in}{2.453544in}}{\pgfqpoint{0.840161in}{2.453544in}}%
\pgfpathclose%
\pgfusepath{stroke,fill}%
\end{pgfscope}%
\begin{pgfscope}%
\pgfsetrectcap%
\pgfsetroundjoin%
\pgfsetlinewidth{1.505625pt}%
\definecolor{currentstroke}{rgb}{0.121569,0.466667,0.705882}%
\pgfsetstrokecolor{currentstroke}%
\pgfsetdash{}{0pt}%
\pgfpathmoveto{\pgfqpoint{0.862383in}{3.178210in}}%
\pgfpathlineto{\pgfqpoint{1.084605in}{3.178210in}}%
\pgfusepath{stroke}%
\end{pgfscope}%
\begin{pgfscope}%
\pgftext[x=1.173494in,y=3.139321in,left,base]{\rmfamily\fontsize{8.000000}{9.600000}\selectfont OGI}%
\end{pgfscope}%
\begin{pgfscope}%
\pgfsetbuttcap%
\pgfsetroundjoin%
\pgfsetlinewidth{1.505625pt}%
\definecolor{currentstroke}{rgb}{1.000000,0.498039,0.054902}%
\pgfsetstrokecolor{currentstroke}%
\pgfsetdash{{5.550000pt}{2.400000pt}}{0.000000pt}%
\pgfpathmoveto{\pgfqpoint{0.862383in}{3.023277in}}%
\pgfpathlineto{\pgfqpoint{1.084605in}{3.023277in}}%
\pgfusepath{stroke}%
\end{pgfscope}%
\begin{pgfscope}%
\pgftext[x=1.173494in,y=2.984388in,left,base]{\rmfamily\fontsize{8.000000}{9.600000}\selectfont IDS}%
\end{pgfscope}%
\begin{pgfscope}%
\pgfsetbuttcap%
\pgfsetroundjoin%
\pgfsetlinewidth{1.505625pt}%
\definecolor{currentstroke}{rgb}{0.172549,0.627451,0.172549}%
\pgfsetstrokecolor{currentstroke}%
\pgfsetdash{{9.600000pt}{2.400000pt}{1.500000pt}{2.400000pt}}{0.000000pt}%
\pgfpathmoveto{\pgfqpoint{0.862383in}{2.868343in}}%
\pgfpathlineto{\pgfqpoint{1.084605in}{2.868343in}}%
\pgfusepath{stroke}%
\end{pgfscope}%
\begin{pgfscope}%
\pgftext[x=1.173494in,y=2.829455in,left,base]{\rmfamily\fontsize{8.000000}{9.600000}\selectfont Thompson}%
\end{pgfscope}%
\begin{pgfscope}%
\pgfsetbuttcap%
\pgfsetroundjoin%
\pgfsetlinewidth{1.505625pt}%
\definecolor{currentstroke}{rgb}{0.839216,0.152941,0.156863}%
\pgfsetstrokecolor{currentstroke}%
\pgfsetdash{{1.500000pt}{2.475000pt}}{0.000000pt}%
\pgfpathmoveto{\pgfqpoint{0.862383in}{2.713410in}}%
\pgfpathlineto{\pgfqpoint{1.084605in}{2.713410in}}%
\pgfusepath{stroke}%
\end{pgfscope}%
\begin{pgfscope}%
\pgftext[x=1.173494in,y=2.674522in,left,base]{\rmfamily\fontsize{8.000000}{9.600000}\selectfont Bayes UCB}%
\end{pgfscope}%
\begin{pgfscope}%
\pgfsetrectcap%
\pgfsetroundjoin%
\pgfsetlinewidth{1.505625pt}%
\definecolor{currentstroke}{rgb}{0.580392,0.403922,0.741176}%
\pgfsetstrokecolor{currentstroke}%
\pgfsetdash{}{0pt}%
\pgfpathmoveto{\pgfqpoint{0.862383in}{2.558477in}}%
\pgfpathlineto{\pgfqpoint{1.084605in}{2.558477in}}%
\pgfusepath{stroke}%
\end{pgfscope}%
\begin{pgfscope}%
\pgfsetbuttcap%
\pgfsetroundjoin%
\definecolor{currentfill}{rgb}{0.580392,0.403922,0.741176}%
\pgfsetfillcolor{currentfill}%
\pgfsetlinewidth{1.003750pt}%
\definecolor{currentstroke}{rgb}{0.580392,0.403922,0.741176}%
\pgfsetstrokecolor{currentstroke}%
\pgfsetdash{}{0pt}%
\pgfsys@defobject{currentmarker}{\pgfqpoint{-0.020833in}{-0.020833in}}{\pgfqpoint{0.020833in}{0.020833in}}{%
\pgfpathmoveto{\pgfqpoint{0.000000in}{-0.020833in}}%
\pgfpathcurveto{\pgfqpoint{0.005525in}{-0.020833in}}{\pgfqpoint{0.010825in}{-0.018638in}}{\pgfqpoint{0.014731in}{-0.014731in}}%
\pgfpathcurveto{\pgfqpoint{0.018638in}{-0.010825in}}{\pgfqpoint{0.020833in}{-0.005525in}}{\pgfqpoint{0.020833in}{0.000000in}}%
\pgfpathcurveto{\pgfqpoint{0.020833in}{0.005525in}}{\pgfqpoint{0.018638in}{0.010825in}}{\pgfqpoint{0.014731in}{0.014731in}}%
\pgfpathcurveto{\pgfqpoint{0.010825in}{0.018638in}}{\pgfqpoint{0.005525in}{0.020833in}}{\pgfqpoint{0.000000in}{0.020833in}}%
\pgfpathcurveto{\pgfqpoint{-0.005525in}{0.020833in}}{\pgfqpoint{-0.010825in}{0.018638in}}{\pgfqpoint{-0.014731in}{0.014731in}}%
\pgfpathcurveto{\pgfqpoint{-0.018638in}{0.010825in}}{\pgfqpoint{-0.020833in}{0.005525in}}{\pgfqpoint{-0.020833in}{0.000000in}}%
\pgfpathcurveto{\pgfqpoint{-0.020833in}{-0.005525in}}{\pgfqpoint{-0.018638in}{-0.010825in}}{\pgfqpoint{-0.014731in}{-0.014731in}}%
\pgfpathcurveto{\pgfqpoint{-0.010825in}{-0.018638in}}{\pgfqpoint{-0.005525in}{-0.020833in}}{\pgfqpoint{0.000000in}{-0.020833in}}%
\pgfpathclose%
\pgfusepath{stroke,fill}%
}%
\begin{pgfscope}%
\pgfsys@transformshift{0.973494in}{2.558477in}%
\pgfsys@useobject{currentmarker}{}%
\end{pgfscope}%
\end{pgfscope}%
\begin{pgfscope}%
\pgftext[x=1.173494in,y=2.519588in,left,base]{\rmfamily\fontsize{8.000000}{9.600000}\selectfont KL-UCB}%
\end{pgfscope}%
\end{pgfpicture}%
\makeatother%
\endgroup%

	\caption{Frequentist regret. The OGI policy is configured $K=1$ and $\alpha=100$.}
	\label{fig:kaufmann_regret}
\end{figure}

\subsection{Additional tables for Section~\ref{sec:experiments}}
\begin{table}
	\centering
	\begin{tabular}{rrrrrr} 
		\toprule
		{}    $\alpha$ &   $\beta$ &  OGI(1) &  OGI(3) &  OGI(5) &  Gittins \\
		\midrule
		   1 & 1 &   0.760 &   0.721 &   0.712 &    0.703 \\
		   1 & 2 &   0.571 &   0.522 &   0.511 &    0.500 \\
		   1 & 3 &   0.452 &   0.401 &   0.389 &    0.380 \\
		   1 & 4 &   0.374 &   0.321 &   0.312 &    0.302 \\
		  2 & 1 &   0.853 &   0.818 &   0.809 &    0.800 \\
		  2 & 2 &   0.702 &   0.657 &   0.646 &    0.635 \\
		  2 & 3 &   0.591 &   0.543 &   0.530 &    0.516 \\
		  2 & 4 &   0.508 &   0.458 &   0.445 &    0.434 \\
		  3 & 1 &   0.893 &   0.864 &   0.855 &    0.845 \\
		  3 & 2 &   0.771 &   0.729 &   0.719 &    0.707 \\
		  3 & 3 &   0.671 &   0.626 &   0.613 &    0.601 \\
		  3 & 4 &   0.592 &   0.545 &   0.532 &    0.518 \\
		 4 & 1 &   0.916 &   0.890 &   0.882 &    0.872 \\
		  4 & 2 &   0.813 &   0.776 &   0.765 &    0.754 \\
		  4 & 3 &   0.724 &   0.682 &   0.670 &    0.658 \\
		  4 & 4 &   0.651 &   0.607 &   0.593 &    0.581 \\
		\bottomrule
	\end{tabular}
	\caption{Optimistic and exact Gittins Indices when $\gamma = 0.9$ for different Beta-Bernoulli parameters}
	\label{table:ogi_table_for_gamma_9}
\end{table}

\begin{table}
	\centering
	\begin{tabular}{rrrrrr}
		\toprule
		{}    $\alpha$ &   $\beta$ &  OGI(1) &  OGI(3) &  OGI(5) &  Gittins \\
		\midrule
		 1.0 & 1.0 &   0.817 &   0.784 &   0.774 &    0.761 \\
		 1.0 & 2.0 &   0.637 &   0.590 &   0.577 &    0.560 \\
		 1.0 & 3.0 &   0.514 &   0.463 &   0.449 &    0.433 \\
		 1.0 & 4.0 &   0.430 &   0.376 &   0.364 &    0.348 \\
		 2.0 & 1.0 &   0.890 &   0.860 &   0.851 &    0.838 \\
		 2.0 & 2.0 &   0.752 &   0.710 &   0.698 &    0.681 \\
		 2.0 & 3.0 &   0.643 &   0.596 &   0.581 &    0.562 \\
		2.0 & 4.0 &   0.558 &   0.509 &   0.494 &    0.475 \\
		 3.0 & 1.0 &   0.921 &   0.896 &   0.887 &    0.874 \\
		3.0 & 2.0 &   0.811 &   0.773 &   0.762 &    0.744 \\
		 3.0 & 3.0 &   0.715 &   0.672 &   0.658 &    0.639 \\
		 3.0 & 4.0 &   0.637 &   0.591 &   0.575 &    0.556 \\
		4.0 & 1.0 &   0.938 &   0.916 &   0.908 &    0.895 \\
		4.0 & 2.0 &   0.847 &   0.812 &   0.801 &    0.784 \\
		4.0 & 3.0 &   0.763 &   0.722 &   0.709 &    0.690 \\
		4.0 & 4.0 &   0.691 &   0.648 &   0.633 &    0.613 \\
		\bottomrule
	\end{tabular}
	\caption{Optimistic and exact Gittins Indices when $\gamma = 0.95$ for different Beta-Bernoulli parameters}
	\label{table:ogi_table_for_gamma_95}
\end{table}

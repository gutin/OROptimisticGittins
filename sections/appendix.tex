\appendix

\section{Proof of Proposition~\ref{prop:gittins_log3T}} \label{proof:prop_log3T}
\begin{myproof}[Proof.]
	First, letting $\gamma_n = 1 - 1/n$, we show that
	\begin{equation} \label{eq:basic_finite_time_gittins}
	\Regret{\pi^{G, \gamma_n}, n} = O\left( \log^2(n) \right).
	\end{equation}
	Let $H \sim \text{Geo}(1/n)$ be an exogenous geometric random variable that is independent of $\theta$ and not observed by the agent. As an abbreviation, define $\mu^* = \E_{q}[\mu^*(\theta)]$. We then have
	\begin{align}
	\sum_{t=1}^\infty \gamma^{t-1} \E\left[X_{\pi^{G,\gamma_n},t}\right] & = \E\left[\sum_{t=1}^H X_{\pi^{G,\gamma_n}_t,t}\right] \nonumber \\
	& =  \E\left[H \mu^*(\theta) -\Regret{\pi^{G,\gamma_n}, H}\right] \\
	& =  n \mu^* - \E\left[\Regret{\pi^{G,\gamma_n}, H}\right] \nonumber \\
	& \le n \mu^* - \E\left[\Regret{\pi^{G,\gamma_n}, H} \given H > n\right]\P{H > n} \nonumber \\
	& \le n \mu^* - \E\left[\Regret{\pi^{G,\gamma_n}, n}\right](1-1/n)^n \nonumber \\
	& = n \mu^* - \E\left[\Regret{\pi^{G,\gamma_n}, n}\right](e^{-1} + o(1)) \label{bound:geometric_regret1}.
	\end{align}
	Let $q, Q$ be the density and CDF, respectively, of the prior distribution. Now, by Theorem 3, part 1 of \cite{lai1987adaptive}, there exists (an efficient) policy $\tilde \pi$, such that as $n$ becomes sufficiently large
	\begin{equation*}
	\Regret{\tilde \pi, n} \sim \left( A(A-1)  \int_\Theta q^2(\theta) Q^{A-2}(\theta) \; d\theta \right) \log^2 n.
	\end{equation*}
	Therefore for some prior-dependent constant $C_q$, we have $\Regret{\tilde \pi, n} \le C_q \log^2 n$. Let $\Delta(\theta)$ denote worst case  single period regret under parameter $\theta$, that is, $\Delta(\theta) =  \max_{i} \mu(\theta^*) - \mu(\theta_i)$. Let $\Delta$ denote its expectation over $\theta$, from which we obtain the lower bound,
	\begin{align}
	\E{\sum_{t=1}^H X_{\pi^{G,\gamma_n}_t,t}} & \ge \E\left[\sum_{t=1}^H X_{ \tilde \pi_t,t}\right] \label{bound:gittins_optimality}\\
	&  = \E\left[H\mu^*(\theta) - \Regret{\tilde \pi, H}\right] \nonumber \\
	& \ge \E\left[H\mu^*(\theta) -\Regret{\tilde \pi, H}\ind{H \ge e} - 2\ind{H < e}\Delta(\theta)\right] \nonumber \\
	& \ge n \mu^* - C_q \Ee{(\log (H))^2 \ind{H \ge 3} } - 2\Delta \nonumber \\
	& \ge n \mu^* - C_q \Ee{(\log (H))^2 \given H \ge 3}\P{H \ge 3} - 2\Delta \nonumber \\
	& \ge n \mu^* - C_q \log^2 (n+3)\P{H \ge 3} - 2\Delta\label{bound:jensens_ineq_for_log3t_proof}
	\end{align}
	where \eqref{bound:gittins_optimality} holds by optimality of the Gittins Index. The bound \eqref{bound:jensens_ineq_for_log3t_proof} follows from the memoryless property of the Geometric distribution, from Jensen's inequality and the fact that function $\log^2 x$ is a concave function on $[e,+\infty)$. Thus, equation \eqref{eq:basic_finite_time_gittins} is implied by the bounds \eqref{bound:geometric_regret1} and \eqref{bound:jensens_ineq_for_log3t_proof}.
	
	Now, for any policy $\pi$, we define $\tilde L_\pi(m) := m\mu^* - \sum_{t=1}^m X_{\pi_t,t}$ to be the random $m$ period shortfall against the expected Bayes' optimal arm and let $g_k = 1 - 1/2^{k-1}$. We break up the time horizon $T$ into geometrically growing epochs and bound, conservatively, the Bayes' risk in each one:
	\begin{align}
	\Regret{\pi^D, T} & \le \Regret{\pi^D, 2^{\ceil{\log_2 T}}}\\
	& = \sum_{k=1}^{\ceil{\log_2 T}} \Ee{\Ee{ \tilde L_{\pi^D}(2^{k-1}) \given[\Big] \mathcal F_{2^{k-1}-1}}} \nonumber \\
	& =\sum_{k=1}^{\ceil{\log_2 T}} \Ee{\Ee{ \tilde L_{\pi^{G, g_k}}(2^{k-1}) \given[\Big] \mathcal F_{2^{k-1}-1}}} \nonumber \\
	& \le \sum_{k=1}^{\ceil{\log_2 T}} \Ee{\Ee{ \tilde L_{\pi^{G,g_k}}(2^{k-1}) \given[\Big] \mathcal F_{0}}} \label{bound:super_mg} \\
	& = \sum_{k=1}^{\ceil{\log_2 T}} \Regret{\pi^{G, g_k}, 2^{k-1}} =  O\left(\sum_{k=1}^{\ceil{\log_2 T}} k^2 \right)\label{eq:order_optimal_bayes} \\
	& = O(\log^3 T) \nonumber
	\end{align}
	where \eqref{eq:order_optimal_bayes} follows from equation \eqref{eq:basic_finite_time_gittins} and \eqref{bound:super_mg} holds because regret increases if history is discarded.
\end{myproof}

\subsection{Proof of Lemma~\ref{lemma:approx_bound}} \label{prf:approx_bound}
\begin{myproof}[Proof.]
	Fix $\lambda > 0$ and an arm $i$. Let $V_\lambda(y)$ be the value of the RHS of \eqref{eqn:gittins_index} with the per-period reward  of $\lambda$, and define $\hat V^K_\lambda(y)$, similarly, for problem \eqref{eqn:ogi_general} (where $y$ is, as before, the state of an arm). Notice that because rewards are generated according to an unknown parameter $\theta_i$, which needs to be learned, that if we condition on a fixed $\theta_i$, we have for any stopping time $\tau$ that
	\begin{align} 
	\Ee{ \sum_{t=1}^{\tau-1} \gamma^{t-1}X_{i,t} + \gamma^{\tau-1} \frac{\lambda}{1-\gamma} \given[\Big] \theta_i} & \le  \Ee{ \sum_{t=1}^{\tau-1} \gamma^{t-1}\mu(\theta_i) + \gamma^{\tau-1} \frac{\lambda}{1-\gamma} \given[\Big] \theta_i} \label{bound:perfect_info}
	\end{align}
	where the expectation is also taken over the agent's prior on $\theta_i$. Simply put, the best performance in the bandit game can be achieved if the parameter governing expected rewards is known from the beginning by the agent. Now recall that $R(y_{i,t})$ is a random variable drawn from the prior on the arm's mean reward at time $t$. We also define the function 
	\[
	r_{\lambda,K}(t,x) = \begin{cases}
	\lambda & t < K \\
	\max(x,\lambda) & \text{otherwise}
	\end{cases}
	\] 
	Let $\tau$ be the stopping time at which the agent retires and define $\tau_K = \tau \wedge (K+1)$. We then bound $V_\lambda(y)$,
	\begin{align}
	V_\lambda(y) & =  \sup_{\tau > 1}\Ee{\sum_{t=1}^{\tau-1} \gamma^{t-1} X_{i,t} + \gamma^{\tau-1} \frac{\lambda}{1-\gamma}
		\given[\Big] y_{i,1} = y} \nonumber \\
	& = \sup_{\tau > 1}\Ee{\Ee{\sum_{t=1}^{\tau-1} \gamma^{t-1} X_{i,t} + \gamma^{\tau-1} \frac{\lambda}{1-\gamma}
			\given[\Big] \theta_i}\given[\Big] y_{i,1} = y} \nonumber \\
	& = \sup_{\tau > 1}\Ee{\sum_{t=1}^{\tau_K-1} \gamma^{t-1} X_{i,t} + \Ee{\sum_{t=\tau_K}^{\tau-1} \gamma^{t-1} X_{i,t} + \gamma^{\tau-1} \frac{\lambda}{1-\gamma} \given[\Big] \theta_i}\given[\Big] y_{i,1} = y} \nonumber \\
	& \le \sup_{\tau > 1}\Ee{\sum_{t=1}^{\tau_K-1} \gamma^{t-1} X_{i,t} + \Ee{\sum_{t=\tau_K}^{\tau-1} \gamma^{t-1} \mu(\theta_i) + \gamma^{\tau-1} \frac{\lambda}{1-\gamma} \given[\Big] \theta_i}\given[\Big] y_{i,1} = y} \label{bound:appl_of_perfect_info} \\
	& = \sup_{\tau > 1}\Ee{\sum_{t=1}^{\tau_K-1} \gamma^{t-1} X_{i,t} +  \Ee{ \gamma^{\tau_K - 1} \frac{r_{\lambda,K}(\tau_K, \mu(\theta_i))}{1-\gamma} \given[\Big] \theta_i}\given[\Big] y_{i,1} = y} \nonumber \\
	& = \sup_{\tau > 1}\Ee{\sum_{t=1}^{\tau_K-1} \gamma^{t-1} X_{i,t} +  \frac{\gamma^{\tau_K - 1}}{1-\gamma}\Ee{\Ee{  r_{\lambda,K}( \tau_K, \mu(\theta_i)) \given[\Big] \theta_i} \given[\Big] \mathcal{F}_{\tau_K-1}}} \nonumber \\
	& = \sup_{\tau > 1}\Ee{\sum_{t=1}^{\tau_K-1} \gamma^{t-1} X_{i,t} +  \frac{\gamma^{\tau_K - 1}}{1-\gamma}\underbrace{\Ee{  r_{\lambda,K}( \tau_K, R(y_{i,\tau_K}))}}_{ = R_{\lambda,K}(\tau_K, y_{i,\tau_K})}\given[\Big] y_{i,1} = y} \nonumber \\
	& = \sup_{1 < \tau \le K+1}\Ee{\sum_{t=1}^{\tau} \gamma^{t-1} X_{i,t} +  \gamma^{\tau - 1} \frac{ R_{\lambda,K}(\tau, y_{i,\tau})}{1-\gamma}\given[\Big] y_{i,1} = y}  = \hat V^K_\lambda(y). \nonumber
	\end{align}
	The main step in the above is \eqref{bound:appl_of_perfect_info} where we bound on the inner conditional expectation (in terms of $\theta_i$) by applying \eqref{bound:perfect_info}. We also used the fact that $\mu(\theta_i) \; | \; \mathcal F_{t-1} \sim R(y_{i,t})$ for all $t$. Finally observe that both $\hat V^K_\lambda(y)$ and $V_\lambda(y)$ are increasing in $\lambda$ for any fixed $y$. Therefore if $\lambda_1 = (1-\gamma) \hat V^K_{\lambda_1}(y)$ and $\lambda_2 = (1-\gamma) V_{\lambda_2}(y)$, then, because $V_{\lambda}(y) \le \hat V^K_{\lambda}(y)$ for any $\lambda$, it must be that  $\lambda_1 \ge \lambda_2$. A simple argument shows this, which we omit.
\end{myproof}
\subsection{Results for the frequentist regret bound proof}
\subsubsection{Definitions and properties of Binomial distributions.}
We list notation and facts related to Beta and Binomial distributions, which are used through this section.
\begin{definition}
	$F^B_{n,p}(.)$ is the CDF of the Binomial distribution with parameters $n$ and $p$, and $F^\beta_{a,b}(.)$ is the CDF of the Beta distribution with parameters $a$ and $b$.
\end{definition}

\begin{lemma} \label{fact:equation_for_beta_binomial_cdfs}
	Let $a$ and $b$ be positive integers and $y \in [0,1]$, 
	\[
	F^\beta_{a,b}(y) = 1 - F^B_{a+b-1,y}(a-1)
	\]
\end{lemma}
\begin{myproof}[Proof.]
	Proof is found in \cite{agrawalanalysis}.
\end{myproof}
\begin{lemma} \label{fact:median_of_binomial_dist}
	The median of a Binomial$(n,p)$ distribution is either $\ceil{np}$ or $\floor{np}$.
\end{lemma}
\begin{myproof}[Proof]
	Proof is found in \cite{jogdeo1968monotone}.
\end{myproof}

\begin{corollary}[Corollary of Fact~\ref{fact:median_of_binomial_dist}] \label{cor:corollarly_of_binomial_median_property}
	Let $n$ be a positive integer and $p \in (0,1)$. For any non-negative integer $s < np$
	\[
	F_{n,p}(s) \le 1/2
	\]
\end{corollary}

\begin{lemma} \label{fact:relationship_with_binom_cdfs}
	Let $n$ be a positive integer and $p \in [0,1]$. Then for any $k \in \{0,\ldots,n\}$,
	\[
	(1-p)F_{n-1,p}(k)\le F_{n,p}(k) \le F^B_{n-1,p}(k)
	\] 
\end{lemma}
\begin{myproof}[Proof]
	To prove $F_{n,p}(k) \le F^B_{n-1,p}(k)$, we let $X_1,\ldots,X_{n}$ be i.i.d samples from a Bernoulli($p$) distribution. We then have
	\begin{align*}
	F^B_{n,p}(k)  = \P{\sum_{i=1}^{n} X_i \le k}  \le  \P{\sum_{i=1}^{n-1} X_i \le k}  = F^B_{n-1,p}(k)
	\end{align*}
	Now to prove $(1-p)F_{n-1,p}(k)\le F_{n,p}(k)$, it's enough to observe that $F_{n,p}(k) = p F_{n-1,p}(k-1) + (1-p) F_{n-1,p}(k)$.
\end{myproof}

\subsubsection{Ratio of Binomial CDFs.} \label{sec:ratio_of_bin_cdfs}
\begin{lemma} \label{lemma:ratio_of_cdfs}
	Let $0< q < p < 1$. Let $n$ be a positive integer such that $e^{\frac{n}{2} d(q,p)} \ge (n+1)^4$ and let $k$ be a non-negative integer such that $k < nq$. It then follows that
	\[
	F^B_{n,q}(k)/F^B_{n,p}(k) >  e^{\frac{n}{2} d(q,p)}.
	\]
\end{lemma}
\begin{proof}[Proof.]
	From the method of types  (see \cite{cover2012elements}), we have for any $r \in (0,1)$ and $j < nr$
	\begin{equation} \label{eqn:appl_of_sanovs}
	\frac{e^{-nd(j/n, r)}}{(1+n)^2}\le F_{n,r}(j) \le (n+1)^2 e^{- n d(j/n, r)}.
	\end{equation}
	Because $k < nq < np$, by applying \eqref{eqn:appl_of_sanovs} to both the numerator and denominator, we get
	\begin{align*}
	\frac{F_{n,q}(k)}{F_{n,p}(k)} & \ge  \frac{e^{-nd(k/n, q)}}{(n+1)^4 e^{- n d(k/n, p)}} = \frac{e^{n(d(k/n,p) - d(k/n,q))}}{(n+1)^4}.
	\end{align*}
	Examining the exponent, we find
	\begin{align*}
	d(k/n, p) - d(k/n,q) & = \frac{k}{n} \log \frac{q}{p} + \left(1-\frac{k}{n}\right)\log \frac{1-q}{1-p} \\
	& > q \log \frac{q}{p} + (1-q)\log \frac{1-q}{1-p} \\
	& = d(q,p)
	\end{align*}
	where the bound holds because the expression is decreasing in $k$, and $k < nq$. Therefore,
	\begin{align}
	\frac{F_{n,q}(k)}{F_{n,p}(k)} & > \frac{e^{n  d(q,p)}}{(n+1)^4} = \frac{e^{\frac{n}{2}d(q,p)}}{(n+1)^4} e^{\frac{n}{2}d(q,p)} \ge e^{\frac{n}{2}d(q,p)} \label{bound:log_1minusq_etc}.
	\end{align}
	The final lower bound in \eqref{bound:log_1minusq_etc} follows from the assumption on $n$.
\end{proof}

\section{Optimistic Gittins Index results.} \label{sec:amgi_results}
For this section it is useful to define the value of a stopping problem used in the calculation of the Optimistic Gittins Index. For any fixed arm $i$, we write
\begin{align*}
V_K(y; x, \gamma) & := \sup_{1 < \tau \le K} \Ee{\sum_{t=1}^{\tau-1} (1-\gamma)X_{i,t} + \gamma^{\tau-1} R_{x, K}(\tau, y_{i,\tau})	\given y_{i,1} = y}.
\end{align*}
Therefore the Optimistic Gittins Index, in state $y$ with discount factor $\gamma$, is the solution in $x$ to $V_K(y; x, \gamma) = x$. We show some key properties of $V_K$, which we exploit later on. For fixed $y$, $K$ and $\gamma$, we will call $V(x) \defeq V(y; x, \gamma)$.
\begin{fact}
	$V(x)$ is convex and differentiable.
\end{fact}
\begin{myproof}[Proof.]
	We prove this by induction. For $K = 1$, we have $\tau = 2$ almost surely and so
	\begin{align*}
	V(x) & = V_1(y; x, \gamma) \\
	& = (1-\gamma_t)\Ee{X_{i,1} \given y_{i,1 = y}} + \gamma_t \Ee{R_{x, 1}(2, y_{i,1}) \given y_{i,1} = y} \\
	& = (1-\gamma_t)\Ee{X_{i,1} \given y_{i,1 = y}} + \gamma_t \Ee{\max(x, \E{X_{i,1}}) \given y_{i,1} = y}
	\end{align*}
	The function is convex because for any random variable $Z$ the term $\max(g, Z)$ is convex and taking expectations preserves convexity. Also, we verify through the bounded convergence theorem that $V(x)$ is differentiable and the event $\{\E X_{i,1}=z \given y_{i,1} = y\}$, at which $V$ is not differentiable, has measure zero.
	
	For $K > 1$, assume that $V_{K-1}$ is convex and differentiable. By writing the Bellman equation
	\begin{align*}
		V_K(y; x, \gamma) & = (1-\gamma_t)\Ee{X_{i,1} \given y_{i,1}=y} + \Ee{\gamma_t \max(x, V_{K-1}(y_{i,t+1}; x, \gamma)) \given y_{i,1} = y}
	\end{align*}
	we again notice a maximum of convex functions in $x$. This form for $V_K$ implies that it is convex and differentiable.
\end{myproof}
\begin{lemma} \label{eq:important_fact}
	Let $\gamma \in (0,1)$ and
	\begin{equation} \label{eq:def_lambda_in_important_equiv}
	\lambda = \sup\{ x \in [0,1] : V(x) \ge x\}
	\end{equation} 
	For all $x \in (0,1)$, the following equivalence holds
	\begin{equation} \label{eq:equivlance_lambda1}
	\lambda < x  \Longleftrightarrow V(x) < x.
	\end{equation}
\end{lemma}
\begin{myproof}[Proof.]
	Figure~\ref{fig:visaulize_gx_proof} gives a visualization of the proof. In the plot, the point $x^*$, where $V(x^*)$ and $x^*$ intersect, is the Optimistic Gittins Index. Notice how if $x > x^*$, $V(x)$ is below $x$ and otherwise $V(x)$ is above $x$; this is the crux of the proof.
	
	We also give a formal proof. Firstly let's assume that $\lambda < x$. If $x\le V(x)$, then $\lambda$ would not be the supremum over all $z \in [0,1]$ such that $ z \le V(z)$. Therefore $V(x) < x$.
	
	\begin{figure}
		\centering
		%% Creator: Matplotlib, PGF backend
%%
%% To include the figure in your LaTeX document, write
%%   \input{<filename>.pgf}
%%
%% Make sure the required packages are loaded in your preamble
%%   \usepackage{pgf}
%%
%% Figures using additional raster images can only be included by \input if
%% they are in the same directory as the main LaTeX file. For loading figures
%% from other directories you can use the `import` package
%%   \usepackage{import}
%% and then include the figures with
%%   \import{<path to file>}{<filename>.pgf}
%%
%% Matplotlib used the following preamble
%%   \usepackage[utf8x]{inputenc}
%%   \usepackage[T1]{fontenc}
%%
\begingroup%
\makeatletter%
\begin{pgfpicture}%
\pgfpathrectangle{\pgfpointorigin}{\pgfqpoint{3.900000in}{2.410333in}}%
\pgfusepath{use as bounding box, clip}%
\begin{pgfscope}%
\pgfsetbuttcap%
\pgfsetmiterjoin%
\definecolor{currentfill}{rgb}{1.000000,1.000000,1.000000}%
\pgfsetfillcolor{currentfill}%
\pgfsetlinewidth{0.000000pt}%
\definecolor{currentstroke}{rgb}{1.000000,1.000000,1.000000}%
\pgfsetstrokecolor{currentstroke}%
\pgfsetdash{}{0pt}%
\pgfpathmoveto{\pgfqpoint{0.000000in}{0.000000in}}%
\pgfpathlineto{\pgfqpoint{3.900000in}{0.000000in}}%
\pgfpathlineto{\pgfqpoint{3.900000in}{2.410333in}}%
\pgfpathlineto{\pgfqpoint{0.000000in}{2.410333in}}%
\pgfpathclose%
\pgfusepath{fill}%
\end{pgfscope}%
\begin{pgfscope}%
\pgfsetbuttcap%
\pgfsetmiterjoin%
\definecolor{currentfill}{rgb}{1.000000,1.000000,1.000000}%
\pgfsetfillcolor{currentfill}%
\pgfsetlinewidth{0.000000pt}%
\definecolor{currentstroke}{rgb}{0.000000,0.000000,0.000000}%
\pgfsetstrokecolor{currentstroke}%
\pgfsetstrokeopacity{0.000000}%
\pgfsetdash{}{0pt}%
\pgfpathmoveto{\pgfqpoint{0.487500in}{0.301292in}}%
\pgfpathlineto{\pgfqpoint{3.510000in}{0.301292in}}%
\pgfpathlineto{\pgfqpoint{3.510000in}{2.169299in}}%
\pgfpathlineto{\pgfqpoint{0.487500in}{2.169299in}}%
\pgfpathclose%
\pgfusepath{fill}%
\end{pgfscope}%
\begin{pgfscope}%
\pgfpathrectangle{\pgfqpoint{0.487500in}{0.301292in}}{\pgfqpoint{3.022500in}{1.868008in}} %
\pgfusepath{clip}%
\pgfsetbuttcap%
\pgfsetroundjoin%
\pgfsetlinewidth{1.003750pt}%
\definecolor{currentstroke}{rgb}{1.000000,0.000000,0.000000}%
\pgfsetstrokecolor{currentstroke}%
\pgfsetdash{{6.000000pt}{6.000000pt}}{0.000000pt}%
\pgfpathmoveto{\pgfqpoint{0.487500in}{0.301292in}}%
\pgfpathlineto{\pgfqpoint{0.591724in}{0.365706in}}%
\pgfpathlineto{\pgfqpoint{0.695948in}{0.430120in}}%
\pgfpathlineto{\pgfqpoint{0.800172in}{0.494534in}}%
\pgfpathlineto{\pgfqpoint{0.904397in}{0.558948in}}%
\pgfpathlineto{\pgfqpoint{1.008621in}{0.623362in}}%
\pgfpathlineto{\pgfqpoint{1.112845in}{0.687776in}}%
\pgfpathlineto{\pgfqpoint{1.217069in}{0.752190in}}%
\pgfpathlineto{\pgfqpoint{1.321293in}{0.816604in}}%
\pgfpathlineto{\pgfqpoint{1.425517in}{0.881018in}}%
\pgfpathlineto{\pgfqpoint{1.529741in}{0.945432in}}%
\pgfpathlineto{\pgfqpoint{1.633966in}{1.009846in}}%
\pgfpathlineto{\pgfqpoint{1.738190in}{1.074260in}}%
\pgfpathlineto{\pgfqpoint{1.842414in}{1.138674in}}%
\pgfpathlineto{\pgfqpoint{1.946638in}{1.203088in}}%
\pgfpathlineto{\pgfqpoint{2.050862in}{1.267502in}}%
\pgfpathlineto{\pgfqpoint{2.155086in}{1.331917in}}%
\pgfpathlineto{\pgfqpoint{2.259310in}{1.396331in}}%
\pgfpathlineto{\pgfqpoint{2.363534in}{1.460745in}}%
\pgfpathlineto{\pgfqpoint{2.467759in}{1.525159in}}%
\pgfpathlineto{\pgfqpoint{2.571983in}{1.589573in}}%
\pgfpathlineto{\pgfqpoint{2.676207in}{1.653987in}}%
\pgfpathlineto{\pgfqpoint{2.780431in}{1.718401in}}%
\pgfpathlineto{\pgfqpoint{2.884655in}{1.782815in}}%
\pgfpathlineto{\pgfqpoint{2.988879in}{1.847229in}}%
\pgfpathlineto{\pgfqpoint{3.093103in}{1.911643in}}%
\pgfpathlineto{\pgfqpoint{3.197328in}{1.976057in}}%
\pgfpathlineto{\pgfqpoint{3.301552in}{2.040471in}}%
\pgfpathlineto{\pgfqpoint{3.405776in}{2.104885in}}%
\pgfpathlineto{\pgfqpoint{3.510000in}{2.169299in}}%
\pgfusepath{stroke}%
\end{pgfscope}%
\begin{pgfscope}%
\pgfpathrectangle{\pgfqpoint{0.487500in}{0.301292in}}{\pgfqpoint{3.022500in}{1.868008in}} %
\pgfusepath{clip}%
\pgfsetrectcap%
\pgfsetroundjoin%
\pgfsetlinewidth{1.003750pt}%
\definecolor{currentstroke}{rgb}{0.000000,0.000000,1.000000}%
\pgfsetstrokecolor{currentstroke}%
\pgfsetdash{}{0pt}%
\pgfpathmoveto{\pgfqpoint{0.487500in}{0.835008in}}%
\pgfpathlineto{\pgfqpoint{0.591724in}{0.835131in}}%
\pgfpathlineto{\pgfqpoint{0.695948in}{0.835923in}}%
\pgfpathlineto{\pgfqpoint{0.800172in}{0.837886in}}%
\pgfpathlineto{\pgfqpoint{0.904397in}{0.841366in}}%
\pgfpathlineto{\pgfqpoint{1.008621in}{0.846579in}}%
\pgfpathlineto{\pgfqpoint{1.112845in}{0.853633in}}%
\pgfpathlineto{\pgfqpoint{1.217069in}{0.862552in}}%
\pgfpathlineto{\pgfqpoint{1.321293in}{0.873295in}}%
\pgfpathlineto{\pgfqpoint{1.425517in}{0.898119in}}%
\pgfpathlineto{\pgfqpoint{1.529741in}{0.932953in}}%
\pgfpathlineto{\pgfqpoint{1.633966in}{0.968286in}}%
\pgfpathlineto{\pgfqpoint{1.738190in}{1.004555in}}%
\pgfpathlineto{\pgfqpoint{1.842414in}{1.047574in}}%
\pgfpathlineto{\pgfqpoint{1.946638in}{1.092664in}}%
\pgfpathlineto{\pgfqpoint{2.050862in}{1.137754in}}%
\pgfpathlineto{\pgfqpoint{2.155086in}{1.182844in}}%
\pgfpathlineto{\pgfqpoint{2.259310in}{1.227934in}}%
\pgfpathlineto{\pgfqpoint{2.363534in}{1.273024in}}%
\pgfpathlineto{\pgfqpoint{2.467759in}{1.318114in}}%
\pgfpathlineto{\pgfqpoint{2.571983in}{1.363203in}}%
\pgfpathlineto{\pgfqpoint{2.676207in}{1.408293in}}%
\pgfpathlineto{\pgfqpoint{2.780431in}{1.453383in}}%
\pgfpathlineto{\pgfqpoint{2.884655in}{1.498473in}}%
\pgfpathlineto{\pgfqpoint{2.988879in}{1.543563in}}%
\pgfpathlineto{\pgfqpoint{3.093103in}{1.588653in}}%
\pgfpathlineto{\pgfqpoint{3.197328in}{1.633742in}}%
\pgfpathlineto{\pgfqpoint{3.301552in}{1.678832in}}%
\pgfpathlineto{\pgfqpoint{3.405776in}{1.723922in}}%
\pgfpathlineto{\pgfqpoint{3.510000in}{1.769012in}}%
\pgfusepath{stroke}%
\end{pgfscope}%
\begin{pgfscope}%
\pgfsetrectcap%
\pgfsetmiterjoin%
\pgfsetlinewidth{1.003750pt}%
\definecolor{currentstroke}{rgb}{0.000000,0.000000,0.000000}%
\pgfsetstrokecolor{currentstroke}%
\pgfsetdash{}{0pt}%
\pgfpathmoveto{\pgfqpoint{0.487500in}{2.169299in}}%
\pgfpathlineto{\pgfqpoint{3.510000in}{2.169299in}}%
\pgfusepath{stroke}%
\end{pgfscope}%
\begin{pgfscope}%
\pgfsetrectcap%
\pgfsetmiterjoin%
\pgfsetlinewidth{1.003750pt}%
\definecolor{currentstroke}{rgb}{0.000000,0.000000,0.000000}%
\pgfsetstrokecolor{currentstroke}%
\pgfsetdash{}{0pt}%
\pgfpathmoveto{\pgfqpoint{3.510000in}{0.301292in}}%
\pgfpathlineto{\pgfqpoint{3.510000in}{2.169299in}}%
\pgfusepath{stroke}%
\end{pgfscope}%
\begin{pgfscope}%
\pgfsetrectcap%
\pgfsetmiterjoin%
\pgfsetlinewidth{1.003750pt}%
\definecolor{currentstroke}{rgb}{0.000000,0.000000,0.000000}%
\pgfsetstrokecolor{currentstroke}%
\pgfsetdash{}{0pt}%
\pgfpathmoveto{\pgfqpoint{0.487500in}{0.301292in}}%
\pgfpathlineto{\pgfqpoint{3.510000in}{0.301292in}}%
\pgfusepath{stroke}%
\end{pgfscope}%
\begin{pgfscope}%
\pgfsetrectcap%
\pgfsetmiterjoin%
\pgfsetlinewidth{1.003750pt}%
\definecolor{currentstroke}{rgb}{0.000000,0.000000,0.000000}%
\pgfsetstrokecolor{currentstroke}%
\pgfsetdash{}{0pt}%
\pgfpathmoveto{\pgfqpoint{0.487500in}{0.301292in}}%
\pgfpathlineto{\pgfqpoint{0.487500in}{2.169299in}}%
\pgfusepath{stroke}%
\end{pgfscope}%
\begin{pgfscope}%
\pgfsetbuttcap%
\pgfsetroundjoin%
\definecolor{currentfill}{rgb}{0.000000,0.000000,0.000000}%
\pgfsetfillcolor{currentfill}%
\pgfsetlinewidth{0.501875pt}%
\definecolor{currentstroke}{rgb}{0.000000,0.000000,0.000000}%
\pgfsetstrokecolor{currentstroke}%
\pgfsetdash{}{0pt}%
\pgfsys@defobject{currentmarker}{\pgfqpoint{0.000000in}{0.000000in}}{\pgfqpoint{0.000000in}{0.055556in}}{%
\pgfpathmoveto{\pgfqpoint{0.000000in}{0.000000in}}%
\pgfpathlineto{\pgfqpoint{0.000000in}{0.055556in}}%
\pgfusepath{stroke,fill}%
}%
\begin{pgfscope}%
\pgfsys@transformshift{0.487500in}{0.301292in}%
\pgfsys@useobject{currentmarker}{}%
\end{pgfscope}%
\end{pgfscope}%
\begin{pgfscope}%
\pgfsetbuttcap%
\pgfsetroundjoin%
\definecolor{currentfill}{rgb}{0.000000,0.000000,0.000000}%
\pgfsetfillcolor{currentfill}%
\pgfsetlinewidth{0.501875pt}%
\definecolor{currentstroke}{rgb}{0.000000,0.000000,0.000000}%
\pgfsetstrokecolor{currentstroke}%
\pgfsetdash{}{0pt}%
\pgfsys@defobject{currentmarker}{\pgfqpoint{0.000000in}{-0.055556in}}{\pgfqpoint{0.000000in}{0.000000in}}{%
\pgfpathmoveto{\pgfqpoint{0.000000in}{0.000000in}}%
\pgfpathlineto{\pgfqpoint{0.000000in}{-0.055556in}}%
\pgfusepath{stroke,fill}%
}%
\begin{pgfscope}%
\pgfsys@transformshift{0.487500in}{2.169299in}%
\pgfsys@useobject{currentmarker}{}%
\end{pgfscope}%
\end{pgfscope}%
\begin{pgfscope}%
\pgftext[x=0.487500in,y=0.245736in,,top]{\rmfamily\fontsize{8.000000}{9.600000}\selectfont \(\displaystyle 0.0\)}%
\end{pgfscope}%
\begin{pgfscope}%
\pgfsetbuttcap%
\pgfsetroundjoin%
\definecolor{currentfill}{rgb}{0.000000,0.000000,0.000000}%
\pgfsetfillcolor{currentfill}%
\pgfsetlinewidth{0.501875pt}%
\definecolor{currentstroke}{rgb}{0.000000,0.000000,0.000000}%
\pgfsetstrokecolor{currentstroke}%
\pgfsetdash{}{0pt}%
\pgfsys@defobject{currentmarker}{\pgfqpoint{0.000000in}{0.000000in}}{\pgfqpoint{0.000000in}{0.055556in}}{%
\pgfpathmoveto{\pgfqpoint{0.000000in}{0.000000in}}%
\pgfpathlineto{\pgfqpoint{0.000000in}{0.055556in}}%
\pgfusepath{stroke,fill}%
}%
\begin{pgfscope}%
\pgfsys@transformshift{1.092000in}{0.301292in}%
\pgfsys@useobject{currentmarker}{}%
\end{pgfscope}%
\end{pgfscope}%
\begin{pgfscope}%
\pgfsetbuttcap%
\pgfsetroundjoin%
\definecolor{currentfill}{rgb}{0.000000,0.000000,0.000000}%
\pgfsetfillcolor{currentfill}%
\pgfsetlinewidth{0.501875pt}%
\definecolor{currentstroke}{rgb}{0.000000,0.000000,0.000000}%
\pgfsetstrokecolor{currentstroke}%
\pgfsetdash{}{0pt}%
\pgfsys@defobject{currentmarker}{\pgfqpoint{0.000000in}{-0.055556in}}{\pgfqpoint{0.000000in}{0.000000in}}{%
\pgfpathmoveto{\pgfqpoint{0.000000in}{0.000000in}}%
\pgfpathlineto{\pgfqpoint{0.000000in}{-0.055556in}}%
\pgfusepath{stroke,fill}%
}%
\begin{pgfscope}%
\pgfsys@transformshift{1.092000in}{2.169299in}%
\pgfsys@useobject{currentmarker}{}%
\end{pgfscope}%
\end{pgfscope}%
\begin{pgfscope}%
\pgftext[x=1.092000in,y=0.245736in,,top]{\rmfamily\fontsize{8.000000}{9.600000}\selectfont \(\displaystyle 0.2\)}%
\end{pgfscope}%
\begin{pgfscope}%
\pgfsetbuttcap%
\pgfsetroundjoin%
\definecolor{currentfill}{rgb}{0.000000,0.000000,0.000000}%
\pgfsetfillcolor{currentfill}%
\pgfsetlinewidth{0.501875pt}%
\definecolor{currentstroke}{rgb}{0.000000,0.000000,0.000000}%
\pgfsetstrokecolor{currentstroke}%
\pgfsetdash{}{0pt}%
\pgfsys@defobject{currentmarker}{\pgfqpoint{0.000000in}{0.000000in}}{\pgfqpoint{0.000000in}{0.055556in}}{%
\pgfpathmoveto{\pgfqpoint{0.000000in}{0.000000in}}%
\pgfpathlineto{\pgfqpoint{0.000000in}{0.055556in}}%
\pgfusepath{stroke,fill}%
}%
\begin{pgfscope}%
\pgfsys@transformshift{1.696500in}{0.301292in}%
\pgfsys@useobject{currentmarker}{}%
\end{pgfscope}%
\end{pgfscope}%
\begin{pgfscope}%
\pgfsetbuttcap%
\pgfsetroundjoin%
\definecolor{currentfill}{rgb}{0.000000,0.000000,0.000000}%
\pgfsetfillcolor{currentfill}%
\pgfsetlinewidth{0.501875pt}%
\definecolor{currentstroke}{rgb}{0.000000,0.000000,0.000000}%
\pgfsetstrokecolor{currentstroke}%
\pgfsetdash{}{0pt}%
\pgfsys@defobject{currentmarker}{\pgfqpoint{0.000000in}{-0.055556in}}{\pgfqpoint{0.000000in}{0.000000in}}{%
\pgfpathmoveto{\pgfqpoint{0.000000in}{0.000000in}}%
\pgfpathlineto{\pgfqpoint{0.000000in}{-0.055556in}}%
\pgfusepath{stroke,fill}%
}%
\begin{pgfscope}%
\pgfsys@transformshift{1.696500in}{2.169299in}%
\pgfsys@useobject{currentmarker}{}%
\end{pgfscope}%
\end{pgfscope}%
\begin{pgfscope}%
\pgftext[x=1.696500in,y=0.245736in,,top]{\rmfamily\fontsize{8.000000}{9.600000}\selectfont \(\displaystyle 0.4\)}%
\end{pgfscope}%
\begin{pgfscope}%
\pgfsetbuttcap%
\pgfsetroundjoin%
\definecolor{currentfill}{rgb}{0.000000,0.000000,0.000000}%
\pgfsetfillcolor{currentfill}%
\pgfsetlinewidth{0.501875pt}%
\definecolor{currentstroke}{rgb}{0.000000,0.000000,0.000000}%
\pgfsetstrokecolor{currentstroke}%
\pgfsetdash{}{0pt}%
\pgfsys@defobject{currentmarker}{\pgfqpoint{0.000000in}{0.000000in}}{\pgfqpoint{0.000000in}{0.055556in}}{%
\pgfpathmoveto{\pgfqpoint{0.000000in}{0.000000in}}%
\pgfpathlineto{\pgfqpoint{0.000000in}{0.055556in}}%
\pgfusepath{stroke,fill}%
}%
\begin{pgfscope}%
\pgfsys@transformshift{2.301000in}{0.301292in}%
\pgfsys@useobject{currentmarker}{}%
\end{pgfscope}%
\end{pgfscope}%
\begin{pgfscope}%
\pgfsetbuttcap%
\pgfsetroundjoin%
\definecolor{currentfill}{rgb}{0.000000,0.000000,0.000000}%
\pgfsetfillcolor{currentfill}%
\pgfsetlinewidth{0.501875pt}%
\definecolor{currentstroke}{rgb}{0.000000,0.000000,0.000000}%
\pgfsetstrokecolor{currentstroke}%
\pgfsetdash{}{0pt}%
\pgfsys@defobject{currentmarker}{\pgfqpoint{0.000000in}{-0.055556in}}{\pgfqpoint{0.000000in}{0.000000in}}{%
\pgfpathmoveto{\pgfqpoint{0.000000in}{0.000000in}}%
\pgfpathlineto{\pgfqpoint{0.000000in}{-0.055556in}}%
\pgfusepath{stroke,fill}%
}%
\begin{pgfscope}%
\pgfsys@transformshift{2.301000in}{2.169299in}%
\pgfsys@useobject{currentmarker}{}%
\end{pgfscope}%
\end{pgfscope}%
\begin{pgfscope}%
\pgftext[x=2.301000in,y=0.245736in,,top]{\rmfamily\fontsize{8.000000}{9.600000}\selectfont \(\displaystyle 0.6\)}%
\end{pgfscope}%
\begin{pgfscope}%
\pgfsetbuttcap%
\pgfsetroundjoin%
\definecolor{currentfill}{rgb}{0.000000,0.000000,0.000000}%
\pgfsetfillcolor{currentfill}%
\pgfsetlinewidth{0.501875pt}%
\definecolor{currentstroke}{rgb}{0.000000,0.000000,0.000000}%
\pgfsetstrokecolor{currentstroke}%
\pgfsetdash{}{0pt}%
\pgfsys@defobject{currentmarker}{\pgfqpoint{0.000000in}{0.000000in}}{\pgfqpoint{0.000000in}{0.055556in}}{%
\pgfpathmoveto{\pgfqpoint{0.000000in}{0.000000in}}%
\pgfpathlineto{\pgfqpoint{0.000000in}{0.055556in}}%
\pgfusepath{stroke,fill}%
}%
\begin{pgfscope}%
\pgfsys@transformshift{2.905500in}{0.301292in}%
\pgfsys@useobject{currentmarker}{}%
\end{pgfscope}%
\end{pgfscope}%
\begin{pgfscope}%
\pgfsetbuttcap%
\pgfsetroundjoin%
\definecolor{currentfill}{rgb}{0.000000,0.000000,0.000000}%
\pgfsetfillcolor{currentfill}%
\pgfsetlinewidth{0.501875pt}%
\definecolor{currentstroke}{rgb}{0.000000,0.000000,0.000000}%
\pgfsetstrokecolor{currentstroke}%
\pgfsetdash{}{0pt}%
\pgfsys@defobject{currentmarker}{\pgfqpoint{0.000000in}{-0.055556in}}{\pgfqpoint{0.000000in}{0.000000in}}{%
\pgfpathmoveto{\pgfqpoint{0.000000in}{0.000000in}}%
\pgfpathlineto{\pgfqpoint{0.000000in}{-0.055556in}}%
\pgfusepath{stroke,fill}%
}%
\begin{pgfscope}%
\pgfsys@transformshift{2.905500in}{2.169299in}%
\pgfsys@useobject{currentmarker}{}%
\end{pgfscope}%
\end{pgfscope}%
\begin{pgfscope}%
\pgftext[x=2.905500in,y=0.245736in,,top]{\rmfamily\fontsize{8.000000}{9.600000}\selectfont \(\displaystyle 0.8\)}%
\end{pgfscope}%
\begin{pgfscope}%
\pgfsetbuttcap%
\pgfsetroundjoin%
\definecolor{currentfill}{rgb}{0.000000,0.000000,0.000000}%
\pgfsetfillcolor{currentfill}%
\pgfsetlinewidth{0.501875pt}%
\definecolor{currentstroke}{rgb}{0.000000,0.000000,0.000000}%
\pgfsetstrokecolor{currentstroke}%
\pgfsetdash{}{0pt}%
\pgfsys@defobject{currentmarker}{\pgfqpoint{0.000000in}{0.000000in}}{\pgfqpoint{0.000000in}{0.055556in}}{%
\pgfpathmoveto{\pgfqpoint{0.000000in}{0.000000in}}%
\pgfpathlineto{\pgfqpoint{0.000000in}{0.055556in}}%
\pgfusepath{stroke,fill}%
}%
\begin{pgfscope}%
\pgfsys@transformshift{3.510000in}{0.301292in}%
\pgfsys@useobject{currentmarker}{}%
\end{pgfscope}%
\end{pgfscope}%
\begin{pgfscope}%
\pgfsetbuttcap%
\pgfsetroundjoin%
\definecolor{currentfill}{rgb}{0.000000,0.000000,0.000000}%
\pgfsetfillcolor{currentfill}%
\pgfsetlinewidth{0.501875pt}%
\definecolor{currentstroke}{rgb}{0.000000,0.000000,0.000000}%
\pgfsetstrokecolor{currentstroke}%
\pgfsetdash{}{0pt}%
\pgfsys@defobject{currentmarker}{\pgfqpoint{0.000000in}{-0.055556in}}{\pgfqpoint{0.000000in}{0.000000in}}{%
\pgfpathmoveto{\pgfqpoint{0.000000in}{0.000000in}}%
\pgfpathlineto{\pgfqpoint{0.000000in}{-0.055556in}}%
\pgfusepath{stroke,fill}%
}%
\begin{pgfscope}%
\pgfsys@transformshift{3.510000in}{2.169299in}%
\pgfsys@useobject{currentmarker}{}%
\end{pgfscope}%
\end{pgfscope}%
\begin{pgfscope}%
\pgftext[x=3.510000in,y=0.245736in,,top]{\rmfamily\fontsize{8.000000}{9.600000}\selectfont \(\displaystyle 1.0\)}%
\end{pgfscope}%
\begin{pgfscope}%
\pgftext[x=1.998750in,y=0.078167in,,top]{\rmfamily\fontsize{10.000000}{12.000000}\selectfont \(\displaystyle x\)}%
\end{pgfscope}%
\begin{pgfscope}%
\pgfsetbuttcap%
\pgfsetroundjoin%
\definecolor{currentfill}{rgb}{0.000000,0.000000,0.000000}%
\pgfsetfillcolor{currentfill}%
\pgfsetlinewidth{0.501875pt}%
\definecolor{currentstroke}{rgb}{0.000000,0.000000,0.000000}%
\pgfsetstrokecolor{currentstroke}%
\pgfsetdash{}{0pt}%
\pgfsys@defobject{currentmarker}{\pgfqpoint{0.000000in}{0.000000in}}{\pgfqpoint{0.055556in}{0.000000in}}{%
\pgfpathmoveto{\pgfqpoint{0.000000in}{0.000000in}}%
\pgfpathlineto{\pgfqpoint{0.055556in}{0.000000in}}%
\pgfusepath{stroke,fill}%
}%
\begin{pgfscope}%
\pgfsys@transformshift{0.487500in}{0.301292in}%
\pgfsys@useobject{currentmarker}{}%
\end{pgfscope}%
\end{pgfscope}%
\begin{pgfscope}%
\pgfsetbuttcap%
\pgfsetroundjoin%
\definecolor{currentfill}{rgb}{0.000000,0.000000,0.000000}%
\pgfsetfillcolor{currentfill}%
\pgfsetlinewidth{0.501875pt}%
\definecolor{currentstroke}{rgb}{0.000000,0.000000,0.000000}%
\pgfsetstrokecolor{currentstroke}%
\pgfsetdash{}{0pt}%
\pgfsys@defobject{currentmarker}{\pgfqpoint{-0.055556in}{0.000000in}}{\pgfqpoint{0.000000in}{0.000000in}}{%
\pgfpathmoveto{\pgfqpoint{0.000000in}{0.000000in}}%
\pgfpathlineto{\pgfqpoint{-0.055556in}{0.000000in}}%
\pgfusepath{stroke,fill}%
}%
\begin{pgfscope}%
\pgfsys@transformshift{3.510000in}{0.301292in}%
\pgfsys@useobject{currentmarker}{}%
\end{pgfscope}%
\end{pgfscope}%
\begin{pgfscope}%
\pgftext[x=0.431944in,y=0.301292in,right,]{\rmfamily\fontsize{8.000000}{9.600000}\selectfont \(\displaystyle 0.0\)}%
\end{pgfscope}%
\begin{pgfscope}%
\pgfsetbuttcap%
\pgfsetroundjoin%
\definecolor{currentfill}{rgb}{0.000000,0.000000,0.000000}%
\pgfsetfillcolor{currentfill}%
\pgfsetlinewidth{0.501875pt}%
\definecolor{currentstroke}{rgb}{0.000000,0.000000,0.000000}%
\pgfsetstrokecolor{currentstroke}%
\pgfsetdash{}{0pt}%
\pgfsys@defobject{currentmarker}{\pgfqpoint{0.000000in}{0.000000in}}{\pgfqpoint{0.055556in}{0.000000in}}{%
\pgfpathmoveto{\pgfqpoint{0.000000in}{0.000000in}}%
\pgfpathlineto{\pgfqpoint{0.055556in}{0.000000in}}%
\pgfusepath{stroke,fill}%
}%
\begin{pgfscope}%
\pgfsys@transformshift{0.487500in}{0.674893in}%
\pgfsys@useobject{currentmarker}{}%
\end{pgfscope}%
\end{pgfscope}%
\begin{pgfscope}%
\pgfsetbuttcap%
\pgfsetroundjoin%
\definecolor{currentfill}{rgb}{0.000000,0.000000,0.000000}%
\pgfsetfillcolor{currentfill}%
\pgfsetlinewidth{0.501875pt}%
\definecolor{currentstroke}{rgb}{0.000000,0.000000,0.000000}%
\pgfsetstrokecolor{currentstroke}%
\pgfsetdash{}{0pt}%
\pgfsys@defobject{currentmarker}{\pgfqpoint{-0.055556in}{0.000000in}}{\pgfqpoint{0.000000in}{0.000000in}}{%
\pgfpathmoveto{\pgfqpoint{0.000000in}{0.000000in}}%
\pgfpathlineto{\pgfqpoint{-0.055556in}{0.000000in}}%
\pgfusepath{stroke,fill}%
}%
\begin{pgfscope}%
\pgfsys@transformshift{3.510000in}{0.674893in}%
\pgfsys@useobject{currentmarker}{}%
\end{pgfscope}%
\end{pgfscope}%
\begin{pgfscope}%
\pgftext[x=0.431944in,y=0.674893in,right,]{\rmfamily\fontsize{8.000000}{9.600000}\selectfont \(\displaystyle 0.2\)}%
\end{pgfscope}%
\begin{pgfscope}%
\pgfsetbuttcap%
\pgfsetroundjoin%
\definecolor{currentfill}{rgb}{0.000000,0.000000,0.000000}%
\pgfsetfillcolor{currentfill}%
\pgfsetlinewidth{0.501875pt}%
\definecolor{currentstroke}{rgb}{0.000000,0.000000,0.000000}%
\pgfsetstrokecolor{currentstroke}%
\pgfsetdash{}{0pt}%
\pgfsys@defobject{currentmarker}{\pgfqpoint{0.000000in}{0.000000in}}{\pgfqpoint{0.055556in}{0.000000in}}{%
\pgfpathmoveto{\pgfqpoint{0.000000in}{0.000000in}}%
\pgfpathlineto{\pgfqpoint{0.055556in}{0.000000in}}%
\pgfusepath{stroke,fill}%
}%
\begin{pgfscope}%
\pgfsys@transformshift{0.487500in}{1.048495in}%
\pgfsys@useobject{currentmarker}{}%
\end{pgfscope}%
\end{pgfscope}%
\begin{pgfscope}%
\pgfsetbuttcap%
\pgfsetroundjoin%
\definecolor{currentfill}{rgb}{0.000000,0.000000,0.000000}%
\pgfsetfillcolor{currentfill}%
\pgfsetlinewidth{0.501875pt}%
\definecolor{currentstroke}{rgb}{0.000000,0.000000,0.000000}%
\pgfsetstrokecolor{currentstroke}%
\pgfsetdash{}{0pt}%
\pgfsys@defobject{currentmarker}{\pgfqpoint{-0.055556in}{0.000000in}}{\pgfqpoint{0.000000in}{0.000000in}}{%
\pgfpathmoveto{\pgfqpoint{0.000000in}{0.000000in}}%
\pgfpathlineto{\pgfqpoint{-0.055556in}{0.000000in}}%
\pgfusepath{stroke,fill}%
}%
\begin{pgfscope}%
\pgfsys@transformshift{3.510000in}{1.048495in}%
\pgfsys@useobject{currentmarker}{}%
\end{pgfscope}%
\end{pgfscope}%
\begin{pgfscope}%
\pgftext[x=0.431944in,y=1.048495in,right,]{\rmfamily\fontsize{8.000000}{9.600000}\selectfont \(\displaystyle 0.4\)}%
\end{pgfscope}%
\begin{pgfscope}%
\pgfsetbuttcap%
\pgfsetroundjoin%
\definecolor{currentfill}{rgb}{0.000000,0.000000,0.000000}%
\pgfsetfillcolor{currentfill}%
\pgfsetlinewidth{0.501875pt}%
\definecolor{currentstroke}{rgb}{0.000000,0.000000,0.000000}%
\pgfsetstrokecolor{currentstroke}%
\pgfsetdash{}{0pt}%
\pgfsys@defobject{currentmarker}{\pgfqpoint{0.000000in}{0.000000in}}{\pgfqpoint{0.055556in}{0.000000in}}{%
\pgfpathmoveto{\pgfqpoint{0.000000in}{0.000000in}}%
\pgfpathlineto{\pgfqpoint{0.055556in}{0.000000in}}%
\pgfusepath{stroke,fill}%
}%
\begin{pgfscope}%
\pgfsys@transformshift{0.487500in}{1.422096in}%
\pgfsys@useobject{currentmarker}{}%
\end{pgfscope}%
\end{pgfscope}%
\begin{pgfscope}%
\pgfsetbuttcap%
\pgfsetroundjoin%
\definecolor{currentfill}{rgb}{0.000000,0.000000,0.000000}%
\pgfsetfillcolor{currentfill}%
\pgfsetlinewidth{0.501875pt}%
\definecolor{currentstroke}{rgb}{0.000000,0.000000,0.000000}%
\pgfsetstrokecolor{currentstroke}%
\pgfsetdash{}{0pt}%
\pgfsys@defobject{currentmarker}{\pgfqpoint{-0.055556in}{0.000000in}}{\pgfqpoint{0.000000in}{0.000000in}}{%
\pgfpathmoveto{\pgfqpoint{0.000000in}{0.000000in}}%
\pgfpathlineto{\pgfqpoint{-0.055556in}{0.000000in}}%
\pgfusepath{stroke,fill}%
}%
\begin{pgfscope}%
\pgfsys@transformshift{3.510000in}{1.422096in}%
\pgfsys@useobject{currentmarker}{}%
\end{pgfscope}%
\end{pgfscope}%
\begin{pgfscope}%
\pgftext[x=0.431944in,y=1.422096in,right,]{\rmfamily\fontsize{8.000000}{9.600000}\selectfont \(\displaystyle 0.6\)}%
\end{pgfscope}%
\begin{pgfscope}%
\pgfsetbuttcap%
\pgfsetroundjoin%
\definecolor{currentfill}{rgb}{0.000000,0.000000,0.000000}%
\pgfsetfillcolor{currentfill}%
\pgfsetlinewidth{0.501875pt}%
\definecolor{currentstroke}{rgb}{0.000000,0.000000,0.000000}%
\pgfsetstrokecolor{currentstroke}%
\pgfsetdash{}{0pt}%
\pgfsys@defobject{currentmarker}{\pgfqpoint{0.000000in}{0.000000in}}{\pgfqpoint{0.055556in}{0.000000in}}{%
\pgfpathmoveto{\pgfqpoint{0.000000in}{0.000000in}}%
\pgfpathlineto{\pgfqpoint{0.055556in}{0.000000in}}%
\pgfusepath{stroke,fill}%
}%
\begin{pgfscope}%
\pgfsys@transformshift{0.487500in}{1.795698in}%
\pgfsys@useobject{currentmarker}{}%
\end{pgfscope}%
\end{pgfscope}%
\begin{pgfscope}%
\pgfsetbuttcap%
\pgfsetroundjoin%
\definecolor{currentfill}{rgb}{0.000000,0.000000,0.000000}%
\pgfsetfillcolor{currentfill}%
\pgfsetlinewidth{0.501875pt}%
\definecolor{currentstroke}{rgb}{0.000000,0.000000,0.000000}%
\pgfsetstrokecolor{currentstroke}%
\pgfsetdash{}{0pt}%
\pgfsys@defobject{currentmarker}{\pgfqpoint{-0.055556in}{0.000000in}}{\pgfqpoint{0.000000in}{0.000000in}}{%
\pgfpathmoveto{\pgfqpoint{0.000000in}{0.000000in}}%
\pgfpathlineto{\pgfqpoint{-0.055556in}{0.000000in}}%
\pgfusepath{stroke,fill}%
}%
\begin{pgfscope}%
\pgfsys@transformshift{3.510000in}{1.795698in}%
\pgfsys@useobject{currentmarker}{}%
\end{pgfscope}%
\end{pgfscope}%
\begin{pgfscope}%
\pgftext[x=0.431944in,y=1.795698in,right,]{\rmfamily\fontsize{8.000000}{9.600000}\selectfont \(\displaystyle 0.8\)}%
\end{pgfscope}%
\begin{pgfscope}%
\pgfsetbuttcap%
\pgfsetroundjoin%
\definecolor{currentfill}{rgb}{0.000000,0.000000,0.000000}%
\pgfsetfillcolor{currentfill}%
\pgfsetlinewidth{0.501875pt}%
\definecolor{currentstroke}{rgb}{0.000000,0.000000,0.000000}%
\pgfsetstrokecolor{currentstroke}%
\pgfsetdash{}{0pt}%
\pgfsys@defobject{currentmarker}{\pgfqpoint{0.000000in}{0.000000in}}{\pgfqpoint{0.055556in}{0.000000in}}{%
\pgfpathmoveto{\pgfqpoint{0.000000in}{0.000000in}}%
\pgfpathlineto{\pgfqpoint{0.055556in}{0.000000in}}%
\pgfusepath{stroke,fill}%
}%
\begin{pgfscope}%
\pgfsys@transformshift{0.487500in}{2.169299in}%
\pgfsys@useobject{currentmarker}{}%
\end{pgfscope}%
\end{pgfscope}%
\begin{pgfscope}%
\pgfsetbuttcap%
\pgfsetroundjoin%
\definecolor{currentfill}{rgb}{0.000000,0.000000,0.000000}%
\pgfsetfillcolor{currentfill}%
\pgfsetlinewidth{0.501875pt}%
\definecolor{currentstroke}{rgb}{0.000000,0.000000,0.000000}%
\pgfsetstrokecolor{currentstroke}%
\pgfsetdash{}{0pt}%
\pgfsys@defobject{currentmarker}{\pgfqpoint{-0.055556in}{0.000000in}}{\pgfqpoint{0.000000in}{0.000000in}}{%
\pgfpathmoveto{\pgfqpoint{0.000000in}{0.000000in}}%
\pgfpathlineto{\pgfqpoint{-0.055556in}{0.000000in}}%
\pgfusepath{stroke,fill}%
}%
\begin{pgfscope}%
\pgfsys@transformshift{3.510000in}{2.169299in}%
\pgfsys@useobject{currentmarker}{}%
\end{pgfscope}%
\end{pgfscope}%
\begin{pgfscope}%
\pgftext[x=0.431944in,y=2.169299in,right,]{\rmfamily\fontsize{8.000000}{9.600000}\selectfont \(\displaystyle 1.0\)}%
\end{pgfscope}%
\begin{pgfscope}%
\pgfsetbuttcap%
\pgfsetmiterjoin%
\definecolor{currentfill}{rgb}{1.000000,1.000000,1.000000}%
\pgfsetfillcolor{currentfill}%
\pgfsetlinewidth{1.003750pt}%
\definecolor{currentstroke}{rgb}{0.000000,0.000000,0.000000}%
\pgfsetstrokecolor{currentstroke}%
\pgfsetdash{}{0pt}%
\pgfpathmoveto{\pgfqpoint{0.543056in}{1.758811in}}%
\pgfpathlineto{\pgfqpoint{1.196319in}{1.758811in}}%
\pgfpathlineto{\pgfqpoint{1.196319in}{2.113744in}}%
\pgfpathlineto{\pgfqpoint{0.543056in}{2.113744in}}%
\pgfpathclose%
\pgfusepath{stroke,fill}%
\end{pgfscope}%
\begin{pgfscope}%
\pgfsetbuttcap%
\pgfsetroundjoin%
\pgfsetlinewidth{1.003750pt}%
\definecolor{currentstroke}{rgb}{1.000000,0.000000,0.000000}%
\pgfsetstrokecolor{currentstroke}%
\pgfsetdash{{6.000000pt}{6.000000pt}}{0.000000pt}%
\pgfpathmoveto{\pgfqpoint{0.620833in}{2.030410in}}%
\pgfpathlineto{\pgfqpoint{0.776389in}{2.030410in}}%
\pgfusepath{stroke}%
\end{pgfscope}%
\begin{pgfscope}%
\pgftext[x=0.898611in,y=1.991522in,left,base]{\rmfamily\fontsize{8.000000}{9.600000}\selectfont \(\displaystyle x\)}%
\end{pgfscope}%
\begin{pgfscope}%
\pgfsetrectcap%
\pgfsetroundjoin%
\pgfsetlinewidth{1.003750pt}%
\definecolor{currentstroke}{rgb}{0.000000,0.000000,1.000000}%
\pgfsetstrokecolor{currentstroke}%
\pgfsetdash{}{0pt}%
\pgfpathmoveto{\pgfqpoint{0.620833in}{1.869922in}}%
\pgfpathlineto{\pgfqpoint{0.776389in}{1.869922in}}%
\pgfusepath{stroke}%
\end{pgfscope}%
\begin{pgfscope}%
\pgftext[x=0.898611in,y=1.831033in,left,base]{\rmfamily\fontsize{8.000000}{9.600000}\selectfont \(\displaystyle V(x)\)}%
\end{pgfscope}%
\end{pgfpicture}%
\makeatother%
\endgroup%

		\caption{Visualization of Lemma~\ref{eq:important_fact}'s proof for a Beta-Bernoulli problem with parameters $(3,7)$ and $\gamma=0.7$. The intersection of the two lines represents the Optimistic Gittins Index.}
		\label{fig:visaulize_gx_proof}
	\end{figure}
	
	For the converse, assume $\lambda \ge x$. Recall that $V(z)$ is convex. Since $V$ is continuous on $[0,1]$, by the Intermediate Value Theorem, there exists a point $\lambda$ at which $\lambda =V(\lambda)$. Therefore let $\eps < (1-\lambda)/2$ and from the first direction of the proof, we have $\lambda + \eps > V(\lambda + \eps)$. Thus
	\[
	V(\lambda + \eps) \ge V(\lambda) + \eps V'(\lambda) = \lambda + \eps V'(\lambda)
	\]
	where the inequality follows from $V$ being convex and differentiable. This implies that $V'(\lambda) < 1$ and, moreover, because $V$ is also increasing, it follows that $V'(\lambda) \in (0,1)$, whence
	\begin{align*}
	V(x) & \ge V(\lambda) - (\lambda - x) V'(\lambda) \\
	& = \lambda - (\lambda - x) V'(\lambda) \\
	& = (1-V'(\lambda)) \lambda + V'(\lambda) x \\
	& \ge \min(x,\lambda) = x.
	\end{align*}
	This completes the proof.
\end{myproof}
\begin{corollary} \label{cor:equivalent_event}
	Let $v_{i,t}$ be the Optimistic Gittins Index of arm $i$ at time $t$ and let $x \in (0,1)$. The following equivalence holds
	\[
	\{v^K_{i,t} < x \} = \{V_K(y_{i,t}; x, \gamma_t) < x\}\]
	where $y_{i,t}$ is the sufficient statistic for estimating the $i$th arm's parameter $\theta_i$.
\end{corollary}
\begin{myproof}[Proof.]
	By the definition in Equation~\eqref{eqn:ogi_k1}, $v^K_{i,t}$ can be characterized with the relation 
	\begin{equation*}% \label{eq:form4}
	v_{i,t} = \sup\left\{ x \in [0,1] : x \le V_K(y_{i,t}; x, \gamma_t)  \right\}.
	\end{equation*}
	The conclusion then follows from Lemma~\ref{eq:important_fact}.
\end{myproof}
The next Lemma will be the final property for the function $V_K$ that we prove. This will subsequently be used in the proof of Lemma~\ref{lemma:underestimation}.
\begin{lemma} \label{lemma:vk_bound}
	Let $i$ be any arm. For any state $y$, look-ahead parameter $K \in \mathbb{Z}_+$ and $\gamma, \eta \in [0,1]$, we have
	\begin{equation*}
		\Ee{V_K(y'; \gamma, \eta) \given y} \ge V_K(y; \gamma, \eta)
	\end{equation*}
	where $y'$ is the next state (posterior) for that arm.
\end{lemma}
\begin{myproof}[Proof.]
	Let $\tau^\star(y)$ be the optimal stopping time for an initial state $y$ in the problem for computing $V_K$. Then
	\begin{align}
		\Ee{V_K(y'; \gamma, \eta) \given y} & = \Ee{\Ee{\sum_{s=1}^{\tau^\star(y')-1} \gamma^{s-1} X_{i,s} + \frac{\gamma^{\tau^\star(y')-1} R_{\eta, K}(\tau, y_{i,\tau^\star(y')})}{1-\gamma} \given[\Bigg] y_{i,1} = y'} \given[\Bigg] y} \nonumber \\
		& \ge  \Ee{\Ee{\sum_{s=1}^{\tau^\star(y)-1} \gamma^{s-1} X_{i,s} + \frac{\gamma^{\tau^\star(y)-1} R_{\eta, K}(\tau, y_{i,\tau^\star(y)})}{1-\gamma} \given[\Bigg] y_{i,1} = y'} \given[\Bigg] y} \label{ineq:subopt_of_y}\\
		& = \Ee{\sum_{s=1}^{\tau^\star(y)-1} \gamma^{s-1} X_{i,s} + \frac{\gamma^{\tau^\star(y)-1} R_{\eta, K}(\tau, y_{i,\tau^\star(y)})}{1-\gamma} \given[\Bigg] y_{i,1} = y} \label{eq:tower_prop} \\
		& = V_K(y; \gamma, \eta). \nonumber
	\end{align}
	The bound \eqref{ineq:subopt_of_y} is due to the sub-optimality of $\tau^\star(y)$ when the actual starting state is $y'$ and equation \eqref{eq:tower_prop} is the tower property.
\end{myproof}
\subsection{Proof of Lemma~\ref{lemma:underestimation}} \label{proof:underestimation_proof}
\begin{myproof}[Proof.]
	This proof involves induction. We state what the base case is, in the following Lemma, but we defer its proof to the end of the section.
	\begin{lemma} \label{lemma:underestimation_base_case}
		Suppose the OGI algorithm uses only one look-ahead step, and so $K = 1$. Then we have
		\[
		\P{v^1_{1,t} < \eta} = \mathcal{O}\left(\frac{1}{t^{1 + h(\eta)}}\right)
		\]
		where $h$ is a positive yet decreasing function of $\eta$.
	\end{lemma}
	Now we show the induction step. Suppose that for $K \ge 2$, it holds that
	\[
	\P{v^{K-1}_{1,t} < \eta} = \mathcal{O}\left(\frac{1}{t^{1 + h(\eta)}}\right).
	\]
	We show the same is true for $v^{K}_{1,t}$. Indeed, we have for $t > 1$
	\begin{align}
	\P{v^K_{1,t} < \eta} & = \P{V_K(y_{1,t}; \eta, \gamma_t) < \eta} \label{eq:appl_of_lemma_9}\\
	& = \P{(1-\gamma_t)\Ee{X_{1,t} \given y_{1,t}} + \gamma_t \Ee{\max(\eta, V_{K-1}(y_{1,t+1}; \eta, \gamma_{t})) \given y_{1,t}}< \eta} \nonumber \\
	& \le \P{\frac{1}{t}\Ee{X_{1,t} \given y_{1,t}} + \gamma_t \max(\eta, \Ee{V_{K-1}(y_{1,t+1}; \eta, \gamma_{t}) \given y_{1,t}})< \eta} \label{ineq:appl_jensens} \\
	& \le \P{\frac{1}{t}\Ee{X_{1,t} \given y_{1,t}} + \left(1-\frac{1}{t}\right) \Ee{V_{K-1}(y_{1,t+1}; \eta, \gamma_{t}) \given y_{1,t} }< \eta} \nonumber \\
	& \le \P{\frac{1}{t}\Ee{X_{1,t} \given y_{1,t}} + \left(1-\frac{1}{t}\right) V_{K-1}(y_{1,t}; \eta, \gamma_{t})< \eta} \label{ineq:missing_step} \\
	& \le \P{ V_{K-1}(y_{1,t}; \eta, \gamma_{t})< \eta \left(\frac{t}{t-1}\right)} \nonumber \\
	& \le \mathcal{O}\left(\frac{1}{t^{1 + h(\eta t/(t-1))}}\right) =  \mathcal{O}\left(\frac{1}{t^{1 + h(\eta)}}\right) \nonumber
	\end{align}
	where \eqref{eq:appl_of_lemma_9} follows from Lemma~\ref{cor:equivalent_event} and \eqref{ineq:appl_jensens} uses Jensen's inequality. Lemma~\ref{lemma:vk_bound} give us \eqref{ineq:missing_step}.
	
	We finish the proof by using the following asymptotic argument. Take $M$ to be a large enough integer, then we have, using the result of the induction proof,
	\begin{align*}
	\sum_{t=1}^\infty \P{v^K_{1,t} < \eta} & \le M +  \sum_{t=M+1}^\infty \frac{C_1}{t^{1 + h(\eta)}} \le M +  C_2
	\end{align*}
	where $C_2 = C_2(\eta)$ is the limit of the series and $C_1$ is a constant used in the definition of the big-Oh.
\end{myproof}
\subsubsection{Proof of the base case in Lemma~\ref{lemma:underestimation_base_case}}
\begin{myproof}[Proof.]
	For simplicity let's abbreviate $v^1_{1,t}$ as $v_{1,t}$. Define $\delta := (\theta_1 - \eta)/2$ and  $\eta' :=  \eta + \delta$. In other words, $\delta$ is half the distance between $\eta$ and $\theta_1$; $\eta'$ is the point half-way.
	
	The proof consists of showing two claims
	\subsubsection*{Claim 1: $\{v_{1,t} < \eta\} \subseteq \left\{F^B_{N_1(t)+1, \eta'}(S_1(t)) < \frac{1}{\delta t}\right\}$:}
	Let $V_t \sim $Beta$(S_1(t)+1,N_1(t) - S_1(t) + 1)$ be the agent's posterior on the optimal arm. Using Corollary~\ref{cor:equivalent_event} and the simplified form for $K=1$ \[V_K((S_1(t)+1, N_1(t) - S_1(t) + 1); \eta, \gamma_t) = \E{V_t} + \gamma_t\Ee{(\eta - V_t)^+}\] we find that
	\begin{align}
	\left\{v_{1,t} < \eta \right\} & = \left\{ \Ee{V_t } + \gamma_t\Ee{(\eta - V_t)^+} < \eta \right\} \nonumber\\
	& =  \left\{ (1-1/t)\Ee{(\eta - V_t)^+} < \Ee{\eta - V_t} \right\} \label{eq:def_gamma_t} \\
	& =  \left\{ \Ee{(\eta - V_t)^+} - \Ee{\eta - V_t} <  \frac{1}{t}\Ee{(\eta - V_t)^+}\right\} \nonumber\\
	& =  \left\{ \Ee{(V_t - \eta)^+}<  \frac{1}{t}\Ee{(\eta - V_t)^+}\right\} \nonumber\\
	& \subseteq \left\{ \Ee{ (V_t - \eta)^+}< \frac{1}{t}  \right\}  \label{eq:intermediate_event}
	\end{align}
	where \eqref{eq:def_gamma_t} follows from the definition of $\gamma_t$ and \eqref{eq:intermediate_event} is due to $V_t, \eta$ both lying in the interval $[0,1]$. We approximate the conditional expectation in \eqref{eq:intermediate_event} with
	\begin{align}
	\Ee{(V_t - \eta)^+ \given S_1(t), N_1(t)} & = \Ee{(V_t - \eta) \ind{V_t \ge \eta} }\nonumber \\
	& = \Ee{(V_t - \eta) \ind{\eta + \delta > V_t \ge \eta} }  \nonumber \\
	& \qquad + \Ee{(V_t - \eta) \ind{ V_t \ge \eta + \delta} } \nonumber \\
	& > \Ee{(V_t - \eta) \ind{ V_t \ge \eta + \delta} } \nonumber \\
	& \ge \delta\P{ V_t \ge \eta' } \nonumber \\
	& = \delta (1 - F_{S_1(t)+1,N_1(t)-S_1(t)+1}(\eta')) = \delta F^B_{N_1(t)+1,\eta'}(S_1(t)) \label{ineq:lower_bound_on_ppart_term}
	\end{align}
	The last equality is due to Fact~\ref{fact:equation_for_beta_binomial_cdfs} and this proves the claim.
	\subsubsection*{Claim 2: $ \sum_{t=1}^\infty \mathbb{P}\left(F^B_{N_1(t)+1, \eta'}(S_1(t)) < \frac{1}{\delta t}\right) \le C_1$ where $C_1$ is a constant:}
	Let us fix the sequence $f_t = -\frac{\log \delta t }{\log (1-\eta')}-1 = O(\log t)$. We then have
	\begin{align}
	\P{F^B_{N_1(t)+1, \eta'}(S_1(t)) < \frac{1}{\delta t}} & = \P{F^B_{N_1(t)+1, \eta'}(S_1(t)) < \frac{1}{\delta t}, \; N_1(t) > f_t}  \nonumber \\
	& \qquad + \P{F^B_{N_1(t)+1, \eta'}(S_1(t)) < \frac{1}{\delta t}, \; N_1(t) \le f_t} \label{eq:decomp2}.
	\end{align}
	For the second term in the RHS of \eqref{eq:decomp2} we have the following bound,
	\begin{align}
	\P{F^B_{N_1(t)+1, \eta'}(S_1(t)) < \frac{1}{\delta t}, \; N_1(t) \le f_t}  &  \le \P{F^B_{N_1(t)+1,\eta'}(0) < \frac{1}{\delta  t}, \; N_1(t) \le f_t} \nonumber \\
	& = \P{(1-\eta')^{N_1(t)+1} <  \frac{1}{\delta  t}, \; N_1(t) \le f_t} \nonumber \\
	& \le \P{(1-\eta')^{f_t+1} <  \frac{1}{\delta  t}} = 0.\label{bound:bdd_by_zero}
	\end{align}
	Now we use the following fact to bound the left term on the RHS of \eqref{eq:decomp2}. Define the function
	\[
	F^{-B}_{n,p}(u) := \inf\{x : F^B_{n,p}(x) \ge u\}
	\]
	which is the inverse CDF. Then it is known that if $U \sim \text{Unif}(0,1)$, then $F^{-B}_{n,p}(U) \sim \text{Binomial}(n,p)$. Furthermore, $F^B_{n,p}(F^{-B}_{n,p}(U)) \ge U$ due to the definition of the inverse CDF.
	
	Now let us only consider large $t$, in particular $t > M = M(\theta_1, \eta')$ where:
	\begin{enumerate}
		\item $M$ is such that $e^{d(\eta', \theta_1)f_{M}/2} > (f_M + 1)^4$
		\item $M > \frac{4}{(1-\eta')\delta }$
		\item $\ceil{f_M} > 0$ and $F^B_{\ceil{f_M},\eta'}(f_M \eta') > 1/4$. Note that there is a large enough integer for this because $F^B_{\ceil{f_t},\eta'}(f_t \eta') \to \frac{1}{2}$ as $t \to \infty$.
	\end{enumerate} 
	Suppose that $t > M$. It then follows that the event $\{F^B_{N_1(t), \eta'}(S_1(t)) < \frac{1}{(1-\eta')\delta t},\; S_1(t) \ge N_1(t) \eta', \; N_1(t) > f_t\}$ has measure zero because of the assumptions made on $M$. Therefore if $t > M$, we have
	\begin{align}
	\mathbb{P}\bigg(F^B_{N_1(t)+1,  \eta'}(S_1(t)) &< \frac{1}{\delta t }  , \; N_1(t) > f_t \bigg) \nonumber \\
	& \le \P{F^B_{N_1(t),  \eta'}(S_1(t)) < \frac{1}{(1-\eta')\delta t}, \; N_1(t) > f_t} \label{eqn:part1_decomp_the_cdf_of_y} \\ 
	& = \P{F^B_{N_1(t),  \eta'}(S_1(t)) < \frac{1}{(1-\eta')\delta t}, \; S_1(t) < N_1(t) \eta', \; N_1(t) > f_t} \nonumber \\ 
	& =  \P{F^B_{N_1(t),\theta_1}(S_1(t)) \frac{F^B_{N_1(t),\eta'}(S_1(t))}{F^B_{N_1(t),\theta_1}(S_1(t))} < \frac{1}{(1-\eta')\delta  t}, \;S_1(t) < N_1(t) \eta', \; N_1(t) > f_t} \nonumber \\
	& \le  \P{F_{N_1(t),\theta_1}^B(S_1(t))  e^{N_1(t) D} < \frac{1}{(1-\eta')\delta  t} , \; N_1(t) > f_t} \label{eqn:part1_app_of_lemma2} \\
	& \le  \P{F_{N_1(t),\theta_1}^B(S_1(t)) e^{f_t D} < \frac{1}{(1-\eta')\delta  t}} \nonumber \\
	& =  \P{F_{N_1(t),\theta_1}^B(F^{-B}_{N_1(t),\theta_1}(U)) < \frac{e^{-f_t D}}{(1-\eta')\delta  t} } \label{eqn:part1_propert_of_inverse_sampling}\\
	& \le  \P{U < \frac{e^{-f_t D}}{(1-\eta')\delta  t} } \nonumber \\  
	& =  \frac{e^{-f_t D}}{(1-\eta')\delta  t} \nonumber  \nonumber\\
	& = O\left( \frac{1}{t^{1+Dc_{\eta'}}} \right)  \label{bound:one_over_t_plus_eps} 
	\end{align}
	where $D = d(\eta',\theta_1) > 0$ and $c_{\eta'} = -\log^{-1}(1-\eta') > 0$ are constant. The bound \eqref{eqn:part1_decomp_the_cdf_of_y} holds due to Fact~\eqref{fact:relationship_with_binom_cdfs}. Bound \eqref{eqn:part1_app_of_lemma2} follows from an application of Lemma~\ref{lemma:ratio_of_cdfs} and the fact that $t > M$. Equation \eqref{eqn:part1_propert_of_inverse_sampling} follows from $S_1(t) \sim \text{Binomial}(N_1(t), \theta_1)$ and the inverse sampling technique. By combining bounds \eqref{bound:one_over_t_plus_eps}, \eqref{bound:bdd_by_zero} and \eqref{eq:decomp2}, we get the big-Oh bound.
\end{myproof}

\subsection{Proof of Lemma~\ref{lemma:overestimation}} \label{proof:overestimation_proof}

\begin{proof}[Proof.]
	See the main proof of Theorem~\ref{thm:frequentist_optimal_bound} to recall the definition of constants $\eta_1$, $\eta_3$ and their relationship with $\theta_2$ and $\theta_1$. As an abbreviation we let $L = L(T)$. Moreover, because for any arm $i$ $v^K_{i,t} \le v^{K-1}_{i,t} \le \ldots \le v^1_{i,t}$, it will be sufficient to consider this proof only for $v^1_{2,t}$, which we also will abbreviate as $v_{2,t} \defeq v^1_{2,t}$.
	
	Firstly, by the law of total probability, we find that
	\begin{align} 
	\sum_{t=1}^T \mathbb{P}(v_{2,t} & \ge \eta_3 ,\; N_2(t) \ge L,\; \pi^{\rm OG}_t = 2) \nonumber \\
	& = \sum_{t=1}^T \P{v_{2,t} \ge \eta_3 ,\; N_2(t) \ge L, \; S_2(t) < \floor{N_2(t) \eta_1}, \; \pi^{\rm OG}_t = 2} \nonumber \\
	& \qquad + \sum_{t=1}^T \P{v_{2,t} \ge \eta_3 ,\; N_2(t) \ge L, \; S_2(t) \ge \floor{N_2(t) \eta_1},\; \pi^{\rm OG}_t = 2} \nonumber \\
	& \le \sum_{t=1}^T \P{v_{2,t} \ge \eta_3 ,\; N_2(t) \ge L, \; S_2(t) < \floor{N_2(t) \eta_1}} + \sum_{t=1}^T \P{\pi^{\rm OG}_t = 2,\; S_2(t) \ge \floor{N_2(t) \eta_1}} \label{eqn:splitting_not_underestimate}
	\end{align}
	Let $V_t \sim \text{Beta}(S_2(t) + 1, N_2(t)- S_2(t) + 1)$ denote the agent's posterior on the second arm at time $t$, then
	\begin{align}
	\sum_{t=1}^T \mathbb{P}(v_{2,t} \ge \eta_3 ,\; & \; N_2(t) \ge L,\; S_2(t) < \floor{N_2(t) \eta_1})  \nonumber\\
	& = \sum_{t=1}^T \P{\Ee{V_t} + \gamma_t \Ee{(\eta_3 - V_t)^+} \ge \eta_3, \; N_2(t) \ge L,\; S_2(t) < \floor{N_2(t) \eta_1}} \nonumber \\
	& = \sum_{t=1}^T \P{\frac{\Ee{(\eta_3-V_t)^+ }}{  \Ee{(V_t - \eta_3)^+ }} \le t , \; N_2(t) \ge L,\; S_2(t) < \floor{N_2(t) \eta_1}} \label{eq:complicated_rv_in_part2}
	\end{align}
	where the second equality follows from Corollary~\ref{cor:equivalent_event} in Appendix~\ref{sec:amgi_results}. The following result lets us bound \eqref{eq:complicated_rv_in_part2},
	\begin{lemma} \label{lem:lb_rv2}
		Let $0 < x < y < 1$. For any non-negative integers $s$ and $k$ with $s < \floor{kx}$, it holds that
		\begin{equation*}
		\frac{\Ee{(y-V)^+ }}{  \Ee{(V - y)^+ } } \ge \frac{(y-x) \exp(k d(x,y))}{2}
		\end{equation*}
		where $V \sim \text{Beta}(s+1,k-s+1)$.
	\end{lemma}
	\begin{myproof}[Proof.]
		See Appendix~\ref{prf:proof_of_lb_rv2}.
	\end{myproof}
	Therefore, from equation \eqref{eq:complicated_rv_in_part2} and Lemma~\ref{lem:lb_rv2}, we find that whenever $T > \left(\frac{2}{\eta_3-\eta_1}\right)^{1/\eps} =: T^*(\eps, \theta)$,
	\begin{align}
	\sum_{t=1}^T \mathbb{P}(v_{2,t} \ge \eta_3 ,\; & \; N_2(t) \ge L,\; S_2(t) < \floor{N_2(t) \eta_1}) \nonumber \\
	& \le  \sum_{t=1}^T\P{  (\eta_3-\eta_1) \exp\{N_2(t) d(\eta_1,\eta_3) \} \le 2t,\; N_2(t) \ge L} \nonumber \\
	& \le  \sum_{t=1}^T\P{  (\eta_3-\eta_1) \exp\{L d(\eta_1,\eta_3) \} \le 2t} \nonumber \\
	& =   \sum_{t=1}^T\P{  (\eta_3-\eta_1) T^{1+\eps} \le 2t} = 0 \label{bound:equal_to_zero}
	\end{align}
	All that is left is to bound the second term in \eqref{eqn:splitting_not_underestimate}, and to do so we apply the following Lemma whose proof is in Appendix~\ref{prf:proof_of_acc_sub_means}
	\begin{lemma} \label{lem:accurate_suboptimal_mean}
		There exist positive constants $C = C(\theta_2,\eta_1)$ and $x' > \theta_2$ such that
		\begin{equation*}
		\sum_{t=1}^T \P{S_2(t) \ge \floor{N_2(t) \eta_1}, \; \pi^{\rm OG}_t = 2} \le  K + \frac{1}{1 - e^{-d(x',\theta_2)}} 
		\end{equation*}
	\end{lemma}
	Combining Lemma~\ref{lem:accurate_suboptimal_mean}, \eqref{bound:equal_to_zero}, \eqref{eqn:splitting_not_underestimate} and \eqref{eq:complicated_rv_in_part2} shows the claim.
\end{proof}

\subsubsection{Proof of Lemma~\ref{lem:lb_rv2}.} \label{prf:proof_of_lb_rv2}
\begin{myproof}[Proof.]
	We upper bound the denominator as follows. Given that $s < \floor{k x}$, we have $s \le kx - 1$. Let $B(a,b)$ denote the Beta function, then
	\begin{align}
	\Ee{(V - y)^+ } & = \frac{1}{B(s+1,k-s+1)}\int_{y}^1 (t-y) t^s (1-t)^{k-s} \; dt \nonumber \\
	& = \frac{1}{B(s+1,k-s+1)}\int_{y}^1 t^{s+1} (1-t)^{k-s}  dt - y \P{V \ge y} \nonumber \\
	& = \frac{B(s+2,k-s+1)}{B(s+1,j-s+1)}\left( \frac{1}{B(s+2,k-s+1)} \right)\int_{y}^1 t^{s+1} (1-t)^{k-s}  dt - y \P{V \ge y} \nonumber \\
	& = \frac{s+1}{k+2} F^B_{k+2,y}(s+1)  - y \P{V \ge y} \label{eq:part2_use_of_equiv_between_beta_and_binom} \\
	& \le \frac{s+1}{k+2} F^B_{k+2,y}(s+1) \le  F^B_{k,y}(k x) \le \exp\left\{- k d(x,y) \label{ineq:chernoff_app} \right\}
	\end{align}
	where we use Fact~\ref{fact:equation_for_beta_binomial_cdfs} and the definition of the Beta CDF to establish equation \eqref{eq:part2_use_of_equiv_between_beta_and_binom}. The final bound in \eqref{ineq:chernoff_app} is the result of the Chernoff-Hoeffding theorem and Fact~\ref{fact:relationship_with_binom_cdfs}. Let $\delta:=y-x$, and note that $s < kx \Longrightarrow s \le \floor{(k+1)x}$ due to $s$ being integer, whence
	\begin{align}
	\Ee{(y - V)^+ } & =  \Ee{(y - V) \ind{V \le y} \given s, k} \nonumber \\
	& = \Ee{(y - V) \ind{y - \delta \le V \le y} \given s, k} +  \Ee{(y - V) \ind{V < y - \delta} \given s, k} \nonumber\\
	& > \Ee{(y - V) \ind{V < y - \delta} \given s, k}\nonumber \\
	& \ge \delta\Ee{\ind{V < y-\delta} \given s, k}\\
	& = \delta \P{V < x \given s} \nonumber \\
	& = \delta\left(1 - F^B_{k+1,x}(s) \right) \label{eq:use_of_bin_beta_identity}  \\
	& \ge \delta/2  \label{eq:use_of_median_prop}
	\end{align}
	where equation \eqref{eq:use_of_bin_beta_identity} relies on Fact~\ref{fact:equation_for_beta_binomial_cdfs}. The bound \eqref{eq:use_of_median_prop} is justified from Fact~\ref{fact:median_of_binomial_dist} and $s \le \floor{(k+1) x}$. Thus using the inequalities for both the numerator and denominator, we obtain the desired bound.
\end{myproof}
\subsubsection{Proof of Lemma~\ref{lem:accurate_suboptimal_mean}.} \label{prf:proof_of_acc_sub_means}
\begin{proof}[Proof.]
	The steps in this proof follow a similar one in \cite{agrawal2013further} but we show them for completeness. We bound the number of times the sub-optimal arm's mean is overestimated. Let $\tau_\ell$ be the time step in which the  sub-optimal arm is sampled for the $\ell$\textsuperscript{th} time. Because for any $x$, $\lim_{n\to\infty}\frac{\floor{nx}}{nx} = 1$, we can let $N$ be a large enough integer so that if $\ell \ge N$, then $\eta_1 \frac{\floor{\ell \eta_1}}{\ell \eta_1} > x' := (\theta_2 + \eta_1)/2 > \theta_2$. In that case,
	\begin{align}
	\sum_{t=1}^T\P{S_2(t) \ge \floor{N_2(t) \eta_1}, \; \pi^{\rm OG}_t = 2} & \le \Ee{\sum_{\ell=1}^T \sum_{t=\tau_\ell}^{\tau_{\ell+1}-1}\ind{S_2(\ell) \ge \floor{N_1(\ell) \eta_1}} \ind{\pi^{\rm OG}_t = 2}} \nonumber \\
	& = \Ee{\sum_{\ell=1}^T \ind{S_2(\tau_{\ell}) \ge \floor{(\ell-1) \eta_1}} \sum_{t=\tau_\ell}^{\tau_{\ell+1}-1} \ind{\pi^{\rm OG}_t = 2}} \nonumber\\
	& = \Ee{\sum_{\ell=0}^{T-1} \ind{S_2(\tau_{\ell+1}) \ge \floor{\ell \eta_1}}} \nonumber\\
	& \le  N + \sum_{\ell=N+1}^{T-1} \P{ S_2(\tau_{\ell+1}) \ge \ell \eta_1 \frac{\floor{\ell \eta_1}}{\ell \eta_1}} \nonumber \\
	& \le N + \sum_{\ell=N+1}^{T-1} \P{ S_2(\tau_{\ell+1}) \ge \ell x'} \nonumber \\
	& \le  N + \sum_{\ell=1}^{\infty} \exp(-\ell d(x', \theta_2)) \label{bound:cf_thm} \\
	& = N + \frac{1}{1 - e^{-d(x',\theta_2)}} \nonumber
	\end{align}
\end{proof}
The bound \eqref{bound:cf_thm} follows from the Chernoff-Hoeffding theorem and that $S_2(\tau_{\ell+1}) \sim \text{Binomial}(N_1(\ell+1), \theta_2) \sim \text{Binomial}(\ell, \theta_2)$.
\appendix

\section{Proof of Proposition~\ref{prop:gittins_log3T}} \label{proof:prop_log3T}
\begin{myproof}[Proof.]
	First, letting $\gamma_n = 1 - 1/n$, we show that
	\begin{equation} \label{eq:basic_finite_time_gittins}
	\Regret{\pi^{G, \gamma_n}, n} = O\left( \log^2(n) \right).
	\end{equation}
	Let $H \sim \text{Geo}(1/n)$ be an exogenous geometric random variable that is independent of $\theta$ and not observed by the agent. As an abbreviation, define $\mu^* = \E_{q}[\mu^*(\theta)]$. We then have
	\begin{align}
	\sum_{t=1}^\infty \gamma^{t-1} \E\left[X_{\pi^{G,\gamma_n},t}\right] & = \E\left[\sum_{t=1}^H X_{\pi^{G,\gamma_n}_t,t}\right] \nonumber \\
	& =  \E\left[H \mu^*(\theta) -\Regret{\pi^{G,\gamma_n}, H}\right] \\
	& =  n \mu^* - \E\left[\Regret{\pi^{G,\gamma_n}, H}\right] \nonumber \\
	& \le n \mu^* - \E\left[\Regret{\pi^{G,\gamma_n}, H} \given H > n\right]\P{H > n} \nonumber \\
	& \le n \mu^* - \E\left[\Regret{\pi^{G,\gamma_n}, n}\right](1-1/n)^n \nonumber \\
	& = n \mu^* - \E\left[\Regret{\pi^{G,\gamma_n}, n}\right](e^{-1} + o(1)) \label{bound:geometric_regret1}.
	\end{align}
	Let $q, Q$ be the density and CDF, respectively, of the prior distribution. Now, by Theorem 3, part 1 of \cite{lai1987adaptive}, there exists (an efficient) policy $\tilde \pi$, such that as $n$ becomes sufficiently large
	\begin{equation*}
	\Regret{\tilde \pi, n} \sim \left( A(A-1)  \int_\Theta q^2(\theta) Q^{A-2}(\theta) \; d\theta \right) \log^2 n.
	\end{equation*}
	Therefore for some prior-dependent constant $C_q$, we have $\Regret{\tilde \pi, n} \le C_q \log^2 n$. Let $\Delta(\theta)$ denote worst case  single period regret under parameter $\theta$, that is, $\Delta(\theta) =  \max_{i} \mu(\theta^*) - \mu(\theta_i)$. Let $\Delta$ denote its expectation over $\theta$, from which we obtain the lower bound,
	\begin{align}
	\E{\sum_{t=1}^H X_{\pi^{G,\gamma_n}_t,t}} & \ge \E\left[\sum_{t=1}^H X_{ \tilde \pi_t,t}\right] \label{bound:gittins_optimality}\\
	&  = \E\left[H\mu^*(\theta) - \Regret{\tilde \pi, H}\right] \nonumber \\
	& \ge \E\left[H\mu^*(\theta) -\Regret{\tilde \pi, H}\ind{H \ge e} - 2\ind{H < e}\Delta(\theta)\right] \nonumber \\
	& \ge n \mu^* - C_q \Ee{(\log (H))^2 \ind{H \ge 3} } - 2\Delta \nonumber \\
	& \ge n \mu^* - C_q \Ee{(\log (H))^2 \given H \ge 3}\P{H \ge 3} - 2\Delta \nonumber \\
	& \ge n \mu^* - C_q \log^2 (n+3)\P{H \ge 3} - 2\Delta\label{bound:jensens_ineq_for_log3t_proof}
	\end{align}
	where \eqref{bound:gittins_optimality} holds by optimality of the Gittins Index. The bound \eqref{bound:jensens_ineq_for_log3t_proof} follows from the memoryless property of the Geometric distribution, from Jensen's inequality and the fact that function $\log^2 x$ is a concave function on $[e,+\infty)$. Thus, equation \eqref{eq:basic_finite_time_gittins} is implied by the bounds \eqref{bound:geometric_regret1} and \eqref{bound:jensens_ineq_for_log3t_proof}.
	
	Now, for any policy $\pi$, we define $\tilde L_\pi(m) := m\mu^* - \sum_{t=1}^m X_{\pi_t,t}$ to be the random $m$ period shortfall against the expected Bayes' optimal arm and let $g_k = 1 - 1/2^{k-1}$. We break up the time horizon $T$ into geometrically growing epochs and bound, conservatively, the Bayes' risk in each one:
	\begin{align}
	\Regret{\pi^D, T} & \le \Regret{\pi^D, 2^{\ceil{\log_2 T}}}\\
	& = \sum_{k=1}^{\ceil{\log_2 T}} \Ee{\Ee{ \tilde L_{\pi^D}(2^{k-1}) \given[\Big] \mathcal F_{2^{k-1}-1}}} \nonumber \\
	& =\sum_{k=1}^{\ceil{\log_2 T}} \Ee{\Ee{ \tilde L_{\pi^{G, g_k}}(2^{k-1}) \given[\Big] \mathcal F_{2^{k-1}-1}}} \nonumber \\
	& \le \sum_{k=1}^{\ceil{\log_2 T}} \Ee{\Ee{ \tilde L_{\pi^{G,g_k}}(2^{k-1}) \given[\Big] \mathcal F_{0}}} \label{bound:super_mg} \\
	& = \sum_{k=1}^{\ceil{\log_2 T}} \Regret{\pi^{G, g_k}, 2^{k-1}} =  O\left(\sum_{k=1}^{\ceil{\log_2 T}} k^2 \right)\label{eq:order_optimal_bayes} \\
	& = O(\log^3 T) \nonumber
	\end{align}
	where \eqref{eq:order_optimal_bayes} follows from equation \eqref{eq:basic_finite_time_gittins} and \eqref{bound:super_mg} holds because regret increases if history is discarded.
\end{myproof}

\subsection{Proof of Lemma~\ref{lemma:approx_bound}} \label{prf:approx_bound}
\begin{myproof}[Proof.]
	We begin with a fundamental step. Let $\tau_1$ and $\tau_2$ be any stopping times such that $\tau_1$ precedes $\tau_2$ almost surely, that is $\tau_1 \le \tau_2$. Recall that the expected reward of the $i$th arm satisfies $\Ee{X_{i,t} \given \theta_i} = \mu(\theta_i)$ for all $t$. Let $\hat \theta_i \in \Theta$ denote a realization of the random variable $\theta_i$ and let us suppose first that $\lambda > \mu(\hat \theta_i)$. In this case, we have that
	\begin{align*}
	\Ee{\sum_{t=\tau_1}^{\tau_2-1} \gamma^{t-1} X_{i,t} + \gamma^{\tau_2-1}\frac{\lambda}{1 - \gamma} \given[\Bigg] \theta_i = \hat \theta_i} & =\mu(\hat \theta_i) \Ee{ \sum_{t=\tau_1}^{\tau_2-1} \gamma^{t-1}  \given[\Bigg] \theta_i = \hat \theta_i} + \Ee{\frac{\gamma^{\tau_2-1}}{1-\gamma}\given[\Bigg] \theta_i = \hat \theta_i}\lambda \\
	& =\mu(\hat \theta_i) \Ee{ \sum_{t=\tau_1}^{\tau_2-1} \gamma^{t-1} \given[\Bigg] \theta_i = \hat \theta_i } + \Ee{\sum_{t=\tau_2}^{\infty} \gamma^{t-1} \given[\Bigg] \theta_i = \hat \theta_i}\frac{\lambda}{1 - \gamma} \\
	& \le \Ee{\gamma^{\tau_1-1} \given \theta_i = \hat \theta_i} \frac{\lambda}{1-\gamma} \\
	& = \Ee{\gamma^{\tau_1-1} \given \theta_i = \hat \theta_i}  \frac{\max(\lambda, \mu(\hat \theta_i))}{1-\gamma}
	\end{align*}
	where the inequality is tight for $\tau_2 = \tau_1$. The same bound, for the other case when $\lambda \le \mu(\hat \theta_i)$, can be shown in a similar way. Thus we conclude, because $\hat \theta_i$ was arbitrary, that almost surely,
	\begin{equation} \label{ineq:fundamntal_bound_for_lemma_2}
	\Ee{\sum_{t=\tau_1}^{\tau_2-1} \gamma^{t-1} X_{i,t} + \gamma^{\tau_2-1}\frac{\lambda}{1 - \gamma} \given[\Bigg] \theta_i}  \le \Ee{\gamma^{\tau_1-1} \given \theta_i }  \frac{\max(\lambda, \mu(\theta_i))}{1-\gamma}.
	\end{equation}
	Let $\tau^\star$ be a stopping time that achieves the supremum in the RHS of \eqref{eqn:gittins_index} and define the stopping time $\tau^\star_K \defeq (K+1) \wedge \tau^\star$. Consider the (conditional) cumulative rewards in the Gittins problem, from time $\tau^\star_K$ onwards, given the sufficient statistic observed at time $\tau_K^\star - 1$, that is, $\E\left[\sum_{t=\tau_K^\star}^{\tau^\star}  \gamma^{t-1} X_{i,t} + \gamma^{\tau^\star-1} \lambda/(1-\gamma)
	\given[\Bigg] y_{i,\tau_K^\star-1} \right]$. We upper bound this random variable as follows. Firstly, we note that, at any time $s$ and for any statistic $\hat y \in \mathcal{Y}$, the following statement holds
	\begin{equation}\label{eqn:dist_equal_theta_i}
	\P{R(\hat y_{i,s}) \le r} = \P{\mu(\theta_i) \le r \given y_{i,s} = \hat y}, \qquad \forall r \in \Re
	\end{equation}
	meaning that the posterior distribution of the arm's expected reward $R(y_{i,s})$ is the same as $\mu(\theta_i)$ \emph{conditioned} on having observed statistic $\hat y$ about the arm. This is true by definition, but we emphasize the point here.
	
	Now, let us condition on the event $\tau^\star \ge (K+1)$. In that case we have $\tau^\star_K = K+1$ and therefore
	\begin{align} 
	&\E\left[\sum_{t=\tau_K^\star}^{\tau^\star-1} \gamma^{t-1} X_{i,t} + \gamma^{\tau^\star-1} \frac{\lambda}{1-\gamma}
	\given[\Bigg] \tau^\star \ge (K+1), \;y_{i,\tau^\star_K-1} \right] \nonumber \\
	&\qquad = \E\left[ \Ee{\sum_{t=(K+1)}^{\tau^\star-1} \gamma^{t-1} X_{i,t} + \gamma^{\tau^\star-1}\frac{\lambda}{1 - \gamma} \given[\Bigg] \theta_i} 
	\given[\Bigg]\tau^\star \ge (K+1), \; y_{i,\tau^\star_K-1} \right] \label{eqn:proof_lemma2_toer_prop} \\
	&\qquad \le \E\left[ \gamma^{\tau^\star_K - 1} \frac{\max(\mu(\theta_i), \lambda)}{1 - \gamma}\given[\Bigg] \tau^\star \ge (K+1), \; y_{i,\tau^\star_K-1}  \right]  \label{ineq:proof_lem_2_use_of_first_bound} \\
	&\qquad = \E\left[ \gamma^{\tau^\star_K - 1} \frac{\max(R(y_{i,\tau^\star_K-1}), \lambda)}{1 - \gamma}\given[\Bigg] \tau^\star \ge (K+1), \; y_{i,\tau^\star_K-1}  \right]  \label{ineq:obvious_step} \\
	&\qquad = \Ee{\frac{\gamma^{\tau^\star_K-1}R_{\lambda,K}(\tau^\star_K, y_{i,\tau^\star_K-1})}{1-\gamma} \given[\Bigg]  \tau^\star \ge (K+1), \; y_{i,\tau^\star_K-1}} \label{eqn:use_of_def_of_R}
	\end{align}
	where \eqref{eqn:proof_lemma2_toer_prop} is the tower property and \eqref{ineq:proof_lem_2_use_of_first_bound} follows from the bound in \eqref{ineq:fundamntal_bound_for_lemma_2} because $\tau^\star_K \le \tau^\star$, almost surely. Equation \eqref{ineq:obvious_step} follows from statement \eqref{eqn:dist_equal_theta_i} and that the event $\tau^\star \ge K + 1$ is $\mathcal{F}_K$-measurable (we can decide whether to pull arm $i$ or retire based on prior information). Finally equation \eqref{eqn:use_of_def_of_R} is derived by substituting in the definition of $R_{\lambda,K}$ (as given in Section~\ref{sec:gittins_and_approx}) and noting that $\tau^\star_K = K+1$.
	
	We now condition on the complement of the previous event we considered, namely, $\tau^\star < (K+1)$. Under that event, $\tau^\star$ occured early enough before time $K+1$ and thus $\tau^\star_K = \tau^\star$. Therefore, it follows that
	\begin{align} 
	&\E\left[\sum_{t=\tau_K^\star}^{\tau^\star-1}\gamma^{t-1} X_{i,t} + \gamma^{\tau^\star-1} \frac{\lambda}{1-\gamma}
	\given[\Bigg] \tau^\star < (K+1), \;y_{i,\tau^\star_K-1} \right] \nonumber  \\
	&\qquad = \E\left[\gamma^{\tau^\star-1}  \frac{\lambda}{1-\gamma}
	\given[\Bigg]\tau^\star < (K+1), \; y_{i,\tau^\star_K-1} \right] \nonumber \\
	&\qquad = \E\left[\gamma^{\tau^\star_K-1}  \frac{R_{\lambda,K}(\tau^\star_K, y_{i,\tau^\star_K-1})}{1-\gamma}
	\given[\Bigg]\tau^\star < (K+1), \; y_{i,\tau^\star_K-1} \right] \label{eqn:proof_lem_2_second_use_of_R_def}
	\end{align}
	where \eqref{eqn:proof_lem_2_second_use_of_R_def} is obtained by using the definition of $R_{\lambda,K}$ and the fact that we conditioned on $\tau^\star < (K+1)$. Thus, by the law of total expectation and \eqref{eqn:use_of_def_of_R}, \eqref{eqn:proof_lem_2_second_use_of_R_def}, we establish that
	\begin{equation} \label{ineq:proof_lem_2_main_bound_in_proof}
	\E\left[\sum_{t=\tau_K^\star}^{\tau^\star-1}\gamma^{t-1} X_{i,t} + \gamma^{\tau^\star-1} \frac{\lambda}{1-\gamma}
	\given[\Bigg] \;y_{i,\tau^\star_K-1} \right] \le \Ee{\gamma^{\tau^\star_K-1}  \frac{R_{\lambda,K}(\tau^\star_K, y_{i,\tau^\star_K-1})}{1-\gamma} \given[\Bigg] y_{i,\tau^\star_K-1}}.
	\end{equation}
	We are ready to complete our main argument in this proof by using the above bound and `breaking up' the RHS of \eqref{eqn:gittins_index} into rewards from times before $\tau_K^\star$ and after (and bounding the latter terms). More precisely, we obtain that
	\begin{align}
	\E_{y}\left[\sum_{t=1}^{\tau^\star-1}\gamma^{t-1} X_{i,t} + \gamma^{\tau^\star-1}\frac{\lambda}{1-\gamma}\right] & = \E_{y}\left[\sum_{t=1}^{\tau^\star_K-1}\gamma^{t-1} X_{i,t} + \sum_{t'=\tau^\star_K}^{\tau^\star-1}\gamma^{t'-1} X_{i,t'} +  \gamma^{\tau^\star-1}\frac{\lambda}{1-\gamma}\right] \nonumber \\
	& = \E_{y}\left[\sum_{t=1}^{\tau^\star_K-1}\gamma^{t-1} X_{i,t} + \Ee{\sum_{t'=\tau^\star_K}^{\tau^\star-1}\gamma^{t'-1} X_{i,t'} +  \gamma^{\tau^\star-1}\frac{\lambda}{1-\gamma} \given[\Bigg] y_{i,\tau^\star_K-1} }\right] \label{eqn:proof_lem_2_tower_prop_again} \\
	& \le  \E_{y}\left[\sum_{t=1}^{\tau^\star_K-1}\gamma^{t-1} X_{i,t} + \Ee{\gamma^{\tau^\star_K-1}  \frac{R_{\lambda,K}(\tau^\star_K, y_{i,\tau^\star_K-1})}{1-\gamma} \given[\Bigg] y_{i,\tau^\star_K-1}}\right] \label{eqn:proof_lem2_using_the_main_argument} \\
	& =  \E_{y}\left[\sum_{t=1}^{\tau^\star_K-1}\gamma^{t-1} X_{i,t} +  \gamma^{\tau^\star_K-1}  \frac{R_{\lambda,K}(\tau^\star_K, y_{i,\tau^\star_K-1})}{1-\gamma}\right] \label{eqn:proof_lem_2_tower_prop_yet_again} \\
	& \le \sup_{1 < \tau \le (K+1)}  \E_{y}\left[\sum_{t=1}^{\tau-1}\gamma^{t-1} X_{i,t} +  \gamma^{\tau-1}  \frac{R_{\lambda,K}(\tau, y_{i,\tau-1})}{1-\gamma}\right] \nonumber
	\end{align}
	where Equations \eqref{eqn:proof_lem_2_tower_prop_again}, \eqref{eqn:proof_lem_2_tower_prop_yet_again} use the tower property and \eqref{eqn:proof_lem2_using_the_main_argument} is gotten immediately by using the bound of \eqref{ineq:proof_lem_2_main_bound_in_proof}. 
	
	Let us now use the shorthand $V^K_{\gamma}( y, \lambda) \defeq \sup_{1 < \tau \le (K+1)} \E_{y}\left[\sum_{t=1}^{\tau-1}\gamma^{t-1} X_{i,t} + \gamma^{\tau-1} \frac{R_{\lambda,K}(\tau, y_{\tau-1})}{1-\gamma}\right]$ and $V_{\gamma}(y, \lambda) \defeq \sup_{\tau > 1} \E_{y}\left[\sum_{t=1}^{\tau-1}\gamma^{t-1} X_{i,t} +  \gamma^{\tau-1} \frac{\lambda}{1-\gamma}\right]$. We have just shown that $V^K_{\gamma}( y, \lambda) \ge V_{\gamma}(y, \lambda)$ for all $K \ge 1$. 
	
	To finish the proof let us assume that $v^K_\gamma(y) < v_\gamma(y)$ in order to get a contradiction. As shown in \cite{gittins2011multi}, Chapter 2, the function $V_\gamma(y, \lambda) - \lambda/(1-\gamma)$ is convex and decreasing in $\lambda$, therefore it has a single root, which we know is $v_\gamma(y)$. Because $v^K_\gamma < v_\gamma(y)$, by our current assumption, we have that 
	\begin{equation}
		V_\gamma(y, v^K_\gamma(y)) > v^K_\gamma(y)/(1-\gamma) = V^K_\gamma(y, v^K_\gamma(y)),
	\end{equation}
	which contradicts what we just showed, that $V^K_{\gamma}( y, \lambda) \ge V_{\gamma}(y, \lambda)$ for all $\lambda$.
\end{myproof}
\subsection{Results for the frequentist regret bound proof}
\subsubsection{Definitions and properties of Binomial distributions.}
We list notation and facts related to Beta and Binomial distributions, which are used through this section.
\begin{definition}
	$F^B_{n,p}(.)$ is the CDF of the Binomial distribution with parameters $n$ and $p$, and $F^\beta_{a,b}(.)$ is the CDF of the Beta distribution with parameters $a$ and $b$.
\end{definition}

\begin{lemma} \label{fact:equation_for_beta_binomial_cdfs}
	Let $a$ and $b$ be positive integers and $y \in [0,1]$, 
	\[
	F^\beta_{a,b}(y) = 1 - F^B_{a+b-1,y}(a-1)
	\]
\end{lemma}
\begin{myproof}[Proof.]
	Proof is found in \cite{agrawalanalysis}.
\end{myproof}
\begin{lemma} \label{fact:median_of_binomial_dist}
	The median of a Binomial$(n,p)$ distribution is either $\ceil{np}$ or $\floor{np}$.
\end{lemma}
\begin{myproof}[Proof]
	Proof is found in \cite{jogdeo1968monotone}.
\end{myproof}

\begin{corollary}[Corollary of Fact~\ref{fact:median_of_binomial_dist}] \label{cor:corollarly_of_binomial_median_property}
	Let $n$ be a positive integer and $p \in (0,1)$. For any non-negative integer $s < np$
	\[
	F_{n,p}(s) \le 1/2
	\]
\end{corollary}

\begin{lemma} \label{fact:relationship_with_binom_cdfs}
	Let $n$ be a positive integer and $p \in [0,1]$. Then for any $k \in \{0,\ldots,n\}$,
	\[
	(1-p)F_{n-1,p}(k)\le F_{n,p}(k) \le F^B_{n-1,p}(k)
	\] 
\end{lemma}
\begin{myproof}[Proof]
	To prove $F_{n,p}(k) \le F^B_{n-1,p}(k)$, we let $X_1,\ldots,X_{n}$ be i.i.d samples from a Bernoulli($p$) distribution. We then have
	\begin{align*}
	F^B_{n,p}(k)  = \P{\sum_{i=1}^{n} X_i \le k}  \le  \P{\sum_{i=1}^{n-1} X_i \le k}  = F^B_{n-1,p}(k)
	\end{align*}
	Now to prove $(1-p)F_{n-1,p}(k)\le F_{n,p}(k)$, it's enough to observe that $F_{n,p}(k) = p F_{n-1,p}(k-1) + (1-p) F_{n-1,p}(k)$.
\end{myproof}

\subsubsection{Ratio of Binomial CDFs.} \label{sec:ratio_of_bin_cdfs}
\begin{lemma} \label{lemma:ratio_of_cdfs}
	Let $0< q < p < 1$. Let $n$ be a positive integer such that $e^{\frac{n}{2} d(q,p)} \ge (n+1)^4$ and let $k$ be a non-negative integer such that $k < nq$. It then follows that
	\[
	F^B_{n,q}(k)/F^B_{n,p}(k) >  e^{\frac{n}{2} d(q,p)}.
	\]
\end{lemma}
\begin{proof}[Proof.]
	From the method of types  (see \cite{cover2012elements}), we have for any $r \in (0,1)$ and $j < nr$
	\begin{equation} \label{eqn:appl_of_sanovs}
	\frac{e^{-nd(j/n, r)}}{(1+n)^2}\le F_{n,r}(j) \le (n+1)^2 e^{- n d(j/n, r)}.
	\end{equation}
	Because $k < nq < np$, by applying \eqref{eqn:appl_of_sanovs} to both the numerator and denominator, we get
	\begin{align*}
	\frac{F_{n,q}(k)}{F_{n,p}(k)} & \ge  \frac{e^{-nd(k/n, q)}}{(n+1)^4 e^{- n d(k/n, p)}} = \frac{e^{n(d(k/n,p) - d(k/n,q))}}{(n+1)^4}.
	\end{align*}
	Examining the exponent, we find
	\begin{align*}
	d(k/n, p) - d(k/n,q) & = \frac{k}{n} \log \frac{q}{p} + \left(1-\frac{k}{n}\right)\log \frac{1-q}{1-p} \\
	& > q \log \frac{q}{p} + (1-q)\log \frac{1-q}{1-p} \\
	& = d(q,p)
	\end{align*}
	where the bound holds because the expression is decreasing in $k$, and $k < nq$. Therefore,
	\begin{align}
	\frac{F_{n,q}(k)}{F_{n,p}(k)} & > \frac{e^{n  d(q,p)}}{(n+1)^4} = \frac{e^{\frac{n}{2}d(q,p)}}{(n+1)^4} e^{\frac{n}{2}d(q,p)} \ge e^{\frac{n}{2}d(q,p)} \label{bound:log_1minusq_etc}.
	\end{align}
	The final lower bound in \eqref{bound:log_1minusq_etc} follows from the assumption on $n$.
\end{proof}

\section{Optimistic Gittins Index results.} \label{sec:amgi_results}
For this section it is useful to define the value of a stopping problem used in the calculation of the Optimistic Gittins Index. For any fixed arm $i$, we write
\begin{align*}
V_K(y; x, \gamma) & := \sup_{1 < \tau \le K} \Ee{\sum_{t=1}^{\tau-1} (1-\gamma)X_{i,t} + \gamma^{\tau-1} R_{x, K}(\tau, y_{i,\tau})	\given y_{i,1} = y}.
\end{align*}
Therefore the Optimistic Gittins Index, in state $y$ with discount factor $\gamma$, is the solution in $x$ to $V_K(y; x, \gamma) = x$. We show some key properties of $V_K$, which we exploit later on. For fixed $y$, $K$ and $\gamma$, we will call $V(x) \defeq V(y; x, \gamma)$.
\begin{fact}
	$V(x)$ is convex and differentiable.
\end{fact}
\begin{myproof}[Proof.]
	We prove this by induction. For $K = 1$, we have $\tau = 2$ almost surely and so
	\begin{align*}
	V(x) & = V_1(y; x, \gamma) \\
	& = (1-\gamma_t)\Ee{X_{i,1} \given y_{i,1 = y}} + \gamma_t \Ee{R_{x, 1}(2, y_{i,1}) \given y_{i,1} = y} \\
	& = (1-\gamma_t)\Ee{X_{i,1} \given y_{i,1 = y}} + \gamma_t \Ee{\max(x, \E{X_{i,1}}) \given y_{i,1} = y}
	\end{align*}
	The function is convex because for any random variable $Z$ the term $\max(g, Z)$ is convex and taking expectations preserves convexity. Also, we verify through the bounded convergence theorem that $V(x)$ is differentiable and the event $\{\E X_{i,1}=z \given y_{i,1} = y\}$, at which $V$ is not differentiable, has measure zero.
	
	For $K > 1$, assume that $V_{K-1}$ is convex and differentiable. By writing the Bellman equation
	\begin{align*}
		V_K(y; x, \gamma) & = (1-\gamma_t)\Ee{X_{i,1} \given y_{i,1}=y} + \Ee{\gamma_t \max(x, V_{K-1}(y_{i,t+1}; x, \gamma)) \given y_{i,1} = y}
	\end{align*}
	we again notice a maximum of convex functions in $x$. This form for $V_K$ implies that it is convex and differentiable.
\end{myproof}
\begin{lemma} \label{eq:important_fact}
	Let $\gamma \in (0,1)$ and
	\begin{equation} \label{eq:def_lambda_in_important_equiv}
	\lambda = \sup\{ x \in [0,1] : V(x) \ge x\}
	\end{equation} 
	For all $x \in (0,1)$, the following equivalence holds
	\begin{equation} \label{eq:equivlance_lambda1}
	\lambda < x  \Longleftrightarrow V(x) < x.
	\end{equation}
\end{lemma}
\begin{myproof}[Proof.]
	Figure~\ref{fig:visaulize_gx_proof} gives a visualization of the proof. In the plot, the point $x^*$, where $V(x^*)$ and $x^*$ intersect, is the Optimistic Gittins Index. Notice how if $x > x^*$, $V(x)$ is below $x$ and otherwise $V(x)$ is above $x$; this is the crux of the proof.
	
	We also give a formal proof. Firstly let's assume that $\lambda < x$. If $x\le V(x)$, then $\lambda$ would not be the supremum over all $z \in [0,1]$ such that $ z \le V(z)$. Therefore $V(x) < x$.
	
	\begin{figure}
		\centering
		%% Creator: Matplotlib, PGF backend
%%
%% To include the figure in your LaTeX document, write
%%   \input{<filename>.pgf}
%%
%% Make sure the required packages are loaded in your preamble
%%   \usepackage{pgf}
%%
%% Figures using additional raster images can only be included by \input if
%% they are in the same directory as the main LaTeX file. For loading figures
%% from other directories you can use the `import` package
%%   \usepackage{import}
%% and then include the figures with
%%   \import{<path to file>}{<filename>.pgf}
%%
%% Matplotlib used the following preamble
%%   \usepackage[utf8x]{inputenc}
%%   \usepackage[T1]{fontenc}
%%
\begingroup%
\makeatletter%
\begin{pgfpicture}%
\pgfpathrectangle{\pgfpointorigin}{\pgfqpoint{3.900000in}{2.410333in}}%
\pgfusepath{use as bounding box, clip}%
\begin{pgfscope}%
\pgfsetbuttcap%
\pgfsetmiterjoin%
\definecolor{currentfill}{rgb}{1.000000,1.000000,1.000000}%
\pgfsetfillcolor{currentfill}%
\pgfsetlinewidth{0.000000pt}%
\definecolor{currentstroke}{rgb}{1.000000,1.000000,1.000000}%
\pgfsetstrokecolor{currentstroke}%
\pgfsetdash{}{0pt}%
\pgfpathmoveto{\pgfqpoint{0.000000in}{0.000000in}}%
\pgfpathlineto{\pgfqpoint{3.900000in}{0.000000in}}%
\pgfpathlineto{\pgfqpoint{3.900000in}{2.410333in}}%
\pgfpathlineto{\pgfqpoint{0.000000in}{2.410333in}}%
\pgfpathclose%
\pgfusepath{fill}%
\end{pgfscope}%
\begin{pgfscope}%
\pgfsetbuttcap%
\pgfsetmiterjoin%
\definecolor{currentfill}{rgb}{1.000000,1.000000,1.000000}%
\pgfsetfillcolor{currentfill}%
\pgfsetlinewidth{0.000000pt}%
\definecolor{currentstroke}{rgb}{0.000000,0.000000,0.000000}%
\pgfsetstrokecolor{currentstroke}%
\pgfsetstrokeopacity{0.000000}%
\pgfsetdash{}{0pt}%
\pgfpathmoveto{\pgfqpoint{0.487500in}{0.301292in}}%
\pgfpathlineto{\pgfqpoint{3.510000in}{0.301292in}}%
\pgfpathlineto{\pgfqpoint{3.510000in}{2.169299in}}%
\pgfpathlineto{\pgfqpoint{0.487500in}{2.169299in}}%
\pgfpathclose%
\pgfusepath{fill}%
\end{pgfscope}%
\begin{pgfscope}%
\pgfpathrectangle{\pgfqpoint{0.487500in}{0.301292in}}{\pgfqpoint{3.022500in}{1.868008in}} %
\pgfusepath{clip}%
\pgfsetbuttcap%
\pgfsetroundjoin%
\pgfsetlinewidth{1.003750pt}%
\definecolor{currentstroke}{rgb}{1.000000,0.000000,0.000000}%
\pgfsetstrokecolor{currentstroke}%
\pgfsetdash{{6.000000pt}{6.000000pt}}{0.000000pt}%
\pgfpathmoveto{\pgfqpoint{0.487500in}{0.301292in}}%
\pgfpathlineto{\pgfqpoint{0.591724in}{0.365706in}}%
\pgfpathlineto{\pgfqpoint{0.695948in}{0.430120in}}%
\pgfpathlineto{\pgfqpoint{0.800172in}{0.494534in}}%
\pgfpathlineto{\pgfqpoint{0.904397in}{0.558948in}}%
\pgfpathlineto{\pgfqpoint{1.008621in}{0.623362in}}%
\pgfpathlineto{\pgfqpoint{1.112845in}{0.687776in}}%
\pgfpathlineto{\pgfqpoint{1.217069in}{0.752190in}}%
\pgfpathlineto{\pgfqpoint{1.321293in}{0.816604in}}%
\pgfpathlineto{\pgfqpoint{1.425517in}{0.881018in}}%
\pgfpathlineto{\pgfqpoint{1.529741in}{0.945432in}}%
\pgfpathlineto{\pgfqpoint{1.633966in}{1.009846in}}%
\pgfpathlineto{\pgfqpoint{1.738190in}{1.074260in}}%
\pgfpathlineto{\pgfqpoint{1.842414in}{1.138674in}}%
\pgfpathlineto{\pgfqpoint{1.946638in}{1.203088in}}%
\pgfpathlineto{\pgfqpoint{2.050862in}{1.267502in}}%
\pgfpathlineto{\pgfqpoint{2.155086in}{1.331917in}}%
\pgfpathlineto{\pgfqpoint{2.259310in}{1.396331in}}%
\pgfpathlineto{\pgfqpoint{2.363534in}{1.460745in}}%
\pgfpathlineto{\pgfqpoint{2.467759in}{1.525159in}}%
\pgfpathlineto{\pgfqpoint{2.571983in}{1.589573in}}%
\pgfpathlineto{\pgfqpoint{2.676207in}{1.653987in}}%
\pgfpathlineto{\pgfqpoint{2.780431in}{1.718401in}}%
\pgfpathlineto{\pgfqpoint{2.884655in}{1.782815in}}%
\pgfpathlineto{\pgfqpoint{2.988879in}{1.847229in}}%
\pgfpathlineto{\pgfqpoint{3.093103in}{1.911643in}}%
\pgfpathlineto{\pgfqpoint{3.197328in}{1.976057in}}%
\pgfpathlineto{\pgfqpoint{3.301552in}{2.040471in}}%
\pgfpathlineto{\pgfqpoint{3.405776in}{2.104885in}}%
\pgfpathlineto{\pgfqpoint{3.510000in}{2.169299in}}%
\pgfusepath{stroke}%
\end{pgfscope}%
\begin{pgfscope}%
\pgfpathrectangle{\pgfqpoint{0.487500in}{0.301292in}}{\pgfqpoint{3.022500in}{1.868008in}} %
\pgfusepath{clip}%
\pgfsetrectcap%
\pgfsetroundjoin%
\pgfsetlinewidth{1.003750pt}%
\definecolor{currentstroke}{rgb}{0.000000,0.000000,1.000000}%
\pgfsetstrokecolor{currentstroke}%
\pgfsetdash{}{0pt}%
\pgfpathmoveto{\pgfqpoint{0.487500in}{0.835008in}}%
\pgfpathlineto{\pgfqpoint{0.591724in}{0.835131in}}%
\pgfpathlineto{\pgfqpoint{0.695948in}{0.835923in}}%
\pgfpathlineto{\pgfqpoint{0.800172in}{0.837886in}}%
\pgfpathlineto{\pgfqpoint{0.904397in}{0.841366in}}%
\pgfpathlineto{\pgfqpoint{1.008621in}{0.846579in}}%
\pgfpathlineto{\pgfqpoint{1.112845in}{0.853633in}}%
\pgfpathlineto{\pgfqpoint{1.217069in}{0.862552in}}%
\pgfpathlineto{\pgfqpoint{1.321293in}{0.873295in}}%
\pgfpathlineto{\pgfqpoint{1.425517in}{0.898119in}}%
\pgfpathlineto{\pgfqpoint{1.529741in}{0.932953in}}%
\pgfpathlineto{\pgfqpoint{1.633966in}{0.968286in}}%
\pgfpathlineto{\pgfqpoint{1.738190in}{1.004555in}}%
\pgfpathlineto{\pgfqpoint{1.842414in}{1.047574in}}%
\pgfpathlineto{\pgfqpoint{1.946638in}{1.092664in}}%
\pgfpathlineto{\pgfqpoint{2.050862in}{1.137754in}}%
\pgfpathlineto{\pgfqpoint{2.155086in}{1.182844in}}%
\pgfpathlineto{\pgfqpoint{2.259310in}{1.227934in}}%
\pgfpathlineto{\pgfqpoint{2.363534in}{1.273024in}}%
\pgfpathlineto{\pgfqpoint{2.467759in}{1.318114in}}%
\pgfpathlineto{\pgfqpoint{2.571983in}{1.363203in}}%
\pgfpathlineto{\pgfqpoint{2.676207in}{1.408293in}}%
\pgfpathlineto{\pgfqpoint{2.780431in}{1.453383in}}%
\pgfpathlineto{\pgfqpoint{2.884655in}{1.498473in}}%
\pgfpathlineto{\pgfqpoint{2.988879in}{1.543563in}}%
\pgfpathlineto{\pgfqpoint{3.093103in}{1.588653in}}%
\pgfpathlineto{\pgfqpoint{3.197328in}{1.633742in}}%
\pgfpathlineto{\pgfqpoint{3.301552in}{1.678832in}}%
\pgfpathlineto{\pgfqpoint{3.405776in}{1.723922in}}%
\pgfpathlineto{\pgfqpoint{3.510000in}{1.769012in}}%
\pgfusepath{stroke}%
\end{pgfscope}%
\begin{pgfscope}%
\pgfsetrectcap%
\pgfsetmiterjoin%
\pgfsetlinewidth{1.003750pt}%
\definecolor{currentstroke}{rgb}{0.000000,0.000000,0.000000}%
\pgfsetstrokecolor{currentstroke}%
\pgfsetdash{}{0pt}%
\pgfpathmoveto{\pgfqpoint{0.487500in}{2.169299in}}%
\pgfpathlineto{\pgfqpoint{3.510000in}{2.169299in}}%
\pgfusepath{stroke}%
\end{pgfscope}%
\begin{pgfscope}%
\pgfsetrectcap%
\pgfsetmiterjoin%
\pgfsetlinewidth{1.003750pt}%
\definecolor{currentstroke}{rgb}{0.000000,0.000000,0.000000}%
\pgfsetstrokecolor{currentstroke}%
\pgfsetdash{}{0pt}%
\pgfpathmoveto{\pgfqpoint{3.510000in}{0.301292in}}%
\pgfpathlineto{\pgfqpoint{3.510000in}{2.169299in}}%
\pgfusepath{stroke}%
\end{pgfscope}%
\begin{pgfscope}%
\pgfsetrectcap%
\pgfsetmiterjoin%
\pgfsetlinewidth{1.003750pt}%
\definecolor{currentstroke}{rgb}{0.000000,0.000000,0.000000}%
\pgfsetstrokecolor{currentstroke}%
\pgfsetdash{}{0pt}%
\pgfpathmoveto{\pgfqpoint{0.487500in}{0.301292in}}%
\pgfpathlineto{\pgfqpoint{3.510000in}{0.301292in}}%
\pgfusepath{stroke}%
\end{pgfscope}%
\begin{pgfscope}%
\pgfsetrectcap%
\pgfsetmiterjoin%
\pgfsetlinewidth{1.003750pt}%
\definecolor{currentstroke}{rgb}{0.000000,0.000000,0.000000}%
\pgfsetstrokecolor{currentstroke}%
\pgfsetdash{}{0pt}%
\pgfpathmoveto{\pgfqpoint{0.487500in}{0.301292in}}%
\pgfpathlineto{\pgfqpoint{0.487500in}{2.169299in}}%
\pgfusepath{stroke}%
\end{pgfscope}%
\begin{pgfscope}%
\pgfsetbuttcap%
\pgfsetroundjoin%
\definecolor{currentfill}{rgb}{0.000000,0.000000,0.000000}%
\pgfsetfillcolor{currentfill}%
\pgfsetlinewidth{0.501875pt}%
\definecolor{currentstroke}{rgb}{0.000000,0.000000,0.000000}%
\pgfsetstrokecolor{currentstroke}%
\pgfsetdash{}{0pt}%
\pgfsys@defobject{currentmarker}{\pgfqpoint{0.000000in}{0.000000in}}{\pgfqpoint{0.000000in}{0.055556in}}{%
\pgfpathmoveto{\pgfqpoint{0.000000in}{0.000000in}}%
\pgfpathlineto{\pgfqpoint{0.000000in}{0.055556in}}%
\pgfusepath{stroke,fill}%
}%
\begin{pgfscope}%
\pgfsys@transformshift{0.487500in}{0.301292in}%
\pgfsys@useobject{currentmarker}{}%
\end{pgfscope}%
\end{pgfscope}%
\begin{pgfscope}%
\pgfsetbuttcap%
\pgfsetroundjoin%
\definecolor{currentfill}{rgb}{0.000000,0.000000,0.000000}%
\pgfsetfillcolor{currentfill}%
\pgfsetlinewidth{0.501875pt}%
\definecolor{currentstroke}{rgb}{0.000000,0.000000,0.000000}%
\pgfsetstrokecolor{currentstroke}%
\pgfsetdash{}{0pt}%
\pgfsys@defobject{currentmarker}{\pgfqpoint{0.000000in}{-0.055556in}}{\pgfqpoint{0.000000in}{0.000000in}}{%
\pgfpathmoveto{\pgfqpoint{0.000000in}{0.000000in}}%
\pgfpathlineto{\pgfqpoint{0.000000in}{-0.055556in}}%
\pgfusepath{stroke,fill}%
}%
\begin{pgfscope}%
\pgfsys@transformshift{0.487500in}{2.169299in}%
\pgfsys@useobject{currentmarker}{}%
\end{pgfscope}%
\end{pgfscope}%
\begin{pgfscope}%
\pgftext[x=0.487500in,y=0.245736in,,top]{\rmfamily\fontsize{8.000000}{9.600000}\selectfont \(\displaystyle 0.0\)}%
\end{pgfscope}%
\begin{pgfscope}%
\pgfsetbuttcap%
\pgfsetroundjoin%
\definecolor{currentfill}{rgb}{0.000000,0.000000,0.000000}%
\pgfsetfillcolor{currentfill}%
\pgfsetlinewidth{0.501875pt}%
\definecolor{currentstroke}{rgb}{0.000000,0.000000,0.000000}%
\pgfsetstrokecolor{currentstroke}%
\pgfsetdash{}{0pt}%
\pgfsys@defobject{currentmarker}{\pgfqpoint{0.000000in}{0.000000in}}{\pgfqpoint{0.000000in}{0.055556in}}{%
\pgfpathmoveto{\pgfqpoint{0.000000in}{0.000000in}}%
\pgfpathlineto{\pgfqpoint{0.000000in}{0.055556in}}%
\pgfusepath{stroke,fill}%
}%
\begin{pgfscope}%
\pgfsys@transformshift{1.092000in}{0.301292in}%
\pgfsys@useobject{currentmarker}{}%
\end{pgfscope}%
\end{pgfscope}%
\begin{pgfscope}%
\pgfsetbuttcap%
\pgfsetroundjoin%
\definecolor{currentfill}{rgb}{0.000000,0.000000,0.000000}%
\pgfsetfillcolor{currentfill}%
\pgfsetlinewidth{0.501875pt}%
\definecolor{currentstroke}{rgb}{0.000000,0.000000,0.000000}%
\pgfsetstrokecolor{currentstroke}%
\pgfsetdash{}{0pt}%
\pgfsys@defobject{currentmarker}{\pgfqpoint{0.000000in}{-0.055556in}}{\pgfqpoint{0.000000in}{0.000000in}}{%
\pgfpathmoveto{\pgfqpoint{0.000000in}{0.000000in}}%
\pgfpathlineto{\pgfqpoint{0.000000in}{-0.055556in}}%
\pgfusepath{stroke,fill}%
}%
\begin{pgfscope}%
\pgfsys@transformshift{1.092000in}{2.169299in}%
\pgfsys@useobject{currentmarker}{}%
\end{pgfscope}%
\end{pgfscope}%
\begin{pgfscope}%
\pgftext[x=1.092000in,y=0.245736in,,top]{\rmfamily\fontsize{8.000000}{9.600000}\selectfont \(\displaystyle 0.2\)}%
\end{pgfscope}%
\begin{pgfscope}%
\pgfsetbuttcap%
\pgfsetroundjoin%
\definecolor{currentfill}{rgb}{0.000000,0.000000,0.000000}%
\pgfsetfillcolor{currentfill}%
\pgfsetlinewidth{0.501875pt}%
\definecolor{currentstroke}{rgb}{0.000000,0.000000,0.000000}%
\pgfsetstrokecolor{currentstroke}%
\pgfsetdash{}{0pt}%
\pgfsys@defobject{currentmarker}{\pgfqpoint{0.000000in}{0.000000in}}{\pgfqpoint{0.000000in}{0.055556in}}{%
\pgfpathmoveto{\pgfqpoint{0.000000in}{0.000000in}}%
\pgfpathlineto{\pgfqpoint{0.000000in}{0.055556in}}%
\pgfusepath{stroke,fill}%
}%
\begin{pgfscope}%
\pgfsys@transformshift{1.696500in}{0.301292in}%
\pgfsys@useobject{currentmarker}{}%
\end{pgfscope}%
\end{pgfscope}%
\begin{pgfscope}%
\pgfsetbuttcap%
\pgfsetroundjoin%
\definecolor{currentfill}{rgb}{0.000000,0.000000,0.000000}%
\pgfsetfillcolor{currentfill}%
\pgfsetlinewidth{0.501875pt}%
\definecolor{currentstroke}{rgb}{0.000000,0.000000,0.000000}%
\pgfsetstrokecolor{currentstroke}%
\pgfsetdash{}{0pt}%
\pgfsys@defobject{currentmarker}{\pgfqpoint{0.000000in}{-0.055556in}}{\pgfqpoint{0.000000in}{0.000000in}}{%
\pgfpathmoveto{\pgfqpoint{0.000000in}{0.000000in}}%
\pgfpathlineto{\pgfqpoint{0.000000in}{-0.055556in}}%
\pgfusepath{stroke,fill}%
}%
\begin{pgfscope}%
\pgfsys@transformshift{1.696500in}{2.169299in}%
\pgfsys@useobject{currentmarker}{}%
\end{pgfscope}%
\end{pgfscope}%
\begin{pgfscope}%
\pgftext[x=1.696500in,y=0.245736in,,top]{\rmfamily\fontsize{8.000000}{9.600000}\selectfont \(\displaystyle 0.4\)}%
\end{pgfscope}%
\begin{pgfscope}%
\pgfsetbuttcap%
\pgfsetroundjoin%
\definecolor{currentfill}{rgb}{0.000000,0.000000,0.000000}%
\pgfsetfillcolor{currentfill}%
\pgfsetlinewidth{0.501875pt}%
\definecolor{currentstroke}{rgb}{0.000000,0.000000,0.000000}%
\pgfsetstrokecolor{currentstroke}%
\pgfsetdash{}{0pt}%
\pgfsys@defobject{currentmarker}{\pgfqpoint{0.000000in}{0.000000in}}{\pgfqpoint{0.000000in}{0.055556in}}{%
\pgfpathmoveto{\pgfqpoint{0.000000in}{0.000000in}}%
\pgfpathlineto{\pgfqpoint{0.000000in}{0.055556in}}%
\pgfusepath{stroke,fill}%
}%
\begin{pgfscope}%
\pgfsys@transformshift{2.301000in}{0.301292in}%
\pgfsys@useobject{currentmarker}{}%
\end{pgfscope}%
\end{pgfscope}%
\begin{pgfscope}%
\pgfsetbuttcap%
\pgfsetroundjoin%
\definecolor{currentfill}{rgb}{0.000000,0.000000,0.000000}%
\pgfsetfillcolor{currentfill}%
\pgfsetlinewidth{0.501875pt}%
\definecolor{currentstroke}{rgb}{0.000000,0.000000,0.000000}%
\pgfsetstrokecolor{currentstroke}%
\pgfsetdash{}{0pt}%
\pgfsys@defobject{currentmarker}{\pgfqpoint{0.000000in}{-0.055556in}}{\pgfqpoint{0.000000in}{0.000000in}}{%
\pgfpathmoveto{\pgfqpoint{0.000000in}{0.000000in}}%
\pgfpathlineto{\pgfqpoint{0.000000in}{-0.055556in}}%
\pgfusepath{stroke,fill}%
}%
\begin{pgfscope}%
\pgfsys@transformshift{2.301000in}{2.169299in}%
\pgfsys@useobject{currentmarker}{}%
\end{pgfscope}%
\end{pgfscope}%
\begin{pgfscope}%
\pgftext[x=2.301000in,y=0.245736in,,top]{\rmfamily\fontsize{8.000000}{9.600000}\selectfont \(\displaystyle 0.6\)}%
\end{pgfscope}%
\begin{pgfscope}%
\pgfsetbuttcap%
\pgfsetroundjoin%
\definecolor{currentfill}{rgb}{0.000000,0.000000,0.000000}%
\pgfsetfillcolor{currentfill}%
\pgfsetlinewidth{0.501875pt}%
\definecolor{currentstroke}{rgb}{0.000000,0.000000,0.000000}%
\pgfsetstrokecolor{currentstroke}%
\pgfsetdash{}{0pt}%
\pgfsys@defobject{currentmarker}{\pgfqpoint{0.000000in}{0.000000in}}{\pgfqpoint{0.000000in}{0.055556in}}{%
\pgfpathmoveto{\pgfqpoint{0.000000in}{0.000000in}}%
\pgfpathlineto{\pgfqpoint{0.000000in}{0.055556in}}%
\pgfusepath{stroke,fill}%
}%
\begin{pgfscope}%
\pgfsys@transformshift{2.905500in}{0.301292in}%
\pgfsys@useobject{currentmarker}{}%
\end{pgfscope}%
\end{pgfscope}%
\begin{pgfscope}%
\pgfsetbuttcap%
\pgfsetroundjoin%
\definecolor{currentfill}{rgb}{0.000000,0.000000,0.000000}%
\pgfsetfillcolor{currentfill}%
\pgfsetlinewidth{0.501875pt}%
\definecolor{currentstroke}{rgb}{0.000000,0.000000,0.000000}%
\pgfsetstrokecolor{currentstroke}%
\pgfsetdash{}{0pt}%
\pgfsys@defobject{currentmarker}{\pgfqpoint{0.000000in}{-0.055556in}}{\pgfqpoint{0.000000in}{0.000000in}}{%
\pgfpathmoveto{\pgfqpoint{0.000000in}{0.000000in}}%
\pgfpathlineto{\pgfqpoint{0.000000in}{-0.055556in}}%
\pgfusepath{stroke,fill}%
}%
\begin{pgfscope}%
\pgfsys@transformshift{2.905500in}{2.169299in}%
\pgfsys@useobject{currentmarker}{}%
\end{pgfscope}%
\end{pgfscope}%
\begin{pgfscope}%
\pgftext[x=2.905500in,y=0.245736in,,top]{\rmfamily\fontsize{8.000000}{9.600000}\selectfont \(\displaystyle 0.8\)}%
\end{pgfscope}%
\begin{pgfscope}%
\pgfsetbuttcap%
\pgfsetroundjoin%
\definecolor{currentfill}{rgb}{0.000000,0.000000,0.000000}%
\pgfsetfillcolor{currentfill}%
\pgfsetlinewidth{0.501875pt}%
\definecolor{currentstroke}{rgb}{0.000000,0.000000,0.000000}%
\pgfsetstrokecolor{currentstroke}%
\pgfsetdash{}{0pt}%
\pgfsys@defobject{currentmarker}{\pgfqpoint{0.000000in}{0.000000in}}{\pgfqpoint{0.000000in}{0.055556in}}{%
\pgfpathmoveto{\pgfqpoint{0.000000in}{0.000000in}}%
\pgfpathlineto{\pgfqpoint{0.000000in}{0.055556in}}%
\pgfusepath{stroke,fill}%
}%
\begin{pgfscope}%
\pgfsys@transformshift{3.510000in}{0.301292in}%
\pgfsys@useobject{currentmarker}{}%
\end{pgfscope}%
\end{pgfscope}%
\begin{pgfscope}%
\pgfsetbuttcap%
\pgfsetroundjoin%
\definecolor{currentfill}{rgb}{0.000000,0.000000,0.000000}%
\pgfsetfillcolor{currentfill}%
\pgfsetlinewidth{0.501875pt}%
\definecolor{currentstroke}{rgb}{0.000000,0.000000,0.000000}%
\pgfsetstrokecolor{currentstroke}%
\pgfsetdash{}{0pt}%
\pgfsys@defobject{currentmarker}{\pgfqpoint{0.000000in}{-0.055556in}}{\pgfqpoint{0.000000in}{0.000000in}}{%
\pgfpathmoveto{\pgfqpoint{0.000000in}{0.000000in}}%
\pgfpathlineto{\pgfqpoint{0.000000in}{-0.055556in}}%
\pgfusepath{stroke,fill}%
}%
\begin{pgfscope}%
\pgfsys@transformshift{3.510000in}{2.169299in}%
\pgfsys@useobject{currentmarker}{}%
\end{pgfscope}%
\end{pgfscope}%
\begin{pgfscope}%
\pgftext[x=3.510000in,y=0.245736in,,top]{\rmfamily\fontsize{8.000000}{9.600000}\selectfont \(\displaystyle 1.0\)}%
\end{pgfscope}%
\begin{pgfscope}%
\pgftext[x=1.998750in,y=0.078167in,,top]{\rmfamily\fontsize{10.000000}{12.000000}\selectfont \(\displaystyle x\)}%
\end{pgfscope}%
\begin{pgfscope}%
\pgfsetbuttcap%
\pgfsetroundjoin%
\definecolor{currentfill}{rgb}{0.000000,0.000000,0.000000}%
\pgfsetfillcolor{currentfill}%
\pgfsetlinewidth{0.501875pt}%
\definecolor{currentstroke}{rgb}{0.000000,0.000000,0.000000}%
\pgfsetstrokecolor{currentstroke}%
\pgfsetdash{}{0pt}%
\pgfsys@defobject{currentmarker}{\pgfqpoint{0.000000in}{0.000000in}}{\pgfqpoint{0.055556in}{0.000000in}}{%
\pgfpathmoveto{\pgfqpoint{0.000000in}{0.000000in}}%
\pgfpathlineto{\pgfqpoint{0.055556in}{0.000000in}}%
\pgfusepath{stroke,fill}%
}%
\begin{pgfscope}%
\pgfsys@transformshift{0.487500in}{0.301292in}%
\pgfsys@useobject{currentmarker}{}%
\end{pgfscope}%
\end{pgfscope}%
\begin{pgfscope}%
\pgfsetbuttcap%
\pgfsetroundjoin%
\definecolor{currentfill}{rgb}{0.000000,0.000000,0.000000}%
\pgfsetfillcolor{currentfill}%
\pgfsetlinewidth{0.501875pt}%
\definecolor{currentstroke}{rgb}{0.000000,0.000000,0.000000}%
\pgfsetstrokecolor{currentstroke}%
\pgfsetdash{}{0pt}%
\pgfsys@defobject{currentmarker}{\pgfqpoint{-0.055556in}{0.000000in}}{\pgfqpoint{0.000000in}{0.000000in}}{%
\pgfpathmoveto{\pgfqpoint{0.000000in}{0.000000in}}%
\pgfpathlineto{\pgfqpoint{-0.055556in}{0.000000in}}%
\pgfusepath{stroke,fill}%
}%
\begin{pgfscope}%
\pgfsys@transformshift{3.510000in}{0.301292in}%
\pgfsys@useobject{currentmarker}{}%
\end{pgfscope}%
\end{pgfscope}%
\begin{pgfscope}%
\pgftext[x=0.431944in,y=0.301292in,right,]{\rmfamily\fontsize{8.000000}{9.600000}\selectfont \(\displaystyle 0.0\)}%
\end{pgfscope}%
\begin{pgfscope}%
\pgfsetbuttcap%
\pgfsetroundjoin%
\definecolor{currentfill}{rgb}{0.000000,0.000000,0.000000}%
\pgfsetfillcolor{currentfill}%
\pgfsetlinewidth{0.501875pt}%
\definecolor{currentstroke}{rgb}{0.000000,0.000000,0.000000}%
\pgfsetstrokecolor{currentstroke}%
\pgfsetdash{}{0pt}%
\pgfsys@defobject{currentmarker}{\pgfqpoint{0.000000in}{0.000000in}}{\pgfqpoint{0.055556in}{0.000000in}}{%
\pgfpathmoveto{\pgfqpoint{0.000000in}{0.000000in}}%
\pgfpathlineto{\pgfqpoint{0.055556in}{0.000000in}}%
\pgfusepath{stroke,fill}%
}%
\begin{pgfscope}%
\pgfsys@transformshift{0.487500in}{0.674893in}%
\pgfsys@useobject{currentmarker}{}%
\end{pgfscope}%
\end{pgfscope}%
\begin{pgfscope}%
\pgfsetbuttcap%
\pgfsetroundjoin%
\definecolor{currentfill}{rgb}{0.000000,0.000000,0.000000}%
\pgfsetfillcolor{currentfill}%
\pgfsetlinewidth{0.501875pt}%
\definecolor{currentstroke}{rgb}{0.000000,0.000000,0.000000}%
\pgfsetstrokecolor{currentstroke}%
\pgfsetdash{}{0pt}%
\pgfsys@defobject{currentmarker}{\pgfqpoint{-0.055556in}{0.000000in}}{\pgfqpoint{0.000000in}{0.000000in}}{%
\pgfpathmoveto{\pgfqpoint{0.000000in}{0.000000in}}%
\pgfpathlineto{\pgfqpoint{-0.055556in}{0.000000in}}%
\pgfusepath{stroke,fill}%
}%
\begin{pgfscope}%
\pgfsys@transformshift{3.510000in}{0.674893in}%
\pgfsys@useobject{currentmarker}{}%
\end{pgfscope}%
\end{pgfscope}%
\begin{pgfscope}%
\pgftext[x=0.431944in,y=0.674893in,right,]{\rmfamily\fontsize{8.000000}{9.600000}\selectfont \(\displaystyle 0.2\)}%
\end{pgfscope}%
\begin{pgfscope}%
\pgfsetbuttcap%
\pgfsetroundjoin%
\definecolor{currentfill}{rgb}{0.000000,0.000000,0.000000}%
\pgfsetfillcolor{currentfill}%
\pgfsetlinewidth{0.501875pt}%
\definecolor{currentstroke}{rgb}{0.000000,0.000000,0.000000}%
\pgfsetstrokecolor{currentstroke}%
\pgfsetdash{}{0pt}%
\pgfsys@defobject{currentmarker}{\pgfqpoint{0.000000in}{0.000000in}}{\pgfqpoint{0.055556in}{0.000000in}}{%
\pgfpathmoveto{\pgfqpoint{0.000000in}{0.000000in}}%
\pgfpathlineto{\pgfqpoint{0.055556in}{0.000000in}}%
\pgfusepath{stroke,fill}%
}%
\begin{pgfscope}%
\pgfsys@transformshift{0.487500in}{1.048495in}%
\pgfsys@useobject{currentmarker}{}%
\end{pgfscope}%
\end{pgfscope}%
\begin{pgfscope}%
\pgfsetbuttcap%
\pgfsetroundjoin%
\definecolor{currentfill}{rgb}{0.000000,0.000000,0.000000}%
\pgfsetfillcolor{currentfill}%
\pgfsetlinewidth{0.501875pt}%
\definecolor{currentstroke}{rgb}{0.000000,0.000000,0.000000}%
\pgfsetstrokecolor{currentstroke}%
\pgfsetdash{}{0pt}%
\pgfsys@defobject{currentmarker}{\pgfqpoint{-0.055556in}{0.000000in}}{\pgfqpoint{0.000000in}{0.000000in}}{%
\pgfpathmoveto{\pgfqpoint{0.000000in}{0.000000in}}%
\pgfpathlineto{\pgfqpoint{-0.055556in}{0.000000in}}%
\pgfusepath{stroke,fill}%
}%
\begin{pgfscope}%
\pgfsys@transformshift{3.510000in}{1.048495in}%
\pgfsys@useobject{currentmarker}{}%
\end{pgfscope}%
\end{pgfscope}%
\begin{pgfscope}%
\pgftext[x=0.431944in,y=1.048495in,right,]{\rmfamily\fontsize{8.000000}{9.600000}\selectfont \(\displaystyle 0.4\)}%
\end{pgfscope}%
\begin{pgfscope}%
\pgfsetbuttcap%
\pgfsetroundjoin%
\definecolor{currentfill}{rgb}{0.000000,0.000000,0.000000}%
\pgfsetfillcolor{currentfill}%
\pgfsetlinewidth{0.501875pt}%
\definecolor{currentstroke}{rgb}{0.000000,0.000000,0.000000}%
\pgfsetstrokecolor{currentstroke}%
\pgfsetdash{}{0pt}%
\pgfsys@defobject{currentmarker}{\pgfqpoint{0.000000in}{0.000000in}}{\pgfqpoint{0.055556in}{0.000000in}}{%
\pgfpathmoveto{\pgfqpoint{0.000000in}{0.000000in}}%
\pgfpathlineto{\pgfqpoint{0.055556in}{0.000000in}}%
\pgfusepath{stroke,fill}%
}%
\begin{pgfscope}%
\pgfsys@transformshift{0.487500in}{1.422096in}%
\pgfsys@useobject{currentmarker}{}%
\end{pgfscope}%
\end{pgfscope}%
\begin{pgfscope}%
\pgfsetbuttcap%
\pgfsetroundjoin%
\definecolor{currentfill}{rgb}{0.000000,0.000000,0.000000}%
\pgfsetfillcolor{currentfill}%
\pgfsetlinewidth{0.501875pt}%
\definecolor{currentstroke}{rgb}{0.000000,0.000000,0.000000}%
\pgfsetstrokecolor{currentstroke}%
\pgfsetdash{}{0pt}%
\pgfsys@defobject{currentmarker}{\pgfqpoint{-0.055556in}{0.000000in}}{\pgfqpoint{0.000000in}{0.000000in}}{%
\pgfpathmoveto{\pgfqpoint{0.000000in}{0.000000in}}%
\pgfpathlineto{\pgfqpoint{-0.055556in}{0.000000in}}%
\pgfusepath{stroke,fill}%
}%
\begin{pgfscope}%
\pgfsys@transformshift{3.510000in}{1.422096in}%
\pgfsys@useobject{currentmarker}{}%
\end{pgfscope}%
\end{pgfscope}%
\begin{pgfscope}%
\pgftext[x=0.431944in,y=1.422096in,right,]{\rmfamily\fontsize{8.000000}{9.600000}\selectfont \(\displaystyle 0.6\)}%
\end{pgfscope}%
\begin{pgfscope}%
\pgfsetbuttcap%
\pgfsetroundjoin%
\definecolor{currentfill}{rgb}{0.000000,0.000000,0.000000}%
\pgfsetfillcolor{currentfill}%
\pgfsetlinewidth{0.501875pt}%
\definecolor{currentstroke}{rgb}{0.000000,0.000000,0.000000}%
\pgfsetstrokecolor{currentstroke}%
\pgfsetdash{}{0pt}%
\pgfsys@defobject{currentmarker}{\pgfqpoint{0.000000in}{0.000000in}}{\pgfqpoint{0.055556in}{0.000000in}}{%
\pgfpathmoveto{\pgfqpoint{0.000000in}{0.000000in}}%
\pgfpathlineto{\pgfqpoint{0.055556in}{0.000000in}}%
\pgfusepath{stroke,fill}%
}%
\begin{pgfscope}%
\pgfsys@transformshift{0.487500in}{1.795698in}%
\pgfsys@useobject{currentmarker}{}%
\end{pgfscope}%
\end{pgfscope}%
\begin{pgfscope}%
\pgfsetbuttcap%
\pgfsetroundjoin%
\definecolor{currentfill}{rgb}{0.000000,0.000000,0.000000}%
\pgfsetfillcolor{currentfill}%
\pgfsetlinewidth{0.501875pt}%
\definecolor{currentstroke}{rgb}{0.000000,0.000000,0.000000}%
\pgfsetstrokecolor{currentstroke}%
\pgfsetdash{}{0pt}%
\pgfsys@defobject{currentmarker}{\pgfqpoint{-0.055556in}{0.000000in}}{\pgfqpoint{0.000000in}{0.000000in}}{%
\pgfpathmoveto{\pgfqpoint{0.000000in}{0.000000in}}%
\pgfpathlineto{\pgfqpoint{-0.055556in}{0.000000in}}%
\pgfusepath{stroke,fill}%
}%
\begin{pgfscope}%
\pgfsys@transformshift{3.510000in}{1.795698in}%
\pgfsys@useobject{currentmarker}{}%
\end{pgfscope}%
\end{pgfscope}%
\begin{pgfscope}%
\pgftext[x=0.431944in,y=1.795698in,right,]{\rmfamily\fontsize{8.000000}{9.600000}\selectfont \(\displaystyle 0.8\)}%
\end{pgfscope}%
\begin{pgfscope}%
\pgfsetbuttcap%
\pgfsetroundjoin%
\definecolor{currentfill}{rgb}{0.000000,0.000000,0.000000}%
\pgfsetfillcolor{currentfill}%
\pgfsetlinewidth{0.501875pt}%
\definecolor{currentstroke}{rgb}{0.000000,0.000000,0.000000}%
\pgfsetstrokecolor{currentstroke}%
\pgfsetdash{}{0pt}%
\pgfsys@defobject{currentmarker}{\pgfqpoint{0.000000in}{0.000000in}}{\pgfqpoint{0.055556in}{0.000000in}}{%
\pgfpathmoveto{\pgfqpoint{0.000000in}{0.000000in}}%
\pgfpathlineto{\pgfqpoint{0.055556in}{0.000000in}}%
\pgfusepath{stroke,fill}%
}%
\begin{pgfscope}%
\pgfsys@transformshift{0.487500in}{2.169299in}%
\pgfsys@useobject{currentmarker}{}%
\end{pgfscope}%
\end{pgfscope}%
\begin{pgfscope}%
\pgfsetbuttcap%
\pgfsetroundjoin%
\definecolor{currentfill}{rgb}{0.000000,0.000000,0.000000}%
\pgfsetfillcolor{currentfill}%
\pgfsetlinewidth{0.501875pt}%
\definecolor{currentstroke}{rgb}{0.000000,0.000000,0.000000}%
\pgfsetstrokecolor{currentstroke}%
\pgfsetdash{}{0pt}%
\pgfsys@defobject{currentmarker}{\pgfqpoint{-0.055556in}{0.000000in}}{\pgfqpoint{0.000000in}{0.000000in}}{%
\pgfpathmoveto{\pgfqpoint{0.000000in}{0.000000in}}%
\pgfpathlineto{\pgfqpoint{-0.055556in}{0.000000in}}%
\pgfusepath{stroke,fill}%
}%
\begin{pgfscope}%
\pgfsys@transformshift{3.510000in}{2.169299in}%
\pgfsys@useobject{currentmarker}{}%
\end{pgfscope}%
\end{pgfscope}%
\begin{pgfscope}%
\pgftext[x=0.431944in,y=2.169299in,right,]{\rmfamily\fontsize{8.000000}{9.600000}\selectfont \(\displaystyle 1.0\)}%
\end{pgfscope}%
\begin{pgfscope}%
\pgfsetbuttcap%
\pgfsetmiterjoin%
\definecolor{currentfill}{rgb}{1.000000,1.000000,1.000000}%
\pgfsetfillcolor{currentfill}%
\pgfsetlinewidth{1.003750pt}%
\definecolor{currentstroke}{rgb}{0.000000,0.000000,0.000000}%
\pgfsetstrokecolor{currentstroke}%
\pgfsetdash{}{0pt}%
\pgfpathmoveto{\pgfqpoint{0.543056in}{1.758811in}}%
\pgfpathlineto{\pgfqpoint{1.196319in}{1.758811in}}%
\pgfpathlineto{\pgfqpoint{1.196319in}{2.113744in}}%
\pgfpathlineto{\pgfqpoint{0.543056in}{2.113744in}}%
\pgfpathclose%
\pgfusepath{stroke,fill}%
\end{pgfscope}%
\begin{pgfscope}%
\pgfsetbuttcap%
\pgfsetroundjoin%
\pgfsetlinewidth{1.003750pt}%
\definecolor{currentstroke}{rgb}{1.000000,0.000000,0.000000}%
\pgfsetstrokecolor{currentstroke}%
\pgfsetdash{{6.000000pt}{6.000000pt}}{0.000000pt}%
\pgfpathmoveto{\pgfqpoint{0.620833in}{2.030410in}}%
\pgfpathlineto{\pgfqpoint{0.776389in}{2.030410in}}%
\pgfusepath{stroke}%
\end{pgfscope}%
\begin{pgfscope}%
\pgftext[x=0.898611in,y=1.991522in,left,base]{\rmfamily\fontsize{8.000000}{9.600000}\selectfont \(\displaystyle x\)}%
\end{pgfscope}%
\begin{pgfscope}%
\pgfsetrectcap%
\pgfsetroundjoin%
\pgfsetlinewidth{1.003750pt}%
\definecolor{currentstroke}{rgb}{0.000000,0.000000,1.000000}%
\pgfsetstrokecolor{currentstroke}%
\pgfsetdash{}{0pt}%
\pgfpathmoveto{\pgfqpoint{0.620833in}{1.869922in}}%
\pgfpathlineto{\pgfqpoint{0.776389in}{1.869922in}}%
\pgfusepath{stroke}%
\end{pgfscope}%
\begin{pgfscope}%
\pgftext[x=0.898611in,y=1.831033in,left,base]{\rmfamily\fontsize{8.000000}{9.600000}\selectfont \(\displaystyle V(x)\)}%
\end{pgfscope}%
\end{pgfpicture}%
\makeatother%
\endgroup%

		\caption{Visualization of Lemma~\ref{eq:important_fact}'s proof for a Beta-Bernoulli problem with parameters $(3,7)$ and $\gamma=0.7$. The intersection of the two lines represents the Optimistic Gittins Index.}
		\label{fig:visaulize_gx_proof}
	\end{figure}
	
	For the converse, assume $\lambda \ge x$. Recall that $V(z)$ is convex. Since $V$ is continuous on $[0,1]$, by the Intermediate Value Theorem, there exists a point $\lambda$ at which $\lambda =V(\lambda)$. Therefore let $\eps < (1-\lambda)/2$ and from the first direction of the proof, we have $\lambda + \eps > V(\lambda + \eps)$. Thus
	\[
	V(\lambda + \eps) \ge V(\lambda) + \eps V'(\lambda) = \lambda + \eps V'(\lambda)
	\]
	where the inequality follows from $V$ being convex and differentiable. This implies that $V'(\lambda) < 1$ and, moreover, because $V$ is also increasing, it follows that $V'(\lambda) \in (0,1)$, whence
	\begin{align*}
	V(x) & \ge V(\lambda) - (\lambda - x) V'(\lambda) \\
	& = \lambda - (\lambda - x) V'(\lambda) \\
	& = (1-V'(\lambda)) \lambda + V'(\lambda) x \\
	& \ge \min(x,\lambda) = x.
	\end{align*}
	This completes the proof.
\end{myproof}
\begin{corollary} \label{cor:equivalent_event}
	Let $v_{i,t}$ be the Optimistic Gittins Index of arm $i$ at time $t$ and let $x \in (0,1)$. The following equivalence holds
	\[
	\{v^K_{i,t} < x \} = \{V_K(y_{i,t}; x, \gamma_t) < x\}\]
	where $y_{i,t}$ is the sufficient statistic for estimating the $i$th arm's parameter $\theta_i$.
\end{corollary}
\begin{myproof}[Proof.]
	By the definition in Equation~\eqref{eqn:ogi_k1}, $v^K_{i,t}$ can be characterized with the relation 
	\begin{equation*}% \label{eq:form4}
	v_{i,t} = \sup\left\{ x \in [0,1] : x \le V_K(y_{i,t}; x, \gamma_t)  \right\}.
	\end{equation*}
	The conclusion then follows from Lemma~\ref{eq:important_fact}.
\end{myproof}
The next Lemma will be the final property for the function $V_K$ that we prove. This will subsequently be used in the proof of Lemma~\ref{lemma:underestimation}.
\begin{lemma} \label{lemma:vk_bound}
	Let $i$ be any arm. For any state $y$, look-ahead parameter $K \in \mathbb{Z}_+$ and $\gamma, \eta \in [0,1]$, we have
	\begin{equation*}
		\Ee{V_K(y'; \gamma, \eta) \given y} \ge V_K(y; \gamma, \eta)
	\end{equation*}
	where $y'$ is the next state (posterior) for that arm.
\end{lemma}
\begin{myproof}[Proof.]
	Let $\tau^\star(y)$ be the optimal stopping time for an initial state $y$ in the problem for computing $V_K$. Then
	\begin{align}
		\Ee{V_K(y'; \gamma, \eta) \given y} & = \Ee{\Ee{\sum_{s=1}^{\tau^\star(y')-1} \gamma^{s-1} X_{i,s} + \frac{\gamma^{\tau^\star(y')-1} R_{\eta, K}(\tau, y_{i,\tau^\star(y')})}{1-\gamma} \given[\Bigg] y_{i,1} = y'} \given[\Bigg] y} \nonumber \\
		& \ge  \Ee{\Ee{\sum_{s=1}^{\tau^\star(y)-1} \gamma^{s-1} X_{i,s} + \frac{\gamma^{\tau^\star(y)-1} R_{\eta, K}(\tau, y_{i,\tau^\star(y)})}{1-\gamma} \given[\Bigg] y_{i,1} = y'} \given[\Bigg] y} \label{ineq:subopt_of_y}\\
		& = \Ee{\sum_{s=1}^{\tau^\star(y)-1} \gamma^{s-1} X_{i,s} + \frac{\gamma^{\tau^\star(y)-1} R_{\eta, K}(\tau, y_{i,\tau^\star(y)})}{1-\gamma} \given[\Bigg] y_{i,1} = y} \label{eq:tower_prop} \\
		& = V_K(y; \gamma, \eta). \nonumber
	\end{align}
	The bound \eqref{ineq:subopt_of_y} is due to the sub-optimality of $\tau^\star(y)$ when the actual starting state is $y'$ and equation \eqref{eq:tower_prop} is the tower property.
\end{myproof}
\subsection{Proof of Lemma~\ref{lemma:underestimation}} \label{proof:underestimation_proof}
\begin{myproof}[Proof.]
	This proof involves induction. We state what the base case is, in the following Lemma, but we defer its proof to the end of the section.
	\begin{lemma} \label{lemma:underestimation_base_case}
		Suppose the OGI algorithm uses only one look-ahead step, and so $K = 1$. Then we have
		\[
		\P{v^1_{1,t} < \eta} = \mathcal{O}\left(\frac{1}{t^{1 + h(\eta)}}\right)
		\]
		where $h$ is a positive yet decreasing function of $\eta$.
	\end{lemma}
	Now we show the induction step. Suppose that for $K \ge 2$, it holds that
	\[
	\P{v^{K-1}_{1,t} < \eta} = \mathcal{O}\left(\frac{1}{t^{1 + h(\eta)}}\right).
	\]
	We show the same is true for $v^{K}_{1,t}$. Indeed, we have for $t > 1$
	\begin{align}
	\P{v^K_{1,t} < \eta} & = \P{V_K(y_{1,t}; \eta, \gamma_t) < \eta} \label{eq:appl_of_lemma_9}\\
	& = \P{(1-\gamma_t)\Ee{X_{1,t} \given y_{1,t}} + \gamma_t \Ee{\max(\eta, V_{K-1}(y_{1,t+1}; \eta, \gamma_{t})) \given y_{1,t}}< \eta} \nonumber \\
	& \le \P{\frac{1}{t}\Ee{X_{1,t} \given y_{1,t}} + \gamma_t \max(\eta, \Ee{V_{K-1}(y_{1,t+1}; \eta, \gamma_{t}) \given y_{1,t}})< \eta} \label{ineq:appl_jensens} \\
	& \le \P{\frac{1}{t}\Ee{X_{1,t} \given y_{1,t}} + \left(1-\frac{1}{t}\right) \Ee{V_{K-1}(y_{1,t+1}; \eta, \gamma_{t}) \given y_{1,t} }< \eta} \nonumber \\
	& \le \P{\frac{1}{t}\Ee{X_{1,t} \given y_{1,t}} + \left(1-\frac{1}{t}\right) V_{K-1}(y_{1,t}; \eta, \gamma_{t})< \eta} \label{ineq:missing_step} \\
	& \le \P{ V_{K-1}(y_{1,t}; \eta, \gamma_{t})< \eta \left(\frac{t}{t-1}\right)} \nonumber \\
	& \le \mathcal{O}\left(\frac{1}{t^{1 + h(\eta t/(t-1))}}\right) =  \mathcal{O}\left(\frac{1}{t^{1 + h(\eta)}}\right) \nonumber
	\end{align}
	where \eqref{eq:appl_of_lemma_9} follows from Lemma~\ref{cor:equivalent_event} and \eqref{ineq:appl_jensens} uses Jensen's inequality. Lemma~\ref{lemma:vk_bound} give us \eqref{ineq:missing_step}.
	
	We finish the proof by using the following asymptotic argument. Take $M$ to be a large enough integer, then we have, using the result of the induction proof,
	\begin{align*}
	\sum_{t=1}^\infty \P{v^K_{1,t} < \eta} & \le M +  \sum_{t=M+1}^\infty \frac{C_1}{t^{1 + h(\eta)}} \le M +  C_2
	\end{align*}
	where $C_2 = C_2(\eta)$ is the limit of the series and $C_1$ is a constant used in the definition of the big-Oh.
\end{myproof}
\subsubsection{Proof of the base case in Lemma~\ref{lemma:underestimation_base_case}}
\begin{myproof}[Proof.]
	For simplicity let's abbreviate $v^1_{1,t}$ as $v_{1,t}$. Define $\delta := (\theta_1 - \eta)/2$ and  $\eta' :=  \eta + \delta$. In other words, $\delta$ is half the distance between $\eta$ and $\theta_1$; $\eta'$ is the point half-way.
	
	The proof consists of showing two claims
	\subsubsection*{Claim 1: $\{v_{1,t} < \eta\} \subseteq \left\{F^B_{N_1(t)+1, \eta'}(S_1(t)) < \frac{1}{\delta t}\right\}$:}
	Let $V_t \sim $Beta$(S_1(t)+1,N_1(t) - S_1(t) + 1)$ be the agent's posterior on the optimal arm. Using Corollary~\ref{cor:equivalent_event} and the simplified form for $K=1$ \[V_K((S_1(t)+1, N_1(t) - S_1(t) + 1); \eta, \gamma_t) = \E{V_t} + \gamma_t\Ee{(\eta - V_t)^+}\] we find that
	\begin{align}
	\left\{v_{1,t} < \eta \right\} & = \left\{ \Ee{V_t } + \gamma_t\Ee{(\eta - V_t)^+} < \eta \right\} \nonumber\\
	& =  \left\{ (1-1/t)\Ee{(\eta - V_t)^+} < \Ee{\eta - V_t} \right\} \label{eq:def_gamma_t} \\
	& =  \left\{ \Ee{(\eta - V_t)^+} - \Ee{\eta - V_t} <  \frac{1}{t}\Ee{(\eta - V_t)^+}\right\} \nonumber\\
	& =  \left\{ \Ee{(V_t - \eta)^+}<  \frac{1}{t}\Ee{(\eta - V_t)^+}\right\} \nonumber\\
	& \subseteq \left\{ \Ee{ (V_t - \eta)^+}< \frac{1}{t}  \right\}  \label{eq:intermediate_event}
	\end{align}
	where \eqref{eq:def_gamma_t} follows from the definition of $\gamma_t$ and \eqref{eq:intermediate_event} is due to $V_t, \eta$ both lying in the interval $[0,1]$. We approximate the conditional expectation in \eqref{eq:intermediate_event} with
	\begin{align}
	\Ee{(V_t - \eta)^+ \given S_1(t), N_1(t)} & = \Ee{(V_t - \eta) \ind{V_t \ge \eta} }\nonumber \\
	& = \Ee{(V_t - \eta) \ind{\eta + \delta > V_t \ge \eta} }  \nonumber \\
	& \qquad + \Ee{(V_t - \eta) \ind{ V_t \ge \eta + \delta} } \nonumber \\
	& > \Ee{(V_t - \eta) \ind{ V_t \ge \eta + \delta} } \nonumber \\
	& \ge \delta\P{ V_t \ge \eta' } \nonumber \\
	& = \delta (1 - F_{S_1(t)+1,N_1(t)-S_1(t)+1}(\eta')) = \delta F^B_{N_1(t)+1,\eta'}(S_1(t)) \label{ineq:lower_bound_on_ppart_term}
	\end{align}
	The last equality is due to Fact~\ref{fact:equation_for_beta_binomial_cdfs} and this proves the claim.
	\subsubsection*{Claim 2: $ \sum_{t=1}^\infty \mathbb{P}\left(F^B_{N_1(t)+1, \eta'}(S_1(t)) < \frac{1}{\delta t}\right) \le C_1$ where $C_1$ is a constant:}
	Let us fix the sequence $f_t = -\frac{\log \delta t }{\log (1-\eta')}-1 = O(\log t)$. We then have
	\begin{align}
	\P{F^B_{N_1(t)+1, \eta'}(S_1(t)) < \frac{1}{\delta t}} & = \P{F^B_{N_1(t)+1, \eta'}(S_1(t)) < \frac{1}{\delta t}, \; N_1(t) > f_t}  \nonumber \\
	& \qquad + \P{F^B_{N_1(t)+1, \eta'}(S_1(t)) < \frac{1}{\delta t}, \; N_1(t) \le f_t} \label{eq:decomp2}.
	\end{align}
	For the second term in the RHS of \eqref{eq:decomp2} we have the following bound,
	\begin{align}
	\P{F^B_{N_1(t)+1, \eta'}(S_1(t)) < \frac{1}{\delta t}, \; N_1(t) \le f_t}  &  \le \P{F^B_{N_1(t)+1,\eta'}(0) < \frac{1}{\delta  t}, \; N_1(t) \le f_t} \nonumber \\
	& = \P{(1-\eta')^{N_1(t)+1} <  \frac{1}{\delta  t}, \; N_1(t) \le f_t} \nonumber \\
	& \le \P{(1-\eta')^{f_t+1} <  \frac{1}{\delta  t}} = 0.\label{bound:bdd_by_zero}
	\end{align}
	Now we use the following fact to bound the left term on the RHS of \eqref{eq:decomp2}. Define the function
	\[
	F^{-B}_{n,p}(u) := \inf\{x : F^B_{n,p}(x) \ge u\}
	\]
	which is the inverse CDF. Then it is known that if $U \sim \text{Unif}(0,1)$, then $F^{-B}_{n,p}(U) \sim \text{Binomial}(n,p)$. Furthermore, $F^B_{n,p}(F^{-B}_{n,p}(U)) \ge U$ due to the definition of the inverse CDF.
	
	Now let us only consider large $t$, in particular $t > M = M(\theta_1, \eta')$ where:
	\begin{enumerate}
		\item $M$ is such that $e^{d(\eta', \theta_1)f_{M}/2} > (f_M + 1)^4$
		\item $M > \frac{4}{(1-\eta')\delta }$
		\item $\ceil{f_M} > 0$ and $F^B_{\ceil{f_M},\eta'}(f_M \eta') > 1/4$. Note that there is a large enough integer for this because $F^B_{\ceil{f_t},\eta'}(f_t \eta') \to \frac{1}{2}$ as $t \to \infty$.
	\end{enumerate} 
	Suppose that $t > M$. It then follows that the event $\{F^B_{N_1(t), \eta'}(S_1(t)) < \frac{1}{(1-\eta')\delta t},\; S_1(t) \ge N_1(t) \eta', \; N_1(t) > f_t\}$ has measure zero because of the assumptions made on $M$. Therefore if $t > M$, we have
	\begin{align}
	\mathbb{P}\bigg(F^B_{N_1(t)+1,  \eta'}(S_1(t)) &< \frac{1}{\delta t }  , \; N_1(t) > f_t \bigg) \nonumber \\
	& \le \P{F^B_{N_1(t),  \eta'}(S_1(t)) < \frac{1}{(1-\eta')\delta t}, \; N_1(t) > f_t} \label{eqn:part1_decomp_the_cdf_of_y} \\ 
	& = \P{F^B_{N_1(t),  \eta'}(S_1(t)) < \frac{1}{(1-\eta')\delta t}, \; S_1(t) < N_1(t) \eta', \; N_1(t) > f_t} \nonumber \\ 
	& =  \P{F^B_{N_1(t),\theta_1}(S_1(t)) \frac{F^B_{N_1(t),\eta'}(S_1(t))}{F^B_{N_1(t),\theta_1}(S_1(t))} < \frac{1}{(1-\eta')\delta  t}, \;S_1(t) < N_1(t) \eta', \; N_1(t) > f_t} \nonumber \\
	& \le  \P{F_{N_1(t),\theta_1}^B(S_1(t))  e^{N_1(t) D} < \frac{1}{(1-\eta')\delta  t} , \; N_1(t) > f_t} \label{eqn:part1_app_of_lemma2} \\
	& \le  \P{F_{N_1(t),\theta_1}^B(S_1(t)) e^{f_t D} < \frac{1}{(1-\eta')\delta  t}} \nonumber \\
	& =  \P{F_{N_1(t),\theta_1}^B(F^{-B}_{N_1(t),\theta_1}(U)) < \frac{e^{-f_t D}}{(1-\eta')\delta  t} } \label{eqn:part1_propert_of_inverse_sampling}\\
	& \le  \P{U < \frac{e^{-f_t D}}{(1-\eta')\delta  t} } \nonumber \\  
	& =  \frac{e^{-f_t D}}{(1-\eta')\delta  t} \nonumber  \nonumber\\
	& = O\left( \frac{1}{t^{1+Dc_{\eta'}}} \right)  \label{bound:one_over_t_plus_eps} 
	\end{align}
	where $D = d(\eta',\theta_1) > 0$ and $c_{\eta'} = -\log^{-1}(1-\eta') > 0$ are constant. The bound \eqref{eqn:part1_decomp_the_cdf_of_y} holds due to Fact~\eqref{fact:relationship_with_binom_cdfs}. Bound \eqref{eqn:part1_app_of_lemma2} follows from an application of Lemma~\ref{lemma:ratio_of_cdfs} and the fact that $t > M$. Equation \eqref{eqn:part1_propert_of_inverse_sampling} follows from $S_1(t) \sim \text{Binomial}(N_1(t), \theta_1)$ and the inverse sampling technique. By combining bounds \eqref{bound:one_over_t_plus_eps}, \eqref{bound:bdd_by_zero} and \eqref{eq:decomp2}, we get the big-Oh bound.
\end{myproof}

\subsection{Proof of Lemma~\ref{lemma:overestimation}} \label{proof:overestimation_proof}

\begin{proof}[Proof.]
	See the main proof of Theorem~\ref{thm:frequentist_optimal_bound} to recall the definition of constants $\eta_1$, $\eta_3$ and their relationship with $\theta_2$ and $\theta_1$. As an abbreviation we let $L = L(T)$. Moreover, because for any arm $i$ $v^K_{i,t} \le v^{K-1}_{i,t} \le \ldots \le v^1_{i,t}$, it will be sufficient to consider this proof only for $v^1_{2,t}$, which we also will abbreviate as $v_{2,t} \defeq v^1_{2,t}$.
	
	Firstly, by the law of total probability, we find that
	\begin{align} 
	\sum_{t=1}^T \mathbb{P}(v_{2,t} & \ge \eta_3 ,\; N_2(t) \ge L,\; \pi^{\rm OG}_t = 2) \nonumber \\
	& = \sum_{t=1}^T \P{v_{2,t} \ge \eta_3 ,\; N_2(t) \ge L, \; S_2(t) < \floor{N_2(t) \eta_1}, \; \pi^{\rm OG}_t = 2} \nonumber \\
	& \qquad + \sum_{t=1}^T \P{v_{2,t} \ge \eta_3 ,\; N_2(t) \ge L, \; S_2(t) \ge \floor{N_2(t) \eta_1},\; \pi^{\rm OG}_t = 2} \nonumber \\
	& \le \sum_{t=1}^T \P{v_{2,t} \ge \eta_3 ,\; N_2(t) \ge L, \; S_2(t) < \floor{N_2(t) \eta_1}} + \sum_{t=1}^T \P{\pi^{\rm OG}_t = 2,\; S_2(t) \ge \floor{N_2(t) \eta_1}} \label{eqn:splitting_not_underestimate}
	\end{align}
	Let $V_t \sim \text{Beta}(S_2(t) + 1, N_2(t)- S_2(t) + 1)$ denote the agent's posterior on the second arm at time $t$, then
	\begin{align}
	\sum_{t=1}^T \mathbb{P}(v_{2,t} \ge \eta_3 ,\; & \; N_2(t) \ge L,\; S_2(t) < \floor{N_2(t) \eta_1})  \nonumber\\
	& = \sum_{t=1}^T \P{\Ee{V_t} + \gamma_t \Ee{(\eta_3 - V_t)^+} \ge \eta_3, \; N_2(t) \ge L,\; S_2(t) < \floor{N_2(t) \eta_1}} \nonumber \\
	& = \sum_{t=1}^T \P{\frac{\Ee{(\eta_3-V_t)^+ }}{  \Ee{(V_t - \eta_3)^+ }} \le t , \; N_2(t) \ge L,\; S_2(t) < \floor{N_2(t) \eta_1}} \label{eq:complicated_rv_in_part2}
	\end{align}
	where the second equality follows from Corollary~\ref{cor:equivalent_event} in Appendix~\ref{sec:amgi_results}. The following result lets us bound \eqref{eq:complicated_rv_in_part2},
	\begin{lemma} \label{lem:lb_rv2}
		Let $0 < x < y < 1$. For any non-negative integers $s$ and $k$ with $s < \floor{kx}$, it holds that
		\begin{equation*}
		\frac{\Ee{(y-V)^+ }}{  \Ee{(V - y)^+ } } \ge \frac{(y-x) \exp(k d(x,y))}{2}
		\end{equation*}
		where $V \sim \text{Beta}(s+1,k-s+1)$.
	\end{lemma}
	\begin{myproof}[Proof.]
		See Appendix~\ref{prf:proof_of_lb_rv2}.
	\end{myproof}
	Therefore, from equation \eqref{eq:complicated_rv_in_part2} and Lemma~\ref{lem:lb_rv2}, we find that whenever $T > \left(\frac{2}{\eta_3-\eta_1}\right)^{1/\eps} =: T^*(\eps, \theta)$,
	\begin{align}
	\sum_{t=1}^T \mathbb{P}(v_{2,t} \ge \eta_3 ,\; & \; N_2(t) \ge L,\; S_2(t) < \floor{N_2(t) \eta_1}) \nonumber \\
	& \le  \sum_{t=1}^T\P{  (\eta_3-\eta_1) \exp\{N_2(t) d(\eta_1,\eta_3) \} \le 2t,\; N_2(t) \ge L} \nonumber \\
	& \le  \sum_{t=1}^T\P{  (\eta_3-\eta_1) \exp\{L d(\eta_1,\eta_3) \} \le 2t} \nonumber \\
	& =   \sum_{t=1}^T\P{  (\eta_3-\eta_1) T^{1+\eps} \le 2t} = 0 \label{bound:equal_to_zero}
	\end{align}
	All that is left is to bound the second term in \eqref{eqn:splitting_not_underestimate}, and to do so we apply the following Lemma whose proof is in Appendix~\ref{prf:proof_of_acc_sub_means}
	\begin{lemma} \label{lem:accurate_suboptimal_mean}
		There exist positive constants $C = C(\theta_2,\eta_1)$ and $x' > \theta_2$ such that
		\begin{equation*}
		\sum_{t=1}^T \P{S_2(t) \ge \floor{N_2(t) \eta_1}, \; \pi^{\rm OG}_t = 2} \le  K + \frac{1}{1 - e^{-d(x',\theta_2)}} 
		\end{equation*}
	\end{lemma}
	Combining Lemma~\ref{lem:accurate_suboptimal_mean}, \eqref{bound:equal_to_zero}, \eqref{eqn:splitting_not_underestimate} and \eqref{eq:complicated_rv_in_part2} shows the claim.
\end{proof}

\subsubsection{Proof of Lemma~\ref{lem:lb_rv2}.} \label{prf:proof_of_lb_rv2}
\begin{myproof}[Proof.]
	We upper bound the denominator as follows. Given that $s < \floor{k x}$, we have $s \le kx - 1$. Let $B(a,b)$ denote the Beta function, then
	\begin{align}
	\Ee{(V - y)^+ } & = \frac{1}{B(s+1,k-s+1)}\int_{y}^1 (t-y) t^s (1-t)^{k-s} \; dt \nonumber \\
	& = \frac{1}{B(s+1,k-s+1)}\int_{y}^1 t^{s+1} (1-t)^{k-s}  dt - y \P{V \ge y} \nonumber \\
	& = \frac{B(s+2,k-s+1)}{B(s+1,j-s+1)}\left( \frac{1}{B(s+2,k-s+1)} \right)\int_{y}^1 t^{s+1} (1-t)^{k-s}  dt - y \P{V \ge y} \nonumber \\
	& = \frac{s+1}{k+2} F^B_{k+2,y}(s+1)  - y \P{V \ge y} \label{eq:part2_use_of_equiv_between_beta_and_binom} \\
	& \le \frac{s+1}{k+2} F^B_{k+2,y}(s+1) \le  F^B_{k,y}(k x) \le \exp\left\{- k d(x,y) \label{ineq:chernoff_app} \right\}
	\end{align}
	where we use Fact~\ref{fact:equation_for_beta_binomial_cdfs} and the definition of the Beta CDF to establish equation \eqref{eq:part2_use_of_equiv_between_beta_and_binom}. The final bound in \eqref{ineq:chernoff_app} is the result of the Chernoff-Hoeffding theorem and Fact~\ref{fact:relationship_with_binom_cdfs}. Let $\delta:=y-x$, and note that $s < kx \Longrightarrow s \le \floor{(k+1)x}$ due to $s$ being integer, whence
	\begin{align}
	\Ee{(y - V)^+ } & =  \Ee{(y - V) \ind{V \le y} \given s, k} \nonumber \\
	& = \Ee{(y - V) \ind{y - \delta \le V \le y} \given s, k} +  \Ee{(y - V) \ind{V < y - \delta} \given s, k} \nonumber\\
	& > \Ee{(y - V) \ind{V < y - \delta} \given s, k}\nonumber \\
	& \ge \delta\Ee{\ind{V < y-\delta} \given s, k}\\
	& = \delta \P{V < x \given s} \nonumber \\
	& = \delta\left(1 - F^B_{k+1,x}(s) \right) \label{eq:use_of_bin_beta_identity}  \\
	& \ge \delta/2  \label{eq:use_of_median_prop}
	\end{align}
	where equation \eqref{eq:use_of_bin_beta_identity} relies on Fact~\ref{fact:equation_for_beta_binomial_cdfs}. The bound \eqref{eq:use_of_median_prop} is justified from Fact~\ref{fact:median_of_binomial_dist} and $s \le \floor{(k+1) x}$. Thus using the inequalities for both the numerator and denominator, we obtain the desired bound.
\end{myproof}
\subsubsection{Proof of Lemma~\ref{lem:accurate_suboptimal_mean}.} \label{prf:proof_of_acc_sub_means}
\begin{proof}[Proof.]
	The steps in this proof follow a similar one in \cite{agrawal2013further} but we show them for completeness. We bound the number of times the sub-optimal arm's mean is overestimated. Let $\tau_\ell$ be the time step in which the  sub-optimal arm is sampled for the $\ell$\textsuperscript{th} time. Because for any $x$, $\lim_{n\to\infty}\frac{\floor{nx}}{nx} = 1$, we can let $N$ be a large enough integer so that if $\ell \ge N$, then $\eta_1 \frac{\floor{\ell \eta_1}}{\ell \eta_1} > x' := (\theta_2 + \eta_1)/2 > \theta_2$. In that case,
	\begin{align}
	\sum_{t=1}^T\P{S_2(t) \ge \floor{N_2(t) \eta_1}, \; \pi^{\rm OG}_t = 2} & \le \Ee{\sum_{\ell=1}^T \sum_{t=\tau_\ell}^{\tau_{\ell+1}-1}\ind{S_2(\ell) \ge \floor{N_1(\ell) \eta_1}} \ind{\pi^{\rm OG}_t = 2}} \nonumber \\
	& = \Ee{\sum_{\ell=1}^T \ind{S_2(\tau_{\ell}) \ge \floor{(\ell-1) \eta_1}} \sum_{t=\tau_\ell}^{\tau_{\ell+1}-1} \ind{\pi^{\rm OG}_t = 2}} \nonumber\\
	& = \Ee{\sum_{\ell=0}^{T-1} \ind{S_2(\tau_{\ell+1}) \ge \floor{\ell \eta_1}}} \nonumber\\
	& \le  N + \sum_{\ell=N+1}^{T-1} \P{ S_2(\tau_{\ell+1}) \ge \ell \eta_1 \frac{\floor{\ell \eta_1}}{\ell \eta_1}} \nonumber \\
	& \le N + \sum_{\ell=N+1}^{T-1} \P{ S_2(\tau_{\ell+1}) \ge \ell x'} \nonumber \\
	& \le  N + \sum_{\ell=1}^{\infty} \exp(-\ell d(x', \theta_2)) \label{bound:cf_thm} \\
	& = N + \frac{1}{1 - e^{-d(x',\theta_2)}} \nonumber
	\end{align}
\end{proof}
The bound \eqref{bound:cf_thm} follows from the Chernoff-Hoeffding theorem and that $S_2(\tau_{\ell+1}) \sim \text{Binomial}(N_1(\ell+1), \theta_2) \sim \text{Binomial}(\ell, \theta_2)$.

\section{Further experiment results} \label{sec:further_exp}
\subsection{Bayes UCB experiment} \label{exp:bayes_ucb}
This experiment is motivated by \cite{kaufmann2012thompson} and in it we simulate the Bernoulli bandit problem with a $T = 500$ and two arms. Since we are interested in measuring expected regret over the prior, we draw the arms' mean rewards at random from the uniform distribution. There are 5,000 independent trials and we show the results in Figures~\ref{fig:kaufmann_regret}. OGI offers notable performance improvements over both Thompson Sampling and IDS for this modest horizon.
\begin{figure}[h!]
	\centering
	%% Creator: Matplotlib, PGF backend
%%
%% To include the figure in your LaTeX document, write
%%   \input{<filename>.pgf}
%%
%% Make sure the required packages are loaded in your preamble
%%   \usepackage{pgf}
%%
%% Figures using additional raster images can only be included by \input if
%% they are in the same directory as the main LaTeX file. For loading figures
%% from other directories you can use the `import` package
%%   \usepackage{import}
%% and then include the figures with
%%   \import{<path to file>}{<filename>.pgf}
%%
%% Matplotlib used the following preamble
%%   \usepackage[utf8x]{inputenc}
%%   \usepackage[T1]{fontenc}
%%
\begingroup%
\makeatletter%
\begin{pgfpicture}%
\pgfpathrectangle{\pgfpointorigin}{\pgfqpoint{3.900000in}{2.410333in}}%
\pgfusepath{use as bounding box, clip}%
\begin{pgfscope}%
\pgfsetbuttcap%
\pgfsetmiterjoin%
\definecolor{currentfill}{rgb}{1.000000,1.000000,1.000000}%
\pgfsetfillcolor{currentfill}%
\pgfsetlinewidth{0.000000pt}%
\definecolor{currentstroke}{rgb}{1.000000,1.000000,1.000000}%
\pgfsetstrokecolor{currentstroke}%
\pgfsetdash{}{0pt}%
\pgfpathmoveto{\pgfqpoint{0.000000in}{0.000000in}}%
\pgfpathlineto{\pgfqpoint{3.900000in}{0.000000in}}%
\pgfpathlineto{\pgfqpoint{3.900000in}{2.410333in}}%
\pgfpathlineto{\pgfqpoint{0.000000in}{2.410333in}}%
\pgfpathclose%
\pgfusepath{fill}%
\end{pgfscope}%
\begin{pgfscope}%
\pgfsetbuttcap%
\pgfsetmiterjoin%
\definecolor{currentfill}{rgb}{1.000000,1.000000,1.000000}%
\pgfsetfillcolor{currentfill}%
\pgfsetlinewidth{0.000000pt}%
\definecolor{currentstroke}{rgb}{0.000000,0.000000,0.000000}%
\pgfsetstrokecolor{currentstroke}%
\pgfsetstrokeopacity{0.000000}%
\pgfsetdash{}{0pt}%
\pgfpathmoveto{\pgfqpoint{0.487500in}{0.301292in}}%
\pgfpathlineto{\pgfqpoint{3.510000in}{0.301292in}}%
\pgfpathlineto{\pgfqpoint{3.510000in}{2.169299in}}%
\pgfpathlineto{\pgfqpoint{0.487500in}{2.169299in}}%
\pgfpathclose%
\pgfusepath{fill}%
\end{pgfscope}%
\begin{pgfscope}%
\pgfpathrectangle{\pgfqpoint{0.487500in}{0.301292in}}{\pgfqpoint{3.022500in}{1.868008in}} %
\pgfusepath{clip}%
\pgfsetrectcap%
\pgfsetroundjoin%
\pgfsetlinewidth{1.505625pt}%
\definecolor{currentstroke}{rgb}{0.000000,0.000000,1.000000}%
\pgfsetstrokecolor{currentstroke}%
\pgfsetdash{}{0pt}%
\pgfpathmoveto{\pgfqpoint{0.487500in}{0.355294in}}%
\pgfpathlineto{\pgfqpoint{0.493545in}{0.381588in}}%
\pgfpathlineto{\pgfqpoint{0.499590in}{0.407072in}}%
\pgfpathlineto{\pgfqpoint{0.511680in}{0.447517in}}%
\pgfpathlineto{\pgfqpoint{0.560040in}{0.539869in}}%
\pgfpathlineto{\pgfqpoint{0.572130in}{0.559891in}}%
\pgfpathlineto{\pgfqpoint{0.584220in}{0.574682in}}%
\pgfpathlineto{\pgfqpoint{0.602355in}{0.590798in}}%
\pgfpathlineto{\pgfqpoint{0.608400in}{0.600341in}}%
\pgfpathlineto{\pgfqpoint{0.614445in}{0.605215in}}%
\pgfpathlineto{\pgfqpoint{0.620490in}{0.613949in}}%
\pgfpathlineto{\pgfqpoint{0.626535in}{0.619009in}}%
\pgfpathlineto{\pgfqpoint{0.632580in}{0.627371in}}%
\pgfpathlineto{\pgfqpoint{0.638625in}{0.631185in}}%
\pgfpathlineto{\pgfqpoint{0.644670in}{0.636683in}}%
\pgfpathlineto{\pgfqpoint{0.650715in}{0.639189in}}%
\pgfpathlineto{\pgfqpoint{0.656760in}{0.644314in}}%
\pgfpathlineto{\pgfqpoint{0.668850in}{0.650634in}}%
\pgfpathlineto{\pgfqpoint{0.674895in}{0.654638in}}%
\pgfpathlineto{\pgfqpoint{0.680940in}{0.662997in}}%
\pgfpathlineto{\pgfqpoint{0.686985in}{0.666191in}}%
\pgfpathlineto{\pgfqpoint{0.693030in}{0.673554in}}%
\pgfpathlineto{\pgfqpoint{0.711165in}{0.685622in}}%
\pgfpathlineto{\pgfqpoint{0.717210in}{0.688691in}}%
\pgfpathlineto{\pgfqpoint{0.723255in}{0.690451in}}%
\pgfpathlineto{\pgfqpoint{0.729300in}{0.696447in}}%
\pgfpathlineto{\pgfqpoint{0.741390in}{0.701711in}}%
\pgfpathlineto{\pgfqpoint{0.747435in}{0.707518in}}%
\pgfpathlineto{\pgfqpoint{0.771615in}{0.718794in}}%
\pgfpathlineto{\pgfqpoint{0.777660in}{0.720927in}}%
\pgfpathlineto{\pgfqpoint{0.789750in}{0.732107in}}%
\pgfpathlineto{\pgfqpoint{0.795795in}{0.733807in}}%
\pgfpathlineto{\pgfqpoint{0.801840in}{0.737870in}}%
\pgfpathlineto{\pgfqpoint{0.807885in}{0.740438in}}%
\pgfpathlineto{\pgfqpoint{0.832065in}{0.755513in}}%
\pgfpathlineto{\pgfqpoint{0.844155in}{0.758412in}}%
\pgfpathlineto{\pgfqpoint{0.856245in}{0.764984in}}%
\pgfpathlineto{\pgfqpoint{0.862290in}{0.769670in}}%
\pgfpathlineto{\pgfqpoint{0.886470in}{0.780199in}}%
\pgfpathlineto{\pgfqpoint{0.892515in}{0.784386in}}%
\pgfpathlineto{\pgfqpoint{0.904605in}{0.786911in}}%
\pgfpathlineto{\pgfqpoint{0.910650in}{0.789234in}}%
\pgfpathlineto{\pgfqpoint{0.922740in}{0.796678in}}%
\pgfpathlineto{\pgfqpoint{0.934830in}{0.798891in}}%
\pgfpathlineto{\pgfqpoint{0.940875in}{0.801460in}}%
\pgfpathlineto{\pgfqpoint{0.946920in}{0.802596in}}%
\pgfpathlineto{\pgfqpoint{0.952965in}{0.806849in}}%
\pgfpathlineto{\pgfqpoint{0.959010in}{0.813776in}}%
\pgfpathlineto{\pgfqpoint{0.965055in}{0.815414in}}%
\pgfpathlineto{\pgfqpoint{0.971100in}{0.820286in}}%
\pgfpathlineto{\pgfqpoint{0.977145in}{0.821364in}}%
\pgfpathlineto{\pgfqpoint{0.983190in}{0.826672in}}%
\pgfpathlineto{\pgfqpoint{0.995280in}{0.829633in}}%
\pgfpathlineto{\pgfqpoint{1.007370in}{0.837513in}}%
\pgfpathlineto{\pgfqpoint{1.013415in}{0.838524in}}%
\pgfpathlineto{\pgfqpoint{1.019460in}{0.842777in}}%
\pgfpathlineto{\pgfqpoint{1.025505in}{0.844287in}}%
\pgfpathlineto{\pgfqpoint{1.031550in}{0.848973in}}%
\pgfpathlineto{\pgfqpoint{1.037595in}{0.848929in}}%
\pgfpathlineto{\pgfqpoint{1.061775in}{0.858836in}}%
\pgfpathlineto{\pgfqpoint{1.067820in}{0.858976in}}%
\pgfpathlineto{\pgfqpoint{1.073865in}{0.860922in}}%
\pgfpathlineto{\pgfqpoint{1.079910in}{0.865361in}}%
\pgfpathlineto{\pgfqpoint{1.085955in}{0.867432in}}%
\pgfpathlineto{\pgfqpoint{1.098045in}{0.868338in}}%
\pgfpathlineto{\pgfqpoint{1.104090in}{0.871408in}}%
\pgfpathlineto{\pgfqpoint{1.110135in}{0.871672in}}%
\pgfpathlineto{\pgfqpoint{1.122225in}{0.874259in}}%
\pgfpathlineto{\pgfqpoint{1.128270in}{0.878945in}}%
\pgfpathlineto{\pgfqpoint{1.134315in}{0.879960in}}%
\pgfpathlineto{\pgfqpoint{1.140360in}{0.883338in}}%
\pgfpathlineto{\pgfqpoint{1.152450in}{0.884493in}}%
\pgfpathlineto{\pgfqpoint{1.164540in}{0.890318in}}%
\pgfpathlineto{\pgfqpoint{1.170585in}{0.887905in}}%
\pgfpathlineto{\pgfqpoint{1.176630in}{0.893777in}}%
\pgfpathlineto{\pgfqpoint{1.182675in}{0.895287in}}%
\pgfpathlineto{\pgfqpoint{1.188720in}{0.899104in}}%
\pgfpathlineto{\pgfqpoint{1.194765in}{0.899306in}}%
\pgfpathlineto{\pgfqpoint{1.206855in}{0.903699in}}%
\pgfpathlineto{\pgfqpoint{1.218945in}{0.905975in}}%
\pgfpathlineto{\pgfqpoint{1.231035in}{0.906632in}}%
\pgfpathlineto{\pgfqpoint{1.237080in}{0.910321in}}%
\pgfpathlineto{\pgfqpoint{1.243125in}{0.911274in}}%
\pgfpathlineto{\pgfqpoint{1.249170in}{0.918014in}}%
\pgfpathlineto{\pgfqpoint{1.255215in}{0.919960in}}%
\pgfpathlineto{\pgfqpoint{1.261260in}{0.923279in}}%
\pgfpathlineto{\pgfqpoint{1.267305in}{0.924166in}}%
\pgfpathlineto{\pgfqpoint{1.273350in}{0.923189in}}%
\pgfpathlineto{\pgfqpoint{1.279395in}{0.925321in}}%
\pgfpathlineto{\pgfqpoint{1.285440in}{0.928952in}}%
\pgfpathlineto{\pgfqpoint{1.297530in}{0.931227in}}%
\pgfpathlineto{\pgfqpoint{1.309620in}{0.933002in}}%
\pgfpathlineto{\pgfqpoint{1.327755in}{0.935172in}}%
\pgfpathlineto{\pgfqpoint{1.345890in}{0.935844in}}%
\pgfpathlineto{\pgfqpoint{1.357980in}{0.940237in}}%
\pgfpathlineto{\pgfqpoint{1.364025in}{0.945546in}}%
\pgfpathlineto{\pgfqpoint{1.370070in}{0.947557in}}%
\pgfpathlineto{\pgfqpoint{1.376115in}{0.948195in}}%
\pgfpathlineto{\pgfqpoint{1.388205in}{0.954767in}}%
\pgfpathlineto{\pgfqpoint{1.394250in}{0.955032in}}%
\pgfpathlineto{\pgfqpoint{1.400295in}{0.960157in}}%
\pgfpathlineto{\pgfqpoint{1.412385in}{0.965857in}}%
\pgfpathlineto{\pgfqpoint{1.418430in}{0.965312in}}%
\pgfpathlineto{\pgfqpoint{1.430520in}{0.970577in}}%
\pgfpathlineto{\pgfqpoint{1.436565in}{0.974702in}}%
\pgfpathlineto{\pgfqpoint{1.460745in}{0.977884in}}%
\pgfpathlineto{\pgfqpoint{1.466790in}{0.978339in}}%
\pgfpathlineto{\pgfqpoint{1.478880in}{0.976380in}}%
\pgfpathlineto{\pgfqpoint{1.490970in}{0.980957in}}%
\pgfpathlineto{\pgfqpoint{1.497015in}{0.984587in}}%
\pgfpathlineto{\pgfqpoint{1.503060in}{0.985288in}}%
\pgfpathlineto{\pgfqpoint{1.509105in}{0.984497in}}%
\pgfpathlineto{\pgfqpoint{1.521195in}{0.987271in}}%
\pgfpathlineto{\pgfqpoint{1.527240in}{0.989715in}}%
\pgfpathlineto{\pgfqpoint{1.533285in}{0.990789in}}%
\pgfpathlineto{\pgfqpoint{1.539330in}{0.993174in}}%
\pgfpathlineto{\pgfqpoint{1.545375in}{0.992006in}}%
\pgfpathlineto{\pgfqpoint{1.551420in}{0.995076in}}%
\pgfpathlineto{\pgfqpoint{1.557465in}{0.996461in}}%
\pgfpathlineto{\pgfqpoint{1.569555in}{0.996745in}}%
\pgfpathlineto{\pgfqpoint{1.587690in}{1.001900in}}%
\pgfpathlineto{\pgfqpoint{1.605825in}{1.005502in}}%
\pgfpathlineto{\pgfqpoint{1.611870in}{1.010064in}}%
\pgfpathlineto{\pgfqpoint{1.617915in}{1.008709in}}%
\pgfpathlineto{\pgfqpoint{1.623960in}{1.009226in}}%
\pgfpathlineto{\pgfqpoint{1.636050in}{1.014989in}}%
\pgfpathlineto{\pgfqpoint{1.642095in}{1.017433in}}%
\pgfpathlineto{\pgfqpoint{1.660230in}{1.017299in}}%
\pgfpathlineto{\pgfqpoint{1.672320in}{1.015649in}}%
\pgfpathlineto{\pgfqpoint{1.690455in}{1.014768in}}%
\pgfpathlineto{\pgfqpoint{1.702545in}{1.018600in}}%
\pgfpathlineto{\pgfqpoint{1.714635in}{1.020998in}}%
\pgfpathlineto{\pgfqpoint{1.720680in}{1.020456in}}%
\pgfpathlineto{\pgfqpoint{1.726725in}{1.025640in}}%
\pgfpathlineto{\pgfqpoint{1.732770in}{1.028647in}}%
\pgfpathlineto{\pgfqpoint{1.738815in}{1.026483in}}%
\pgfpathlineto{\pgfqpoint{1.744860in}{1.031857in}}%
\pgfpathlineto{\pgfqpoint{1.756950in}{1.033569in}}%
\pgfpathlineto{\pgfqpoint{1.781130in}{1.042417in}}%
\pgfpathlineto{\pgfqpoint{1.793220in}{1.043510in}}%
\pgfpathlineto{\pgfqpoint{1.799265in}{1.045086in}}%
\pgfpathlineto{\pgfqpoint{1.811355in}{1.042626in}}%
\pgfpathlineto{\pgfqpoint{1.823445in}{1.047828in}}%
\pgfpathlineto{\pgfqpoint{1.829490in}{1.050960in}}%
\pgfpathlineto{\pgfqpoint{1.847625in}{1.052380in}}%
\pgfpathlineto{\pgfqpoint{1.853670in}{1.054637in}}%
\pgfpathlineto{\pgfqpoint{1.859715in}{1.055528in}}%
\pgfpathlineto{\pgfqpoint{1.865760in}{1.054858in}}%
\pgfpathlineto{\pgfqpoint{1.871805in}{1.056870in}}%
\pgfpathlineto{\pgfqpoint{1.877850in}{1.061119in}}%
\pgfpathlineto{\pgfqpoint{1.883895in}{1.062321in}}%
\pgfpathlineto{\pgfqpoint{1.889940in}{1.067692in}}%
\pgfpathlineto{\pgfqpoint{1.902030in}{1.072334in}}%
\pgfpathlineto{\pgfqpoint{1.926210in}{1.074083in}}%
\pgfpathlineto{\pgfqpoint{1.932255in}{1.076776in}}%
\pgfpathlineto{\pgfqpoint{1.962480in}{1.080472in}}%
\pgfpathlineto{\pgfqpoint{1.968525in}{1.084289in}}%
\pgfpathlineto{\pgfqpoint{1.980615in}{1.089055in}}%
\pgfpathlineto{\pgfqpoint{1.992705in}{1.090207in}}%
\pgfpathlineto{\pgfqpoint{1.998750in}{1.093464in}}%
\pgfpathlineto{\pgfqpoint{2.004795in}{1.093417in}}%
\pgfpathlineto{\pgfqpoint{2.010840in}{1.097421in}}%
\pgfpathlineto{\pgfqpoint{2.022930in}{1.100880in}}%
\pgfpathlineto{\pgfqpoint{2.028975in}{1.099401in}}%
\pgfpathlineto{\pgfqpoint{2.059200in}{1.108140in}}%
\pgfpathlineto{\pgfqpoint{2.077335in}{1.106509in}}%
\pgfpathlineto{\pgfqpoint{2.083380in}{1.107897in}}%
\pgfpathlineto{\pgfqpoint{2.089425in}{1.110528in}}%
\pgfpathlineto{\pgfqpoint{2.101515in}{1.110625in}}%
\pgfpathlineto{\pgfqpoint{2.107560in}{1.113383in}}%
\pgfpathlineto{\pgfqpoint{2.113605in}{1.112216in}}%
\pgfpathlineto{\pgfqpoint{2.119650in}{1.112418in}}%
\pgfpathlineto{\pgfqpoint{2.125695in}{1.113744in}}%
\pgfpathlineto{\pgfqpoint{2.131740in}{1.118118in}}%
\pgfpathlineto{\pgfqpoint{2.149875in}{1.119915in}}%
\pgfpathlineto{\pgfqpoint{2.155920in}{1.119619in}}%
\pgfpathlineto{\pgfqpoint{2.168010in}{1.124261in}}%
\pgfpathlineto{\pgfqpoint{2.174055in}{1.125709in}}%
\pgfpathlineto{\pgfqpoint{2.180100in}{1.124918in}}%
\pgfpathlineto{\pgfqpoint{2.192190in}{1.126135in}}%
\pgfpathlineto{\pgfqpoint{2.198235in}{1.125279in}}%
\pgfpathlineto{\pgfqpoint{2.210325in}{1.128364in}}%
\pgfpathlineto{\pgfqpoint{2.228460in}{1.134330in}}%
\pgfpathlineto{\pgfqpoint{2.234505in}{1.133477in}}%
\pgfpathlineto{\pgfqpoint{2.240550in}{1.134115in}}%
\pgfpathlineto{\pgfqpoint{2.246595in}{1.133573in}}%
\pgfpathlineto{\pgfqpoint{2.252640in}{1.135768in}}%
\pgfpathlineto{\pgfqpoint{2.258685in}{1.135908in}}%
\pgfpathlineto{\pgfqpoint{2.264730in}{1.137421in}}%
\pgfpathlineto{\pgfqpoint{2.270775in}{1.137374in}}%
\pgfpathlineto{\pgfqpoint{2.276820in}{1.138452in}}%
\pgfpathlineto{\pgfqpoint{2.282865in}{1.137782in}}%
\pgfpathlineto{\pgfqpoint{2.313090in}{1.141851in}}%
\pgfpathlineto{\pgfqpoint{2.325180in}{1.145933in}}%
\pgfpathlineto{\pgfqpoint{2.331225in}{1.146201in}}%
\pgfpathlineto{\pgfqpoint{2.337270in}{1.149952in}}%
\pgfpathlineto{\pgfqpoint{2.349360in}{1.149053in}}%
\pgfpathlineto{\pgfqpoint{2.355405in}{1.152119in}}%
\pgfpathlineto{\pgfqpoint{2.361450in}{1.153695in}}%
\pgfpathlineto{\pgfqpoint{2.367495in}{1.152091in}}%
\pgfpathlineto{\pgfqpoint{2.391675in}{1.158262in}}%
\pgfpathlineto{\pgfqpoint{2.397720in}{1.157779in}}%
\pgfpathlineto{\pgfqpoint{2.409810in}{1.159370in}}%
\pgfpathlineto{\pgfqpoint{2.421900in}{1.157661in}}%
\pgfpathlineto{\pgfqpoint{2.427945in}{1.160419in}}%
\pgfpathlineto{\pgfqpoint{2.433990in}{1.165105in}}%
\pgfpathlineto{\pgfqpoint{2.440035in}{1.167113in}}%
\pgfpathlineto{\pgfqpoint{2.458170in}{1.167166in}}%
\pgfpathlineto{\pgfqpoint{2.464215in}{1.168427in}}%
\pgfpathlineto{\pgfqpoint{2.470260in}{1.166453in}}%
\pgfpathlineto{\pgfqpoint{2.476305in}{1.168897in}}%
\pgfpathlineto{\pgfqpoint{2.482350in}{1.167231in}}%
\pgfpathlineto{\pgfqpoint{2.494440in}{1.166083in}}%
\pgfpathlineto{\pgfqpoint{2.500485in}{1.164794in}}%
\pgfpathlineto{\pgfqpoint{2.506530in}{1.165868in}}%
\pgfpathlineto{\pgfqpoint{2.512575in}{1.164579in}}%
\pgfpathlineto{\pgfqpoint{2.518620in}{1.161979in}}%
\pgfpathlineto{\pgfqpoint{2.524665in}{1.163928in}}%
\pgfpathlineto{\pgfqpoint{2.530710in}{1.164193in}}%
\pgfpathlineto{\pgfqpoint{2.536755in}{1.167633in}}%
\pgfpathlineto{\pgfqpoint{2.542800in}{1.166531in}}%
\pgfpathlineto{\pgfqpoint{2.548845in}{1.168788in}}%
\pgfpathlineto{\pgfqpoint{2.566980in}{1.170460in}}%
\pgfpathlineto{\pgfqpoint{2.585115in}{1.170198in}}%
\pgfpathlineto{\pgfqpoint{2.591160in}{1.172207in}}%
\pgfpathlineto{\pgfqpoint{2.603250in}{1.179713in}}%
\pgfpathlineto{\pgfqpoint{2.615340in}{1.179934in}}%
\pgfpathlineto{\pgfqpoint{2.645565in}{1.186431in}}%
\pgfpathlineto{\pgfqpoint{2.663700in}{1.188972in}}%
\pgfpathlineto{\pgfqpoint{2.669745in}{1.187683in}}%
\pgfpathlineto{\pgfqpoint{2.675790in}{1.188570in}}%
\pgfpathlineto{\pgfqpoint{2.681835in}{1.187157in}}%
\pgfpathlineto{\pgfqpoint{2.699970in}{1.186335in}}%
\pgfpathlineto{\pgfqpoint{2.706015in}{1.183987in}}%
\pgfpathlineto{\pgfqpoint{2.712060in}{1.187179in}}%
\pgfpathlineto{\pgfqpoint{2.718105in}{1.187132in}}%
\pgfpathlineto{\pgfqpoint{2.724150in}{1.184473in}}%
\pgfpathlineto{\pgfqpoint{2.730195in}{1.183119in}}%
\pgfpathlineto{\pgfqpoint{2.736240in}{1.183636in}}%
\pgfpathlineto{\pgfqpoint{2.742285in}{1.187014in}}%
\pgfpathlineto{\pgfqpoint{2.748330in}{1.183981in}}%
\pgfpathlineto{\pgfqpoint{2.754375in}{1.184184in}}%
\pgfpathlineto{\pgfqpoint{2.766465in}{1.189012in}}%
\pgfpathlineto{\pgfqpoint{2.772510in}{1.188717in}}%
\pgfpathlineto{\pgfqpoint{2.778555in}{1.191475in}}%
\pgfpathlineto{\pgfqpoint{2.784600in}{1.192362in}}%
\pgfpathlineto{\pgfqpoint{2.796690in}{1.191463in}}%
\pgfpathlineto{\pgfqpoint{2.808780in}{1.193365in}}%
\pgfpathlineto{\pgfqpoint{2.814825in}{1.194813in}}%
\pgfpathlineto{\pgfqpoint{2.820870in}{1.198380in}}%
\pgfpathlineto{\pgfqpoint{2.839005in}{1.200734in}}%
\pgfpathlineto{\pgfqpoint{2.845050in}{1.199318in}}%
\pgfpathlineto{\pgfqpoint{2.851095in}{1.199399in}}%
\pgfpathlineto{\pgfqpoint{2.881320in}{1.193131in}}%
\pgfpathlineto{\pgfqpoint{2.887365in}{1.192590in}}%
\pgfpathlineto{\pgfqpoint{2.893410in}{1.193166in}}%
\pgfpathlineto{\pgfqpoint{2.905500in}{1.197870in}}%
\pgfpathlineto{\pgfqpoint{2.911545in}{1.197076in}}%
\pgfpathlineto{\pgfqpoint{2.923635in}{1.199227in}}%
\pgfpathlineto{\pgfqpoint{2.929680in}{1.199806in}}%
\pgfpathlineto{\pgfqpoint{2.947815in}{1.196867in}}%
\pgfpathlineto{\pgfqpoint{2.965950in}{1.198847in}}%
\pgfpathlineto{\pgfqpoint{2.984085in}{1.207428in}}%
\pgfpathlineto{\pgfqpoint{2.996175in}{1.207898in}}%
\pgfpathlineto{\pgfqpoint{3.032445in}{1.216967in}}%
\pgfpathlineto{\pgfqpoint{3.038490in}{1.216049in}}%
\pgfpathlineto{\pgfqpoint{3.050580in}{1.224240in}}%
\pgfpathlineto{\pgfqpoint{3.056625in}{1.225255in}}%
\pgfpathlineto{\pgfqpoint{3.062670in}{1.220600in}}%
\pgfpathlineto{\pgfqpoint{3.068715in}{1.224666in}}%
\pgfpathlineto{\pgfqpoint{3.080805in}{1.223826in}}%
\pgfpathlineto{\pgfqpoint{3.086850in}{1.222101in}}%
\pgfpathlineto{\pgfqpoint{3.092895in}{1.224047in}}%
\pgfpathlineto{\pgfqpoint{3.098940in}{1.223630in}}%
\pgfpathlineto{\pgfqpoint{3.117075in}{1.230467in}}%
\pgfpathlineto{\pgfqpoint{3.123120in}{1.233097in}}%
\pgfpathlineto{\pgfqpoint{3.129165in}{1.231746in}}%
\pgfpathlineto{\pgfqpoint{3.141255in}{1.231282in}}%
\pgfpathlineto{\pgfqpoint{3.153345in}{1.235239in}}%
\pgfpathlineto{\pgfqpoint{3.159390in}{1.232951in}}%
\pgfpathlineto{\pgfqpoint{3.165435in}{1.236142in}}%
\pgfpathlineto{\pgfqpoint{3.177525in}{1.236114in}}%
\pgfpathlineto{\pgfqpoint{3.183570in}{1.238125in}}%
\pgfpathlineto{\pgfqpoint{3.189615in}{1.236958in}}%
\pgfpathlineto{\pgfqpoint{3.201705in}{1.244527in}}%
\pgfpathlineto{\pgfqpoint{3.213795in}{1.244623in}}%
\pgfpathlineto{\pgfqpoint{3.231930in}{1.249841in}}%
\pgfpathlineto{\pgfqpoint{3.237975in}{1.249673in}}%
\pgfpathlineto{\pgfqpoint{3.244020in}{1.251868in}}%
\pgfpathlineto{\pgfqpoint{3.250065in}{1.252385in}}%
\pgfpathlineto{\pgfqpoint{3.256110in}{1.251715in}}%
\pgfpathlineto{\pgfqpoint{3.262155in}{1.254035in}}%
\pgfpathlineto{\pgfqpoint{3.268200in}{1.254801in}}%
\pgfpathlineto{\pgfqpoint{3.286335in}{1.249558in}}%
\pgfpathlineto{\pgfqpoint{3.292380in}{1.250946in}}%
\pgfpathlineto{\pgfqpoint{3.298425in}{1.250526in}}%
\pgfpathlineto{\pgfqpoint{3.304470in}{1.252160in}}%
\pgfpathlineto{\pgfqpoint{3.316560in}{1.250513in}}%
\pgfpathlineto{\pgfqpoint{3.322605in}{1.250532in}}%
\pgfpathlineto{\pgfqpoint{3.334695in}{1.255236in}}%
\pgfpathlineto{\pgfqpoint{3.346785in}{1.256886in}}%
\pgfpathlineto{\pgfqpoint{3.352830in}{1.255971in}}%
\pgfpathlineto{\pgfqpoint{3.370965in}{1.258885in}}%
\pgfpathlineto{\pgfqpoint{3.377010in}{1.262080in}}%
\pgfpathlineto{\pgfqpoint{3.383055in}{1.258670in}}%
\pgfpathlineto{\pgfqpoint{3.395145in}{1.258891in}}%
\pgfpathlineto{\pgfqpoint{3.401190in}{1.261647in}}%
\pgfpathlineto{\pgfqpoint{3.419325in}{1.258898in}}%
\pgfpathlineto{\pgfqpoint{3.431415in}{1.261423in}}%
\pgfpathlineto{\pgfqpoint{3.437460in}{1.260940in}}%
\pgfpathlineto{\pgfqpoint{3.449550in}{1.263091in}}%
\pgfpathlineto{\pgfqpoint{3.455595in}{1.261488in}}%
\pgfpathlineto{\pgfqpoint{3.473730in}{1.259237in}}%
\pgfpathlineto{\pgfqpoint{3.479775in}{1.261307in}}%
\pgfpathlineto{\pgfqpoint{3.485820in}{1.266616in}}%
\pgfpathlineto{\pgfqpoint{3.497910in}{1.263225in}}%
\pgfpathlineto{\pgfqpoint{3.503955in}{1.262185in}}%
\pgfpathlineto{\pgfqpoint{3.503955in}{1.262185in}}%
\pgfusepath{stroke}%
\end{pgfscope}%
\begin{pgfscope}%
\pgfpathrectangle{\pgfqpoint{0.487500in}{0.301292in}}{\pgfqpoint{3.022500in}{1.868008in}} %
\pgfusepath{clip}%
\pgfsetrectcap%
\pgfsetroundjoin%
\pgfsetlinewidth{1.505625pt}%
\definecolor{currentstroke}{rgb}{0.000000,0.500000,0.000000}%
\pgfsetstrokecolor{currentstroke}%
\pgfsetdash{}{0pt}%
\pgfpathmoveto{\pgfqpoint{0.487500in}{0.357037in}}%
\pgfpathlineto{\pgfqpoint{0.493545in}{0.384452in}}%
\pgfpathlineto{\pgfqpoint{0.499590in}{0.410559in}}%
\pgfpathlineto{\pgfqpoint{0.511680in}{0.446271in}}%
\pgfpathlineto{\pgfqpoint{0.523770in}{0.475135in}}%
\pgfpathlineto{\pgfqpoint{0.529815in}{0.486235in}}%
\pgfpathlineto{\pgfqpoint{0.535860in}{0.500075in}}%
\pgfpathlineto{\pgfqpoint{0.553995in}{0.528582in}}%
\pgfpathlineto{\pgfqpoint{0.572130in}{0.559953in}}%
\pgfpathlineto{\pgfqpoint{0.578175in}{0.567753in}}%
\pgfpathlineto{\pgfqpoint{0.584220in}{0.572565in}}%
\pgfpathlineto{\pgfqpoint{0.590265in}{0.581548in}}%
\pgfpathlineto{\pgfqpoint{0.602355in}{0.592666in}}%
\pgfpathlineto{\pgfqpoint{0.620490in}{0.614946in}}%
\pgfpathlineto{\pgfqpoint{0.626535in}{0.618012in}}%
\pgfpathlineto{\pgfqpoint{0.638625in}{0.628570in}}%
\pgfpathlineto{\pgfqpoint{0.644670in}{0.632449in}}%
\pgfpathlineto{\pgfqpoint{0.656760in}{0.642882in}}%
\pgfpathlineto{\pgfqpoint{0.662805in}{0.645139in}}%
\pgfpathlineto{\pgfqpoint{0.686985in}{0.669554in}}%
\pgfpathlineto{\pgfqpoint{0.699075in}{0.680734in}}%
\pgfpathlineto{\pgfqpoint{0.705120in}{0.685731in}}%
\pgfpathlineto{\pgfqpoint{0.717210in}{0.690497in}}%
\pgfpathlineto{\pgfqpoint{0.729300in}{0.699560in}}%
\pgfpathlineto{\pgfqpoint{0.735345in}{0.699700in}}%
\pgfpathlineto{\pgfqpoint{0.741390in}{0.704451in}}%
\pgfpathlineto{\pgfqpoint{0.753480in}{0.718745in}}%
\pgfpathlineto{\pgfqpoint{0.759525in}{0.721624in}}%
\pgfpathlineto{\pgfqpoint{0.765570in}{0.723134in}}%
\pgfpathlineto{\pgfqpoint{0.771615in}{0.726080in}}%
\pgfpathlineto{\pgfqpoint{0.777660in}{0.726469in}}%
\pgfpathlineto{\pgfqpoint{0.795795in}{0.737294in}}%
\pgfpathlineto{\pgfqpoint{0.807885in}{0.747475in}}%
\pgfpathlineto{\pgfqpoint{0.819975in}{0.757347in}}%
\pgfpathlineto{\pgfqpoint{0.826020in}{0.762845in}}%
\pgfpathlineto{\pgfqpoint{0.832065in}{0.765725in}}%
\pgfpathlineto{\pgfqpoint{0.838110in}{0.765495in}}%
\pgfpathlineto{\pgfqpoint{0.844155in}{0.766631in}}%
\pgfpathlineto{\pgfqpoint{0.850200in}{0.769075in}}%
\pgfpathlineto{\pgfqpoint{0.856245in}{0.775818in}}%
\pgfpathlineto{\pgfqpoint{0.874380in}{0.787761in}}%
\pgfpathlineto{\pgfqpoint{0.880425in}{0.792076in}}%
\pgfpathlineto{\pgfqpoint{0.892515in}{0.796030in}}%
\pgfpathlineto{\pgfqpoint{0.898560in}{0.797855in}}%
\pgfpathlineto{\pgfqpoint{0.910650in}{0.803680in}}%
\pgfpathlineto{\pgfqpoint{0.922740in}{0.807824in}}%
\pgfpathlineto{\pgfqpoint{0.940875in}{0.816778in}}%
\pgfpathlineto{\pgfqpoint{0.959010in}{0.827226in}}%
\pgfpathlineto{\pgfqpoint{0.971100in}{0.838842in}}%
\pgfpathlineto{\pgfqpoint{0.977145in}{0.839981in}}%
\pgfpathlineto{\pgfqpoint{0.983190in}{0.844418in}}%
\pgfpathlineto{\pgfqpoint{0.989235in}{0.844932in}}%
\pgfpathlineto{\pgfqpoint{1.007370in}{0.852332in}}%
\pgfpathlineto{\pgfqpoint{1.013415in}{0.853033in}}%
\pgfpathlineto{\pgfqpoint{1.019460in}{0.859589in}}%
\pgfpathlineto{\pgfqpoint{1.031550in}{0.866221in}}%
\pgfpathlineto{\pgfqpoint{1.037595in}{0.866177in}}%
\pgfpathlineto{\pgfqpoint{1.049685in}{0.874182in}}%
\pgfpathlineto{\pgfqpoint{1.061775in}{0.878325in}}%
\pgfpathlineto{\pgfqpoint{1.067820in}{0.880022in}}%
\pgfpathlineto{\pgfqpoint{1.079910in}{0.885785in}}%
\pgfpathlineto{\pgfqpoint{1.092000in}{0.890801in}}%
\pgfpathlineto{\pgfqpoint{1.098045in}{0.892871in}}%
\pgfpathlineto{\pgfqpoint{1.104090in}{0.896190in}}%
\pgfpathlineto{\pgfqpoint{1.128270in}{0.901735in}}%
\pgfpathlineto{\pgfqpoint{1.134315in}{0.901691in}}%
\pgfpathlineto{\pgfqpoint{1.140360in}{0.906813in}}%
\pgfpathlineto{\pgfqpoint{1.152450in}{0.908341in}}%
\pgfpathlineto{\pgfqpoint{1.158495in}{0.914026in}}%
\pgfpathlineto{\pgfqpoint{1.170585in}{0.913435in}}%
\pgfpathlineto{\pgfqpoint{1.176630in}{0.917438in}}%
\pgfpathlineto{\pgfqpoint{1.182675in}{0.919633in}}%
\pgfpathlineto{\pgfqpoint{1.188720in}{0.924820in}}%
\pgfpathlineto{\pgfqpoint{1.212900in}{0.926504in}}%
\pgfpathlineto{\pgfqpoint{1.237080in}{0.938092in}}%
\pgfpathlineto{\pgfqpoint{1.249170in}{0.947467in}}%
\pgfpathlineto{\pgfqpoint{1.273350in}{0.954198in}}%
\pgfpathlineto{\pgfqpoint{1.279395in}{0.952657in}}%
\pgfpathlineto{\pgfqpoint{1.285440in}{0.956972in}}%
\pgfpathlineto{\pgfqpoint{1.291485in}{0.956053in}}%
\pgfpathlineto{\pgfqpoint{1.297530in}{0.957566in}}%
\pgfpathlineto{\pgfqpoint{1.303575in}{0.957706in}}%
\pgfpathlineto{\pgfqpoint{1.309620in}{0.962267in}}%
\pgfpathlineto{\pgfqpoint{1.339845in}{0.966464in}}%
\pgfpathlineto{\pgfqpoint{1.357980in}{0.971433in}}%
\pgfpathlineto{\pgfqpoint{1.364025in}{0.975745in}}%
\pgfpathlineto{\pgfqpoint{1.376115in}{0.981944in}}%
\pgfpathlineto{\pgfqpoint{1.382160in}{0.983021in}}%
\pgfpathlineto{\pgfqpoint{1.406340in}{0.994419in}}%
\pgfpathlineto{\pgfqpoint{1.412385in}{0.998049in}}%
\pgfpathlineto{\pgfqpoint{1.418430in}{0.997380in}}%
\pgfpathlineto{\pgfqpoint{1.424475in}{0.999515in}}%
\pgfpathlineto{\pgfqpoint{1.436565in}{1.006458in}}%
\pgfpathlineto{\pgfqpoint{1.448655in}{1.006617in}}%
\pgfpathlineto{\pgfqpoint{1.454700in}{1.009189in}}%
\pgfpathlineto{\pgfqpoint{1.460745in}{1.007336in}}%
\pgfpathlineto{\pgfqpoint{1.466790in}{1.008974in}}%
\pgfpathlineto{\pgfqpoint{1.472835in}{1.007931in}}%
\pgfpathlineto{\pgfqpoint{1.478880in}{1.008634in}}%
\pgfpathlineto{\pgfqpoint{1.497015in}{1.020266in}}%
\pgfpathlineto{\pgfqpoint{1.503060in}{1.020655in}}%
\pgfpathlineto{\pgfqpoint{1.509105in}{1.018245in}}%
\pgfpathlineto{\pgfqpoint{1.515150in}{1.020378in}}%
\pgfpathlineto{\pgfqpoint{1.521195in}{1.024631in}}%
\pgfpathlineto{\pgfqpoint{1.527240in}{1.024273in}}%
\pgfpathlineto{\pgfqpoint{1.533285in}{1.027277in}}%
\pgfpathlineto{\pgfqpoint{1.545375in}{1.029242in}}%
\pgfpathlineto{\pgfqpoint{1.551420in}{1.036546in}}%
\pgfpathlineto{\pgfqpoint{1.563510in}{1.034587in}}%
\pgfpathlineto{\pgfqpoint{1.569555in}{1.037779in}}%
\pgfpathlineto{\pgfqpoint{1.575600in}{1.038666in}}%
\pgfpathlineto{\pgfqpoint{1.587690in}{1.042312in}}%
\pgfpathlineto{\pgfqpoint{1.593735in}{1.041209in}}%
\pgfpathlineto{\pgfqpoint{1.611870in}{1.050911in}}%
\pgfpathlineto{\pgfqpoint{1.617915in}{1.050802in}}%
\pgfpathlineto{\pgfqpoint{1.623960in}{1.052937in}}%
\pgfpathlineto{\pgfqpoint{1.636050in}{1.059946in}}%
\pgfpathlineto{\pgfqpoint{1.654185in}{1.064914in}}%
\pgfpathlineto{\pgfqpoint{1.660230in}{1.065556in}}%
\pgfpathlineto{\pgfqpoint{1.678365in}{1.063862in}}%
\pgfpathlineto{\pgfqpoint{1.696500in}{1.066714in}}%
\pgfpathlineto{\pgfqpoint{1.702545in}{1.064989in}}%
\pgfpathlineto{\pgfqpoint{1.708590in}{1.067309in}}%
\pgfpathlineto{\pgfqpoint{1.720680in}{1.069896in}}%
\pgfpathlineto{\pgfqpoint{1.726725in}{1.074955in}}%
\pgfpathlineto{\pgfqpoint{1.732770in}{1.077091in}}%
\pgfpathlineto{\pgfqpoint{1.738815in}{1.077854in}}%
\pgfpathlineto{\pgfqpoint{1.756950in}{1.082698in}}%
\pgfpathlineto{\pgfqpoint{1.762995in}{1.083215in}}%
\pgfpathlineto{\pgfqpoint{1.781130in}{1.090238in}}%
\pgfpathlineto{\pgfqpoint{1.799265in}{1.091973in}}%
\pgfpathlineto{\pgfqpoint{1.805310in}{1.092424in}}%
\pgfpathlineto{\pgfqpoint{1.811355in}{1.094681in}}%
\pgfpathlineto{\pgfqpoint{1.817400in}{1.093517in}}%
\pgfpathlineto{\pgfqpoint{1.823445in}{1.093595in}}%
\pgfpathlineto{\pgfqpoint{1.871805in}{1.103570in}}%
\pgfpathlineto{\pgfqpoint{1.877850in}{1.107508in}}%
\pgfpathlineto{\pgfqpoint{1.889940in}{1.110407in}}%
\pgfpathlineto{\pgfqpoint{1.902030in}{1.116481in}}%
\pgfpathlineto{\pgfqpoint{1.908075in}{1.116995in}}%
\pgfpathlineto{\pgfqpoint{1.914120in}{1.115394in}}%
\pgfpathlineto{\pgfqpoint{1.920165in}{1.118461in}}%
\pgfpathlineto{\pgfqpoint{1.926210in}{1.118604in}}%
\pgfpathlineto{\pgfqpoint{1.938300in}{1.124488in}}%
\pgfpathlineto{\pgfqpoint{1.944345in}{1.125067in}}%
\pgfpathlineto{\pgfqpoint{1.956435in}{1.129398in}}%
\pgfpathlineto{\pgfqpoint{1.962480in}{1.128355in}}%
\pgfpathlineto{\pgfqpoint{1.968525in}{1.131487in}}%
\pgfpathlineto{\pgfqpoint{1.980615in}{1.135320in}}%
\pgfpathlineto{\pgfqpoint{1.986660in}{1.137016in}}%
\pgfpathlineto{\pgfqpoint{1.992705in}{1.141577in}}%
\pgfpathlineto{\pgfqpoint{1.998750in}{1.142468in}}%
\pgfpathlineto{\pgfqpoint{2.004795in}{1.145908in}}%
\pgfpathlineto{\pgfqpoint{2.010840in}{1.145865in}}%
\pgfpathlineto{\pgfqpoint{2.016885in}{1.150861in}}%
\pgfpathlineto{\pgfqpoint{2.022930in}{1.149386in}}%
\pgfpathlineto{\pgfqpoint{2.028975in}{1.149650in}}%
\pgfpathlineto{\pgfqpoint{2.035020in}{1.151908in}}%
\pgfpathlineto{\pgfqpoint{2.041065in}{1.151241in}}%
\pgfpathlineto{\pgfqpoint{2.059200in}{1.152101in}}%
\pgfpathlineto{\pgfqpoint{2.065245in}{1.152368in}}%
\pgfpathlineto{\pgfqpoint{2.071290in}{1.153878in}}%
\pgfpathlineto{\pgfqpoint{2.077335in}{1.152586in}}%
\pgfpathlineto{\pgfqpoint{2.083380in}{1.155532in}}%
\pgfpathlineto{\pgfqpoint{2.089425in}{1.156606in}}%
\pgfpathlineto{\pgfqpoint{2.107560in}{1.164255in}}%
\pgfpathlineto{\pgfqpoint{2.113605in}{1.164022in}}%
\pgfpathlineto{\pgfqpoint{2.119650in}{1.165469in}}%
\pgfpathlineto{\pgfqpoint{2.125695in}{1.164118in}}%
\pgfpathlineto{\pgfqpoint{2.131740in}{1.168866in}}%
\pgfpathlineto{\pgfqpoint{2.143830in}{1.173321in}}%
\pgfpathlineto{\pgfqpoint{2.149875in}{1.172655in}}%
\pgfpathlineto{\pgfqpoint{2.155920in}{1.175286in}}%
\pgfpathlineto{\pgfqpoint{2.161965in}{1.175367in}}%
\pgfpathlineto{\pgfqpoint{2.168010in}{1.178371in}}%
\pgfpathlineto{\pgfqpoint{2.174055in}{1.179507in}}%
\pgfpathlineto{\pgfqpoint{2.180100in}{1.182390in}}%
\pgfpathlineto{\pgfqpoint{2.186145in}{1.183714in}}%
\pgfpathlineto{\pgfqpoint{2.192190in}{1.182860in}}%
\pgfpathlineto{\pgfqpoint{2.204280in}{1.186195in}}%
\pgfpathlineto{\pgfqpoint{2.216370in}{1.189090in}}%
\pgfpathlineto{\pgfqpoint{2.222415in}{1.188673in}}%
\pgfpathlineto{\pgfqpoint{2.228460in}{1.189560in}}%
\pgfpathlineto{\pgfqpoint{2.234505in}{1.188583in}}%
\pgfpathlineto{\pgfqpoint{2.240550in}{1.190902in}}%
\pgfpathlineto{\pgfqpoint{2.264730in}{1.193773in}}%
\pgfpathlineto{\pgfqpoint{2.270775in}{1.193477in}}%
\pgfpathlineto{\pgfqpoint{2.276820in}{1.194367in}}%
\pgfpathlineto{\pgfqpoint{2.294955in}{1.200706in}}%
\pgfpathlineto{\pgfqpoint{2.307045in}{1.201488in}}%
\pgfpathlineto{\pgfqpoint{2.325180in}{1.202036in}}%
\pgfpathlineto{\pgfqpoint{2.331225in}{1.199688in}}%
\pgfpathlineto{\pgfqpoint{2.337270in}{1.201385in}}%
\pgfpathlineto{\pgfqpoint{2.349360in}{1.202976in}}%
\pgfpathlineto{\pgfqpoint{2.379585in}{1.207419in}}%
\pgfpathlineto{\pgfqpoint{2.385630in}{1.210800in}}%
\pgfpathlineto{\pgfqpoint{2.391675in}{1.210566in}}%
\pgfpathlineto{\pgfqpoint{2.397720in}{1.208838in}}%
\pgfpathlineto{\pgfqpoint{2.403765in}{1.214523in}}%
\pgfpathlineto{\pgfqpoint{2.409810in}{1.216344in}}%
\pgfpathlineto{\pgfqpoint{2.421900in}{1.214262in}}%
\pgfpathlineto{\pgfqpoint{2.427945in}{1.211105in}}%
\pgfpathlineto{\pgfqpoint{2.433990in}{1.216102in}}%
\pgfpathlineto{\pgfqpoint{2.446080in}{1.222238in}}%
\pgfpathlineto{\pgfqpoint{2.452125in}{1.223437in}}%
\pgfpathlineto{\pgfqpoint{2.458170in}{1.223082in}}%
\pgfpathlineto{\pgfqpoint{2.464215in}{1.225090in}}%
\pgfpathlineto{\pgfqpoint{2.470260in}{1.222120in}}%
\pgfpathlineto{\pgfqpoint{2.476305in}{1.224501in}}%
\pgfpathlineto{\pgfqpoint{2.482350in}{1.223957in}}%
\pgfpathlineto{\pgfqpoint{2.488395in}{1.225657in}}%
\pgfpathlineto{\pgfqpoint{2.494440in}{1.225983in}}%
\pgfpathlineto{\pgfqpoint{2.500485in}{1.222453in}}%
\pgfpathlineto{\pgfqpoint{2.506530in}{1.223963in}}%
\pgfpathlineto{\pgfqpoint{2.512575in}{1.224168in}}%
\pgfpathlineto{\pgfqpoint{2.518620in}{1.222814in}}%
\pgfpathlineto{\pgfqpoint{2.524665in}{1.226133in}}%
\pgfpathlineto{\pgfqpoint{2.530710in}{1.225775in}}%
\pgfpathlineto{\pgfqpoint{2.536755in}{1.229340in}}%
\pgfpathlineto{\pgfqpoint{2.542800in}{1.229234in}}%
\pgfpathlineto{\pgfqpoint{2.548845in}{1.231553in}}%
\pgfpathlineto{\pgfqpoint{2.554890in}{1.229579in}}%
\pgfpathlineto{\pgfqpoint{2.566980in}{1.233474in}}%
\pgfpathlineto{\pgfqpoint{2.573025in}{1.231684in}}%
\pgfpathlineto{\pgfqpoint{2.579070in}{1.233817in}}%
\pgfpathlineto{\pgfqpoint{2.591160in}{1.234038in}}%
\pgfpathlineto{\pgfqpoint{2.597205in}{1.237917in}}%
\pgfpathlineto{\pgfqpoint{2.603250in}{1.240236in}}%
\pgfpathlineto{\pgfqpoint{2.615340in}{1.239337in}}%
\pgfpathlineto{\pgfqpoint{2.627430in}{1.240305in}}%
\pgfpathlineto{\pgfqpoint{2.633475in}{1.244617in}}%
\pgfpathlineto{\pgfqpoint{2.645565in}{1.245025in}}%
\pgfpathlineto{\pgfqpoint{2.657655in}{1.249355in}}%
\pgfpathlineto{\pgfqpoint{2.669745in}{1.246650in}}%
\pgfpathlineto{\pgfqpoint{2.693925in}{1.252634in}}%
\pgfpathlineto{\pgfqpoint{2.699970in}{1.249411in}}%
\pgfpathlineto{\pgfqpoint{2.706015in}{1.248558in}}%
\pgfpathlineto{\pgfqpoint{2.724150in}{1.251099in}}%
\pgfpathlineto{\pgfqpoint{2.736240in}{1.246712in}}%
\pgfpathlineto{\pgfqpoint{2.748330in}{1.250046in}}%
\pgfpathlineto{\pgfqpoint{2.754375in}{1.248817in}}%
\pgfpathlineto{\pgfqpoint{2.772510in}{1.253661in}}%
\pgfpathlineto{\pgfqpoint{2.778555in}{1.253057in}}%
\pgfpathlineto{\pgfqpoint{2.784600in}{1.254069in}}%
\pgfpathlineto{\pgfqpoint{2.790645in}{1.252531in}}%
\pgfpathlineto{\pgfqpoint{2.796690in}{1.253231in}}%
\pgfpathlineto{\pgfqpoint{2.808780in}{1.257002in}}%
\pgfpathlineto{\pgfqpoint{2.814825in}{1.257827in}}%
\pgfpathlineto{\pgfqpoint{2.820870in}{1.261581in}}%
\pgfpathlineto{\pgfqpoint{2.826915in}{1.263776in}}%
\pgfpathlineto{\pgfqpoint{2.832960in}{1.264231in}}%
\pgfpathlineto{\pgfqpoint{2.851095in}{1.259673in}}%
\pgfpathlineto{\pgfqpoint{2.869230in}{1.260968in}}%
\pgfpathlineto{\pgfqpoint{2.881320in}{1.258325in}}%
\pgfpathlineto{\pgfqpoint{2.887365in}{1.261083in}}%
\pgfpathlineto{\pgfqpoint{2.893410in}{1.261784in}}%
\pgfpathlineto{\pgfqpoint{2.899455in}{1.263979in}}%
\pgfpathlineto{\pgfqpoint{2.905500in}{1.263811in}}%
\pgfpathlineto{\pgfqpoint{2.923635in}{1.271332in}}%
\pgfpathlineto{\pgfqpoint{2.929680in}{1.274900in}}%
\pgfpathlineto{\pgfqpoint{2.935725in}{1.273608in}}%
\pgfpathlineto{\pgfqpoint{2.941770in}{1.269576in}}%
\pgfpathlineto{\pgfqpoint{2.947815in}{1.267665in}}%
\pgfpathlineto{\pgfqpoint{2.953860in}{1.269175in}}%
\pgfpathlineto{\pgfqpoint{2.959905in}{1.273615in}}%
\pgfpathlineto{\pgfqpoint{2.978040in}{1.274972in}}%
\pgfpathlineto{\pgfqpoint{2.996175in}{1.277637in}}%
\pgfpathlineto{\pgfqpoint{3.008265in}{1.277920in}}%
\pgfpathlineto{\pgfqpoint{3.014310in}{1.279620in}}%
\pgfpathlineto{\pgfqpoint{3.026400in}{1.278222in}}%
\pgfpathlineto{\pgfqpoint{3.032445in}{1.279732in}}%
\pgfpathlineto{\pgfqpoint{3.038490in}{1.287033in}}%
\pgfpathlineto{\pgfqpoint{3.044535in}{1.285931in}}%
\pgfpathlineto{\pgfqpoint{3.050580in}{1.288686in}}%
\pgfpathlineto{\pgfqpoint{3.074760in}{1.293300in}}%
\pgfpathlineto{\pgfqpoint{3.080805in}{1.295121in}}%
\pgfpathlineto{\pgfqpoint{3.092895in}{1.292976in}}%
\pgfpathlineto{\pgfqpoint{3.098940in}{1.293369in}}%
\pgfpathlineto{\pgfqpoint{3.104985in}{1.292575in}}%
\pgfpathlineto{\pgfqpoint{3.111030in}{1.293403in}}%
\pgfpathlineto{\pgfqpoint{3.117075in}{1.296968in}}%
\pgfpathlineto{\pgfqpoint{3.123120in}{1.301902in}}%
\pgfpathlineto{\pgfqpoint{3.135210in}{1.302248in}}%
\pgfpathlineto{\pgfqpoint{3.141255in}{1.303138in}}%
\pgfpathlineto{\pgfqpoint{3.153345in}{1.307344in}}%
\pgfpathlineto{\pgfqpoint{3.159390in}{1.306177in}}%
\pgfpathlineto{\pgfqpoint{3.171480in}{1.311691in}}%
\pgfpathlineto{\pgfqpoint{3.177525in}{1.311519in}}%
\pgfpathlineto{\pgfqpoint{3.183570in}{1.310231in}}%
\pgfpathlineto{\pgfqpoint{3.189615in}{1.313920in}}%
\pgfpathlineto{\pgfqpoint{3.195660in}{1.313378in}}%
\pgfpathlineto{\pgfqpoint{3.201705in}{1.316320in}}%
\pgfpathlineto{\pgfqpoint{3.207750in}{1.315903in}}%
\pgfpathlineto{\pgfqpoint{3.219840in}{1.317180in}}%
\pgfpathlineto{\pgfqpoint{3.225885in}{1.321557in}}%
\pgfpathlineto{\pgfqpoint{3.231930in}{1.322257in}}%
\pgfpathlineto{\pgfqpoint{3.237975in}{1.324206in}}%
\pgfpathlineto{\pgfqpoint{3.244020in}{1.324845in}}%
\pgfpathlineto{\pgfqpoint{3.250065in}{1.323929in}}%
\pgfpathlineto{\pgfqpoint{3.256110in}{1.325190in}}%
\pgfpathlineto{\pgfqpoint{3.262155in}{1.323649in}}%
\pgfpathlineto{\pgfqpoint{3.268200in}{1.324851in}}%
\pgfpathlineto{\pgfqpoint{3.274245in}{1.324555in}}%
\pgfpathlineto{\pgfqpoint{3.280290in}{1.321149in}}%
\pgfpathlineto{\pgfqpoint{3.286335in}{1.320231in}}%
\pgfpathlineto{\pgfqpoint{3.292380in}{1.322927in}}%
\pgfpathlineto{\pgfqpoint{3.298425in}{1.320701in}}%
\pgfpathlineto{\pgfqpoint{3.304470in}{1.322522in}}%
\pgfpathlineto{\pgfqpoint{3.322605in}{1.324817in}}%
\pgfpathlineto{\pgfqpoint{3.334695in}{1.322609in}}%
\pgfpathlineto{\pgfqpoint{3.340740in}{1.325738in}}%
\pgfpathlineto{\pgfqpoint{3.346785in}{1.326127in}}%
\pgfpathlineto{\pgfqpoint{3.352830in}{1.325150in}}%
\pgfpathlineto{\pgfqpoint{3.370965in}{1.326694in}}%
\pgfpathlineto{\pgfqpoint{3.377010in}{1.325280in}}%
\pgfpathlineto{\pgfqpoint{3.383055in}{1.327164in}}%
\pgfpathlineto{\pgfqpoint{3.389100in}{1.325813in}}%
\pgfpathlineto{\pgfqpoint{3.401190in}{1.327712in}}%
\pgfpathlineto{\pgfqpoint{3.407235in}{1.325240in}}%
\pgfpathlineto{\pgfqpoint{3.413280in}{1.327248in}}%
\pgfpathlineto{\pgfqpoint{3.419325in}{1.326395in}}%
\pgfpathlineto{\pgfqpoint{3.425370in}{1.331143in}}%
\pgfpathlineto{\pgfqpoint{3.431415in}{1.331286in}}%
\pgfpathlineto{\pgfqpoint{3.437460in}{1.333730in}}%
\pgfpathlineto{\pgfqpoint{3.443505in}{1.331878in}}%
\pgfpathlineto{\pgfqpoint{3.449550in}{1.334636in}}%
\pgfpathlineto{\pgfqpoint{3.461640in}{1.333799in}}%
\pgfpathlineto{\pgfqpoint{3.467685in}{1.331822in}}%
\pgfpathlineto{\pgfqpoint{3.473730in}{1.328353in}}%
\pgfpathlineto{\pgfqpoint{3.479775in}{1.332416in}}%
\pgfpathlineto{\pgfqpoint{3.485820in}{1.331996in}}%
\pgfpathlineto{\pgfqpoint{3.491865in}{1.329088in}}%
\pgfpathlineto{\pgfqpoint{3.503955in}{1.328936in}}%
\pgfpathlineto{\pgfqpoint{3.503955in}{1.328936in}}%
\pgfusepath{stroke}%
\end{pgfscope}%
\begin{pgfscope}%
\pgfpathrectangle{\pgfqpoint{0.487500in}{0.301292in}}{\pgfqpoint{3.022500in}{1.868008in}} %
\pgfusepath{clip}%
\pgfsetrectcap%
\pgfsetroundjoin%
\pgfsetlinewidth{1.505625pt}%
\definecolor{currentstroke}{rgb}{1.000000,0.000000,0.000000}%
\pgfsetstrokecolor{currentstroke}%
\pgfsetdash{}{0pt}%
\pgfpathmoveto{\pgfqpoint{0.487500in}{0.355294in}}%
\pgfpathlineto{\pgfqpoint{0.511680in}{0.459098in}}%
\pgfpathlineto{\pgfqpoint{0.529815in}{0.520358in}}%
\pgfpathlineto{\pgfqpoint{0.541905in}{0.556568in}}%
\pgfpathlineto{\pgfqpoint{0.547950in}{0.578005in}}%
\pgfpathlineto{\pgfqpoint{0.566085in}{0.623137in}}%
\pgfpathlineto{\pgfqpoint{0.578175in}{0.647829in}}%
\pgfpathlineto{\pgfqpoint{0.584220in}{0.663100in}}%
\pgfpathlineto{\pgfqpoint{0.596310in}{0.683495in}}%
\pgfpathlineto{\pgfqpoint{0.608400in}{0.708436in}}%
\pgfpathlineto{\pgfqpoint{0.644670in}{0.765326in}}%
\pgfpathlineto{\pgfqpoint{0.656760in}{0.780305in}}%
\pgfpathlineto{\pgfqpoint{0.662805in}{0.788166in}}%
\pgfpathlineto{\pgfqpoint{0.668850in}{0.792976in}}%
\pgfpathlineto{\pgfqpoint{0.686985in}{0.817002in}}%
\pgfpathlineto{\pgfqpoint{0.693030in}{0.828288in}}%
\pgfpathlineto{\pgfqpoint{0.699075in}{0.835218in}}%
\pgfpathlineto{\pgfqpoint{0.705120in}{0.844574in}}%
\pgfpathlineto{\pgfqpoint{0.717210in}{0.858307in}}%
\pgfpathlineto{\pgfqpoint{0.723255in}{0.861996in}}%
\pgfpathlineto{\pgfqpoint{0.753480in}{0.889667in}}%
\pgfpathlineto{\pgfqpoint{0.765570in}{0.904954in}}%
\pgfpathlineto{\pgfqpoint{0.771615in}{0.909580in}}%
\pgfpathlineto{\pgfqpoint{0.777660in}{0.916383in}}%
\pgfpathlineto{\pgfqpoint{0.789750in}{0.926069in}}%
\pgfpathlineto{\pgfqpoint{0.813930in}{0.950110in}}%
\pgfpathlineto{\pgfqpoint{0.826020in}{0.962162in}}%
\pgfpathlineto{\pgfqpoint{0.832065in}{0.967034in}}%
\pgfpathlineto{\pgfqpoint{0.838110in}{0.973528in}}%
\pgfpathlineto{\pgfqpoint{0.850200in}{0.979786in}}%
\pgfpathlineto{\pgfqpoint{0.874380in}{1.000029in}}%
\pgfpathlineto{\pgfqpoint{0.880425in}{1.007022in}}%
\pgfpathlineto{\pgfqpoint{0.898560in}{1.014357in}}%
\pgfpathlineto{\pgfqpoint{0.904605in}{1.019727in}}%
\pgfpathlineto{\pgfqpoint{0.910650in}{1.022486in}}%
\pgfpathlineto{\pgfqpoint{0.916695in}{1.027607in}}%
\pgfpathlineto{\pgfqpoint{0.934830in}{1.037810in}}%
\pgfpathlineto{\pgfqpoint{0.946920in}{1.048862in}}%
\pgfpathlineto{\pgfqpoint{0.965055in}{1.061057in}}%
\pgfpathlineto{\pgfqpoint{0.989235in}{1.083476in}}%
\pgfpathlineto{\pgfqpoint{0.995280in}{1.083993in}}%
\pgfpathlineto{\pgfqpoint{1.001325in}{1.085877in}}%
\pgfpathlineto{\pgfqpoint{1.031550in}{1.102088in}}%
\pgfpathlineto{\pgfqpoint{1.037595in}{1.106403in}}%
\pgfpathlineto{\pgfqpoint{1.043640in}{1.108847in}}%
\pgfpathlineto{\pgfqpoint{1.049685in}{1.114594in}}%
\pgfpathlineto{\pgfqpoint{1.055730in}{1.118595in}}%
\pgfpathlineto{\pgfqpoint{1.073865in}{1.126864in}}%
\pgfpathlineto{\pgfqpoint{1.085955in}{1.134930in}}%
\pgfpathlineto{\pgfqpoint{1.098045in}{1.140195in}}%
\pgfpathlineto{\pgfqpoint{1.116180in}{1.152017in}}%
\pgfpathlineto{\pgfqpoint{1.122225in}{1.151472in}}%
\pgfpathlineto{\pgfqpoint{1.134315in}{1.158729in}}%
\pgfpathlineto{\pgfqpoint{1.140360in}{1.158495in}}%
\pgfpathlineto{\pgfqpoint{1.152450in}{1.165005in}}%
\pgfpathlineto{\pgfqpoint{1.158495in}{1.166083in}}%
\pgfpathlineto{\pgfqpoint{1.164540in}{1.170083in}}%
\pgfpathlineto{\pgfqpoint{1.176630in}{1.171052in}}%
\pgfpathlineto{\pgfqpoint{1.182675in}{1.174430in}}%
\pgfpathlineto{\pgfqpoint{1.188720in}{1.174448in}}%
\pgfpathlineto{\pgfqpoint{1.194765in}{1.178200in}}%
\pgfpathlineto{\pgfqpoint{1.200810in}{1.185192in}}%
\pgfpathlineto{\pgfqpoint{1.212900in}{1.189022in}}%
\pgfpathlineto{\pgfqpoint{1.218945in}{1.190224in}}%
\pgfpathlineto{\pgfqpoint{1.224990in}{1.194286in}}%
\pgfpathlineto{\pgfqpoint{1.237080in}{1.196375in}}%
\pgfpathlineto{\pgfqpoint{1.249170in}{1.203384in}}%
\pgfpathlineto{\pgfqpoint{1.255215in}{1.205143in}}%
\pgfpathlineto{\pgfqpoint{1.267305in}{1.210781in}}%
\pgfpathlineto{\pgfqpoint{1.285440in}{1.217995in}}%
\pgfpathlineto{\pgfqpoint{1.291485in}{1.219629in}}%
\pgfpathlineto{\pgfqpoint{1.303575in}{1.226450in}}%
\pgfpathlineto{\pgfqpoint{1.321710in}{1.232167in}}%
\pgfpathlineto{\pgfqpoint{1.345890in}{1.245311in}}%
\pgfpathlineto{\pgfqpoint{1.357980in}{1.245719in}}%
\pgfpathlineto{\pgfqpoint{1.364025in}{1.248972in}}%
\pgfpathlineto{\pgfqpoint{1.376115in}{1.250501in}}%
\pgfpathlineto{\pgfqpoint{1.382160in}{1.255377in}}%
\pgfpathlineto{\pgfqpoint{1.388205in}{1.255330in}}%
\pgfpathlineto{\pgfqpoint{1.394250in}{1.258646in}}%
\pgfpathlineto{\pgfqpoint{1.406340in}{1.269763in}}%
\pgfpathlineto{\pgfqpoint{1.424475in}{1.278347in}}%
\pgfpathlineto{\pgfqpoint{1.430520in}{1.278985in}}%
\pgfpathlineto{\pgfqpoint{1.436565in}{1.282674in}}%
\pgfpathlineto{\pgfqpoint{1.442610in}{1.287737in}}%
\pgfpathlineto{\pgfqpoint{1.454700in}{1.286463in}}%
\pgfpathlineto{\pgfqpoint{1.466790in}{1.290483in}}%
\pgfpathlineto{\pgfqpoint{1.472835in}{1.294795in}}%
\pgfpathlineto{\pgfqpoint{1.478880in}{1.293942in}}%
\pgfpathlineto{\pgfqpoint{1.484925in}{1.297257in}}%
\pgfpathlineto{\pgfqpoint{1.490970in}{1.298954in}}%
\pgfpathlineto{\pgfqpoint{1.497015in}{1.298039in}}%
\pgfpathlineto{\pgfqpoint{1.503060in}{1.298490in}}%
\pgfpathlineto{\pgfqpoint{1.509105in}{1.297824in}}%
\pgfpathlineto{\pgfqpoint{1.515150in}{1.300953in}}%
\pgfpathlineto{\pgfqpoint{1.521195in}{1.300100in}}%
\pgfpathlineto{\pgfqpoint{1.527240in}{1.300863in}}%
\pgfpathlineto{\pgfqpoint{1.533285in}{1.305797in}}%
\pgfpathlineto{\pgfqpoint{1.539330in}{1.308244in}}%
\pgfpathlineto{\pgfqpoint{1.545375in}{1.308447in}}%
\pgfpathlineto{\pgfqpoint{1.551420in}{1.312575in}}%
\pgfpathlineto{\pgfqpoint{1.557465in}{1.314583in}}%
\pgfpathlineto{\pgfqpoint{1.569555in}{1.313870in}}%
\pgfpathlineto{\pgfqpoint{1.575600in}{1.316999in}}%
\pgfpathlineto{\pgfqpoint{1.593735in}{1.321411in}}%
\pgfpathlineto{\pgfqpoint{1.605825in}{1.330474in}}%
\pgfpathlineto{\pgfqpoint{1.611870in}{1.329617in}}%
\pgfpathlineto{\pgfqpoint{1.623960in}{1.333014in}}%
\pgfpathlineto{\pgfqpoint{1.636050in}{1.334916in}}%
\pgfpathlineto{\pgfqpoint{1.642095in}{1.338045in}}%
\pgfpathlineto{\pgfqpoint{1.648140in}{1.338437in}}%
\pgfpathlineto{\pgfqpoint{1.660230in}{1.344076in}}%
\pgfpathlineto{\pgfqpoint{1.666275in}{1.349197in}}%
\pgfpathlineto{\pgfqpoint{1.672320in}{1.351081in}}%
\pgfpathlineto{\pgfqpoint{1.690455in}{1.348518in}}%
\pgfpathlineto{\pgfqpoint{1.696500in}{1.352644in}}%
\pgfpathlineto{\pgfqpoint{1.708590in}{1.353363in}}%
\pgfpathlineto{\pgfqpoint{1.726725in}{1.364808in}}%
\pgfpathlineto{\pgfqpoint{1.738815in}{1.366461in}}%
\pgfpathlineto{\pgfqpoint{1.744860in}{1.367289in}}%
\pgfpathlineto{\pgfqpoint{1.750905in}{1.366557in}}%
\pgfpathlineto{\pgfqpoint{1.756950in}{1.371928in}}%
\pgfpathlineto{\pgfqpoint{1.781130in}{1.380215in}}%
\pgfpathlineto{\pgfqpoint{1.787175in}{1.383534in}}%
\pgfpathlineto{\pgfqpoint{1.799265in}{1.386246in}}%
\pgfpathlineto{\pgfqpoint{1.805310in}{1.386386in}}%
\pgfpathlineto{\pgfqpoint{1.811355in}{1.389702in}}%
\pgfpathlineto{\pgfqpoint{1.829490in}{1.390627in}}%
\pgfpathlineto{\pgfqpoint{1.835535in}{1.392448in}}%
\pgfpathlineto{\pgfqpoint{1.841580in}{1.391408in}}%
\pgfpathlineto{\pgfqpoint{1.859715in}{1.393824in}}%
\pgfpathlineto{\pgfqpoint{1.865760in}{1.397700in}}%
\pgfpathlineto{\pgfqpoint{1.883895in}{1.404540in}}%
\pgfpathlineto{\pgfqpoint{1.889940in}{1.408852in}}%
\pgfpathlineto{\pgfqpoint{1.895985in}{1.411047in}}%
\pgfpathlineto{\pgfqpoint{1.902030in}{1.411501in}}%
\pgfpathlineto{\pgfqpoint{1.908075in}{1.415378in}}%
\pgfpathlineto{\pgfqpoint{1.920165in}{1.419148in}}%
\pgfpathlineto{\pgfqpoint{1.926210in}{1.423152in}}%
\pgfpathlineto{\pgfqpoint{1.932255in}{1.421175in}}%
\pgfpathlineto{\pgfqpoint{1.938300in}{1.421252in}}%
\pgfpathlineto{\pgfqpoint{1.950390in}{1.424774in}}%
\pgfpathlineto{\pgfqpoint{1.962480in}{1.423936in}}%
\pgfpathlineto{\pgfqpoint{1.968525in}{1.424017in}}%
\pgfpathlineto{\pgfqpoint{1.980615in}{1.427352in}}%
\pgfpathlineto{\pgfqpoint{1.986660in}{1.432909in}}%
\pgfpathlineto{\pgfqpoint{1.998750in}{1.438485in}}%
\pgfpathlineto{\pgfqpoint{2.004795in}{1.438314in}}%
\pgfpathlineto{\pgfqpoint{2.010840in}{1.439702in}}%
\pgfpathlineto{\pgfqpoint{2.016885in}{1.438659in}}%
\pgfpathlineto{\pgfqpoint{2.028975in}{1.440997in}}%
\pgfpathlineto{\pgfqpoint{2.035020in}{1.440017in}}%
\pgfpathlineto{\pgfqpoint{2.041065in}{1.445203in}}%
\pgfpathlineto{\pgfqpoint{2.047110in}{1.447461in}}%
\pgfpathlineto{\pgfqpoint{2.053155in}{1.448040in}}%
\pgfpathlineto{\pgfqpoint{2.059200in}{1.449799in}}%
\pgfpathlineto{\pgfqpoint{2.065245in}{1.449942in}}%
\pgfpathlineto{\pgfqpoint{2.071290in}{1.448961in}}%
\pgfpathlineto{\pgfqpoint{2.083380in}{1.451984in}}%
\pgfpathlineto{\pgfqpoint{2.095470in}{1.457187in}}%
\pgfpathlineto{\pgfqpoint{2.107560in}{1.461144in}}%
\pgfpathlineto{\pgfqpoint{2.119650in}{1.465534in}}%
\pgfpathlineto{\pgfqpoint{2.125695in}{1.465490in}}%
\pgfpathlineto{\pgfqpoint{2.131740in}{1.467311in}}%
\pgfpathlineto{\pgfqpoint{2.137785in}{1.471066in}}%
\pgfpathlineto{\pgfqpoint{2.143830in}{1.471767in}}%
\pgfpathlineto{\pgfqpoint{2.149875in}{1.473715in}}%
\pgfpathlineto{\pgfqpoint{2.155920in}{1.472797in}}%
\pgfpathlineto{\pgfqpoint{2.168010in}{1.474824in}}%
\pgfpathlineto{\pgfqpoint{2.174055in}{1.476583in}}%
\pgfpathlineto{\pgfqpoint{2.180100in}{1.476602in}}%
\pgfpathlineto{\pgfqpoint{2.186145in}{1.479917in}}%
\pgfpathlineto{\pgfqpoint{2.198235in}{1.478333in}}%
\pgfpathlineto{\pgfqpoint{2.204280in}{1.482772in}}%
\pgfpathlineto{\pgfqpoint{2.210325in}{1.480858in}}%
\pgfpathlineto{\pgfqpoint{2.234505in}{1.491138in}}%
\pgfpathlineto{\pgfqpoint{2.240550in}{1.488165in}}%
\pgfpathlineto{\pgfqpoint{2.246595in}{1.488245in}}%
\pgfpathlineto{\pgfqpoint{2.270775in}{1.499519in}}%
\pgfpathlineto{\pgfqpoint{2.288910in}{1.500070in}}%
\pgfpathlineto{\pgfqpoint{2.294955in}{1.505627in}}%
\pgfpathlineto{\pgfqpoint{2.301000in}{1.507573in}}%
\pgfpathlineto{\pgfqpoint{2.325180in}{1.507330in}}%
\pgfpathlineto{\pgfqpoint{2.331225in}{1.506166in}}%
\pgfpathlineto{\pgfqpoint{2.337270in}{1.508859in}}%
\pgfpathlineto{\pgfqpoint{2.349360in}{1.511446in}}%
\pgfpathlineto{\pgfqpoint{2.367495in}{1.520338in}}%
\pgfpathlineto{\pgfqpoint{2.379585in}{1.521867in}}%
\pgfpathlineto{\pgfqpoint{2.391675in}{1.519410in}}%
\pgfpathlineto{\pgfqpoint{2.403765in}{1.522246in}}%
\pgfpathlineto{\pgfqpoint{2.409810in}{1.524628in}}%
\pgfpathlineto{\pgfqpoint{2.415855in}{1.523713in}}%
\pgfpathlineto{\pgfqpoint{2.421900in}{1.526593in}}%
\pgfpathlineto{\pgfqpoint{2.433990in}{1.523264in}}%
\pgfpathlineto{\pgfqpoint{2.452125in}{1.529665in}}%
\pgfpathlineto{\pgfqpoint{2.476305in}{1.532038in}}%
\pgfpathlineto{\pgfqpoint{2.482350in}{1.534793in}}%
\pgfpathlineto{\pgfqpoint{2.488395in}{1.533629in}}%
\pgfpathlineto{\pgfqpoint{2.500485in}{1.538956in}}%
\pgfpathlineto{\pgfqpoint{2.506530in}{1.536979in}}%
\pgfpathlineto{\pgfqpoint{2.512575in}{1.536561in}}%
\pgfpathlineto{\pgfqpoint{2.518620in}{1.534398in}}%
\pgfpathlineto{\pgfqpoint{2.524665in}{1.534977in}}%
\pgfpathlineto{\pgfqpoint{2.530710in}{1.533934in}}%
\pgfpathlineto{\pgfqpoint{2.536755in}{1.534385in}}%
\pgfpathlineto{\pgfqpoint{2.542800in}{1.535961in}}%
\pgfpathlineto{\pgfqpoint{2.554890in}{1.540603in}}%
\pgfpathlineto{\pgfqpoint{2.560935in}{1.542175in}}%
\pgfpathlineto{\pgfqpoint{2.573025in}{1.542832in}}%
\pgfpathlineto{\pgfqpoint{2.585115in}{1.546166in}}%
\pgfpathlineto{\pgfqpoint{2.591160in}{1.545559in}}%
\pgfpathlineto{\pgfqpoint{2.603250in}{1.549578in}}%
\pgfpathlineto{\pgfqpoint{2.609295in}{1.553084in}}%
\pgfpathlineto{\pgfqpoint{2.615340in}{1.553722in}}%
\pgfpathlineto{\pgfqpoint{2.621385in}{1.555917in}}%
\pgfpathlineto{\pgfqpoint{2.633475in}{1.562116in}}%
\pgfpathlineto{\pgfqpoint{2.645565in}{1.564952in}}%
\pgfpathlineto{\pgfqpoint{2.651610in}{1.561422in}}%
\pgfpathlineto{\pgfqpoint{2.657655in}{1.563181in}}%
\pgfpathlineto{\pgfqpoint{2.663700in}{1.561826in}}%
\pgfpathlineto{\pgfqpoint{2.669745in}{1.563090in}}%
\pgfpathlineto{\pgfqpoint{2.681835in}{1.562066in}}%
\pgfpathlineto{\pgfqpoint{2.699970in}{1.565105in}}%
\pgfpathlineto{\pgfqpoint{2.712060in}{1.566447in}}%
\pgfpathlineto{\pgfqpoint{2.724150in}{1.563990in}}%
\pgfpathlineto{\pgfqpoint{2.730195in}{1.564566in}}%
\pgfpathlineto{\pgfqpoint{2.736240in}{1.563153in}}%
\pgfpathlineto{\pgfqpoint{2.742285in}{1.565036in}}%
\pgfpathlineto{\pgfqpoint{2.748330in}{1.563249in}}%
\pgfpathlineto{\pgfqpoint{2.754375in}{1.565880in}}%
\pgfpathlineto{\pgfqpoint{2.766465in}{1.568654in}}%
\pgfpathlineto{\pgfqpoint{2.772510in}{1.570973in}}%
\pgfpathlineto{\pgfqpoint{2.778555in}{1.569311in}}%
\pgfpathlineto{\pgfqpoint{2.784600in}{1.566026in}}%
\pgfpathlineto{\pgfqpoint{2.790645in}{1.570092in}}%
\pgfpathlineto{\pgfqpoint{2.820870in}{1.581384in}}%
\pgfpathlineto{\pgfqpoint{2.826915in}{1.578660in}}%
\pgfpathlineto{\pgfqpoint{2.839005in}{1.583863in}}%
\pgfpathlineto{\pgfqpoint{2.845050in}{1.585248in}}%
\pgfpathlineto{\pgfqpoint{2.851095in}{1.588816in}}%
\pgfpathlineto{\pgfqpoint{2.869230in}{1.590360in}}%
\pgfpathlineto{\pgfqpoint{2.875275in}{1.592869in}}%
\pgfpathlineto{\pgfqpoint{2.899455in}{1.584404in}}%
\pgfpathlineto{\pgfqpoint{2.911545in}{1.589669in}}%
\pgfpathlineto{\pgfqpoint{2.917590in}{1.586761in}}%
\pgfpathlineto{\pgfqpoint{2.923635in}{1.586839in}}%
\pgfpathlineto{\pgfqpoint{2.929680in}{1.588975in}}%
\pgfpathlineto{\pgfqpoint{2.935725in}{1.588679in}}%
\pgfpathlineto{\pgfqpoint{2.947815in}{1.593259in}}%
\pgfpathlineto{\pgfqpoint{2.953860in}{1.592589in}}%
\pgfpathlineto{\pgfqpoint{2.959905in}{1.593355in}}%
\pgfpathlineto{\pgfqpoint{2.965950in}{1.592810in}}%
\pgfpathlineto{\pgfqpoint{2.971995in}{1.594759in}}%
\pgfpathlineto{\pgfqpoint{2.978040in}{1.594152in}}%
\pgfpathlineto{\pgfqpoint{2.984085in}{1.596534in}}%
\pgfpathlineto{\pgfqpoint{2.990130in}{1.593813in}}%
\pgfpathlineto{\pgfqpoint{3.002220in}{1.599264in}}%
\pgfpathlineto{\pgfqpoint{3.038490in}{1.605902in}}%
\pgfpathlineto{\pgfqpoint{3.044535in}{1.610093in}}%
\pgfpathlineto{\pgfqpoint{3.050580in}{1.609672in}}%
\pgfpathlineto{\pgfqpoint{3.062670in}{1.615248in}}%
\pgfpathlineto{\pgfqpoint{3.074760in}{1.615967in}}%
\pgfpathlineto{\pgfqpoint{3.098940in}{1.619211in}}%
\pgfpathlineto{\pgfqpoint{3.104985in}{1.618916in}}%
\pgfpathlineto{\pgfqpoint{3.111030in}{1.619744in}}%
\pgfpathlineto{\pgfqpoint{3.117075in}{1.622437in}}%
\pgfpathlineto{\pgfqpoint{3.123120in}{1.621705in}}%
\pgfpathlineto{\pgfqpoint{3.129165in}{1.623841in}}%
\pgfpathlineto{\pgfqpoint{3.135210in}{1.623670in}}%
\pgfpathlineto{\pgfqpoint{3.141255in}{1.620575in}}%
\pgfpathlineto{\pgfqpoint{3.147300in}{1.619532in}}%
\pgfpathlineto{\pgfqpoint{3.177525in}{1.628956in}}%
\pgfpathlineto{\pgfqpoint{3.183570in}{1.625052in}}%
\pgfpathlineto{\pgfqpoint{3.201705in}{1.634380in}}%
\pgfpathlineto{\pgfqpoint{3.207750in}{1.633340in}}%
\pgfpathlineto{\pgfqpoint{3.213795in}{1.630740in}}%
\pgfpathlineto{\pgfqpoint{3.219840in}{1.635613in}}%
\pgfpathlineto{\pgfqpoint{3.225885in}{1.634635in}}%
\pgfpathlineto{\pgfqpoint{3.237975in}{1.638717in}}%
\pgfpathlineto{\pgfqpoint{3.256110in}{1.638704in}}%
\pgfpathlineto{\pgfqpoint{3.262155in}{1.643763in}}%
\pgfpathlineto{\pgfqpoint{3.268200in}{1.645276in}}%
\pgfpathlineto{\pgfqpoint{3.274245in}{1.648405in}}%
\pgfpathlineto{\pgfqpoint{3.292380in}{1.647773in}}%
\pgfpathlineto{\pgfqpoint{3.298425in}{1.646544in}}%
\pgfpathlineto{\pgfqpoint{3.304470in}{1.646808in}}%
\pgfpathlineto{\pgfqpoint{3.310515in}{1.645332in}}%
\pgfpathlineto{\pgfqpoint{3.316560in}{1.640989in}}%
\pgfpathlineto{\pgfqpoint{3.322605in}{1.645491in}}%
\pgfpathlineto{\pgfqpoint{3.334695in}{1.647954in}}%
\pgfpathlineto{\pgfqpoint{3.340740in}{1.647658in}}%
\pgfpathlineto{\pgfqpoint{3.352830in}{1.652176in}}%
\pgfpathlineto{\pgfqpoint{3.364920in}{1.653268in}}%
\pgfpathlineto{\pgfqpoint{3.389100in}{1.658692in}}%
\pgfpathlineto{\pgfqpoint{3.395145in}{1.656279in}}%
\pgfpathlineto{\pgfqpoint{3.401190in}{1.658349in}}%
\pgfpathlineto{\pgfqpoint{3.407235in}{1.658244in}}%
\pgfpathlineto{\pgfqpoint{3.413280in}{1.656765in}}%
\pgfpathlineto{\pgfqpoint{3.425370in}{1.658169in}}%
\pgfpathlineto{\pgfqpoint{3.431415in}{1.657565in}}%
\pgfpathlineto{\pgfqpoint{3.437460in}{1.658826in}}%
\pgfpathlineto{\pgfqpoint{3.443505in}{1.655915in}}%
\pgfpathlineto{\pgfqpoint{3.449550in}{1.657054in}}%
\pgfpathlineto{\pgfqpoint{3.455595in}{1.661179in}}%
\pgfpathlineto{\pgfqpoint{3.467685in}{1.660342in}}%
\pgfpathlineto{\pgfqpoint{3.473730in}{1.661980in}}%
\pgfpathlineto{\pgfqpoint{3.479775in}{1.658757in}}%
\pgfpathlineto{\pgfqpoint{3.491865in}{1.664458in}}%
\pgfpathlineto{\pgfqpoint{3.503955in}{1.665924in}}%
\pgfpathlineto{\pgfqpoint{3.503955in}{1.665924in}}%
\pgfusepath{stroke}%
\end{pgfscope}%
\begin{pgfscope}%
\pgfpathrectangle{\pgfqpoint{0.487500in}{0.301292in}}{\pgfqpoint{3.022500in}{1.868008in}} %
\pgfusepath{clip}%
\pgfsetrectcap%
\pgfsetroundjoin%
\pgfsetlinewidth{1.505625pt}%
\definecolor{currentstroke}{rgb}{0.000000,0.750000,0.750000}%
\pgfsetstrokecolor{currentstroke}%
\pgfsetdash{}{0pt}%
\pgfpathmoveto{\pgfqpoint{0.487500in}{0.355294in}}%
\pgfpathlineto{\pgfqpoint{0.493545in}{0.411289in}}%
\pgfpathlineto{\pgfqpoint{0.499590in}{0.433846in}}%
\pgfpathlineto{\pgfqpoint{0.511680in}{0.467255in}}%
\pgfpathlineto{\pgfqpoint{0.523770in}{0.493379in}}%
\pgfpathlineto{\pgfqpoint{0.529815in}{0.502923in}}%
\pgfpathlineto{\pgfqpoint{0.535860in}{0.517386in}}%
\pgfpathlineto{\pgfqpoint{0.541905in}{0.524127in}}%
\pgfpathlineto{\pgfqpoint{0.547950in}{0.535539in}}%
\pgfpathlineto{\pgfqpoint{0.553995in}{0.544024in}}%
\pgfpathlineto{\pgfqpoint{0.566085in}{0.568467in}}%
\pgfpathlineto{\pgfqpoint{0.578175in}{0.585562in}}%
\pgfpathlineto{\pgfqpoint{0.584220in}{0.590436in}}%
\pgfpathlineto{\pgfqpoint{0.596310in}{0.602051in}}%
\pgfpathlineto{\pgfqpoint{0.602355in}{0.608232in}}%
\pgfpathlineto{\pgfqpoint{0.620490in}{0.632380in}}%
\pgfpathlineto{\pgfqpoint{0.626535in}{0.637128in}}%
\pgfpathlineto{\pgfqpoint{0.632580in}{0.645740in}}%
\pgfpathlineto{\pgfqpoint{0.656760in}{0.665236in}}%
\pgfpathlineto{\pgfqpoint{0.662805in}{0.667804in}}%
\pgfpathlineto{\pgfqpoint{0.674895in}{0.676742in}}%
\pgfpathlineto{\pgfqpoint{0.680940in}{0.684541in}}%
\pgfpathlineto{\pgfqpoint{0.686985in}{0.689915in}}%
\pgfpathlineto{\pgfqpoint{0.693030in}{0.697589in}}%
\pgfpathlineto{\pgfqpoint{0.705120in}{0.707462in}}%
\pgfpathlineto{\pgfqpoint{0.717210in}{0.714408in}}%
\pgfpathlineto{\pgfqpoint{0.723255in}{0.716167in}}%
\pgfpathlineto{\pgfqpoint{0.729300in}{0.721105in}}%
\pgfpathlineto{\pgfqpoint{0.735345in}{0.727783in}}%
\pgfpathlineto{\pgfqpoint{0.741390in}{0.731039in}}%
\pgfpathlineto{\pgfqpoint{0.765570in}{0.749536in}}%
\pgfpathlineto{\pgfqpoint{0.771615in}{0.751547in}}%
\pgfpathlineto{\pgfqpoint{0.777660in}{0.754800in}}%
\pgfpathlineto{\pgfqpoint{0.789750in}{0.765607in}}%
\pgfpathlineto{\pgfqpoint{0.801840in}{0.770871in}}%
\pgfpathlineto{\pgfqpoint{0.807885in}{0.774249in}}%
\pgfpathlineto{\pgfqpoint{0.838110in}{0.799243in}}%
\pgfpathlineto{\pgfqpoint{0.850200in}{0.805314in}}%
\pgfpathlineto{\pgfqpoint{0.862290in}{0.814564in}}%
\pgfpathlineto{\pgfqpoint{0.874380in}{0.820140in}}%
\pgfpathlineto{\pgfqpoint{0.880425in}{0.825389in}}%
\pgfpathlineto{\pgfqpoint{0.886470in}{0.828020in}}%
\pgfpathlineto{\pgfqpoint{0.898560in}{0.836460in}}%
\pgfpathlineto{\pgfqpoint{0.910650in}{0.840542in}}%
\pgfpathlineto{\pgfqpoint{0.916695in}{0.846348in}}%
\pgfpathlineto{\pgfqpoint{0.922740in}{0.847612in}}%
\pgfpathlineto{\pgfqpoint{0.952965in}{0.862329in}}%
\pgfpathlineto{\pgfqpoint{0.959010in}{0.868571in}}%
\pgfpathlineto{\pgfqpoint{0.965055in}{0.871018in}}%
\pgfpathlineto{\pgfqpoint{0.977145in}{0.879147in}}%
\pgfpathlineto{\pgfqpoint{0.983190in}{0.885825in}}%
\pgfpathlineto{\pgfqpoint{0.989235in}{0.886650in}}%
\pgfpathlineto{\pgfqpoint{0.995280in}{0.888599in}}%
\pgfpathlineto{\pgfqpoint{1.001325in}{0.893908in}}%
\pgfpathlineto{\pgfqpoint{1.031550in}{0.910991in}}%
\pgfpathlineto{\pgfqpoint{1.043640in}{0.912395in}}%
\pgfpathlineto{\pgfqpoint{1.055730in}{0.920026in}}%
\pgfpathlineto{\pgfqpoint{1.067820in}{0.925539in}}%
\pgfpathlineto{\pgfqpoint{1.079910in}{0.931427in}}%
\pgfpathlineto{\pgfqpoint{1.085955in}{0.935116in}}%
\pgfpathlineto{\pgfqpoint{1.092000in}{0.935820in}}%
\pgfpathlineto{\pgfqpoint{1.110135in}{0.944524in}}%
\pgfpathlineto{\pgfqpoint{1.116180in}{0.944481in}}%
\pgfpathlineto{\pgfqpoint{1.128270in}{0.952918in}}%
\pgfpathlineto{\pgfqpoint{1.134315in}{0.955178in}}%
\pgfpathlineto{\pgfqpoint{1.146405in}{0.957454in}}%
\pgfpathlineto{\pgfqpoint{1.152450in}{0.961144in}}%
\pgfpathlineto{\pgfqpoint{1.158495in}{0.960602in}}%
\pgfpathlineto{\pgfqpoint{1.164540in}{0.963482in}}%
\pgfpathlineto{\pgfqpoint{1.170585in}{0.964244in}}%
\pgfpathlineto{\pgfqpoint{1.176630in}{0.970801in}}%
\pgfpathlineto{\pgfqpoint{1.182675in}{0.973868in}}%
\pgfpathlineto{\pgfqpoint{1.188720in}{0.973513in}}%
\pgfpathlineto{\pgfqpoint{1.194765in}{0.977576in}}%
\pgfpathlineto{\pgfqpoint{1.200810in}{0.979462in}}%
\pgfpathlineto{\pgfqpoint{1.206855in}{0.983588in}}%
\pgfpathlineto{\pgfqpoint{1.218945in}{0.985552in}}%
\pgfpathlineto{\pgfqpoint{1.224990in}{0.989366in}}%
\pgfpathlineto{\pgfqpoint{1.231035in}{0.990443in}}%
\pgfpathlineto{\pgfqpoint{1.237080in}{0.989400in}}%
\pgfpathlineto{\pgfqpoint{1.249170in}{0.998961in}}%
\pgfpathlineto{\pgfqpoint{1.255215in}{1.005764in}}%
\pgfpathlineto{\pgfqpoint{1.273350in}{1.011857in}}%
\pgfpathlineto{\pgfqpoint{1.279395in}{1.010067in}}%
\pgfpathlineto{\pgfqpoint{1.285440in}{1.012950in}}%
\pgfpathlineto{\pgfqpoint{1.291485in}{1.018133in}}%
\pgfpathlineto{\pgfqpoint{1.297530in}{1.018463in}}%
\pgfpathlineto{\pgfqpoint{1.303575in}{1.020783in}}%
\pgfpathlineto{\pgfqpoint{1.315665in}{1.021377in}}%
\pgfpathlineto{\pgfqpoint{1.321710in}{1.026001in}}%
\pgfpathlineto{\pgfqpoint{1.327755in}{1.024400in}}%
\pgfpathlineto{\pgfqpoint{1.333800in}{1.029646in}}%
\pgfpathlineto{\pgfqpoint{1.339845in}{1.028046in}}%
\pgfpathlineto{\pgfqpoint{1.345890in}{1.029992in}}%
\pgfpathlineto{\pgfqpoint{1.351935in}{1.029634in}}%
\pgfpathlineto{\pgfqpoint{1.364025in}{1.032782in}}%
\pgfpathlineto{\pgfqpoint{1.370070in}{1.037968in}}%
\pgfpathlineto{\pgfqpoint{1.394250in}{1.052044in}}%
\pgfpathlineto{\pgfqpoint{1.400295in}{1.056608in}}%
\pgfpathlineto{\pgfqpoint{1.406340in}{1.058865in}}%
\pgfpathlineto{\pgfqpoint{1.412385in}{1.062869in}}%
\pgfpathlineto{\pgfqpoint{1.418430in}{1.063943in}}%
\pgfpathlineto{\pgfqpoint{1.424475in}{1.063837in}}%
\pgfpathlineto{\pgfqpoint{1.430520in}{1.070453in}}%
\pgfpathlineto{\pgfqpoint{1.442610in}{1.077710in}}%
\pgfpathlineto{\pgfqpoint{1.454700in}{1.078865in}}%
\pgfpathlineto{\pgfqpoint{1.472835in}{1.075740in}}%
\pgfpathlineto{\pgfqpoint{1.478880in}{1.077377in}}%
\pgfpathlineto{\pgfqpoint{1.484925in}{1.076708in}}%
\pgfpathlineto{\pgfqpoint{1.497015in}{1.081288in}}%
\pgfpathlineto{\pgfqpoint{1.503060in}{1.085101in}}%
\pgfpathlineto{\pgfqpoint{1.509105in}{1.083875in}}%
\pgfpathlineto{\pgfqpoint{1.533285in}{1.092471in}}%
\pgfpathlineto{\pgfqpoint{1.545375in}{1.097175in}}%
\pgfpathlineto{\pgfqpoint{1.551420in}{1.097567in}}%
\pgfpathlineto{\pgfqpoint{1.563510in}{1.101649in}}%
\pgfpathlineto{\pgfqpoint{1.575600in}{1.101991in}}%
\pgfpathlineto{\pgfqpoint{1.587690in}{1.107505in}}%
\pgfpathlineto{\pgfqpoint{1.599780in}{1.113330in}}%
\pgfpathlineto{\pgfqpoint{1.605825in}{1.116089in}}%
\pgfpathlineto{\pgfqpoint{1.611870in}{1.115232in}}%
\pgfpathlineto{\pgfqpoint{1.617915in}{1.115808in}}%
\pgfpathlineto{\pgfqpoint{1.648140in}{1.125236in}}%
\pgfpathlineto{\pgfqpoint{1.666275in}{1.131948in}}%
\pgfpathlineto{\pgfqpoint{1.672320in}{1.127792in}}%
\pgfpathlineto{\pgfqpoint{1.678365in}{1.128059in}}%
\pgfpathlineto{\pgfqpoint{1.684410in}{1.127203in}}%
\pgfpathlineto{\pgfqpoint{1.690455in}{1.128280in}}%
\pgfpathlineto{\pgfqpoint{1.696500in}{1.130849in}}%
\pgfpathlineto{\pgfqpoint{1.702545in}{1.131303in}}%
\pgfpathlineto{\pgfqpoint{1.738815in}{1.145849in}}%
\pgfpathlineto{\pgfqpoint{1.744860in}{1.144249in}}%
\pgfpathlineto{\pgfqpoint{1.750905in}{1.148187in}}%
\pgfpathlineto{\pgfqpoint{1.756950in}{1.149946in}}%
\pgfpathlineto{\pgfqpoint{1.762995in}{1.153327in}}%
\pgfpathlineto{\pgfqpoint{1.769040in}{1.153094in}}%
\pgfpathlineto{\pgfqpoint{1.775085in}{1.155914in}}%
\pgfpathlineto{\pgfqpoint{1.787175in}{1.159436in}}%
\pgfpathlineto{\pgfqpoint{1.793220in}{1.162378in}}%
\pgfpathlineto{\pgfqpoint{1.799265in}{1.163580in}}%
\pgfpathlineto{\pgfqpoint{1.811355in}{1.167347in}}%
\pgfpathlineto{\pgfqpoint{1.817400in}{1.167365in}}%
\pgfpathlineto{\pgfqpoint{1.823445in}{1.169000in}}%
\pgfpathlineto{\pgfqpoint{1.829490in}{1.171945in}}%
\pgfpathlineto{\pgfqpoint{1.841580in}{1.170983in}}%
\pgfpathlineto{\pgfqpoint{1.847625in}{1.173240in}}%
\pgfpathlineto{\pgfqpoint{1.853670in}{1.173629in}}%
\pgfpathlineto{\pgfqpoint{1.859715in}{1.177135in}}%
\pgfpathlineto{\pgfqpoint{1.865760in}{1.178271in}}%
\pgfpathlineto{\pgfqpoint{1.889940in}{1.189548in}}%
\pgfpathlineto{\pgfqpoint{1.895985in}{1.196600in}}%
\pgfpathlineto{\pgfqpoint{1.902030in}{1.199420in}}%
\pgfpathlineto{\pgfqpoint{1.908075in}{1.200681in}}%
\pgfpathlineto{\pgfqpoint{1.920165in}{1.199097in}}%
\pgfpathlineto{\pgfqpoint{1.932255in}{1.202369in}}%
\pgfpathlineto{\pgfqpoint{1.938300in}{1.204688in}}%
\pgfpathlineto{\pgfqpoint{1.944345in}{1.204894in}}%
\pgfpathlineto{\pgfqpoint{1.950390in}{1.203664in}}%
\pgfpathlineto{\pgfqpoint{1.974570in}{1.209212in}}%
\pgfpathlineto{\pgfqpoint{1.980615in}{1.214336in}}%
\pgfpathlineto{\pgfqpoint{1.992705in}{1.217917in}}%
\pgfpathlineto{\pgfqpoint{2.004795in}{1.218325in}}%
\pgfpathlineto{\pgfqpoint{2.010840in}{1.219713in}}%
\pgfpathlineto{\pgfqpoint{2.016885in}{1.223216in}}%
\pgfpathlineto{\pgfqpoint{2.022930in}{1.225289in}}%
\pgfpathlineto{\pgfqpoint{2.028975in}{1.225492in}}%
\pgfpathlineto{\pgfqpoint{2.059200in}{1.232300in}}%
\pgfpathlineto{\pgfqpoint{2.065245in}{1.235308in}}%
\pgfpathlineto{\pgfqpoint{2.071290in}{1.235074in}}%
\pgfpathlineto{\pgfqpoint{2.077335in}{1.238639in}}%
\pgfpathlineto{\pgfqpoint{2.089425in}{1.238922in}}%
\pgfpathlineto{\pgfqpoint{2.101515in}{1.244498in}}%
\pgfpathlineto{\pgfqpoint{2.107560in}{1.244766in}}%
\pgfpathlineto{\pgfqpoint{2.113605in}{1.243723in}}%
\pgfpathlineto{\pgfqpoint{2.125695in}{1.245750in}}%
\pgfpathlineto{\pgfqpoint{2.131740in}{1.249190in}}%
\pgfpathlineto{\pgfqpoint{2.149875in}{1.254910in}}%
\pgfpathlineto{\pgfqpoint{2.155920in}{1.255112in}}%
\pgfpathlineto{\pgfqpoint{2.161965in}{1.253325in}}%
\pgfpathlineto{\pgfqpoint{2.168010in}{1.253963in}}%
\pgfpathlineto{\pgfqpoint{2.180100in}{1.262777in}}%
\pgfpathlineto{\pgfqpoint{2.186145in}{1.262606in}}%
\pgfpathlineto{\pgfqpoint{2.192190in}{1.264430in}}%
\pgfpathlineto{\pgfqpoint{2.198235in}{1.263014in}}%
\pgfpathlineto{\pgfqpoint{2.204280in}{1.266581in}}%
\pgfpathlineto{\pgfqpoint{2.210325in}{1.267406in}}%
\pgfpathlineto{\pgfqpoint{2.216370in}{1.269477in}}%
\pgfpathlineto{\pgfqpoint{2.222415in}{1.274290in}}%
\pgfpathlineto{\pgfqpoint{2.228460in}{1.275426in}}%
\pgfpathlineto{\pgfqpoint{2.234505in}{1.274698in}}%
\pgfpathlineto{\pgfqpoint{2.240550in}{1.276519in}}%
\pgfpathlineto{\pgfqpoint{2.246595in}{1.277161in}}%
\pgfpathlineto{\pgfqpoint{2.258685in}{1.275760in}}%
\pgfpathlineto{\pgfqpoint{2.270775in}{1.280775in}}%
\pgfpathlineto{\pgfqpoint{2.288910in}{1.283941in}}%
\pgfpathlineto{\pgfqpoint{2.294955in}{1.286448in}}%
\pgfpathlineto{\pgfqpoint{2.301000in}{1.286775in}}%
\pgfpathlineto{\pgfqpoint{2.307045in}{1.289658in}}%
\pgfpathlineto{\pgfqpoint{2.313090in}{1.287992in}}%
\pgfpathlineto{\pgfqpoint{2.319135in}{1.291062in}}%
\pgfpathlineto{\pgfqpoint{2.331225in}{1.291656in}}%
\pgfpathlineto{\pgfqpoint{2.349360in}{1.300548in}}%
\pgfpathlineto{\pgfqpoint{2.361450in}{1.303571in}}%
\pgfpathlineto{\pgfqpoint{2.367495in}{1.306326in}}%
\pgfpathlineto{\pgfqpoint{2.373540in}{1.307528in}}%
\pgfpathlineto{\pgfqpoint{2.379585in}{1.306049in}}%
\pgfpathlineto{\pgfqpoint{2.385630in}{1.307313in}}%
\pgfpathlineto{\pgfqpoint{2.391675in}{1.310069in}}%
\pgfpathlineto{\pgfqpoint{2.397720in}{1.309337in}}%
\pgfpathlineto{\pgfqpoint{2.403765in}{1.310165in}}%
\pgfpathlineto{\pgfqpoint{2.415855in}{1.314620in}}%
\pgfpathlineto{\pgfqpoint{2.427945in}{1.314156in}}%
\pgfpathlineto{\pgfqpoint{2.440035in}{1.317861in}}%
\pgfpathlineto{\pgfqpoint{2.446080in}{1.321305in}}%
\pgfpathlineto{\pgfqpoint{2.452125in}{1.322254in}}%
\pgfpathlineto{\pgfqpoint{2.458170in}{1.324701in}}%
\pgfpathlineto{\pgfqpoint{2.464215in}{1.325028in}}%
\pgfpathlineto{\pgfqpoint{2.470260in}{1.323926in}}%
\pgfpathlineto{\pgfqpoint{2.482350in}{1.326448in}}%
\pgfpathlineto{\pgfqpoint{2.488395in}{1.325035in}}%
\pgfpathlineto{\pgfqpoint{2.506530in}{1.327949in}}%
\pgfpathlineto{\pgfqpoint{2.512575in}{1.325539in}}%
\pgfpathlineto{\pgfqpoint{2.524665in}{1.324826in}}%
\pgfpathlineto{\pgfqpoint{2.530710in}{1.326647in}}%
\pgfpathlineto{\pgfqpoint{2.536755in}{1.323674in}}%
\pgfpathlineto{\pgfqpoint{2.542800in}{1.327553in}}%
\pgfpathlineto{\pgfqpoint{2.548845in}{1.325639in}}%
\pgfpathlineto{\pgfqpoint{2.560935in}{1.327292in}}%
\pgfpathlineto{\pgfqpoint{2.566980in}{1.329926in}}%
\pgfpathlineto{\pgfqpoint{2.579070in}{1.329583in}}%
\pgfpathlineto{\pgfqpoint{2.585115in}{1.330411in}}%
\pgfpathlineto{\pgfqpoint{2.591160in}{1.333104in}}%
\pgfpathlineto{\pgfqpoint{2.597205in}{1.333995in}}%
\pgfpathlineto{\pgfqpoint{2.609295in}{1.342995in}}%
\pgfpathlineto{\pgfqpoint{2.615340in}{1.341952in}}%
\pgfpathlineto{\pgfqpoint{2.621385in}{1.345766in}}%
\pgfpathlineto{\pgfqpoint{2.645565in}{1.350318in}}%
\pgfpathlineto{\pgfqpoint{2.651610in}{1.349963in}}%
\pgfpathlineto{\pgfqpoint{2.663700in}{1.353979in}}%
\pgfpathlineto{\pgfqpoint{2.669745in}{1.352379in}}%
\pgfpathlineto{\pgfqpoint{2.675790in}{1.354761in}}%
\pgfpathlineto{\pgfqpoint{2.681835in}{1.354655in}}%
\pgfpathlineto{\pgfqpoint{2.687880in}{1.357037in}}%
\pgfpathlineto{\pgfqpoint{2.699970in}{1.359312in}}%
\pgfpathlineto{\pgfqpoint{2.718105in}{1.354630in}}%
\pgfpathlineto{\pgfqpoint{2.730195in}{1.355785in}}%
\pgfpathlineto{\pgfqpoint{2.736240in}{1.353873in}}%
\pgfpathlineto{\pgfqpoint{2.742285in}{1.355944in}}%
\pgfpathlineto{\pgfqpoint{2.748330in}{1.359574in}}%
\pgfpathlineto{\pgfqpoint{2.754375in}{1.361333in}}%
\pgfpathlineto{\pgfqpoint{2.760420in}{1.360041in}}%
\pgfpathlineto{\pgfqpoint{2.778555in}{1.364951in}}%
\pgfpathlineto{\pgfqpoint{2.784600in}{1.364593in}}%
\pgfpathlineto{\pgfqpoint{2.796690in}{1.368487in}}%
\pgfpathlineto{\pgfqpoint{2.802735in}{1.370869in}}%
\pgfpathlineto{\pgfqpoint{2.808780in}{1.368709in}}%
\pgfpathlineto{\pgfqpoint{2.820870in}{1.375530in}}%
\pgfpathlineto{\pgfqpoint{2.826915in}{1.374861in}}%
\pgfpathlineto{\pgfqpoint{2.839005in}{1.380561in}}%
\pgfpathlineto{\pgfqpoint{2.845050in}{1.386118in}}%
\pgfpathlineto{\pgfqpoint{2.851095in}{1.384954in}}%
\pgfpathlineto{\pgfqpoint{2.857140in}{1.382603in}}%
\pgfpathlineto{\pgfqpoint{2.863185in}{1.383743in}}%
\pgfpathlineto{\pgfqpoint{2.869230in}{1.381953in}}%
\pgfpathlineto{\pgfqpoint{2.875275in}{1.381971in}}%
\pgfpathlineto{\pgfqpoint{2.893410in}{1.377600in}}%
\pgfpathlineto{\pgfqpoint{2.911545in}{1.380265in}}%
\pgfpathlineto{\pgfqpoint{2.917590in}{1.378354in}}%
\pgfpathlineto{\pgfqpoint{2.923635in}{1.380486in}}%
\pgfpathlineto{\pgfqpoint{2.929680in}{1.379571in}}%
\pgfpathlineto{\pgfqpoint{2.941770in}{1.382155in}}%
\pgfpathlineto{\pgfqpoint{2.947815in}{1.385972in}}%
\pgfpathlineto{\pgfqpoint{2.953860in}{1.385303in}}%
\pgfpathlineto{\pgfqpoint{2.965950in}{1.380791in}}%
\pgfpathlineto{\pgfqpoint{2.984085in}{1.381402in}}%
\pgfpathlineto{\pgfqpoint{2.990130in}{1.385468in}}%
\pgfpathlineto{\pgfqpoint{2.996175in}{1.386915in}}%
\pgfpathlineto{\pgfqpoint{3.002220in}{1.384568in}}%
\pgfpathlineto{\pgfqpoint{3.014310in}{1.387778in}}%
\pgfpathlineto{\pgfqpoint{3.026400in}{1.393042in}}%
\pgfpathlineto{\pgfqpoint{3.032445in}{1.393618in}}%
\pgfpathlineto{\pgfqpoint{3.038490in}{1.396125in}}%
\pgfpathlineto{\pgfqpoint{3.044535in}{1.396953in}}%
\pgfpathlineto{\pgfqpoint{3.050580in}{1.399833in}}%
\pgfpathlineto{\pgfqpoint{3.056625in}{1.400349in}}%
\pgfpathlineto{\pgfqpoint{3.062670in}{1.403354in}}%
\pgfpathlineto{\pgfqpoint{3.068715in}{1.403435in}}%
\pgfpathlineto{\pgfqpoint{3.074760in}{1.405318in}}%
\pgfpathlineto{\pgfqpoint{3.080805in}{1.405272in}}%
\pgfpathlineto{\pgfqpoint{3.086850in}{1.409649in}}%
\pgfpathlineto{\pgfqpoint{3.092895in}{1.407734in}}%
\pgfpathlineto{\pgfqpoint{3.098940in}{1.407317in}}%
\pgfpathlineto{\pgfqpoint{3.104985in}{1.412314in}}%
\pgfpathlineto{\pgfqpoint{3.111030in}{1.412893in}}%
\pgfpathlineto{\pgfqpoint{3.123120in}{1.418092in}}%
\pgfpathlineto{\pgfqpoint{3.129165in}{1.419232in}}%
\pgfpathlineto{\pgfqpoint{3.135210in}{1.416695in}}%
\pgfpathlineto{\pgfqpoint{3.153345in}{1.420110in}}%
\pgfpathlineto{\pgfqpoint{3.159390in}{1.423675in}}%
\pgfpathlineto{\pgfqpoint{3.171480in}{1.425017in}}%
\pgfpathlineto{\pgfqpoint{3.177525in}{1.427025in}}%
\pgfpathlineto{\pgfqpoint{3.189615in}{1.429114in}}%
\pgfpathlineto{\pgfqpoint{3.201705in}{1.426969in}}%
\pgfpathlineto{\pgfqpoint{3.213795in}{1.433292in}}%
\pgfpathlineto{\pgfqpoint{3.219840in}{1.436109in}}%
\pgfpathlineto{\pgfqpoint{3.225885in}{1.434136in}}%
\pgfpathlineto{\pgfqpoint{3.244020in}{1.438170in}}%
\pgfpathlineto{\pgfqpoint{3.250065in}{1.437691in}}%
\pgfpathlineto{\pgfqpoint{3.256110in}{1.440509in}}%
\pgfpathlineto{\pgfqpoint{3.268200in}{1.447641in}}%
\pgfpathlineto{\pgfqpoint{3.286335in}{1.448687in}}%
\pgfpathlineto{\pgfqpoint{3.292380in}{1.448395in}}%
\pgfpathlineto{\pgfqpoint{3.304470in}{1.449048in}}%
\pgfpathlineto{\pgfqpoint{3.310515in}{1.445767in}}%
\pgfpathlineto{\pgfqpoint{3.316560in}{1.449083in}}%
\pgfpathlineto{\pgfqpoint{3.328650in}{1.446315in}}%
\pgfpathlineto{\pgfqpoint{3.340740in}{1.444730in}}%
\pgfpathlineto{\pgfqpoint{3.346785in}{1.448793in}}%
\pgfpathlineto{\pgfqpoint{3.352830in}{1.449995in}}%
\pgfpathlineto{\pgfqpoint{3.364920in}{1.450091in}}%
\pgfpathlineto{\pgfqpoint{3.370965in}{1.451477in}}%
\pgfpathlineto{\pgfqpoint{3.377010in}{1.451558in}}%
\pgfpathlineto{\pgfqpoint{3.383055in}{1.456057in}}%
\pgfpathlineto{\pgfqpoint{3.389100in}{1.457009in}}%
\pgfpathlineto{\pgfqpoint{3.395145in}{1.461197in}}%
\pgfpathlineto{\pgfqpoint{3.401190in}{1.460963in}}%
\pgfpathlineto{\pgfqpoint{3.407235in}{1.459176in}}%
\pgfpathlineto{\pgfqpoint{3.419325in}{1.459958in}}%
\pgfpathlineto{\pgfqpoint{3.425370in}{1.461468in}}%
\pgfpathlineto{\pgfqpoint{3.431415in}{1.460241in}}%
\pgfpathlineto{\pgfqpoint{3.437460in}{1.457205in}}%
\pgfpathlineto{\pgfqpoint{3.443505in}{1.459400in}}%
\pgfpathlineto{\pgfqpoint{3.449550in}{1.458859in}}%
\pgfpathlineto{\pgfqpoint{3.479775in}{1.464049in}}%
\pgfpathlineto{\pgfqpoint{3.485820in}{1.465372in}}%
\pgfpathlineto{\pgfqpoint{3.491865in}{1.468192in}}%
\pgfpathlineto{\pgfqpoint{3.497910in}{1.467959in}}%
\pgfpathlineto{\pgfqpoint{3.503955in}{1.468974in}}%
\pgfpathlineto{\pgfqpoint{3.503955in}{1.468974in}}%
\pgfusepath{stroke}%
\end{pgfscope}%
\begin{pgfscope}%
\pgfpathrectangle{\pgfqpoint{0.487500in}{0.301292in}}{\pgfqpoint{3.022500in}{1.868008in}} %
\pgfusepath{clip}%
\pgfsetrectcap%
\pgfsetroundjoin%
\pgfsetlinewidth{1.505625pt}%
\definecolor{currentstroke}{rgb}{0.750000,0.000000,0.750000}%
\pgfsetstrokecolor{currentstroke}%
\pgfsetdash{}{0pt}%
\pgfpathmoveto{\pgfqpoint{0.487500in}{0.355294in}}%
\pgfpathlineto{\pgfqpoint{0.493545in}{0.411289in}}%
\pgfpathlineto{\pgfqpoint{0.499590in}{0.433846in}}%
\pgfpathlineto{\pgfqpoint{0.511680in}{0.467255in}}%
\pgfpathlineto{\pgfqpoint{0.529815in}{0.507966in}}%
\pgfpathlineto{\pgfqpoint{0.535860in}{0.523924in}}%
\pgfpathlineto{\pgfqpoint{0.553995in}{0.556042in}}%
\pgfpathlineto{\pgfqpoint{0.572130in}{0.594324in}}%
\pgfpathlineto{\pgfqpoint{0.590265in}{0.619096in}}%
\pgfpathlineto{\pgfqpoint{0.602355in}{0.636129in}}%
\pgfpathlineto{\pgfqpoint{0.608400in}{0.646294in}}%
\pgfpathlineto{\pgfqpoint{0.620490in}{0.662642in}}%
\pgfpathlineto{\pgfqpoint{0.632580in}{0.680858in}}%
\pgfpathlineto{\pgfqpoint{0.644670in}{0.693657in}}%
\pgfpathlineto{\pgfqpoint{0.650715in}{0.703636in}}%
\pgfpathlineto{\pgfqpoint{0.668850in}{0.716139in}}%
\pgfpathlineto{\pgfqpoint{0.686985in}{0.740413in}}%
\pgfpathlineto{\pgfqpoint{0.705120in}{0.762194in}}%
\pgfpathlineto{\pgfqpoint{0.717210in}{0.771257in}}%
\pgfpathlineto{\pgfqpoint{0.723255in}{0.774822in}}%
\pgfpathlineto{\pgfqpoint{0.759525in}{0.812783in}}%
\pgfpathlineto{\pgfqpoint{0.765570in}{0.816099in}}%
\pgfpathlineto{\pgfqpoint{0.795795in}{0.851678in}}%
\pgfpathlineto{\pgfqpoint{0.801840in}{0.854932in}}%
\pgfpathlineto{\pgfqpoint{0.826020in}{0.881526in}}%
\pgfpathlineto{\pgfqpoint{0.832065in}{0.886336in}}%
\pgfpathlineto{\pgfqpoint{0.838110in}{0.893204in}}%
\pgfpathlineto{\pgfqpoint{0.850200in}{0.899337in}}%
\pgfpathlineto{\pgfqpoint{0.862290in}{0.910206in}}%
\pgfpathlineto{\pgfqpoint{0.886470in}{0.928581in}}%
\pgfpathlineto{\pgfqpoint{0.892515in}{0.932519in}}%
\pgfpathlineto{\pgfqpoint{0.904605in}{0.943575in}}%
\pgfpathlineto{\pgfqpoint{0.922740in}{0.953279in}}%
\pgfpathlineto{\pgfqpoint{0.928785in}{0.955038in}}%
\pgfpathlineto{\pgfqpoint{0.934830in}{0.959229in}}%
\pgfpathlineto{\pgfqpoint{0.940875in}{0.966530in}}%
\pgfpathlineto{\pgfqpoint{0.946920in}{0.970842in}}%
\pgfpathlineto{\pgfqpoint{0.965055in}{0.991443in}}%
\pgfpathlineto{\pgfqpoint{0.971100in}{0.994447in}}%
\pgfpathlineto{\pgfqpoint{0.983190in}{1.006374in}}%
\pgfpathlineto{\pgfqpoint{0.989235in}{1.012741in}}%
\pgfpathlineto{\pgfqpoint{1.019460in}{1.035618in}}%
\pgfpathlineto{\pgfqpoint{1.025505in}{1.035696in}}%
\pgfpathlineto{\pgfqpoint{1.037595in}{1.044198in}}%
\pgfpathlineto{\pgfqpoint{1.049685in}{1.052950in}}%
\pgfpathlineto{\pgfqpoint{1.055730in}{1.059192in}}%
\pgfpathlineto{\pgfqpoint{1.061775in}{1.060830in}}%
\pgfpathlineto{\pgfqpoint{1.067820in}{1.065889in}}%
\pgfpathlineto{\pgfqpoint{1.073865in}{1.069018in}}%
\pgfpathlineto{\pgfqpoint{1.085955in}{1.079264in}}%
\pgfpathlineto{\pgfqpoint{1.098045in}{1.084404in}}%
\pgfpathlineto{\pgfqpoint{1.110135in}{1.094276in}}%
\pgfpathlineto{\pgfqpoint{1.116180in}{1.099961in}}%
\pgfpathlineto{\pgfqpoint{1.128270in}{1.107527in}}%
\pgfpathlineto{\pgfqpoint{1.134315in}{1.111904in}}%
\pgfpathlineto{\pgfqpoint{1.146405in}{1.115114in}}%
\pgfpathlineto{\pgfqpoint{1.152450in}{1.120298in}}%
\pgfpathlineto{\pgfqpoint{1.170585in}{1.130123in}}%
\pgfpathlineto{\pgfqpoint{1.176630in}{1.133878in}}%
\pgfpathlineto{\pgfqpoint{1.182675in}{1.139996in}}%
\pgfpathlineto{\pgfqpoint{1.188720in}{1.140015in}}%
\pgfpathlineto{\pgfqpoint{1.224990in}{1.161347in}}%
\pgfpathlineto{\pgfqpoint{1.237080in}{1.163187in}}%
\pgfpathlineto{\pgfqpoint{1.249170in}{1.170444in}}%
\pgfpathlineto{\pgfqpoint{1.255215in}{1.177932in}}%
\pgfpathlineto{\pgfqpoint{1.273350in}{1.186391in}}%
\pgfpathlineto{\pgfqpoint{1.279395in}{1.188897in}}%
\pgfpathlineto{\pgfqpoint{1.285440in}{1.193212in}}%
\pgfpathlineto{\pgfqpoint{1.291485in}{1.198894in}}%
\pgfpathlineto{\pgfqpoint{1.303575in}{1.207957in}}%
\pgfpathlineto{\pgfqpoint{1.309620in}{1.213390in}}%
\pgfpathlineto{\pgfqpoint{1.321710in}{1.216787in}}%
\pgfpathlineto{\pgfqpoint{1.327755in}{1.221662in}}%
\pgfpathlineto{\pgfqpoint{1.345890in}{1.229682in}}%
\pgfpathlineto{\pgfqpoint{1.351935in}{1.228888in}}%
\pgfpathlineto{\pgfqpoint{1.364025in}{1.234713in}}%
\pgfpathlineto{\pgfqpoint{1.370070in}{1.237721in}}%
\pgfpathlineto{\pgfqpoint{1.376115in}{1.242780in}}%
\pgfpathlineto{\pgfqpoint{1.388205in}{1.247546in}}%
\pgfpathlineto{\pgfqpoint{1.394250in}{1.249928in}}%
\pgfpathlineto{\pgfqpoint{1.400295in}{1.257045in}}%
\pgfpathlineto{\pgfqpoint{1.418430in}{1.271541in}}%
\pgfpathlineto{\pgfqpoint{1.424475in}{1.271311in}}%
\pgfpathlineto{\pgfqpoint{1.430520in}{1.274066in}}%
\pgfpathlineto{\pgfqpoint{1.436565in}{1.280184in}}%
\pgfpathlineto{\pgfqpoint{1.460745in}{1.298123in}}%
\pgfpathlineto{\pgfqpoint{1.472835in}{1.299278in}}%
\pgfpathlineto{\pgfqpoint{1.478880in}{1.298487in}}%
\pgfpathlineto{\pgfqpoint{1.497015in}{1.306881in}}%
\pgfpathlineto{\pgfqpoint{1.509105in}{1.316815in}}%
\pgfpathlineto{\pgfqpoint{1.521195in}{1.317970in}}%
\pgfpathlineto{\pgfqpoint{1.527240in}{1.321286in}}%
\pgfpathlineto{\pgfqpoint{1.533285in}{1.326594in}}%
\pgfpathlineto{\pgfqpoint{1.545375in}{1.330676in}}%
\pgfpathlineto{\pgfqpoint{1.551420in}{1.336174in}}%
\pgfpathlineto{\pgfqpoint{1.569555in}{1.340707in}}%
\pgfpathlineto{\pgfqpoint{1.599780in}{1.347703in}}%
\pgfpathlineto{\pgfqpoint{1.611870in}{1.353901in}}%
\pgfpathlineto{\pgfqpoint{1.617915in}{1.356096in}}%
\pgfpathlineto{\pgfqpoint{1.623960in}{1.359540in}}%
\pgfpathlineto{\pgfqpoint{1.630005in}{1.359306in}}%
\pgfpathlineto{\pgfqpoint{1.648140in}{1.369446in}}%
\pgfpathlineto{\pgfqpoint{1.654185in}{1.376187in}}%
\pgfpathlineto{\pgfqpoint{1.660230in}{1.376890in}}%
\pgfpathlineto{\pgfqpoint{1.690455in}{1.388930in}}%
\pgfpathlineto{\pgfqpoint{1.708590in}{1.389042in}}%
\pgfpathlineto{\pgfqpoint{1.720680in}{1.399661in}}%
\pgfpathlineto{\pgfqpoint{1.726725in}{1.403226in}}%
\pgfpathlineto{\pgfqpoint{1.732770in}{1.402249in}}%
\pgfpathlineto{\pgfqpoint{1.738815in}{1.406312in}}%
\pgfpathlineto{\pgfqpoint{1.756950in}{1.413833in}}%
\pgfpathlineto{\pgfqpoint{1.762995in}{1.418896in}}%
\pgfpathlineto{\pgfqpoint{1.769040in}{1.420094in}}%
\pgfpathlineto{\pgfqpoint{1.775085in}{1.424223in}}%
\pgfpathlineto{\pgfqpoint{1.781130in}{1.435259in}}%
\pgfpathlineto{\pgfqpoint{1.799265in}{1.446458in}}%
\pgfpathlineto{\pgfqpoint{1.817400in}{1.450929in}}%
\pgfpathlineto{\pgfqpoint{1.823445in}{1.450197in}}%
\pgfpathlineto{\pgfqpoint{1.829490in}{1.454201in}}%
\pgfpathlineto{\pgfqpoint{1.835535in}{1.456334in}}%
\pgfpathlineto{\pgfqpoint{1.847625in}{1.457925in}}%
\pgfpathlineto{\pgfqpoint{1.853670in}{1.459123in}}%
\pgfpathlineto{\pgfqpoint{1.859715in}{1.457834in}}%
\pgfpathlineto{\pgfqpoint{1.865760in}{1.460590in}}%
\pgfpathlineto{\pgfqpoint{1.877850in}{1.471147in}}%
\pgfpathlineto{\pgfqpoint{1.889940in}{1.476038in}}%
\pgfpathlineto{\pgfqpoint{1.908075in}{1.491156in}}%
\pgfpathlineto{\pgfqpoint{1.926210in}{1.497498in}}%
\pgfpathlineto{\pgfqpoint{1.932255in}{1.496144in}}%
\pgfpathlineto{\pgfqpoint{1.944345in}{1.504024in}}%
\pgfpathlineto{\pgfqpoint{1.950390in}{1.505347in}}%
\pgfpathlineto{\pgfqpoint{1.962480in}{1.504074in}}%
\pgfpathlineto{\pgfqpoint{1.968525in}{1.510506in}}%
\pgfpathlineto{\pgfqpoint{1.980615in}{1.512346in}}%
\pgfpathlineto{\pgfqpoint{1.986660in}{1.516035in}}%
\pgfpathlineto{\pgfqpoint{1.992705in}{1.517857in}}%
\pgfpathlineto{\pgfqpoint{2.004795in}{1.525861in}}%
\pgfpathlineto{\pgfqpoint{2.010840in}{1.525631in}}%
\pgfpathlineto{\pgfqpoint{2.016885in}{1.528573in}}%
\pgfpathlineto{\pgfqpoint{2.022930in}{1.535129in}}%
\pgfpathlineto{\pgfqpoint{2.028975in}{1.536888in}}%
\pgfpathlineto{\pgfqpoint{2.047110in}{1.545593in}}%
\pgfpathlineto{\pgfqpoint{2.059200in}{1.548990in}}%
\pgfpathlineto{\pgfqpoint{2.071290in}{1.555375in}}%
\pgfpathlineto{\pgfqpoint{2.083380in}{1.558523in}}%
\pgfpathlineto{\pgfqpoint{2.095470in}{1.559367in}}%
\pgfpathlineto{\pgfqpoint{2.101515in}{1.562745in}}%
\pgfpathlineto{\pgfqpoint{2.107560in}{1.563884in}}%
\pgfpathlineto{\pgfqpoint{2.113605in}{1.569192in}}%
\pgfpathlineto{\pgfqpoint{2.119650in}{1.569519in}}%
\pgfpathlineto{\pgfqpoint{2.125695in}{1.572340in}}%
\pgfpathlineto{\pgfqpoint{2.131740in}{1.573601in}}%
\pgfpathlineto{\pgfqpoint{2.155920in}{1.585438in}}%
\pgfpathlineto{\pgfqpoint{2.161965in}{1.583277in}}%
\pgfpathlineto{\pgfqpoint{2.174055in}{1.585301in}}%
\pgfpathlineto{\pgfqpoint{2.180100in}{1.590737in}}%
\pgfpathlineto{\pgfqpoint{2.192190in}{1.594943in}}%
\pgfpathlineto{\pgfqpoint{2.198235in}{1.594211in}}%
\pgfpathlineto{\pgfqpoint{2.204280in}{1.595475in}}%
\pgfpathlineto{\pgfqpoint{2.210325in}{1.595055in}}%
\pgfpathlineto{\pgfqpoint{2.222415in}{1.602748in}}%
\pgfpathlineto{\pgfqpoint{2.228460in}{1.606375in}}%
\pgfpathlineto{\pgfqpoint{2.234505in}{1.603530in}}%
\pgfpathlineto{\pgfqpoint{2.264730in}{1.612954in}}%
\pgfpathlineto{\pgfqpoint{2.270775in}{1.616332in}}%
\pgfpathlineto{\pgfqpoint{2.276820in}{1.614109in}}%
\pgfpathlineto{\pgfqpoint{2.288910in}{1.618688in}}%
\pgfpathlineto{\pgfqpoint{2.301000in}{1.622891in}}%
\pgfpathlineto{\pgfqpoint{2.307045in}{1.622972in}}%
\pgfpathlineto{\pgfqpoint{2.313090in}{1.620186in}}%
\pgfpathlineto{\pgfqpoint{2.319135in}{1.621388in}}%
\pgfpathlineto{\pgfqpoint{2.325180in}{1.626198in}}%
\pgfpathlineto{\pgfqpoint{2.331225in}{1.628832in}}%
\pgfpathlineto{\pgfqpoint{2.337270in}{1.628972in}}%
\pgfpathlineto{\pgfqpoint{2.343315in}{1.631606in}}%
\pgfpathlineto{\pgfqpoint{2.349360in}{1.637474in}}%
\pgfpathlineto{\pgfqpoint{2.355405in}{1.639047in}}%
\pgfpathlineto{\pgfqpoint{2.361450in}{1.644109in}}%
\pgfpathlineto{\pgfqpoint{2.367495in}{1.644374in}}%
\pgfpathlineto{\pgfqpoint{2.373540in}{1.649374in}}%
\pgfpathlineto{\pgfqpoint{2.379585in}{1.651693in}}%
\pgfpathlineto{\pgfqpoint{2.385630in}{1.652210in}}%
\pgfpathlineto{\pgfqpoint{2.397720in}{1.658281in}}%
\pgfpathlineto{\pgfqpoint{2.409810in}{1.664915in}}%
\pgfpathlineto{\pgfqpoint{2.415855in}{1.668857in}}%
\pgfpathlineto{\pgfqpoint{2.427945in}{1.674184in}}%
\pgfpathlineto{\pgfqpoint{2.433990in}{1.674698in}}%
\pgfpathlineto{\pgfqpoint{2.446080in}{1.682702in}}%
\pgfpathlineto{\pgfqpoint{2.452125in}{1.683714in}}%
\pgfpathlineto{\pgfqpoint{2.464215in}{1.689290in}}%
\pgfpathlineto{\pgfqpoint{2.476305in}{1.690009in}}%
\pgfpathlineto{\pgfqpoint{2.482350in}{1.693138in}}%
\pgfpathlineto{\pgfqpoint{2.488395in}{1.698138in}}%
\pgfpathlineto{\pgfqpoint{2.494440in}{1.698652in}}%
\pgfpathlineto{\pgfqpoint{2.512575in}{1.703250in}}%
\pgfpathlineto{\pgfqpoint{2.518620in}{1.701273in}}%
\pgfpathlineto{\pgfqpoint{2.536755in}{1.706740in}}%
\pgfpathlineto{\pgfqpoint{2.542800in}{1.709187in}}%
\pgfpathlineto{\pgfqpoint{2.554890in}{1.712148in}}%
\pgfpathlineto{\pgfqpoint{2.560935in}{1.710296in}}%
\pgfpathlineto{\pgfqpoint{2.566980in}{1.713303in}}%
\pgfpathlineto{\pgfqpoint{2.573025in}{1.714315in}}%
\pgfpathlineto{\pgfqpoint{2.579070in}{1.712151in}}%
\pgfpathlineto{\pgfqpoint{2.585115in}{1.712855in}}%
\pgfpathlineto{\pgfqpoint{2.597205in}{1.719738in}}%
\pgfpathlineto{\pgfqpoint{2.603250in}{1.720003in}}%
\pgfpathlineto{\pgfqpoint{2.615340in}{1.727572in}}%
\pgfpathlineto{\pgfqpoint{2.621385in}{1.729268in}}%
\pgfpathlineto{\pgfqpoint{2.627430in}{1.734517in}}%
\pgfpathlineto{\pgfqpoint{2.639520in}{1.740965in}}%
\pgfpathlineto{\pgfqpoint{2.645565in}{1.743471in}}%
\pgfpathlineto{\pgfqpoint{2.651610in}{1.741809in}}%
\pgfpathlineto{\pgfqpoint{2.657655in}{1.741949in}}%
\pgfpathlineto{\pgfqpoint{2.663700in}{1.746448in}}%
\pgfpathlineto{\pgfqpoint{2.675790in}{1.748786in}}%
\pgfpathlineto{\pgfqpoint{2.681835in}{1.752291in}}%
\pgfpathlineto{\pgfqpoint{2.687880in}{1.751871in}}%
\pgfpathlineto{\pgfqpoint{2.693925in}{1.754630in}}%
\pgfpathlineto{\pgfqpoint{2.699970in}{1.755392in}}%
\pgfpathlineto{\pgfqpoint{2.706015in}{1.758400in}}%
\pgfpathlineto{\pgfqpoint{2.712060in}{1.759661in}}%
\pgfpathlineto{\pgfqpoint{2.724150in}{1.759446in}}%
\pgfpathlineto{\pgfqpoint{2.730195in}{1.761018in}}%
\pgfpathlineto{\pgfqpoint{2.736240in}{1.766018in}}%
\pgfpathlineto{\pgfqpoint{2.742285in}{1.762298in}}%
\pgfpathlineto{\pgfqpoint{2.748330in}{1.761881in}}%
\pgfpathlineto{\pgfqpoint{2.766465in}{1.766600in}}%
\pgfpathlineto{\pgfqpoint{2.772510in}{1.767924in}}%
\pgfpathlineto{\pgfqpoint{2.784600in}{1.776177in}}%
\pgfpathlineto{\pgfqpoint{2.790645in}{1.777067in}}%
\pgfpathlineto{\pgfqpoint{2.796690in}{1.779574in}}%
\pgfpathlineto{\pgfqpoint{2.802735in}{1.779838in}}%
\pgfpathlineto{\pgfqpoint{2.808780in}{1.781476in}}%
\pgfpathlineto{\pgfqpoint{2.857140in}{1.809879in}}%
\pgfpathlineto{\pgfqpoint{2.863185in}{1.811205in}}%
\pgfpathlineto{\pgfqpoint{2.875275in}{1.815785in}}%
\pgfpathlineto{\pgfqpoint{2.881320in}{1.815302in}}%
\pgfpathlineto{\pgfqpoint{2.899455in}{1.822326in}}%
\pgfpathlineto{\pgfqpoint{2.911545in}{1.822236in}}%
\pgfpathlineto{\pgfqpoint{2.923635in}{1.827750in}}%
\pgfpathlineto{\pgfqpoint{2.929680in}{1.826648in}}%
\pgfpathlineto{\pgfqpoint{2.935725in}{1.829839in}}%
\pgfpathlineto{\pgfqpoint{2.941770in}{1.829481in}}%
\pgfpathlineto{\pgfqpoint{2.959905in}{1.838189in}}%
\pgfpathlineto{\pgfqpoint{2.971995in}{1.839468in}}%
\pgfpathlineto{\pgfqpoint{2.984085in}{1.836511in}}%
\pgfpathlineto{\pgfqpoint{3.008265in}{1.848783in}}%
\pgfpathlineto{\pgfqpoint{3.020355in}{1.851869in}}%
\pgfpathlineto{\pgfqpoint{3.026400in}{1.855188in}}%
\pgfpathlineto{\pgfqpoint{3.032445in}{1.857196in}}%
\pgfpathlineto{\pgfqpoint{3.038490in}{1.862006in}}%
\pgfpathlineto{\pgfqpoint{3.044535in}{1.864079in}}%
\pgfpathlineto{\pgfqpoint{3.056625in}{1.865546in}}%
\pgfpathlineto{\pgfqpoint{3.062670in}{1.870231in}}%
\pgfpathlineto{\pgfqpoint{3.068715in}{1.873488in}}%
\pgfpathlineto{\pgfqpoint{3.074760in}{1.872632in}}%
\pgfpathlineto{\pgfqpoint{3.092895in}{1.884388in}}%
\pgfpathlineto{\pgfqpoint{3.098940in}{1.883721in}}%
\pgfpathlineto{\pgfqpoint{3.129165in}{1.892834in}}%
\pgfpathlineto{\pgfqpoint{3.135210in}{1.896523in}}%
\pgfpathlineto{\pgfqpoint{3.153345in}{1.897884in}}%
\pgfpathlineto{\pgfqpoint{3.165435in}{1.901527in}}%
\pgfpathlineto{\pgfqpoint{3.177525in}{1.902931in}}%
\pgfpathlineto{\pgfqpoint{3.183570in}{1.907869in}}%
\pgfpathlineto{\pgfqpoint{3.195660in}{1.909522in}}%
\pgfpathlineto{\pgfqpoint{3.231930in}{1.922697in}}%
\pgfpathlineto{\pgfqpoint{3.244020in}{1.927153in}}%
\pgfpathlineto{\pgfqpoint{3.250065in}{1.930720in}}%
\pgfpathlineto{\pgfqpoint{3.268200in}{1.937246in}}%
\pgfpathlineto{\pgfqpoint{3.280290in}{1.946683in}}%
\pgfpathlineto{\pgfqpoint{3.292380in}{1.950142in}}%
\pgfpathlineto{\pgfqpoint{3.298425in}{1.950655in}}%
\pgfpathlineto{\pgfqpoint{3.304470in}{1.948678in}}%
\pgfpathlineto{\pgfqpoint{3.310515in}{1.952059in}}%
\pgfpathlineto{\pgfqpoint{3.316560in}{1.952386in}}%
\pgfpathlineto{\pgfqpoint{3.328650in}{1.956094in}}%
\pgfpathlineto{\pgfqpoint{3.334695in}{1.954556in}}%
\pgfpathlineto{\pgfqpoint{3.340740in}{1.954883in}}%
\pgfpathlineto{\pgfqpoint{3.346785in}{1.951910in}}%
\pgfpathlineto{\pgfqpoint{3.352830in}{1.952489in}}%
\pgfpathlineto{\pgfqpoint{3.358875in}{1.956116in}}%
\pgfpathlineto{\pgfqpoint{3.364920in}{1.957505in}}%
\pgfpathlineto{\pgfqpoint{3.383055in}{1.966272in}}%
\pgfpathlineto{\pgfqpoint{3.407235in}{1.972754in}}%
\pgfpathlineto{\pgfqpoint{3.413280in}{1.975945in}}%
\pgfpathlineto{\pgfqpoint{3.425370in}{1.984261in}}%
\pgfpathlineto{\pgfqpoint{3.431415in}{1.986459in}}%
\pgfpathlineto{\pgfqpoint{3.437460in}{1.990397in}}%
\pgfpathlineto{\pgfqpoint{3.443505in}{1.990786in}}%
\pgfpathlineto{\pgfqpoint{3.449550in}{1.989435in}}%
\pgfpathlineto{\pgfqpoint{3.455595in}{1.992004in}}%
\pgfpathlineto{\pgfqpoint{3.467685in}{1.993034in}}%
\pgfpathlineto{\pgfqpoint{3.473730in}{1.992492in}}%
\pgfpathlineto{\pgfqpoint{3.479775in}{1.993317in}}%
\pgfpathlineto{\pgfqpoint{3.485820in}{1.992710in}}%
\pgfpathlineto{\pgfqpoint{3.491865in}{1.994348in}}%
\pgfpathlineto{\pgfqpoint{3.497910in}{1.999096in}}%
\pgfpathlineto{\pgfqpoint{3.503955in}{2.001107in}}%
\pgfpathlineto{\pgfqpoint{3.503955in}{2.001107in}}%
\pgfusepath{stroke}%
\end{pgfscope}%
\begin{pgfscope}%
\pgfsetrectcap%
\pgfsetmiterjoin%
\pgfsetlinewidth{1.003750pt}%
\definecolor{currentstroke}{rgb}{0.000000,0.000000,0.000000}%
\pgfsetstrokecolor{currentstroke}%
\pgfsetdash{}{0pt}%
\pgfpathmoveto{\pgfqpoint{0.487500in}{2.169299in}}%
\pgfpathlineto{\pgfqpoint{3.510000in}{2.169299in}}%
\pgfusepath{stroke}%
\end{pgfscope}%
\begin{pgfscope}%
\pgfsetrectcap%
\pgfsetmiterjoin%
\pgfsetlinewidth{1.003750pt}%
\definecolor{currentstroke}{rgb}{0.000000,0.000000,0.000000}%
\pgfsetstrokecolor{currentstroke}%
\pgfsetdash{}{0pt}%
\pgfpathmoveto{\pgfqpoint{3.510000in}{0.301292in}}%
\pgfpathlineto{\pgfqpoint{3.510000in}{2.169299in}}%
\pgfusepath{stroke}%
\end{pgfscope}%
\begin{pgfscope}%
\pgfsetrectcap%
\pgfsetmiterjoin%
\pgfsetlinewidth{1.003750pt}%
\definecolor{currentstroke}{rgb}{0.000000,0.000000,0.000000}%
\pgfsetstrokecolor{currentstroke}%
\pgfsetdash{}{0pt}%
\pgfpathmoveto{\pgfqpoint{0.487500in}{0.301292in}}%
\pgfpathlineto{\pgfqpoint{3.510000in}{0.301292in}}%
\pgfusepath{stroke}%
\end{pgfscope}%
\begin{pgfscope}%
\pgfsetrectcap%
\pgfsetmiterjoin%
\pgfsetlinewidth{1.003750pt}%
\definecolor{currentstroke}{rgb}{0.000000,0.000000,0.000000}%
\pgfsetstrokecolor{currentstroke}%
\pgfsetdash{}{0pt}%
\pgfpathmoveto{\pgfqpoint{0.487500in}{0.301292in}}%
\pgfpathlineto{\pgfqpoint{0.487500in}{2.169299in}}%
\pgfusepath{stroke}%
\end{pgfscope}%
\begin{pgfscope}%
\pgfsetbuttcap%
\pgfsetroundjoin%
\definecolor{currentfill}{rgb}{0.000000,0.000000,0.000000}%
\pgfsetfillcolor{currentfill}%
\pgfsetlinewidth{0.501875pt}%
\definecolor{currentstroke}{rgb}{0.000000,0.000000,0.000000}%
\pgfsetstrokecolor{currentstroke}%
\pgfsetdash{}{0pt}%
\pgfsys@defobject{currentmarker}{\pgfqpoint{0.000000in}{0.000000in}}{\pgfqpoint{0.000000in}{0.055556in}}{%
\pgfpathmoveto{\pgfqpoint{0.000000in}{0.000000in}}%
\pgfpathlineto{\pgfqpoint{0.000000in}{0.055556in}}%
\pgfusepath{stroke,fill}%
}%
\begin{pgfscope}%
\pgfsys@transformshift{0.487500in}{0.301292in}%
\pgfsys@useobject{currentmarker}{}%
\end{pgfscope}%
\end{pgfscope}%
\begin{pgfscope}%
\pgfsetbuttcap%
\pgfsetroundjoin%
\definecolor{currentfill}{rgb}{0.000000,0.000000,0.000000}%
\pgfsetfillcolor{currentfill}%
\pgfsetlinewidth{0.501875pt}%
\definecolor{currentstroke}{rgb}{0.000000,0.000000,0.000000}%
\pgfsetstrokecolor{currentstroke}%
\pgfsetdash{}{0pt}%
\pgfsys@defobject{currentmarker}{\pgfqpoint{0.000000in}{-0.055556in}}{\pgfqpoint{0.000000in}{0.000000in}}{%
\pgfpathmoveto{\pgfqpoint{0.000000in}{0.000000in}}%
\pgfpathlineto{\pgfqpoint{0.000000in}{-0.055556in}}%
\pgfusepath{stroke,fill}%
}%
\begin{pgfscope}%
\pgfsys@transformshift{0.487500in}{2.169299in}%
\pgfsys@useobject{currentmarker}{}%
\end{pgfscope}%
\end{pgfscope}%
\begin{pgfscope}%
\pgftext[x=0.487500in,y=0.245736in,,top]{\rmfamily\fontsize{8.000000}{9.600000}\selectfont \(\displaystyle 0\)}%
\end{pgfscope}%
\begin{pgfscope}%
\pgfsetbuttcap%
\pgfsetroundjoin%
\definecolor{currentfill}{rgb}{0.000000,0.000000,0.000000}%
\pgfsetfillcolor{currentfill}%
\pgfsetlinewidth{0.501875pt}%
\definecolor{currentstroke}{rgb}{0.000000,0.000000,0.000000}%
\pgfsetstrokecolor{currentstroke}%
\pgfsetdash{}{0pt}%
\pgfsys@defobject{currentmarker}{\pgfqpoint{0.000000in}{0.000000in}}{\pgfqpoint{0.000000in}{0.055556in}}{%
\pgfpathmoveto{\pgfqpoint{0.000000in}{0.000000in}}%
\pgfpathlineto{\pgfqpoint{0.000000in}{0.055556in}}%
\pgfusepath{stroke,fill}%
}%
\begin{pgfscope}%
\pgfsys@transformshift{1.092000in}{0.301292in}%
\pgfsys@useobject{currentmarker}{}%
\end{pgfscope}%
\end{pgfscope}%
\begin{pgfscope}%
\pgfsetbuttcap%
\pgfsetroundjoin%
\definecolor{currentfill}{rgb}{0.000000,0.000000,0.000000}%
\pgfsetfillcolor{currentfill}%
\pgfsetlinewidth{0.501875pt}%
\definecolor{currentstroke}{rgb}{0.000000,0.000000,0.000000}%
\pgfsetstrokecolor{currentstroke}%
\pgfsetdash{}{0pt}%
\pgfsys@defobject{currentmarker}{\pgfqpoint{0.000000in}{-0.055556in}}{\pgfqpoint{0.000000in}{0.000000in}}{%
\pgfpathmoveto{\pgfqpoint{0.000000in}{0.000000in}}%
\pgfpathlineto{\pgfqpoint{0.000000in}{-0.055556in}}%
\pgfusepath{stroke,fill}%
}%
\begin{pgfscope}%
\pgfsys@transformshift{1.092000in}{2.169299in}%
\pgfsys@useobject{currentmarker}{}%
\end{pgfscope}%
\end{pgfscope}%
\begin{pgfscope}%
\pgftext[x=1.092000in,y=0.245736in,,top]{\rmfamily\fontsize{8.000000}{9.600000}\selectfont \(\displaystyle 100\)}%
\end{pgfscope}%
\begin{pgfscope}%
\pgfsetbuttcap%
\pgfsetroundjoin%
\definecolor{currentfill}{rgb}{0.000000,0.000000,0.000000}%
\pgfsetfillcolor{currentfill}%
\pgfsetlinewidth{0.501875pt}%
\definecolor{currentstroke}{rgb}{0.000000,0.000000,0.000000}%
\pgfsetstrokecolor{currentstroke}%
\pgfsetdash{}{0pt}%
\pgfsys@defobject{currentmarker}{\pgfqpoint{0.000000in}{0.000000in}}{\pgfqpoint{0.000000in}{0.055556in}}{%
\pgfpathmoveto{\pgfqpoint{0.000000in}{0.000000in}}%
\pgfpathlineto{\pgfqpoint{0.000000in}{0.055556in}}%
\pgfusepath{stroke,fill}%
}%
\begin{pgfscope}%
\pgfsys@transformshift{1.696500in}{0.301292in}%
\pgfsys@useobject{currentmarker}{}%
\end{pgfscope}%
\end{pgfscope}%
\begin{pgfscope}%
\pgfsetbuttcap%
\pgfsetroundjoin%
\definecolor{currentfill}{rgb}{0.000000,0.000000,0.000000}%
\pgfsetfillcolor{currentfill}%
\pgfsetlinewidth{0.501875pt}%
\definecolor{currentstroke}{rgb}{0.000000,0.000000,0.000000}%
\pgfsetstrokecolor{currentstroke}%
\pgfsetdash{}{0pt}%
\pgfsys@defobject{currentmarker}{\pgfqpoint{0.000000in}{-0.055556in}}{\pgfqpoint{0.000000in}{0.000000in}}{%
\pgfpathmoveto{\pgfqpoint{0.000000in}{0.000000in}}%
\pgfpathlineto{\pgfqpoint{0.000000in}{-0.055556in}}%
\pgfusepath{stroke,fill}%
}%
\begin{pgfscope}%
\pgfsys@transformshift{1.696500in}{2.169299in}%
\pgfsys@useobject{currentmarker}{}%
\end{pgfscope}%
\end{pgfscope}%
\begin{pgfscope}%
\pgftext[x=1.696500in,y=0.245736in,,top]{\rmfamily\fontsize{8.000000}{9.600000}\selectfont \(\displaystyle 200\)}%
\end{pgfscope}%
\begin{pgfscope}%
\pgfsetbuttcap%
\pgfsetroundjoin%
\definecolor{currentfill}{rgb}{0.000000,0.000000,0.000000}%
\pgfsetfillcolor{currentfill}%
\pgfsetlinewidth{0.501875pt}%
\definecolor{currentstroke}{rgb}{0.000000,0.000000,0.000000}%
\pgfsetstrokecolor{currentstroke}%
\pgfsetdash{}{0pt}%
\pgfsys@defobject{currentmarker}{\pgfqpoint{0.000000in}{0.000000in}}{\pgfqpoint{0.000000in}{0.055556in}}{%
\pgfpathmoveto{\pgfqpoint{0.000000in}{0.000000in}}%
\pgfpathlineto{\pgfqpoint{0.000000in}{0.055556in}}%
\pgfusepath{stroke,fill}%
}%
\begin{pgfscope}%
\pgfsys@transformshift{2.301000in}{0.301292in}%
\pgfsys@useobject{currentmarker}{}%
\end{pgfscope}%
\end{pgfscope}%
\begin{pgfscope}%
\pgfsetbuttcap%
\pgfsetroundjoin%
\definecolor{currentfill}{rgb}{0.000000,0.000000,0.000000}%
\pgfsetfillcolor{currentfill}%
\pgfsetlinewidth{0.501875pt}%
\definecolor{currentstroke}{rgb}{0.000000,0.000000,0.000000}%
\pgfsetstrokecolor{currentstroke}%
\pgfsetdash{}{0pt}%
\pgfsys@defobject{currentmarker}{\pgfqpoint{0.000000in}{-0.055556in}}{\pgfqpoint{0.000000in}{0.000000in}}{%
\pgfpathmoveto{\pgfqpoint{0.000000in}{0.000000in}}%
\pgfpathlineto{\pgfqpoint{0.000000in}{-0.055556in}}%
\pgfusepath{stroke,fill}%
}%
\begin{pgfscope}%
\pgfsys@transformshift{2.301000in}{2.169299in}%
\pgfsys@useobject{currentmarker}{}%
\end{pgfscope}%
\end{pgfscope}%
\begin{pgfscope}%
\pgftext[x=2.301000in,y=0.245736in,,top]{\rmfamily\fontsize{8.000000}{9.600000}\selectfont \(\displaystyle 300\)}%
\end{pgfscope}%
\begin{pgfscope}%
\pgfsetbuttcap%
\pgfsetroundjoin%
\definecolor{currentfill}{rgb}{0.000000,0.000000,0.000000}%
\pgfsetfillcolor{currentfill}%
\pgfsetlinewidth{0.501875pt}%
\definecolor{currentstroke}{rgb}{0.000000,0.000000,0.000000}%
\pgfsetstrokecolor{currentstroke}%
\pgfsetdash{}{0pt}%
\pgfsys@defobject{currentmarker}{\pgfqpoint{0.000000in}{0.000000in}}{\pgfqpoint{0.000000in}{0.055556in}}{%
\pgfpathmoveto{\pgfqpoint{0.000000in}{0.000000in}}%
\pgfpathlineto{\pgfqpoint{0.000000in}{0.055556in}}%
\pgfusepath{stroke,fill}%
}%
\begin{pgfscope}%
\pgfsys@transformshift{2.905500in}{0.301292in}%
\pgfsys@useobject{currentmarker}{}%
\end{pgfscope}%
\end{pgfscope}%
\begin{pgfscope}%
\pgfsetbuttcap%
\pgfsetroundjoin%
\definecolor{currentfill}{rgb}{0.000000,0.000000,0.000000}%
\pgfsetfillcolor{currentfill}%
\pgfsetlinewidth{0.501875pt}%
\definecolor{currentstroke}{rgb}{0.000000,0.000000,0.000000}%
\pgfsetstrokecolor{currentstroke}%
\pgfsetdash{}{0pt}%
\pgfsys@defobject{currentmarker}{\pgfqpoint{0.000000in}{-0.055556in}}{\pgfqpoint{0.000000in}{0.000000in}}{%
\pgfpathmoveto{\pgfqpoint{0.000000in}{0.000000in}}%
\pgfpathlineto{\pgfqpoint{0.000000in}{-0.055556in}}%
\pgfusepath{stroke,fill}%
}%
\begin{pgfscope}%
\pgfsys@transformshift{2.905500in}{2.169299in}%
\pgfsys@useobject{currentmarker}{}%
\end{pgfscope}%
\end{pgfscope}%
\begin{pgfscope}%
\pgftext[x=2.905500in,y=0.245736in,,top]{\rmfamily\fontsize{8.000000}{9.600000}\selectfont \(\displaystyle 400\)}%
\end{pgfscope}%
\begin{pgfscope}%
\pgfsetbuttcap%
\pgfsetroundjoin%
\definecolor{currentfill}{rgb}{0.000000,0.000000,0.000000}%
\pgfsetfillcolor{currentfill}%
\pgfsetlinewidth{0.501875pt}%
\definecolor{currentstroke}{rgb}{0.000000,0.000000,0.000000}%
\pgfsetstrokecolor{currentstroke}%
\pgfsetdash{}{0pt}%
\pgfsys@defobject{currentmarker}{\pgfqpoint{0.000000in}{0.000000in}}{\pgfqpoint{0.000000in}{0.055556in}}{%
\pgfpathmoveto{\pgfqpoint{0.000000in}{0.000000in}}%
\pgfpathlineto{\pgfqpoint{0.000000in}{0.055556in}}%
\pgfusepath{stroke,fill}%
}%
\begin{pgfscope}%
\pgfsys@transformshift{3.510000in}{0.301292in}%
\pgfsys@useobject{currentmarker}{}%
\end{pgfscope}%
\end{pgfscope}%
\begin{pgfscope}%
\pgfsetbuttcap%
\pgfsetroundjoin%
\definecolor{currentfill}{rgb}{0.000000,0.000000,0.000000}%
\pgfsetfillcolor{currentfill}%
\pgfsetlinewidth{0.501875pt}%
\definecolor{currentstroke}{rgb}{0.000000,0.000000,0.000000}%
\pgfsetstrokecolor{currentstroke}%
\pgfsetdash{}{0pt}%
\pgfsys@defobject{currentmarker}{\pgfqpoint{0.000000in}{-0.055556in}}{\pgfqpoint{0.000000in}{0.000000in}}{%
\pgfpathmoveto{\pgfqpoint{0.000000in}{0.000000in}}%
\pgfpathlineto{\pgfqpoint{0.000000in}{-0.055556in}}%
\pgfusepath{stroke,fill}%
}%
\begin{pgfscope}%
\pgfsys@transformshift{3.510000in}{2.169299in}%
\pgfsys@useobject{currentmarker}{}%
\end{pgfscope}%
\end{pgfscope}%
\begin{pgfscope}%
\pgftext[x=3.510000in,y=0.245736in,,top]{\rmfamily\fontsize{8.000000}{9.600000}\selectfont \(\displaystyle 500\)}%
\end{pgfscope}%
\begin{pgfscope}%
\pgftext[x=1.998750in,y=0.078167in,,top]{\rmfamily\fontsize{10.000000}{12.000000}\selectfont \(\displaystyle x\)}%
\end{pgfscope}%
\begin{pgfscope}%
\pgfsetbuttcap%
\pgfsetroundjoin%
\definecolor{currentfill}{rgb}{0.000000,0.000000,0.000000}%
\pgfsetfillcolor{currentfill}%
\pgfsetlinewidth{0.501875pt}%
\definecolor{currentstroke}{rgb}{0.000000,0.000000,0.000000}%
\pgfsetstrokecolor{currentstroke}%
\pgfsetdash{}{0pt}%
\pgfsys@defobject{currentmarker}{\pgfqpoint{0.000000in}{0.000000in}}{\pgfqpoint{0.055556in}{0.000000in}}{%
\pgfpathmoveto{\pgfqpoint{0.000000in}{0.000000in}}%
\pgfpathlineto{\pgfqpoint{0.055556in}{0.000000in}}%
\pgfusepath{stroke,fill}%
}%
\begin{pgfscope}%
\pgfsys@transformshift{0.487500in}{0.301292in}%
\pgfsys@useobject{currentmarker}{}%
\end{pgfscope}%
\end{pgfscope}%
\begin{pgfscope}%
\pgfsetbuttcap%
\pgfsetroundjoin%
\definecolor{currentfill}{rgb}{0.000000,0.000000,0.000000}%
\pgfsetfillcolor{currentfill}%
\pgfsetlinewidth{0.501875pt}%
\definecolor{currentstroke}{rgb}{0.000000,0.000000,0.000000}%
\pgfsetstrokecolor{currentstroke}%
\pgfsetdash{}{0pt}%
\pgfsys@defobject{currentmarker}{\pgfqpoint{-0.055556in}{0.000000in}}{\pgfqpoint{0.000000in}{0.000000in}}{%
\pgfpathmoveto{\pgfqpoint{0.000000in}{0.000000in}}%
\pgfpathlineto{\pgfqpoint{-0.055556in}{0.000000in}}%
\pgfusepath{stroke,fill}%
}%
\begin{pgfscope}%
\pgfsys@transformshift{3.510000in}{0.301292in}%
\pgfsys@useobject{currentmarker}{}%
\end{pgfscope}%
\end{pgfscope}%
\begin{pgfscope}%
\pgftext[x=0.431944in,y=0.301292in,right,]{\rmfamily\fontsize{8.000000}{9.600000}\selectfont \(\displaystyle 0\)}%
\end{pgfscope}%
\begin{pgfscope}%
\pgfsetbuttcap%
\pgfsetroundjoin%
\definecolor{currentfill}{rgb}{0.000000,0.000000,0.000000}%
\pgfsetfillcolor{currentfill}%
\pgfsetlinewidth{0.501875pt}%
\definecolor{currentstroke}{rgb}{0.000000,0.000000,0.000000}%
\pgfsetstrokecolor{currentstroke}%
\pgfsetdash{}{0pt}%
\pgfsys@defobject{currentmarker}{\pgfqpoint{0.000000in}{0.000000in}}{\pgfqpoint{0.055556in}{0.000000in}}{%
\pgfpathmoveto{\pgfqpoint{0.000000in}{0.000000in}}%
\pgfpathlineto{\pgfqpoint{0.055556in}{0.000000in}}%
\pgfusepath{stroke,fill}%
}%
\begin{pgfscope}%
\pgfsys@transformshift{0.487500in}{0.612626in}%
\pgfsys@useobject{currentmarker}{}%
\end{pgfscope}%
\end{pgfscope}%
\begin{pgfscope}%
\pgfsetbuttcap%
\pgfsetroundjoin%
\definecolor{currentfill}{rgb}{0.000000,0.000000,0.000000}%
\pgfsetfillcolor{currentfill}%
\pgfsetlinewidth{0.501875pt}%
\definecolor{currentstroke}{rgb}{0.000000,0.000000,0.000000}%
\pgfsetstrokecolor{currentstroke}%
\pgfsetdash{}{0pt}%
\pgfsys@defobject{currentmarker}{\pgfqpoint{-0.055556in}{0.000000in}}{\pgfqpoint{0.000000in}{0.000000in}}{%
\pgfpathmoveto{\pgfqpoint{0.000000in}{0.000000in}}%
\pgfpathlineto{\pgfqpoint{-0.055556in}{0.000000in}}%
\pgfusepath{stroke,fill}%
}%
\begin{pgfscope}%
\pgfsys@transformshift{3.510000in}{0.612626in}%
\pgfsys@useobject{currentmarker}{}%
\end{pgfscope}%
\end{pgfscope}%
\begin{pgfscope}%
\pgftext[x=0.431944in,y=0.612626in,right,]{\rmfamily\fontsize{8.000000}{9.600000}\selectfont \(\displaystyle 1\)}%
\end{pgfscope}%
\begin{pgfscope}%
\pgfsetbuttcap%
\pgfsetroundjoin%
\definecolor{currentfill}{rgb}{0.000000,0.000000,0.000000}%
\pgfsetfillcolor{currentfill}%
\pgfsetlinewidth{0.501875pt}%
\definecolor{currentstroke}{rgb}{0.000000,0.000000,0.000000}%
\pgfsetstrokecolor{currentstroke}%
\pgfsetdash{}{0pt}%
\pgfsys@defobject{currentmarker}{\pgfqpoint{0.000000in}{0.000000in}}{\pgfqpoint{0.055556in}{0.000000in}}{%
\pgfpathmoveto{\pgfqpoint{0.000000in}{0.000000in}}%
\pgfpathlineto{\pgfqpoint{0.055556in}{0.000000in}}%
\pgfusepath{stroke,fill}%
}%
\begin{pgfscope}%
\pgfsys@transformshift{0.487500in}{0.923961in}%
\pgfsys@useobject{currentmarker}{}%
\end{pgfscope}%
\end{pgfscope}%
\begin{pgfscope}%
\pgfsetbuttcap%
\pgfsetroundjoin%
\definecolor{currentfill}{rgb}{0.000000,0.000000,0.000000}%
\pgfsetfillcolor{currentfill}%
\pgfsetlinewidth{0.501875pt}%
\definecolor{currentstroke}{rgb}{0.000000,0.000000,0.000000}%
\pgfsetstrokecolor{currentstroke}%
\pgfsetdash{}{0pt}%
\pgfsys@defobject{currentmarker}{\pgfqpoint{-0.055556in}{0.000000in}}{\pgfqpoint{0.000000in}{0.000000in}}{%
\pgfpathmoveto{\pgfqpoint{0.000000in}{0.000000in}}%
\pgfpathlineto{\pgfqpoint{-0.055556in}{0.000000in}}%
\pgfusepath{stroke,fill}%
}%
\begin{pgfscope}%
\pgfsys@transformshift{3.510000in}{0.923961in}%
\pgfsys@useobject{currentmarker}{}%
\end{pgfscope}%
\end{pgfscope}%
\begin{pgfscope}%
\pgftext[x=0.431944in,y=0.923961in,right,]{\rmfamily\fontsize{8.000000}{9.600000}\selectfont \(\displaystyle 2\)}%
\end{pgfscope}%
\begin{pgfscope}%
\pgfsetbuttcap%
\pgfsetroundjoin%
\definecolor{currentfill}{rgb}{0.000000,0.000000,0.000000}%
\pgfsetfillcolor{currentfill}%
\pgfsetlinewidth{0.501875pt}%
\definecolor{currentstroke}{rgb}{0.000000,0.000000,0.000000}%
\pgfsetstrokecolor{currentstroke}%
\pgfsetdash{}{0pt}%
\pgfsys@defobject{currentmarker}{\pgfqpoint{0.000000in}{0.000000in}}{\pgfqpoint{0.055556in}{0.000000in}}{%
\pgfpathmoveto{\pgfqpoint{0.000000in}{0.000000in}}%
\pgfpathlineto{\pgfqpoint{0.055556in}{0.000000in}}%
\pgfusepath{stroke,fill}%
}%
\begin{pgfscope}%
\pgfsys@transformshift{0.487500in}{1.235295in}%
\pgfsys@useobject{currentmarker}{}%
\end{pgfscope}%
\end{pgfscope}%
\begin{pgfscope}%
\pgfsetbuttcap%
\pgfsetroundjoin%
\definecolor{currentfill}{rgb}{0.000000,0.000000,0.000000}%
\pgfsetfillcolor{currentfill}%
\pgfsetlinewidth{0.501875pt}%
\definecolor{currentstroke}{rgb}{0.000000,0.000000,0.000000}%
\pgfsetstrokecolor{currentstroke}%
\pgfsetdash{}{0pt}%
\pgfsys@defobject{currentmarker}{\pgfqpoint{-0.055556in}{0.000000in}}{\pgfqpoint{0.000000in}{0.000000in}}{%
\pgfpathmoveto{\pgfqpoint{0.000000in}{0.000000in}}%
\pgfpathlineto{\pgfqpoint{-0.055556in}{0.000000in}}%
\pgfusepath{stroke,fill}%
}%
\begin{pgfscope}%
\pgfsys@transformshift{3.510000in}{1.235295in}%
\pgfsys@useobject{currentmarker}{}%
\end{pgfscope}%
\end{pgfscope}%
\begin{pgfscope}%
\pgftext[x=0.431944in,y=1.235295in,right,]{\rmfamily\fontsize{8.000000}{9.600000}\selectfont \(\displaystyle 3\)}%
\end{pgfscope}%
\begin{pgfscope}%
\pgfsetbuttcap%
\pgfsetroundjoin%
\definecolor{currentfill}{rgb}{0.000000,0.000000,0.000000}%
\pgfsetfillcolor{currentfill}%
\pgfsetlinewidth{0.501875pt}%
\definecolor{currentstroke}{rgb}{0.000000,0.000000,0.000000}%
\pgfsetstrokecolor{currentstroke}%
\pgfsetdash{}{0pt}%
\pgfsys@defobject{currentmarker}{\pgfqpoint{0.000000in}{0.000000in}}{\pgfqpoint{0.055556in}{0.000000in}}{%
\pgfpathmoveto{\pgfqpoint{0.000000in}{0.000000in}}%
\pgfpathlineto{\pgfqpoint{0.055556in}{0.000000in}}%
\pgfusepath{stroke,fill}%
}%
\begin{pgfscope}%
\pgfsys@transformshift{0.487500in}{1.546630in}%
\pgfsys@useobject{currentmarker}{}%
\end{pgfscope}%
\end{pgfscope}%
\begin{pgfscope}%
\pgfsetbuttcap%
\pgfsetroundjoin%
\definecolor{currentfill}{rgb}{0.000000,0.000000,0.000000}%
\pgfsetfillcolor{currentfill}%
\pgfsetlinewidth{0.501875pt}%
\definecolor{currentstroke}{rgb}{0.000000,0.000000,0.000000}%
\pgfsetstrokecolor{currentstroke}%
\pgfsetdash{}{0pt}%
\pgfsys@defobject{currentmarker}{\pgfqpoint{-0.055556in}{0.000000in}}{\pgfqpoint{0.000000in}{0.000000in}}{%
\pgfpathmoveto{\pgfqpoint{0.000000in}{0.000000in}}%
\pgfpathlineto{\pgfqpoint{-0.055556in}{0.000000in}}%
\pgfusepath{stroke,fill}%
}%
\begin{pgfscope}%
\pgfsys@transformshift{3.510000in}{1.546630in}%
\pgfsys@useobject{currentmarker}{}%
\end{pgfscope}%
\end{pgfscope}%
\begin{pgfscope}%
\pgftext[x=0.431944in,y=1.546630in,right,]{\rmfamily\fontsize{8.000000}{9.600000}\selectfont \(\displaystyle 4\)}%
\end{pgfscope}%
\begin{pgfscope}%
\pgfsetbuttcap%
\pgfsetroundjoin%
\definecolor{currentfill}{rgb}{0.000000,0.000000,0.000000}%
\pgfsetfillcolor{currentfill}%
\pgfsetlinewidth{0.501875pt}%
\definecolor{currentstroke}{rgb}{0.000000,0.000000,0.000000}%
\pgfsetstrokecolor{currentstroke}%
\pgfsetdash{}{0pt}%
\pgfsys@defobject{currentmarker}{\pgfqpoint{0.000000in}{0.000000in}}{\pgfqpoint{0.055556in}{0.000000in}}{%
\pgfpathmoveto{\pgfqpoint{0.000000in}{0.000000in}}%
\pgfpathlineto{\pgfqpoint{0.055556in}{0.000000in}}%
\pgfusepath{stroke,fill}%
}%
\begin{pgfscope}%
\pgfsys@transformshift{0.487500in}{1.857965in}%
\pgfsys@useobject{currentmarker}{}%
\end{pgfscope}%
\end{pgfscope}%
\begin{pgfscope}%
\pgfsetbuttcap%
\pgfsetroundjoin%
\definecolor{currentfill}{rgb}{0.000000,0.000000,0.000000}%
\pgfsetfillcolor{currentfill}%
\pgfsetlinewidth{0.501875pt}%
\definecolor{currentstroke}{rgb}{0.000000,0.000000,0.000000}%
\pgfsetstrokecolor{currentstroke}%
\pgfsetdash{}{0pt}%
\pgfsys@defobject{currentmarker}{\pgfqpoint{-0.055556in}{0.000000in}}{\pgfqpoint{0.000000in}{0.000000in}}{%
\pgfpathmoveto{\pgfqpoint{0.000000in}{0.000000in}}%
\pgfpathlineto{\pgfqpoint{-0.055556in}{0.000000in}}%
\pgfusepath{stroke,fill}%
}%
\begin{pgfscope}%
\pgfsys@transformshift{3.510000in}{1.857965in}%
\pgfsys@useobject{currentmarker}{}%
\end{pgfscope}%
\end{pgfscope}%
\begin{pgfscope}%
\pgftext[x=0.431944in,y=1.857965in,right,]{\rmfamily\fontsize{8.000000}{9.600000}\selectfont \(\displaystyle 5\)}%
\end{pgfscope}%
\begin{pgfscope}%
\pgfsetbuttcap%
\pgfsetroundjoin%
\definecolor{currentfill}{rgb}{0.000000,0.000000,0.000000}%
\pgfsetfillcolor{currentfill}%
\pgfsetlinewidth{0.501875pt}%
\definecolor{currentstroke}{rgb}{0.000000,0.000000,0.000000}%
\pgfsetstrokecolor{currentstroke}%
\pgfsetdash{}{0pt}%
\pgfsys@defobject{currentmarker}{\pgfqpoint{0.000000in}{0.000000in}}{\pgfqpoint{0.055556in}{0.000000in}}{%
\pgfpathmoveto{\pgfqpoint{0.000000in}{0.000000in}}%
\pgfpathlineto{\pgfqpoint{0.055556in}{0.000000in}}%
\pgfusepath{stroke,fill}%
}%
\begin{pgfscope}%
\pgfsys@transformshift{0.487500in}{2.169299in}%
\pgfsys@useobject{currentmarker}{}%
\end{pgfscope}%
\end{pgfscope}%
\begin{pgfscope}%
\pgfsetbuttcap%
\pgfsetroundjoin%
\definecolor{currentfill}{rgb}{0.000000,0.000000,0.000000}%
\pgfsetfillcolor{currentfill}%
\pgfsetlinewidth{0.501875pt}%
\definecolor{currentstroke}{rgb}{0.000000,0.000000,0.000000}%
\pgfsetstrokecolor{currentstroke}%
\pgfsetdash{}{0pt}%
\pgfsys@defobject{currentmarker}{\pgfqpoint{-0.055556in}{0.000000in}}{\pgfqpoint{0.000000in}{0.000000in}}{%
\pgfpathmoveto{\pgfqpoint{0.000000in}{0.000000in}}%
\pgfpathlineto{\pgfqpoint{-0.055556in}{0.000000in}}%
\pgfusepath{stroke,fill}%
}%
\begin{pgfscope}%
\pgfsys@transformshift{3.510000in}{2.169299in}%
\pgfsys@useobject{currentmarker}{}%
\end{pgfscope}%
\end{pgfscope}%
\begin{pgfscope}%
\pgftext[x=0.431944in,y=2.169299in,right,]{\rmfamily\fontsize{8.000000}{9.600000}\selectfont \(\displaystyle 6\)}%
\end{pgfscope}%
\begin{pgfscope}%
\pgfsetbuttcap%
\pgfsetmiterjoin%
\definecolor{currentfill}{rgb}{1.000000,1.000000,1.000000}%
\pgfsetfillcolor{currentfill}%
\pgfsetlinewidth{1.003750pt}%
\definecolor{currentstroke}{rgb}{0.000000,0.000000,0.000000}%
\pgfsetstrokecolor{currentstroke}%
\pgfsetdash{}{0pt}%
\pgfpathmoveto{\pgfqpoint{0.543056in}{1.305745in}}%
\pgfpathlineto{\pgfqpoint{1.536892in}{1.305745in}}%
\pgfpathlineto{\pgfqpoint{1.536892in}{2.113744in}}%
\pgfpathlineto{\pgfqpoint{0.543056in}{2.113744in}}%
\pgfpathclose%
\pgfusepath{stroke,fill}%
\end{pgfscope}%
\begin{pgfscope}%
\pgfsetrectcap%
\pgfsetroundjoin%
\pgfsetlinewidth{1.505625pt}%
\definecolor{currentstroke}{rgb}{0.000000,0.000000,1.000000}%
\pgfsetstrokecolor{currentstroke}%
\pgfsetdash{}{0pt}%
\pgfpathmoveto{\pgfqpoint{0.620833in}{2.030410in}}%
\pgfpathlineto{\pgfqpoint{0.776389in}{2.030410in}}%
\pgfusepath{stroke}%
\end{pgfscope}%
\begin{pgfscope}%
\pgftext[x=0.898611in,y=1.991522in,left,base]{\rmfamily\fontsize{8.000000}{9.600000}\selectfont OGI}%
\end{pgfscope}%
\begin{pgfscope}%
\pgfsetrectcap%
\pgfsetroundjoin%
\pgfsetlinewidth{1.505625pt}%
\definecolor{currentstroke}{rgb}{0.000000,0.500000,0.000000}%
\pgfsetstrokecolor{currentstroke}%
\pgfsetdash{}{0pt}%
\pgfpathmoveto{\pgfqpoint{0.620833in}{1.875477in}}%
\pgfpathlineto{\pgfqpoint{0.776389in}{1.875477in}}%
\pgfusepath{stroke}%
\end{pgfscope}%
\begin{pgfscope}%
\pgftext[x=0.898611in,y=1.836588in,left,base]{\rmfamily\fontsize{8.000000}{9.600000}\selectfont IDS}%
\end{pgfscope}%
\begin{pgfscope}%
\pgfsetrectcap%
\pgfsetroundjoin%
\pgfsetlinewidth{1.505625pt}%
\definecolor{currentstroke}{rgb}{1.000000,0.000000,0.000000}%
\pgfsetstrokecolor{currentstroke}%
\pgfsetdash{}{0pt}%
\pgfpathmoveto{\pgfqpoint{0.620833in}{1.720544in}}%
\pgfpathlineto{\pgfqpoint{0.776389in}{1.720544in}}%
\pgfusepath{stroke}%
\end{pgfscope}%
\begin{pgfscope}%
\pgftext[x=0.898611in,y=1.681655in,left,base]{\rmfamily\fontsize{8.000000}{9.600000}\selectfont Thompson}%
\end{pgfscope}%
\begin{pgfscope}%
\pgfsetrectcap%
\pgfsetroundjoin%
\pgfsetlinewidth{1.505625pt}%
\definecolor{currentstroke}{rgb}{0.000000,0.750000,0.750000}%
\pgfsetstrokecolor{currentstroke}%
\pgfsetdash{}{0pt}%
\pgfpathmoveto{\pgfqpoint{0.620833in}{1.565611in}}%
\pgfpathlineto{\pgfqpoint{0.776389in}{1.565611in}}%
\pgfusepath{stroke}%
\end{pgfscope}%
\begin{pgfscope}%
\pgftext[x=0.898611in,y=1.526722in,left,base]{\rmfamily\fontsize{8.000000}{9.600000}\selectfont Bayes UCB}%
\end{pgfscope}%
\begin{pgfscope}%
\pgfsetrectcap%
\pgfsetroundjoin%
\pgfsetlinewidth{1.505625pt}%
\definecolor{currentstroke}{rgb}{0.750000,0.000000,0.750000}%
\pgfsetstrokecolor{currentstroke}%
\pgfsetdash{}{0pt}%
\pgfpathmoveto{\pgfqpoint{0.620833in}{1.410678in}}%
\pgfpathlineto{\pgfqpoint{0.776389in}{1.410678in}}%
\pgfusepath{stroke}%
\end{pgfscope}%
\begin{pgfscope}%
\pgftext[x=0.898611in,y=1.371789in,left,base]{\rmfamily\fontsize{8.000000}{9.600000}\selectfont KL-UCB}%
\end{pgfscope}%
\end{pgfpicture}%
\makeatother%
\endgroup%

	\caption{Frequentist regret. The OGI policy is configured $K=1$ and $\alpha=100$.}
	\label{fig:kaufmann_regret}
\end{figure}

\subsection{Additional tables for Section~\ref{sec:experiments}}
\begin{table}
	\centering
	\begin{tabular}{rrrrrr} 
		\toprule
		{}    $\alpha$ &   $\beta$ &  OGI(1) &  OGI(3) &  OGI(5) &  Gittins \\
		\midrule
		   1 & 1 &   0.760 &   0.721 &   0.712 &    0.703 \\
		   1 & 2 &   0.571 &   0.522 &   0.511 &    0.500 \\
		   1 & 3 &   0.452 &   0.401 &   0.389 &    0.380 \\
		   1 & 4 &   0.374 &   0.321 &   0.312 &    0.302 \\
		  2 & 1 &   0.853 &   0.818 &   0.809 &    0.800 \\
		  2 & 2 &   0.702 &   0.657 &   0.646 &    0.635 \\
		  2 & 3 &   0.591 &   0.543 &   0.530 &    0.516 \\
		  2 & 4 &   0.508 &   0.458 &   0.445 &    0.434 \\
		  3 & 1 &   0.893 &   0.864 &   0.855 &    0.845 \\
		  3 & 2 &   0.771 &   0.729 &   0.719 &    0.707 \\
		  3 & 3 &   0.671 &   0.626 &   0.613 &    0.601 \\
		  3 & 4 &   0.592 &   0.545 &   0.532 &    0.518 \\
		 4 & 1 &   0.916 &   0.890 &   0.882 &    0.872 \\
		  4 & 2 &   0.813 &   0.776 &   0.765 &    0.754 \\
		  4 & 3 &   0.724 &   0.682 &   0.670 &    0.658 \\
		  4 & 4 &   0.651 &   0.607 &   0.593 &    0.581 \\
		\bottomrule
	\end{tabular}
	\caption{Optimistic and exact Gittins Indices when $\gamma = 0.9$ for different Beta-Bernoulli parameters}
	\label{table:ogi_table_for_gamma_9}
\end{table}

\begin{table}
	\centering
	\begin{tabular}{rrrrrr}
		\toprule
		{}    $\alpha$ &   $\beta$ &  OGI(1) &  OGI(3) &  OGI(5) &  Gittins \\
		\midrule
		 1.0 & 1.0 &   0.817 &   0.784 &   0.774 &    0.761 \\
		 1.0 & 2.0 &   0.637 &   0.590 &   0.577 &    0.560 \\
		 1.0 & 3.0 &   0.514 &   0.463 &   0.449 &    0.433 \\
		 1.0 & 4.0 &   0.430 &   0.376 &   0.364 &    0.348 \\
		 2.0 & 1.0 &   0.890 &   0.860 &   0.851 &    0.838 \\
		 2.0 & 2.0 &   0.752 &   0.710 &   0.698 &    0.681 \\
		 2.0 & 3.0 &   0.643 &   0.596 &   0.581 &    0.562 \\
		2.0 & 4.0 &   0.558 &   0.509 &   0.494 &    0.475 \\
		 3.0 & 1.0 &   0.921 &   0.896 &   0.887 &    0.874 \\
		3.0 & 2.0 &   0.811 &   0.773 &   0.762 &    0.744 \\
		 3.0 & 3.0 &   0.715 &   0.672 &   0.658 &    0.639 \\
		 3.0 & 4.0 &   0.637 &   0.591 &   0.575 &    0.556 \\
		4.0 & 1.0 &   0.938 &   0.916 &   0.908 &    0.895 \\
		4.0 & 2.0 &   0.847 &   0.812 &   0.801 &    0.784 \\
		4.0 & 3.0 &   0.763 &   0.722 &   0.709 &    0.690 \\
		4.0 & 4.0 &   0.691 &   0.648 &   0.633 &    0.613 \\
		\bottomrule
	\end{tabular}
	\caption{Optimistic and exact Gittins Indices when $\gamma = 0.95$ for different Beta-Bernoulli parameters}
	\label{table:ogi_table_for_gamma_95}
\end{table}
\begin{table}
	\centering
	\begin{tabular}{llrrrrr}
		\toprule
		&       &  Diff. B-UCB &  Diff. TS &    OGI &  Thompson &  Bayes UCB \\
		\midrule
		& mean &           28.9 &            3.0 &    99.7 &      102.7 &      128.5 \\
		& SD &          407.2 &          410.6 &   493.8 &      297.5 &      302.3 \\
		$A = 10$& 25\% &           20.0 &           12.0 &   -79.7 &      -62.6 &      -43.5 \\
		& 50\% &           33.0 &           22.0 &    57.3 &       70.4 &       86.5 \\
		& 75\% &           59.0 &           34.0 &   228.2 &      244.9 &      270.1 \\ \hline
		& mean &           48.4 &           20.8 &   184.1 &      205.0 &      232.5 \\
		& SD &          540.8 &          556.9 &   565.2 &      237.7 &      214.0 \\
		$A = 40$& 25\% &           46.0 &           31.0 &    37.4 &       79.7 &       92.8 \\
		& 50\% &           87.0 &           67.0 &   119.6 &      159.8 &      183.1 \\
		& 75\% &          127.0 &           91.0 &   236.1 &      286.9 &      324.0 \\ \hline
		& mean &           75.2 &           42.4 &   239.4 &      281.7 &      314.5 \\
		& SD &          443.2 &          438.3 &   451.8 &      181.4 &      203.1 \\
		$A = 70$ & 25\% &           50.0 &           37.0 &    98.9 &      157.5 &      170.3 \\
		& 50\% &          124.0 &           97.0 &   180.3 &      241.0 &      265.7 \\
		& 75\% &          182.0 &          139.0 &   276.6 &      366.0 &      412.3 \\ \hline
		& mean &           93.6 &           60.6 &   287.1 &      347.7 &      380.7 \\
		& SD &          448.3 &          442.7 &   447.2 &      184.3 &      208.8 \\
		$A = 100$  & 25\% &           57.0 &           41.0 &   149.3 &      217.9 &      234.0 \\
		& 50\% &          144.0 &          121.5 &   228.5 &      309.0 &      334.5 \\
		& 75\% &          232.0 &          182.0 &   325.4 &      434.0 &      483.7 \\
		\bottomrule
	\end{tabular}
	\caption{Summary statistics (mean, standard deviation, lower, median and upper quartiles) for regret in the \cite{chapelle2011empirical} inspired experiment. For each value of $A$ we simulated 5,000 trials. The first two columns show statistics for the absolute difference in realized regret between the two competing algorithms and OGI.}
	\label{table:additional_cli_table}
\end{table}
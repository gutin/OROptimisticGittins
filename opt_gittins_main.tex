%%%%%%%%%%%%%%%%%%%%%%%%%%%%%%%%%%%%%%%%%%%%%%%%%%%%%%%%%%%%%%%%%%%%%%%%%%%%
%% Author template for Operations Reseacrh (opre) for articles with no e-companion (EC)
%% Mirko Janc, Ph.D., INFORMS, mirko.janc@informs.org
%% ver. 0.95, December 2010
%%%%%%%%%%%%%%%%%%%%%%%%%%%%%%%%%%%%%%%%%%%%%%%%%%%%%%%%%%%%%%%%%%%%%%%%%%%%
%\documentclass[msom,blindrev]{informs3}
%\documentclass[opre,nonblindrev]{informs3} % current default for manuscript submission

\documentclass[11pt, letterpaper]{article}

\usepackage[final]{microtype}
\usepackage{color}
\usepackage[ruled]{algorithm2e}
\usepackage{subcaption}
\usepackage{varwidth}
\usepackage{capt-of}
%\usepackage[blocks]{authblk}
\usepackage{boldline}
\usepackage{epsfig}
\usepackage{pgf}
\usepackage{latexsym}
\usepackage{url}
\usepackage{float}
\usepackage[hypertexnames=false,hyperfootnotes=false]{hyperref}
\usepackage{texnansi}
\usepackage{color}
\usepackage{tikz}
\usepackage{afterpage}
\usepackage{enumerate}
\usepackage[normalem]{ulem}
\usepackage{lmodern}
\usepackage{rotating}
\usepackage[ruled]{algorithm2e}
\usepackage[noend]{algpseudocode}
\usepackage{graphicx}
\usepackage{caption}
\usepackage{subcaption}
\usepackage{bbm}
\usepackage{layouts}
% DS
\usepackage{booktabs}
\newcommand{\rara}[1]{\renewcommand{\arraystretch}{#1}}
% 
\usepackage{sectsty}
%
\usepackage{ifthen}
%DS

\usepackage[leqno]{amsmath}
\usepackage{amsthm}
\usepackage{amssymb}
\usepackage{amsfonts}
\usepackage{mathtools}
% DS 
\usepackage[margin=10pt,font=small,labelfont=bf]{caption}


%\DoubleSpacedXI % Made default 4/4/2014 at request
%\OneAndAHalfSpacedXI % current default line spacing
%%\OneAndAHalfSpacedXII
%%\DoubleSpacedXII

% If hyperref is used, dvi-to-ps driver of choice must be declared as
%   an additional option to the \documentclass. For example
%\documentclass[dvips,opre]{informs3}      % if dvips is used
%\documentclass[dvipsone,opre]{informs3}   % if dvipsone is used, etc.

%%% OPRE uses endnotes. If you do not use them, put a percent sign before
%%% the \theendnotes command. This template does show how to use them.
%\usepackage{endnotes}
%\let\footnote=\endnote
%\let\enotesize=\normalsize
%\def\notesname{Endnotes}%
%\def\makeenmark{$^{\theenmark}$}
%\def\enoteformat{\rightskip0pt\leftskip0pt\parindent=1.75em
%  \leavevmode\llap{\theenmark.\enskip}}

% Private macros here (check that there is no clash with the style)

% Natbib setup for author-year style
\usepackage{natbib}
 \bibpunct[, ]{(}{)}{,}{a}{}{,}%
 \def\bibfont{\small}%
 \def\bibsep{\smallskipamount}%
 \def\bibhang{24pt}%
 \def\newblock{\ }%
 \def\BIBand{and}%


\makeatletter
\g@addto@macro{\UrlBreaks}{\UrlOrds}
\makeatother

%% Setup of theorem styles. Outcomment only one.
%% Preferred default is the first option.
%\TheoremsNumberedThrough     % Preferred (Theorem 1, Lemma 1, Theorem 2)
%\TheoremsNumberedByChapter  % (Theorem 1.1, Lema 1.1, Theorem 1.2)
%\ECRepeatTheorems

%% Setup of the equation numbering system. Outcomment only one.
%% Preferred default is the first option.
%\EquationsNumberedThrough    % Default: (1), (2), ...
%\EquationsNumberedBySection % (1.1), (1.2), ...

% In the reviewing and copyediting stage enter the manuscript number.
%\MANUSCRIPTNO{} % When the article is logged in and DOI assigned to it,
                 %   this manuscript number is no longer necessary

%% useful math macros
\newcommand{\field}[1]{\ensuremath{\mathbb{#1}}}
\newcommand{\N}{\ensuremath{\field{N}}} % natural numbers
\newcommand{\R}{\ensuremath{\field{R}}} % real numbers
\newcommand{\C}{\ensuremath{\field{C}}} % real numbers
\newcommand{\Rp}{\ensuremath{\R_+}} % positive real numbers
\newcommand{\Z}{\ensuremath{\field{Z}}} % integers
\newcommand{\Zp}{\ensuremath{\Z_+}} % positive integers
\newcommand{\1}{\ensuremath{\mathbf{1}}} % vector of all 1's
\newcommand{\ind}[1]{\ensuremath{\mathbbm{1}\left(#1\right)}} % indicator function
\newcommand{\Inb}[1]{\ensuremath{\mathbb{I}_{#1}}} % indicator function, no brackets
\newcommand{\tends}{\ensuremath{\rightarrow}} % arrow for limits
\newcommand{\ra}{\ensuremath{\rightarrow}} % abbreviation for right arrow
\newcommand{\PR}{\ensuremath{\mathsf{P}}} % probability
\newcommand{\E}{\ensuremath{\mathsf{E}}} % expectation
\newcommand{\defeq}{\ensuremath{\triangleq}}
\newcommand{\subjectto}{\text{\rm subject to}} % subject to
\renewcommand{\Re}{\ensuremath{\R}} % expectation
%% some caligraphic symbols
\newcommand{\Ascr}{\ensuremath{\mathcal A}}
\newcommand{\Bscr}{\ensuremath{\mathcal B}}
\newcommand{\Cscr}{\ensuremath{\mathcal C}}
\newcommand{\Dscr}{\ensuremath{\mathcal D}}
\newcommand{\Escr}{\ensuremath{\mathcal E}}
\newcommand{\Fscr}{\ensuremath{\mathcal F}}
\newcommand{\Gscr}{\ensuremath{\mathcal G}}
\newcommand{\Hscr}{\ensuremath{\mathcal H}}
\newcommand{\Iscr}{\ensuremath{\mathcal I}}
\newcommand{\Jscr}{\ensuremath{\mathcal J}}
\newcommand{\Kscr}{\ensuremath{\mathcal K}}
\newcommand{\Lscr}{\ensuremath{\mathcal L}}
\newcommand{\Mscr}{\ensuremath{\mathcal M}}
\newcommand{\Nscr}{\ensuremath{\mathcal N}}
\newcommand{\Oscr}{\ensuremath{\mathcal O}}
\newcommand{\Pscr}{\ensuremath{\mathcal P}}
\newcommand{\Qscr}{\ensuremath{\mathcal Q}}
\newcommand{\Rscr}{\ensuremath{\mathcal R}}
\newcommand{\Sscr}{\ensuremath{\mathcal S}}
\newcommand{\Tscr}{\ensuremath{\mathcal T}}
\newcommand{\Uscr}{\ensuremath{\mathcal U}}
\newcommand{\Vscr}{\ensuremath{\mathcal V}}
\newcommand{\Wscr}{\ensuremath{\mathcal W}}
\newcommand{\Xscr}{\ensuremath{\mathcal X}}
\newcommand{\Yscr}{\ensuremath{\mathcal Y}}
\newcommand{\Zscr}{\ensuremath{\mathcal Z}}

%% Macros introduced by Eli...... %%
% Bandit notation
\newcommand{\Regret}[1]{\textnormal{Regret} \left( #1 \right)}
% Probability notation not already included
\newcommand\given[1][]{\:#1\vert\:}
\renewcommand{\P}[1]{\mathbb{P} \left( #1 \right)}
% For some reason these operators are not included in the LaTeX packages
\DeclarePairedDelimiter\ceil{\lceil}{\rceil}
\DeclarePairedDelimiter\floor{\lfloor}{\rfloor}

%% math operators
\DeclareMathOperator{\diag}{diag}
\DeclareMathOperator{\mean}{mean}
\DeclareMathOperator{\Var}{Var}
\DeclareMathOperator{\vect}{vec}
%\DeclareMathOperator{\vec}{vec}
\DeclareMathOperator{\interior}{int}
\DeclareMathOperator{\st}{subject\;to}
\DeclareMathOperator*{\argmin}{\mathrm{argmin}}
\DeclareMathOperator*{\argmax}{\mathrm{argmax}}
\newcommand{\minimize}{\ensuremath{\mathop{\mathrm{minimize}}\limits}}
\newcommand{\maximize}{\ensuremath{\mathop{\mathrm{maximize}}\limits}}
%% theorem environments
\newtheoremstyle{thm-sf}{}{}{\itshape}{}{\sffamily\bfseries}{.}{ }{}
\theoremstyle{thm-sf}
\newtheorem{remark}{Remark}
\newtheorem{assumption}{Assumption}
\newtheorem{definition}{Definition}
\newtheorem{example}{Example}
\newtheorem{theorem}{Theorem}
\newtheorem{conjecture}{Conjecture}
\newtheorem{corollary}{Corollary}
\newtheorem{lemma}{Lemma}
\newtheorem{fact}{Fact}
\newtheorem{proposition}{Proposition}
\renewcommand{\qedsymbol}{\ensuremath{\blacksquare}}
%\renewcommand{\proofname}{{\normalfont\sffamily\bfseries Proof}}
\renewcommand{\proofname}[1]{{\normalfont\sffamily\bfseries #1}}

\newenvironment{myproof}[1][\proofname]{%
  \proof[\normalfont\sffamily\bfseries #1]%
}{\endproof}

%% Misc. new symbols
\newcommand{\calN}{\mathcal{N}}
\newcommand{\calS}{\mathcal{S}}
\newcommand{\calM}{\mathcal{M}}
\newcommand{\bone}{\mathbf{1}}
\newcommand{\prods}{\calN}
\newcommand{\nprods}{N}
\newcommand{\shelf}{\Mscr}
\newcommand{\shelfcap}{C}
\newcommand{\profit}{p}
\newcommand{\expprofit}{R}
\newcommand{\permgrp}{S}
\newcommand{\sPur}{\calS_j(\shelf)}
\newcommand{\sPurbar}{\overline{\calS}_j(\shelf)}
\newcommand{\sPurd}{\calS_{jd}(\shelf)}
\newcommand{\sPurbard}{\overline{\calS}_{jd}(\shelf)}
\newcommand{\defined}{\overset{\defi}{=}}
\newcommand{\set}[1]{\left\{ #1 \right\}}
\newcommand{\co}[1]{\conv \left( #1 \right)}
\newcommand{\Pairs}{\mathcal{P}}
\newcommand{\Triples}{\mathcal{T}}
\newcommand{\calI}{\mathcal{I}}
\newcommand{\calK}{\mathcal{K}}
\newcommand{\event}{\mathcal{E}}
\newcommand{\normzero}[1]{\lVert #1 \rVert_{0}}
\newcommand{\Ee}[1]{\E \left[ #1 \right]}
\newcommand{\abs}[1]{\lvert #1 \rvert}


\newcommand{\hM}{\hat{M}}
\newcommand{\hS}{\hat{S}}
\newcommand{\hU}{\hat{U}}
\newcommand{\hV}{\hat{V}}
\newcommand{\hX}{\hat{X}}

\newcommand{\Ln}{\left\|}
\newcommand{\Rn}{\right\|}
\newcommand{\La}{\left\langle}
\newcommand{\Ra}{\right\rangle}

\renewcommand{\Pr}{\mathbb{P}}
\newcommand{\T}{^\intercal}
\newcommand{\Ber}{{\rm Ber}}
\newcommand{\Uni}{{\rm Uniform}}
\newcommand{\Beta}{{\rm Beta}}
\newcommand{\eps}{\epsilon}
\newcommand\numberthis{\addtocounter{equation}{1}\tag{\theequation}}

%% Tikz customizations
\usetikzlibrary{arrows,patterns,plotmarks}
\tikzstyle{every picture} += [>=stealth]
%% customize section titles
\allsectionsfont{\sffamily}


\makeatletter
\def\@seccntformat#1{\csname the#1\endcsname.\quad}
\makeatother



%% abstract, table, figure names
\renewcommand{\figurename}{\normalfont\sffamily\bfseries Figure}
\renewcommand{\tablename}{\normalfont\sffamily\bfseries Table}
\renewcommand{\abstractname}{\normalfont\sffamily\bfseries Abstract}
%% other customizations
%\numberwithin{equation}{section}
\floatstyle{ruled}
\newcommand{\emailhref}[1]{\href{mailto:#1}{\tt #1}} % hyperlinked email address
%% boolean for fast compilation
\provideboolean{fastcompile}
\newcommand{\hidefastcompile}[1]{\ifthenelse{\boolean{fastcompile}}{}{#1}}
%%%%%%%% end of generic header

\usepackage[margin=1in]{geometry}
% spacing for Operations Research
\usepackage{setspace}
%\topmargin -0.925in \textwidth 6.5in \textheight 8.75in
%\topmargin -0.525in \textwidth 6.5in \textheight 8.75in
%\oddsidemargin 0.0in
%\doublespacing
%\onehalfspacing
%\renewcommand{\baselinestretch}{1.5}
%\linespread{1.45}
\setstretch{1.00}

%\setlength\topmargin{0pt}
%\addtolength\topmargin{-\headheight}
%\addtolength\topmargin{-\headsep}
%\setlength\oddsidemargin{0pt}
%\setlength\textwidth{\paperwidth}
%\addtolength\textwidth{-2in}
%\setlength\textheight{\paperheight}
%\addtolength\textheight{-2in}



% for float placement
\renewcommand{\textfraction}{0.05}
\renewcommand{\topfraction}{0.95}
\renewcommand{\bottomfraction}{0.95}
\renewcommand{\floatpagefraction}{0.35}

% spacing of captions
%\setlength{\abovecaptionskip}{-5pt}

% citations


%\setcitestyle{authoryear,round,semicolon,aysep={,},yysep={,},notesep={, }}
%\bibliographystyle{plain}


% plots, figures
%\usepackage{pgfplots}
%\usetikzlibrary{calc}

% for editing comments
\newcommand{\todo}[1]{{\color{red} #1}}
\newcommand{\vff}[1]{{\color{blue} \noindent {\sffamily\bfseries VFF:} #1}}

% theorems stated more than once
\newtheorem*{le:ell}{Lemma~\ref{le:ell}}
\newtheorem*{le:feas}{Lemma~\ref{le:feas}}
\newtheorem*{le:sample_complexity}{Lemma~\ref{le:sample_complexity}}
\newtheorem*{p1}{Proposition~1}
% uncomment to skip figures and compile quickly
%\setboolean{fastcompile}{true}


\usepackage{setspace}
%\topmargin -0.525in \textwidth 6.5in \textheight 8.75in
%\oddsidemargin 0.0in

\title{\textsf{\textbf{Optimistic Gittins Indices}}}
%\author{Vivek F. Farias}
%\affil{Sloan School of Management \\ MIT \\ email}
%\author{Andrew A. Li}
%\affil{Operations Research Center \\ MIT \\ email}
\author{Vivek F. Farias \\ Sloan School of Management \\ Massachusetts Institute of Technology \\ email: \url{vivekf@mit.edu}
\and Eli Gutin \\ Operations Research Center \\ Massachusetts Institute of Technology \\ email: \url{gutin@mit.edu}}
\date{}

        %%%%%%%%%%%%%%%%%%%%%%%%%%%%%%
        \oddsidemargin  0.0in
        \evensidemargin 0.0in
        \textwidth      6.5in
        \headheight     -0.0in
        \topmargin       -0.55in
        \textheight 9.0in
        %%%%%%%%%%%%%%%%%%%%%%%%%%%%%%

\begin{document}
\maketitle


% Outcomment only when entries are known. Otherwise leave as is and
%   default values will be used.
%\setcounter{page}{1}
%\VOLUME{00}%
%\NO{0}%
%\MONTH{Xxxxx}% (month or a similar seasonal id)
%\YEAR{0000}% e.g., 2005
%\FIRSTPAGE{000}%
%\LASTPAGE{000}%
%\SHORTYEAR{00}% shortened year (two-digit)
%\ISSUE{0000} %
%\LONGFIRSTPAGE{0001} %
%\DOI{10.1287/xxxx.0000.0000}%

% Author's names for the running heads
% Sample depending on the number of authors;
% \RUNAUTHOR{Jones}
% \RUNAUTHOR{Jones and Wilson}
% \RUNAUTHOR{Jones, Miller, and Wilson}
% \RUNAUTHOR{Jones et al.} % for four or more authors
% Enter authors following the given pattern:
%\RUNAUTHOR{Farias and Gutin}

% Title or shortened title suitable for running heads. Sample:
% \RUNTITLE{Bundling Information Goods of Decreasing Value}
% Enter the (shortened) title:
%\RUNTITLE{Learning Preferences with Side-Information}

% Full title. Sample:
% \TITLE{Bundling Information Goods of Decreasing Value}
% Enter the full title:
%\TITLE{Learning Preferences with Side-Information:\\ Near Optimal Recovery of Tensors}

% Block of authors and their affiliations starts here:
% NOTE: Authors with same affiliation, if the order of authors allows,

\begin{abstract}
\noindent
Starting with the Thomspon Sampling algorithm, recent years have seen a resurgence of interest in Bayesian algorithms for the Multi-armed Bandit (MAB) problem. These algorithms seek to exploit prior information on arm biases and while several have been shown to be regret optimal, their design has not emerged from a principled approach. In contrast, if one cared about Bayesian regret discounted over an infinite horizon at a {\em fixed, pre-specified} rate, the celebrated Gittins index theorem offers an {\em optimal} algorithm. Unfortunately, the Gittins analysis does not appear to carry over to minimizing Bayesian regret over all sufficiently large horizons and computing a Gittins index is onerous relative to essentially any incumbent index scheme for the Bayesian MAB problem. 

The present paper proposes a sequence of `optimistic' approximations to the Gittins index. We show that the use of these approximations in concert with the use of an increasing discount factor appears to offer a compelling alternative to state-of-the-art index schemes proposed for the Bayesian MAB problem in recent years by offering substantially improved performance with little to no additional computational overhead. In addition, we prove that the simplest of these approximations yields frequentist regret that matches the Lai-Robbins lower bound, including achieving matching constants. 

\vskip 5pt
\noindent {\it Keywords:} multi-armed bandits; Gittins index; online learning;
\end{abstract}

%\maketitle
%%%%%%%%%%%%%%%%%%%%%%%%%%%%%%%%%%%%%%%%%%%%%%%%%%%%%%%%%%%%%%%%%%%%%%
%\setstretch{1.45}
\setstretch{1.3}

% Actual content here:
\section{Introduction} \label{sec:intro}

%points to make: (1) current bayesian landscape appears to offer v. good algos. (2) Bayesian setup yields an MDP, but this has not been leveraged. (3) Gitins index is optimal for MDPs. (4) Challenges with Gittins. (5) Our contribution. 

The Multi-Armed Bandit (MAB) problem is perhaps the simplest example of a learning problem that exposes the tension between exploration and exploitation. In its simplest form, we are given a collection of random variables or `arms'. By adaptively sampling these random variables, we seek to eventually sample consistently from the random variable with the highest mean. This is typically formalized by asking that we minimize cumulative `regret'; a notion we make precise in a later section. 

Recent years have seen a resurgence of interest in {\em Bayesian} algorithms for the MAB problem. In this variant of the MAB problem, we are endowed with a prior on arm means, and a number of algorithms that exploit this prior have been proposed and analyzed. These include Thompson Sampling \citep{thompson1933likelihood}, Bayes-UCB \citep{kaufmann2012thompson}, KL-UCB \citep{garivier2011kl}, and Information Directed Sampling \citep{russo2014learning}. The ultimate motivation for these algorithms appears to be the empirical performance they offer. Specifically, these Bayesian algorithms appear to incur smaller regret than their frequentist counterparts such as \cite{auer2002finite}, even when regret is measured in a frequentist sense. This empirical evidence has, very recently, been reinforced by theoretical performance guarantees. For instance, it has been shown that both Thompson sampling and Bayes-UCB enjoy upper bounds on frequentist regret that match the Lai-Robbins lower bound \citep{lai1985asymptotically}. Interestingly, even amongst the various Bayesian algorithm proposed there appears to be a wide range in empirical performance. For instance, empirical evidence presented in \cite{russo2014learning} suggests that the IDS algorithm offer a substantial improvement in frequentist regret over Thompson sampling and the Bayes-UCB algorithm, among others. The former algorithm does not however enjoy the optimal data dependent frequentist regret bounds that the latter two do. Perhaps more importantly, these algorithms also vary substantially in their design (as opposed to being variations on a theme).    

Now a prior on arm means endows us with the structure of a Markov Decision Process (MDP) and none of the Bayesian algorithms alluded to above exploit this structure. This is especially surprising in light of the celebrated Gittins Index Theorem. That breakthrough result proved the optimality of a certain index policy for a {\em horizon dependent} variant of the Bayesian MAB. Specifically, imagine that we cared about the expected (Bayes) regret incurred over an exponentially distributed horizon, where the mean horizon length is known to the algorithm designer. This problem is nominally a high dimensional MDP. Gittins, however, proved that a simple to compute index rule was optimal for this task resolving a problem that had remained open for several decades \citep{gittins1979bandit}. Why does the Gittins Index Theorem not immediately help resolve the design of an optimal algorithm for the variant of the Bayesian MAB problem that is the subject of the approaches discussed in the preceding paragraph? As we will discus more carefully in our literature review, this is certainly not from lack of research effort \citep{lattimore2016bayesregret}. In fact, one must deal with several substantial challenges:
\begin{enumerate}
\item
Dependence on Horizon: The notion of regret optimality as popularized by \cite{lai1985asymptotically} is `anytime'. Colloquially, this can be thought of as follows: we desire an algorithm that performs well for {\em any} time horizon. This fact is fundamentally at odds with Gittins' variant of the MAB problem that (via a discount factor) effectively specifies a (exponentially distributed) horizon. Gittins' result is intimately connected to this choice of horizon; even seemingly minor changes appear to render the problem intractable. For instance, it is known that a Gittins-like index strategy is sub-optimal for a fixed, finite-horizon \citep{berry1985bandit}. Algorithms for other notions of optimality that one may reasonably conjecture are better aligned with `anytime' regret optimality (such as, say, Cesaro-overtaking optimality) are similarly elusive \citep{katehakis1996finite}.  
\item
Computation: Separate from the issues made in the previous point, consider the task of computing a Gittins index at every point in time. 
The computation of a Gittins index can be reduced to the solution of a certain infinite horizon stopping problem. For the Bayesian MAB, the state space for this problem must describe all possible posteriors one may encounter on a given arm. Assuming conjugate priors, one may hope for a finite dimensional state space, but tractable computation will typically call for some form of state-space truncation. This computation is far more onerous than any of the aforementioned indices. Furthermore, it is reasonable to conjecture that as time progresses one may require increasingly more accurate estimates of the Gittins index, which further complicates computation, and calls into question the correctness of a naive state-space truncation scheme. 
\end{enumerate}
Against this backdrop, the present paper makes the following contribution: 
\newline
\newline
\noindent
{\em We show that picking arms according to a certain tractable approximation to their Gittins index, computed for a time dependent discount factor we characterize precisely, constitutes a regret optimal bandit policy. The resulting index rule is both simple to compute and in computational experiments appears to outperform state-of-the-art bandit algorithms by a material margin.} 
\newline
\newline
In greater detail, we outline our contributions as follows:

\begin{enumerate}

\item Optimistic Approximations: We propose a sequence of `optimistic' approximations to the Gittins index. These optimistic approximations can be interpreted as providing a tightening sequence of upper bounds on the optimal stopping problem defining a Gittins index, yielding the index itself in the limit. The computation associated with the simplest of these approximations is no more burdensome than the computation of indices for the Bayes UCB algorithm, and several orders of magnitude faster than the best performing alternative from an empirical perspective (the IDS algorithm). 
\item Regret Optimality: We establish that an arm selection rule that is greedy with respect to any optimistic approximation to the Gittins index achieves optimal regret in the sense of meeting the Lai-Robbins lower bound (including matching constants) for the canonical case of Beta-Bernoulli bandits. A crucial ingredient required for this scheme to work is that as time progresses, the discount factor employed in computing the index must be increased at a certain rate which we characterize precisely. This implicitly resolves the challenge of horizon dependence. 
\item Empirical Performance: We show empirically that even the simplest optimistic approximation to the Gittins index outperforms the state-of-the-art incumbent schemes discussed in this introduction by a non-trivial margin. Our empirical study is careful to recreate several ensembles of problem instances considered by previous authors (including a particularly computationally intensive study by \cite{chapelle2011empirical} that prompted the reexamination of the Thompson sampling algorithm in recent years). The margin of improvement we demonstrate increases further as one employs successfully tighter optimistic approximations, at the cost of computational effort. 
%We view these: the Bayesian MAB problem is fundamental, making the performance improvements we demonstrate important.  
\end{enumerate}

In summary, we propose a new index rule for the Baysian MAB problem that employs Gittins indices in a novel way. This new index rule enjoys the strongest possible data-dependent regret guarantees while also offering excellent empirical performance. 

%The ultimate motivation for these algorithms appears to be two-fold: superior empirical performance and light computational burden. The strongest performance results available for these algorithms establish regret lower bounds that match the Lai-Robbins lower bound \citep{lai1985asymptotically}.  Even among this set of recently proposed algorithms, there is a wide spread in empirically observed performance. Table~\ref{table:intro_algorithm_summary} lists well-known algorithms in the literature and what is known about them.

%\begin{table}
%	\centering
%	\begin{tabular}{@{}llll}\toprule
%		Algorithm & Bayes/Frequentist & Regret Optimal & Framework \\ \midrule
%		KL-UCB & Frequentist & Yes & Index-based \\
%		UCB & Frequentist & Unknown & Index-based \\
%		MOSS & Frequentist & Unknown & Index-based \\
%		Thompson Sampling &Bayes & Yes & Posterior Sampling \\
%		Bayes UCB & Bayes & Yes & Index-based \\
%		IDS & Bayes & Unknown  & Mixed \\
%		Gittins Index & Bayes & No & Index-based \\
%		\bottomrule
%	\end{tabular}
%	\caption{Summary of some famous bandit policies and their properties.}
%	\label{table:intro_algorithm_summary}
%\end{table}

%Interestingly, the design of the index policies referenced above has been somewhat ad-hoc as opposed to having emerged from a principled analysis of the underlying Markov Decision Process. Now if in contrast to requiring `small' regret for all sufficiently large time horizons, we cared about minimizing Bayesian regret over an infinite horizon, discounted at a fixed, pre-specified rate (or equivalently, maximizing discounted infinite horizon rewards), the celebrated Gittin's index theorem provides an {\em optimal, efficient} solution. Importing this celebrated result to the fundamental problem of designing algorithms that achieve low regret (either frequentist or Bayesian) simultaneously over all sufficiently large time horizons runs into two substantial challenges:

%\noindent {\em High-Dimensional State Space: }Even minor `tweaks' to the discounted infinite horizon objective appear to render the corresponding Markov Decision problem for the Bayesian MAB problem intractable. For instance, it is known that a Gittins-like index strategy is sub-optimal for a fixed, finite-horizon \citep{berry1985bandit}. Moreover, the problem of minimizing regret simultaneously over all sufficiently large horizons is not well understood.
%\newline
%\newline
%\noindent {\em Computational Burden: }Even in the context of the discounted infinite horizon problem, the computational burden of calculating a Gittins index is substantially larger than that required for any of the index schemes for the multi-armed bandit discussed thus far. 

%The present paper attempts to make progress on these challenges. Specifically, we make the following contributions:
%\begin{itemize}
%	\item We propose a class of `optimistic' approximations to the Gittins index that can be computed with significantly less effort. In fact, the computation of the simplest of these approximations is no more burdensome than the computation of indices for the Bayes UCB algorithm, and several orders of magnitude faster than the nearest competitor, IDS. 
%	\item We establish that an arm selection rule that is greedy with respect to the simplest of these optimistic approximations achieves optimal regret in the sense of meeting the Lai-Robbins lower bound (including matching constants) provided the discount factor is increased at a certain rate.
%	\item We show empirically that even the simplest optimistic approximation to the Gittins index proposed here {\em outperforms the state-of-the-art incumbent schemes discussed in this introduction by a non-trivial margin}. We view this as our primary contribution -- the Bayesian MAB problem is fundamental making the performance improvements we demonstrate important.  
%\end{itemize}



\subsection{Relevant Literature}
We organize our literature review around the primary topics that this paper touches on. The study of exploration-exploitation problems is vast, even if it is restricted to a problem with a finite number of arms. Consequently, our review will be focused on stochastic, non-contextual, versions of the MAB problem. Even with this restriction, the literature remains vast, and so we focus on papers that are either seminal in nature or particularly relevant to our own work; this review is by no means a comprehensive with respect to the MAB problem. 
\newline
\noindent\textbf{\textsf{Regret optimality and the bandit problem: }}\cite{robbins1952some} motivated the study of the MAB problem and left open questions on how to design effective policies. Since then \cite{lai1985asymptotically} proved a cornerstone result, namely an asymptotic lower bound on regret that any consistent strategy incurs. The same paper proposes an upper-confidence bound (UCB) algorithm that asymptotically achieves the lower bound. Computationally efficient UCB algorithms were developed in \citep{agrawal1995sample, katehakis1995sequential}. Later, \cite{auer2002finite} and \cite{audibert2010regret} proved finite time regret bounds for UCB algorithms and demonstrated ways to tune them in order to improve performance. Other algorithms are proposed in \citep{garivier2011kl,maillard2011finite} where the confidence bounds are calculated using the KL-divergence function. Those authors provide a finite-time analysis and their algorithms are shown to achieve the Lai-Robbins bound.
\newline
\noindent\textbf{\textsf{Bayesian bandit algorithms: }} Another powerful approach to bandit problems is to work with a Bayesian prior to model one's uncertainty about an arm's expected reward. \cite{lai1987adaptive}  proves an asymptotic lower bound on Bayes' risk and develops a horizon-dependent algorithm that achieves it.
%The advantage of such algorithms is that they can make use of existing knowledge about an arm, such as the family of distributions that its rewards comes from, to make better decisions.
Thompson Sampling \citep{thompson1933likelihood}, one of the earliest algorithms proposed for the MAB problem, is in fact a Bayesian one. Empirical studies in \citep{chapelle2011empirical,scott2010modern} highlight Thompson Sampling's hugely superior performance over some UCB algorithms even when the prior is mismatched. A series of tight regret bounds for Thompson Sampling are proven in \citep{agrawalanalysis,agrawal2013further} and \citep{kaufmann2012thompson}. For specific instances such as the Beta-Bernoulli bandit problem, Thompson Sampling was proven to be asymptotically optimal. Recently, \cite{korda2013thompson} generalized the aforementioned results to bandit problems where the arm distributions belong to a 1-D exponential family. Interestingly enough, \cite{robbins1952some} seems to have been unaware of Thompson Sampling and its effectiveness in the non-Bayesian setting.

Several other Bayesian algorithms exist. \cite{kaufmann2012thompson} propose Bayes UCB, which they show is competitive with Thompson Sampling. The main idea behind Bayes UCB is to treat quantiles of the arm's prior as an upper confidence bound and let the quantile grows at some pre-specified rate. \cite{russo2014learning} propose Information Directed Sampling (IDS), an algorithm that exploits information theoretic quantities arising from the prior distributions over the arms. In simulations, IDS is shown to dominate many of the aforementioned algorithms, including Thompson Sampling, Bayes UCB and KL-UCB. In our empirical investigation, we will see that IDS is the closest competitor to the approach we propose here (we recreate the experiments from \cite{russo2014learning}).

\noindent\textbf{\textsf{Gittins index and its approximations: }}
There is another stream of literature that models the MAB problem as an MDP. For the case of two arms, where one arm's reward is deterministic, \cite{bradt1956sequential} show that for this one-dimensional DP, an index rule is an optimal strategy. When the objective to maximize the infinite sum of expected \emph{discounted} rewards \cite{gittins1979bandit} shows that another index policy is also optimal (where the index is similar to \citep{bradt1956sequential} but takes discounting into account). Several alternative proofs of the same result are shown in \citep{tsitsiklis1994short,weber1992gittins,whittle1980multi,bertsimas1996conservation}, which also offer their own interpretations of it. 
 
Computing the Gittins index can be an onerous task, especially when the state space corresponding to posterior sufficient statistics is large or high dimensional. As such, several approximations to it have been proposed in \citep{yao2006some,katehakis1987multi,varaiya1985extensions} with a survey in \citep{chakravorty2013multi}. This paper also relies on Gittins index approximations and we develop simple, general ones that enable our algorithm to be regret optimal. 

Finally, we note that others have contemporaneously attempted to leverage the Gittins index in the construction of a Bayesian MAB algorithm. For instance, \cite{kaufmann2016bayesian} considers a variety of heuristics based on a finite horizon version of the Gittins index (essentially, the index proposed by \cite{bradt1956sequential}), and shows promising empirical results. \cite{lattimore2016bayesregret} analyzes the regret under a similar index and shows it to be logarithmic for a {\em fixed} horizon. Unfortunately, the index policies studied in both \citep{kaufmann2016bayesian} and \citep{lattimore2016bayesregret} {\em require a-priori knowledge of a horizon}. As such this does not yield an index rule that works for any sufficiently large horizon, but rather one that only works for a fixed pre-specified horizon. In fact, such schemes cannot be expected to work well for time horizons other than the pre-specified horizon determining the index. In contrast, we seek to provide a compelling alternative to the host of state-of-the-art `anytime' regret optimal index rules discussed heretofore.


 

%we became aware of recent work in \cite{lattimore2016bayesregret} which also focuses on regret minimization using (approximated) Gittins indices. However, the algorithm there is horizon-dependent and the proofs assume a Gaussian prior over the arms. More heuristics inspired by `finite-horizon' Gittins indices have been tested out recently in \cite{kaufmann2016bayesian}.

\subsection{Structure of the paper}
The remainder of this paper is organized as follows: in the next section, we state our notation, objectives of interest 
and key results such as the Lai-Robbins lower bound. The third section focuses on the Gittins Index and explains how it fails to minimize regret in a sense that is made clear later. At the end, we address another issue, namely the computational cost of calculating the Gittins Index, which inspires us to develop the Optimistic Gittins Index (OGI) policy. Section~\ref{sec:analysis_of_regret} establishes an optimal regret bound for the algorithm; namely, one that matches the Lai-Robbins lower bound. Following that, Section~\ref{sec:experiments} presents experiments showing how OGI achieves lower Bayesian regret than state of the art policies and is computationally efficient. In addition to the problem studied in earlier sections of the paper, we also demonstrate computationally the algorithm's effectiveness in a more general setting where it is possible to pull several arms at once in every iteration. Finally, in Section~\ref{sec:conclusions} we state open questions that remain following this work.
\section{Model and Preliminaries} \label{sec:model_and_prelim}

The multi-armed bandit problem is described via a handful of primitives. These include the notion of an `arm', the concept of an arm selection rule or policy and the notion of regret. This section seeks to formalize each of these notions.

\noindent\textbf{\textsf{Arms:}}
We consider a multi-armed bandit problem with $A > 1$ arms. We index arms by $i$ and denote by $\Ascr$ the set of all arm indices, $\{1,\ldots,A\}$. At each point in time, $t \in \N$, we are permitted to select or `pull' a single arm. We denote by $N_i(t)$ the cumulative number of pulls of arm $i$ up to and including time $t$. If arm $i$ were pulled at time $t$, we collect a reward  $X_{i,N_i(t)} \in \R$. 

All random variables are generated on a common probability space $\left(\Omega, \Fscr, \mathbb{P}\right)$. For a given arm $i$, $(X_{i,s}, s \in \N)$ is assumed to be an i.i.d. sequence of random variables, each distributed according to a distribution $p_{\theta_i}(\cdot)$. Denote by $\mu(\theta_i)$ the mean of this distribution. Thus, $\theta_i$ is a parameter specifying the reward distribution for arm $i$ and we denote by $\Theta$ the set of all possible values of $\theta_i$. 
We let 
\[
\theta \triangleq \left(\theta_1,\theta_2, \dots, \theta_A\right)
\]
denote a tuple of the parameters defining the reward distributions for all of the arms. $(X_{i,s}, i \in \Ascr, s \in \mathbb{N})$ is itself assumed to be an independent sequence of random variables so that the arms are independent. 


\noindent\textbf{\textsf{Policies:}} At every point in time, we choose an arm to pull according to some history dependent policy $\pi$. Formally, any policy $\pi$ is specified by an $\Ascr$-valued stochastic process $(\pi_t, t \in \mathbb{N})$. Denote by $\Fscr_t$ the filtration generated by the sequence of indices of the first $t$ arms pulled, as well as their corresponding rewards
\[
\mathcal{F}_t
\triangleq
\sigma\left(
\left(
\pi_s, X_{\pi_s,N_{\pi_s}(s)}
\right)
, s=1,2,\dots,t
\right).
\]
We require that the process $\pi_t$ be $\mathcal{F}_t$-predictable,\footnote{in order to capture the possibility of a randomized policy $\mathcal{F}_t$ must also contain the realization of a random variable describing the randomization, but we ignore this here for notational brevity} and denote by $\Pi$ the space of all such policies. 

\noindent\textbf{\textsf{Frequentist Regret and Regret Optimality:}} Over time, the agent accumulates rewards, and we denote by 
\[
V(\pi, T, \theta) := 
\E \left[
	\sum_{t=1}^T
	X_{
		\pi_t,
		N_{\pi_t}(t)
	}
\Bigg |
\theta
\right]
\] 
the reward accumulated up to time $T$ when using policy $\pi$. Denote by $\mu^*(\theta)$ the maximum expected reward across arms for a given $\theta$:  $\mu^*(\theta) \triangleq \max_i \mu(\theta_i)$. The frequentist regret of a policy over $T$ time periods, for a given $\theta \in \Theta^A$, is the expected shortfall against always pulling the optimal arm for that $\theta$, namely
\[
\Regret{\pi, T, \theta} := 
T \mu^*(\theta) -
V(\pi, T, \theta).
\]
In a seminal paper, \cite{lai1985asymptotically} established a lower bound on achievable regret. They showed that for any policy $\pi \in \Pi$, and any $\theta$ such that the set of arms with expected reward $\mu^*(\theta)$ is a singleton, we must have
\begin{equation}
\label{eq:lai_robbins_lb}
\liminf_T
\frac{
	\Regret{\pi, T, \theta}
}
{
	\log T
}
\geq
\sum_{i}
\frac{\mu^*(\theta) - \mu(\theta_i)}{d_{\rm KL}\left(p_{\theta_i},p_{\theta_{i^*}} \right)}
\end{equation}
where $d_{\rm KL}$ is the Kullback-Liebler divergence. A policy $\pi'$ that achieves this lower bound is considered {\em regret optimal}. Specifically, $\pi'$ is regret optimal iff 
\[
\limsup_T
\frac{
	\Regret{\pi', T, \theta}
}
{
	\log T
}
\leq
\sum_{i}
\frac{\mu^*(\theta) - \mu(\theta_i)}{d_{\rm KL}\left(p_{\theta_i},p_{\theta_{i^*}} \right)}
\]

\noindent\textbf{\textsf{Bayesian Bandits: }}A {\em Bayesian} MAB problem is endowed with additional structure: we are given a prior on $\theta$. Specifically, we suppose that each $\theta_i$ is, in fact, an independent draw according to some prior distribution $q$ that is supported on $\Theta$. We assume that $q$ is conjugate to $p_{\theta_i}$ and that $\Ee{|\mu(\theta_i)|} < \infty$. With a minor abuse of notation, we denote by $y$ the sufficient statistic specifying $q$ and by $\Yscr$ the set of all possible values of $y$. 

An algorithm that leverages knowledge of $q$ will frequently maintain a posterior distribution on $\theta_i$ given observations from that arm. To that end, denote by $q_{i,s}$ the posterior distribution on $\theta_i$ given the first $s$ rewards from that arm, $X_{i,1},X_{i,2},\dots,X_{i,s}$. Denote by $y_{i,s}$ the corresponding values of the sufficient statistic describing the posterior. Of course, $q_{i,0} \triangleq q$. 

Now, one can define a notion of regret that depends on the prior $q$. Specifically, the Bayes risk (or Bayesian regret) for any policy $\pi$ is simply the expected regret over draws of $\theta$ according to the prior $q$:
\[
\Regret{\pi, T} := \int_\Theta \Regret{\pi, T, \theta} q^A\left(d\theta\right).
\]
In yet another landmark paper, \cite{lai1987adaptive} showed that for a restricted class of priors $q$ a similar class of algorithms to those found to be regret optimal in \citep{lai1985asymptotically} were {also `Bayes' optimal in the sense that they achieved a lower bound on the Bayes risk (also established in \cite{lai1987adaptive})}. Interestingly, however, this class of algorithms ignores information about the prior altogether -- i.e. they do not require knowledge of $q$.
However, this class of algorithms is not anytime and does require knowledge of the problem horizon.
A number of algorithms that {\em do} exploit prior information have in recent years received a good deal of attention; these include Thompson sampling \citep{thompson1933likelihood}, Bayes-UCB \citep{kaufmann2012thompson}, KL-UCB \citep{garivier2011kl}, and Information Directed Sampling \citep{russo2014learning}. All of these algorithms maintain a posterior on the mean of an arm, but leverage this posterior in different ways. It has been empirically observed that these approaches offer excellent performance, even in a frequentist sense. In fact, Thompson sampling, Bayes-UCB and KL-UCB have each been shown to be regret optimal in the sense of meeting the lower bound  \eqref{eq:lai_robbins_lb}. 
 
\noindent\textbf{\textsf{The Discounted Infinite Horizon Objective: }}Assuming the structure afforded by the Bayesian setting, i.e. the prior $q$, one may consider a distinct objective to Bayesian regret. Specifically, given some fixed discount factor $\gamma < 1$, one could consider the problem of maximizing discounted infinite horizon rewards. Assume we start with a prior $q$ on the mean of any arm; and as before denote by $y$ the sufficient statistic corresponding to this prior. In the parlance of Markov Decisions Processes, we might refer to this as starting with every arm in state $y$. For a given policy $\pi$, we define the expected discounted infinite horizon reward under that policy according to
\[
V^\pi_\gamma(\mathbf{y}) 
=
\E_{\mathbf{y}}
\left[
	\sum_{t=1}^\infty \gamma^{t-1} X_{\pi_t,N_{\pi_t}(t)}
\right]
\]
where $\mathbf{y}$ is an $A$-tuple with every entry equal to $y$. The subscript on the expectation indicates that $\theta_i$ is drawn according to a prior with sufficient statistic $y$ for each arm $i$. An optimal such policy must solve the problem
\[
V^*_\gamma(\mathbf{y}) 
\triangleq 
\max_{\pi \in \Pi} V^\pi_\gamma(\mathbf{y}). 
\]
This is, of course a challenging MDP in that it has a high dimensional state space ($\Yscr^A$). The celebrated Gittins index theorem (which we present in the next section) provides an approach to computing an optimal policy by instead simply solving a dynamic program on the state space $\Yscr$. 


%We write $V(\pi,T) := \E{V(\pi,T,\theta)}$. 
%
%The regret of a policy over $T$ time periods, for a given $\theta \in \Theta^A$, is the expected shortfall against always pulling the optimal arm, namely
%\[
%\Regret{\pi, T, \theta} := 
%T \mu^*(\theta) -
%V(\pi, T, \theta)
%\]
%
%
%
%We denote by $\mu^*(\theta)$ the maximum expected reward across arms;  $\mu^*(\theta) := \max_i \mu_i(\theta_i)$ and let $i^*$ be an optimal arm. The present paper will focus on the Bayesian setting, and so we suppose that each $\theta_i$ is an independent draw from some prior distribution $q$ over $\Theta$. 
%
%We define a policy, $\pi := (\pi_t, t \in \mathbb{N})$, to be a stochastic process taking values in $\mathcal{A}$. We require that $\pi$ be adapted to the filtration $\mathcal{F}_t$ generated by the history of arm pulls and their corresponding rewards up to and including time $t-1$.
%
%
%
%The Bayesian setting endows us with the structure of a (high dimensional) Markov Decision process. An alternative objective to minimizing Bayes risk, is the maximization of the cumulative reward discounted over an infinite horizon. Specifically, for any positive discount factor $\gamma < 1$, define
%\[
%V_\gamma(\pi) := 
%\E_q \left[
%	\sum_{t=1}^\infty \gamma^{t-1} X_{\pi_t,N_{\pi_t}(t)}
%\right].
%\]
%The celebrated Gittin's index theorem provides an {\em optimal, efficient} solution to this problem that we will describe in greater detail shortly; unfortunately as alluded to earlier even a minor `tweak' to the objective above -- such as maximizing cumulative expected reward over a finite horizon renders the Gittins index sub-optimal \cite{nino2011computing}. 
%
%As a final point of notation, every scheme we consider will maintain a posterior on the mean of an arm at every point in time. We denote by $q_{i,s}$ the posterior on the mean of the $i$th arm after $s-1$ pulls of that arm; $q_{i,1} := q$. Since our prior on $\theta_i$ will frequently be conjugate to the distribution of the reward $X_i$, $q_{i,s}$ will permit a succinct description via a sufficient statistic we will denote by $y_{i,s}$; denote the set of all such sufficient statistics $\mathcal{Y}$. We will thus use $q_{i,s}$ and $y_{i,s}$ interchangeably and refer to the latter as the `state' of the $i$th arm after $s-1$ pulls.  
%

\section{The Optimistic Gittins Index Algorithm} \label{sec:gittins_and_approx}

This section introduces the notion of an optimistic Gittins index, and presents an algorithm for the MAB problem that we will subsequently show is optimal in that it achieves the Lai-Robbins lower bound. We will begin with reviewing the Gittins index theorem for the discounted infinite horizon bandit problem and show that one cannot expect the use of the index from that well known result to yield a regret optimal policy for the MAB problem. We then show that the use of the Gittins index in concert with an increasing discount factor yields poly-logarithmic Bayesian regret. This coarse result motivates the discount factor schedule we eventually propose. Finally, we present a series of `optimistic' approximations to the Gittins index with the view of minimizing the computational burden of index computation. Putting these ingredients together yields the optimistic Gittins index algorithm that is the subject of our paper. The regret optimality of the optimistic Gittins index, for Beta-Bernoulli bandits, is proved in Section~\ref{sec:analysis_of_regret} (Theorem~\ref{thm:frequentist_optimal_bound}). That is our main theoretical result. 

\subsection{The Gittins Index and Regret}

The Gittins index theorem presents a surprisingly simple solution to the problem of computing an optimal policy for the discounted infinite horizon bandit problem. Specifically, the theorem defines for each arm state $y \in \Yscr$, an index we denote $v_\gamma(y)$; we define this index shortly. The theorem shows that an arm selection rule which at every time selects the arm with the highest index is optimal. The result is powerful in that the computation of the index for a given arm requires the solution of an MDP on the state space $\Yscr$, as opposed to solving an MDP on the considerably larger state space $\Yscr^A$. 

One way to compute the Gittins Index $v_\gamma(y)$ for an arm in state $y$ is via the so-called retirement value formulation \citep{whittle1980multi}. Specifically, $v_\gamma(y)$ is defined as the value of $\lambda$ that solves
\begin{equation} \label{eqn:gittins_index}
\frac{\lambda}{1-\gamma} = \sup_{\tau > 0}\E_{y}\left[\sum_{t=1}^{\tau} \gamma^{t-1} X_{i,t} + \gamma^{\tau} \frac{\lambda}{1-\gamma}
%	\given y_{i,1} = y
\right],
\end{equation}
where the subscript on the expectation indicates that the prior on the (say, $i$th) arm's mean at time $t=1$, $y_{i,0}$, equals $y$. If one thought of the notion of retiring as receiving a deterministic reward $\lambda$ in every period, then the value of $\lambda$ that solves the above equation could be interpreted as the per-period retirement reward that makes us indifferent between retiring immediately, and the option of continuing to play arm $i$ with the potential of retiring at some future time. The Gittins index policy itself, which we denote by $\pi^{G,\gamma}$, can succinctly be stated as follows: 
\begin{center}
{\em At time $t$, play an arm in the set 
$\argmax_{i} v_\gamma
\left(
y_{i,N_i(t-1)}
\right)$,
} 
\end{center}
where $N_i(0) \equiv 0$ and $y_{i,0}$ is understood to be the sufficient statistic corresponding to the prior on that arm. Ignoring computational considerations, we cannot hope for the policy $\pi^{G,\gamma}$ to be regret optimal. In fact, as the result below indicates, one cannot even hope for such a policy to be consistent (i.e. have sub-linear regret) in the sense of \cite{lai1985asymptotically}:
{
\color{blue}
\begin{lemma}
\label{le:linearregret}
	
	For any $\gamma > 0$, there exists an instance of the multi armed bandit problem, with parameters $\theta \in \Theta$, for which 
	\[
	{\rm Regret}\left(
	\pi^{G,\gamma},T,\theta
	\right)
	= 
	\Omega(T).
	\]
\end{lemma}
}
The proof, given in Appendix~\ref{proof:linearregret}, rests on the simple fact that for any fixed discount factor, if the posterior means on the two arms are sufficiently apart, the Gittins index policy will pick the arm with the larger posterior mean. The threshold beyond which the Gittins policy `exploits' depends on the discount factor and with a fixed discount factor there is a positive probability that the superior arm is never explored sufficiently so as to establish that it is, in fact, the superior arm. 

\subsection{Increasing Discount Factors yield sub-linear Bayesian Regret}

Lemma~\ref{le:linearregret} tells us that we cannot hope for sub-linear regret by applying the Gittins index policy with a constant discount factor. One may naturally wonder whether an increasing discount factor might fix this issue. Now observe that any schedule of increasing discount factors effectively implies a change in the trade-off between exploration and exploitation. With a fixed discount factor, we have already seen that once the priors between two arms are sufficiently far apart, the Gittins policy will not explore, thereby leading to the possibility of linear regret. As the discount factor increases, the `gap' between priors above which exploration is not justified goes up over time. If we increase this `gap' too fast, we might incur too much exploration. Too slow, and we might incur too little exploration. As such, the schedule at which we increase the discount factor is likely to play a significant role in determining the regret of the resulting policy. 

Now notice that the Gittins index policy for a discount factor $\gamma$ can be viewed as optimal for a {\em random} finite horizon, distributed geometrically with parameter $1-\gamma$. As $\gamma$ approaches one, this may be thought of as a near optimal policy for the fixed finite horizon $1/(1-\gamma)$. Now consider for a moment that we had access to a policy that has optimal $T$ period expected regret (assuming $T$ is known in advance). One way to convert such a policy into a policy that has `low' regret for any $T$ is to employ the so-called doubling trick: Apply the optimal policy for the horizon $T$ for $T$ steps, then the optimal policy for horizon $2T$ for the following $2T$ steps, followed by the optimal policy for $4T$ for the next $4T$ steps, and so-forth. {\color{blue} Intuitively, such doubling tricks leverage finite time horizon results to get so-called `anytime' results by picking an update schedule such that the choice of horizon at any point is roughly consistent with the elapsed time up to that point; \cite{besson2018doubling} provide an insightful analysis of this trick and its broad applicability. Such a policy will be `near'-optimal for any horizon, in a manner we now make precise.}

Consider employing discount factors that increase at roughly the rate $1 - 1/t$; specifically, consider setting 
\[
\gamma_t = 
1 - 
\frac{1}
{2^{\floor{\log_2 t}}}%+1}}
\]
and consider using the policy that at time $t$ picks an arm from the set $\argmax_i \nu_{\gamma_t}(y_{i,N_i(t-1)})$. Denote this policy by $\pi^{\rm D}$. The following proposition shows that this `doubling' policy achieves Bayes risk that is within a factor of $\log T$ of the optimal Bayes risk. Specifically, we have:	and consider using the policy that at time $t$ picks an arm from the set $\argmax_i \nu_{\gamma_t}(y_{i,N_i(t-1)})$. Denote this policy by $\pi^{\rm D}$. The following proposition shows that this `doubling' policy achieves Bayes risk that is within a factor of $\log T$ of the optimal Bayes risk. Specifically, we have:
\begin{proposition}
	\label{prop:gittins_log3T}
	Assume that $q$ satisfies the requirements of Theorem 3 in \cite{lai1987adaptive}. Then, 
	\[
	{\rm Regret}(\pi^{\rm D},T)
	=
	O
	\left(
	\log^4 T
	\right),
	\] 
	where the constant in the big-Oh term depends on the prior $q$ and $A$.
\end{proposition}
The proof of this result (Appendix~\ref{proof:prop_log3T}) relies on showing that the finite horizon regret achieved by using a Gittins index with an appropriate fixed discount factor is within a constant factor of the optimal finite horizon regret. The second ingredient is the doubling trick described above. The coarse analysis above illustrates that the use of the Gittins index policy together with an increasing discount factor does indeed yield an algorithm with sub-linear Bayesian regret. It is worth noting that the result above does not show that such a policy achieves {\em optimal} Bayesian regret (the achievable lower bound being on the order of $\log^2T$ \citep{lai1987adaptive}). The analysis does however suggest a candidate discount rate schedule that we will eventually show to yield a regret optimal policy (in a frequentist sense). 

 
\subsection{Optimistic Approximations to The Gittins Index}\label{sec:approx_agi_deriv}

The retirement value formulation makes clear that computing a Gittins index is equivalent to solving a discounted, infinite horizon stopping problem. Solving this problem requires substantially more computational effort than, say, Thompson Sampling or the Bayes UCB algorithm. In fact, this computation can even be rendered intractable in practice. Specifically, the set $\Yscr$ can be high dimensional; see \citep{chapelle2011empirical} for one such example that arises in the context of contextual news recommendations. This motivates an approximation to the Gittins index that is the subject of this section. Specifically, we introduce a sequence of `optimistic' approximations to the Gittins index that will alleviate computational burden. 

Consider the following alternative stopping problem that requires as input the parameters $\lambda$ (which has the same interpretation as it did before), and $K$, an integer limiting the number of steps that we need to look ahead. For an arm in state $y$ (recall that the state specifies sufficient statistics for the current prior on the arm reward), let $R(y)$ be a random variable distributed as the prior on expected arm reward specified by $y$. Define the retirement value $R_{\lambda,K}(s,y)$ according to 
\[
R_{\lambda,K}(s, y) = 
\begin{cases}
\lambda ,& \text{if } s < K\\
\max\left(\lambda, R(y) \right), & \text{otherwise}
\end{cases}
\]
For a given $K$, the \emph{Optimistic Gittins Index} for arm $i$ in state $y$ is now defined as the value for $\lambda$ that solves
\begin{equation} \label{eqn:ogi_general}
\frac{ \lambda}{1-\gamma} = \sup_{1 \le \tau \leq K}
\E_y\left[
	\sum_{s=1}^{\tau} \gamma^{s-1}X_{i,s} + \gamma^{\tau} \frac{R_{\lambda,K}(\tau, y_{i,\tau-1})}{1-\gamma}
\right],
\end{equation}
where we recall that the subscript on the expectation indicates that $y_{i,0} = y$. 
We denote the solution to this equation by $v_\gamma^{K}(y)$.

Let us interpret the stopping problem above. Assume we choose to retire after {\color{blue}$\tau-1$} pulls of the arm. If {\color{blue}$\tau-1$} were less than $K$, we then receive a reward $\lambda$ per period, over the rest of time, discounted at the rate $\gamma$. This is no different from what happens in the stopping problem defining the usual Gittins index, \eqref{eqn:gittins_index}. On the other hand, unlike that formulation we are forced to retire after the $K$th arm pull if we have not done so already. Should we retire at that time, nature reveals the `true' mean reward of the arm, and we receive the greater of that quantity and $\lambda$ as our per period retirement payoff. In this manner one is better off than in the stopping problem inherent to the definition of the Gittins index, \eqref{eqn:gittins_index}, so that we use the moniker `optimistic'.
%nature reveals the {\em true} mean reward for the arm at time $K+1$ should we choose to not retire prior to that time, which enables the decision maker to then instantaneously decide whether to retire at time $K+1$ or else, never retire. In this manner one is better off than in the stopping problem inherent to the definition of the Gittins index, so that we use the moniker optimistic. 
The following Lemma formalizes this intuition

\begin{lemma} \label{lemma:approx_bound}
$v_\gamma^{K}(y)$ is non-increasing in $K$ for all discount factors $\gamma$ and states $y \in \mathcal Y$.
Moreover, $v^K_\gamma(y) \to v_\gamma(y)$ as $K \to \infty$.	
%	\[
%	v^1_\gamma(y) \geq v^2_\gamma(y) \ldots \geq v_\gamma^{K}(y) \geq v_\gamma(y) \quad \forall K,
%	\]
%	and, furthermore, $v^K_\gamma(y) \to v_\gamma(y)$ as $K \to \infty$.
\end{lemma}
\begin{myproof}[Proof]
	See Appendix~\ref{prf:approx_bound}
\end{myproof}

Now, since we need to look ahead at most $K$ steps in solving the stopping problem implicit in the definition above, the computational burden in index computation is limited. In fact, we will see in a subsequent section that {\em even the choice of $K=1$} will make for a compelling policy. 


\subsection{The Optimistic Gittins Index Algorithm}

The discussion thus far suggests a simple class of bandit algorithms we dub the Optimistic Gittins Index (OGI) algorithm. The algorithm itself requires as input a prior on arm means (as does any Bayesian algorithm for the MAB), and a parameter $K$. 


\noindent The OGI algorithm may be summarized succinctly as follows:

\begin{center}
{\em At time $t$ play an arm in the set $\argmax_i v^K_{\gamma_t}(y_{i,N_i(t-1)})$}
\end{center}

\noindent where $\gamma_t = 1 - \frac{1}{t}$. 

The following Section will establish that the algorithm above achieves the Lai-Robbins lower bound (and thus is regret optimal), for {\em any} finite $K$. We will establish this result for Beta priors and Bernoulli rewards. While we do not state this result formally until the next Section (see Theorem~\ref{thm:frequentist_optimal_bound}), it is worth pausing to reflect on the implications of such a result:
\begin{enumerate}
\item As $K$ grows large the optimistic Gittins index approaches the Gittins index. The result thus establishes that the use of a set of arbitrarily close approximations to the Gittins index with the discount factor schedule $\gamma_t = 1-1/t$ is a regret optimal algorithm. This is a simple, surprising result that bridges two very different flavors of the multi-armed bandit problem. It also suggest the natural conjecture that the use of the Gittins index itself with the discount factor schedule $\gamma_t = 1-1/t$ is a regret optimal algorithm.
\item At the other end, since the result establishes regret optimality for any finite $K$, we have regret optimality for $K=1$. Computing the optimistic Gittins index in this case is a particularly trivial task, and offers the spectre of a computationally practical algorithm. In fact, in Section~\ref{sec:experiments} we shall see precisely this -- the choice of $K=1$ yields an index that materially outperforms a host of state-of-the-art alternatives, while requiring little to no computational overhead relative to even the simplest schemes. 
\end{enumerate}

We end this section, with some brief commentary on computation. For concreteness, let us focus on the case of a Beta-Bernoulli bandit. First, we note that solving the stopping problem implicit in the definition of $v^K_\gamma(y)$ for any given value of the retirement subsidy $\lambda$ requires the solution of a relatively simple dynamic program with just $O(K)$ states. This dynamic program can be solved exactly in $O(K^2)$ time. The optimal value of $\lambda$ can be found by bisection. For small values of $K$ this is substantially less effort than computing a Gittins index. The case of $K=1$ is particularly appealing. There, we note that equation~\eqref{eqn:ogi_general} simplifies to
\begin{equation} \label{eqn:ogi_k1}
\lambda = \E[R(y)] + \gamma \E[(\lambda - R(y))^+].
\end{equation}
This equation is easily solved via a method such as Newton-Raphson\footnote{Intuitively, we solve a `degenerate' stopping problem, where we one is forced to stop at time $1$. Such a problem involves no recourse decisions (or learning) and `nature' immediately reveals the true mean reward after pulling the arm once. This is why the index is especially easy to compute when $K=1$.}.
In fact, the gradients required for the use of the Newton-Raphson approach are often readily available in closed form. To wit, in the case of a Beta prior with sufficient statistics $(a,b)$, \eqref{eqn:ogi_k1} reduces to
\[
	\lambda  =  \frac{a}{a+b}\left(1 - \gamma F^\beta_{a+1,b}(\lambda)\right) + \gamma \lambda F^\beta_{a,b}(\lambda) \triangleq g_{a,b}(\lambda)
\]
wherein we see that $\frac{\partial}{\partial \lambda} g_{a,b}(\lambda)$ can be computed in closed form. This makes the use of the Newton-Raphson method for the solution of the equation $\lambda =  g_{a,b}(\lambda)$ particularly simple. In our computational experiments, we will see that the choice of $K=1$ already provides a material improvement in empirical performance over state-of-the-art alternatives. 


%First, note that when $K=1$, equation~\eqref{eqn:ogi_general} simplifies to
%\begin{equation} \label{eqn:ogi_k1}
%\lambda = \E[R(y)] + \gamma \E[(\lambda - R(y))^+].
%\end{equation}
%The equation for $\lambda$ above can also be viewed as an upper confidence bound to an arm's expected reward. 
%Solving equation~\eqref{eqn:ogi_k1} is often simple in practice, and we list a few examples to illustrate this.
%\begin{example}[Beta]
%	In this case $y$ is the pair $(a,b)$, which specifies a Beta prior distribution. The 1-step Optimistic Gittins Index, is the value of $\lambda$ that solves
%	\begin{align*}
%	\lambda & = \frac{a}{a+b} + \gamma \E[(\lambda - {\rm Beta}(a,b))^+] \\
%	 & =  \frac{a}{a+b}(1 - \gamma F^\beta_{a+1,b}(\lambda)) + \gamma \lambda (1-F^\beta_{a,b}(\lambda))
%	\end{align*}
%	where $F^\beta_{a,b}$ is the CDF of a Beta distribution with parameters $a, b$.
%\end{example}
%
%\begin{example}[Gaussian]
%	Here $y = (\mu,\sigma^2)$, which specifies a Gaussian prior and the corresponding equation is
%	\begin{align*}
%	\lambda & = \mu  + \gamma \E[(\lambda - {\rm \mathcal{N}}(\mu,\sigma^2))^+]  \\
%	& = \mu + \gamma \left[( \lambda - \mu) \Phi\left(\frac{\mu - \lambda}{\sigma}\right) + \sigma\phi\left(\frac{\mu-\lambda}{\sigma}\right)\right].
%	\end{align*}
%\end{example}
%
%Notice that in both the Beta and Gaussian examples, the equations for $\lambda$ are in terms of distribution functions. Therefore it is straightforward to compute a derivative for these equations (which would be in terms of the density and CDF of the prior) and makes finding a solution, using a method such as Newton-Raphson, simple and efficient. 


\section{Analysis and Regret bounds} \label{sec:analysis_of_regret}
We establish a regret bound for the OGI algorithm that applies when the prior distribution $q$ is uniform and arm rewards are Bernoulli. The result shows that the algorithm, in that case, meets the Lai-Robbins lower bound and is thus asymptotically optimal in both a frequentist and Bayesian sense. After stating the main theorem, we briefly discuss two generalizations to the algorithm.

In the sequel, we will simplify notation  and let $d(x,y) := d_{\rm KL}(\text{Ber}(x),\text{Ber}(y))$ denote the KL divergence between Bernoulli random variables with parameters $x$ and $y$. We will also refer to the OGI policy, which uses a look-ahead parameter of $K$, as $\pi^{{\rm OG},K}$ and will write the Optimistic Gittins index of the $i$\textsuperscript{th} arm at time $t$ as $v^K_{i,t} \defeq v^K_{1-1/t}(y_{i,N_i(t-1)})$. That way, for the sake of brevity, we will suppress the index's dependence on $y_{i,N_i(t-1)}$. We are ready to state the main result below.
\begin{theorem} \label{thm:frequentist_optimal_bound}
	Let $\epsilon > 0$ and consider an OGI policy configured with a parameter $K \in \N$ and that assumes Beta($1,1$) priors. For the multi-armed bandit problem with Bernoulli rewards and any parameter vector $\theta \subset [0,1]^A$, there exists $T^* = T^*(\epsilon, \theta)$ and $C = C(\epsilon,\theta)$ such that for all $T \ge T^*$,
	\begin{equation}
	\Regret{\pi^{{\rm OG},K}, T, \theta} \le \sum_{\substack{i=1,\ldots,A \\ i \ne i^*}} \frac{(1+\epsilon)^2(\theta^* - \theta_i)}{d(\theta_i, \theta^*)} \log T  + C(\epsilon,\theta)
	\end{equation}
	where $C(\epsilon,\theta)$ is a constant that is determined by $\epsilon$ and the parameter $\theta$.
\end{theorem}
\begin{myproof}[Proof.]	
	Assume, without loss of generality, that the first arm is uniquely optimal so that $\theta^* = \theta_1$. Fix an arbitrary sub-optimal arm, which for convenience we will say is the second arm. We will strategically fix three constants in between the expected rewards of the first and second arms, namely $\theta_1$ and $\theta_2$. In particular, we let $\eta_1,\eta_2,\eta_3 \in (\theta_2, \theta_1)$ be chosen such that $\eta_1 < \eta_2 < \eta_3$, $d(\eta_1, \eta_3) = \frac{d(\theta_2, \theta_1)}{1+\epsilon}$ and $d(\eta_2,\eta_3) =\frac{d(\eta_1, \eta_3)}{1+\epsilon} $. Next, we define the constant $L(T) := \frac{\log T}{d(\eta_2,\eta_3)}$ to be, intuitively, the optimal length of the exploration period.
	
	The main step in this proof will be to upper bound the expected number of pulls of the second arm, as follows,
	\begin{align}
	\E[N_2(T)] & \le L(T) + \sum_{t=\floor{L(T)}+1}^T \P{\pi^{{\rm OG},K}_t = 2, \; N_2(t-1) \ge L(T)} \nonumber \\
	& \le L(T) +   \sum_{t=1}^T \P{v^K_{1,t} < \eta_3} + \sum_{t=1}^T \P{\pi^{{\rm OG},K}_t = 2,\; v^K_{1,t} \ge \eta_3, \; N_2(t-1) \ge L(T)} \nonumber \\
	& \le L(T) +   \sum_{t=1}^T \P{v^K_{1,t} < \eta_3} + \sum_{t=1}^T \P{\pi^{{\rm OG},K}_t = 2,\; v^K_{2,t} \ge \eta_3, \; N_2(t-1) \ge L(T)} \nonumber \\
	& \le \frac{(1+\epsilon)^2 \log T}{d(\theta_2, \theta_1)} + \underbrace{\sum_{t=1}^\infty \P{v^K_{1,t} < \eta_3}}_{A} + \underbrace{\sum_{t=1}^T \P{\pi^{{\rm OG},K}_t = 2,\;v^K_{2,t} \ge \eta_3, \; N_2(t-1) \ge L(T)}}_{B}, \label{bound:final_step_in_freq_regret}
	\end{align}
	where the first step is the same as in the analysis of \cite{auer2002finite} and applies to any bandit policy. All that remains is to show that terms $A$ and $B$ are bounded by constants. These bounds are given in Lemmas~\ref{lemma:underestimation} and \ref{lemma:overestimation} whose proofs we will now describe at a high-level and defer the complete proofs to the Appendix.
	\begin{lemma}[Bound on term A] \label{lemma:underestimation}
		For any $\eta < \theta_1$, the following bounds holds for some constant $C_1 = C_1(\epsilon, \theta_1)$
		\begin{equation*}
		\sum_{t=1}^\infty \P{v^K_{1,t} < \eta} \le C_1.
		\end{equation*}
	\end{lemma}
	\begin{myproof}[Proof outline]
		The goal is to bound $\P{v^K_{1,t} < \eta}$ by an expression that decays fast enough in $t$ so that the series converges. Specifically, we show that there exists a positive constant $h$ so that $\P{v^K_{1,t} < \eta} = O\left(\frac{1}{t^{1+h}}\right)$ using an induction argument. Proving the base case requires using properties specific to Beta and Bernoulli random variables while the inductive step is more general.
		The full proof is contained in Appendix~\ref{proof:underestimation_proof}.
		
		We remark that the core steps in the proof of Lemma~\ref{lemma:underestimation}, at least in the base case of the induction, rely on properties of the Beta and Bernoulli variables. Because of this, we strongly suspect our analysis can strengthen a similar theoretical result for the Bayes UCB algorithm. In particular, the main theorem of \cite{kaufmann2012thompson} states that the quantile parameter in the Bayes UCB algorithm should be $1 - 1/(t \log^c T)$ for some constant $c \ge 5$. However, what is perplexing is that their simulation experiments suggest that using a simpler sequence of quantiles, namely $1 -1/t$, practically results in a lower regret. By utilizing techniques in our analysis, it is possible to prove that the use of the quantiles $1-1/t$ would lead to the same optimal regret bound. Therefore the `scaling' by $\log^c T$ is unnecessary.
	\end{myproof}
	\begin{lemma}[Bound on term B] \label{lemma:overestimation}
		There exists $T^* = T^*(\epsilon, \theta)$ sufficiently large and a constant $C_2 = C_2(\epsilon, \theta_1, \theta_2)$ so that for any $T \ge T^*$, we have
		\begin{equation*}
		\sum_{t=1}^T \P{\pi^{{\rm OG},K}_t = 2,\; v^K_{2,t} \ge \eta_3, \; N_2(t-1) \ge L(T)} \le C_2.
		\end{equation*}
	\end{lemma}
	\begin{myproof}[Proof outline]
		This relies on a concentration of measure result and the assumption that the 2\textsuperscript{nd} arm was sampled at least $L(T)$ times. The full proof is given in Appendix~\ref{proof:overestimation_proof}.
	\end{myproof}
	Lemma~\ref{lemma:underestimation} and \ref{lemma:overestimation}, together with \eqref{bound:final_step_in_freq_regret}, imply that
	\[
	\E[N_2(T)] \le \frac{(1+\epsilon)^2 \log T}{d(\theta_2, \theta_1)} +  C_1 +  C_2,
	\]
	and from this the regret bound follows.
\end{myproof}
\subsection{Generalizations and a tuning parameter}
As we have shown, the OGI algorithm is regret optimal for the Bernoulli bandit problem. However, Theorem~\ref{thm:frequentist_optimal_bound} automatically provides us with another important insight. In particular, \cite{agrawalanalysis} show that \emph{any} bandit algorithm that is regret optimal for the Bernoulli bandit problem, can be modified to yield an algorithm that has $O(\log T)$ regret in a general setting with (bounded) stochastic rewards. Informally, this would require `emulating ' a Bernoulli bandit problem but running the same algorithm as the one just discussed. Thus, we have indirectly given an algorithm with $\mathcal{O}(\log T)$ regret in arbitrary problems with (bounded) stochastic rewards.

% slight modification to Optimistic Gittins Indices gives an algorithm that has $O(\log T)$ regret for the general stochastic bandit problem with bounded rewards. Specifically, when arms have arbitrary reward distributions, bounded in the interval $[a,b]$, the approach is to each time sample the reward $X_{i,t}$, then generate an `artificial' Bernoulli reward with probability $X_{i,t}/(b-a)$ and provide that as input to the policy. The choice of arm pulls from the resulting algorithm leads to an $O(\log T)$ regret bound, as is shown in that paper. The constant in front of $\log T$, however, depends on the KL divergences of \emph{Bernoulli} random variables as opposed to the actual underlying distributions.
Yet another key observation is that the discount factor for Optimistic Gittins Indices does not need to be exactly $1-1/t$. In fact, a tuning parameter can be included to make the discount factor $\gamma_{t+\alpha} = 1-1/(t+\alpha)$ instead. Intuitively, this would encourage a greater degree of `exploration' over the arms. An inspection of the proofs of Lemmas~\ref{lemma:underestimation} and \ref{lemma:overestimation} shows that the result in Theorem~\ref{thm:frequentist_optimal_bound} would still hold were one to use such a tuning parameter. In practice, performance is remarkably robust to our choice of $K$ and $\alpha$.
%\section{Multiple simultaneous arm pulls} \label{sec:multiple_pulls}
In this section we show how the approach in this paper can be applied to a more general Multi-Armed Bandit problem, where the decision maker is able to ``pull" up to a certain number (say $m < A$) of the arms simultaneously. \citep{whittle1988restless} considers a slightly more general version of the problem just discussed, where arms that are not pulled (idled) are able to change state and proposes an index scheme, Whittle's heuristic, for it. However, if arms that are idled are frozen in  state Whittle's heuristic becomes equivalent to pulling arms with the $m$ largest Gittins indices.

For the purposes of this section, we denote by the action space $\mathcal{A}$ the set of all binary vectors with $K$ ones in them, and $X_t$ to be a tuple of (potential) rewards from all $A$ arms at time $t$. A policy  $\pi$ is then a non-anticipative sequence of such vectors and we define its regret over $T$ periods to be
\[
\Regret{\pi, T} = T \cdot \max_{a \in A} a^\top \E[ X_t] - \sum_{t=1}^T \E[\pi_t^\top X_t ]
\]
where the expectation is over both the randomness in the rewards, the prior and the policy's actions.

We give two examples of policies that empirically have low regret. Both rely on the use of an increasing discount factor, that is $\gamma_t = 1 - 1/t$, and on approximating solutions to a sequence of Markov Decision problems whose rewards are discounted by $\gamma_t$ (if it's the $t$\textsuperscript{th} problem). The first of these heuristics involves pulling arms with $m$ largest Optimistic Gittins Indices, which is essentially an approximation to Whittle's heuristic. The second is approximating the solution of the $t$\textsuperscript{th} Markov Decision problem by an using a Linear Programming relaxation of it \citep{bertsimas1996conservation}.
\section{Computational Experiments} \label{sec:experiments}
Our goal is to benchmark Optimistic Gittins Indices (OGI) against state-of-the-art Bayesian algorithms. Specifically, we compare ourselves against Thomson Sampling, Bayes UCB and IDS. Each of these algorithms has in turn been shown to substantially dominate other extant schemes. Our experimental setup closely follows that of \cite{russo2014learning,kaufmann2012bayesian} and \cite{chapelle2011empirical}. 
%The only difference in our paper is we will randomize arm parameters rather than set them to specific constants. This is for consistency and so that we focus on evaluating algorithms purely on their Bayesian performance. 
The experiment from \cite{kaufmann2012bayesian} is deferred to Appendix~\ref{exp:bayes_ucb} because it is brief and sends a similar message to the rest of this section. We conclude with a novel experiment to test the problem with multiple simultaneous arm pulls.

For the majority of experiments, we configure the OGI algorithm with $K =1$ to keep the computational burden under control. In one experiment, included for completeness, we test OGI with $K = 3$ and $K=\infty$, where the latter is equivalent to using Gittins indices. The purpose of those experiments is to show the (limited) value of a higher lookahead in the OGI algorithm. 

We use a common discount factor schedule in all experiments setting $\gamma_t = 1 - 1/(100 + t)$. The choice of $\alpha =å 100$ is second order; our conclusions remain unchanged, and actually appear to improve in an absolute sense with other choices of $\alpha$. In addition, in one experiment we examine the regret of OGI relative to its competitors up to a horizon of $10^6$ epochs, so that this choice of $\alpha$ does not represent an attempt to tune the performance of OGI for a specific time horizon. 



\subsection{Smaller scale experiments with IDS}

This section considers a series of smaller scale experiments (10 arms, 1000 time periods) drawn from the paper introducing the IDS algorithm, \citep{russo2014learning}. A major consideration in running these experiments is that the CPU time required to execute IDS, the closest competitor, based on the current suggested implementation is orders of magnitudes greater than that of the index schemes or Thompson Sampling. The main bottleneck is that IDS uses numerical integration, requiring the calculation of a CDF over, at least, hundreds of iterations. By contrast, the version of OGI with $K=1$ uses 10 iterations of the Newton-Raphson method. 

\paragraph{Gaussian}We replicate the Gaussian experiments from \cite{russo2014learning}. In the first experiment (Table~\ref{table:gaussian_experiment1}), the arms generate Gaussian rewards  $X_{i,t} \sim \mathcal{N}(\theta_i, 1)$ where each $\theta_i$ is independently drawn from a standard Gaussian distribution. {\color{blue}That is, the prior $q$ on each arm's reward is a stand Gaussian prior.} We simulate 1,000 independent trials with 10 arms and 1,000 time periods. The implementation of OGI in this experiment uses $K = 1$. It is difficult to compute exact Gittins indices in this setting, but a classical approximation for Gaussian bandits does exist; see \cite{powell2012optimal}, Chapter 6.1.3. We term the use of that approximation `OGI($\infty$) Approx'.  In addition to regret, we  show the average CPU time taken, in seconds, to execute each trial.
%We also evaluate a policy (labeled `OGI Approx' in the table) that computes a particular closed-form approximation to the Gittins Index given in Chapter 6.1.3 of Powell and Ryzhov \cite{powell2012optimal}. 

\begin{table}[h!]
	\centering
	\begin{tabular}{cccccc} \toprule
		\textbf{Algorithm}  & \textbf{OGI(1)} & \textbf{OGI($\infty$) Approx.} & \textbf{IDS} & \textbf{TS} & \textbf{Bayes UCB}\\ \midrule
		Mean   & 49.19 & 47.64  &  55.83 & 67.40 & 60.30  \\ 
		Standard error  & 1.61 & 1.6 & 2.08 & 1.5 & 1.43 \\ 
		25\%  & 17.49 & 16.88  & 18.61 & 37.46 & 31.41 \\
		50\%   & 41.72 & 40.99 & 40.79 & 63.06 & 57.71 \\ 
		75\%  & 73.24 & 72.26 & 78.76 & 94.52 & 86.40 \\ 
		CPU time (s) & 0.02 & 0.01 & 11.18 & 0.01 & 0.02 \\
		\bottomrule
	\end{tabular}
	\caption[Table caption text]{Gaussian experiment. OGI(1) denotes OGI with $K =1$, while OGI Approx. uses the approximation to the Gaussian Gittins Index from \cite{powell2012optimal}.}
	\label{table:gaussian_experiment1}
\end{table}

The key feature of the results here is that OGI offers an approximately 10\% improvement in regret over its nearest competitor IDS, and larger improvements (20 and 40 \% respectively) over Bayes UCB and Thompson Sampling. The best performing policy is OGI with the specialized Gaussian approximation since it gives a closer approximation to the Gittins Index. At the same time, OGI is essentially as fast as Thompson sampling, and three orders of magnitude faster than its nearest competitor (in terms of regret). 


\paragraph{Bernoulli}
We next replicate the Beta-Bernoulli experiments from \cite{russo2014learning}.
In this experiment regret is simulated over 1,000 periods, with 10 arms each having a uniformly distributed Bernoulli parameter. We simulate 1,000 independent trials and Table~\ref{table:bernoulli_experiment1} summarizes the results.

\begin{table}[h!]
	\centering
	\begin{tabular}{ccccccc} \toprule
		\textbf{Algorithm} & \textbf{OGI(1)} & \textbf{OGI(3)} &  \textbf{OGI($\infty$)} & \textbf{IDS} & \textbf{TS} & \textbf{Bayes UCB}  \\ \midrule
		Mean &  18.12 & 18.00 & 17.52 & 19.03 & 27.39 & 22.71 \\ 
		Standard error & 0.65 & 0.64 &  0.68 & 0.67 & 0.57 & 0.56 \\ 
		25\% & 6.26 & 5.60 & 4.45 & 5.85 & 14.62 & 10.09 \\
		50\% & 15.08 & 14.84 &12.06 & 14.06 & 23.53 & 18.52 \\
		75\% & 27.63 & 27.74 & 24.93 & 26.48 & 36.11 & 30.58 \\
		CPU time (s) & 0.19 & 0.89 & (?) hours & 8.11 & 0.01 & 0.05  \\ \bottomrule
	\end{tabular}
	\caption[Table caption text]{Bernoulli experiment. OGI($K$) denotes the OGI algorithm with a $K$ step approximation and tuning parameter $\alpha = 100$. OGI($\infty$) is the algorithm that uses Gittins Indices.}
	\label{table:bernoulli_experiment1}
\end{table}
Each version of OGI outperforms other algorithms and the one that uses exact Gittins Indices shows the lowest mean regret.
Perhaps, unsurprisingly, when OGI  looks ahead 3 steps (or when the lookahead is not limited), it performs better than with a single step. It is however apparent that in each of these cases the improvement over simply setting $K=1$ is marginal. Indeed, looking ahead 1 step is a reasonably close approximation to the Gittins Index in the Bernoulli problem. In the Appendix, we report the approximation error in approximating the Gittins index for various choice of $K$. When using an optimistic 1 step approximation, the error is around 15\% and if $K$ is increased to 3, the error drops to around 4\% (see Tables~\ref{table:ogi_table_for_gamma_9} and \ref{table:ogi_table_for_gamma_95} in the Appendix). 

As an aside, we note that the regret we computed for the IDS algorithm is slightly different from that reported by \cite{russo2014learning}. Specifically, we obtain slightly lower regret for IDS than they report in the Gaussian experiments, and slightly higher values for the Beta-Bernoulli case; we include a link to the code we used to implement the algorithms\footnote{\url{https://github.com/gutin/FastGittins}} as a reference.  

\subsection{Large scale experiment} \label{exp:ts_sampling_experiment}
This experiment replicates a large scale synthetic experiment in \cite{chapelle2011empirical}.
The key feature here is that we simulate a  longer horizon of $T = 10^6$ and include a large number of arms, particularly we let $A = 100$. This is an order of magnitude greater than in the majority of synthetic bandit experiments we are aware of.
Our goal is to see how the algorithms scale both computationally and in terms of performance.
Such a setup is practically relevant because in applications such as e-commerce or online advertising, the problems of interest are typically modeled with many arms relative to the horizon, where each arm could represent a product or ad.

Because all the methods we test in our numerical experiments are regret optimal, any relative difference in regret must shrink after a sufficiently large number of time periods. The length of time for this `burn in' period intuitively depends on the number of arms in the problem.
In particular, we can think of the horizon as giving us a rough budget on the number of trials per arm via the ratio $T/A$.
The idea is that with more trials per arm we should expect a smaller relative difference between the algorithms {\color{blue}(and indeed the theoretical guarantees for the algorithms require this to happen)}. 
We will see that even when the ratio $T/A$ and $A$ itself are large, there is a substantial difference between OGI and the competing benchmarks in both a relative and absolute sense.

As this experiment requires an order of magnitude more iterations than the earlier ones, we are only able to simulate the fastest algorithms, which are OGI with $K=1$, Thompson Sampling and Bayes UCB. 
It was not possible to include IDS because its performance is hindered by the fact that each arm pull decision requires time that is quadratic in the number of arms to compute. 
Again, this is a Bernoulli experiment where arm means are independently sampled from a uniform prior and each algorithm assumes this same prior over the unknown mean rewards from the arms.
We show the algorithms' regret averaged over 5,000 trials in Figure~\ref{fig:chapelle_and_li} and Table~\ref{table:additional_cli_table}.
\begin{figure}[h!]
	\centering
	%% Creator: Matplotlib, PGF backend
%%
%% To include the figure in your LaTeX document, write
%%   \input{<filename>.pgf}
%%
%% Make sure the required packages are loaded in your preamble
%%   \usepackage{pgf}
%%
%% Figures using additional raster images can only be included by \input if
%% they are in the same directory as the main LaTeX file. For loading figures
%% from other directories you can use the `import` package
%%   \usepackage{import}
%% and then include the figures with
%%   \import{<path to file>}{<filename>.pgf}
%%
%% Matplotlib used the following preamble
%%   \usepackage[utf8x]{inputenc}
%%   \usepackage[T1]{fontenc}
%%
\begingroup%
\makeatletter%
\begin{pgfpicture}%
\pgfpathrectangle{\pgfpointorigin}{\pgfqpoint{5.082555in}{3.141192in}}%
\pgfusepath{use as bounding box, clip}%
\begin{pgfscope}%
\pgfsetbuttcap%
\pgfsetmiterjoin%
\definecolor{currentfill}{rgb}{1.000000,1.000000,1.000000}%
\pgfsetfillcolor{currentfill}%
\pgfsetlinewidth{0.000000pt}%
\definecolor{currentstroke}{rgb}{1.000000,1.000000,1.000000}%
\pgfsetstrokecolor{currentstroke}%
\pgfsetdash{}{0pt}%
\pgfpathmoveto{\pgfqpoint{0.000000in}{0.000000in}}%
\pgfpathlineto{\pgfqpoint{5.082555in}{0.000000in}}%
\pgfpathlineto{\pgfqpoint{5.082555in}{3.141192in}}%
\pgfpathlineto{\pgfqpoint{0.000000in}{3.141192in}}%
\pgfpathclose%
\pgfusepath{fill}%
\end{pgfscope}%
\begin{pgfscope}%
\pgfsetbuttcap%
\pgfsetmiterjoin%
\definecolor{currentfill}{rgb}{1.000000,1.000000,1.000000}%
\pgfsetfillcolor{currentfill}%
\pgfsetlinewidth{0.000000pt}%
\definecolor{currentstroke}{rgb}{0.000000,0.000000,0.000000}%
\pgfsetstrokecolor{currentstroke}%
\pgfsetstrokeopacity{0.000000}%
\pgfsetdash{}{0pt}%
\pgfpathmoveto{\pgfqpoint{0.635319in}{0.392649in}}%
\pgfpathlineto{\pgfqpoint{4.574300in}{0.392649in}}%
\pgfpathlineto{\pgfqpoint{4.574300in}{2.764249in}}%
\pgfpathlineto{\pgfqpoint{0.635319in}{2.764249in}}%
\pgfpathclose%
\pgfusepath{fill}%
\end{pgfscope}%
\begin{pgfscope}%
\pgfsetbuttcap%
\pgfsetroundjoin%
\definecolor{currentfill}{rgb}{0.000000,0.000000,0.000000}%
\pgfsetfillcolor{currentfill}%
\pgfsetlinewidth{0.803000pt}%
\definecolor{currentstroke}{rgb}{0.000000,0.000000,0.000000}%
\pgfsetstrokecolor{currentstroke}%
\pgfsetdash{}{0pt}%
\pgfsys@defobject{currentmarker}{\pgfqpoint{0.000000in}{-0.048611in}}{\pgfqpoint{0.000000in}{0.000000in}}{%
\pgfpathmoveto{\pgfqpoint{0.000000in}{0.000000in}}%
\pgfpathlineto{\pgfqpoint{0.000000in}{-0.048611in}}%
\pgfusepath{stroke,fill}%
}%
\begin{pgfscope}%
\pgfsys@transformshift{0.814364in}{0.392649in}%
\pgfsys@useobject{currentmarker}{}%
\end{pgfscope}%
\end{pgfscope}%
\begin{pgfscope}%
\pgftext[x=0.814364in,y=0.295427in,,top]{\rmfamily\fontsize{8.000000}{9.600000}\selectfont \(\displaystyle 10^{4}\)}%
\end{pgfscope}%
\begin{pgfscope}%
\pgfsetbuttcap%
\pgfsetroundjoin%
\definecolor{currentfill}{rgb}{0.000000,0.000000,0.000000}%
\pgfsetfillcolor{currentfill}%
\pgfsetlinewidth{0.803000pt}%
\definecolor{currentstroke}{rgb}{0.000000,0.000000,0.000000}%
\pgfsetstrokecolor{currentstroke}%
\pgfsetdash{}{0pt}%
\pgfsys@defobject{currentmarker}{\pgfqpoint{0.000000in}{-0.048611in}}{\pgfqpoint{0.000000in}{0.000000in}}{%
\pgfpathmoveto{\pgfqpoint{0.000000in}{0.000000in}}%
\pgfpathlineto{\pgfqpoint{0.000000in}{-0.048611in}}%
\pgfusepath{stroke,fill}%
}%
\begin{pgfscope}%
\pgfsys@transformshift{2.608725in}{0.392649in}%
\pgfsys@useobject{currentmarker}{}%
\end{pgfscope}%
\end{pgfscope}%
\begin{pgfscope}%
\pgftext[x=2.608725in,y=0.295427in,,top]{\rmfamily\fontsize{8.000000}{9.600000}\selectfont \(\displaystyle 10^{5}\)}%
\end{pgfscope}%
\begin{pgfscope}%
\pgfsetbuttcap%
\pgfsetroundjoin%
\definecolor{currentfill}{rgb}{0.000000,0.000000,0.000000}%
\pgfsetfillcolor{currentfill}%
\pgfsetlinewidth{0.803000pt}%
\definecolor{currentstroke}{rgb}{0.000000,0.000000,0.000000}%
\pgfsetstrokecolor{currentstroke}%
\pgfsetdash{}{0pt}%
\pgfsys@defobject{currentmarker}{\pgfqpoint{0.000000in}{-0.048611in}}{\pgfqpoint{0.000000in}{0.000000in}}{%
\pgfpathmoveto{\pgfqpoint{0.000000in}{0.000000in}}%
\pgfpathlineto{\pgfqpoint{0.000000in}{-0.048611in}}%
\pgfusepath{stroke,fill}%
}%
\begin{pgfscope}%
\pgfsys@transformshift{4.403087in}{0.392649in}%
\pgfsys@useobject{currentmarker}{}%
\end{pgfscope}%
\end{pgfscope}%
\begin{pgfscope}%
\pgftext[x=4.403087in,y=0.295427in,,top]{\rmfamily\fontsize{8.000000}{9.600000}\selectfont \(\displaystyle 10^{6}\)}%
\end{pgfscope}%
\begin{pgfscope}%
\pgfsetbuttcap%
\pgfsetroundjoin%
\definecolor{currentfill}{rgb}{0.000000,0.000000,0.000000}%
\pgfsetfillcolor{currentfill}%
\pgfsetlinewidth{0.602250pt}%
\definecolor{currentstroke}{rgb}{0.000000,0.000000,0.000000}%
\pgfsetstrokecolor{currentstroke}%
\pgfsetdash{}{0pt}%
\pgfsys@defobject{currentmarker}{\pgfqpoint{0.000000in}{-0.027778in}}{\pgfqpoint{0.000000in}{0.000000in}}{%
\pgfpathmoveto{\pgfqpoint{0.000000in}{0.000000in}}%
\pgfpathlineto{\pgfqpoint{0.000000in}{-0.027778in}}%
\pgfusepath{stroke,fill}%
}%
\begin{pgfscope}%
\pgfsys@transformshift{0.640472in}{0.392649in}%
\pgfsys@useobject{currentmarker}{}%
\end{pgfscope}%
\end{pgfscope}%
\begin{pgfscope}%
\pgfsetbuttcap%
\pgfsetroundjoin%
\definecolor{currentfill}{rgb}{0.000000,0.000000,0.000000}%
\pgfsetfillcolor{currentfill}%
\pgfsetlinewidth{0.602250pt}%
\definecolor{currentstroke}{rgb}{0.000000,0.000000,0.000000}%
\pgfsetstrokecolor{currentstroke}%
\pgfsetdash{}{0pt}%
\pgfsys@defobject{currentmarker}{\pgfqpoint{0.000000in}{-0.027778in}}{\pgfqpoint{0.000000in}{0.000000in}}{%
\pgfpathmoveto{\pgfqpoint{0.000000in}{0.000000in}}%
\pgfpathlineto{\pgfqpoint{0.000000in}{-0.027778in}}%
\pgfusepath{stroke,fill}%
}%
\begin{pgfscope}%
\pgfsys@transformshift{0.732258in}{0.392649in}%
\pgfsys@useobject{currentmarker}{}%
\end{pgfscope}%
\end{pgfscope}%
\begin{pgfscope}%
\pgfsetbuttcap%
\pgfsetroundjoin%
\definecolor{currentfill}{rgb}{0.000000,0.000000,0.000000}%
\pgfsetfillcolor{currentfill}%
\pgfsetlinewidth{0.602250pt}%
\definecolor{currentstroke}{rgb}{0.000000,0.000000,0.000000}%
\pgfsetstrokecolor{currentstroke}%
\pgfsetdash{}{0pt}%
\pgfsys@defobject{currentmarker}{\pgfqpoint{0.000000in}{-0.027778in}}{\pgfqpoint{0.000000in}{0.000000in}}{%
\pgfpathmoveto{\pgfqpoint{0.000000in}{0.000000in}}%
\pgfpathlineto{\pgfqpoint{0.000000in}{-0.027778in}}%
\pgfusepath{stroke,fill}%
}%
\begin{pgfscope}%
\pgfsys@transformshift{1.354521in}{0.392649in}%
\pgfsys@useobject{currentmarker}{}%
\end{pgfscope}%
\end{pgfscope}%
\begin{pgfscope}%
\pgfsetbuttcap%
\pgfsetroundjoin%
\definecolor{currentfill}{rgb}{0.000000,0.000000,0.000000}%
\pgfsetfillcolor{currentfill}%
\pgfsetlinewidth{0.602250pt}%
\definecolor{currentstroke}{rgb}{0.000000,0.000000,0.000000}%
\pgfsetstrokecolor{currentstroke}%
\pgfsetdash{}{0pt}%
\pgfsys@defobject{currentmarker}{\pgfqpoint{0.000000in}{-0.027778in}}{\pgfqpoint{0.000000in}{0.000000in}}{%
\pgfpathmoveto{\pgfqpoint{0.000000in}{0.000000in}}%
\pgfpathlineto{\pgfqpoint{0.000000in}{-0.027778in}}%
\pgfusepath{stroke,fill}%
}%
\begin{pgfscope}%
\pgfsys@transformshift{1.670492in}{0.392649in}%
\pgfsys@useobject{currentmarker}{}%
\end{pgfscope}%
\end{pgfscope}%
\begin{pgfscope}%
\pgfsetbuttcap%
\pgfsetroundjoin%
\definecolor{currentfill}{rgb}{0.000000,0.000000,0.000000}%
\pgfsetfillcolor{currentfill}%
\pgfsetlinewidth{0.602250pt}%
\definecolor{currentstroke}{rgb}{0.000000,0.000000,0.000000}%
\pgfsetstrokecolor{currentstroke}%
\pgfsetdash{}{0pt}%
\pgfsys@defobject{currentmarker}{\pgfqpoint{0.000000in}{-0.027778in}}{\pgfqpoint{0.000000in}{0.000000in}}{%
\pgfpathmoveto{\pgfqpoint{0.000000in}{0.000000in}}%
\pgfpathlineto{\pgfqpoint{0.000000in}{-0.027778in}}%
\pgfusepath{stroke,fill}%
}%
\begin{pgfscope}%
\pgfsys@transformshift{1.894677in}{0.392649in}%
\pgfsys@useobject{currentmarker}{}%
\end{pgfscope}%
\end{pgfscope}%
\begin{pgfscope}%
\pgfsetbuttcap%
\pgfsetroundjoin%
\definecolor{currentfill}{rgb}{0.000000,0.000000,0.000000}%
\pgfsetfillcolor{currentfill}%
\pgfsetlinewidth{0.602250pt}%
\definecolor{currentstroke}{rgb}{0.000000,0.000000,0.000000}%
\pgfsetstrokecolor{currentstroke}%
\pgfsetdash{}{0pt}%
\pgfsys@defobject{currentmarker}{\pgfqpoint{0.000000in}{-0.027778in}}{\pgfqpoint{0.000000in}{0.000000in}}{%
\pgfpathmoveto{\pgfqpoint{0.000000in}{0.000000in}}%
\pgfpathlineto{\pgfqpoint{0.000000in}{-0.027778in}}%
\pgfusepath{stroke,fill}%
}%
\begin{pgfscope}%
\pgfsys@transformshift{2.068569in}{0.392649in}%
\pgfsys@useobject{currentmarker}{}%
\end{pgfscope}%
\end{pgfscope}%
\begin{pgfscope}%
\pgfsetbuttcap%
\pgfsetroundjoin%
\definecolor{currentfill}{rgb}{0.000000,0.000000,0.000000}%
\pgfsetfillcolor{currentfill}%
\pgfsetlinewidth{0.602250pt}%
\definecolor{currentstroke}{rgb}{0.000000,0.000000,0.000000}%
\pgfsetstrokecolor{currentstroke}%
\pgfsetdash{}{0pt}%
\pgfsys@defobject{currentmarker}{\pgfqpoint{0.000000in}{-0.027778in}}{\pgfqpoint{0.000000in}{0.000000in}}{%
\pgfpathmoveto{\pgfqpoint{0.000000in}{0.000000in}}%
\pgfpathlineto{\pgfqpoint{0.000000in}{-0.027778in}}%
\pgfusepath{stroke,fill}%
}%
\begin{pgfscope}%
\pgfsys@transformshift{2.210649in}{0.392649in}%
\pgfsys@useobject{currentmarker}{}%
\end{pgfscope}%
\end{pgfscope}%
\begin{pgfscope}%
\pgfsetbuttcap%
\pgfsetroundjoin%
\definecolor{currentfill}{rgb}{0.000000,0.000000,0.000000}%
\pgfsetfillcolor{currentfill}%
\pgfsetlinewidth{0.602250pt}%
\definecolor{currentstroke}{rgb}{0.000000,0.000000,0.000000}%
\pgfsetstrokecolor{currentstroke}%
\pgfsetdash{}{0pt}%
\pgfsys@defobject{currentmarker}{\pgfqpoint{0.000000in}{-0.027778in}}{\pgfqpoint{0.000000in}{0.000000in}}{%
\pgfpathmoveto{\pgfqpoint{0.000000in}{0.000000in}}%
\pgfpathlineto{\pgfqpoint{0.000000in}{-0.027778in}}%
\pgfusepath{stroke,fill}%
}%
\begin{pgfscope}%
\pgfsys@transformshift{2.330775in}{0.392649in}%
\pgfsys@useobject{currentmarker}{}%
\end{pgfscope}%
\end{pgfscope}%
\begin{pgfscope}%
\pgfsetbuttcap%
\pgfsetroundjoin%
\definecolor{currentfill}{rgb}{0.000000,0.000000,0.000000}%
\pgfsetfillcolor{currentfill}%
\pgfsetlinewidth{0.602250pt}%
\definecolor{currentstroke}{rgb}{0.000000,0.000000,0.000000}%
\pgfsetstrokecolor{currentstroke}%
\pgfsetdash{}{0pt}%
\pgfsys@defobject{currentmarker}{\pgfqpoint{0.000000in}{-0.027778in}}{\pgfqpoint{0.000000in}{0.000000in}}{%
\pgfpathmoveto{\pgfqpoint{0.000000in}{0.000000in}}%
\pgfpathlineto{\pgfqpoint{0.000000in}{-0.027778in}}%
\pgfusepath{stroke,fill}%
}%
\begin{pgfscope}%
\pgfsys@transformshift{2.434834in}{0.392649in}%
\pgfsys@useobject{currentmarker}{}%
\end{pgfscope}%
\end{pgfscope}%
\begin{pgfscope}%
\pgfsetbuttcap%
\pgfsetroundjoin%
\definecolor{currentfill}{rgb}{0.000000,0.000000,0.000000}%
\pgfsetfillcolor{currentfill}%
\pgfsetlinewidth{0.602250pt}%
\definecolor{currentstroke}{rgb}{0.000000,0.000000,0.000000}%
\pgfsetstrokecolor{currentstroke}%
\pgfsetdash{}{0pt}%
\pgfsys@defobject{currentmarker}{\pgfqpoint{0.000000in}{-0.027778in}}{\pgfqpoint{0.000000in}{0.000000in}}{%
\pgfpathmoveto{\pgfqpoint{0.000000in}{0.000000in}}%
\pgfpathlineto{\pgfqpoint{0.000000in}{-0.027778in}}%
\pgfusepath{stroke,fill}%
}%
\begin{pgfscope}%
\pgfsys@transformshift{2.526620in}{0.392649in}%
\pgfsys@useobject{currentmarker}{}%
\end{pgfscope}%
\end{pgfscope}%
\begin{pgfscope}%
\pgfsetbuttcap%
\pgfsetroundjoin%
\definecolor{currentfill}{rgb}{0.000000,0.000000,0.000000}%
\pgfsetfillcolor{currentfill}%
\pgfsetlinewidth{0.602250pt}%
\definecolor{currentstroke}{rgb}{0.000000,0.000000,0.000000}%
\pgfsetstrokecolor{currentstroke}%
\pgfsetdash{}{0pt}%
\pgfsys@defobject{currentmarker}{\pgfqpoint{0.000000in}{-0.027778in}}{\pgfqpoint{0.000000in}{0.000000in}}{%
\pgfpathmoveto{\pgfqpoint{0.000000in}{0.000000in}}%
\pgfpathlineto{\pgfqpoint{0.000000in}{-0.027778in}}%
\pgfusepath{stroke,fill}%
}%
\begin{pgfscope}%
\pgfsys@transformshift{3.148882in}{0.392649in}%
\pgfsys@useobject{currentmarker}{}%
\end{pgfscope}%
\end{pgfscope}%
\begin{pgfscope}%
\pgfsetbuttcap%
\pgfsetroundjoin%
\definecolor{currentfill}{rgb}{0.000000,0.000000,0.000000}%
\pgfsetfillcolor{currentfill}%
\pgfsetlinewidth{0.602250pt}%
\definecolor{currentstroke}{rgb}{0.000000,0.000000,0.000000}%
\pgfsetstrokecolor{currentstroke}%
\pgfsetdash{}{0pt}%
\pgfsys@defobject{currentmarker}{\pgfqpoint{0.000000in}{-0.027778in}}{\pgfqpoint{0.000000in}{0.000000in}}{%
\pgfpathmoveto{\pgfqpoint{0.000000in}{0.000000in}}%
\pgfpathlineto{\pgfqpoint{0.000000in}{-0.027778in}}%
\pgfusepath{stroke,fill}%
}%
\begin{pgfscope}%
\pgfsys@transformshift{3.464853in}{0.392649in}%
\pgfsys@useobject{currentmarker}{}%
\end{pgfscope}%
\end{pgfscope}%
\begin{pgfscope}%
\pgfsetbuttcap%
\pgfsetroundjoin%
\definecolor{currentfill}{rgb}{0.000000,0.000000,0.000000}%
\pgfsetfillcolor{currentfill}%
\pgfsetlinewidth{0.602250pt}%
\definecolor{currentstroke}{rgb}{0.000000,0.000000,0.000000}%
\pgfsetstrokecolor{currentstroke}%
\pgfsetdash{}{0pt}%
\pgfsys@defobject{currentmarker}{\pgfqpoint{0.000000in}{-0.027778in}}{\pgfqpoint{0.000000in}{0.000000in}}{%
\pgfpathmoveto{\pgfqpoint{0.000000in}{0.000000in}}%
\pgfpathlineto{\pgfqpoint{0.000000in}{-0.027778in}}%
\pgfusepath{stroke,fill}%
}%
\begin{pgfscope}%
\pgfsys@transformshift{3.689039in}{0.392649in}%
\pgfsys@useobject{currentmarker}{}%
\end{pgfscope}%
\end{pgfscope}%
\begin{pgfscope}%
\pgfsetbuttcap%
\pgfsetroundjoin%
\definecolor{currentfill}{rgb}{0.000000,0.000000,0.000000}%
\pgfsetfillcolor{currentfill}%
\pgfsetlinewidth{0.602250pt}%
\definecolor{currentstroke}{rgb}{0.000000,0.000000,0.000000}%
\pgfsetstrokecolor{currentstroke}%
\pgfsetdash{}{0pt}%
\pgfsys@defobject{currentmarker}{\pgfqpoint{0.000000in}{-0.027778in}}{\pgfqpoint{0.000000in}{0.000000in}}{%
\pgfpathmoveto{\pgfqpoint{0.000000in}{0.000000in}}%
\pgfpathlineto{\pgfqpoint{0.000000in}{-0.027778in}}%
\pgfusepath{stroke,fill}%
}%
\begin{pgfscope}%
\pgfsys@transformshift{3.862930in}{0.392649in}%
\pgfsys@useobject{currentmarker}{}%
\end{pgfscope}%
\end{pgfscope}%
\begin{pgfscope}%
\pgfsetbuttcap%
\pgfsetroundjoin%
\definecolor{currentfill}{rgb}{0.000000,0.000000,0.000000}%
\pgfsetfillcolor{currentfill}%
\pgfsetlinewidth{0.602250pt}%
\definecolor{currentstroke}{rgb}{0.000000,0.000000,0.000000}%
\pgfsetstrokecolor{currentstroke}%
\pgfsetdash{}{0pt}%
\pgfsys@defobject{currentmarker}{\pgfqpoint{0.000000in}{-0.027778in}}{\pgfqpoint{0.000000in}{0.000000in}}{%
\pgfpathmoveto{\pgfqpoint{0.000000in}{0.000000in}}%
\pgfpathlineto{\pgfqpoint{0.000000in}{-0.027778in}}%
\pgfusepath{stroke,fill}%
}%
\begin{pgfscope}%
\pgfsys@transformshift{4.005010in}{0.392649in}%
\pgfsys@useobject{currentmarker}{}%
\end{pgfscope}%
\end{pgfscope}%
\begin{pgfscope}%
\pgfsetbuttcap%
\pgfsetroundjoin%
\definecolor{currentfill}{rgb}{0.000000,0.000000,0.000000}%
\pgfsetfillcolor{currentfill}%
\pgfsetlinewidth{0.602250pt}%
\definecolor{currentstroke}{rgb}{0.000000,0.000000,0.000000}%
\pgfsetstrokecolor{currentstroke}%
\pgfsetdash{}{0pt}%
\pgfsys@defobject{currentmarker}{\pgfqpoint{0.000000in}{-0.027778in}}{\pgfqpoint{0.000000in}{0.000000in}}{%
\pgfpathmoveto{\pgfqpoint{0.000000in}{0.000000in}}%
\pgfpathlineto{\pgfqpoint{0.000000in}{-0.027778in}}%
\pgfusepath{stroke,fill}%
}%
\begin{pgfscope}%
\pgfsys@transformshift{4.125137in}{0.392649in}%
\pgfsys@useobject{currentmarker}{}%
\end{pgfscope}%
\end{pgfscope}%
\begin{pgfscope}%
\pgfsetbuttcap%
\pgfsetroundjoin%
\definecolor{currentfill}{rgb}{0.000000,0.000000,0.000000}%
\pgfsetfillcolor{currentfill}%
\pgfsetlinewidth{0.602250pt}%
\definecolor{currentstroke}{rgb}{0.000000,0.000000,0.000000}%
\pgfsetstrokecolor{currentstroke}%
\pgfsetdash{}{0pt}%
\pgfsys@defobject{currentmarker}{\pgfqpoint{0.000000in}{-0.027778in}}{\pgfqpoint{0.000000in}{0.000000in}}{%
\pgfpathmoveto{\pgfqpoint{0.000000in}{0.000000in}}%
\pgfpathlineto{\pgfqpoint{0.000000in}{-0.027778in}}%
\pgfusepath{stroke,fill}%
}%
\begin{pgfscope}%
\pgfsys@transformshift{4.229195in}{0.392649in}%
\pgfsys@useobject{currentmarker}{}%
\end{pgfscope}%
\end{pgfscope}%
\begin{pgfscope}%
\pgfsetbuttcap%
\pgfsetroundjoin%
\definecolor{currentfill}{rgb}{0.000000,0.000000,0.000000}%
\pgfsetfillcolor{currentfill}%
\pgfsetlinewidth{0.602250pt}%
\definecolor{currentstroke}{rgb}{0.000000,0.000000,0.000000}%
\pgfsetstrokecolor{currentstroke}%
\pgfsetdash{}{0pt}%
\pgfsys@defobject{currentmarker}{\pgfqpoint{0.000000in}{-0.027778in}}{\pgfqpoint{0.000000in}{0.000000in}}{%
\pgfpathmoveto{\pgfqpoint{0.000000in}{0.000000in}}%
\pgfpathlineto{\pgfqpoint{0.000000in}{-0.027778in}}%
\pgfusepath{stroke,fill}%
}%
\begin{pgfscope}%
\pgfsys@transformshift{4.320982in}{0.392649in}%
\pgfsys@useobject{currentmarker}{}%
\end{pgfscope}%
\end{pgfscope}%
\begin{pgfscope}%
\pgftext[x=2.604809in,y=0.140010in,,top]{\rmfamily\fontsize{10.000000}{12.000000}\selectfont Period}%
\end{pgfscope}%
\begin{pgfscope}%
\pgfsetbuttcap%
\pgfsetroundjoin%
\definecolor{currentfill}{rgb}{0.000000,0.000000,0.000000}%
\pgfsetfillcolor{currentfill}%
\pgfsetlinewidth{0.803000pt}%
\definecolor{currentstroke}{rgb}{0.000000,0.000000,0.000000}%
\pgfsetstrokecolor{currentstroke}%
\pgfsetdash{}{0pt}%
\pgfsys@defobject{currentmarker}{\pgfqpoint{-0.048611in}{0.000000in}}{\pgfqpoint{0.000000in}{0.000000in}}{%
\pgfpathmoveto{\pgfqpoint{0.000000in}{0.000000in}}%
\pgfpathlineto{\pgfqpoint{-0.048611in}{0.000000in}}%
\pgfusepath{stroke,fill}%
}%
\begin{pgfscope}%
\pgfsys@transformshift{0.635319in}{0.747708in}%
\pgfsys@useobject{currentmarker}{}%
\end{pgfscope}%
\end{pgfscope}%
\begin{pgfscope}%
\pgftext[x=0.361011in,y=0.709446in,left,base]{\rmfamily\fontsize{8.000000}{9.600000}\selectfont \(\displaystyle 150\)}%
\end{pgfscope}%
\begin{pgfscope}%
\pgfsetbuttcap%
\pgfsetroundjoin%
\definecolor{currentfill}{rgb}{0.000000,0.000000,0.000000}%
\pgfsetfillcolor{currentfill}%
\pgfsetlinewidth{0.803000pt}%
\definecolor{currentstroke}{rgb}{0.000000,0.000000,0.000000}%
\pgfsetstrokecolor{currentstroke}%
\pgfsetdash{}{0pt}%
\pgfsys@defobject{currentmarker}{\pgfqpoint{-0.048611in}{0.000000in}}{\pgfqpoint{0.000000in}{0.000000in}}{%
\pgfpathmoveto{\pgfqpoint{0.000000in}{0.000000in}}%
\pgfpathlineto{\pgfqpoint{-0.048611in}{0.000000in}}%
\pgfusepath{stroke,fill}%
}%
\begin{pgfscope}%
\pgfsys@transformshift{0.635319in}{1.161341in}%
\pgfsys@useobject{currentmarker}{}%
\end{pgfscope}%
\end{pgfscope}%
\begin{pgfscope}%
\pgftext[x=0.361011in,y=1.123079in,left,base]{\rmfamily\fontsize{8.000000}{9.600000}\selectfont \(\displaystyle 200\)}%
\end{pgfscope}%
\begin{pgfscope}%
\pgfsetbuttcap%
\pgfsetroundjoin%
\definecolor{currentfill}{rgb}{0.000000,0.000000,0.000000}%
\pgfsetfillcolor{currentfill}%
\pgfsetlinewidth{0.803000pt}%
\definecolor{currentstroke}{rgb}{0.000000,0.000000,0.000000}%
\pgfsetstrokecolor{currentstroke}%
\pgfsetdash{}{0pt}%
\pgfsys@defobject{currentmarker}{\pgfqpoint{-0.048611in}{0.000000in}}{\pgfqpoint{0.000000in}{0.000000in}}{%
\pgfpathmoveto{\pgfqpoint{0.000000in}{0.000000in}}%
\pgfpathlineto{\pgfqpoint{-0.048611in}{0.000000in}}%
\pgfusepath{stroke,fill}%
}%
\begin{pgfscope}%
\pgfsys@transformshift{0.635319in}{1.574973in}%
\pgfsys@useobject{currentmarker}{}%
\end{pgfscope}%
\end{pgfscope}%
\begin{pgfscope}%
\pgftext[x=0.361011in,y=1.536711in,left,base]{\rmfamily\fontsize{8.000000}{9.600000}\selectfont \(\displaystyle 250\)}%
\end{pgfscope}%
\begin{pgfscope}%
\pgfsetbuttcap%
\pgfsetroundjoin%
\definecolor{currentfill}{rgb}{0.000000,0.000000,0.000000}%
\pgfsetfillcolor{currentfill}%
\pgfsetlinewidth{0.803000pt}%
\definecolor{currentstroke}{rgb}{0.000000,0.000000,0.000000}%
\pgfsetstrokecolor{currentstroke}%
\pgfsetdash{}{0pt}%
\pgfsys@defobject{currentmarker}{\pgfqpoint{-0.048611in}{0.000000in}}{\pgfqpoint{0.000000in}{0.000000in}}{%
\pgfpathmoveto{\pgfqpoint{0.000000in}{0.000000in}}%
\pgfpathlineto{\pgfqpoint{-0.048611in}{0.000000in}}%
\pgfusepath{stroke,fill}%
}%
\begin{pgfscope}%
\pgfsys@transformshift{0.635319in}{1.988606in}%
\pgfsys@useobject{currentmarker}{}%
\end{pgfscope}%
\end{pgfscope}%
\begin{pgfscope}%
\pgftext[x=0.361011in,y=1.950344in,left,base]{\rmfamily\fontsize{8.000000}{9.600000}\selectfont \(\displaystyle 300\)}%
\end{pgfscope}%
\begin{pgfscope}%
\pgfsetbuttcap%
\pgfsetroundjoin%
\definecolor{currentfill}{rgb}{0.000000,0.000000,0.000000}%
\pgfsetfillcolor{currentfill}%
\pgfsetlinewidth{0.803000pt}%
\definecolor{currentstroke}{rgb}{0.000000,0.000000,0.000000}%
\pgfsetstrokecolor{currentstroke}%
\pgfsetdash{}{0pt}%
\pgfsys@defobject{currentmarker}{\pgfqpoint{-0.048611in}{0.000000in}}{\pgfqpoint{0.000000in}{0.000000in}}{%
\pgfpathmoveto{\pgfqpoint{0.000000in}{0.000000in}}%
\pgfpathlineto{\pgfqpoint{-0.048611in}{0.000000in}}%
\pgfusepath{stroke,fill}%
}%
\begin{pgfscope}%
\pgfsys@transformshift{0.635319in}{2.402239in}%
\pgfsys@useobject{currentmarker}{}%
\end{pgfscope}%
\end{pgfscope}%
\begin{pgfscope}%
\pgftext[x=0.361011in,y=2.363976in,left,base]{\rmfamily\fontsize{8.000000}{9.600000}\selectfont \(\displaystyle 350\)}%
\end{pgfscope}%
\begin{pgfscope}%
\pgftext[x=0.305456in,y=1.578449in,,bottom,rotate=90.000000]{\rmfamily\fontsize{10.000000}{12.000000}\selectfont Avg. Bayesian Regret}%
\end{pgfscope}%
\begin{pgfscope}%
\pgfpathrectangle{\pgfqpoint{0.635319in}{0.392649in}}{\pgfqpoint{3.938980in}{2.371600in}} %
\pgfusepath{clip}%
\pgfsetrectcap%
\pgfsetroundjoin%
\pgfsetlinewidth{1.505625pt}%
\definecolor{currentstroke}{rgb}{0.121569,0.466667,0.705882}%
\pgfsetstrokecolor{currentstroke}%
\pgfsetdash{}{0pt}%
\pgfpathmoveto{\pgfqpoint{0.814364in}{0.500449in}}%
\pgfpathlineto{\pgfqpoint{1.354521in}{0.715479in}}%
\pgfpathlineto{\pgfqpoint{1.670492in}{0.841740in}}%
\pgfpathlineto{\pgfqpoint{1.894677in}{0.928736in}}%
\pgfpathlineto{\pgfqpoint{2.068569in}{1.000307in}}%
\pgfpathlineto{\pgfqpoint{2.210649in}{1.056527in}}%
\pgfpathlineto{\pgfqpoint{2.330775in}{1.106157in}}%
\pgfpathlineto{\pgfqpoint{2.434834in}{1.147815in}}%
\pgfpathlineto{\pgfqpoint{2.526620in}{1.187179in}}%
\pgfpathlineto{\pgfqpoint{2.608725in}{1.220844in}}%
\pgfpathlineto{\pgfqpoint{2.682999in}{1.248470in}}%
\pgfpathlineto{\pgfqpoint{2.750805in}{1.274705in}}%
\pgfpathlineto{\pgfqpoint{2.813181in}{1.300029in}}%
\pgfpathlineto{\pgfqpoint{2.870932in}{1.321061in}}%
\pgfpathlineto{\pgfqpoint{2.924697in}{1.343735in}}%
\pgfpathlineto{\pgfqpoint{2.974991in}{1.361653in}}%
\pgfpathlineto{\pgfqpoint{3.022234in}{1.379430in}}%
\pgfpathlineto{\pgfqpoint{3.066777in}{1.397384in}}%
\pgfpathlineto{\pgfqpoint{3.108910in}{1.414039in}}%
\pgfpathlineto{\pgfqpoint{3.148882in}{1.430125in}}%
\pgfpathlineto{\pgfqpoint{3.186903in}{1.445713in}}%
\pgfpathlineto{\pgfqpoint{3.223156in}{1.460101in}}%
\pgfpathlineto{\pgfqpoint{3.257796in}{1.473218in}}%
\pgfpathlineto{\pgfqpoint{3.290962in}{1.485836in}}%
\pgfpathlineto{\pgfqpoint{3.322774in}{1.497936in}}%
\pgfpathlineto{\pgfqpoint{3.353338in}{1.509726in}}%
\pgfpathlineto{\pgfqpoint{3.382748in}{1.519871in}}%
\pgfpathlineto{\pgfqpoint{3.411089in}{1.526866in}}%
\pgfpathlineto{\pgfqpoint{3.438435in}{1.536513in}}%
\pgfpathlineto{\pgfqpoint{3.464853in}{1.545701in}}%
\pgfpathlineto{\pgfqpoint{3.490406in}{1.553216in}}%
\pgfpathlineto{\pgfqpoint{3.515147in}{1.561802in}}%
\pgfpathlineto{\pgfqpoint{3.539127in}{1.569294in}}%
\pgfpathlineto{\pgfqpoint{3.562391in}{1.576416in}}%
\pgfpathlineto{\pgfqpoint{3.584980in}{1.583653in}}%
\pgfpathlineto{\pgfqpoint{3.606933in}{1.589801in}}%
\pgfpathlineto{\pgfqpoint{3.628285in}{1.597837in}}%
\pgfpathlineto{\pgfqpoint{3.649067in}{1.604878in}}%
\pgfpathlineto{\pgfqpoint{3.669309in}{1.613653in}}%
\pgfpathlineto{\pgfqpoint{3.689039in}{1.620770in}}%
\pgfpathlineto{\pgfqpoint{3.708281in}{1.628255in}}%
\pgfpathlineto{\pgfqpoint{3.727060in}{1.634785in}}%
\pgfpathlineto{\pgfqpoint{3.745397in}{1.641761in}}%
\pgfpathlineto{\pgfqpoint{3.763312in}{1.648257in}}%
\pgfpathlineto{\pgfqpoint{3.780825in}{1.654619in}}%
\pgfpathlineto{\pgfqpoint{3.797953in}{1.660629in}}%
\pgfpathlineto{\pgfqpoint{3.814712in}{1.666475in}}%
\pgfpathlineto{\pgfqpoint{3.831119in}{1.672411in}}%
\pgfpathlineto{\pgfqpoint{3.847187in}{1.678254in}}%
\pgfpathlineto{\pgfqpoint{3.862930in}{1.683031in}}%
\pgfpathlineto{\pgfqpoint{3.878362in}{1.688524in}}%
\pgfpathlineto{\pgfqpoint{3.893494in}{1.697076in}}%
\pgfpathlineto{\pgfqpoint{3.908338in}{1.702216in}}%
\pgfpathlineto{\pgfqpoint{3.922905in}{1.705424in}}%
\pgfpathlineto{\pgfqpoint{3.937204in}{1.709711in}}%
\pgfpathlineto{\pgfqpoint{3.951245in}{1.714631in}}%
\pgfpathlineto{\pgfqpoint{3.965038in}{1.718597in}}%
\pgfpathlineto{\pgfqpoint{3.978591in}{1.722965in}}%
\pgfpathlineto{\pgfqpoint{3.991913in}{1.728868in}}%
\pgfpathlineto{\pgfqpoint{4.005010in}{1.733140in}}%
\pgfpathlineto{\pgfqpoint{4.017891in}{1.739352in}}%
\pgfpathlineto{\pgfqpoint{4.030563in}{1.744911in}}%
\pgfpathlineto{\pgfqpoint{4.043031in}{1.749155in}}%
\pgfpathlineto{\pgfqpoint{4.055304in}{1.753718in}}%
\pgfpathlineto{\pgfqpoint{4.067386in}{1.758115in}}%
\pgfpathlineto{\pgfqpoint{4.079284in}{1.762915in}}%
\pgfpathlineto{\pgfqpoint{4.091002in}{1.767658in}}%
\pgfpathlineto{\pgfqpoint{4.102547in}{1.773223in}}%
\pgfpathlineto{\pgfqpoint{4.113924in}{1.778905in}}%
\pgfpathlineto{\pgfqpoint{4.125137in}{1.782900in}}%
\pgfpathlineto{\pgfqpoint{4.136191in}{1.787289in}}%
\pgfpathlineto{\pgfqpoint{4.147090in}{1.791509in}}%
\pgfpathlineto{\pgfqpoint{4.157839in}{1.795367in}}%
\pgfpathlineto{\pgfqpoint{4.168441in}{1.800169in}}%
\pgfpathlineto{\pgfqpoint{4.178902in}{1.803909in}}%
\pgfpathlineto{\pgfqpoint{4.189223in}{1.806668in}}%
\pgfpathlineto{\pgfqpoint{4.199410in}{1.808377in}}%
\pgfpathlineto{\pgfqpoint{4.209466in}{1.812116in}}%
\pgfpathlineto{\pgfqpoint{4.219393in}{1.815443in}}%
\pgfpathlineto{\pgfqpoint{4.229195in}{1.819638in}}%
\pgfpathlineto{\pgfqpoint{4.238876in}{1.824777in}}%
\pgfpathlineto{\pgfqpoint{4.248438in}{1.828979in}}%
\pgfpathlineto{\pgfqpoint{4.257884in}{1.833107in}}%
\pgfpathlineto{\pgfqpoint{4.267217in}{1.836095in}}%
\pgfpathlineto{\pgfqpoint{4.276439in}{1.839207in}}%
\pgfpathlineto{\pgfqpoint{4.285554in}{1.842095in}}%
\pgfpathlineto{\pgfqpoint{4.294563in}{1.845947in}}%
\pgfpathlineto{\pgfqpoint{4.303469in}{1.848644in}}%
\pgfpathlineto{\pgfqpoint{4.312274in}{1.854361in}}%
\pgfpathlineto{\pgfqpoint{4.320982in}{1.856366in}}%
\pgfpathlineto{\pgfqpoint{4.329592in}{1.860029in}}%
\pgfpathlineto{\pgfqpoint{4.338109in}{1.863009in}}%
\pgfpathlineto{\pgfqpoint{4.346534in}{1.864282in}}%
\pgfpathlineto{\pgfqpoint{4.354869in}{1.868059in}}%
\pgfpathlineto{\pgfqpoint{4.363115in}{1.870339in}}%
\pgfpathlineto{\pgfqpoint{4.371275in}{1.872054in}}%
\pgfpathlineto{\pgfqpoint{4.379351in}{1.875687in}}%
\pgfpathlineto{\pgfqpoint{4.387343in}{1.878561in}}%
\pgfpathlineto{\pgfqpoint{4.395255in}{1.881747in}}%
\pgfusepath{stroke}%
\end{pgfscope}%
\begin{pgfscope}%
\pgfpathrectangle{\pgfqpoint{0.635319in}{0.392649in}}{\pgfqpoint{3.938980in}{2.371600in}} %
\pgfusepath{clip}%
\pgfsetbuttcap%
\pgfsetroundjoin%
\pgfsetlinewidth{1.505625pt}%
\definecolor{currentstroke}{rgb}{1.000000,0.498039,0.054902}%
\pgfsetstrokecolor{currentstroke}%
\pgfsetdash{{1.500000pt}{2.475000pt}}{0.000000pt}%
\pgfpathmoveto{\pgfqpoint{0.814364in}{1.003474in}}%
\pgfpathlineto{\pgfqpoint{1.354521in}{1.196221in}}%
\pgfpathlineto{\pgfqpoint{1.670492in}{1.309744in}}%
\pgfpathlineto{\pgfqpoint{1.894677in}{1.390068in}}%
\pgfpathlineto{\pgfqpoint{2.068569in}{1.455483in}}%
\pgfpathlineto{\pgfqpoint{2.210649in}{1.506574in}}%
\pgfpathlineto{\pgfqpoint{2.330775in}{1.552407in}}%
\pgfpathlineto{\pgfqpoint{2.434834in}{1.592716in}}%
\pgfpathlineto{\pgfqpoint{2.526620in}{1.629018in}}%
\pgfpathlineto{\pgfqpoint{2.608725in}{1.661311in}}%
\pgfpathlineto{\pgfqpoint{2.682999in}{1.689570in}}%
\pgfpathlineto{\pgfqpoint{2.750805in}{1.715643in}}%
\pgfpathlineto{\pgfqpoint{2.813181in}{1.741107in}}%
\pgfpathlineto{\pgfqpoint{2.870932in}{1.763406in}}%
\pgfpathlineto{\pgfqpoint{2.924697in}{1.785478in}}%
\pgfpathlineto{\pgfqpoint{2.974991in}{1.804414in}}%
\pgfpathlineto{\pgfqpoint{3.022234in}{1.823359in}}%
\pgfpathlineto{\pgfqpoint{3.066777in}{1.841400in}}%
\pgfpathlineto{\pgfqpoint{3.108910in}{1.859756in}}%
\pgfpathlineto{\pgfqpoint{3.148882in}{1.875784in}}%
\pgfpathlineto{\pgfqpoint{3.186903in}{1.891288in}}%
\pgfpathlineto{\pgfqpoint{3.223156in}{1.905248in}}%
\pgfpathlineto{\pgfqpoint{3.257796in}{1.920377in}}%
\pgfpathlineto{\pgfqpoint{3.290962in}{1.933953in}}%
\pgfpathlineto{\pgfqpoint{3.322774in}{1.947309in}}%
\pgfpathlineto{\pgfqpoint{3.353338in}{1.959146in}}%
\pgfpathlineto{\pgfqpoint{3.382748in}{1.969810in}}%
\pgfpathlineto{\pgfqpoint{3.411089in}{1.979606in}}%
\pgfpathlineto{\pgfqpoint{3.438435in}{1.991017in}}%
\pgfpathlineto{\pgfqpoint{3.464853in}{2.002531in}}%
\pgfpathlineto{\pgfqpoint{3.490406in}{2.012209in}}%
\pgfpathlineto{\pgfqpoint{3.515147in}{2.022064in}}%
\pgfpathlineto{\pgfqpoint{3.539127in}{2.031476in}}%
\pgfpathlineto{\pgfqpoint{3.562391in}{2.040342in}}%
\pgfpathlineto{\pgfqpoint{3.584980in}{2.049184in}}%
\pgfpathlineto{\pgfqpoint{3.606933in}{2.056970in}}%
\pgfpathlineto{\pgfqpoint{3.628285in}{2.065490in}}%
\pgfpathlineto{\pgfqpoint{3.649067in}{2.073511in}}%
\pgfpathlineto{\pgfqpoint{3.669309in}{2.084896in}}%
\pgfpathlineto{\pgfqpoint{3.689039in}{2.093434in}}%
\pgfpathlineto{\pgfqpoint{3.708281in}{2.101422in}}%
\pgfpathlineto{\pgfqpoint{3.727060in}{2.108578in}}%
\pgfpathlineto{\pgfqpoint{3.745397in}{2.115723in}}%
\pgfpathlineto{\pgfqpoint{3.763312in}{2.123233in}}%
\pgfpathlineto{\pgfqpoint{3.780825in}{2.130028in}}%
\pgfpathlineto{\pgfqpoint{3.797953in}{2.136992in}}%
\pgfpathlineto{\pgfqpoint{3.814712in}{2.143901in}}%
\pgfpathlineto{\pgfqpoint{3.831119in}{2.150234in}}%
\pgfpathlineto{\pgfqpoint{3.847187in}{2.157015in}}%
\pgfpathlineto{\pgfqpoint{3.862930in}{2.161761in}}%
\pgfpathlineto{\pgfqpoint{3.878362in}{2.168700in}}%
\pgfpathlineto{\pgfqpoint{3.893494in}{2.178080in}}%
\pgfpathlineto{\pgfqpoint{3.908338in}{2.184067in}}%
\pgfpathlineto{\pgfqpoint{3.922905in}{2.187702in}}%
\pgfpathlineto{\pgfqpoint{3.937204in}{2.192834in}}%
\pgfpathlineto{\pgfqpoint{3.951245in}{2.198174in}}%
\pgfpathlineto{\pgfqpoint{3.965038in}{2.202972in}}%
\pgfpathlineto{\pgfqpoint{3.978591in}{2.208445in}}%
\pgfpathlineto{\pgfqpoint{3.991913in}{2.215417in}}%
\pgfpathlineto{\pgfqpoint{4.005010in}{2.220333in}}%
\pgfpathlineto{\pgfqpoint{4.017891in}{2.226917in}}%
\pgfpathlineto{\pgfqpoint{4.030563in}{2.233226in}}%
\pgfpathlineto{\pgfqpoint{4.043031in}{2.237948in}}%
\pgfpathlineto{\pgfqpoint{4.055304in}{2.242623in}}%
\pgfpathlineto{\pgfqpoint{4.067386in}{2.247292in}}%
\pgfpathlineto{\pgfqpoint{4.079284in}{2.252642in}}%
\pgfpathlineto{\pgfqpoint{4.091002in}{2.257858in}}%
\pgfpathlineto{\pgfqpoint{4.102547in}{2.264001in}}%
\pgfpathlineto{\pgfqpoint{4.113924in}{2.270044in}}%
\pgfpathlineto{\pgfqpoint{4.125137in}{2.274909in}}%
\pgfpathlineto{\pgfqpoint{4.136191in}{2.280591in}}%
\pgfpathlineto{\pgfqpoint{4.147090in}{2.285460in}}%
\pgfpathlineto{\pgfqpoint{4.157839in}{2.289815in}}%
\pgfpathlineto{\pgfqpoint{4.168441in}{2.295739in}}%
\pgfpathlineto{\pgfqpoint{4.178902in}{2.299778in}}%
\pgfpathlineto{\pgfqpoint{4.189223in}{2.302617in}}%
\pgfpathlineto{\pgfqpoint{4.199410in}{2.304590in}}%
\pgfpathlineto{\pgfqpoint{4.209466in}{2.308584in}}%
\pgfpathlineto{\pgfqpoint{4.219393in}{2.312689in}}%
\pgfpathlineto{\pgfqpoint{4.229195in}{2.317751in}}%
\pgfpathlineto{\pgfqpoint{4.238876in}{2.322997in}}%
\pgfpathlineto{\pgfqpoint{4.248438in}{2.327701in}}%
\pgfpathlineto{\pgfqpoint{4.257884in}{2.332124in}}%
\pgfpathlineto{\pgfqpoint{4.267217in}{2.334321in}}%
\pgfpathlineto{\pgfqpoint{4.276439in}{2.337761in}}%
\pgfpathlineto{\pgfqpoint{4.285554in}{2.340588in}}%
\pgfpathlineto{\pgfqpoint{4.294563in}{2.344687in}}%
\pgfpathlineto{\pgfqpoint{4.303469in}{2.347663in}}%
\pgfpathlineto{\pgfqpoint{4.312274in}{2.353814in}}%
\pgfpathlineto{\pgfqpoint{4.320982in}{2.356267in}}%
\pgfpathlineto{\pgfqpoint{4.329592in}{2.360495in}}%
\pgfpathlineto{\pgfqpoint{4.338109in}{2.363054in}}%
\pgfpathlineto{\pgfqpoint{4.346534in}{2.364311in}}%
\pgfpathlineto{\pgfqpoint{4.354869in}{2.367734in}}%
\pgfpathlineto{\pgfqpoint{4.363115in}{2.370450in}}%
\pgfpathlineto{\pgfqpoint{4.371275in}{2.372731in}}%
\pgfpathlineto{\pgfqpoint{4.379351in}{2.376451in}}%
\pgfpathlineto{\pgfqpoint{4.387343in}{2.379439in}}%
\pgfpathlineto{\pgfqpoint{4.395255in}{2.383075in}}%
\pgfusepath{stroke}%
\end{pgfscope}%
\begin{pgfscope}%
\pgfpathrectangle{\pgfqpoint{0.635319in}{0.392649in}}{\pgfqpoint{3.938980in}{2.371600in}} %
\pgfusepath{clip}%
\pgfsetbuttcap%
\pgfsetroundjoin%
\pgfsetlinewidth{1.505625pt}%
\definecolor{currentstroke}{rgb}{0.172549,0.627451,0.172549}%
\pgfsetstrokecolor{currentstroke}%
\pgfsetdash{{5.550000pt}{2.400000pt}}{0.000000pt}%
\pgfpathmoveto{\pgfqpoint{0.814364in}{0.837183in}}%
\pgfpathlineto{\pgfqpoint{1.354521in}{1.090555in}}%
\pgfpathlineto{\pgfqpoint{1.670492in}{1.242222in}}%
\pgfpathlineto{\pgfqpoint{1.894677in}{1.351065in}}%
\pgfpathlineto{\pgfqpoint{2.068569in}{1.438175in}}%
\pgfpathlineto{\pgfqpoint{2.210649in}{1.506619in}}%
\pgfpathlineto{\pgfqpoint{2.330775in}{1.567917in}}%
\pgfpathlineto{\pgfqpoint{2.434834in}{1.618646in}}%
\pgfpathlineto{\pgfqpoint{2.526620in}{1.666624in}}%
\pgfpathlineto{\pgfqpoint{2.608725in}{1.709570in}}%
\pgfpathlineto{\pgfqpoint{2.682999in}{1.748134in}}%
\pgfpathlineto{\pgfqpoint{2.750805in}{1.781747in}}%
\pgfpathlineto{\pgfqpoint{2.813181in}{1.814745in}}%
\pgfpathlineto{\pgfqpoint{2.870932in}{1.843133in}}%
\pgfpathlineto{\pgfqpoint{2.924697in}{1.872735in}}%
\pgfpathlineto{\pgfqpoint{2.974991in}{1.898641in}}%
\pgfpathlineto{\pgfqpoint{3.022234in}{1.923464in}}%
\pgfpathlineto{\pgfqpoint{3.066777in}{1.946915in}}%
\pgfpathlineto{\pgfqpoint{3.108910in}{1.971414in}}%
\pgfpathlineto{\pgfqpoint{3.148882in}{1.992657in}}%
\pgfpathlineto{\pgfqpoint{3.186903in}{2.012624in}}%
\pgfpathlineto{\pgfqpoint{3.223156in}{2.032417in}}%
\pgfpathlineto{\pgfqpoint{3.257796in}{2.051183in}}%
\pgfpathlineto{\pgfqpoint{3.290962in}{2.068025in}}%
\pgfpathlineto{\pgfqpoint{3.322774in}{2.085468in}}%
\pgfpathlineto{\pgfqpoint{3.353338in}{2.102088in}}%
\pgfpathlineto{\pgfqpoint{3.382748in}{2.117061in}}%
\pgfpathlineto{\pgfqpoint{3.411089in}{2.129250in}}%
\pgfpathlineto{\pgfqpoint{3.438435in}{2.143050in}}%
\pgfpathlineto{\pgfqpoint{3.464853in}{2.158238in}}%
\pgfpathlineto{\pgfqpoint{3.490406in}{2.172056in}}%
\pgfpathlineto{\pgfqpoint{3.515147in}{2.184406in}}%
\pgfpathlineto{\pgfqpoint{3.539127in}{2.195420in}}%
\pgfpathlineto{\pgfqpoint{3.562391in}{2.206607in}}%
\pgfpathlineto{\pgfqpoint{3.584980in}{2.218755in}}%
\pgfpathlineto{\pgfqpoint{3.606933in}{2.230208in}}%
\pgfpathlineto{\pgfqpoint{3.628285in}{2.243229in}}%
\pgfpathlineto{\pgfqpoint{3.649067in}{2.252934in}}%
\pgfpathlineto{\pgfqpoint{3.669309in}{2.265537in}}%
\pgfpathlineto{\pgfqpoint{3.689039in}{2.276261in}}%
\pgfpathlineto{\pgfqpoint{3.708281in}{2.287495in}}%
\pgfpathlineto{\pgfqpoint{3.727060in}{2.297385in}}%
\pgfpathlineto{\pgfqpoint{3.745397in}{2.308348in}}%
\pgfpathlineto{\pgfqpoint{3.763312in}{2.318392in}}%
\pgfpathlineto{\pgfqpoint{3.780825in}{2.327559in}}%
\pgfpathlineto{\pgfqpoint{3.797953in}{2.336694in}}%
\pgfpathlineto{\pgfqpoint{3.814712in}{2.345877in}}%
\pgfpathlineto{\pgfqpoint{3.831119in}{2.354245in}}%
\pgfpathlineto{\pgfqpoint{3.847187in}{2.362627in}}%
\pgfpathlineto{\pgfqpoint{3.862930in}{2.369352in}}%
\pgfpathlineto{\pgfqpoint{3.878362in}{2.378289in}}%
\pgfpathlineto{\pgfqpoint{3.893494in}{2.389425in}}%
\pgfpathlineto{\pgfqpoint{3.908338in}{2.397394in}}%
\pgfpathlineto{\pgfqpoint{3.922905in}{2.403846in}}%
\pgfpathlineto{\pgfqpoint{3.937204in}{2.410794in}}%
\pgfpathlineto{\pgfqpoint{3.951245in}{2.418281in}}%
\pgfpathlineto{\pgfqpoint{3.965038in}{2.425240in}}%
\pgfpathlineto{\pgfqpoint{3.978591in}{2.431820in}}%
\pgfpathlineto{\pgfqpoint{3.991913in}{2.439634in}}%
\pgfpathlineto{\pgfqpoint{4.005010in}{2.446790in}}%
\pgfpathlineto{\pgfqpoint{4.017891in}{2.454998in}}%
\pgfpathlineto{\pgfqpoint{4.030563in}{2.462291in}}%
\pgfpathlineto{\pgfqpoint{4.043031in}{2.468460in}}%
\pgfpathlineto{\pgfqpoint{4.055304in}{2.473969in}}%
\pgfpathlineto{\pgfqpoint{4.067386in}{2.479773in}}%
\pgfpathlineto{\pgfqpoint{4.079284in}{2.486567in}}%
\pgfpathlineto{\pgfqpoint{4.091002in}{2.493682in}}%
\pgfpathlineto{\pgfqpoint{4.102547in}{2.501542in}}%
\pgfpathlineto{\pgfqpoint{4.113924in}{2.509296in}}%
\pgfpathlineto{\pgfqpoint{4.125137in}{2.515337in}}%
\pgfpathlineto{\pgfqpoint{4.136191in}{2.522132in}}%
\pgfpathlineto{\pgfqpoint{4.147090in}{2.528520in}}%
\pgfpathlineto{\pgfqpoint{4.157839in}{2.534868in}}%
\pgfpathlineto{\pgfqpoint{4.168441in}{2.541797in}}%
\pgfpathlineto{\pgfqpoint{4.178902in}{2.546513in}}%
\pgfpathlineto{\pgfqpoint{4.189223in}{2.550899in}}%
\pgfpathlineto{\pgfqpoint{4.199410in}{2.554557in}}%
\pgfpathlineto{\pgfqpoint{4.209466in}{2.559616in}}%
\pgfpathlineto{\pgfqpoint{4.219393in}{2.564166in}}%
\pgfpathlineto{\pgfqpoint{4.229195in}{2.570064in}}%
\pgfpathlineto{\pgfqpoint{4.238876in}{2.576488in}}%
\pgfpathlineto{\pgfqpoint{4.248438in}{2.582776in}}%
\pgfpathlineto{\pgfqpoint{4.257884in}{2.587968in}}%
\pgfpathlineto{\pgfqpoint{4.267217in}{2.592218in}}%
\pgfpathlineto{\pgfqpoint{4.276439in}{2.597270in}}%
\pgfpathlineto{\pgfqpoint{4.285554in}{2.601011in}}%
\pgfpathlineto{\pgfqpoint{4.294563in}{2.606224in}}%
\pgfpathlineto{\pgfqpoint{4.303469in}{2.610413in}}%
\pgfpathlineto{\pgfqpoint{4.312274in}{2.617038in}}%
\pgfpathlineto{\pgfqpoint{4.320982in}{2.620232in}}%
\pgfpathlineto{\pgfqpoint{4.329592in}{2.625375in}}%
\pgfpathlineto{\pgfqpoint{4.338109in}{2.629538in}}%
\pgfpathlineto{\pgfqpoint{4.346534in}{2.631647in}}%
\pgfpathlineto{\pgfqpoint{4.354869in}{2.636117in}}%
\pgfpathlineto{\pgfqpoint{4.363115in}{2.640288in}}%
\pgfpathlineto{\pgfqpoint{4.371275in}{2.643941in}}%
\pgfpathlineto{\pgfqpoint{4.379351in}{2.648927in}}%
\pgfpathlineto{\pgfqpoint{4.387343in}{2.652628in}}%
\pgfpathlineto{\pgfqpoint{4.395255in}{2.656449in}}%
\pgfusepath{stroke}%
\end{pgfscope}%
\begin{pgfscope}%
\pgfsetrectcap%
\pgfsetmiterjoin%
\pgfsetlinewidth{0.803000pt}%
\definecolor{currentstroke}{rgb}{0.000000,0.000000,0.000000}%
\pgfsetstrokecolor{currentstroke}%
\pgfsetdash{}{0pt}%
\pgfpathmoveto{\pgfqpoint{0.635319in}{0.392649in}}%
\pgfpathlineto{\pgfqpoint{0.635319in}{2.764249in}}%
\pgfusepath{stroke}%
\end{pgfscope}%
\begin{pgfscope}%
\pgfsetrectcap%
\pgfsetmiterjoin%
\pgfsetlinewidth{0.803000pt}%
\definecolor{currentstroke}{rgb}{0.000000,0.000000,0.000000}%
\pgfsetstrokecolor{currentstroke}%
\pgfsetdash{}{0pt}%
\pgfpathmoveto{\pgfqpoint{4.574300in}{0.392649in}}%
\pgfpathlineto{\pgfqpoint{4.574300in}{2.764249in}}%
\pgfusepath{stroke}%
\end{pgfscope}%
\begin{pgfscope}%
\pgfsetrectcap%
\pgfsetmiterjoin%
\pgfsetlinewidth{0.803000pt}%
\definecolor{currentstroke}{rgb}{0.000000,0.000000,0.000000}%
\pgfsetstrokecolor{currentstroke}%
\pgfsetdash{}{0pt}%
\pgfpathmoveto{\pgfqpoint{0.635319in}{0.392649in}}%
\pgfpathlineto{\pgfqpoint{4.574300in}{0.392649in}}%
\pgfusepath{stroke}%
\end{pgfscope}%
\begin{pgfscope}%
\pgfsetrectcap%
\pgfsetmiterjoin%
\pgfsetlinewidth{0.803000pt}%
\definecolor{currentstroke}{rgb}{0.000000,0.000000,0.000000}%
\pgfsetstrokecolor{currentstroke}%
\pgfsetdash{}{0pt}%
\pgfpathmoveto{\pgfqpoint{0.635319in}{2.764249in}}%
\pgfpathlineto{\pgfqpoint{4.574300in}{2.764249in}}%
\pgfusepath{stroke}%
\end{pgfscope}%
\begin{pgfscope}%
\pgfsetbuttcap%
\pgfsetmiterjoin%
\definecolor{currentfill}{rgb}{1.000000,1.000000,1.000000}%
\pgfsetfillcolor{currentfill}%
\pgfsetfillopacity{0.800000}%
\pgfsetlinewidth{1.003750pt}%
\definecolor{currentstroke}{rgb}{0.800000,0.800000,0.800000}%
\pgfsetstrokecolor{currentstroke}%
\pgfsetstrokeopacity{0.800000}%
\pgfsetdash{}{0pt}%
\pgfpathmoveto{\pgfqpoint{0.713097in}{2.210561in}}%
\pgfpathlineto{\pgfqpoint{1.662489in}{2.210561in}}%
\pgfpathquadraticcurveto{\pgfqpoint{1.684711in}{2.210561in}}{\pgfqpoint{1.684711in}{2.232783in}}%
\pgfpathlineto{\pgfqpoint{1.684711in}{2.686471in}}%
\pgfpathquadraticcurveto{\pgfqpoint{1.684711in}{2.708693in}}{\pgfqpoint{1.662489in}{2.708693in}}%
\pgfpathlineto{\pgfqpoint{0.713097in}{2.708693in}}%
\pgfpathquadraticcurveto{\pgfqpoint{0.690875in}{2.708693in}}{\pgfqpoint{0.690875in}{2.686471in}}%
\pgfpathlineto{\pgfqpoint{0.690875in}{2.232783in}}%
\pgfpathquadraticcurveto{\pgfqpoint{0.690875in}{2.210561in}}{\pgfqpoint{0.713097in}{2.210561in}}%
\pgfpathclose%
\pgfusepath{stroke,fill}%
\end{pgfscope}%
\begin{pgfscope}%
\pgfsetrectcap%
\pgfsetroundjoin%
\pgfsetlinewidth{1.505625pt}%
\definecolor{currentstroke}{rgb}{0.121569,0.466667,0.705882}%
\pgfsetstrokecolor{currentstroke}%
\pgfsetdash{}{0pt}%
\pgfpathmoveto{\pgfqpoint{0.735319in}{2.625360in}}%
\pgfpathlineto{\pgfqpoint{0.957542in}{2.625360in}}%
\pgfusepath{stroke}%
\end{pgfscope}%
\begin{pgfscope}%
\pgftext[x=1.046430in,y=2.586471in,left,base]{\rmfamily\fontsize{8.000000}{9.600000}\selectfont OGI}%
\end{pgfscope}%
\begin{pgfscope}%
\pgfsetbuttcap%
\pgfsetroundjoin%
\pgfsetlinewidth{1.505625pt}%
\definecolor{currentstroke}{rgb}{1.000000,0.498039,0.054902}%
\pgfsetstrokecolor{currentstroke}%
\pgfsetdash{{1.500000pt}{2.475000pt}}{0.000000pt}%
\pgfpathmoveto{\pgfqpoint{0.735319in}{2.470427in}}%
\pgfpathlineto{\pgfqpoint{0.957542in}{2.470427in}}%
\pgfusepath{stroke}%
\end{pgfscope}%
\begin{pgfscope}%
\pgftext[x=1.046430in,y=2.431538in,left,base]{\rmfamily\fontsize{8.000000}{9.600000}\selectfont Thompson}%
\end{pgfscope}%
\begin{pgfscope}%
\pgfsetbuttcap%
\pgfsetroundjoin%
\pgfsetlinewidth{1.505625pt}%
\definecolor{currentstroke}{rgb}{0.172549,0.627451,0.172549}%
\pgfsetstrokecolor{currentstroke}%
\pgfsetdash{{5.550000pt}{2.400000pt}}{0.000000pt}%
\pgfpathmoveto{\pgfqpoint{0.735319in}{2.315494in}}%
\pgfpathlineto{\pgfqpoint{0.957542in}{2.315494in}}%
\pgfusepath{stroke}%
\end{pgfscope}%
\begin{pgfscope}%
\pgftext[x=1.046430in,y=2.276605in,left,base]{\rmfamily\fontsize{8.000000}{9.600000}\selectfont Bayes UCB}%
\end{pgfscope}%
\end{pgfpicture}%
\makeatother%
\endgroup%

	\caption{Cumulative regret in the large-scale problem of this section averaged over 5,000 independent trials.
		We plot the number of periods, $T$ on a logarithmic scale.}
	\label{fig:chapelle_and_li}
\end{figure}

\begin{table}[h!]
	\centering
	\color{blue}
	\begin{tabular}{c|ccccc}
		\toprule
		$T/A$ &   OGI &  Thompson &   Bayes UCB &  Rel. improvement (\%) &  Abs. improvement \\
		\midrule
		2,000 & 230.5 &     284.4 & 297.9 &                            18.9 &                  53.9 \\
		4,000 & 254.7 &     311.6 & 333.5 &                            18.3 &                  57.0 \\
		6,000 & 268.6 &     327.4 & 354.5 &                            18.0 &                  58.8 \\
		8,000 & 279.1 &     339.2 & 369.6 &                            17.7 &                  60.1 \\
		10,000 & 287.1 &     347.7 & 380.7 &                            17.4 &                  60.6 \\
		\bottomrule
	\end{tabular}
	
	\caption{Regret in the large scale experiment from OGI, Thompson Sampling and Bayes UCB. The last two columns show the relative and absolute difference from Thompson Sampling, which is the closest competitor to OGI.}
	\label{table:additional_cli_table}
\end{table}

As before, the OGI scheme consistently dominates the other two. What is particularly interesting is that despite going out to a horizon of $10^6$ time periods, the relative improvement in regret over these algorithms remains substantial. For instance, going from a horizon length of $2 \times 10^5$ (corresponding to a heuristic budget of {\color{blue}$T/A = 2,000$} pulls per arm) 
to a horizon length of $10^6$ (corresponding to a heuristic budget of {\color{blue}$T/A = 10,000$ }pulls per arm) resulted in the relative improvement offered by OGI shrining only marginally, from $18.9\%$ to $17.4\%$. 

% and the relative margin by which OGI outperforms the other algorithms appears to decline, albeit steadily with the horizon.
%In particular, we notice that the improvement in regret, over the nearest competitor Thompson Sampling, is 18.9\% when there are $2\times 10^5$ periods but this improvement decreases to almost 17.4\% with $10^6$ time periods.
%On the other hand, the \emph{absolute} difference in regret increases monotonically with $T$.
%Therefore our method appears to consistently retain its advantage over Thompson Sampling when $T$ is large even if, in a relative sense, the performance gap shrinks (as we would expect from the asymptotic bound).



\subsection{Bandits with multiple arm pulls}

In this section, we consider a somewhat exploratory experiment; we seek to adapt OGI to a more complex bandit problem (here, we allow for multiple simultaneous arm pulls). Again, in the discounted infinite horizon setting, a number of heuristic approaches have been proposed to adapt the Gittins index to more complex settings; a good example is the so-called Whittle relaxation for restless bandits. One might consider applying those same heuristic strategies to the optimistic gittins index. 

For this experiment, we consider a more general MAB problem, where the agent is able to pull up to a certain number (say $m < A$) of the arms simultaneously. In order to describe the problem, we recall that $A$ is the total number of arms and define  $\mathcal{D}_m := \{d \in \{0,1\}^A : \sum_i d_i \le m\}$ to be the set of all $A$-dimensional binary vectors with up to $m$ ones in them, which we take to be the action space. Let $X_t = (X_{1,t}, \ldots, {\color{blue}X_{A,t})}$ be a tuple of (potential) rewards from the $A$ arms at time $t$, where the definition of $X_{i,t}$ for any arm $i$ is the same as in Section~\ref{sec:model_and_prelim}. Given a decision $d \in \mathcal D_m$, which encodes the subset of arms pulled, the reward $d^\top X_t$ is earned and an arm $j$'s reward $X_{j,t}$ is observed if and only if that arm is pulled, i.e. $d_{j} = 1$. We can then define a policy $(\pi_t, t \in \mathbb{N})$ to be a $\mathcal{D}_m$-valued stochastic process adapted to an information set generated by past actions and observed feedback. 
%where $\pi_{t+1}$ is measurable with respect to $\sigma\left( (\pi_s, (\pi_{i,s} X_{i,s}, \;i =1,\ldots,A)), \; s=1,\ldots,t\right)$. That is a policy's information set comes from its previous decisions and observed feedback.
 A policy $\pi$'s regret is given by the equation 
\[
\color{blue}
\Regret{\pi, T} = T \cdot  \Ee{\max_{d \in \mathcal D_m}   \sum_{i=1}^A d_i \mu(\theta_i)} - \sum_{t=1}^T \E[\pi_t^\top X_t ]
\]
where the expectation is over both the randomness in the rewards, the prior and the policy's actions.

We propose a heuristic to this problem using our scheme, which is to compute the Optimistic Gittins Index of every arm, at time $t$, using a discount factor of $1-1/t$ (just as before). However, for this problem, we pick $m$ arms with the largest indices. This is essentially Whittle's heuristic \citep{whittle1988restless}, which was originally given for the restless bandit problem but can be described as picking several arms with the largest Gittins indices. {\color{blue} We break ties randomly and it would be interesting to explore smarter tiebreaking schemes that could improve performance \citep{brown2017index}}.

To test our policy, we simulate $A = 6$ binary arms with uniformly distributed biases and fix $m=3$. 
We benchmark our heuristic against Thompson Sampling and IDS. Because the arms give independent Bernoulli rewards, we will use a flat Beta prior for all of the algorithms. We implement the version of IDS designed for the linear bandit problem because this experiment is a special case of a linear bandit. Our implementation of IDS also uses {\color{blue} 10,000 Monte Carlo} samples per iteration.

The results, produced from 1,000 independent trials, are summarized in Figure~\ref{fig:restless1} and Table~\ref{table:restless1_summary}. We notice a significant spread in the performance between OGI and both Thompson Sampling and IDS. Just like for our main algorithm, the primary computational bottleneck in using OGI comes from solving the stopping problem and this can be onerous if $K$ is large. However, as the results suggest, the policy works well even for low to moderate look-ahead parameters. The experiment here sets the stage for an exploration of the appropriate extensions to the OGI algorithm for more complex bandit problems (such as contextual bandits) which we leave for future work. 

The results, produced from 2,000 independent trials, are summarized in Figure~\ref{fig:restless1} and Table~\ref{table:restless1_summary}. The horizon is limited to 250 time periods because of the increased computational effort required to execute a single trial of both the IDS algorithm and Whittle's heuristic, when $K > 1$. This extra time is on the order of minutes for these algorithms.
For the sake of simplicity, we dub this algorithm as exactly `Whittle's heuristic' for the remainder of this section.

{\color{blue}We notice a significant spread in performance between Whittle's heuristic and Thompson Sampling. Meanwhile, IDS and Whittle's heuristic show similar performance, however the computational cost of running the latter algorithm is up to 25 times lower (IDS requires generating 10,000 Monte Carlo samples on each iteration). Just like for our main algorithm, the primary computational bottleneck in using Whittle's heuristic comes from solving the stopping problem and this can be onerous if $K$ is large. However, as the results suggest, the policy works well even for low to moderate look-ahead parameters but nonetheless improves slightly when $K$ increases.% By contrast, IDS is one of the slowest algorithm because it needs to generate a 10,000 Monte Carlo samples in every iteration.}
}

\begin{figure}
	\centering
	%% Creator: Matplotlib, PGF backend
%%
%% To include the figure in your LaTeX document, write
%%   \input{<filename>.pgf}
%%
%% Make sure the required packages are loaded in your preamble
%%   \usepackage{pgf}
%%
%% Figures using additional raster images can only be included by \input if
%% they are in the same directory as the main LaTeX file. For loading figures
%% from other directories you can use the `import` package
%%   \usepackage{import}
%% and then include the figures with
%%   \import{<path to file>}{<filename>.pgf}
%%
%% Matplotlib used the following preamble
%%   \usepackage[utf8x]{inputenc}
%%   \usepackage[T1]{fontenc}
%%
\begingroup%
\makeatletter%
\begin{pgfpicture}%
\pgfpathrectangle{\pgfpointorigin}{\pgfqpoint{4.875000in}{3.012916in}}%
\pgfusepath{use as bounding box, clip}%
\begin{pgfscope}%
\pgfsetbuttcap%
\pgfsetmiterjoin%
\definecolor{currentfill}{rgb}{1.000000,1.000000,1.000000}%
\pgfsetfillcolor{currentfill}%
\pgfsetlinewidth{0.000000pt}%
\definecolor{currentstroke}{rgb}{1.000000,1.000000,1.000000}%
\pgfsetstrokecolor{currentstroke}%
\pgfsetdash{}{0pt}%
\pgfpathmoveto{\pgfqpoint{0.000000in}{0.000000in}}%
\pgfpathlineto{\pgfqpoint{4.875000in}{0.000000in}}%
\pgfpathlineto{\pgfqpoint{4.875000in}{3.012916in}}%
\pgfpathlineto{\pgfqpoint{0.000000in}{3.012916in}}%
\pgfpathclose%
\pgfusepath{fill}%
\end{pgfscope}%
\begin{pgfscope}%
\pgfsetbuttcap%
\pgfsetmiterjoin%
\definecolor{currentfill}{rgb}{1.000000,1.000000,1.000000}%
\pgfsetfillcolor{currentfill}%
\pgfsetlinewidth{0.000000pt}%
\definecolor{currentstroke}{rgb}{0.000000,0.000000,0.000000}%
\pgfsetstrokecolor{currentstroke}%
\pgfsetstrokeopacity{0.000000}%
\pgfsetdash{}{0pt}%
\pgfpathmoveto{\pgfqpoint{0.609375in}{0.376614in}}%
\pgfpathlineto{\pgfqpoint{4.387500in}{0.376614in}}%
\pgfpathlineto{\pgfqpoint{4.387500in}{2.651366in}}%
\pgfpathlineto{\pgfqpoint{0.609375in}{2.651366in}}%
\pgfpathclose%
\pgfusepath{fill}%
\end{pgfscope}%
\begin{pgfscope}%
\pgfsetbuttcap%
\pgfsetroundjoin%
\definecolor{currentfill}{rgb}{0.000000,0.000000,0.000000}%
\pgfsetfillcolor{currentfill}%
\pgfsetlinewidth{0.803000pt}%
\definecolor{currentstroke}{rgb}{0.000000,0.000000,0.000000}%
\pgfsetstrokecolor{currentstroke}%
\pgfsetdash{}{0pt}%
\pgfsys@defobject{currentmarker}{\pgfqpoint{0.000000in}{-0.048611in}}{\pgfqpoint{0.000000in}{0.000000in}}{%
\pgfpathmoveto{\pgfqpoint{0.000000in}{0.000000in}}%
\pgfpathlineto{\pgfqpoint{0.000000in}{-0.048611in}}%
\pgfusepath{stroke,fill}%
}%
\begin{pgfscope}%
\pgfsys@transformshift{0.781108in}{0.376614in}%
\pgfsys@useobject{currentmarker}{}%
\end{pgfscope}%
\end{pgfscope}%
\begin{pgfscope}%
\pgftext[x=0.781108in,y=0.279392in,,top]{\rmfamily\fontsize{8.000000}{9.600000}\selectfont \(\displaystyle 0\)}%
\end{pgfscope}%
\begin{pgfscope}%
\pgfsetbuttcap%
\pgfsetroundjoin%
\definecolor{currentfill}{rgb}{0.000000,0.000000,0.000000}%
\pgfsetfillcolor{currentfill}%
\pgfsetlinewidth{0.803000pt}%
\definecolor{currentstroke}{rgb}{0.000000,0.000000,0.000000}%
\pgfsetstrokecolor{currentstroke}%
\pgfsetdash{}{0pt}%
\pgfsys@defobject{currentmarker}{\pgfqpoint{0.000000in}{-0.048611in}}{\pgfqpoint{0.000000in}{0.000000in}}{%
\pgfpathmoveto{\pgfqpoint{0.000000in}{0.000000in}}%
\pgfpathlineto{\pgfqpoint{0.000000in}{-0.048611in}}%
\pgfusepath{stroke,fill}%
}%
\begin{pgfscope}%
\pgfsys@transformshift{1.470799in}{0.376614in}%
\pgfsys@useobject{currentmarker}{}%
\end{pgfscope}%
\end{pgfscope}%
\begin{pgfscope}%
\pgftext[x=1.470799in,y=0.279392in,,top]{\rmfamily\fontsize{8.000000}{9.600000}\selectfont \(\displaystyle 50\)}%
\end{pgfscope}%
\begin{pgfscope}%
\pgfsetbuttcap%
\pgfsetroundjoin%
\definecolor{currentfill}{rgb}{0.000000,0.000000,0.000000}%
\pgfsetfillcolor{currentfill}%
\pgfsetlinewidth{0.803000pt}%
\definecolor{currentstroke}{rgb}{0.000000,0.000000,0.000000}%
\pgfsetstrokecolor{currentstroke}%
\pgfsetdash{}{0pt}%
\pgfsys@defobject{currentmarker}{\pgfqpoint{0.000000in}{-0.048611in}}{\pgfqpoint{0.000000in}{0.000000in}}{%
\pgfpathmoveto{\pgfqpoint{0.000000in}{0.000000in}}%
\pgfpathlineto{\pgfqpoint{0.000000in}{-0.048611in}}%
\pgfusepath{stroke,fill}%
}%
\begin{pgfscope}%
\pgfsys@transformshift{2.160489in}{0.376614in}%
\pgfsys@useobject{currentmarker}{}%
\end{pgfscope}%
\end{pgfscope}%
\begin{pgfscope}%
\pgftext[x=2.160489in,y=0.279392in,,top]{\rmfamily\fontsize{8.000000}{9.600000}\selectfont \(\displaystyle 100\)}%
\end{pgfscope}%
\begin{pgfscope}%
\pgfsetbuttcap%
\pgfsetroundjoin%
\definecolor{currentfill}{rgb}{0.000000,0.000000,0.000000}%
\pgfsetfillcolor{currentfill}%
\pgfsetlinewidth{0.803000pt}%
\definecolor{currentstroke}{rgb}{0.000000,0.000000,0.000000}%
\pgfsetstrokecolor{currentstroke}%
\pgfsetdash{}{0pt}%
\pgfsys@defobject{currentmarker}{\pgfqpoint{0.000000in}{-0.048611in}}{\pgfqpoint{0.000000in}{0.000000in}}{%
\pgfpathmoveto{\pgfqpoint{0.000000in}{0.000000in}}%
\pgfpathlineto{\pgfqpoint{0.000000in}{-0.048611in}}%
\pgfusepath{stroke,fill}%
}%
\begin{pgfscope}%
\pgfsys@transformshift{2.850180in}{0.376614in}%
\pgfsys@useobject{currentmarker}{}%
\end{pgfscope}%
\end{pgfscope}%
\begin{pgfscope}%
\pgftext[x=2.850180in,y=0.279392in,,top]{\rmfamily\fontsize{8.000000}{9.600000}\selectfont \(\displaystyle 150\)}%
\end{pgfscope}%
\begin{pgfscope}%
\pgfsetbuttcap%
\pgfsetroundjoin%
\definecolor{currentfill}{rgb}{0.000000,0.000000,0.000000}%
\pgfsetfillcolor{currentfill}%
\pgfsetlinewidth{0.803000pt}%
\definecolor{currentstroke}{rgb}{0.000000,0.000000,0.000000}%
\pgfsetstrokecolor{currentstroke}%
\pgfsetdash{}{0pt}%
\pgfsys@defobject{currentmarker}{\pgfqpoint{0.000000in}{-0.048611in}}{\pgfqpoint{0.000000in}{0.000000in}}{%
\pgfpathmoveto{\pgfqpoint{0.000000in}{0.000000in}}%
\pgfpathlineto{\pgfqpoint{0.000000in}{-0.048611in}}%
\pgfusepath{stroke,fill}%
}%
\begin{pgfscope}%
\pgfsys@transformshift{3.539870in}{0.376614in}%
\pgfsys@useobject{currentmarker}{}%
\end{pgfscope}%
\end{pgfscope}%
\begin{pgfscope}%
\pgftext[x=3.539870in,y=0.279392in,,top]{\rmfamily\fontsize{8.000000}{9.600000}\selectfont \(\displaystyle 200\)}%
\end{pgfscope}%
\begin{pgfscope}%
\pgfsetbuttcap%
\pgfsetroundjoin%
\definecolor{currentfill}{rgb}{0.000000,0.000000,0.000000}%
\pgfsetfillcolor{currentfill}%
\pgfsetlinewidth{0.803000pt}%
\definecolor{currentstroke}{rgb}{0.000000,0.000000,0.000000}%
\pgfsetstrokecolor{currentstroke}%
\pgfsetdash{}{0pt}%
\pgfsys@defobject{currentmarker}{\pgfqpoint{0.000000in}{-0.048611in}}{\pgfqpoint{0.000000in}{0.000000in}}{%
\pgfpathmoveto{\pgfqpoint{0.000000in}{0.000000in}}%
\pgfpathlineto{\pgfqpoint{0.000000in}{-0.048611in}}%
\pgfusepath{stroke,fill}%
}%
\begin{pgfscope}%
\pgfsys@transformshift{4.229561in}{0.376614in}%
\pgfsys@useobject{currentmarker}{}%
\end{pgfscope}%
\end{pgfscope}%
\begin{pgfscope}%
\pgftext[x=4.229561in,y=0.279392in,,top]{\rmfamily\fontsize{8.000000}{9.600000}\selectfont \(\displaystyle 250\)}%
\end{pgfscope}%
\begin{pgfscope}%
\pgftext[x=2.498438in,y=0.125712in,,top]{\rmfamily\fontsize{10.000000}{12.000000}\selectfont \(\displaystyle T\)}%
\end{pgfscope}%
\begin{pgfscope}%
\pgfsetbuttcap%
\pgfsetroundjoin%
\definecolor{currentfill}{rgb}{0.000000,0.000000,0.000000}%
\pgfsetfillcolor{currentfill}%
\pgfsetlinewidth{0.803000pt}%
\definecolor{currentstroke}{rgb}{0.000000,0.000000,0.000000}%
\pgfsetstrokecolor{currentstroke}%
\pgfsetdash{}{0pt}%
\pgfsys@defobject{currentmarker}{\pgfqpoint{-0.048611in}{0.000000in}}{\pgfqpoint{0.000000in}{0.000000in}}{%
\pgfpathmoveto{\pgfqpoint{0.000000in}{0.000000in}}%
\pgfpathlineto{\pgfqpoint{-0.048611in}{0.000000in}}%
\pgfusepath{stroke,fill}%
}%
\begin{pgfscope}%
\pgfsys@transformshift{0.609375in}{0.389795in}%
\pgfsys@useobject{currentmarker}{}%
\end{pgfscope}%
\end{pgfscope}%
\begin{pgfscope}%
\pgftext[x=0.453124in,y=0.351533in,left,base]{\rmfamily\fontsize{8.000000}{9.600000}\selectfont \(\displaystyle 0\)}%
\end{pgfscope}%
\begin{pgfscope}%
\pgfsetbuttcap%
\pgfsetroundjoin%
\definecolor{currentfill}{rgb}{0.000000,0.000000,0.000000}%
\pgfsetfillcolor{currentfill}%
\pgfsetlinewidth{0.803000pt}%
\definecolor{currentstroke}{rgb}{0.000000,0.000000,0.000000}%
\pgfsetstrokecolor{currentstroke}%
\pgfsetdash{}{0pt}%
\pgfsys@defobject{currentmarker}{\pgfqpoint{-0.048611in}{0.000000in}}{\pgfqpoint{0.000000in}{0.000000in}}{%
\pgfpathmoveto{\pgfqpoint{0.000000in}{0.000000in}}%
\pgfpathlineto{\pgfqpoint{-0.048611in}{0.000000in}}%
\pgfusepath{stroke,fill}%
}%
\begin{pgfscope}%
\pgfsys@transformshift{0.609375in}{0.669940in}%
\pgfsys@useobject{currentmarker}{}%
\end{pgfscope}%
\end{pgfscope}%
\begin{pgfscope}%
\pgftext[x=0.453124in,y=0.631677in,left,base]{\rmfamily\fontsize{8.000000}{9.600000}\selectfont \(\displaystyle 2\)}%
\end{pgfscope}%
\begin{pgfscope}%
\pgfsetbuttcap%
\pgfsetroundjoin%
\definecolor{currentfill}{rgb}{0.000000,0.000000,0.000000}%
\pgfsetfillcolor{currentfill}%
\pgfsetlinewidth{0.803000pt}%
\definecolor{currentstroke}{rgb}{0.000000,0.000000,0.000000}%
\pgfsetstrokecolor{currentstroke}%
\pgfsetdash{}{0pt}%
\pgfsys@defobject{currentmarker}{\pgfqpoint{-0.048611in}{0.000000in}}{\pgfqpoint{0.000000in}{0.000000in}}{%
\pgfpathmoveto{\pgfqpoint{0.000000in}{0.000000in}}%
\pgfpathlineto{\pgfqpoint{-0.048611in}{0.000000in}}%
\pgfusepath{stroke,fill}%
}%
\begin{pgfscope}%
\pgfsys@transformshift{0.609375in}{0.950084in}%
\pgfsys@useobject{currentmarker}{}%
\end{pgfscope}%
\end{pgfscope}%
\begin{pgfscope}%
\pgftext[x=0.453124in,y=0.911822in,left,base]{\rmfamily\fontsize{8.000000}{9.600000}\selectfont \(\displaystyle 4\)}%
\end{pgfscope}%
\begin{pgfscope}%
\pgfsetbuttcap%
\pgfsetroundjoin%
\definecolor{currentfill}{rgb}{0.000000,0.000000,0.000000}%
\pgfsetfillcolor{currentfill}%
\pgfsetlinewidth{0.803000pt}%
\definecolor{currentstroke}{rgb}{0.000000,0.000000,0.000000}%
\pgfsetstrokecolor{currentstroke}%
\pgfsetdash{}{0pt}%
\pgfsys@defobject{currentmarker}{\pgfqpoint{-0.048611in}{0.000000in}}{\pgfqpoint{0.000000in}{0.000000in}}{%
\pgfpathmoveto{\pgfqpoint{0.000000in}{0.000000in}}%
\pgfpathlineto{\pgfqpoint{-0.048611in}{0.000000in}}%
\pgfusepath{stroke,fill}%
}%
\begin{pgfscope}%
\pgfsys@transformshift{0.609375in}{1.230229in}%
\pgfsys@useobject{currentmarker}{}%
\end{pgfscope}%
\end{pgfscope}%
\begin{pgfscope}%
\pgftext[x=0.453124in,y=1.191967in,left,base]{\rmfamily\fontsize{8.000000}{9.600000}\selectfont \(\displaystyle 6\)}%
\end{pgfscope}%
\begin{pgfscope}%
\pgfsetbuttcap%
\pgfsetroundjoin%
\definecolor{currentfill}{rgb}{0.000000,0.000000,0.000000}%
\pgfsetfillcolor{currentfill}%
\pgfsetlinewidth{0.803000pt}%
\definecolor{currentstroke}{rgb}{0.000000,0.000000,0.000000}%
\pgfsetstrokecolor{currentstroke}%
\pgfsetdash{}{0pt}%
\pgfsys@defobject{currentmarker}{\pgfqpoint{-0.048611in}{0.000000in}}{\pgfqpoint{0.000000in}{0.000000in}}{%
\pgfpathmoveto{\pgfqpoint{0.000000in}{0.000000in}}%
\pgfpathlineto{\pgfqpoint{-0.048611in}{0.000000in}}%
\pgfusepath{stroke,fill}%
}%
\begin{pgfscope}%
\pgfsys@transformshift{0.609375in}{1.510374in}%
\pgfsys@useobject{currentmarker}{}%
\end{pgfscope}%
\end{pgfscope}%
\begin{pgfscope}%
\pgftext[x=0.453124in,y=1.472112in,left,base]{\rmfamily\fontsize{8.000000}{9.600000}\selectfont \(\displaystyle 8\)}%
\end{pgfscope}%
\begin{pgfscope}%
\pgfsetbuttcap%
\pgfsetroundjoin%
\definecolor{currentfill}{rgb}{0.000000,0.000000,0.000000}%
\pgfsetfillcolor{currentfill}%
\pgfsetlinewidth{0.803000pt}%
\definecolor{currentstroke}{rgb}{0.000000,0.000000,0.000000}%
\pgfsetstrokecolor{currentstroke}%
\pgfsetdash{}{0pt}%
\pgfsys@defobject{currentmarker}{\pgfqpoint{-0.048611in}{0.000000in}}{\pgfqpoint{0.000000in}{0.000000in}}{%
\pgfpathmoveto{\pgfqpoint{0.000000in}{0.000000in}}%
\pgfpathlineto{\pgfqpoint{-0.048611in}{0.000000in}}%
\pgfusepath{stroke,fill}%
}%
\begin{pgfscope}%
\pgfsys@transformshift{0.609375in}{1.790518in}%
\pgfsys@useobject{currentmarker}{}%
\end{pgfscope}%
\end{pgfscope}%
\begin{pgfscope}%
\pgftext[x=0.394096in,y=1.752256in,left,base]{\rmfamily\fontsize{8.000000}{9.600000}\selectfont \(\displaystyle 10\)}%
\end{pgfscope}%
\begin{pgfscope}%
\pgfsetbuttcap%
\pgfsetroundjoin%
\definecolor{currentfill}{rgb}{0.000000,0.000000,0.000000}%
\pgfsetfillcolor{currentfill}%
\pgfsetlinewidth{0.803000pt}%
\definecolor{currentstroke}{rgb}{0.000000,0.000000,0.000000}%
\pgfsetstrokecolor{currentstroke}%
\pgfsetdash{}{0pt}%
\pgfsys@defobject{currentmarker}{\pgfqpoint{-0.048611in}{0.000000in}}{\pgfqpoint{0.000000in}{0.000000in}}{%
\pgfpathmoveto{\pgfqpoint{0.000000in}{0.000000in}}%
\pgfpathlineto{\pgfqpoint{-0.048611in}{0.000000in}}%
\pgfusepath{stroke,fill}%
}%
\begin{pgfscope}%
\pgfsys@transformshift{0.609375in}{2.070663in}%
\pgfsys@useobject{currentmarker}{}%
\end{pgfscope}%
\end{pgfscope}%
\begin{pgfscope}%
\pgftext[x=0.394096in,y=2.032401in,left,base]{\rmfamily\fontsize{8.000000}{9.600000}\selectfont \(\displaystyle 12\)}%
\end{pgfscope}%
\begin{pgfscope}%
\pgfsetbuttcap%
\pgfsetroundjoin%
\definecolor{currentfill}{rgb}{0.000000,0.000000,0.000000}%
\pgfsetfillcolor{currentfill}%
\pgfsetlinewidth{0.803000pt}%
\definecolor{currentstroke}{rgb}{0.000000,0.000000,0.000000}%
\pgfsetstrokecolor{currentstroke}%
\pgfsetdash{}{0pt}%
\pgfsys@defobject{currentmarker}{\pgfqpoint{-0.048611in}{0.000000in}}{\pgfqpoint{0.000000in}{0.000000in}}{%
\pgfpathmoveto{\pgfqpoint{0.000000in}{0.000000in}}%
\pgfpathlineto{\pgfqpoint{-0.048611in}{0.000000in}}%
\pgfusepath{stroke,fill}%
}%
\begin{pgfscope}%
\pgfsys@transformshift{0.609375in}{2.350808in}%
\pgfsys@useobject{currentmarker}{}%
\end{pgfscope}%
\end{pgfscope}%
\begin{pgfscope}%
\pgftext[x=0.394096in,y=2.312546in,left,base]{\rmfamily\fontsize{8.000000}{9.600000}\selectfont \(\displaystyle 14\)}%
\end{pgfscope}%
\begin{pgfscope}%
\pgfsetbuttcap%
\pgfsetroundjoin%
\definecolor{currentfill}{rgb}{0.000000,0.000000,0.000000}%
\pgfsetfillcolor{currentfill}%
\pgfsetlinewidth{0.803000pt}%
\definecolor{currentstroke}{rgb}{0.000000,0.000000,0.000000}%
\pgfsetstrokecolor{currentstroke}%
\pgfsetdash{}{0pt}%
\pgfsys@defobject{currentmarker}{\pgfqpoint{-0.048611in}{0.000000in}}{\pgfqpoint{0.000000in}{0.000000in}}{%
\pgfpathmoveto{\pgfqpoint{0.000000in}{0.000000in}}%
\pgfpathlineto{\pgfqpoint{-0.048611in}{0.000000in}}%
\pgfusepath{stroke,fill}%
}%
\begin{pgfscope}%
\pgfsys@transformshift{0.609375in}{2.630953in}%
\pgfsys@useobject{currentmarker}{}%
\end{pgfscope}%
\end{pgfscope}%
\begin{pgfscope}%
\pgftext[x=0.394096in,y=2.592690in,left,base]{\rmfamily\fontsize{8.000000}{9.600000}\selectfont \(\displaystyle 16\)}%
\end{pgfscope}%
\begin{pgfscope}%
\pgftext[x=0.338540in,y=1.513990in,,bottom,rotate=90.000000]{\rmfamily\fontsize{10.000000}{12.000000}\selectfont Regret}%
\end{pgfscope}%
\begin{pgfscope}%
\pgfpathrectangle{\pgfqpoint{0.609375in}{0.376614in}}{\pgfqpoint{3.778125in}{2.274751in}} %
\pgfusepath{clip}%
\pgfsetrectcap%
\pgfsetroundjoin%
\pgfsetlinewidth{2.007500pt}%
\definecolor{currentstroke}{rgb}{0.121569,0.466667,0.705882}%
\pgfsetstrokecolor{currentstroke}%
\pgfsetdash{}{0pt}%
\pgfpathmoveto{\pgfqpoint{0.781108in}{0.481787in}}%
\pgfpathlineto{\pgfqpoint{0.794902in}{0.557249in}}%
\pgfpathlineto{\pgfqpoint{0.808696in}{0.617211in}}%
\pgfpathlineto{\pgfqpoint{0.836283in}{0.721819in}}%
\pgfpathlineto{\pgfqpoint{0.863871in}{0.804763in}}%
\pgfpathlineto{\pgfqpoint{0.891458in}{0.876875in}}%
\pgfpathlineto{\pgfqpoint{0.919046in}{0.938435in}}%
\pgfpathlineto{\pgfqpoint{0.974221in}{1.054737in}}%
\pgfpathlineto{\pgfqpoint{1.015603in}{1.123404in}}%
\pgfpathlineto{\pgfqpoint{1.043190in}{1.152094in}}%
\pgfpathlineto{\pgfqpoint{1.070778in}{1.184518in}}%
\pgfpathlineto{\pgfqpoint{1.098366in}{1.218624in}}%
\pgfpathlineto{\pgfqpoint{1.112159in}{1.230774in}}%
\pgfpathlineto{\pgfqpoint{1.153541in}{1.277590in}}%
\pgfpathlineto{\pgfqpoint{1.181128in}{1.311882in}}%
\pgfpathlineto{\pgfqpoint{1.194922in}{1.325433in}}%
\pgfpathlineto{\pgfqpoint{1.208716in}{1.341599in}}%
\pgfpathlineto{\pgfqpoint{1.222510in}{1.353655in}}%
\pgfpathlineto{\pgfqpoint{1.236304in}{1.362911in}}%
\pgfpathlineto{\pgfqpoint{1.277685in}{1.404310in}}%
\pgfpathlineto{\pgfqpoint{1.305273in}{1.425809in}}%
\pgfpathlineto{\pgfqpoint{1.319067in}{1.439454in}}%
\pgfpathlineto{\pgfqpoint{1.332860in}{1.447869in}}%
\pgfpathlineto{\pgfqpoint{1.374242in}{1.483012in}}%
\pgfpathlineto{\pgfqpoint{1.388036in}{1.490306in}}%
\pgfpathlineto{\pgfqpoint{1.401829in}{1.500309in}}%
\pgfpathlineto{\pgfqpoint{1.429417in}{1.529278in}}%
\pgfpathlineto{\pgfqpoint{1.457005in}{1.549937in}}%
\pgfpathlineto{\pgfqpoint{1.470799in}{1.557138in}}%
\pgfpathlineto{\pgfqpoint{1.484592in}{1.568821in}}%
\pgfpathlineto{\pgfqpoint{1.512180in}{1.584250in}}%
\pgfpathlineto{\pgfqpoint{1.567355in}{1.626221in}}%
\pgfpathlineto{\pgfqpoint{1.594943in}{1.636514in}}%
\pgfpathlineto{\pgfqpoint{1.608737in}{1.649598in}}%
\pgfpathlineto{\pgfqpoint{1.622530in}{1.657172in}}%
\pgfpathlineto{\pgfqpoint{1.636324in}{1.661478in}}%
\pgfpathlineto{\pgfqpoint{1.650118in}{1.672788in}}%
\pgfpathlineto{\pgfqpoint{1.663912in}{1.680269in}}%
\pgfpathlineto{\pgfqpoint{1.677706in}{1.681587in}}%
\pgfpathlineto{\pgfqpoint{1.691500in}{1.690843in}}%
\pgfpathlineto{\pgfqpoint{1.705293in}{1.697950in}}%
\pgfpathlineto{\pgfqpoint{1.719087in}{1.707299in}}%
\pgfpathlineto{\pgfqpoint{1.732881in}{1.711512in}}%
\pgfpathlineto{\pgfqpoint{1.746675in}{1.722915in}}%
\pgfpathlineto{\pgfqpoint{1.788056in}{1.749467in}}%
\pgfpathlineto{\pgfqpoint{1.815644in}{1.766390in}}%
\pgfpathlineto{\pgfqpoint{1.829438in}{1.773124in}}%
\pgfpathlineto{\pgfqpoint{1.843231in}{1.776310in}}%
\pgfpathlineto{\pgfqpoint{1.870819in}{1.789965in}}%
\pgfpathlineto{\pgfqpoint{1.884613in}{1.796232in}}%
\pgfpathlineto{\pgfqpoint{1.912201in}{1.815490in}}%
\pgfpathlineto{\pgfqpoint{1.925994in}{1.819235in}}%
\pgfpathlineto{\pgfqpoint{1.939788in}{1.827464in}}%
\pgfpathlineto{\pgfqpoint{1.953582in}{1.831116in}}%
\pgfpathlineto{\pgfqpoint{1.994963in}{1.849264in}}%
\pgfpathlineto{\pgfqpoint{2.008757in}{1.853850in}}%
\pgfpathlineto{\pgfqpoint{2.050139in}{1.874893in}}%
\pgfpathlineto{\pgfqpoint{2.063932in}{1.876771in}}%
\pgfpathlineto{\pgfqpoint{2.077726in}{1.883225in}}%
\pgfpathlineto{\pgfqpoint{2.091520in}{1.886597in}}%
\pgfpathlineto{\pgfqpoint{2.105314in}{1.895199in}}%
\pgfpathlineto{\pgfqpoint{2.119108in}{1.898758in}}%
\pgfpathlineto{\pgfqpoint{2.132901in}{1.906986in}}%
\pgfpathlineto{\pgfqpoint{2.146695in}{1.912600in}}%
\pgfpathlineto{\pgfqpoint{2.160489in}{1.915505in}}%
\pgfpathlineto{\pgfqpoint{2.174283in}{1.928869in}}%
\pgfpathlineto{\pgfqpoint{2.201871in}{1.941870in}}%
\pgfpathlineto{\pgfqpoint{2.215664in}{1.946457in}}%
\pgfpathlineto{\pgfqpoint{2.243252in}{1.961232in}}%
\pgfpathlineto{\pgfqpoint{2.270840in}{1.966483in}}%
\pgfpathlineto{\pgfqpoint{2.284633in}{1.974337in}}%
\pgfpathlineto{\pgfqpoint{2.298427in}{1.979017in}}%
\pgfpathlineto{\pgfqpoint{2.312221in}{1.987432in}}%
\pgfpathlineto{\pgfqpoint{2.339809in}{1.999126in}}%
\pgfpathlineto{\pgfqpoint{2.353602in}{2.003432in}}%
\pgfpathlineto{\pgfqpoint{2.367396in}{2.010633in}}%
\pgfpathlineto{\pgfqpoint{2.422572in}{2.030659in}}%
\pgfpathlineto{\pgfqpoint{2.436365in}{2.036459in}}%
\pgfpathlineto{\pgfqpoint{2.450159in}{2.040298in}}%
\pgfpathlineto{\pgfqpoint{2.463953in}{2.040589in}}%
\pgfpathlineto{\pgfqpoint{2.491541in}{2.047707in}}%
\pgfpathlineto{\pgfqpoint{2.505334in}{2.052387in}}%
\pgfpathlineto{\pgfqpoint{2.519128in}{2.055292in}}%
\pgfpathlineto{\pgfqpoint{2.546716in}{2.069227in}}%
\pgfpathlineto{\pgfqpoint{2.560510in}{2.069798in}}%
\pgfpathlineto{\pgfqpoint{2.574303in}{2.076159in}}%
\pgfpathlineto{\pgfqpoint{2.588097in}{2.078317in}}%
\pgfpathlineto{\pgfqpoint{2.601891in}{2.088319in}}%
\pgfpathlineto{\pgfqpoint{2.615685in}{2.091038in}}%
\pgfpathlineto{\pgfqpoint{2.629479in}{2.103188in}}%
\pgfpathlineto{\pgfqpoint{2.643273in}{2.105720in}}%
\pgfpathlineto{\pgfqpoint{2.657066in}{2.109840in}}%
\pgfpathlineto{\pgfqpoint{2.670860in}{2.111344in}}%
\pgfpathlineto{\pgfqpoint{2.684654in}{2.117892in}}%
\pgfpathlineto{\pgfqpoint{2.698448in}{2.128081in}}%
\pgfpathlineto{\pgfqpoint{2.712242in}{2.133974in}}%
\pgfpathlineto{\pgfqpoint{2.739829in}{2.139131in}}%
\pgfpathlineto{\pgfqpoint{2.753623in}{2.143998in}}%
\pgfpathlineto{\pgfqpoint{2.767417in}{2.153440in}}%
\pgfpathlineto{\pgfqpoint{2.781211in}{2.157466in}}%
\pgfpathlineto{\pgfqpoint{2.795004in}{2.159718in}}%
\pgfpathlineto{\pgfqpoint{2.808798in}{2.167759in}}%
\pgfpathlineto{\pgfqpoint{2.836386in}{2.179360in}}%
\pgfpathlineto{\pgfqpoint{2.850180in}{2.180211in}}%
\pgfpathlineto{\pgfqpoint{2.863974in}{2.186198in}}%
\pgfpathlineto{\pgfqpoint{2.877767in}{2.188636in}}%
\pgfpathlineto{\pgfqpoint{2.891561in}{2.196584in}}%
\pgfpathlineto{\pgfqpoint{2.905355in}{2.199209in}}%
\pgfpathlineto{\pgfqpoint{2.919149in}{2.209305in}}%
\pgfpathlineto{\pgfqpoint{2.946736in}{2.216610in}}%
\pgfpathlineto{\pgfqpoint{2.960530in}{2.226706in}}%
\pgfpathlineto{\pgfqpoint{2.974324in}{2.232319in}}%
\pgfpathlineto{\pgfqpoint{2.988118in}{2.235692in}}%
\pgfpathlineto{\pgfqpoint{3.001912in}{2.235702in}}%
\pgfpathlineto{\pgfqpoint{3.015705in}{2.242717in}}%
\pgfpathlineto{\pgfqpoint{3.029499in}{2.246929in}}%
\pgfpathlineto{\pgfqpoint{3.043293in}{2.246380in}}%
\pgfpathlineto{\pgfqpoint{3.057087in}{2.252834in}}%
\pgfpathlineto{\pgfqpoint{3.070881in}{2.251350in}}%
\pgfpathlineto{\pgfqpoint{3.098468in}{2.261923in}}%
\pgfpathlineto{\pgfqpoint{3.112262in}{2.268284in}}%
\pgfpathlineto{\pgfqpoint{3.139850in}{2.275869in}}%
\pgfpathlineto{\pgfqpoint{3.153644in}{2.278961in}}%
\pgfpathlineto{\pgfqpoint{3.167437in}{2.280372in}}%
\pgfpathlineto{\pgfqpoint{3.195025in}{2.287771in}}%
\pgfpathlineto{\pgfqpoint{3.263994in}{2.303792in}}%
\pgfpathlineto{\pgfqpoint{3.277788in}{2.306417in}}%
\pgfpathlineto{\pgfqpoint{3.305375in}{2.309053in}}%
\pgfpathlineto{\pgfqpoint{3.346757in}{2.318330in}}%
\pgfpathlineto{\pgfqpoint{3.360551in}{2.319554in}}%
\pgfpathlineto{\pgfqpoint{3.374345in}{2.324981in}}%
\pgfpathlineto{\pgfqpoint{3.388138in}{2.322750in}}%
\pgfpathlineto{\pgfqpoint{3.415726in}{2.329028in}}%
\pgfpathlineto{\pgfqpoint{3.429520in}{2.326330in}}%
\pgfpathlineto{\pgfqpoint{3.443314in}{2.334185in}}%
\pgfpathlineto{\pgfqpoint{3.457107in}{2.338398in}}%
\pgfpathlineto{\pgfqpoint{3.484695in}{2.353453in}}%
\pgfpathlineto{\pgfqpoint{3.512283in}{2.358891in}}%
\pgfpathlineto{\pgfqpoint{3.526076in}{2.361423in}}%
\pgfpathlineto{\pgfqpoint{3.539870in}{2.368343in}}%
\pgfpathlineto{\pgfqpoint{3.553664in}{2.371809in}}%
\pgfpathlineto{\pgfqpoint{3.567458in}{2.369952in}}%
\pgfpathlineto{\pgfqpoint{3.581252in}{2.370990in}}%
\pgfpathlineto{\pgfqpoint{3.608839in}{2.377081in}}%
\pgfpathlineto{\pgfqpoint{3.622633in}{2.384282in}}%
\pgfpathlineto{\pgfqpoint{3.636427in}{2.388121in}}%
\pgfpathlineto{\pgfqpoint{3.650221in}{2.389906in}}%
\pgfpathlineto{\pgfqpoint{3.664015in}{2.388049in}}%
\pgfpathlineto{\pgfqpoint{3.677808in}{2.392915in}}%
\pgfpathlineto{\pgfqpoint{3.691602in}{2.392833in}}%
\pgfpathlineto{\pgfqpoint{3.705396in}{2.395645in}}%
\pgfpathlineto{\pgfqpoint{3.719190in}{2.402005in}}%
\pgfpathlineto{\pgfqpoint{3.732984in}{2.404070in}}%
\pgfpathlineto{\pgfqpoint{3.746777in}{2.403427in}}%
\pgfpathlineto{\pgfqpoint{3.760571in}{2.406706in}}%
\pgfpathlineto{\pgfqpoint{3.774365in}{2.412319in}}%
\pgfpathlineto{\pgfqpoint{3.788159in}{2.420081in}}%
\pgfpathlineto{\pgfqpoint{3.829540in}{2.428050in}}%
\pgfpathlineto{\pgfqpoint{3.843334in}{2.428714in}}%
\pgfpathlineto{\pgfqpoint{3.857128in}{2.432647in}}%
\pgfpathlineto{\pgfqpoint{3.884716in}{2.437337in}}%
\pgfpathlineto{\pgfqpoint{3.898509in}{2.436414in}}%
\pgfpathlineto{\pgfqpoint{3.912303in}{2.443801in}}%
\pgfpathlineto{\pgfqpoint{3.926097in}{2.439890in}}%
\pgfpathlineto{\pgfqpoint{3.939891in}{2.445877in}}%
\pgfpathlineto{\pgfqpoint{3.953685in}{2.442993in}}%
\pgfpathlineto{\pgfqpoint{3.967478in}{2.444404in}}%
\pgfpathlineto{\pgfqpoint{3.995066in}{2.454510in}}%
\pgfpathlineto{\pgfqpoint{4.008860in}{2.457976in}}%
\pgfpathlineto{\pgfqpoint{4.022654in}{2.456586in}}%
\pgfpathlineto{\pgfqpoint{4.036447in}{2.461546in}}%
\pgfpathlineto{\pgfqpoint{4.050241in}{2.471081in}}%
\pgfpathlineto{\pgfqpoint{4.091623in}{2.486895in}}%
\pgfpathlineto{\pgfqpoint{4.105417in}{2.493722in}}%
\pgfpathlineto{\pgfqpoint{4.133004in}{2.502054in}}%
\pgfpathlineto{\pgfqpoint{4.160592in}{2.503569in}}%
\pgfpathlineto{\pgfqpoint{4.174386in}{2.506755in}}%
\pgfpathlineto{\pgfqpoint{4.188179in}{2.511902in}}%
\pgfpathlineto{\pgfqpoint{4.201973in}{2.514900in}}%
\pgfpathlineto{\pgfqpoint{4.215767in}{2.516032in}}%
\pgfpathlineto{\pgfqpoint{4.215767in}{2.516032in}}%
\pgfusepath{stroke}%
\end{pgfscope}%
\begin{pgfscope}%
\pgfpathrectangle{\pgfqpoint{0.609375in}{0.376614in}}{\pgfqpoint{3.778125in}{2.274751in}} %
\pgfusepath{clip}%
\pgfsetrectcap%
\pgfsetroundjoin%
\pgfsetlinewidth{2.007500pt}%
\definecolor{currentstroke}{rgb}{1.000000,0.498039,0.054902}%
\pgfsetstrokecolor{currentstroke}%
\pgfsetdash{}{0pt}%
\pgfpathmoveto{\pgfqpoint{0.781108in}{0.480012in}}%
\pgfpathlineto{\pgfqpoint{0.794902in}{0.533437in}}%
\pgfpathlineto{\pgfqpoint{0.822489in}{0.616661in}}%
\pgfpathlineto{\pgfqpoint{0.850077in}{0.679341in}}%
\pgfpathlineto{\pgfqpoint{0.863871in}{0.707180in}}%
\pgfpathlineto{\pgfqpoint{0.877665in}{0.729042in}}%
\pgfpathlineto{\pgfqpoint{0.891458in}{0.756600in}}%
\pgfpathlineto{\pgfqpoint{0.932840in}{0.825921in}}%
\pgfpathlineto{\pgfqpoint{0.974221in}{0.884409in}}%
\pgfpathlineto{\pgfqpoint{0.988015in}{0.898334in}}%
\pgfpathlineto{\pgfqpoint{1.015603in}{0.931038in}}%
\pgfpathlineto{\pgfqpoint{1.029397in}{0.942722in}}%
\pgfpathlineto{\pgfqpoint{1.070778in}{0.969834in}}%
\pgfpathlineto{\pgfqpoint{1.098366in}{0.996562in}}%
\pgfpathlineto{\pgfqpoint{1.112159in}{1.006752in}}%
\pgfpathlineto{\pgfqpoint{1.139747in}{1.033106in}}%
\pgfpathlineto{\pgfqpoint{1.153541in}{1.042642in}}%
\pgfpathlineto{\pgfqpoint{1.181128in}{1.069090in}}%
\pgfpathlineto{\pgfqpoint{1.194922in}{1.079933in}}%
\pgfpathlineto{\pgfqpoint{1.208716in}{1.096005in}}%
\pgfpathlineto{\pgfqpoint{1.222510in}{1.107221in}}%
\pgfpathlineto{\pgfqpoint{1.236304in}{1.115543in}}%
\pgfpathlineto{\pgfqpoint{1.250098in}{1.128440in}}%
\pgfpathlineto{\pgfqpoint{1.291479in}{1.159008in}}%
\pgfpathlineto{\pgfqpoint{1.360448in}{1.212662in}}%
\pgfpathlineto{\pgfqpoint{1.374242in}{1.224718in}}%
\pgfpathlineto{\pgfqpoint{1.401829in}{1.239214in}}%
\pgfpathlineto{\pgfqpoint{1.429417in}{1.261180in}}%
\pgfpathlineto{\pgfqpoint{1.457005in}{1.280904in}}%
\pgfpathlineto{\pgfqpoint{1.470799in}{1.288385in}}%
\pgfpathlineto{\pgfqpoint{1.498386in}{1.309044in}}%
\pgfpathlineto{\pgfqpoint{1.512180in}{1.317272in}}%
\pgfpathlineto{\pgfqpoint{1.539768in}{1.344187in}}%
\pgfpathlineto{\pgfqpoint{1.581149in}{1.372607in}}%
\pgfpathlineto{\pgfqpoint{1.594943in}{1.380275in}}%
\pgfpathlineto{\pgfqpoint{1.608737in}{1.392238in}}%
\pgfpathlineto{\pgfqpoint{1.622530in}{1.401774in}}%
\pgfpathlineto{\pgfqpoint{1.636324in}{1.407947in}}%
\pgfpathlineto{\pgfqpoint{1.663912in}{1.426645in}}%
\pgfpathlineto{\pgfqpoint{1.677706in}{1.430297in}}%
\pgfpathlineto{\pgfqpoint{1.691500in}{1.438059in}}%
\pgfpathlineto{\pgfqpoint{1.705293in}{1.442925in}}%
\pgfpathlineto{\pgfqpoint{1.719087in}{1.451060in}}%
\pgfpathlineto{\pgfqpoint{1.732881in}{1.456206in}}%
\pgfpathlineto{\pgfqpoint{1.746675in}{1.466956in}}%
\pgfpathlineto{\pgfqpoint{1.760469in}{1.479666in}}%
\pgfpathlineto{\pgfqpoint{1.774262in}{1.489482in}}%
\pgfpathlineto{\pgfqpoint{1.788056in}{1.494442in}}%
\pgfpathlineto{\pgfqpoint{1.801850in}{1.505005in}}%
\pgfpathlineto{\pgfqpoint{1.829438in}{1.521461in}}%
\pgfpathlineto{\pgfqpoint{1.843231in}{1.527821in}}%
\pgfpathlineto{\pgfqpoint{1.857025in}{1.537917in}}%
\pgfpathlineto{\pgfqpoint{1.870819in}{1.543531in}}%
\pgfpathlineto{\pgfqpoint{1.898407in}{1.561855in}}%
\pgfpathlineto{\pgfqpoint{1.912201in}{1.571577in}}%
\pgfpathlineto{\pgfqpoint{1.925994in}{1.574296in}}%
\pgfpathlineto{\pgfqpoint{1.953582in}{1.589631in}}%
\pgfpathlineto{\pgfqpoint{1.994963in}{1.612822in}}%
\pgfpathlineto{\pgfqpoint{2.050139in}{1.641065in}}%
\pgfpathlineto{\pgfqpoint{2.063932in}{1.644344in}}%
\pgfpathlineto{\pgfqpoint{2.077726in}{1.650518in}}%
\pgfpathlineto{\pgfqpoint{2.091520in}{1.658839in}}%
\pgfpathlineto{\pgfqpoint{2.119108in}{1.670907in}}%
\pgfpathlineto{\pgfqpoint{2.132901in}{1.681843in}}%
\pgfpathlineto{\pgfqpoint{2.160489in}{1.693163in}}%
\pgfpathlineto{\pgfqpoint{2.174283in}{1.705687in}}%
\pgfpathlineto{\pgfqpoint{2.243252in}{1.746081in}}%
\pgfpathlineto{\pgfqpoint{2.257046in}{1.751414in}}%
\pgfpathlineto{\pgfqpoint{2.270840in}{1.753012in}}%
\pgfpathlineto{\pgfqpoint{2.284633in}{1.765163in}}%
\pgfpathlineto{\pgfqpoint{2.312221in}{1.780498in}}%
\pgfpathlineto{\pgfqpoint{2.326015in}{1.788913in}}%
\pgfpathlineto{\pgfqpoint{2.353602in}{1.798646in}}%
\pgfpathlineto{\pgfqpoint{2.367396in}{1.808648in}}%
\pgfpathlineto{\pgfqpoint{2.394984in}{1.817821in}}%
\pgfpathlineto{\pgfqpoint{2.422572in}{1.832970in}}%
\pgfpathlineto{\pgfqpoint{2.450159in}{1.844944in}}%
\pgfpathlineto{\pgfqpoint{2.463953in}{1.848130in}}%
\pgfpathlineto{\pgfqpoint{2.546716in}{1.882091in}}%
\pgfpathlineto{\pgfqpoint{2.560510in}{1.884996in}}%
\pgfpathlineto{\pgfqpoint{2.574303in}{1.893784in}}%
\pgfpathlineto{\pgfqpoint{2.588097in}{1.899024in}}%
\pgfpathlineto{\pgfqpoint{2.601891in}{1.909494in}}%
\pgfpathlineto{\pgfqpoint{2.615685in}{1.913240in}}%
\pgfpathlineto{\pgfqpoint{2.629479in}{1.924549in}}%
\pgfpathlineto{\pgfqpoint{2.643273in}{1.927642in}}%
\pgfpathlineto{\pgfqpoint{2.684654in}{1.946069in}}%
\pgfpathlineto{\pgfqpoint{2.698448in}{1.955512in}}%
\pgfpathlineto{\pgfqpoint{2.712242in}{1.962713in}}%
\pgfpathlineto{\pgfqpoint{2.726035in}{1.966832in}}%
\pgfpathlineto{\pgfqpoint{2.753623in}{1.980393in}}%
\pgfpathlineto{\pgfqpoint{2.767417in}{1.992170in}}%
\pgfpathlineto{\pgfqpoint{2.795004in}{1.998261in}}%
\pgfpathlineto{\pgfqpoint{2.822592in}{2.016211in}}%
\pgfpathlineto{\pgfqpoint{2.836386in}{2.023693in}}%
\pgfpathlineto{\pgfqpoint{2.850180in}{2.026318in}}%
\pgfpathlineto{\pgfqpoint{2.905355in}{2.047278in}}%
\pgfpathlineto{\pgfqpoint{2.919149in}{2.057467in}}%
\pgfpathlineto{\pgfqpoint{2.946736in}{2.071028in}}%
\pgfpathlineto{\pgfqpoint{2.960530in}{2.084486in}}%
\pgfpathlineto{\pgfqpoint{3.001912in}{2.098431in}}%
\pgfpathlineto{\pgfqpoint{3.015705in}{2.107780in}}%
\pgfpathlineto{\pgfqpoint{3.029499in}{2.113674in}}%
\pgfpathlineto{\pgfqpoint{3.043293in}{2.114898in}}%
\pgfpathlineto{\pgfqpoint{3.057087in}{2.123033in}}%
\pgfpathlineto{\pgfqpoint{3.070881in}{2.124071in}}%
\pgfpathlineto{\pgfqpoint{3.084674in}{2.129031in}}%
\pgfpathlineto{\pgfqpoint{3.112262in}{2.146888in}}%
\pgfpathlineto{\pgfqpoint{3.139850in}{2.156060in}}%
\pgfpathlineto{\pgfqpoint{3.167437in}{2.163926in}}%
\pgfpathlineto{\pgfqpoint{3.181231in}{2.172901in}}%
\pgfpathlineto{\pgfqpoint{3.208819in}{2.179832in}}%
\pgfpathlineto{\pgfqpoint{3.222613in}{2.187220in}}%
\pgfpathlineto{\pgfqpoint{3.236406in}{2.191806in}}%
\pgfpathlineto{\pgfqpoint{3.250200in}{2.192470in}}%
\pgfpathlineto{\pgfqpoint{3.277788in}{2.201923in}}%
\pgfpathlineto{\pgfqpoint{3.305375in}{2.207454in}}%
\pgfpathlineto{\pgfqpoint{3.319169in}{2.211386in}}%
\pgfpathlineto{\pgfqpoint{3.332963in}{2.213451in}}%
\pgfpathlineto{\pgfqpoint{3.346757in}{2.220652in}}%
\pgfpathlineto{\pgfqpoint{3.401932in}{2.235823in}}%
\pgfpathlineto{\pgfqpoint{3.415726in}{2.241716in}}%
\pgfpathlineto{\pgfqpoint{3.429520in}{2.242474in}}%
\pgfpathlineto{\pgfqpoint{3.443314in}{2.252290in}}%
\pgfpathlineto{\pgfqpoint{3.457107in}{2.257996in}}%
\pgfpathlineto{\pgfqpoint{3.470901in}{2.265758in}}%
\pgfpathlineto{\pgfqpoint{3.498489in}{2.276424in}}%
\pgfpathlineto{\pgfqpoint{3.512283in}{2.281291in}}%
\pgfpathlineto{\pgfqpoint{3.526076in}{2.283916in}}%
\pgfpathlineto{\pgfqpoint{3.553664in}{2.296544in}}%
\pgfpathlineto{\pgfqpoint{3.567458in}{2.298328in}}%
\pgfpathlineto{\pgfqpoint{3.595046in}{2.305633in}}%
\pgfpathlineto{\pgfqpoint{3.608839in}{2.312181in}}%
\pgfpathlineto{\pgfqpoint{3.622633in}{2.321063in}}%
\pgfpathlineto{\pgfqpoint{3.636427in}{2.325462in}}%
\pgfpathlineto{\pgfqpoint{3.664015in}{2.328471in}}%
\pgfpathlineto{\pgfqpoint{3.677808in}{2.334458in}}%
\pgfpathlineto{\pgfqpoint{3.691602in}{2.336337in}}%
\pgfpathlineto{\pgfqpoint{3.705396in}{2.342697in}}%
\pgfpathlineto{\pgfqpoint{3.732984in}{2.360554in}}%
\pgfpathlineto{\pgfqpoint{3.760571in}{2.367953in}}%
\pgfpathlineto{\pgfqpoint{3.774365in}{2.374873in}}%
\pgfpathlineto{\pgfqpoint{3.788159in}{2.383662in}}%
\pgfpathlineto{\pgfqpoint{3.815747in}{2.395729in}}%
\pgfpathlineto{\pgfqpoint{3.829540in}{2.398168in}}%
\pgfpathlineto{\pgfqpoint{3.857128in}{2.410795in}}%
\pgfpathlineto{\pgfqpoint{3.870922in}{2.410059in}}%
\pgfpathlineto{\pgfqpoint{3.884716in}{2.416233in}}%
\pgfpathlineto{\pgfqpoint{3.898509in}{2.416523in}}%
\pgfpathlineto{\pgfqpoint{3.912303in}{2.426713in}}%
\pgfpathlineto{\pgfqpoint{3.926097in}{2.427097in}}%
\pgfpathlineto{\pgfqpoint{3.939891in}{2.434298in}}%
\pgfpathlineto{\pgfqpoint{3.953685in}{2.434495in}}%
\pgfpathlineto{\pgfqpoint{3.967478in}{2.438801in}}%
\pgfpathlineto{\pgfqpoint{3.981272in}{2.447870in}}%
\pgfpathlineto{\pgfqpoint{3.995066in}{2.451055in}}%
\pgfpathlineto{\pgfqpoint{4.008860in}{2.458910in}}%
\pgfpathlineto{\pgfqpoint{4.022654in}{2.460882in}}%
\pgfpathlineto{\pgfqpoint{4.036447in}{2.467709in}}%
\pgfpathlineto{\pgfqpoint{4.050241in}{2.477525in}}%
\pgfpathlineto{\pgfqpoint{4.091623in}{2.498381in}}%
\pgfpathlineto{\pgfqpoint{4.133004in}{2.519330in}}%
\pgfpathlineto{\pgfqpoint{4.174386in}{2.532902in}}%
\pgfpathlineto{\pgfqpoint{4.188179in}{2.539729in}}%
\pgfpathlineto{\pgfqpoint{4.215767in}{2.547968in}}%
\pgfpathlineto{\pgfqpoint{4.215767in}{2.547968in}}%
\pgfusepath{stroke}%
\end{pgfscope}%
\begin{pgfscope}%
\pgfpathrectangle{\pgfqpoint{0.609375in}{0.376614in}}{\pgfqpoint{3.778125in}{2.274751in}} %
\pgfusepath{clip}%
\pgfsetrectcap%
\pgfsetroundjoin%
\pgfsetlinewidth{2.007500pt}%
\definecolor{currentstroke}{rgb}{0.172549,0.627451,0.172549}%
\pgfsetstrokecolor{currentstroke}%
\pgfsetdash{}{0pt}%
\pgfpathmoveto{\pgfqpoint{0.781108in}{0.484401in}}%
\pgfpathlineto{\pgfqpoint{0.794902in}{0.540721in}}%
\pgfpathlineto{\pgfqpoint{0.822489in}{0.617408in}}%
\pgfpathlineto{\pgfqpoint{0.850077in}{0.676633in}}%
\pgfpathlineto{\pgfqpoint{0.863871in}{0.703538in}}%
\pgfpathlineto{\pgfqpoint{0.932840in}{0.811540in}}%
\pgfpathlineto{\pgfqpoint{0.960428in}{0.851622in}}%
\pgfpathlineto{\pgfqpoint{1.001809in}{0.897971in}}%
\pgfpathlineto{\pgfqpoint{1.015603in}{0.911335in}}%
\pgfpathlineto{\pgfqpoint{1.029397in}{0.921337in}}%
\pgfpathlineto{\pgfqpoint{1.056984in}{0.938354in}}%
\pgfpathlineto{\pgfqpoint{1.098366in}{0.969295in}}%
\pgfpathlineto{\pgfqpoint{1.112159in}{0.976496in}}%
\pgfpathlineto{\pgfqpoint{1.153541in}{1.008931in}}%
\pgfpathlineto{\pgfqpoint{1.167335in}{1.022856in}}%
\pgfpathlineto{\pgfqpoint{1.181128in}{1.034539in}}%
\pgfpathlineto{\pgfqpoint{1.194922in}{1.041273in}}%
\pgfpathlineto{\pgfqpoint{1.208716in}{1.055197in}}%
\pgfpathlineto{\pgfqpoint{1.222510in}{1.064453in}}%
\pgfpathlineto{\pgfqpoint{1.236304in}{1.070533in}}%
\pgfpathlineto{\pgfqpoint{1.263891in}{1.091378in}}%
\pgfpathlineto{\pgfqpoint{1.277685in}{1.100354in}}%
\pgfpathlineto{\pgfqpoint{1.305273in}{1.114475in}}%
\pgfpathlineto{\pgfqpoint{1.319067in}{1.124198in}}%
\pgfpathlineto{\pgfqpoint{1.332860in}{1.128597in}}%
\pgfpathlineto{\pgfqpoint{1.346654in}{1.135425in}}%
\pgfpathlineto{\pgfqpoint{1.374242in}{1.154776in}}%
\pgfpathlineto{\pgfqpoint{1.388036in}{1.161416in}}%
\pgfpathlineto{\pgfqpoint{1.401829in}{1.164509in}}%
\pgfpathlineto{\pgfqpoint{1.443211in}{1.192928in}}%
\pgfpathlineto{\pgfqpoint{1.484592in}{1.213878in}}%
\pgfpathlineto{\pgfqpoint{1.498386in}{1.222199in}}%
\pgfpathlineto{\pgfqpoint{1.512180in}{1.226412in}}%
\pgfpathlineto{\pgfqpoint{1.553561in}{1.251376in}}%
\pgfpathlineto{\pgfqpoint{1.594943in}{1.268777in}}%
\pgfpathlineto{\pgfqpoint{1.608737in}{1.278499in}}%
\pgfpathlineto{\pgfqpoint{1.663912in}{1.304408in}}%
\pgfpathlineto{\pgfqpoint{1.677706in}{1.305166in}}%
\pgfpathlineto{\pgfqpoint{1.691500in}{1.309566in}}%
\pgfpathlineto{\pgfqpoint{1.705293in}{1.311444in}}%
\pgfpathlineto{\pgfqpoint{1.719087in}{1.317898in}}%
\pgfpathlineto{\pgfqpoint{1.732881in}{1.320056in}}%
\pgfpathlineto{\pgfqpoint{1.774262in}{1.344834in}}%
\pgfpathlineto{\pgfqpoint{1.788056in}{1.349047in}}%
\pgfpathlineto{\pgfqpoint{1.815644in}{1.362421in}}%
\pgfpathlineto{\pgfqpoint{1.898407in}{1.389192in}}%
\pgfpathlineto{\pgfqpoint{1.912201in}{1.397607in}}%
\pgfpathlineto{\pgfqpoint{1.925994in}{1.400699in}}%
\pgfpathlineto{\pgfqpoint{1.953582in}{1.409965in}}%
\pgfpathlineto{\pgfqpoint{1.967376in}{1.414645in}}%
\pgfpathlineto{\pgfqpoint{1.981170in}{1.421939in}}%
\pgfpathlineto{\pgfqpoint{2.036345in}{1.442245in}}%
\pgfpathlineto{\pgfqpoint{2.050139in}{1.446458in}}%
\pgfpathlineto{\pgfqpoint{2.063932in}{1.446282in}}%
\pgfpathlineto{\pgfqpoint{2.105314in}{1.460975in}}%
\pgfpathlineto{\pgfqpoint{2.119108in}{1.465654in}}%
\pgfpathlineto{\pgfqpoint{2.132901in}{1.472762in}}%
\pgfpathlineto{\pgfqpoint{2.146695in}{1.476321in}}%
\pgfpathlineto{\pgfqpoint{2.160489in}{1.477359in}}%
\pgfpathlineto{\pgfqpoint{2.174283in}{1.487548in}}%
\pgfpathlineto{\pgfqpoint{2.188077in}{1.494562in}}%
\pgfpathlineto{\pgfqpoint{2.215664in}{1.503735in}}%
\pgfpathlineto{\pgfqpoint{2.229458in}{1.512243in}}%
\pgfpathlineto{\pgfqpoint{2.243252in}{1.518137in}}%
\pgfpathlineto{\pgfqpoint{2.270840in}{1.520212in}}%
\pgfpathlineto{\pgfqpoint{2.284633in}{1.526666in}}%
\pgfpathlineto{\pgfqpoint{2.298427in}{1.529105in}}%
\pgfpathlineto{\pgfqpoint{2.339809in}{1.545665in}}%
\pgfpathlineto{\pgfqpoint{2.353602in}{1.546329in}}%
\pgfpathlineto{\pgfqpoint{2.367396in}{1.551382in}}%
\pgfpathlineto{\pgfqpoint{2.422572in}{1.564872in}}%
\pgfpathlineto{\pgfqpoint{2.450159in}{1.572737in}}%
\pgfpathlineto{\pgfqpoint{2.463953in}{1.570226in}}%
\pgfpathlineto{\pgfqpoint{2.519128in}{1.580447in}}%
\pgfpathlineto{\pgfqpoint{2.546716in}{1.591300in}}%
\pgfpathlineto{\pgfqpoint{2.560510in}{1.592058in}}%
\pgfpathlineto{\pgfqpoint{2.574303in}{1.598419in}}%
\pgfpathlineto{\pgfqpoint{2.588097in}{1.600297in}}%
\pgfpathlineto{\pgfqpoint{2.601891in}{1.607965in}}%
\pgfpathlineto{\pgfqpoint{2.615685in}{1.611244in}}%
\pgfpathlineto{\pgfqpoint{2.629479in}{1.620592in}}%
\pgfpathlineto{\pgfqpoint{2.643273in}{1.621630in}}%
\pgfpathlineto{\pgfqpoint{2.670860in}{1.629589in}}%
\pgfpathlineto{\pgfqpoint{2.684654in}{1.635296in}}%
\pgfpathlineto{\pgfqpoint{2.698448in}{1.643711in}}%
\pgfpathlineto{\pgfqpoint{2.726035in}{1.651669in}}%
\pgfpathlineto{\pgfqpoint{2.739829in}{1.653921in}}%
\pgfpathlineto{\pgfqpoint{2.753623in}{1.659534in}}%
\pgfpathlineto{\pgfqpoint{2.767417in}{1.670097in}}%
\pgfpathlineto{\pgfqpoint{2.781211in}{1.672349in}}%
\pgfpathlineto{\pgfqpoint{2.795004in}{1.672173in}}%
\pgfpathlineto{\pgfqpoint{2.808798in}{1.678907in}}%
\pgfpathlineto{\pgfqpoint{2.822592in}{1.682746in}}%
\pgfpathlineto{\pgfqpoint{2.836386in}{1.689573in}}%
\pgfpathlineto{\pgfqpoint{2.850180in}{1.690798in}}%
\pgfpathlineto{\pgfqpoint{2.863974in}{1.694731in}}%
\pgfpathlineto{\pgfqpoint{2.877767in}{1.695675in}}%
\pgfpathlineto{\pgfqpoint{2.891561in}{1.700821in}}%
\pgfpathlineto{\pgfqpoint{2.905355in}{1.702139in}}%
\pgfpathlineto{\pgfqpoint{2.919149in}{1.712329in}}%
\pgfpathlineto{\pgfqpoint{2.932943in}{1.714954in}}%
\pgfpathlineto{\pgfqpoint{2.946736in}{1.719540in}}%
\pgfpathlineto{\pgfqpoint{2.960530in}{1.729076in}}%
\pgfpathlineto{\pgfqpoint{2.988118in}{1.734793in}}%
\pgfpathlineto{\pgfqpoint{3.001912in}{1.735271in}}%
\pgfpathlineto{\pgfqpoint{3.015705in}{1.741725in}}%
\pgfpathlineto{\pgfqpoint{3.029499in}{1.744817in}}%
\pgfpathlineto{\pgfqpoint{3.043293in}{1.743893in}}%
\pgfpathlineto{\pgfqpoint{3.057087in}{1.751094in}}%
\pgfpathlineto{\pgfqpoint{3.070881in}{1.750265in}}%
\pgfpathlineto{\pgfqpoint{3.084674in}{1.752236in}}%
\pgfpathlineto{\pgfqpoint{3.112262in}{1.763183in}}%
\pgfpathlineto{\pgfqpoint{3.126056in}{1.764688in}}%
\pgfpathlineto{\pgfqpoint{3.153644in}{1.771246in}}%
\pgfpathlineto{\pgfqpoint{3.167437in}{1.771536in}}%
\pgfpathlineto{\pgfqpoint{3.181231in}{1.777056in}}%
\pgfpathlineto{\pgfqpoint{3.208819in}{1.779039in}}%
\pgfpathlineto{\pgfqpoint{3.236406in}{1.785503in}}%
\pgfpathlineto{\pgfqpoint{3.250200in}{1.784486in}}%
\pgfpathlineto{\pgfqpoint{3.277788in}{1.788150in}}%
\pgfpathlineto{\pgfqpoint{3.346757in}{1.794739in}}%
\pgfpathlineto{\pgfqpoint{3.360551in}{1.795217in}}%
\pgfpathlineto{\pgfqpoint{3.374345in}{1.797375in}}%
\pgfpathlineto{\pgfqpoint{3.388138in}{1.794024in}}%
\pgfpathlineto{\pgfqpoint{3.415726in}{1.799088in}}%
\pgfpathlineto{\pgfqpoint{3.429520in}{1.798631in}}%
\pgfpathlineto{\pgfqpoint{3.443314in}{1.805085in}}%
\pgfpathlineto{\pgfqpoint{3.457107in}{1.807804in}}%
\pgfpathlineto{\pgfqpoint{3.470901in}{1.815845in}}%
\pgfpathlineto{\pgfqpoint{3.484695in}{1.820338in}}%
\pgfpathlineto{\pgfqpoint{3.512283in}{1.824935in}}%
\pgfpathlineto{\pgfqpoint{3.526076in}{1.826720in}}%
\pgfpathlineto{\pgfqpoint{3.539870in}{1.832520in}}%
\pgfpathlineto{\pgfqpoint{3.553664in}{1.836639in}}%
\pgfpathlineto{\pgfqpoint{3.567458in}{1.835156in}}%
\pgfpathlineto{\pgfqpoint{3.608839in}{1.839016in}}%
\pgfpathlineto{\pgfqpoint{3.622633in}{1.846031in}}%
\pgfpathlineto{\pgfqpoint{3.636427in}{1.849683in}}%
\pgfpathlineto{\pgfqpoint{3.650221in}{1.850627in}}%
\pgfpathlineto{\pgfqpoint{3.664015in}{1.849331in}}%
\pgfpathlineto{\pgfqpoint{3.677808in}{1.852610in}}%
\pgfpathlineto{\pgfqpoint{3.691602in}{1.851686in}}%
\pgfpathlineto{\pgfqpoint{3.705396in}{1.854405in}}%
\pgfpathlineto{\pgfqpoint{3.732984in}{1.865258in}}%
\pgfpathlineto{\pgfqpoint{3.760571in}{1.867521in}}%
\pgfpathlineto{\pgfqpoint{3.774365in}{1.869119in}}%
\pgfpathlineto{\pgfqpoint{3.788159in}{1.875946in}}%
\pgfpathlineto{\pgfqpoint{3.801953in}{1.877825in}}%
\pgfpathlineto{\pgfqpoint{3.815747in}{1.881384in}}%
\pgfpathlineto{\pgfqpoint{3.829540in}{1.881768in}}%
\pgfpathlineto{\pgfqpoint{3.843334in}{1.880097in}}%
\pgfpathlineto{\pgfqpoint{3.857128in}{1.885244in}}%
\pgfpathlineto{\pgfqpoint{3.870922in}{1.884974in}}%
\pgfpathlineto{\pgfqpoint{3.884716in}{1.886946in}}%
\pgfpathlineto{\pgfqpoint{3.898509in}{1.886303in}}%
\pgfpathlineto{\pgfqpoint{3.912303in}{1.893037in}}%
\pgfpathlineto{\pgfqpoint{3.926097in}{1.889219in}}%
\pgfpathlineto{\pgfqpoint{3.939891in}{1.892498in}}%
\pgfpathlineto{\pgfqpoint{3.953685in}{1.890734in}}%
\pgfpathlineto{\pgfqpoint{3.967478in}{1.893733in}}%
\pgfpathlineto{\pgfqpoint{3.981272in}{1.899440in}}%
\pgfpathlineto{\pgfqpoint{3.995066in}{1.900011in}}%
\pgfpathlineto{\pgfqpoint{4.008860in}{1.904971in}}%
\pgfpathlineto{\pgfqpoint{4.022654in}{1.901713in}}%
\pgfpathlineto{\pgfqpoint{4.036447in}{1.906299in}}%
\pgfpathlineto{\pgfqpoint{4.064035in}{1.918086in}}%
\pgfpathlineto{\pgfqpoint{4.105417in}{1.932686in}}%
\pgfpathlineto{\pgfqpoint{4.133004in}{1.939150in}}%
\pgfpathlineto{\pgfqpoint{4.160592in}{1.940105in}}%
\pgfpathlineto{\pgfqpoint{4.174386in}{1.941050in}}%
\pgfpathlineto{\pgfqpoint{4.188179in}{1.945916in}}%
\pgfpathlineto{\pgfqpoint{4.215767in}{1.949766in}}%
\pgfpathlineto{\pgfqpoint{4.215767in}{1.949766in}}%
\pgfusepath{stroke}%
\end{pgfscope}%
\begin{pgfscope}%
\pgfpathrectangle{\pgfqpoint{0.609375in}{0.376614in}}{\pgfqpoint{3.778125in}{2.274751in}} %
\pgfusepath{clip}%
\pgfsetrectcap%
\pgfsetroundjoin%
\pgfsetlinewidth{2.007500pt}%
\definecolor{currentstroke}{rgb}{0.839216,0.152941,0.156863}%
\pgfsetstrokecolor{currentstroke}%
\pgfsetdash{}{0pt}%
\pgfpathmoveto{\pgfqpoint{0.781108in}{0.484401in}}%
\pgfpathlineto{\pgfqpoint{0.794902in}{0.540721in}}%
\pgfpathlineto{\pgfqpoint{0.822489in}{0.618342in}}%
\pgfpathlineto{\pgfqpoint{0.850077in}{0.678221in}}%
\pgfpathlineto{\pgfqpoint{0.863871in}{0.705966in}}%
\pgfpathlineto{\pgfqpoint{0.919046in}{0.794067in}}%
\pgfpathlineto{\pgfqpoint{0.974221in}{0.868067in}}%
\pgfpathlineto{\pgfqpoint{1.015603in}{0.912549in}}%
\pgfpathlineto{\pgfqpoint{1.029397in}{0.922645in}}%
\pgfpathlineto{\pgfqpoint{1.084572in}{0.953596in}}%
\pgfpathlineto{\pgfqpoint{1.098366in}{0.964159in}}%
\pgfpathlineto{\pgfqpoint{1.112159in}{0.971547in}}%
\pgfpathlineto{\pgfqpoint{1.153541in}{1.006223in}}%
\pgfpathlineto{\pgfqpoint{1.167335in}{1.019494in}}%
\pgfpathlineto{\pgfqpoint{1.181128in}{1.029590in}}%
\pgfpathlineto{\pgfqpoint{1.194922in}{1.036230in}}%
\pgfpathlineto{\pgfqpoint{1.208716in}{1.049127in}}%
\pgfpathlineto{\pgfqpoint{1.222510in}{1.057169in}}%
\pgfpathlineto{\pgfqpoint{1.236304in}{1.062409in}}%
\pgfpathlineto{\pgfqpoint{1.263891in}{1.084001in}}%
\pgfpathlineto{\pgfqpoint{1.291479in}{1.098870in}}%
\pgfpathlineto{\pgfqpoint{1.305273in}{1.107845in}}%
\pgfpathlineto{\pgfqpoint{1.319067in}{1.118968in}}%
\pgfpathlineto{\pgfqpoint{1.332860in}{1.123274in}}%
\pgfpathlineto{\pgfqpoint{1.346654in}{1.129822in}}%
\pgfpathlineto{\pgfqpoint{1.374242in}{1.147118in}}%
\pgfpathlineto{\pgfqpoint{1.388036in}{1.153572in}}%
\pgfpathlineto{\pgfqpoint{1.401829in}{1.157318in}}%
\pgfpathlineto{\pgfqpoint{1.415623in}{1.166013in}}%
\pgfpathlineto{\pgfqpoint{1.429417in}{1.176669in}}%
\pgfpathlineto{\pgfqpoint{1.457005in}{1.193126in}}%
\pgfpathlineto{\pgfqpoint{1.470799in}{1.198179in}}%
\pgfpathlineto{\pgfqpoint{1.484592in}{1.210609in}}%
\pgfpathlineto{\pgfqpoint{1.512180in}{1.222023in}}%
\pgfpathlineto{\pgfqpoint{1.553561in}{1.250536in}}%
\pgfpathlineto{\pgfqpoint{1.594943in}{1.268123in}}%
\pgfpathlineto{\pgfqpoint{1.608737in}{1.278126in}}%
\pgfpathlineto{\pgfqpoint{1.622530in}{1.284860in}}%
\pgfpathlineto{\pgfqpoint{1.636324in}{1.288326in}}%
\pgfpathlineto{\pgfqpoint{1.663912in}{1.303475in}}%
\pgfpathlineto{\pgfqpoint{1.677706in}{1.304606in}}%
\pgfpathlineto{\pgfqpoint{1.691500in}{1.309846in}}%
\pgfpathlineto{\pgfqpoint{1.705293in}{1.312098in}}%
\pgfpathlineto{\pgfqpoint{1.719087in}{1.318551in}}%
\pgfpathlineto{\pgfqpoint{1.732881in}{1.320430in}}%
\pgfpathlineto{\pgfqpoint{1.760469in}{1.338660in}}%
\pgfpathlineto{\pgfqpoint{1.829438in}{1.373171in}}%
\pgfpathlineto{\pgfqpoint{1.843231in}{1.376637in}}%
\pgfpathlineto{\pgfqpoint{1.857025in}{1.382157in}}%
\pgfpathlineto{\pgfqpoint{1.870819in}{1.384502in}}%
\pgfpathlineto{\pgfqpoint{1.898407in}{1.394235in}}%
\pgfpathlineto{\pgfqpoint{1.912201in}{1.401716in}}%
\pgfpathlineto{\pgfqpoint{1.925994in}{1.404154in}}%
\pgfpathlineto{\pgfqpoint{1.967376in}{1.417540in}}%
\pgfpathlineto{\pgfqpoint{1.981170in}{1.423433in}}%
\pgfpathlineto{\pgfqpoint{2.022551in}{1.435791in}}%
\pgfpathlineto{\pgfqpoint{2.036345in}{1.443833in}}%
\pgfpathlineto{\pgfqpoint{2.050139in}{1.448232in}}%
\pgfpathlineto{\pgfqpoint{2.063932in}{1.447963in}}%
\pgfpathlineto{\pgfqpoint{2.077726in}{1.454230in}}%
\pgfpathlineto{\pgfqpoint{2.091520in}{1.457602in}}%
\pgfpathlineto{\pgfqpoint{2.160489in}{1.481467in}}%
\pgfpathlineto{\pgfqpoint{2.174283in}{1.491563in}}%
\pgfpathlineto{\pgfqpoint{2.188077in}{1.497457in}}%
\pgfpathlineto{\pgfqpoint{2.201871in}{1.501296in}}%
\pgfpathlineto{\pgfqpoint{2.215664in}{1.507003in}}%
\pgfpathlineto{\pgfqpoint{2.229458in}{1.514764in}}%
\pgfpathlineto{\pgfqpoint{2.243252in}{1.519911in}}%
\pgfpathlineto{\pgfqpoint{2.257046in}{1.522536in}}%
\pgfpathlineto{\pgfqpoint{2.270840in}{1.522080in}}%
\pgfpathlineto{\pgfqpoint{2.298427in}{1.534147in}}%
\pgfpathlineto{\pgfqpoint{2.339809in}{1.550801in}}%
\pgfpathlineto{\pgfqpoint{2.353602in}{1.553053in}}%
\pgfpathlineto{\pgfqpoint{2.408778in}{1.569623in}}%
\pgfpathlineto{\pgfqpoint{2.422572in}{1.575984in}}%
\pgfpathlineto{\pgfqpoint{2.450159in}{1.583196in}}%
\pgfpathlineto{\pgfqpoint{2.463953in}{1.581152in}}%
\pgfpathlineto{\pgfqpoint{2.477747in}{1.583870in}}%
\pgfpathlineto{\pgfqpoint{2.491541in}{1.589017in}}%
\pgfpathlineto{\pgfqpoint{2.505334in}{1.589961in}}%
\pgfpathlineto{\pgfqpoint{2.588097in}{1.614584in}}%
\pgfpathlineto{\pgfqpoint{2.601891in}{1.622532in}}%
\pgfpathlineto{\pgfqpoint{2.615685in}{1.624224in}}%
\pgfpathlineto{\pgfqpoint{2.629479in}{1.632639in}}%
\pgfpathlineto{\pgfqpoint{2.657066in}{1.638916in}}%
\pgfpathlineto{\pgfqpoint{2.684654in}{1.649396in}}%
\pgfpathlineto{\pgfqpoint{2.698448in}{1.657624in}}%
\pgfpathlineto{\pgfqpoint{2.712242in}{1.663144in}}%
\pgfpathlineto{\pgfqpoint{2.739829in}{1.668022in}}%
\pgfpathlineto{\pgfqpoint{2.753623in}{1.674662in}}%
\pgfpathlineto{\pgfqpoint{2.767417in}{1.683357in}}%
\pgfpathlineto{\pgfqpoint{2.781211in}{1.685422in}}%
\pgfpathlineto{\pgfqpoint{2.795004in}{1.685620in}}%
\pgfpathlineto{\pgfqpoint{2.808798in}{1.693474in}}%
\pgfpathlineto{\pgfqpoint{2.836386in}{1.703954in}}%
\pgfpathlineto{\pgfqpoint{2.850180in}{1.703218in}}%
\pgfpathlineto{\pgfqpoint{2.877767in}{1.707534in}}%
\pgfpathlineto{\pgfqpoint{2.905355in}{1.717080in}}%
\pgfpathlineto{\pgfqpoint{2.919149in}{1.726803in}}%
\pgfpathlineto{\pgfqpoint{2.946736in}{1.735041in}}%
\pgfpathlineto{\pgfqpoint{2.960530in}{1.746351in}}%
\pgfpathlineto{\pgfqpoint{3.001912in}{1.752359in}}%
\pgfpathlineto{\pgfqpoint{3.015705in}{1.759467in}}%
\pgfpathlineto{\pgfqpoint{3.029499in}{1.762839in}}%
\pgfpathlineto{\pgfqpoint{3.043293in}{1.762850in}}%
\pgfpathlineto{\pgfqpoint{3.057087in}{1.767156in}}%
\pgfpathlineto{\pgfqpoint{3.070881in}{1.765953in}}%
\pgfpathlineto{\pgfqpoint{3.084674in}{1.768111in}}%
\pgfpathlineto{\pgfqpoint{3.098468in}{1.772604in}}%
\pgfpathlineto{\pgfqpoint{3.112262in}{1.778964in}}%
\pgfpathlineto{\pgfqpoint{3.126056in}{1.780843in}}%
\pgfpathlineto{\pgfqpoint{3.153644in}{1.788334in}}%
\pgfpathlineto{\pgfqpoint{3.167437in}{1.789746in}}%
\pgfpathlineto{\pgfqpoint{3.181231in}{1.795079in}}%
\pgfpathlineto{\pgfqpoint{3.195025in}{1.795650in}}%
\pgfpathlineto{\pgfqpoint{3.208819in}{1.794633in}}%
\pgfpathlineto{\pgfqpoint{3.236406in}{1.801471in}}%
\pgfpathlineto{\pgfqpoint{3.250200in}{1.800828in}}%
\pgfpathlineto{\pgfqpoint{3.263994in}{1.804201in}}%
\pgfpathlineto{\pgfqpoint{3.291582in}{1.806650in}}%
\pgfpathlineto{\pgfqpoint{3.305375in}{1.806473in}}%
\pgfpathlineto{\pgfqpoint{3.319169in}{1.808819in}}%
\pgfpathlineto{\pgfqpoint{3.332963in}{1.808923in}}%
\pgfpathlineto{\pgfqpoint{3.374345in}{1.817826in}}%
\pgfpathlineto{\pgfqpoint{3.388138in}{1.815408in}}%
\pgfpathlineto{\pgfqpoint{3.401932in}{1.816726in}}%
\pgfpathlineto{\pgfqpoint{3.415726in}{1.819725in}}%
\pgfpathlineto{\pgfqpoint{3.429520in}{1.820109in}}%
\pgfpathlineto{\pgfqpoint{3.443314in}{1.828057in}}%
\pgfpathlineto{\pgfqpoint{3.457107in}{1.830402in}}%
\pgfpathlineto{\pgfqpoint{3.470901in}{1.839097in}}%
\pgfpathlineto{\pgfqpoint{3.484695in}{1.843310in}}%
\pgfpathlineto{\pgfqpoint{3.498489in}{1.843601in}}%
\pgfpathlineto{\pgfqpoint{3.553664in}{1.858584in}}%
\pgfpathlineto{\pgfqpoint{3.567458in}{1.855700in}}%
\pgfpathlineto{\pgfqpoint{3.608839in}{1.862549in}}%
\pgfpathlineto{\pgfqpoint{3.622633in}{1.869656in}}%
\pgfpathlineto{\pgfqpoint{3.636427in}{1.872935in}}%
\pgfpathlineto{\pgfqpoint{3.650221in}{1.874626in}}%
\pgfpathlineto{\pgfqpoint{3.664015in}{1.872303in}}%
\pgfpathlineto{\pgfqpoint{3.677808in}{1.875768in}}%
\pgfpathlineto{\pgfqpoint{3.691602in}{1.874471in}}%
\pgfpathlineto{\pgfqpoint{3.705396in}{1.875509in}}%
\pgfpathlineto{\pgfqpoint{3.719190in}{1.881683in}}%
\pgfpathlineto{\pgfqpoint{3.732984in}{1.883841in}}%
\pgfpathlineto{\pgfqpoint{3.760571in}{1.884049in}}%
\pgfpathlineto{\pgfqpoint{3.774365in}{1.886114in}}%
\pgfpathlineto{\pgfqpoint{3.788159in}{1.894062in}}%
\pgfpathlineto{\pgfqpoint{3.829540in}{1.898950in}}%
\pgfpathlineto{\pgfqpoint{3.843334in}{1.898120in}}%
\pgfpathlineto{\pgfqpoint{3.857128in}{1.905321in}}%
\pgfpathlineto{\pgfqpoint{3.884716in}{1.906183in}}%
\pgfpathlineto{\pgfqpoint{3.898509in}{1.904232in}}%
\pgfpathlineto{\pgfqpoint{3.912303in}{1.912274in}}%
\pgfpathlineto{\pgfqpoint{3.926097in}{1.909296in}}%
\pgfpathlineto{\pgfqpoint{3.939891in}{1.913322in}}%
\pgfpathlineto{\pgfqpoint{3.953685in}{1.911278in}}%
\pgfpathlineto{\pgfqpoint{3.981272in}{1.920077in}}%
\pgfpathlineto{\pgfqpoint{3.995066in}{1.920741in}}%
\pgfpathlineto{\pgfqpoint{4.008860in}{1.926168in}}%
\pgfpathlineto{\pgfqpoint{4.022654in}{1.925618in}}%
\pgfpathlineto{\pgfqpoint{4.036447in}{1.928897in}}%
\pgfpathlineto{\pgfqpoint{4.050241in}{1.935258in}}%
\pgfpathlineto{\pgfqpoint{4.133004in}{1.958200in}}%
\pgfpathlineto{\pgfqpoint{4.146798in}{1.958117in}}%
\pgfpathlineto{\pgfqpoint{4.174386in}{1.963928in}}%
\pgfpathlineto{\pgfqpoint{4.188179in}{1.968514in}}%
\pgfpathlineto{\pgfqpoint{4.201973in}{1.971046in}}%
\pgfpathlineto{\pgfqpoint{4.215767in}{1.971524in}}%
\pgfpathlineto{\pgfqpoint{4.215767in}{1.971524in}}%
\pgfusepath{stroke}%
\end{pgfscope}%
\begin{pgfscope}%
\pgfpathrectangle{\pgfqpoint{0.609375in}{0.376614in}}{\pgfqpoint{3.778125in}{2.274751in}} %
\pgfusepath{clip}%
\pgfsetrectcap%
\pgfsetroundjoin%
\pgfsetlinewidth{2.007500pt}%
\definecolor{currentstroke}{rgb}{0.580392,0.403922,0.741176}%
\pgfsetstrokecolor{currentstroke}%
\pgfsetdash{}{0pt}%
\pgfpathmoveto{\pgfqpoint{0.781108in}{0.484401in}}%
\pgfpathlineto{\pgfqpoint{0.794902in}{0.540721in}}%
\pgfpathlineto{\pgfqpoint{0.822489in}{0.618716in}}%
\pgfpathlineto{\pgfqpoint{0.850077in}{0.679528in}}%
\pgfpathlineto{\pgfqpoint{0.863871in}{0.706993in}}%
\pgfpathlineto{\pgfqpoint{0.919046in}{0.796495in}}%
\pgfpathlineto{\pgfqpoint{0.932840in}{0.814528in}}%
\pgfpathlineto{\pgfqpoint{0.960428in}{0.854330in}}%
\pgfpathlineto{\pgfqpoint{0.988015in}{0.885447in}}%
\pgfpathlineto{\pgfqpoint{1.015603in}{0.914697in}}%
\pgfpathlineto{\pgfqpoint{1.029397in}{0.924232in}}%
\pgfpathlineto{\pgfqpoint{1.056984in}{0.938727in}}%
\pgfpathlineto{\pgfqpoint{1.084572in}{0.955277in}}%
\pgfpathlineto{\pgfqpoint{1.139747in}{0.994260in}}%
\pgfpathlineto{\pgfqpoint{1.167335in}{1.019774in}}%
\pgfpathlineto{\pgfqpoint{1.181128in}{1.029870in}}%
\pgfpathlineto{\pgfqpoint{1.194922in}{1.037071in}}%
\pgfpathlineto{\pgfqpoint{1.208716in}{1.049501in}}%
\pgfpathlineto{\pgfqpoint{1.236304in}{1.063156in}}%
\pgfpathlineto{\pgfqpoint{1.250098in}{1.075680in}}%
\pgfpathlineto{\pgfqpoint{1.263891in}{1.085775in}}%
\pgfpathlineto{\pgfqpoint{1.277685in}{1.093817in}}%
\pgfpathlineto{\pgfqpoint{1.305273in}{1.107098in}}%
\pgfpathlineto{\pgfqpoint{1.319067in}{1.117941in}}%
\pgfpathlineto{\pgfqpoint{1.332860in}{1.122247in}}%
\pgfpathlineto{\pgfqpoint{1.374242in}{1.147118in}}%
\pgfpathlineto{\pgfqpoint{1.388036in}{1.152078in}}%
\pgfpathlineto{\pgfqpoint{1.401829in}{1.154890in}}%
\pgfpathlineto{\pgfqpoint{1.415623in}{1.163772in}}%
\pgfpathlineto{\pgfqpoint{1.429417in}{1.175362in}}%
\pgfpathlineto{\pgfqpoint{1.443211in}{1.181442in}}%
\pgfpathlineto{\pgfqpoint{1.457005in}{1.189857in}}%
\pgfpathlineto{\pgfqpoint{1.470799in}{1.196405in}}%
\pgfpathlineto{\pgfqpoint{1.484592in}{1.205940in}}%
\pgfpathlineto{\pgfqpoint{1.512180in}{1.216420in}}%
\pgfpathlineto{\pgfqpoint{1.539768in}{1.236051in}}%
\pgfpathlineto{\pgfqpoint{1.567355in}{1.249333in}}%
\pgfpathlineto{\pgfqpoint{1.581149in}{1.254012in}}%
\pgfpathlineto{\pgfqpoint{1.594943in}{1.256171in}}%
\pgfpathlineto{\pgfqpoint{1.608737in}{1.264586in}}%
\pgfpathlineto{\pgfqpoint{1.636324in}{1.273478in}}%
\pgfpathlineto{\pgfqpoint{1.663912in}{1.286199in}}%
\pgfpathlineto{\pgfqpoint{1.677706in}{1.286116in}}%
\pgfpathlineto{\pgfqpoint{1.691500in}{1.290609in}}%
\pgfpathlineto{\pgfqpoint{1.705293in}{1.292861in}}%
\pgfpathlineto{\pgfqpoint{1.719087in}{1.298661in}}%
\pgfpathlineto{\pgfqpoint{1.732881in}{1.300633in}}%
\pgfpathlineto{\pgfqpoint{1.746675in}{1.311569in}}%
\pgfpathlineto{\pgfqpoint{1.774262in}{1.327278in}}%
\pgfpathlineto{\pgfqpoint{1.788056in}{1.331865in}}%
\pgfpathlineto{\pgfqpoint{1.815644in}{1.344492in}}%
\pgfpathlineto{\pgfqpoint{1.829438in}{1.349826in}}%
\pgfpathlineto{\pgfqpoint{1.843231in}{1.353198in}}%
\pgfpathlineto{\pgfqpoint{1.857025in}{1.359932in}}%
\pgfpathlineto{\pgfqpoint{1.884613in}{1.367891in}}%
\pgfpathlineto{\pgfqpoint{1.898407in}{1.373317in}}%
\pgfpathlineto{\pgfqpoint{1.912201in}{1.382666in}}%
\pgfpathlineto{\pgfqpoint{1.925994in}{1.385105in}}%
\pgfpathlineto{\pgfqpoint{2.022551in}{1.417956in}}%
\pgfpathlineto{\pgfqpoint{2.036345in}{1.424596in}}%
\pgfpathlineto{\pgfqpoint{2.050139in}{1.427969in}}%
\pgfpathlineto{\pgfqpoint{2.063932in}{1.428072in}}%
\pgfpathlineto{\pgfqpoint{2.077726in}{1.435180in}}%
\pgfpathlineto{\pgfqpoint{2.119108in}{1.441935in}}%
\pgfpathlineto{\pgfqpoint{2.132901in}{1.448483in}}%
\pgfpathlineto{\pgfqpoint{2.160489in}{1.454760in}}%
\pgfpathlineto{\pgfqpoint{2.174283in}{1.466537in}}%
\pgfpathlineto{\pgfqpoint{2.188077in}{1.474298in}}%
\pgfpathlineto{\pgfqpoint{2.201871in}{1.477857in}}%
\pgfpathlineto{\pgfqpoint{2.257046in}{1.500498in}}%
\pgfpathlineto{\pgfqpoint{2.270840in}{1.502563in}}%
\pgfpathlineto{\pgfqpoint{2.284633in}{1.508830in}}%
\pgfpathlineto{\pgfqpoint{2.298427in}{1.511455in}}%
\pgfpathlineto{\pgfqpoint{2.339809in}{1.526335in}}%
\pgfpathlineto{\pgfqpoint{2.353602in}{1.527186in}}%
\pgfpathlineto{\pgfqpoint{2.408778in}{1.540302in}}%
\pgfpathlineto{\pgfqpoint{2.450159in}{1.554527in}}%
\pgfpathlineto{\pgfqpoint{2.463953in}{1.553324in}}%
\pgfpathlineto{\pgfqpoint{2.477747in}{1.556416in}}%
\pgfpathlineto{\pgfqpoint{2.491541in}{1.561656in}}%
\pgfpathlineto{\pgfqpoint{2.519128in}{1.566160in}}%
\pgfpathlineto{\pgfqpoint{2.546716in}{1.577573in}}%
\pgfpathlineto{\pgfqpoint{2.560510in}{1.578331in}}%
\pgfpathlineto{\pgfqpoint{2.574303in}{1.585345in}}%
\pgfpathlineto{\pgfqpoint{2.588097in}{1.587597in}}%
\pgfpathlineto{\pgfqpoint{2.601891in}{1.595918in}}%
\pgfpathlineto{\pgfqpoint{2.615685in}{1.599011in}}%
\pgfpathlineto{\pgfqpoint{2.629479in}{1.607986in}}%
\pgfpathlineto{\pgfqpoint{2.657066in}{1.613890in}}%
\pgfpathlineto{\pgfqpoint{2.684654in}{1.623810in}}%
\pgfpathlineto{\pgfqpoint{2.698448in}{1.631104in}}%
\pgfpathlineto{\pgfqpoint{2.712242in}{1.635597in}}%
\pgfpathlineto{\pgfqpoint{2.739829in}{1.639073in}}%
\pgfpathlineto{\pgfqpoint{2.753623in}{1.643753in}}%
\pgfpathlineto{\pgfqpoint{2.767417in}{1.652635in}}%
\pgfpathlineto{\pgfqpoint{2.781211in}{1.655820in}}%
\pgfpathlineto{\pgfqpoint{2.795004in}{1.656391in}}%
\pgfpathlineto{\pgfqpoint{2.808798in}{1.663779in}}%
\pgfpathlineto{\pgfqpoint{2.836386in}{1.674259in}}%
\pgfpathlineto{\pgfqpoint{2.850180in}{1.674830in}}%
\pgfpathlineto{\pgfqpoint{2.905355in}{1.685331in}}%
\pgfpathlineto{\pgfqpoint{2.919149in}{1.694026in}}%
\pgfpathlineto{\pgfqpoint{2.946736in}{1.701611in}}%
\pgfpathlineto{\pgfqpoint{2.960530in}{1.711146in}}%
\pgfpathlineto{\pgfqpoint{2.988118in}{1.718265in}}%
\pgfpathlineto{\pgfqpoint{3.001912in}{1.718462in}}%
\pgfpathlineto{\pgfqpoint{3.015705in}{1.725196in}}%
\pgfpathlineto{\pgfqpoint{3.029499in}{1.727915in}}%
\pgfpathlineto{\pgfqpoint{3.043293in}{1.726805in}}%
\pgfpathlineto{\pgfqpoint{3.057087in}{1.734379in}}%
\pgfpathlineto{\pgfqpoint{3.070881in}{1.734110in}}%
\pgfpathlineto{\pgfqpoint{3.084674in}{1.735427in}}%
\pgfpathlineto{\pgfqpoint{3.098468in}{1.739360in}}%
\pgfpathlineto{\pgfqpoint{3.112262in}{1.746188in}}%
\pgfpathlineto{\pgfqpoint{3.153644in}{1.754063in}}%
\pgfpathlineto{\pgfqpoint{3.167437in}{1.755195in}}%
\pgfpathlineto{\pgfqpoint{3.181231in}{1.760154in}}%
\pgfpathlineto{\pgfqpoint{3.222613in}{1.765229in}}%
\pgfpathlineto{\pgfqpoint{3.236406in}{1.768508in}}%
\pgfpathlineto{\pgfqpoint{3.250200in}{1.767771in}}%
\pgfpathlineto{\pgfqpoint{3.263994in}{1.770583in}}%
\pgfpathlineto{\pgfqpoint{3.291582in}{1.772098in}}%
\pgfpathlineto{\pgfqpoint{3.332963in}{1.772877in}}%
\pgfpathlineto{\pgfqpoint{3.360551in}{1.776540in}}%
\pgfpathlineto{\pgfqpoint{3.374345in}{1.780006in}}%
\pgfpathlineto{\pgfqpoint{3.388138in}{1.776281in}}%
\pgfpathlineto{\pgfqpoint{3.415726in}{1.779011in}}%
\pgfpathlineto{\pgfqpoint{3.429520in}{1.776967in}}%
\pgfpathlineto{\pgfqpoint{3.443314in}{1.784728in}}%
\pgfpathlineto{\pgfqpoint{3.457107in}{1.786980in}}%
\pgfpathlineto{\pgfqpoint{3.470901in}{1.794648in}}%
\pgfpathlineto{\pgfqpoint{3.484695in}{1.799794in}}%
\pgfpathlineto{\pgfqpoint{3.498489in}{1.800085in}}%
\pgfpathlineto{\pgfqpoint{3.539870in}{1.809735in}}%
\pgfpathlineto{\pgfqpoint{3.553664in}{1.811613in}}%
\pgfpathlineto{\pgfqpoint{3.567458in}{1.809103in}}%
\pgfpathlineto{\pgfqpoint{3.608839in}{1.812963in}}%
\pgfpathlineto{\pgfqpoint{3.622633in}{1.819977in}}%
\pgfpathlineto{\pgfqpoint{3.650221in}{1.822893in}}%
\pgfpathlineto{\pgfqpoint{3.664015in}{1.820849in}}%
\pgfpathlineto{\pgfqpoint{3.677808in}{1.824875in}}%
\pgfpathlineto{\pgfqpoint{3.691602in}{1.823205in}}%
\pgfpathlineto{\pgfqpoint{3.705396in}{1.824710in}}%
\pgfpathlineto{\pgfqpoint{3.732984in}{1.832668in}}%
\pgfpathlineto{\pgfqpoint{3.760571in}{1.835117in}}%
\pgfpathlineto{\pgfqpoint{3.774365in}{1.838396in}}%
\pgfpathlineto{\pgfqpoint{3.788159in}{1.844103in}}%
\pgfpathlineto{\pgfqpoint{3.829540in}{1.849831in}}%
\pgfpathlineto{\pgfqpoint{3.843334in}{1.849842in}}%
\pgfpathlineto{\pgfqpoint{3.857128in}{1.857603in}}%
\pgfpathlineto{\pgfqpoint{3.870922in}{1.856960in}}%
\pgfpathlineto{\pgfqpoint{3.884716in}{1.859025in}}%
\pgfpathlineto{\pgfqpoint{3.898509in}{1.858195in}}%
\pgfpathlineto{\pgfqpoint{3.912303in}{1.864556in}}%
\pgfpathlineto{\pgfqpoint{3.926097in}{1.861298in}}%
\pgfpathlineto{\pgfqpoint{3.939891in}{1.862803in}}%
\pgfpathlineto{\pgfqpoint{3.953685in}{1.860572in}}%
\pgfpathlineto{\pgfqpoint{3.967478in}{1.863104in}}%
\pgfpathlineto{\pgfqpoint{3.981272in}{1.867877in}}%
\pgfpathlineto{\pgfqpoint{3.995066in}{1.869568in}}%
\pgfpathlineto{\pgfqpoint{4.008860in}{1.875742in}}%
\pgfpathlineto{\pgfqpoint{4.022654in}{1.874632in}}%
\pgfpathlineto{\pgfqpoint{4.077829in}{1.894098in}}%
\pgfpathlineto{\pgfqpoint{4.091623in}{1.897003in}}%
\pgfpathlineto{\pgfqpoint{4.105417in}{1.902897in}}%
\pgfpathlineto{\pgfqpoint{4.133004in}{1.908521in}}%
\pgfpathlineto{\pgfqpoint{4.146798in}{1.908438in}}%
\pgfpathlineto{\pgfqpoint{4.174386in}{1.913875in}}%
\pgfpathlineto{\pgfqpoint{4.188179in}{1.919022in}}%
\pgfpathlineto{\pgfqpoint{4.201973in}{1.921180in}}%
\pgfpathlineto{\pgfqpoint{4.215767in}{1.921565in}}%
\pgfpathlineto{\pgfqpoint{4.215767in}{1.921565in}}%
\pgfusepath{stroke}%
\end{pgfscope}%
\begin{pgfscope}%
\pgfpathrectangle{\pgfqpoint{0.609375in}{0.376614in}}{\pgfqpoint{3.778125in}{2.274751in}} %
\pgfusepath{clip}%
\pgfsetrectcap%
\pgfsetroundjoin%
\pgfsetlinewidth{2.007500pt}%
\definecolor{currentstroke}{rgb}{0.549020,0.337255,0.294118}%
\pgfsetstrokecolor{currentstroke}%
\pgfsetdash{}{0pt}%
\pgfpathmoveto{\pgfqpoint{0.781108in}{0.484401in}}%
\pgfpathlineto{\pgfqpoint{0.794902in}{0.540721in}}%
\pgfpathlineto{\pgfqpoint{0.822489in}{0.617595in}}%
\pgfpathlineto{\pgfqpoint{0.850077in}{0.675980in}}%
\pgfpathlineto{\pgfqpoint{0.863871in}{0.703071in}}%
\pgfpathlineto{\pgfqpoint{0.919046in}{0.791265in}}%
\pgfpathlineto{\pgfqpoint{0.974221in}{0.865359in}}%
\pgfpathlineto{\pgfqpoint{1.015603in}{0.909561in}}%
\pgfpathlineto{\pgfqpoint{1.043190in}{0.927324in}}%
\pgfpathlineto{\pgfqpoint{1.112159in}{0.970519in}}%
\pgfpathlineto{\pgfqpoint{1.181128in}{1.027442in}}%
\pgfpathlineto{\pgfqpoint{1.194922in}{1.035483in}}%
\pgfpathlineto{\pgfqpoint{1.208716in}{1.048847in}}%
\pgfpathlineto{\pgfqpoint{1.222510in}{1.058009in}}%
\pgfpathlineto{\pgfqpoint{1.236304in}{1.063810in}}%
\pgfpathlineto{\pgfqpoint{1.263891in}{1.085308in}}%
\pgfpathlineto{\pgfqpoint{1.305273in}{1.108125in}}%
\pgfpathlineto{\pgfqpoint{1.319067in}{1.118595in}}%
\pgfpathlineto{\pgfqpoint{1.332860in}{1.123181in}}%
\pgfpathlineto{\pgfqpoint{1.346654in}{1.130289in}}%
\pgfpathlineto{\pgfqpoint{1.374242in}{1.149546in}}%
\pgfpathlineto{\pgfqpoint{1.388036in}{1.155813in}}%
\pgfpathlineto{\pgfqpoint{1.401829in}{1.158532in}}%
\pgfpathlineto{\pgfqpoint{1.415623in}{1.166293in}}%
\pgfpathlineto{\pgfqpoint{1.429417in}{1.176669in}}%
\pgfpathlineto{\pgfqpoint{1.470799in}{1.198739in}}%
\pgfpathlineto{\pgfqpoint{1.484592in}{1.209489in}}%
\pgfpathlineto{\pgfqpoint{1.498386in}{1.216876in}}%
\pgfpathlineto{\pgfqpoint{1.512180in}{1.222116in}}%
\pgfpathlineto{\pgfqpoint{1.553561in}{1.246988in}}%
\pgfpathlineto{\pgfqpoint{1.594943in}{1.262521in}}%
\pgfpathlineto{\pgfqpoint{1.608737in}{1.271496in}}%
\pgfpathlineto{\pgfqpoint{1.622530in}{1.277389in}}%
\pgfpathlineto{\pgfqpoint{1.636324in}{1.281322in}}%
\pgfpathlineto{\pgfqpoint{1.663912in}{1.293950in}}%
\pgfpathlineto{\pgfqpoint{1.677706in}{1.293960in}}%
\pgfpathlineto{\pgfqpoint{1.705293in}{1.300425in}}%
\pgfpathlineto{\pgfqpoint{1.719087in}{1.308093in}}%
\pgfpathlineto{\pgfqpoint{1.732881in}{1.309317in}}%
\pgfpathlineto{\pgfqpoint{1.760469in}{1.326521in}}%
\pgfpathlineto{\pgfqpoint{1.774262in}{1.333535in}}%
\pgfpathlineto{\pgfqpoint{1.788056in}{1.338588in}}%
\pgfpathlineto{\pgfqpoint{1.815644in}{1.352803in}}%
\pgfpathlineto{\pgfqpoint{1.829438in}{1.358790in}}%
\pgfpathlineto{\pgfqpoint{1.843231in}{1.362443in}}%
\pgfpathlineto{\pgfqpoint{1.857025in}{1.368990in}}%
\pgfpathlineto{\pgfqpoint{1.884613in}{1.378256in}}%
\pgfpathlineto{\pgfqpoint{1.925994in}{1.396777in}}%
\pgfpathlineto{\pgfqpoint{1.953582in}{1.405763in}}%
\pgfpathlineto{\pgfqpoint{1.967376in}{1.411657in}}%
\pgfpathlineto{\pgfqpoint{1.994963in}{1.426245in}}%
\pgfpathlineto{\pgfqpoint{2.050139in}{1.444497in}}%
\pgfpathlineto{\pgfqpoint{2.063932in}{1.444414in}}%
\pgfpathlineto{\pgfqpoint{2.077726in}{1.450681in}}%
\pgfpathlineto{\pgfqpoint{2.119108in}{1.461545in}}%
\pgfpathlineto{\pgfqpoint{2.132901in}{1.468560in}}%
\pgfpathlineto{\pgfqpoint{2.160489in}{1.474557in}}%
\pgfpathlineto{\pgfqpoint{2.174283in}{1.485400in}}%
\pgfpathlineto{\pgfqpoint{2.188077in}{1.492228in}}%
\pgfpathlineto{\pgfqpoint{2.201871in}{1.494666in}}%
\pgfpathlineto{\pgfqpoint{2.215664in}{1.501400in}}%
\pgfpathlineto{\pgfqpoint{2.229458in}{1.506080in}}%
\pgfpathlineto{\pgfqpoint{2.243252in}{1.512440in}}%
\pgfpathlineto{\pgfqpoint{2.270840in}{1.516664in}}%
\pgfpathlineto{\pgfqpoint{2.284633in}{1.523024in}}%
\pgfpathlineto{\pgfqpoint{2.298427in}{1.526770in}}%
\pgfpathlineto{\pgfqpoint{2.326015in}{1.537530in}}%
\pgfpathlineto{\pgfqpoint{2.339809in}{1.542863in}}%
\pgfpathlineto{\pgfqpoint{2.367396in}{1.547367in}}%
\pgfpathlineto{\pgfqpoint{2.450159in}{1.572643in}}%
\pgfpathlineto{\pgfqpoint{2.463953in}{1.570133in}}%
\pgfpathlineto{\pgfqpoint{2.477747in}{1.573225in}}%
\pgfpathlineto{\pgfqpoint{2.491541in}{1.577905in}}%
\pgfpathlineto{\pgfqpoint{2.505334in}{1.578662in}}%
\pgfpathlineto{\pgfqpoint{2.519128in}{1.581754in}}%
\pgfpathlineto{\pgfqpoint{2.546716in}{1.591487in}}%
\pgfpathlineto{\pgfqpoint{2.560510in}{1.592432in}}%
\pgfpathlineto{\pgfqpoint{2.574303in}{1.598512in}}%
\pgfpathlineto{\pgfqpoint{2.588097in}{1.600017in}}%
\pgfpathlineto{\pgfqpoint{2.601891in}{1.608618in}}%
\pgfpathlineto{\pgfqpoint{2.615685in}{1.609656in}}%
\pgfpathlineto{\pgfqpoint{2.629479in}{1.618351in}}%
\pgfpathlineto{\pgfqpoint{2.643273in}{1.619296in}}%
\pgfpathlineto{\pgfqpoint{2.670860in}{1.626320in}}%
\pgfpathlineto{\pgfqpoint{2.684654in}{1.631840in}}%
\pgfpathlineto{\pgfqpoint{2.712242in}{1.645682in}}%
\pgfpathlineto{\pgfqpoint{2.726035in}{1.648401in}}%
\pgfpathlineto{\pgfqpoint{2.739829in}{1.649252in}}%
\pgfpathlineto{\pgfqpoint{2.753623in}{1.653465in}}%
\pgfpathlineto{\pgfqpoint{2.767417in}{1.663374in}}%
\pgfpathlineto{\pgfqpoint{2.781211in}{1.666466in}}%
\pgfpathlineto{\pgfqpoint{2.795004in}{1.665916in}}%
\pgfpathlineto{\pgfqpoint{2.808798in}{1.673771in}}%
\pgfpathlineto{\pgfqpoint{2.836386in}{1.684344in}}%
\pgfpathlineto{\pgfqpoint{2.850180in}{1.683514in}}%
\pgfpathlineto{\pgfqpoint{2.863974in}{1.686513in}}%
\pgfpathlineto{\pgfqpoint{2.877767in}{1.686804in}}%
\pgfpathlineto{\pgfqpoint{2.891561in}{1.692324in}}%
\pgfpathlineto{\pgfqpoint{2.905355in}{1.695883in}}%
\pgfpathlineto{\pgfqpoint{2.919149in}{1.706632in}}%
\pgfpathlineto{\pgfqpoint{2.946736in}{1.712910in}}%
\pgfpathlineto{\pgfqpoint{2.960530in}{1.722446in}}%
\pgfpathlineto{\pgfqpoint{2.988118in}{1.727042in}}%
\pgfpathlineto{\pgfqpoint{3.001912in}{1.726866in}}%
\pgfpathlineto{\pgfqpoint{3.015705in}{1.731919in}}%
\pgfpathlineto{\pgfqpoint{3.029499in}{1.734825in}}%
\pgfpathlineto{\pgfqpoint{3.043293in}{1.734929in}}%
\pgfpathlineto{\pgfqpoint{3.057087in}{1.742036in}}%
\pgfpathlineto{\pgfqpoint{3.070881in}{1.740833in}}%
\pgfpathlineto{\pgfqpoint{3.084674in}{1.743085in}}%
\pgfpathlineto{\pgfqpoint{3.112262in}{1.752257in}}%
\pgfpathlineto{\pgfqpoint{3.126056in}{1.752922in}}%
\pgfpathlineto{\pgfqpoint{3.167437in}{1.761918in}}%
\pgfpathlineto{\pgfqpoint{3.181231in}{1.768279in}}%
\pgfpathlineto{\pgfqpoint{3.222613in}{1.773633in}}%
\pgfpathlineto{\pgfqpoint{3.236406in}{1.778406in}}%
\pgfpathlineto{\pgfqpoint{3.250200in}{1.776829in}}%
\pgfpathlineto{\pgfqpoint{3.277788in}{1.780119in}}%
\pgfpathlineto{\pgfqpoint{3.360551in}{1.787186in}}%
\pgfpathlineto{\pgfqpoint{3.374345in}{1.791305in}}%
\pgfpathlineto{\pgfqpoint{3.388138in}{1.789635in}}%
\pgfpathlineto{\pgfqpoint{3.415726in}{1.794699in}}%
\pgfpathlineto{\pgfqpoint{3.429520in}{1.793962in}}%
\pgfpathlineto{\pgfqpoint{3.443314in}{1.802657in}}%
\pgfpathlineto{\pgfqpoint{3.457107in}{1.804816in}}%
\pgfpathlineto{\pgfqpoint{3.470901in}{1.812764in}}%
\pgfpathlineto{\pgfqpoint{3.484695in}{1.818190in}}%
\pgfpathlineto{\pgfqpoint{3.498489in}{1.818108in}}%
\pgfpathlineto{\pgfqpoint{3.512283in}{1.821573in}}%
\pgfpathlineto{\pgfqpoint{3.526076in}{1.823078in}}%
\pgfpathlineto{\pgfqpoint{3.539870in}{1.828785in}}%
\pgfpathlineto{\pgfqpoint{3.553664in}{1.831690in}}%
\pgfpathlineto{\pgfqpoint{3.567458in}{1.829927in}}%
\pgfpathlineto{\pgfqpoint{3.608839in}{1.836869in}}%
\pgfpathlineto{\pgfqpoint{3.622633in}{1.844443in}}%
\pgfpathlineto{\pgfqpoint{3.636427in}{1.847348in}}%
\pgfpathlineto{\pgfqpoint{3.650221in}{1.847919in}}%
\pgfpathlineto{\pgfqpoint{3.664015in}{1.845969in}}%
\pgfpathlineto{\pgfqpoint{3.677808in}{1.848781in}}%
\pgfpathlineto{\pgfqpoint{3.691602in}{1.847858in}}%
\pgfpathlineto{\pgfqpoint{3.705396in}{1.849362in}}%
\pgfpathlineto{\pgfqpoint{3.719190in}{1.856096in}}%
\pgfpathlineto{\pgfqpoint{3.732984in}{1.858815in}}%
\pgfpathlineto{\pgfqpoint{3.760571in}{1.859210in}}%
\pgfpathlineto{\pgfqpoint{3.774365in}{1.862395in}}%
\pgfpathlineto{\pgfqpoint{3.788159in}{1.868569in}}%
\pgfpathlineto{\pgfqpoint{3.801953in}{1.870447in}}%
\pgfpathlineto{\pgfqpoint{3.815747in}{1.874940in}}%
\pgfpathlineto{\pgfqpoint{3.843334in}{1.874401in}}%
\pgfpathlineto{\pgfqpoint{3.857128in}{1.881322in}}%
\pgfpathlineto{\pgfqpoint{3.870922in}{1.881239in}}%
\pgfpathlineto{\pgfqpoint{3.884716in}{1.883304in}}%
\pgfpathlineto{\pgfqpoint{3.898509in}{1.881634in}}%
\pgfpathlineto{\pgfqpoint{3.912303in}{1.887434in}}%
\pgfpathlineto{\pgfqpoint{3.926097in}{1.885017in}}%
\pgfpathlineto{\pgfqpoint{3.939891in}{1.888202in}}%
\pgfpathlineto{\pgfqpoint{3.953685in}{1.885785in}}%
\pgfpathlineto{\pgfqpoint{3.967478in}{1.888504in}}%
\pgfpathlineto{\pgfqpoint{3.981272in}{1.894024in}}%
\pgfpathlineto{\pgfqpoint{3.995066in}{1.895435in}}%
\pgfpathlineto{\pgfqpoint{4.008860in}{1.900302in}}%
\pgfpathlineto{\pgfqpoint{4.022654in}{1.899472in}}%
\pgfpathlineto{\pgfqpoint{4.036447in}{1.902004in}}%
\pgfpathlineto{\pgfqpoint{4.050241in}{1.908551in}}%
\pgfpathlineto{\pgfqpoint{4.077829in}{1.916883in}}%
\pgfpathlineto{\pgfqpoint{4.119210in}{1.927934in}}%
\pgfpathlineto{\pgfqpoint{4.133004in}{1.930559in}}%
\pgfpathlineto{\pgfqpoint{4.174386in}{1.934046in}}%
\pgfpathlineto{\pgfqpoint{4.201973in}{1.941724in}}%
\pgfpathlineto{\pgfqpoint{4.215767in}{1.941361in}}%
\pgfpathlineto{\pgfqpoint{4.215767in}{1.941361in}}%
\pgfusepath{stroke}%
\end{pgfscope}%
\begin{pgfscope}%
\pgfsetrectcap%
\pgfsetmiterjoin%
\pgfsetlinewidth{0.803000pt}%
\definecolor{currentstroke}{rgb}{0.000000,0.000000,0.000000}%
\pgfsetstrokecolor{currentstroke}%
\pgfsetdash{}{0pt}%
\pgfpathmoveto{\pgfqpoint{0.609375in}{0.376614in}}%
\pgfpathlineto{\pgfqpoint{0.609375in}{2.651366in}}%
\pgfusepath{stroke}%
\end{pgfscope}%
\begin{pgfscope}%
\pgfsetrectcap%
\pgfsetmiterjoin%
\pgfsetlinewidth{0.803000pt}%
\definecolor{currentstroke}{rgb}{0.000000,0.000000,0.000000}%
\pgfsetstrokecolor{currentstroke}%
\pgfsetdash{}{0pt}%
\pgfpathmoveto{\pgfqpoint{4.387500in}{0.376614in}}%
\pgfpathlineto{\pgfqpoint{4.387500in}{2.651366in}}%
\pgfusepath{stroke}%
\end{pgfscope}%
\begin{pgfscope}%
\pgfsetrectcap%
\pgfsetmiterjoin%
\pgfsetlinewidth{0.803000pt}%
\definecolor{currentstroke}{rgb}{0.000000,0.000000,0.000000}%
\pgfsetstrokecolor{currentstroke}%
\pgfsetdash{}{0pt}%
\pgfpathmoveto{\pgfqpoint{0.609375in}{0.376614in}}%
\pgfpathlineto{\pgfqpoint{4.387500in}{0.376614in}}%
\pgfusepath{stroke}%
\end{pgfscope}%
\begin{pgfscope}%
\pgfsetrectcap%
\pgfsetmiterjoin%
\pgfsetlinewidth{0.803000pt}%
\definecolor{currentstroke}{rgb}{0.000000,0.000000,0.000000}%
\pgfsetstrokecolor{currentstroke}%
\pgfsetdash{}{0pt}%
\pgfpathmoveto{\pgfqpoint{0.609375in}{2.651366in}}%
\pgfpathlineto{\pgfqpoint{4.387500in}{2.651366in}}%
\pgfusepath{stroke}%
\end{pgfscope}%
\begin{pgfscope}%
\pgfsetbuttcap%
\pgfsetmiterjoin%
\definecolor{currentfill}{rgb}{1.000000,1.000000,1.000000}%
\pgfsetfillcolor{currentfill}%
\pgfsetfillopacity{0.800000}%
\pgfsetlinewidth{1.003750pt}%
\definecolor{currentstroke}{rgb}{0.800000,0.800000,0.800000}%
\pgfsetstrokecolor{currentstroke}%
\pgfsetstrokeopacity{0.800000}%
\pgfsetdash{}{0pt}%
\pgfpathmoveto{\pgfqpoint{3.406769in}{0.432170in}}%
\pgfpathlineto{\pgfqpoint{4.309722in}{0.432170in}}%
\pgfpathquadraticcurveto{\pgfqpoint{4.331944in}{0.432170in}}{\pgfqpoint{4.331944in}{0.454392in}}%
\pgfpathlineto{\pgfqpoint{4.331944in}{1.419706in}}%
\pgfpathquadraticcurveto{\pgfqpoint{4.331944in}{1.441928in}}{\pgfqpoint{4.309722in}{1.441928in}}%
\pgfpathlineto{\pgfqpoint{3.406769in}{1.441928in}}%
\pgfpathquadraticcurveto{\pgfqpoint{3.384547in}{1.441928in}}{\pgfqpoint{3.384547in}{1.419706in}}%
\pgfpathlineto{\pgfqpoint{3.384547in}{0.454392in}}%
\pgfpathquadraticcurveto{\pgfqpoint{3.384547in}{0.432170in}}{\pgfqpoint{3.406769in}{0.432170in}}%
\pgfpathclose%
\pgfusepath{stroke,fill}%
\end{pgfscope}%
\begin{pgfscope}%
\pgfsetrectcap%
\pgfsetroundjoin%
\pgfsetlinewidth{2.007500pt}%
\definecolor{currentstroke}{rgb}{0.121569,0.466667,0.705882}%
\pgfsetstrokecolor{currentstroke}%
\pgfsetdash{}{0pt}%
\pgfpathmoveto{\pgfqpoint{3.428992in}{1.358594in}}%
\pgfpathlineto{\pgfqpoint{3.651214in}{1.358594in}}%
\pgfusepath{stroke}%
\end{pgfscope}%
\begin{pgfscope}%
\pgftext[x=3.740103in,y=1.319706in,left,base]{\rmfamily\fontsize{8.000000}{9.600000}\selectfont Thompson}%
\end{pgfscope}%
\begin{pgfscope}%
\pgfsetrectcap%
\pgfsetroundjoin%
\pgfsetlinewidth{2.007500pt}%
\definecolor{currentstroke}{rgb}{1.000000,0.498039,0.054902}%
\pgfsetstrokecolor{currentstroke}%
\pgfsetdash{}{0pt}%
\pgfpathmoveto{\pgfqpoint{3.428992in}{1.203661in}}%
\pgfpathlineto{\pgfqpoint{3.651214in}{1.203661in}}%
\pgfusepath{stroke}%
\end{pgfscope}%
\begin{pgfscope}%
\pgftext[x=3.740103in,y=1.164773in,left,base]{\rmfamily\fontsize{8.000000}{9.600000}\selectfont IDS}%
\end{pgfscope}%
\begin{pgfscope}%
\pgfsetrectcap%
\pgfsetroundjoin%
\pgfsetlinewidth{2.007500pt}%
\definecolor{currentstroke}{rgb}{0.172549,0.627451,0.172549}%
\pgfsetstrokecolor{currentstroke}%
\pgfsetdash{}{0pt}%
\pgfpathmoveto{\pgfqpoint{3.428992in}{1.043193in}}%
\pgfpathlineto{\pgfqpoint{3.651214in}{1.043193in}}%
\pgfusepath{stroke}%
\end{pgfscope}%
\begin{pgfscope}%
\pgftext[x=3.740103in,y=1.004304in,left,base]{\rmfamily\fontsize{8.000000}{9.600000}\selectfont Whittle(1)}%
\end{pgfscope}%
\begin{pgfscope}%
\pgfsetrectcap%
\pgfsetroundjoin%
\pgfsetlinewidth{2.007500pt}%
\definecolor{currentstroke}{rgb}{0.839216,0.152941,0.156863}%
\pgfsetstrokecolor{currentstroke}%
\pgfsetdash{}{0pt}%
\pgfpathmoveto{\pgfqpoint{3.428992in}{0.876553in}}%
\pgfpathlineto{\pgfqpoint{3.651214in}{0.876553in}}%
\pgfusepath{stroke}%
\end{pgfscope}%
\begin{pgfscope}%
\pgftext[x=3.740103in,y=0.837665in,left,base]{\rmfamily\fontsize{8.000000}{9.600000}\selectfont Whittle(2)}%
\end{pgfscope}%
\begin{pgfscope}%
\pgfsetrectcap%
\pgfsetroundjoin%
\pgfsetlinewidth{2.007500pt}%
\definecolor{currentstroke}{rgb}{0.580392,0.403922,0.741176}%
\pgfsetstrokecolor{currentstroke}%
\pgfsetdash{}{0pt}%
\pgfpathmoveto{\pgfqpoint{3.428992in}{0.709914in}}%
\pgfpathlineto{\pgfqpoint{3.651214in}{0.709914in}}%
\pgfusepath{stroke}%
\end{pgfscope}%
\begin{pgfscope}%
\pgftext[x=3.740103in,y=0.671025in,left,base]{\rmfamily\fontsize{8.000000}{9.600000}\selectfont Whittle(3)}%
\end{pgfscope}%
\begin{pgfscope}%
\pgfsetrectcap%
\pgfsetroundjoin%
\pgfsetlinewidth{2.007500pt}%
\definecolor{currentstroke}{rgb}{0.549020,0.337255,0.294118}%
\pgfsetstrokecolor{currentstroke}%
\pgfsetdash{}{0pt}%
\pgfpathmoveto{\pgfqpoint{3.428992in}{0.543274in}}%
\pgfpathlineto{\pgfqpoint{3.651214in}{0.543274in}}%
\pgfusepath{stroke}%
\end{pgfscope}%
\begin{pgfscope}%
\pgftext[x=3.740103in,y=0.504385in,left,base]{\rmfamily\fontsize{8.000000}{9.600000}\selectfont Whittle(4)}%
\end{pgfscope}%
\end{pgfpicture}%
\makeatother%
\endgroup%

	\caption{Regret for bandits with multiple simultaneous arm pulls}
	\label{fig:restless1}
\end{figure}

\begin{table}
	\centering
	{\color{blue}
	\begin{tabular}{lrrrrrr}
		\toprule
		{} &    IDS &  Thompson &  Whittle(1) &  Whittle(2) &  Whittle(3) &  Whittle(4) \\
		\midrule
		Mean           &  10.97 &     15.23 &       11.13 &       11.29 &       10.93 &       11.07 \\
		Standard error &   0.21 &      0.13 &        0.14 &        0.15 &        0.15 &        0.15 \\
		25\%            &   1.18 &      6.60 &        1.66 &        1.57 &        1.20 &        1.39 \\
		50\%            &  10.84 &     14.75 &       10.34 &        9.96 &        9.91 &        9.74 \\
		75\%            &  24.60 &     23.52 &       19.62 &       19.41 &       19.27 &       19.13 \\
		CPU time (s)   & 54.45 &      2.07 &       14.20 &     1122.12 &     2196.83 &     4106.89 \\
		\bottomrule
	\end{tabular}
	
	\caption{Regret from the multiple arm pulls experiment. ``Whittle($K$)" refers to the Whittle heuristic policy, where $K$ look-ahead steps are used in computing the Optimistic Gittins index.}
}
	\label{table:restless1_summary}
\end{table}

\section{Conclusions} \label{sec:conclusions}
This paper proposed a novel way for designing Bayesian Multi-Armed Bandit algorithms by treating the problem of minimizing regret as a sequence of separate Markov Decision problems where the discount factor increases from one problem to the next, according to a carefully chosen rate. We  showed that the fundamental idea of using such a heuristic results in sub-linear regret and, when applied to a binary bandit problem, that a simple and efficient algorithm with a flat Beta prior achieves the optimal rate of growth in regret.

There are some open questions following this work. First, it remains to be proven that playing arms with maximum (exact) Gittins indices together with the increasing discount factor schedule, does produce an algorithm whose regret matches the Lai-Robbins lower bound. We have a strong reason to suspect this due to the findings in our numerical experiments. Secondly, it is worth exploring whether the idea of this framework can be extended to contextual bandit problems where dependencies between arms exist. In our setting, the fact that arms were independent allowed us to exploit the Gittins index but there could be other ways to approximate optimal solutions to bandit problems with dependent arms.

\bibliographystyle{informs2014} % outcomment this and next line in Case 1
%\singlespacing

\bibliography{references} % if more than one, comma separated
\appendix

\section{Proof of Lemma~\ref{le:linearregret}} \label{proof:linearregret}
{
\begin{myproof}[Proof.]
	Consider an instance of the MAB with $A=2$ arms and Bernoulli rewards. We assume that the prior on arm $1$ is degenerate with mean $\lambda = 1/2$ while arm $2$ has a {\rm Beta}$(\alpha,\alpha+1)$ prior where $\alpha$ is a parameter we set later. {Furthermore, we assume that the true paremeter for arm $2$ is equal to $\theta \in (1/2, 1)$, which represents a draw from the prior distribution on that arm}. Now, the continuation value from pulling arm $1$ at this stage is lower bounded by $\frac{1/2}{1-\gamma}$, while the continuation value from pulling arm $2$ is upper bounded by
	\[
	\frac{\alpha}{1+2\alpha}
	+
	\frac{
		\gamma
		\E
		\left[
		\max\left(
		R(y_{2,0}),1/2
		\right)
		\Big| y_{2,0} = \left(\alpha, \alpha+1 \right)
		\right]
	}
	{
		1-\gamma
	}.
	\]
	It follows that the Gittins index policy must pull arm $1$ if
	\[
	\frac{1/2}{1-\gamma}
	>
	\frac{\alpha}{1+2\alpha}
	+
	\frac{
		\gamma
		\E
		\left[
		\max\left(
		R(y_{2,0}),1/2
		\right)
		\Big| y_{2,0} = \left(\alpha, \alpha+1 \right)
		\right]
	}
	{
		1-\gamma
	},
	\]
	{an inequality which in turn is satisfied if
	\[
	1/2
	>
	\frac{(1-\gamma)\alpha}{1+2\alpha}
	+
	\gamma
	\left(\P{
		R(y_{2,0})>1/2 \
		\Big| y_{2,0} = \left(\alpha, \alpha+1 \right)
	} + 1/2\right) .
	\]
	But the right hand side of the above expression goes to $\gamma/2 < 1/2$ as $\alpha \tends 0$}. Consequently, we can choose an $\alpha$ such that the Gittins index policy chooses to pull the first arm; let $\alpha^*$ be the largest such $\alpha$. Since the state of the first arm does not change (the prior on that arm was assumed degenerate), the same condition must hold at subsequent iterations. {Consequently, $\pi^{G,\gamma}$ must incur a $T$-period frequentist regret lower bounded by 
	$
	T
	\left(
	\theta - 1/2
	\right) .
	$
	The result follows. }
\end{myproof}
}
\section{Proof of Proposition~\ref{prop:gittins_log3T}} \label{proof:prop_log3T}

{
We first establish notation useful in the proof. Specifically, we let $h_t^{t'}$ denote the sequence of arms pulled and the corresponding rewards earned between periods $t$ and $t'$ ($h_t^{t'} = \emptyset$ if $t' < t$) . Recall that $\mathbf y$ denotes the $A$-tuple of sufficient statistics of priors on the the arm means of each of the $A$ arms. We denote by $g(\mathbf y, h_t^{t'})$ the $A$-tuple of sufficient statistics of posteriors on the arm means of each of the $A$ arms, obtained starting with the prior $\mathbf y$ and observing the sequence of arm pulls $h_t^{t'}$. 


We begin by establishing a simple Lemma concerning the allocation rule proposed in \cite{lai1987adaptive}; we note that the allocation rule there is specified for a {\em fixed} time horizon and the Lemma below extends it to a policy that specifies a choice of arm for every epoch. 

\begin{lemma} 
\label{lemma:laianytime}
Let the prior with sufficient statistic $\mathbf y$ satisfy the requirements of Theorem~$3$ in \cite{lai1987adaptive}. Then, there exists a policy $\tilde \pi_{\mathbf y}$, and a constant $C_{\mathbf y}$ for which
\[
{\rm Regret}\left(
\tilde \pi_{\mathbf y}, T
\right)
\triangleq 
\E_{\mathbf y}
\left[
{\rm Regret}\left(
\tilde \pi_{\mathbf y}, T,\theta
\right)
\right]
\leq
C_{\mathbf y} \log^3 T
\] 
for all $T$. 
\end{lemma}
\begin{myproof}[Proof.]
By Theorem $3$ in \cite{lai1987adaptive}, we know that there exists a constant $\hat C_{\mathbf y}$ and a sequence of policies, $\tilde \pi_{\mathbf y, T}$ \footnote{this entails a minor abuse of our definition of a policy: $\tilde \pi_{\mathbf y, T}$ is not specified for $t > T$}  such that 
\[
{\lim_{T\to \infty}}
\frac{
\E
_{\mathbf y}
\left[
{\rm Regret}\left(
\tilde \pi_{\mathbf y, T}, T,\theta
\right)
\right]
}
{
\log^2 T
}
=
\hat C_{\mathbf y}.
\]
Consequently, there exists a constant $\bar C_{\mathbf y}$, so that for any $T$, 
\[
\E
_{\mathbf y}
\left[
{\rm Regret}\left(
\tilde \pi_{\mathbf y, T}, T,\theta
\right)
\right]
\leq
\bar C_{\mathbf y} \log^2 T
\]
Now consider the `doubling' policy $\tilde \pi_{\mathbf y}$ which at time $t$ selects arms according to the policy $\tilde \pi_{\mathbf y,2^k}$ applied at the state $g\left(\mathbf y, h_{2^{k(t)}-1}^{t-1}\right)$ where $k(t) = \lfloor \log_2(t+1) \rfloor$. In words, this is the policy obtained wherein (a) time is divided into epochs such that the $k$th epoch extends from time $2^{k}-1$ to $2^{k+1}-2$, and (b) at the start of the $k$th epoch we `forget' everything learned up until that time and subsequently use the policy $\tilde \pi_{\mathbf y,2^k}$ over the course of that epoch. Thus,
\[
\begin{split}
\E
_{\mathbf y}
\left[
{\rm Regret}\left(
\tilde \pi_{\mathbf y}, T,\theta
\right)
\right]
&
\leq
\sum_{k=1}^{\lfloor \log (T+2) \rfloor}
\E_{\mathbf y}
\left[
\E_{\mathbf y_{2^k -2}}
\left[
{\rm Regret}\left(
\tilde \pi_{\mathbf y,2^k}, 2^k,\theta
\right)
\right]
\right]
\\
&
=
\sum_{k=1}^{\lfloor \log (T+2) \rfloor}
\E_{\mathbf y}
\left[
{\rm Regret}\left(
\tilde \pi_{\mathbf y,2^k}, 2^k,\theta
\right)
\right]
\\
&
\leq
\sum_{k=1}^{\lfloor \log (T+2) \rfloor}
\bar C_{\mathbf y} \log^2 2^k
\\
&
\leq
\bar C_{\mathbf y} C \log^3 T
\end{split}
\]
where the equality follows from the tower property and where $C$ is some absolute constant. 
\end{myproof}

Next, we establish a simple result related to the Gittins index policy that relates the finite horizon performance of the policy to the (discounted) infinite horizon performance. 

\begin{lemma}
\label{lemma:gittinsfiniteinfinite}
For any $\mathbf{\hat y} \in \Yscr$, and horizon $T' \geq 2$, we have
\[
\E
_{\hat{\mathbf y}}
\left[
{\rm Regret}\left(
\pi^{G,1-1/{T'}}, T',\theta
\right)
\right]
\leq
4
\E
_{\hat{\mathbf y}}
\left[
{\rm Regret}\left(
\pi^{G,1-1/{T'}}, H_{T'},\theta
\right)
\right]
\]
where $H_{T'}$ is an independent Geometrically distributed random variable with mean $T'$. 
\end{lemma}
\begin{myproof}[Proof.]
We have
\[
\begin{split}
\E_{\hat{\mathbf y}}\left[\Regret{\pi^{G, 1-1/{T'}}, T', \theta}\right]  
& = 
\E_{\hat{\mathbf y}}\left[\Regret{\pi^{G, 1-1/{T'}}, T', \theta}\right]\frac{\P{H_{T'} > T'}}{(1 - 1/{T'})^{T'}}
\\
& \leq
\E_{\hat{\mathbf y}}\left[\Regret{\pi^{G, 1-1/{T'}}, H_{T'}, \theta} \given H_{T'} > T'\right]\frac{\P{H_{T'} > T'} }{(1 - 1/{T'})^{T'}}
\\ 
& \leq
{(1 - 1/{T'})^{-T'}} \E_{\hat{\mathbf y}}\left[\Regret{\pi^{G, 1-1/{T'}}, H_{T'}, \theta}\right]
\\
& \leq
4 \E_{\hat{\mathbf y}}\left[\Regret{\pi^{G, 1-1/{T'}}, H_{T'}, \theta}\right]
\end{split}
\]
where the first inequality follows from the fact that $\E_{\hat{\mathbf y}}\left[\Regret{\pi^{G, 1-1/{T'}}, n, \theta}\right]$ is non-decreasing in $n$, the second inequality follows from the fact that regret is non-negative. 
\end{myproof}

We can now proceed with the proof of the proposition. First, we note that since the Gittins policy with discount factor $1-1/{T'}$ is optimal for a geometrically distributed horizon with mean $T'$, we must have for any $\mathbf{\hat y} \in \Yscr$, 
\begin{equation}
\label{eq:gittinsopt}
\E_{\hat{\mathbf y}}\left[\Regret{\pi^{G, 1-1/{T'}}, H_{T'}, \theta}\right]
\leq
\E_{\hat{\mathbf y}}\left[\Regret{\tilde \pi_{\mathbf{y}}, H_{T'}, \theta}\right]
\end{equation}
But we have
\[
\begin{split}
\E_{{\mathbf y}}\left[\Regret{\pi^{D}, T, \theta}\right]
&\leq
\sum_{k=1}^{\lfloor \log (T+2) \rfloor}
\E_{{\mathbf y}}\left[
\E_{\hat{\mathbf y}_{2^k-2}}\left[
\Regret{\pi^{G,1-1/2^k}, 2^k, \theta}\right]
\right]
\\
&\leq
\sum_{k=1}^{\lfloor \log (T+2) \rfloor}
\E_{{\mathbf y}}\left[
4
\E_{\hat{\mathbf y}_{2^k-2}}\left[
\Regret{\pi^{G,1-1/2^k}, H_{2^k}, \theta}\right]
\right]
\\
&\leq
\sum_{k=1}^{\lfloor \log (T+2) \rfloor}
\E_{{\mathbf y}}\left[
4
\E_{\hat{\mathbf y}_{2^k-2}}\left[
\Regret{\tilde \pi_{\mathbf{y}}, H_{2^k}, \theta}
\right]
\right]
\\
&
=
4\sum_{k=1}^{\lfloor \log (T+2) \rfloor}
\E_{{\mathbf y}}\left[
\Regret{\tilde \pi_{\mathbf{y}}, H_{2^k}, \theta}
\right]
\\
&
\leq 
4 C_{\mathbf y}\sum_{k=1}^{\lfloor \log (T+2) \rfloor}
\E \left[ \log^3 H_{2^k} \right]
\\
&
\leq 
4 C_{\mathbf y}\sum_{k=1}^{\lfloor \log (T+2) \rfloor}
k^3
\end{split}
\]
where the first inequality follows simply from the definition of $\pi^D$, the second inequality follows from Lemma~\ref{lemma:gittinsfiniteinfinite}, the third inequality follows from the aforementioned optimality of the Gittins policy (namely, \eqref{eq:gittinsopt}), the first equality follows from the tower property, the fourth inequality follows from Lemma~\ref{lemma:laianytime}, and the fifth and final inequality is simply Jensen's inequality. 
}



%\begin{myproof}[Proof.]
%	%First, letting $\gamma_n = 1 - 1/n$, we show that
%	%\begin{equation} \label{eq:basic_finite_time_gittins}
%	%\Regret{\pi^{G, \gamma_n}, n} = O\left( \log^2(n) \right).
%	%\end{equation}
%	We start by proving a regret bound for the Gittins index policy over a finite horizon.
%	Since we are dealing with Bayesian regret, we will prove our bound by comparing against the Information Directed Sampling (IDS) policy, denoted by $\pi^{IDS}$. It was shown in \citep{russo2014learning} (Proposition 2) that for any horizon $T \in \mathbb N$ and any prior, described by a vector $\hat{\mathbf y}$, that
%	\[
%	\E_{\hat{\mathbf y}}\left[ \Regret{\pi^{IDS}, T, \theta} \right] 
%	 \le  \sqrt{\frac{1}{2} (A \log A) T}.
%	\]
%	whenever the reward distribution is uniformly bounded.
%	We will leverage this useful result in the next step of the proof.
%	
%	Fix any $n \ge 2$ and sufficient statistic $\hat{\mathbf y}$. Define $H$ to be a geometric random variable with mean $n$, then	
%	\begin{align}
%	\E_{\hat{\mathbf y}}\left[\Regret{\pi^{G, \gamma_n}, n, \theta}\right]  & = \E_{\hat{\mathbf y}}\left[\Regret{\pi^{G, \gamma_n}, n, \theta}\right]\frac{\P{H > n}}{(1 - 1/n)^n} \nonumber \\
%	&  \le  \E_{\hat{\mathbf y}}\left[\Regret{\pi^{G, \gamma_n}, H, \theta} \given H > n\right]\frac{\P{H > n} }{(1 - 1/n)^n} \nonumber\\
%	& \le  (1 - 1/n)^{-n}  \E_{\hat{\mathbf y}}\left[ \Regret{\pi^{G,\gamma_n}, H, \theta} \right] \nonumber   \\
%	& \le  4 \E_{\hat{\mathbf y}}\left[ \Regret{\pi^{G,\gamma_n}, H, \theta} \right]   \nonumber \\
%	& \le 4\E_{\hat{\mathbf y}}\left[ \Regret{\pi^{IDS}, H, \theta} \right] \label{eqn:opt_gittins_over_geo_horizon} \\
%	& \le 4 \E\left[  \sqrt{\frac{1}{2} (A \log A) H} \right] \nonumber\\
%	& \le  \E\left[  \sqrt{8 (A \log A) H} \right] \nonumber \\
%	& \le  \sqrt{8 (A \log A) \E[H]}  \label{eqn:jensens_ids} \\
%	& =  \sqrt{8 (A \log A) n} \label{eqn:regret_bound_for_ids}   
%	\end{align}
%	where \eqref{eqn:opt_gittins_over_geo_horizon} follows from the Gittins index policy being optimal over the random \& geometrically distributed horizon of $H$ (because $\E[H] = n$ and the discount factor is assumed to be $1 - 1/n$ here). Equation~\eqref{eqn:jensens_ids} follows from Jensen's inequality.
%	
%	Back to the doubling trick policy. Recall that this entails pulling the arms according to $\pi^{G,1 - 1/2^{k-1}}$ during periods $\{2^{k-1},\ldots,2^{k}-1\}$ for each $k \in \mathbb{N}$.
%	For convenience, we define the function $\text{Regret}(\pi, t_1, t_2, \theta)$ to be the total regret accumulated from periods $t_1$ up to $t_2$ under policy $\pi$ and where the arms' parameters are set to $\theta$. We will also denote the tuple of sufficient statistics for the arms in period $t$ as $\mathbf y_t$, to reflect all the Bayesian updates up to and inlcuding that time.
%	
%	Now we can bound the Bayes risk up to time $T$ as follows:
%	\begin{align}
%		\E_{\mathbf y}\left[\Regret{\pi^D, T, \theta} \right] &
%		\leq 1 + \sum_{k=2}^{\ceil{\log_2 T}} \E_{\mathbf y}\left[\Regret{\pi^D, 2^{k-1}, 2^k - 1, \theta} \right] \nonumber \\
%		& = 1+ \sum_{k=2}^{\ceil{\log_2 T}} \E_{\mathbf y}\left[ \E_{\mathbf y_{2^k-1}}\left[\Regret{\pi^{G, 1 - 1/2^{k-1}}, 2^{k-1}, \theta} \right] \right] \nonumber \\
%		& \le 1 + \sum_{k=2}^{\ceil{\log_2 T}}  \sqrt{8(A \log A) 2^{k-1}} \label{eqn:application_of_ids}\\
%		& = \mathcal O\left( \sqrt{T} \right), \nonumber
%	\end{align}
%	where inequality \eqref{eqn:application_of_ids} is implied by  \eqref{eqn:regret_bound_for_ids}.
%\end{myproof}

\section{Properties of the Optimistic Gittins index}\label{sec:appendix_properties_of_ogi}
This section gives proofs for a few properties of the Optimistic Gittins index that are used throughout the paper and particularly in the proof of Theorem~\ref{thm:frequentist_optimal_bound}.  
It shall be useful, in what follows, to define the continuation value for the Vittles's retirement problem (\cite{whittle1980multi}) as
\[
V_\gamma(y, \lambda)  \defeq \sup_{\tau > 0} \E_y\left[\sum_{t=1}^{\tau} \gamma^{t-1} X_{i,t} + \gamma^{\tau} \frac{\lambda}{1-\gamma}\right],
\]
so that the Gittins index is then the solution in $\lambda$ to $\lambda/(1-\gamma) = V_\gamma(y, \lambda)$. In an analogous fashion, we define the optimistic continuation value, for parameters $K$ and $\lambda$, to be
\[
V^K_\gamma(y, \lambda) \defeq \sup_{1 \le \tau \le K} \E_y\left[\sum_{t=1}^{\tau} \gamma^{t-1}  X_{i,t} + \gamma^{\tau} \frac{R_{\lambda, K}(\tau, y_{i,\tau-1})}{1-\gamma}\right].
\]
From this definition, it follows that the solution for $\lambda$ to the equation $\lambda/(1-\gamma) = V^K_\gamma(y, \lambda)$ is the Optimistic Gittins index.

Throughout this section, we will sometimes discuss the value of the index at some particular time $t$ during the execution of the algorithm, which depends on the statistic gathered about the arm using information up to but strictly \emph{not including} time $t$. As such, we will define the number of pulls of arm $i$ up to time $t-1$ as
\[
\Ntg{i}{t} \defeq N_i(t-1)
\]
where we recall $N_i(t)$ is the counter for the number of total number of pulls up to and including $t$. From the $\Ntg{i}{t}$ pulls of the arm, the total reward accumulated is defined as
\[
S_i(t) \defeq \sum_{s=1}^{\Ntg{i}{t}} X_{i,s}.
\]

We begin by investigating the effect of the parameter $\lambda$, which gives the deterministic payoff in \eqref{eqn:ogi_general}, on the continuation value $V^K_\gamma(y, \lambda)$ and use that to find out how close an approximation $v^K_\gamma(y)$ is to the Gittins index.
\begin{fact}\label{fact:v_is_convex}
	For any state $y \in \mathcal{Y}$, discount factor $\gamma$ and parameter $K$, the function $V^K_\gamma(y,\lambda)$ is convex in $\lambda$.
	{Moreover, the function $V^K_\gamma(y,\lambda)$ is Lipschitz continuous in $\lambda$ with a Lipshitz constant of $\gamma/(1-\gamma)$.}
	% Moreover, if $R(y)$ is a continuous random variable, the function is also differentiable.
\end{fact}
\begin{myproof}[Proof.]
	Fix an arbitrary state $y$ and discount factor $\gamma \in (0,1)$. Our proof is by induction on the parameter $K$. For $K = 1$, recall from Section~\ref{sec:gittins_and_approx} that
	\begin{align*}
	{V^1_\gamma(y, \lambda) = \E_y\left[X_{i,1} \right] +  \frac{\gamma}{1-\gamma} \E_y\left[\max(\lambda, R(y_{i,0}))\right].}
	\end{align*}
	Thus the function is convex because it is an expectation over a convex piecewise linear function of random variables $X_{i,1}$ and $R(y_{i,0})$.
	{To prove Lipschitz continuity, it's enough to note that
	for any $\lambda_1, \lambda_2 \in \R$, that
	\begin{align*}
		| V^1_\gamma(y, \lambda_1) - V^1_\gamma(y, \lambda_2)| & =  \frac{\gamma}{1-\gamma} |\E_y[\max(\lambda_1, R(y_{i,0})) - \max(\lambda_2, R(y_{i,0}))  ]  | \\
	& \leq \frac{\gamma}{1-\gamma} |\lambda_1 - \lambda_2 |	.
	\end{align*}
		
	}
	
	%Now assume that $R(y)$ is a continuous random variable. We will verify through the bounded convergence theorem that $V^1_\gamma(y, \lambda)$ is differentiable. Indeed, this holds because the event $\{ R(y)= \lambda/(1-\gamma)\}$, at which the random variable inside the expectation is not differentiable, has measure zero, precisely because $R(y)$ is a continuous random variable.
	Now we prove the inductive step. For any $K > 1$, assume that $V^{K-1}_\gamma(y, \lambda)$ is convex and Lipshitz continuous. By writing the Bellman equation,
	{
	\begin{align*}
	V^K_\gamma(y, \lambda) & = \E_y\left[X_{i,1}\right] + \gamma\E_y\left[\max\left(\lambda, V^{K-1}_\gamma(y_{i,1}, \lambda)\right) \right],
	\end{align*}
	}
	we again notice an expectation over a maximum of convex functions in $\lambda$. This form for $V^K_\gamma(y, \lambda)$ implies that  it is also convex in $\lambda$. % If $V^{K-1}$ is differentiable and $R(y)$ is a continuous random variable, then the form of $V^{K-1}$ also implies that it is differentiable in $\lambda$.
	{Finally, to prove Lipschitz continuity, we will use the fact that the pointwise maximum of two Lipschitz functions, having respective constants $L_1$ and $L_2$, is also Lipshitz with a constant of $\max(L_1, L_2)$. Because of this, by fixing $\lambda_1, \lambda_2 \in \R$, we have that
		\begin{align*}
		| V^K_\gamma(y, \lambda_1) - V^K_\gamma(y, \lambda_2)| & =  \gamma \left|\E_y[\max(\lambda_1, V^{K-1}_\gamma(y_{i,1}, \lambda_1)) - \max(\lambda_2, V^{K-1}_\gamma(y_{i,1}, \lambda_2))  ]  \right| \\
		& \leq \frac{\gamma^2}{1-\gamma} |\lambda_1 - \lambda_2| \\
		& \leq \frac{\gamma}{1-\gamma} |\lambda_1 - \lambda_2|
		\end{align*}
	where the second-to-last inequality follows from the induction hypothesis that $V^{K-1}_\gamma(y_{i,1}, \lambda))$ is Lipschitz continuous in $\lambda$ with a constant of $\gamma/(1-\gamma)$, and also from the fact that the identity function for $\lambda$ (within the maximum expression) is trivially Lipschitz continuous.
	}
\end{myproof}

\begin{lemma} \label{cor:equivalent_event}
	Suppose that arm rewards are bounded. That is, there exists a constant $B \in \Re_+$ such that $X_{i,t} \in [0, B]$ for every arm $i$ and time $t$. {Further assume that $X_{i,t}$ is not almost surely equal to $B$}.
	
	Let $v^K_{i,t}$ be the Optimistic Gittins Index of arm $i$ at time $t$ and let $\eta$ be a scalar, then the following equivalence holds
	\[
	\{v^K_{i,t} < \eta \} = \{(1-\gamma_t)V^K_{\gamma_t}(y_{i,\Ntg{i}{t}}, \eta) < \eta\}\]
	where {$y_{i,\Ntg{i}{t}}$} is the sufficient statistic for estimating the $i$th arm's parameter $\theta_i$ at time $t$.
\end{lemma}
{
\begin{myproof}[Proof.]
	Fix any state $y$ and discount factor $\gamma$. At $\lambda = 0$, we have
	\[
		(1-\gamma)V^K_{\gamma}(y, 0) \ge 0
	\]
	because $V^K_{\gamma}(0,y)$ is the expectation of a sum of nonnegative terms. Also, in the other extreme case when $\lambda = B$, the function in question evaluates to
	\[
		V^K_{\gamma}(y, B) = \E_y\left[X_{i,1}\right] + \frac{\gamma B}{(1-\gamma)} < \frac{B}{(1-\gamma)}.
	\]
	and so $(1-\gamma)V^{K}_\gamma(y, B) < B$.
	
	Next we prove that $V^K_\gamma(y, \lambda)$ is monotonically increasing in $\lambda$. To show this, pick any $\lambda' < \lambda''$ and let $\tau^*(\lambda')$ and $\tau^*(\lambda'')$ denote the two optimal stopping times under $\lambda', \lambda''$, respectively. It then follows, from $R_{\lambda,\tau}(.)$ being an increasing function of $\lambda$ on every sample path, that
	\begin{align*}
	V^K_{\gamma}(y, \lambda') & = \E_y\left[\sum_{t=1}^{\tau^*(\lambda')} \gamma^{t-1}  X_{i,t} + \gamma^{\tau^*(\lambda')} \frac{R_{\lambda', K}(\tau, y_{i,\tau^*(\lambda')-1})}{1-\gamma}\right] \\
	& \leq \E_y\left[\sum_{t=1}^{\tau^*(\lambda')} \gamma^{t-1}  X_{i,t} + \gamma^{\tau^*(\lambda')} \frac{R_{\lambda'', K}(\tau, y_{i,\tau^*(\lambda')-1})}{1-\gamma}\right] \\
	& \leq \E_y\left[\sum_{t=1}^{\tau^*(\lambda'')} \gamma^{t-1}  X_{i,t} + \gamma^{\tau^*(\lambda'')} \frac{R_{\lambda'', K}(\tau, y_{i,\tau^*(\lambda'')-1})}{1-\gamma}\right] \\
	& = V^K_{\gamma}(y, \lambda'').
	\end{align*}
	Let's put together these observations:
	\begin{itemize}
		\item $(1-\gamma)V^{K}_\gamma(y, \lambda) \ge \lambda $ at $\lambda = 0$.
		\item   $(1-\gamma)V^{K}_\gamma(y, \lambda) < \lambda $ at $\lambda = B$
		\item $(1-\gamma)V^{K}_\gamma(y, \lambda)$ is monotonically increasing in $\lambda$.
	\end{itemize}
	These together with Fact~\ref{fact:v_is_convex} show that the univariate function  $(1-\gamma)V^K_\gamma(y, \lambda) - \lambda$ is continuous and decreasing in $\lambda$. Moreover, this function is non-negative for any $\lambda \le v^K_\gamma(y)$ (since $v^K_\gamma(y)$ is the root of the function) and is also negative for $\lambda > v^K_\gamma(y)$.
	\begin{figure}
		\centering
		%% Creator: Matplotlib, PGF backend
%%
%% To include the figure in your LaTeX document, write
%%   \input{<filename>.pgf}
%%
%% Make sure the required packages are loaded in your preamble
%%   \usepackage{pgf}
%%
%% Figures using additional raster images can only be included by \input if
%% they are in the same directory as the main LaTeX file. For loading figures
%% from other directories you can use the `import` package
%%   \usepackage{import}
%% and then include the figures with
%%   \import{<path to file>}{<filename>.pgf}
%%
%% Matplotlib used the following preamble
%%   \usepackage[utf8x]{inputenc}
%%   \usepackage[T1]{fontenc}
%%
\begingroup%
\makeatletter%
\begin{pgfpicture}%
\pgfpathrectangle{\pgfpointorigin}{\pgfqpoint{3.900000in}{2.410333in}}%
\pgfusepath{use as bounding box, clip}%
\begin{pgfscope}%
\pgfsetbuttcap%
\pgfsetmiterjoin%
\definecolor{currentfill}{rgb}{1.000000,1.000000,1.000000}%
\pgfsetfillcolor{currentfill}%
\pgfsetlinewidth{0.000000pt}%
\definecolor{currentstroke}{rgb}{1.000000,1.000000,1.000000}%
\pgfsetstrokecolor{currentstroke}%
\pgfsetdash{}{0pt}%
\pgfpathmoveto{\pgfqpoint{0.000000in}{0.000000in}}%
\pgfpathlineto{\pgfqpoint{3.900000in}{0.000000in}}%
\pgfpathlineto{\pgfqpoint{3.900000in}{2.410333in}}%
\pgfpathlineto{\pgfqpoint{0.000000in}{2.410333in}}%
\pgfpathclose%
\pgfusepath{fill}%
\end{pgfscope}%
\begin{pgfscope}%
\pgfsetbuttcap%
\pgfsetmiterjoin%
\definecolor{currentfill}{rgb}{1.000000,1.000000,1.000000}%
\pgfsetfillcolor{currentfill}%
\pgfsetlinewidth{0.000000pt}%
\definecolor{currentstroke}{rgb}{0.000000,0.000000,0.000000}%
\pgfsetstrokecolor{currentstroke}%
\pgfsetstrokeopacity{0.000000}%
\pgfsetdash{}{0pt}%
\pgfpathmoveto{\pgfqpoint{0.487500in}{0.301292in}}%
\pgfpathlineto{\pgfqpoint{3.510000in}{0.301292in}}%
\pgfpathlineto{\pgfqpoint{3.510000in}{2.169299in}}%
\pgfpathlineto{\pgfqpoint{0.487500in}{2.169299in}}%
\pgfpathclose%
\pgfusepath{fill}%
\end{pgfscope}%
\begin{pgfscope}%
\pgfpathrectangle{\pgfqpoint{0.487500in}{0.301292in}}{\pgfqpoint{3.022500in}{1.868008in}} %
\pgfusepath{clip}%
\pgfsetrectcap%
\pgfsetroundjoin%
\pgfsetlinewidth{1.003750pt}%
\definecolor{currentstroke}{rgb}{0.000000,0.000000,1.000000}%
\pgfsetstrokecolor{currentstroke}%
\pgfsetdash{}{0pt}%
\pgfpathmoveto{\pgfqpoint{0.487500in}{0.835008in}}%
\pgfpathlineto{\pgfqpoint{0.517725in}{0.835008in}}%
\pgfpathlineto{\pgfqpoint{0.547950in}{0.835009in}}%
\pgfpathlineto{\pgfqpoint{0.578175in}{0.835013in}}%
\pgfpathlineto{\pgfqpoint{0.608400in}{0.835028in}}%
\pgfpathlineto{\pgfqpoint{0.638625in}{0.835067in}}%
\pgfpathlineto{\pgfqpoint{0.668850in}{0.835145in}}%
\pgfpathlineto{\pgfqpoint{0.699075in}{0.835287in}}%
\pgfpathlineto{\pgfqpoint{0.729300in}{0.835520in}}%
\pgfpathlineto{\pgfqpoint{0.759525in}{0.835876in}}%
\pgfpathlineto{\pgfqpoint{0.789750in}{0.836390in}}%
\pgfpathlineto{\pgfqpoint{0.819975in}{0.837101in}}%
\pgfpathlineto{\pgfqpoint{0.850200in}{0.838049in}}%
\pgfpathlineto{\pgfqpoint{0.880425in}{0.839275in}}%
\pgfpathlineto{\pgfqpoint{0.910650in}{0.840819in}}%
\pgfpathlineto{\pgfqpoint{0.940875in}{0.842721in}}%
\pgfpathlineto{\pgfqpoint{0.971100in}{0.845019in}}%
\pgfpathlineto{\pgfqpoint{1.001325in}{0.847750in}}%
\pgfpathlineto{\pgfqpoint{1.031550in}{0.850947in}}%
\pgfpathlineto{\pgfqpoint{1.061775in}{0.854640in}}%
\pgfpathlineto{\pgfqpoint{1.092000in}{0.858855in}}%
\pgfpathlineto{\pgfqpoint{1.122225in}{0.863615in}}%
\pgfpathlineto{\pgfqpoint{1.152450in}{0.868938in}}%
\pgfpathlineto{\pgfqpoint{1.182675in}{0.874838in}}%
\pgfpathlineto{\pgfqpoint{1.212900in}{0.881325in}}%
\pgfpathlineto{\pgfqpoint{1.243125in}{0.888405in}}%
\pgfpathlineto{\pgfqpoint{1.273350in}{0.896081in}}%
\pgfpathlineto{\pgfqpoint{1.303575in}{0.904349in}}%
\pgfpathlineto{\pgfqpoint{1.333800in}{0.913206in}}%
\pgfpathlineto{\pgfqpoint{1.364025in}{0.922642in}}%
\pgfpathlineto{\pgfqpoint{1.394250in}{0.932645in}}%
\pgfpathlineto{\pgfqpoint{1.424475in}{0.943200in}}%
\pgfpathlineto{\pgfqpoint{1.454700in}{0.954292in}}%
\pgfpathlineto{\pgfqpoint{1.484925in}{0.965900in}}%
\pgfpathlineto{\pgfqpoint{1.515150in}{0.978005in}}%
\pgfpathlineto{\pgfqpoint{1.545375in}{0.990583in}}%
\pgfpathlineto{\pgfqpoint{1.575600in}{1.003613in}}%
\pgfpathlineto{\pgfqpoint{1.605825in}{1.017068in}}%
\pgfpathlineto{\pgfqpoint{1.636050in}{1.030926in}}%
\pgfpathlineto{\pgfqpoint{1.666275in}{1.045160in}}%
\pgfpathlineto{\pgfqpoint{1.696500in}{1.059747in}}%
\pgfpathlineto{\pgfqpoint{1.726725in}{1.074661in}}%
\pgfpathlineto{\pgfqpoint{1.756950in}{1.089878in}}%
\pgfpathlineto{\pgfqpoint{1.787175in}{1.105375in}}%
\pgfpathlineto{\pgfqpoint{1.817400in}{1.121128in}}%
\pgfpathlineto{\pgfqpoint{1.847625in}{1.137115in}}%
\pgfpathlineto{\pgfqpoint{1.877850in}{1.153315in}}%
\pgfpathlineto{\pgfqpoint{1.908075in}{1.169708in}}%
\pgfpathlineto{\pgfqpoint{1.938300in}{1.186275in}}%
\pgfpathlineto{\pgfqpoint{1.968525in}{1.202998in}}%
\pgfpathlineto{\pgfqpoint{1.998750in}{1.219861in}}%
\pgfpathlineto{\pgfqpoint{2.028975in}{1.236848in}}%
\pgfpathlineto{\pgfqpoint{2.059200in}{1.253944in}}%
\pgfpathlineto{\pgfqpoint{2.089425in}{1.271138in}}%
\pgfpathlineto{\pgfqpoint{2.119650in}{1.288415in}}%
\pgfpathlineto{\pgfqpoint{2.149875in}{1.305767in}}%
\pgfpathlineto{\pgfqpoint{2.180100in}{1.323182in}}%
\pgfpathlineto{\pgfqpoint{2.210325in}{1.340653in}}%
\pgfpathlineto{\pgfqpoint{2.240550in}{1.358171in}}%
\pgfpathlineto{\pgfqpoint{2.270775in}{1.375729in}}%
\pgfpathlineto{\pgfqpoint{2.301000in}{1.393322in}}%
\pgfpathlineto{\pgfqpoint{2.331225in}{1.410943in}}%
\pgfpathlineto{\pgfqpoint{2.361450in}{1.428589in}}%
\pgfpathlineto{\pgfqpoint{2.391675in}{1.446254in}}%
\pgfpathlineto{\pgfqpoint{2.421900in}{1.463936in}}%
\pgfpathlineto{\pgfqpoint{2.452125in}{1.481632in}}%
\pgfpathlineto{\pgfqpoint{2.482350in}{1.499338in}}%
\pgfpathlineto{\pgfqpoint{2.512575in}{1.517054in}}%
\pgfpathlineto{\pgfqpoint{2.542800in}{1.534776in}}%
\pgfpathlineto{\pgfqpoint{2.573025in}{1.552505in}}%
\pgfpathlineto{\pgfqpoint{2.603250in}{1.570237in}}%
\pgfpathlineto{\pgfqpoint{2.633475in}{1.587973in}}%
\pgfpathlineto{\pgfqpoint{2.663700in}{1.605712in}}%
\pgfpathlineto{\pgfqpoint{2.693925in}{1.623453in}}%
\pgfpathlineto{\pgfqpoint{2.724150in}{1.641195in}}%
\pgfpathlineto{\pgfqpoint{2.754375in}{1.658938in}}%
\pgfpathlineto{\pgfqpoint{2.784600in}{1.676683in}}%
\pgfpathlineto{\pgfqpoint{2.814825in}{1.694427in}}%
\pgfpathlineto{\pgfqpoint{2.845050in}{1.712173in}}%
\pgfpathlineto{\pgfqpoint{2.875275in}{1.729918in}}%
\pgfpathlineto{\pgfqpoint{2.905500in}{1.747664in}}%
\pgfpathlineto{\pgfqpoint{2.935725in}{1.765410in}}%
\pgfpathlineto{\pgfqpoint{2.965950in}{1.783156in}}%
\pgfpathlineto{\pgfqpoint{2.996175in}{1.800902in}}%
\pgfpathlineto{\pgfqpoint{3.026400in}{1.818648in}}%
\pgfpathlineto{\pgfqpoint{3.056625in}{1.836394in}}%
\pgfpathlineto{\pgfqpoint{3.086850in}{1.854140in}}%
\pgfpathlineto{\pgfqpoint{3.117075in}{1.871886in}}%
\pgfpathlineto{\pgfqpoint{3.147300in}{1.889632in}}%
\pgfpathlineto{\pgfqpoint{3.177525in}{1.907378in}}%
\pgfpathlineto{\pgfqpoint{3.207750in}{1.925124in}}%
\pgfpathlineto{\pgfqpoint{3.237975in}{1.942870in}}%
\pgfpathlineto{\pgfqpoint{3.268200in}{1.960616in}}%
\pgfpathlineto{\pgfqpoint{3.298425in}{1.978362in}}%
\pgfpathlineto{\pgfqpoint{3.328650in}{1.996108in}}%
\pgfpathlineto{\pgfqpoint{3.358875in}{2.013854in}}%
\pgfpathlineto{\pgfqpoint{3.389100in}{2.031600in}}%
\pgfpathlineto{\pgfqpoint{3.419325in}{2.049347in}}%
\pgfpathlineto{\pgfqpoint{3.449550in}{2.067093in}}%
\pgfpathlineto{\pgfqpoint{3.479775in}{2.084839in}}%
\pgfpathlineto{\pgfqpoint{3.510000in}{2.102585in}}%
\pgfusepath{stroke}%
\end{pgfscope}%
\begin{pgfscope}%
\pgfpathrectangle{\pgfqpoint{0.487500in}{0.301292in}}{\pgfqpoint{3.022500in}{1.868008in}} %
\pgfusepath{clip}%
\pgfsetbuttcap%
\pgfsetroundjoin%
\pgfsetlinewidth{1.003750pt}%
\definecolor{currentstroke}{rgb}{1.000000,0.000000,0.000000}%
\pgfsetstrokecolor{currentstroke}%
\pgfsetdash{{6.000000pt}{6.000000pt}}{0.000000pt}%
\pgfpathmoveto{\pgfqpoint{0.487500in}{0.301292in}}%
\pgfpathlineto{\pgfqpoint{0.517725in}{0.319972in}}%
\pgfpathlineto{\pgfqpoint{0.547950in}{0.338652in}}%
\pgfpathlineto{\pgfqpoint{0.578175in}{0.357332in}}%
\pgfpathlineto{\pgfqpoint{0.608400in}{0.376012in}}%
\pgfpathlineto{\pgfqpoint{0.638625in}{0.394692in}}%
\pgfpathlineto{\pgfqpoint{0.668850in}{0.413372in}}%
\pgfpathlineto{\pgfqpoint{0.699075in}{0.432052in}}%
\pgfpathlineto{\pgfqpoint{0.729300in}{0.450732in}}%
\pgfpathlineto{\pgfqpoint{0.759525in}{0.469412in}}%
\pgfpathlineto{\pgfqpoint{0.789750in}{0.488092in}}%
\pgfpathlineto{\pgfqpoint{0.819975in}{0.506772in}}%
\pgfpathlineto{\pgfqpoint{0.850200in}{0.525452in}}%
\pgfpathlineto{\pgfqpoint{0.880425in}{0.544133in}}%
\pgfpathlineto{\pgfqpoint{0.910650in}{0.562813in}}%
\pgfpathlineto{\pgfqpoint{0.940875in}{0.581493in}}%
\pgfpathlineto{\pgfqpoint{0.971100in}{0.600173in}}%
\pgfpathlineto{\pgfqpoint{1.001325in}{0.618853in}}%
\pgfpathlineto{\pgfqpoint{1.031550in}{0.637533in}}%
\pgfpathlineto{\pgfqpoint{1.061775in}{0.656213in}}%
\pgfpathlineto{\pgfqpoint{1.092000in}{0.674893in}}%
\pgfpathlineto{\pgfqpoint{1.122225in}{0.693573in}}%
\pgfpathlineto{\pgfqpoint{1.152450in}{0.712253in}}%
\pgfpathlineto{\pgfqpoint{1.182675in}{0.730933in}}%
\pgfpathlineto{\pgfqpoint{1.212900in}{0.749613in}}%
\pgfpathlineto{\pgfqpoint{1.243125in}{0.768294in}}%
\pgfpathlineto{\pgfqpoint{1.273350in}{0.786974in}}%
\pgfpathlineto{\pgfqpoint{1.303575in}{0.805654in}}%
\pgfpathlineto{\pgfqpoint{1.333800in}{0.824334in}}%
\pgfpathlineto{\pgfqpoint{1.364025in}{0.843014in}}%
\pgfpathlineto{\pgfqpoint{1.394250in}{0.861694in}}%
\pgfpathlineto{\pgfqpoint{1.424475in}{0.880374in}}%
\pgfpathlineto{\pgfqpoint{1.454700in}{0.899054in}}%
\pgfpathlineto{\pgfqpoint{1.484925in}{0.917734in}}%
\pgfpathlineto{\pgfqpoint{1.515150in}{0.936414in}}%
\pgfpathlineto{\pgfqpoint{1.545375in}{0.955094in}}%
\pgfpathlineto{\pgfqpoint{1.575600in}{0.973774in}}%
\pgfpathlineto{\pgfqpoint{1.605825in}{0.992454in}}%
\pgfpathlineto{\pgfqpoint{1.636050in}{1.011135in}}%
\pgfpathlineto{\pgfqpoint{1.666275in}{1.029815in}}%
\pgfpathlineto{\pgfqpoint{1.696500in}{1.048495in}}%
\pgfpathlineto{\pgfqpoint{1.726725in}{1.067175in}}%
\pgfpathlineto{\pgfqpoint{1.756950in}{1.085855in}}%
\pgfpathlineto{\pgfqpoint{1.787175in}{1.104535in}}%
\pgfpathlineto{\pgfqpoint{1.817400in}{1.123215in}}%
\pgfpathlineto{\pgfqpoint{1.847625in}{1.141895in}}%
\pgfpathlineto{\pgfqpoint{1.877850in}{1.160575in}}%
\pgfpathlineto{\pgfqpoint{1.908075in}{1.179255in}}%
\pgfpathlineto{\pgfqpoint{1.938300in}{1.197935in}}%
\pgfpathlineto{\pgfqpoint{1.968525in}{1.216615in}}%
\pgfpathlineto{\pgfqpoint{1.998750in}{1.235295in}}%
\pgfpathlineto{\pgfqpoint{2.028975in}{1.253976in}}%
\pgfpathlineto{\pgfqpoint{2.059200in}{1.272656in}}%
\pgfpathlineto{\pgfqpoint{2.089425in}{1.291336in}}%
\pgfpathlineto{\pgfqpoint{2.119650in}{1.310016in}}%
\pgfpathlineto{\pgfqpoint{2.149875in}{1.328696in}}%
\pgfpathlineto{\pgfqpoint{2.180100in}{1.347376in}}%
\pgfpathlineto{\pgfqpoint{2.210325in}{1.366056in}}%
\pgfpathlineto{\pgfqpoint{2.240550in}{1.384736in}}%
\pgfpathlineto{\pgfqpoint{2.270775in}{1.403416in}}%
\pgfpathlineto{\pgfqpoint{2.301000in}{1.422096in}}%
\pgfpathlineto{\pgfqpoint{2.331225in}{1.440776in}}%
\pgfpathlineto{\pgfqpoint{2.361450in}{1.459456in}}%
\pgfpathlineto{\pgfqpoint{2.391675in}{1.478136in}}%
\pgfpathlineto{\pgfqpoint{2.421900in}{1.496817in}}%
\pgfpathlineto{\pgfqpoint{2.452125in}{1.515497in}}%
\pgfpathlineto{\pgfqpoint{2.482350in}{1.534177in}}%
\pgfpathlineto{\pgfqpoint{2.512575in}{1.552857in}}%
\pgfpathlineto{\pgfqpoint{2.542800in}{1.571537in}}%
\pgfpathlineto{\pgfqpoint{2.573025in}{1.590217in}}%
\pgfpathlineto{\pgfqpoint{2.603250in}{1.608897in}}%
\pgfpathlineto{\pgfqpoint{2.633475in}{1.627577in}}%
\pgfpathlineto{\pgfqpoint{2.663700in}{1.646257in}}%
\pgfpathlineto{\pgfqpoint{2.693925in}{1.664937in}}%
\pgfpathlineto{\pgfqpoint{2.724150in}{1.683617in}}%
\pgfpathlineto{\pgfqpoint{2.754375in}{1.702297in}}%
\pgfpathlineto{\pgfqpoint{2.784600in}{1.720977in}}%
\pgfpathlineto{\pgfqpoint{2.814825in}{1.739658in}}%
\pgfpathlineto{\pgfqpoint{2.845050in}{1.758338in}}%
\pgfpathlineto{\pgfqpoint{2.875275in}{1.777018in}}%
\pgfpathlineto{\pgfqpoint{2.905500in}{1.795698in}}%
\pgfpathlineto{\pgfqpoint{2.935725in}{1.814378in}}%
\pgfpathlineto{\pgfqpoint{2.965950in}{1.833058in}}%
\pgfpathlineto{\pgfqpoint{2.996175in}{1.851738in}}%
\pgfpathlineto{\pgfqpoint{3.026400in}{1.870418in}}%
\pgfpathlineto{\pgfqpoint{3.056625in}{1.889098in}}%
\pgfpathlineto{\pgfqpoint{3.086850in}{1.907778in}}%
\pgfpathlineto{\pgfqpoint{3.117075in}{1.926458in}}%
\pgfpathlineto{\pgfqpoint{3.147300in}{1.945138in}}%
\pgfpathlineto{\pgfqpoint{3.177525in}{1.963818in}}%
\pgfpathlineto{\pgfqpoint{3.207750in}{1.982499in}}%
\pgfpathlineto{\pgfqpoint{3.237975in}{2.001179in}}%
\pgfpathlineto{\pgfqpoint{3.268200in}{2.019859in}}%
\pgfpathlineto{\pgfqpoint{3.298425in}{2.038539in}}%
\pgfpathlineto{\pgfqpoint{3.328650in}{2.057219in}}%
\pgfpathlineto{\pgfqpoint{3.358875in}{2.075899in}}%
\pgfpathlineto{\pgfqpoint{3.389100in}{2.094579in}}%
\pgfpathlineto{\pgfqpoint{3.419325in}{2.113259in}}%
\pgfpathlineto{\pgfqpoint{3.449550in}{2.131939in}}%
\pgfpathlineto{\pgfqpoint{3.479775in}{2.150619in}}%
\pgfpathlineto{\pgfqpoint{3.510000in}{2.169299in}}%
\pgfusepath{stroke}%
\end{pgfscope}%
\begin{pgfscope}%
\pgfsetrectcap%
\pgfsetmiterjoin%
\pgfsetlinewidth{1.003750pt}%
\definecolor{currentstroke}{rgb}{0.000000,0.000000,0.000000}%
\pgfsetstrokecolor{currentstroke}%
\pgfsetdash{}{0pt}%
\pgfpathmoveto{\pgfqpoint{0.487500in}{2.169299in}}%
\pgfpathlineto{\pgfqpoint{3.510000in}{2.169299in}}%
\pgfusepath{stroke}%
\end{pgfscope}%
\begin{pgfscope}%
\pgfsetrectcap%
\pgfsetmiterjoin%
\pgfsetlinewidth{1.003750pt}%
\definecolor{currentstroke}{rgb}{0.000000,0.000000,0.000000}%
\pgfsetstrokecolor{currentstroke}%
\pgfsetdash{}{0pt}%
\pgfpathmoveto{\pgfqpoint{3.510000in}{0.301292in}}%
\pgfpathlineto{\pgfqpoint{3.510000in}{2.169299in}}%
\pgfusepath{stroke}%
\end{pgfscope}%
\begin{pgfscope}%
\pgfsetrectcap%
\pgfsetmiterjoin%
\pgfsetlinewidth{1.003750pt}%
\definecolor{currentstroke}{rgb}{0.000000,0.000000,0.000000}%
\pgfsetstrokecolor{currentstroke}%
\pgfsetdash{}{0pt}%
\pgfpathmoveto{\pgfqpoint{0.487500in}{0.301292in}}%
\pgfpathlineto{\pgfqpoint{3.510000in}{0.301292in}}%
\pgfusepath{stroke}%
\end{pgfscope}%
\begin{pgfscope}%
\pgfsetrectcap%
\pgfsetmiterjoin%
\pgfsetlinewidth{1.003750pt}%
\definecolor{currentstroke}{rgb}{0.000000,0.000000,0.000000}%
\pgfsetstrokecolor{currentstroke}%
\pgfsetdash{}{0pt}%
\pgfpathmoveto{\pgfqpoint{0.487500in}{0.301292in}}%
\pgfpathlineto{\pgfqpoint{0.487500in}{2.169299in}}%
\pgfusepath{stroke}%
\end{pgfscope}%
\begin{pgfscope}%
\pgfsetbuttcap%
\pgfsetroundjoin%
\definecolor{currentfill}{rgb}{0.000000,0.000000,0.000000}%
\pgfsetfillcolor{currentfill}%
\pgfsetlinewidth{0.501875pt}%
\definecolor{currentstroke}{rgb}{0.000000,0.000000,0.000000}%
\pgfsetstrokecolor{currentstroke}%
\pgfsetdash{}{0pt}%
\pgfsys@defobject{currentmarker}{\pgfqpoint{0.000000in}{0.000000in}}{\pgfqpoint{0.000000in}{0.055556in}}{%
\pgfpathmoveto{\pgfqpoint{0.000000in}{0.000000in}}%
\pgfpathlineto{\pgfqpoint{0.000000in}{0.055556in}}%
\pgfusepath{stroke,fill}%
}%
\begin{pgfscope}%
\pgfsys@transformshift{0.487500in}{0.301292in}%
\pgfsys@useobject{currentmarker}{}%
\end{pgfscope}%
\end{pgfscope}%
\begin{pgfscope}%
\pgfsetbuttcap%
\pgfsetroundjoin%
\definecolor{currentfill}{rgb}{0.000000,0.000000,0.000000}%
\pgfsetfillcolor{currentfill}%
\pgfsetlinewidth{0.501875pt}%
\definecolor{currentstroke}{rgb}{0.000000,0.000000,0.000000}%
\pgfsetstrokecolor{currentstroke}%
\pgfsetdash{}{0pt}%
\pgfsys@defobject{currentmarker}{\pgfqpoint{0.000000in}{-0.055556in}}{\pgfqpoint{0.000000in}{0.000000in}}{%
\pgfpathmoveto{\pgfqpoint{0.000000in}{0.000000in}}%
\pgfpathlineto{\pgfqpoint{0.000000in}{-0.055556in}}%
\pgfusepath{stroke,fill}%
}%
\begin{pgfscope}%
\pgfsys@transformshift{0.487500in}{2.169299in}%
\pgfsys@useobject{currentmarker}{}%
\end{pgfscope}%
\end{pgfscope}%
\begin{pgfscope}%
\pgftext[x=0.487500in,y=0.245736in,,top]{\rmfamily\fontsize{8.000000}{9.600000}\selectfont \(\displaystyle 0.0\)}%
\end{pgfscope}%
\begin{pgfscope}%
\pgfsetbuttcap%
\pgfsetroundjoin%
\definecolor{currentfill}{rgb}{0.000000,0.000000,0.000000}%
\pgfsetfillcolor{currentfill}%
\pgfsetlinewidth{0.501875pt}%
\definecolor{currentstroke}{rgb}{0.000000,0.000000,0.000000}%
\pgfsetstrokecolor{currentstroke}%
\pgfsetdash{}{0pt}%
\pgfsys@defobject{currentmarker}{\pgfqpoint{0.000000in}{0.000000in}}{\pgfqpoint{0.000000in}{0.055556in}}{%
\pgfpathmoveto{\pgfqpoint{0.000000in}{0.000000in}}%
\pgfpathlineto{\pgfqpoint{0.000000in}{0.055556in}}%
\pgfusepath{stroke,fill}%
}%
\begin{pgfscope}%
\pgfsys@transformshift{1.092000in}{0.301292in}%
\pgfsys@useobject{currentmarker}{}%
\end{pgfscope}%
\end{pgfscope}%
\begin{pgfscope}%
\pgfsetbuttcap%
\pgfsetroundjoin%
\definecolor{currentfill}{rgb}{0.000000,0.000000,0.000000}%
\pgfsetfillcolor{currentfill}%
\pgfsetlinewidth{0.501875pt}%
\definecolor{currentstroke}{rgb}{0.000000,0.000000,0.000000}%
\pgfsetstrokecolor{currentstroke}%
\pgfsetdash{}{0pt}%
\pgfsys@defobject{currentmarker}{\pgfqpoint{0.000000in}{-0.055556in}}{\pgfqpoint{0.000000in}{0.000000in}}{%
\pgfpathmoveto{\pgfqpoint{0.000000in}{0.000000in}}%
\pgfpathlineto{\pgfqpoint{0.000000in}{-0.055556in}}%
\pgfusepath{stroke,fill}%
}%
\begin{pgfscope}%
\pgfsys@transformshift{1.092000in}{2.169299in}%
\pgfsys@useobject{currentmarker}{}%
\end{pgfscope}%
\end{pgfscope}%
\begin{pgfscope}%
\pgftext[x=1.092000in,y=0.245736in,,top]{\rmfamily\fontsize{8.000000}{9.600000}\selectfont \(\displaystyle 0.2\)}%
\end{pgfscope}%
\begin{pgfscope}%
\pgfsetbuttcap%
\pgfsetroundjoin%
\definecolor{currentfill}{rgb}{0.000000,0.000000,0.000000}%
\pgfsetfillcolor{currentfill}%
\pgfsetlinewidth{0.501875pt}%
\definecolor{currentstroke}{rgb}{0.000000,0.000000,0.000000}%
\pgfsetstrokecolor{currentstroke}%
\pgfsetdash{}{0pt}%
\pgfsys@defobject{currentmarker}{\pgfqpoint{0.000000in}{0.000000in}}{\pgfqpoint{0.000000in}{0.055556in}}{%
\pgfpathmoveto{\pgfqpoint{0.000000in}{0.000000in}}%
\pgfpathlineto{\pgfqpoint{0.000000in}{0.055556in}}%
\pgfusepath{stroke,fill}%
}%
\begin{pgfscope}%
\pgfsys@transformshift{1.696500in}{0.301292in}%
\pgfsys@useobject{currentmarker}{}%
\end{pgfscope}%
\end{pgfscope}%
\begin{pgfscope}%
\pgfsetbuttcap%
\pgfsetroundjoin%
\definecolor{currentfill}{rgb}{0.000000,0.000000,0.000000}%
\pgfsetfillcolor{currentfill}%
\pgfsetlinewidth{0.501875pt}%
\definecolor{currentstroke}{rgb}{0.000000,0.000000,0.000000}%
\pgfsetstrokecolor{currentstroke}%
\pgfsetdash{}{0pt}%
\pgfsys@defobject{currentmarker}{\pgfqpoint{0.000000in}{-0.055556in}}{\pgfqpoint{0.000000in}{0.000000in}}{%
\pgfpathmoveto{\pgfqpoint{0.000000in}{0.000000in}}%
\pgfpathlineto{\pgfqpoint{0.000000in}{-0.055556in}}%
\pgfusepath{stroke,fill}%
}%
\begin{pgfscope}%
\pgfsys@transformshift{1.696500in}{2.169299in}%
\pgfsys@useobject{currentmarker}{}%
\end{pgfscope}%
\end{pgfscope}%
\begin{pgfscope}%
\pgftext[x=1.696500in,y=0.245736in,,top]{\rmfamily\fontsize{8.000000}{9.600000}\selectfont \(\displaystyle 0.4\)}%
\end{pgfscope}%
\begin{pgfscope}%
\pgfsetbuttcap%
\pgfsetroundjoin%
\definecolor{currentfill}{rgb}{0.000000,0.000000,0.000000}%
\pgfsetfillcolor{currentfill}%
\pgfsetlinewidth{0.501875pt}%
\definecolor{currentstroke}{rgb}{0.000000,0.000000,0.000000}%
\pgfsetstrokecolor{currentstroke}%
\pgfsetdash{}{0pt}%
\pgfsys@defobject{currentmarker}{\pgfqpoint{0.000000in}{0.000000in}}{\pgfqpoint{0.000000in}{0.055556in}}{%
\pgfpathmoveto{\pgfqpoint{0.000000in}{0.000000in}}%
\pgfpathlineto{\pgfqpoint{0.000000in}{0.055556in}}%
\pgfusepath{stroke,fill}%
}%
\begin{pgfscope}%
\pgfsys@transformshift{2.301000in}{0.301292in}%
\pgfsys@useobject{currentmarker}{}%
\end{pgfscope}%
\end{pgfscope}%
\begin{pgfscope}%
\pgfsetbuttcap%
\pgfsetroundjoin%
\definecolor{currentfill}{rgb}{0.000000,0.000000,0.000000}%
\pgfsetfillcolor{currentfill}%
\pgfsetlinewidth{0.501875pt}%
\definecolor{currentstroke}{rgb}{0.000000,0.000000,0.000000}%
\pgfsetstrokecolor{currentstroke}%
\pgfsetdash{}{0pt}%
\pgfsys@defobject{currentmarker}{\pgfqpoint{0.000000in}{-0.055556in}}{\pgfqpoint{0.000000in}{0.000000in}}{%
\pgfpathmoveto{\pgfqpoint{0.000000in}{0.000000in}}%
\pgfpathlineto{\pgfqpoint{0.000000in}{-0.055556in}}%
\pgfusepath{stroke,fill}%
}%
\begin{pgfscope}%
\pgfsys@transformshift{2.301000in}{2.169299in}%
\pgfsys@useobject{currentmarker}{}%
\end{pgfscope}%
\end{pgfscope}%
\begin{pgfscope}%
\pgftext[x=2.301000in,y=0.245736in,,top]{\rmfamily\fontsize{8.000000}{9.600000}\selectfont \(\displaystyle 0.6\)}%
\end{pgfscope}%
\begin{pgfscope}%
\pgfsetbuttcap%
\pgfsetroundjoin%
\definecolor{currentfill}{rgb}{0.000000,0.000000,0.000000}%
\pgfsetfillcolor{currentfill}%
\pgfsetlinewidth{0.501875pt}%
\definecolor{currentstroke}{rgb}{0.000000,0.000000,0.000000}%
\pgfsetstrokecolor{currentstroke}%
\pgfsetdash{}{0pt}%
\pgfsys@defobject{currentmarker}{\pgfqpoint{0.000000in}{0.000000in}}{\pgfqpoint{0.000000in}{0.055556in}}{%
\pgfpathmoveto{\pgfqpoint{0.000000in}{0.000000in}}%
\pgfpathlineto{\pgfqpoint{0.000000in}{0.055556in}}%
\pgfusepath{stroke,fill}%
}%
\begin{pgfscope}%
\pgfsys@transformshift{2.905500in}{0.301292in}%
\pgfsys@useobject{currentmarker}{}%
\end{pgfscope}%
\end{pgfscope}%
\begin{pgfscope}%
\pgfsetbuttcap%
\pgfsetroundjoin%
\definecolor{currentfill}{rgb}{0.000000,0.000000,0.000000}%
\pgfsetfillcolor{currentfill}%
\pgfsetlinewidth{0.501875pt}%
\definecolor{currentstroke}{rgb}{0.000000,0.000000,0.000000}%
\pgfsetstrokecolor{currentstroke}%
\pgfsetdash{}{0pt}%
\pgfsys@defobject{currentmarker}{\pgfqpoint{0.000000in}{-0.055556in}}{\pgfqpoint{0.000000in}{0.000000in}}{%
\pgfpathmoveto{\pgfqpoint{0.000000in}{0.000000in}}%
\pgfpathlineto{\pgfqpoint{0.000000in}{-0.055556in}}%
\pgfusepath{stroke,fill}%
}%
\begin{pgfscope}%
\pgfsys@transformshift{2.905500in}{2.169299in}%
\pgfsys@useobject{currentmarker}{}%
\end{pgfscope}%
\end{pgfscope}%
\begin{pgfscope}%
\pgftext[x=2.905500in,y=0.245736in,,top]{\rmfamily\fontsize{8.000000}{9.600000}\selectfont \(\displaystyle 0.8\)}%
\end{pgfscope}%
\begin{pgfscope}%
\pgfsetbuttcap%
\pgfsetroundjoin%
\definecolor{currentfill}{rgb}{0.000000,0.000000,0.000000}%
\pgfsetfillcolor{currentfill}%
\pgfsetlinewidth{0.501875pt}%
\definecolor{currentstroke}{rgb}{0.000000,0.000000,0.000000}%
\pgfsetstrokecolor{currentstroke}%
\pgfsetdash{}{0pt}%
\pgfsys@defobject{currentmarker}{\pgfqpoint{0.000000in}{0.000000in}}{\pgfqpoint{0.000000in}{0.055556in}}{%
\pgfpathmoveto{\pgfqpoint{0.000000in}{0.000000in}}%
\pgfpathlineto{\pgfqpoint{0.000000in}{0.055556in}}%
\pgfusepath{stroke,fill}%
}%
\begin{pgfscope}%
\pgfsys@transformshift{3.510000in}{0.301292in}%
\pgfsys@useobject{currentmarker}{}%
\end{pgfscope}%
\end{pgfscope}%
\begin{pgfscope}%
\pgfsetbuttcap%
\pgfsetroundjoin%
\definecolor{currentfill}{rgb}{0.000000,0.000000,0.000000}%
\pgfsetfillcolor{currentfill}%
\pgfsetlinewidth{0.501875pt}%
\definecolor{currentstroke}{rgb}{0.000000,0.000000,0.000000}%
\pgfsetstrokecolor{currentstroke}%
\pgfsetdash{}{0pt}%
\pgfsys@defobject{currentmarker}{\pgfqpoint{0.000000in}{-0.055556in}}{\pgfqpoint{0.000000in}{0.000000in}}{%
\pgfpathmoveto{\pgfqpoint{0.000000in}{0.000000in}}%
\pgfpathlineto{\pgfqpoint{0.000000in}{-0.055556in}}%
\pgfusepath{stroke,fill}%
}%
\begin{pgfscope}%
\pgfsys@transformshift{3.510000in}{2.169299in}%
\pgfsys@useobject{currentmarker}{}%
\end{pgfscope}%
\end{pgfscope}%
\begin{pgfscope}%
\pgftext[x=3.510000in,y=0.245736in,,top]{\rmfamily\fontsize{8.000000}{9.600000}\selectfont \(\displaystyle 1.0\)}%
\end{pgfscope}%
\begin{pgfscope}%
\pgftext[x=1.998750in,y=0.078167in,,top]{\rmfamily\fontsize{10.000000}{12.000000}\selectfont \(\displaystyle \eta\)}%
\end{pgfscope}%
\begin{pgfscope}%
\pgfsetbuttcap%
\pgfsetroundjoin%
\definecolor{currentfill}{rgb}{0.000000,0.000000,0.000000}%
\pgfsetfillcolor{currentfill}%
\pgfsetlinewidth{0.501875pt}%
\definecolor{currentstroke}{rgb}{0.000000,0.000000,0.000000}%
\pgfsetstrokecolor{currentstroke}%
\pgfsetdash{}{0pt}%
\pgfsys@defobject{currentmarker}{\pgfqpoint{0.000000in}{0.000000in}}{\pgfqpoint{0.055556in}{0.000000in}}{%
\pgfpathmoveto{\pgfqpoint{0.000000in}{0.000000in}}%
\pgfpathlineto{\pgfqpoint{0.055556in}{0.000000in}}%
\pgfusepath{stroke,fill}%
}%
\begin{pgfscope}%
\pgfsys@transformshift{0.487500in}{0.301292in}%
\pgfsys@useobject{currentmarker}{}%
\end{pgfscope}%
\end{pgfscope}%
\begin{pgfscope}%
\pgfsetbuttcap%
\pgfsetroundjoin%
\definecolor{currentfill}{rgb}{0.000000,0.000000,0.000000}%
\pgfsetfillcolor{currentfill}%
\pgfsetlinewidth{0.501875pt}%
\definecolor{currentstroke}{rgb}{0.000000,0.000000,0.000000}%
\pgfsetstrokecolor{currentstroke}%
\pgfsetdash{}{0pt}%
\pgfsys@defobject{currentmarker}{\pgfqpoint{-0.055556in}{0.000000in}}{\pgfqpoint{0.000000in}{0.000000in}}{%
\pgfpathmoveto{\pgfqpoint{0.000000in}{0.000000in}}%
\pgfpathlineto{\pgfqpoint{-0.055556in}{0.000000in}}%
\pgfusepath{stroke,fill}%
}%
\begin{pgfscope}%
\pgfsys@transformshift{3.510000in}{0.301292in}%
\pgfsys@useobject{currentmarker}{}%
\end{pgfscope}%
\end{pgfscope}%
\begin{pgfscope}%
\pgftext[x=0.431944in,y=0.301292in,right,]{\rmfamily\fontsize{8.000000}{9.600000}\selectfont \(\displaystyle 0.0\)}%
\end{pgfscope}%
\begin{pgfscope}%
\pgfsetbuttcap%
\pgfsetroundjoin%
\definecolor{currentfill}{rgb}{0.000000,0.000000,0.000000}%
\pgfsetfillcolor{currentfill}%
\pgfsetlinewidth{0.501875pt}%
\definecolor{currentstroke}{rgb}{0.000000,0.000000,0.000000}%
\pgfsetstrokecolor{currentstroke}%
\pgfsetdash{}{0pt}%
\pgfsys@defobject{currentmarker}{\pgfqpoint{0.000000in}{0.000000in}}{\pgfqpoint{0.055556in}{0.000000in}}{%
\pgfpathmoveto{\pgfqpoint{0.000000in}{0.000000in}}%
\pgfpathlineto{\pgfqpoint{0.055556in}{0.000000in}}%
\pgfusepath{stroke,fill}%
}%
\begin{pgfscope}%
\pgfsys@transformshift{0.487500in}{0.674893in}%
\pgfsys@useobject{currentmarker}{}%
\end{pgfscope}%
\end{pgfscope}%
\begin{pgfscope}%
\pgfsetbuttcap%
\pgfsetroundjoin%
\definecolor{currentfill}{rgb}{0.000000,0.000000,0.000000}%
\pgfsetfillcolor{currentfill}%
\pgfsetlinewidth{0.501875pt}%
\definecolor{currentstroke}{rgb}{0.000000,0.000000,0.000000}%
\pgfsetstrokecolor{currentstroke}%
\pgfsetdash{}{0pt}%
\pgfsys@defobject{currentmarker}{\pgfqpoint{-0.055556in}{0.000000in}}{\pgfqpoint{0.000000in}{0.000000in}}{%
\pgfpathmoveto{\pgfqpoint{0.000000in}{0.000000in}}%
\pgfpathlineto{\pgfqpoint{-0.055556in}{0.000000in}}%
\pgfusepath{stroke,fill}%
}%
\begin{pgfscope}%
\pgfsys@transformshift{3.510000in}{0.674893in}%
\pgfsys@useobject{currentmarker}{}%
\end{pgfscope}%
\end{pgfscope}%
\begin{pgfscope}%
\pgftext[x=0.431944in,y=0.674893in,right,]{\rmfamily\fontsize{8.000000}{9.600000}\selectfont \(\displaystyle 0.2\)}%
\end{pgfscope}%
\begin{pgfscope}%
\pgfsetbuttcap%
\pgfsetroundjoin%
\definecolor{currentfill}{rgb}{0.000000,0.000000,0.000000}%
\pgfsetfillcolor{currentfill}%
\pgfsetlinewidth{0.501875pt}%
\definecolor{currentstroke}{rgb}{0.000000,0.000000,0.000000}%
\pgfsetstrokecolor{currentstroke}%
\pgfsetdash{}{0pt}%
\pgfsys@defobject{currentmarker}{\pgfqpoint{0.000000in}{0.000000in}}{\pgfqpoint{0.055556in}{0.000000in}}{%
\pgfpathmoveto{\pgfqpoint{0.000000in}{0.000000in}}%
\pgfpathlineto{\pgfqpoint{0.055556in}{0.000000in}}%
\pgfusepath{stroke,fill}%
}%
\begin{pgfscope}%
\pgfsys@transformshift{0.487500in}{1.048495in}%
\pgfsys@useobject{currentmarker}{}%
\end{pgfscope}%
\end{pgfscope}%
\begin{pgfscope}%
\pgfsetbuttcap%
\pgfsetroundjoin%
\definecolor{currentfill}{rgb}{0.000000,0.000000,0.000000}%
\pgfsetfillcolor{currentfill}%
\pgfsetlinewidth{0.501875pt}%
\definecolor{currentstroke}{rgb}{0.000000,0.000000,0.000000}%
\pgfsetstrokecolor{currentstroke}%
\pgfsetdash{}{0pt}%
\pgfsys@defobject{currentmarker}{\pgfqpoint{-0.055556in}{0.000000in}}{\pgfqpoint{0.000000in}{0.000000in}}{%
\pgfpathmoveto{\pgfqpoint{0.000000in}{0.000000in}}%
\pgfpathlineto{\pgfqpoint{-0.055556in}{0.000000in}}%
\pgfusepath{stroke,fill}%
}%
\begin{pgfscope}%
\pgfsys@transformshift{3.510000in}{1.048495in}%
\pgfsys@useobject{currentmarker}{}%
\end{pgfscope}%
\end{pgfscope}%
\begin{pgfscope}%
\pgftext[x=0.431944in,y=1.048495in,right,]{\rmfamily\fontsize{8.000000}{9.600000}\selectfont \(\displaystyle 0.4\)}%
\end{pgfscope}%
\begin{pgfscope}%
\pgfsetbuttcap%
\pgfsetroundjoin%
\definecolor{currentfill}{rgb}{0.000000,0.000000,0.000000}%
\pgfsetfillcolor{currentfill}%
\pgfsetlinewidth{0.501875pt}%
\definecolor{currentstroke}{rgb}{0.000000,0.000000,0.000000}%
\pgfsetstrokecolor{currentstroke}%
\pgfsetdash{}{0pt}%
\pgfsys@defobject{currentmarker}{\pgfqpoint{0.000000in}{0.000000in}}{\pgfqpoint{0.055556in}{0.000000in}}{%
\pgfpathmoveto{\pgfqpoint{0.000000in}{0.000000in}}%
\pgfpathlineto{\pgfqpoint{0.055556in}{0.000000in}}%
\pgfusepath{stroke,fill}%
}%
\begin{pgfscope}%
\pgfsys@transformshift{0.487500in}{1.422096in}%
\pgfsys@useobject{currentmarker}{}%
\end{pgfscope}%
\end{pgfscope}%
\begin{pgfscope}%
\pgfsetbuttcap%
\pgfsetroundjoin%
\definecolor{currentfill}{rgb}{0.000000,0.000000,0.000000}%
\pgfsetfillcolor{currentfill}%
\pgfsetlinewidth{0.501875pt}%
\definecolor{currentstroke}{rgb}{0.000000,0.000000,0.000000}%
\pgfsetstrokecolor{currentstroke}%
\pgfsetdash{}{0pt}%
\pgfsys@defobject{currentmarker}{\pgfqpoint{-0.055556in}{0.000000in}}{\pgfqpoint{0.000000in}{0.000000in}}{%
\pgfpathmoveto{\pgfqpoint{0.000000in}{0.000000in}}%
\pgfpathlineto{\pgfqpoint{-0.055556in}{0.000000in}}%
\pgfusepath{stroke,fill}%
}%
\begin{pgfscope}%
\pgfsys@transformshift{3.510000in}{1.422096in}%
\pgfsys@useobject{currentmarker}{}%
\end{pgfscope}%
\end{pgfscope}%
\begin{pgfscope}%
\pgftext[x=0.431944in,y=1.422096in,right,]{\rmfamily\fontsize{8.000000}{9.600000}\selectfont \(\displaystyle 0.6\)}%
\end{pgfscope}%
\begin{pgfscope}%
\pgfsetbuttcap%
\pgfsetroundjoin%
\definecolor{currentfill}{rgb}{0.000000,0.000000,0.000000}%
\pgfsetfillcolor{currentfill}%
\pgfsetlinewidth{0.501875pt}%
\definecolor{currentstroke}{rgb}{0.000000,0.000000,0.000000}%
\pgfsetstrokecolor{currentstroke}%
\pgfsetdash{}{0pt}%
\pgfsys@defobject{currentmarker}{\pgfqpoint{0.000000in}{0.000000in}}{\pgfqpoint{0.055556in}{0.000000in}}{%
\pgfpathmoveto{\pgfqpoint{0.000000in}{0.000000in}}%
\pgfpathlineto{\pgfqpoint{0.055556in}{0.000000in}}%
\pgfusepath{stroke,fill}%
}%
\begin{pgfscope}%
\pgfsys@transformshift{0.487500in}{1.795698in}%
\pgfsys@useobject{currentmarker}{}%
\end{pgfscope}%
\end{pgfscope}%
\begin{pgfscope}%
\pgfsetbuttcap%
\pgfsetroundjoin%
\definecolor{currentfill}{rgb}{0.000000,0.000000,0.000000}%
\pgfsetfillcolor{currentfill}%
\pgfsetlinewidth{0.501875pt}%
\definecolor{currentstroke}{rgb}{0.000000,0.000000,0.000000}%
\pgfsetstrokecolor{currentstroke}%
\pgfsetdash{}{0pt}%
\pgfsys@defobject{currentmarker}{\pgfqpoint{-0.055556in}{0.000000in}}{\pgfqpoint{0.000000in}{0.000000in}}{%
\pgfpathmoveto{\pgfqpoint{0.000000in}{0.000000in}}%
\pgfpathlineto{\pgfqpoint{-0.055556in}{0.000000in}}%
\pgfusepath{stroke,fill}%
}%
\begin{pgfscope}%
\pgfsys@transformshift{3.510000in}{1.795698in}%
\pgfsys@useobject{currentmarker}{}%
\end{pgfscope}%
\end{pgfscope}%
\begin{pgfscope}%
\pgftext[x=0.431944in,y=1.795698in,right,]{\rmfamily\fontsize{8.000000}{9.600000}\selectfont \(\displaystyle 0.8\)}%
\end{pgfscope}%
\begin{pgfscope}%
\pgfsetbuttcap%
\pgfsetroundjoin%
\definecolor{currentfill}{rgb}{0.000000,0.000000,0.000000}%
\pgfsetfillcolor{currentfill}%
\pgfsetlinewidth{0.501875pt}%
\definecolor{currentstroke}{rgb}{0.000000,0.000000,0.000000}%
\pgfsetstrokecolor{currentstroke}%
\pgfsetdash{}{0pt}%
\pgfsys@defobject{currentmarker}{\pgfqpoint{0.000000in}{0.000000in}}{\pgfqpoint{0.055556in}{0.000000in}}{%
\pgfpathmoveto{\pgfqpoint{0.000000in}{0.000000in}}%
\pgfpathlineto{\pgfqpoint{0.055556in}{0.000000in}}%
\pgfusepath{stroke,fill}%
}%
\begin{pgfscope}%
\pgfsys@transformshift{0.487500in}{2.169299in}%
\pgfsys@useobject{currentmarker}{}%
\end{pgfscope}%
\end{pgfscope}%
\begin{pgfscope}%
\pgfsetbuttcap%
\pgfsetroundjoin%
\definecolor{currentfill}{rgb}{0.000000,0.000000,0.000000}%
\pgfsetfillcolor{currentfill}%
\pgfsetlinewidth{0.501875pt}%
\definecolor{currentstroke}{rgb}{0.000000,0.000000,0.000000}%
\pgfsetstrokecolor{currentstroke}%
\pgfsetdash{}{0pt}%
\pgfsys@defobject{currentmarker}{\pgfqpoint{-0.055556in}{0.000000in}}{\pgfqpoint{0.000000in}{0.000000in}}{%
\pgfpathmoveto{\pgfqpoint{0.000000in}{0.000000in}}%
\pgfpathlineto{\pgfqpoint{-0.055556in}{0.000000in}}%
\pgfusepath{stroke,fill}%
}%
\begin{pgfscope}%
\pgfsys@transformshift{3.510000in}{2.169299in}%
\pgfsys@useobject{currentmarker}{}%
\end{pgfscope}%
\end{pgfscope}%
\begin{pgfscope}%
\pgftext[x=0.431944in,y=2.169299in,right,]{\rmfamily\fontsize{8.000000}{9.600000}\selectfont \(\displaystyle 1.0\)}%
\end{pgfscope}%
\begin{pgfscope}%
\pgftext[x=0.211649in,y=1.235295in,,bottom,rotate=90.000000]{\rmfamily\fontsize{10.000000}{12.000000}\selectfont \(\displaystyle y\)}%
\end{pgfscope}%
\begin{pgfscope}%
\pgfsetbuttcap%
\pgfsetmiterjoin%
\definecolor{currentfill}{rgb}{1.000000,1.000000,1.000000}%
\pgfsetfillcolor{currentfill}%
\pgfsetlinewidth{1.003750pt}%
\definecolor{currentstroke}{rgb}{0.000000,0.000000,0.000000}%
\pgfsetstrokecolor{currentstroke}%
\pgfsetdash{}{0pt}%
\pgfpathmoveto{\pgfqpoint{0.543056in}{1.722833in}}%
\pgfpathlineto{\pgfqpoint{1.988345in}{1.722833in}}%
\pgfpathlineto{\pgfqpoint{1.988345in}{2.113744in}}%
\pgfpathlineto{\pgfqpoint{0.543056in}{2.113744in}}%
\pgfpathclose%
\pgfusepath{stroke,fill}%
\end{pgfscope}%
\begin{pgfscope}%
\pgfsetrectcap%
\pgfsetroundjoin%
\pgfsetlinewidth{1.003750pt}%
\definecolor{currentstroke}{rgb}{0.000000,0.000000,1.000000}%
\pgfsetstrokecolor{currentstroke}%
\pgfsetdash{}{0pt}%
\pgfpathmoveto{\pgfqpoint{0.620833in}{2.005086in}}%
\pgfpathlineto{\pgfqpoint{0.776389in}{2.005086in}}%
\pgfusepath{stroke}%
\end{pgfscope}%
\begin{pgfscope}%
\pgftext[x=0.898611in,y=1.966197in,left,base]{\rmfamily\fontsize{8.000000}{9.600000}\selectfont \(\displaystyle y = (1-\gamma)V^K_{\gamma}(y, \eta)\)}%
\end{pgfscope}%
\begin{pgfscope}%
\pgfsetbuttcap%
\pgfsetroundjoin%
\pgfsetlinewidth{1.003750pt}%
\definecolor{currentstroke}{rgb}{1.000000,0.000000,0.000000}%
\pgfsetstrokecolor{currentstroke}%
\pgfsetdash{{6.000000pt}{6.000000pt}}{0.000000pt}%
\pgfpathmoveto{\pgfqpoint{0.620833in}{1.827771in}}%
\pgfpathlineto{\pgfqpoint{0.776389in}{1.827771in}}%
\pgfusepath{stroke}%
\end{pgfscope}%
\begin{pgfscope}%
\pgftext[x=0.898611in,y=1.788882in,left,base]{\rmfamily\fontsize{8.000000}{9.600000}\selectfont \(\displaystyle y = \eta\)}%
\end{pgfscope}%
\end{pgfpicture}%
\makeatother%
\endgroup%
 
		\caption{Visualization of Lemma~\ref{cor:equivalent_event}'s proof for a instance of the problem with a Beta prior corresponding to the pair $y = (4,5)$, a discount factor of $\gamma=0.95$ and $K = 2$. The intersection of the two lines represents the Optimistic Gittins index.}
		\label{fig:visaulize_gx_proof}
	\end{figure}
	This proves the result in question. Figure~\ref{fig:visaulize_gx_proof} also provides a visualization.
\end{myproof}
}

\subsection{Proof of Lemma~\ref{lemma:approx_bound}} \label{prf:approx_bound}
\begin{myproof}[Proof.]
	Let $K < M$ be two look-ahead parameters used in the definition of OGI. We will show that $V^K_\gamma(y, \lambda) \le V^M_\gamma(y, \lambda)$ where we recall the definitions of these functions from the beginning of Section~\ref{sec:appendix_properties_of_ogi}.
	
	We begin with a fundamental step. Let $\tau_1$ and $\tau_2$ be any predictable stopping times (i.e. $\mathcal F_{t-1}$-measurable random times) such that $\tau_1$ precedes $\tau_2$ almost surely, that is $\tau_1 < \tau_2$. Recall that the expected reward of the $i$th arm satisfies $\Ee{X_{i,t} \given \theta_i} = \mu(\theta_i)$ for all $t$. Let $\hat \theta_i \in \Theta$ denote a realization of the random variable $\theta_i$ and let $\zeta(\hat \theta_i)$ be a real-valued, measurable function of $\hat \theta_i$. In this case, we have that
	\begin{align*}
	\Ee{\sum_{t=\tau_1+1}^{\tau_2} \gamma^{t-1} X_{i,t} + \gamma^{\tau_2}\frac{\zeta(\hat \theta_i)}{1 - \gamma} \given[\Bigg] \theta_i = \hat \theta_i} & =\mu(\hat \theta_i) \Ee{ \sum_{t=\tau_1+1}^{\tau_2} \gamma^{t-1}  \given[\Bigg] \theta_i = \hat \theta_i} + \Ee{\frac{\gamma^{\tau_2}}{1-\gamma}\given[\Bigg] \theta_i = \hat \theta_i}\zeta(\hat \theta_i) \\
	%& =\mu(\hat \theta_i) \Ee{ \sum_{t=\tau_1+1}^{\tau_2} \gamma^{t-1} \given[\Bigg] \theta_i = \hat \theta_i } + \Ee{\sum_{t=\tau_2+1}^{\infty} \gamma^{t-1} \given[\Bigg] \theta_i = \hat \theta_i}\frac{\zeta(\hat \theta_i)}{1 - \gamma} \\
	& \le \Ee{\gamma^{\tau_1} \given \theta_i = \hat \theta_i}  \frac{\max(\zeta(\hat \theta_i), \mu(\hat \theta_i))}{1-\gamma}.
	\end{align*}
	Thus we conclude, because $\hat \theta_i$ was arbitrary, that almost surely,
	\begin{equation} \label{ineq:fundamntal_bound_for_lemma_2}
	\Ee{\sum_{t=\tau_1+1}^{\tau_2} \gamma^{t-1} X_{i,t} + \gamma^{\tau_2}\frac{\zeta( \theta_i)}{1 - \gamma} \given[\Bigg] \theta_i}  \le \Ee{\gamma^{\tau_1} \given \theta_i }  \frac{\max(\zeta( \theta_i), \mu(\theta_i))}{1-\gamma}.
	\end{equation}
	Let $\tau^\star$ be a stopping time that achieves the supremum in  $V_\gamma^M(y, \lambda)$ and define the predictable stopping time $\tau^\star_K \defeq K \wedge \tau^\star$. Consider the (conditional) cumulative rewards in the definition of $V^M_\gamma(y)$, from time $\tau^\star_K+1$ onwards, given the sufficient statistic observed at time $\tau_K^\star$. That is, 
	\[
		\E\left[\sum_{t=\tau_K^\star+1}^{\tau^\star}  \gamma^{t-1} X_{i,t} + \gamma^{\tau^\star} R_{\lambda,M}(\tau^\star, y_{i,\tau^\star-1})/(1-\gamma)
	\given[\Bigg] y_{i,\tau_K^\star-1} \right].
	\]
	We upper bound this random variable as follows. First, we note that, at any time $s$ and for any statistic $\hat y \in \mathcal{Y}$, the following statement holds
	\begin{equation}\label{eqn:dist_equal_theta_i}
	\P{R(\hat y) \le r} = \P{\mu(\theta_i) \le r \given y_{i,s} = \hat y}, \qquad \forall r \in \Re
	\end{equation}
	meaning that the posterior distribution of the arm's expected reward $R(y_{i,s})$ is the same as $\mu(\theta_i)$ \emph{conditioned} on having observed statistic $\hat y$ about the arm. This holds by definition of the random variable $R(y)$. Because of this observation, we have that the following inequality  holds almost surely,
	{
	\begin{align*}
		& \gamma^{\tau^\star} \frac{R_{\lambda,M}(\tau^\star, y_{i,\tau^\star-1})}{1-\gamma}
		 \nonumber \\
		&\quad =  \gamma^{\tau^\star}\left( \ind{\tau^\star =M}\frac{\max(\lambda, R(y_{i,\tau^\star-1}))}{1-\gamma} + \ind{\tau^\star < M}\frac{\lambda}{1-\gamma}\right)
		 \nonumber\\
		&\quad= \ind{\tau^\star =M} \gamma^{M}\frac{\max(\lambda, R(y_{i,M-1}))}{1-\gamma} + \ind{\tau^\star < M} \gamma^{\tau^\star}\frac{\lambda}{1-\gamma}
	\nonumber\\
		& \quad\overset{(*)}{=} \ind{\tau^\star =M} \gamma^{M}\frac{\Ee{\max(\lambda, R(y_{i,M-1})) \given y_{i,M-1} }}{1-\gamma} + \ind{\tau^\star < M} \gamma^{\tau^\star}\frac{\lambda}{1-\gamma}
		\\
		& \quad\overset{(\dagger)}{=} \ind{\tau^\star =M} \gamma^{M}\frac{\Ee{\max(\lambda, \mu(\theta_i)) \given y_{i,M-1} }}{1-\gamma} + \ind{\tau^\star < M} \gamma^{\tau^\star}\frac{\lambda}{1-\gamma}
		 \\
		&\quad \overset{(**)}{\le} \frac{  \Ee{\gamma^{\tau^\star}\max(\lambda, \mu(\theta_i)) \given y_{i,\tau^\star-1} }}{1-\gamma}
	\end{align*}
	}
	where $(*)$ and $(**)$ both use the fact that for any $t$, $\tau^\star \le t$ is measurable with respect to the $\sigma$-algebra generated by $y_{i,t-1}$, namely $\mathcal F_{t-1}$. Equation ($\dagger$) follows from \eqref{eqn:dist_equal_theta_i}. Therefore, immediately using the above inequality and conditioning on the event $\tau^\star > K$, we have that
	\begin{align} 
	&\E\left[\sum_{t=\tau_K^\star+1}^{\tau^\star} \gamma^{t-1} X_{i,t} + \gamma^{\tau^\star} \frac{R_{\lambda,M}(\tau^\star, y_{i,\tau^\star-1})}{1-\gamma}
	\given[\Bigg] \tau^\star > K, \;y_{i,\tau^\star_K-1} \right] \nonumber \\
	&\qquad \le  \E\left[\sum_{t=\tau_K^\star+1}^{\tau^\star} \gamma^{t-1} X_{i,t} +  \Ee{ \gamma^{\tau^\star}\frac{\max(\lambda,\mu(\theta_i)))}{1-\gamma}\given[\Bigg] y_{i,\tau^\star-1}}
	\given[\Bigg] \tau^\star > K, \;y_{i,\tau^\star_K-1} \right] \nonumber\\
	&\qquad = \E\left[\sum_{t=\tau_K^\star+1}^{\tau^\star} \gamma^{t-1} X_{i,t} + \gamma^{\tau^\star} \frac{\max(\lambda, \mu(\theta_i))}{1-\gamma}
	\given[\Bigg] \tau^\star > K, \;y_{i,\tau^\star_K-1} \right] \label{eqn:another_toer_prop_use}\\
	&\qquad = \E\left[ \Ee{\sum_{t=K+1}^{\tau^\star} \gamma^{t-1} X_{i,t} + \gamma^{\tau^\star}\frac{\max(\lambda,\mu(\theta_i))}{1 - \gamma} \given[\Bigg] \theta_i} 
	\given[\Bigg]\tau^\star > K, \; y_{i,\tau^\star_K-1} \right] \label{eqn:proof_lemma2_toer_prop} \\
	&\qquad \le \E\left[ \gamma^{\tau^\star_K} \frac{\max(\mu(\theta_i), \lambda)}{1 - \gamma}\given[\Bigg] \tau^\star > K, \; y_{i,\tau^\star_K-1}  \right]  \label{ineq:proof_lem_2_use_of_first_bound} \\
	&\qquad = \E\left[ \gamma^{\tau^\star_K} \frac{\max(R(y_{i,\tau^\star_K-1}), \lambda)}{1 - \gamma}\given[\Bigg] \tau^\star > K, \; y_{i,\tau^\star_K-1}  \right]  \label{ineq:obvious_step} \\
	&\qquad = \Ee{\frac{\gamma^{\tau^\star_K}R_{\lambda,K}(\tau^\star_K, y_{i,\tau^\star_K-1})}{1-\gamma} \given[\Bigg]  \tau^\star > K, \; y_{i,\tau^\star_K-1}} \label{eqn:use_of_def_of_R}
	\end{align}
	where \eqref{eqn:another_toer_prop_use}, \eqref{eqn:proof_lemma2_toer_prop} use the tower property and \eqref{ineq:proof_lem_2_use_of_first_bound} follows from the bound in \eqref{ineq:fundamntal_bound_for_lemma_2} because $\tau^\star_K < \tau^\star$, almost surely. Equation \eqref{ineq:obvious_step} follows from statement \eqref{eqn:dist_equal_theta_i} and that the event $\tau^\star > K$ is $\mathcal{F}_{K-1}$-measurable (we can decide whether to pull arm $i$ or retire based on information up to and including time $K-1$). Finally equation \eqref{eqn:use_of_def_of_R} is derived by substituting in the definition of $R_{\lambda,K}$ (as given in Section~\ref{sec:gittins_and_approx}) and noting that $\tau^\star_K = K$ under the above conditioning.
	
	We now condition on the complement of the previous event we considered, namely, $\tau^\star \le K$. Under that event, $\tau^\star$ occurred early enough before time $K+1$ and thus $\tau^\star_K = \tau^\star$. Therefore, it follows from this observation that
	\begin{align} 
	&\E\left[\sum_{t=\tau_K^\star+1}^{\tau^\star}\gamma^{t-1} X_{i,t} + \gamma^{\tau^\star} \frac{R_{\lambda,M}(\tau^\star, y_{i,\tau^\star-1})}{1-\gamma}
	\given[\Bigg] \tau^\star \le K, \;y_{i,\tau^\star_K-1} \right] \nonumber  \\
	&\qquad = \E\left[\gamma^{\tau^\star}  \frac{\lambda}{1-\gamma}
	\given[\Bigg]\tau^\star \le K, \; y_{i,\tau^\star_K-1} \right] \nonumber \\
	&\qquad \le \E\left[\gamma^{\tau^\star_K}  \frac{R_{\lambda,K}(\tau^\star_K, y_{i,\tau^\star_K-1})}{1-\gamma}
	\given[\Bigg]\tau^\star \le K, \; y_{i,\tau^\star_K-1} \right] \label{eqn:proof_lem_2_second_use_of_R_def}
	\end{align}
	where \eqref{eqn:proof_lem_2_second_use_of_R_def} is obtained by noting that $R_{\lambda,K}(\tau, y) \ge \lambda$ for any choice of $\tau, K$ and $y$. Thus, by the law of total expectation and \eqref{eqn:use_of_def_of_R}, \eqref{eqn:proof_lem_2_second_use_of_R_def}, we establish that
	\begin{equation} \label{ineq:proof_lem_2_main_bound_in_proof}
	\E\left[\sum_{t=\tau_K^\star+1}^{\tau^\star}\gamma^{t-1} X_{i,t} + \gamma^{\tau^\star} \frac{R_{\lambda,M}(\tau^\star, y_{i,\tau^\star-1})}{1-\gamma}
	\given[\Bigg] \;y_{i,\tau^\star_K-1} \right] \le \Ee{\gamma^{\tau^\star_K}  \frac{R_{\lambda,K}(\tau^\star_K, y_{i,\tau^\star_K-1})}{1-\gamma} \given[\Bigg] y_{i,\tau^\star_K-1}}.
	\end{equation}
	We are ready to complete our main argument in this proof by using the above bound and `breaking up' the $V_\gamma^M(y, \lambda)$ into rewards from times before $\tau_K^\star$ and after (and bounding the latter terms). More precisely, we obtain that
	\begin{align}
	V_\gamma^M(y,\lambda) & = \E_{y}\left[\sum_{t=1}^{\tau^\star}\gamma^{t-1} X_{i,t} + \gamma^{\tau^\star}\frac{R_{\lambda,M}(\tau^\star, y_{i,\tau^\star-1})}{1-\gamma}\right] \\
	& = \E_{y}\left[\sum_{t=1}^{\tau^\star_K}\gamma^{t-1} X_{i,t} + \sum_{t'=\tau^\star_K+1}^{\tau^\star}\gamma^{t'-1} X_{i,t'} +  \gamma^{\tau^\star}\frac{R_{\lambda,M}(\tau^\star, y_{i,\tau^\star-1})}{1-\gamma}\right] \nonumber \\
	& = \E_{y}\left[\sum_{t=1}^{\tau^\star_K}\gamma^{t-1} X_{i,t} + \Ee{\sum_{t'=\tau^\star_K+1}^{\tau^\star}\gamma^{t'-1} X_{i,t'} +  \gamma^{\tau^\star}\frac{R_{\lambda,M}(\tau^\star, y_{i,\tau^\star-1})}{1-\gamma} \given[\Bigg] y_{i,\tau^\star_K-1} }\right] \label{eqn:proof_lem_2_tower_prop_again} \\
	& \le  \E_{y}\left[\sum_{t=1}^{\tau^\star_K}\gamma^{t-1} X_{i,t} + \Ee{\gamma^{\tau^\star_K}  \frac{R_{\lambda,K}(\tau^\star_K, y_{i,\tau^\star_K-1})}{1-\gamma} \given[\Bigg] y_{i,\tau^\star_K-1}}\right] \label{eqn:proof_lem2_using_the_main_argument} \\
	& =  \E_{y}\left[\sum_{t=1}^{\tau^\star_K}\gamma^{t-1} X_{i,t} +  \gamma^{\tau^\star_K}  \frac{R_{\lambda,K}(\tau^\star_K, y_{i,\tau^\star_K-1})}{1-\gamma}\right] \label{eqn:proof_lem_2_tower_prop_yet_again} \\
	& \le \sup_{1 \le \tau \le K}  \E_{y}\left[\sum_{t=1}^{\tau}\gamma^{t-1} X_{i,t} +  \gamma^{\tau}  \frac{R_{\lambda,K}(\tau, y_{i,\tau-1})}{1-\gamma}\right] \nonumber \\
	& = V^K_{\gamma}(y, \lambda)
	\end{align}
	where Equations \eqref{eqn:proof_lem_2_tower_prop_again}, \eqref{eqn:proof_lem_2_tower_prop_yet_again} use the tower property and \eqref{eqn:proof_lem2_using_the_main_argument} is immediately derived by using the bound of \eqref{ineq:proof_lem_2_main_bound_in_proof}. Finally, we note that an almost identical proof can be given to show that $V^K_\gamma(y, \lambda) \ge V_\gamma(y, \lambda)$ where the lower bound is the continuation value used to compute the Gittins index.
	
	We have shown that for any $\lambda$ and $y$,  that $V^K_\gamma(y, \lambda)$ is non-increasing in $K$, and that $V_\gamma(y, \lambda)$ is a lower bound to this sequence. We make use of these facts to now prove that $v^K_\gamma(y)$ is also non-increasing in $K$. To this end, let us suppose for contradiction that there exist two integers $K_1 \le K_2$ and $v^{K_1}_\gamma(y) < v^{K_2}_\gamma(y)$. From Lemma~\ref{cor:equivalent_event} we know that
	\begin{equation}
		V^{K_2}_\gamma(y, v^K_\gamma(y)) > v^K_\gamma(y)/(1-\gamma) = V^K_\gamma(y, v^K_\gamma(y)),
	\end{equation}
	which contradicts the claim just shown. Therefore, $v^K_\gamma(y)$ must also be a  non-increasing sequence in $K$. The same argument can be used to further show that $v^K_\gamma(y) \ge v_\gamma(y)$.
	
	We now turn our attention to proving the convergence property stated in the Lemma. The first step will be to prove that for all $y \in \mathcal{Y}$ and $\lambda \in \Re_+$, that 
	\begin{equation} \label{eqn:proof_vk_bound_convergence_of_continuation_value}
	\lim_{K \to \infty}V^K_\gamma(y, \lambda) = V_\gamma(y, \lambda).
	\end{equation}
	Indeed, we upper bound the optimistic continuation value for a fixed parameter $M$ as follows:
	\begin{align}
		V^M_\gamma(y, \lambda) & = \sup_{1 \le \tau \le M} \E_y\left[\sum_{t=1}^{\tau} \gamma^{t-1} X_{i,t} + \frac{\gamma^{\tau}R_{\lambda,M}(\tau, y_{i,\tau-1})}{1-\gamma}\right] \nonumber\\
		& = \sup_{1 \le \tau \le M} \E_y\left[\sum_{t=1}^{\tau} \gamma^{t-1} X_{i,t} + \frac{\gamma^{\tau}\lambda}{1-\gamma} + \frac{\gamma^{\tau}R_{\lambda,M}(\tau, y_{i,\tau-1})}{1-\gamma} - \frac{\gamma^{\tau}\lambda}{1-\gamma}\right] \nonumber \\
		& \le \sup_{\tau \ge 1} \E_y\left[\sum_{t=1}^{\tau } \gamma^{t-1} X_{i,t} + \frac{\gamma^{\tau}\lambda}{1-\gamma}\right] + \sup_{1 \le \tau \le M}\E_y\left[\frac{\gamma^{\tau}R_{\lambda,M}(\tau, y_{i,\tau-1})}{1-\gamma} - \frac{\gamma^{\tau -1}\lambda}{1-\gamma}\right] \nonumber \\
		& =V_{\gamma}(y, \lambda) + \sup_{1 \le \tau \le M}\E_y\left[\frac{\gamma^{\tau}[R_{\lambda,M}(\tau, y_{i,\tau-1})-\lambda]}{1-\gamma}\right] \nonumber \\
		& \le V_{\gamma}(y, \lambda)  + \gamma^{M}\E_y\left[\frac{R_{\lambda,M}(M, y_{i,M-1})-\lambda}{1-\gamma}\right] \nonumber \\
		& = V_{\gamma}(y, \lambda)  + \gamma^{M}\E_y\left[\frac{( R(y_{i,M-1}) - \lambda)^+}{1-\gamma}\right] \nonumber \\
		& \le V_{\gamma}(y, \lambda)  + \gamma^{M}\E_y\left[\frac{|R(y_{i,M-1})|}{1-\gamma}\right] \nonumber  \\
		& = V_{\gamma}(y, \lambda)  + \gamma^M \E_y\left[\frac{|\mu(\theta_i)|}{1-\gamma}\right] \label{eqn:proof_lem_vk_bound_iter_exp},
	\end{align}
	where equation \eqref{eqn:proof_lem_vk_bound_iter_exp} follows from the definition of the random variable $R(.)$ and the law of iterated expectation. Now because $0 < \gamma < 1$ and $\E_y\left[|\mu(\theta_i)|\right] < \infty$, the right hand side above converges to $V_{\gamma}(y, \lambda)$. Finally, notice that $V^M_\gamma(y, \lambda) \ge V_\gamma(y, \lambda)$, and from this equation~\eqref{eqn:proof_vk_bound_convergence_of_continuation_value} follows. 
	
	{
	To finish the proof, we consider the sequence of fixed points of the equations $\lambda = V^K_\gamma(y, \lambda)$, $\{v^K_\gamma(y)\}$. 
	Because this sequence is monotone (established in the first part of this proof) and bounded, we know that this sequence, has a limit; $v^K_\gamma(y) \tends \hat v_\gamma(y)$. 	
%	there exists a convergent sub-sequence $\{ \lambda_k \}$ whose elements we index with $k$. Let $\hat \lambda$ denote the limit of this sub-sequence. We will prove that this limit is the Gittins index. 

It remains to show that $\hat v_\gamma(y) = V_\gamma(y, \hat v_\gamma(y))$. For this it suffices to show that $v^K_\gamma(y) \tends V_\gamma(y, \hat v_\gamma(y))$, which we establish as follows: 
%	
%	Indeed, we can bound the difference between each element $\lambda_k$ and $V_\gamma(y, \hat \lambda)$ as follows
	\begin{align*}
	|v^K_\gamma(y) - V_\gamma(y, \hat v_\gamma(y))| &  = |V^K_\gamma(y, v^K_\gamma(y)) - V_\gamma(y, \hat v_\gamma(y)) | \\
	&  \leq \underbrace{|V^K_\gamma(y, v^K_\gamma(y)) - V^K_\gamma(y, \hat v_\gamma(y))|}_{=: a_k} + \underbrace{|V^K_\gamma(y, \hat v_\gamma(y)) - V_\gamma(y, \hat v_\gamma(y))|}_{=:b_k}.
	\end{align*}
	We already proved \eqref{eqn:proof_vk_bound_convergence_of_continuation_value} and therefore know that $b_k \to 0$ as $k \to \infty$. As for the $a_k$ sequence, we have
	\begin{align}
	a_k & = |V^K_\gamma(y, v^K_\gamma(y)) - V^K_\gamma(y, \hat v_\gamma(y))| \nonumber \\
	& \leq | v^K_\gamma(y) - \hat v_\gamma(y) | \label{eqn:use_of_lipshitz_bound}\\
	&  \to 0 \nonumber
	\end{align}
	where \eqref{eqn:use_of_lipshitz_bound} follows from the Lipschitz continuity of $V^K_\gamma(\cdot, \cdot)$ in its second argument as shown in Fact~\ref{fact:v_is_convex}. Therefore $v^K_\gamma(y) \tends V_\gamma(y, \hat v_\gamma(y))$, which completes the proof. 	
%	$\lambda_k \to V_\gamma(y, \hat \lambda)$, which implies that $\hat \lambda = V_\gamma(y, \hat \lambda)$. This proves that $\hat \lambda$ is a fixed point of the function $V_\gamma(y, .)$, namely the Gittins index.
	}
\end{myproof}


The next Lemma will be the final property of the function $V^K_\gamma$ that we prove. This will subsequently be used in the proof of Lemma~\ref{lemma:underestimation}.
\begin{lemma} \label{lemma:vk_bound}
	Let $i$ be any arm. For any look-ahead parameter $K \in \mathbb{Z}_+$, discount factor $\gamma$ and any constant $\eta$, we have
	\begin{equation*}
	\E_y\left[V^K_\gamma(y_{i,1}, \eta) \right] \ge V^K_\gamma(y, \eta)
	\end{equation*}
	where we recall that $y_{i,1}$ is the summary statistic corresponding to the posterior obtained from pulling arm $i$ once.
\end{lemma}
\begin{myproof}[Proof.]
	For any $\hat y \in \mathcal{Y}$, let $\tau^\star(\hat y)$ be the (predictable) optimal stopping time for the problem (involving computing $V^K_\gamma$) whose initial state is $y_{i,0} = \hat y$. With this notation in hand, we conclude that
	\begin{align}
	\E_y \left[V^K_\gamma(y_{i,1}, \eta) \right] & = \E_y\left[\E_{y_{i,1}}\left[\sum_{s=1}^{\tau^\star(y_{i,1})} \gamma^{s-1} X_{i,s} + \frac{\gamma^{\tau^\star(y_{i,1})} R_{\eta, K}(\tau, y_{i,\tau^\star(y_{i,1})-1})}{1-\gamma} \right]\right]\label{eqn:pf_vk_bound_tower_prop1}  \\
	& \ge  \E_y\left[\E_{y_{i,2}}\left[\sum_{s=1}^{\tau^\star(y)} \gamma^{s-1} X_{i,s} + \frac{\gamma^{\tau^\star(y)} R_{\eta, K}(\tau, y_{i,\tau^\star(y)-1})}{1-\gamma}\right] \right] \label{ineq:subopt_of_y}\\
	& = \E_y\left[\sum_{s=1}^{\tau^\star(y)} \gamma^{s-1} X_{i,s} + \frac{\gamma^{\tau^\star(y)} R_{\eta, K}(\tau, y_{i,\tau^\star(y)-1})}{1-\gamma}\right] \label{eqn:pf_vk_bound_tower_prop2} \\
	& = V^K_\gamma(y, \eta) \nonumber
	\end{align}
	where \eqref{eqn:pf_vk_bound_tower_prop1}, \eqref{eqn:pf_vk_bound_tower_prop2} both follow from the tower property and \eqref{ineq:subopt_of_y} is due to the sub-optimality of the stopping rule $\tau^\star(y)$ when the actual starting state is $y_{i,1}$. Intuitively, we lose out revenue by throwing away information about the arm.
\end{myproof}


\section{Results for the frequentist regret bound}
This section contains proofs of results required to show Theorem~\ref{thm:frequentist_optimal_bound}. It is helpful to go over the definitions and some general properties of the Optimistic Gittins index given in Section~\ref{sec:appendix_properties_of_ogi} when reading this.
\subsection{Definitions and properties of Binomial distributions.}
We list notation and facts related to Beta and Binomial distributions, which are used through this section.
\begin{definition}
	$F^B_{n,p}(.)$ is the CDF of the Binomial distribution with parameters $n$ and $p$, and $F^\beta_{a,b}(.)$ is the CDF of the Beta distribution with parameters $a$ and $b$.
\end{definition}

\begin{lemma} \label{fact:equation_for_beta_binomial_cdfs}
	Let $a$ and $b$ be positive integers and $y \in [0,1]$, 
	\[
	F^\beta_{a,b}(y) = 1 - F^B_{a+b-1,y}(a-1)
	\]
\end{lemma}
\begin{myproof}[Proof.]
	Proof is found in \cite{agrawalanalysis}.
\end{myproof}
\begin{lemma} \label{fact:median_of_binomial_dist}
	The median of a Binomial$(n,p)$ distribution is either $\ceil{np}$ or $\floor{np}$.
\end{lemma}
\begin{myproof}[Proof]
	A proof of this fact can be found in \cite{jogdeo1968monotone}.
\end{myproof}

\begin{corollary}[Corollary of Fact~\ref{fact:median_of_binomial_dist}] \label{cor:corollarly_of_binomial_median_property}
	Let $n$ be a positive integer and $p \in (0,1)$. For any non-negative integer $s < np$
	\[
	F^B_{n,p}(s) \le 1/2
	\]
\end{corollary}

\begin{lemma} \label{fact:relationship_with_binom_cdfs}
	Let $n$ be a positive integer and $p \in [0,1]$. Then for any $k \in \{0,\ldots,n\}$,
	\[
	(1-p)F^B_{n-1,p}(k)\le F^B_{n,p}(k) \le F^B_{n-1,p}(k)
	\] 
\end{lemma}
\begin{myproof}[Proof]
	To prove $F^B_{n,p}(k) \le F^B_{n-1,p}(k)$, we let $X_1,\ldots,X_{n}$ be i.i.d samples from a Bernoulli($p$) distribution. We then have
	\begin{align*}
	F^B_{n,p}(k)  = \P{\sum_{i=1}^{n} X_i \le k}  \le  \P{\sum_{i=1}^{n-1} X_i \le k}  = F^B_{n-1,p}(k)
	\end{align*}
	Now to prove $(1-p)F^B_{n-1,p}(k)\le F^B_{n,p}(k)$, it's enough to observe that $F^B_{n,p}(k) = p F^B_{n-1,p}(k-1) + (1-p) F^B_{n-1,p}(k)$.
\end{myproof}

\subsubsection{Ratio of Binomial CDFs.} \label{sec:ratio_of_bin_cdfs}
\begin{lemma} \label{lemma:ratio_of_cdfs}
	Let $0< q < p < 1$. Let $n$ be a positive integer such that $e^{\frac{n}{2} d(q,p)} \ge (n+1)^4$ and let $k$ be a non-negative integer such that $k < nq$. It then follows that
	\[
	F^B_{n,q}(k)/F^B_{n,p}(k) >  e^{\frac{n}{2} d(q,p)}.
	\]
\end{lemma}
\begin{proof}[Proof.]
	From the method of types  (see \cite{cover2012elements}), we have for any $r \in (0,1)$ and $j < nr$
	\begin{equation} \label{eqn:appl_of_sanovs}
	\frac{e^{-nd(j/n, r)}}{(1+n)^2}\le F^B_{n,r}(j) \le (n+1)^2 e^{- n d(j/n, r)}.
	\end{equation}
	Because $k < nq < np$, by applying \eqref{eqn:appl_of_sanovs} to both the numerator and denominator, we get
	\begin{align*}
	\frac{F^B_{n,q}(k)}{F^B_{n,p}(k)} & \ge  \frac{e^{-nd(k/n, q)}}{(n+1)^4 e^{- n d(k/n, p)}} = \frac{e^{n(d(k/n,p) - d(k/n,q))}}{(n+1)^4}.
	\end{align*}
	Examining the exponent, we find
	\begin{align*}
	d(k/n, p) - d(k/n,q) & = \frac{k}{n} \log \frac{q}{p} + \left(1-\frac{k}{n}\right)\log \frac{1-q}{1-p} \\
	& > q \log \frac{q}{p} + (1-q)\log \frac{1-q}{1-p} \\
	& = d(q,p)
	\end{align*}
	where the bound holds because the expression is decreasing in $k$, and $k < nq$. Therefore,
	\begin{align}
	\frac{F^B_{n,q}(k)}{F^B_{n,p}(k)} & > \frac{e^{n  d(q,p)}}{(n+1)^4} = \frac{e^{\frac{n}{2}d(q,p)}}{(n+1)^4} e^{\frac{n}{2}d(q,p)} \ge e^{\frac{n}{2}d(q,p)} \label{bound:log_1minusq_etc}.
	\end{align}
	The final lower bound in \eqref{bound:log_1minusq_etc} follows from the assumption on $n$.
\end{proof}

\subsection{Proof of Lemma~\ref{lemma:underestimation}} \label{proof:underestimation_proof}
\begin{myproof}[Proof.]
	The proof hinges on showing that for any $K$, which is the number of look-ahead steps used to compute the Optimistic Gittins index, that
	\begin{equation} \label{eqn:big_oh_result_for_ogi}
	\P{v^K_{1,t} < \eta} = O\left(\frac{1}{t^{1 + h_\eta}}\right)
	\end{equation}
	where $h_\eta > 0$ is some constant that depends on $\eta$. After showing the above statement, the result would follow due to convergence of the series $\sum_{t=1}^\infty \P{v^K_{1,t} < \eta}$. The first step will be to show that for any $K \ge 1$ and any $\zeta \ge 0$, that there exists $h'_\eta > 0$, such that
	\begin{equation} \label{eqn:big_oh_result_for_vk}
		\P{(1-\gamma_t)V^K_{\gamma_t}(y_{1,\Nt{1}}, \eta) < \eta + \zeta/t} = O_{\eta,\zeta}\left(\frac{1}{t^{1 + h'_\eta}}\right),
	\end{equation}
	where $V^K_{\gamma_t}$ is the continuation value defined in Section~\ref{sec:appendix_properties_of_ogi} and $O_{\eta,\zeta}$ means that the constant in front the big-Oh depends on both $\zeta$ and $\eta$. After showing the above claim, Lemma~\ref{cor:equivalent_event} would imply Equation~\eqref{eqn:big_oh_result_for_ogi} because we know from that result that,
	\begin{align*}
		\P{v^K_{1,t} < \eta} & = \P{(1-\gamma_t)V^K_{\gamma_t}(y_{1,\Nt{1}}, \eta) < \eta} \\
		& = O\left(\frac{1}{t^{1 + h_\eta}}\right)
	\end{align*}
	for some $h_\eta > 0$. The second equation above is just a special case of \eqref{eqn:big_oh_result_for_vk} when $\zeta = 0$.
	
	Ultimately, showing equation \eqref{eqn:big_oh_result_for_vk}, and thus proving the Lemma, is an induction over the parameter $K$ and we begin with the base case, which requires some work using properties of the Beta and Binomial distributions.
	\subsubsection*{Proof of the base case}
	Let us fix $\zeta \ge 0$. We prove that when the algorithm uses a look-ahead parameter of $K = 1$, that there exists a positive constant $h_\eta$ such that
	 \begin{equation} \label{eqn:base_case_lemma_underestimation}
	 \P{(1-\gamma_t)V^1_{\gamma_t}(y_{1,\Nt{1}}, \eta) < \eta + \zeta/t} = O_{\eta,\zeta}\left(\frac{1}{t^{1 + h_\eta}}\right).
	 \end{equation}
	 %To simplify notation, let us abbreviate $v^1_{1,t}$ as $v_{1,t}$. 
	 First, we define $\delta := (\theta_1 - \eta)/2$ and  $\eta' :=  \eta + \delta$. In other words, $\delta$ is half the distance between $\eta$ and $\theta_1$; $\eta'$ is the point half-way. Recall that $\Nt{i}$ refers to the counting process for the number of pulls of arm $i$ up to \emph{but not including} time $t$ and that $S_i(t)$ is the corresponding total reward (or number of successes from all the Bernoulli trials). Showing this base case consists of showing two claims:
	\subsubsection*{Claim 1: $\{(1-\gamma_t)V^1_{\gamma_t}(y_{1,\Nt{1}}, \eta) < \eta + \zeta/t\} \subseteq \left\{F^B_{\Nt{1}+1, \eta'}(S_1(t)) < \frac{\zeta + 1}{\delta t}\right\}$}
	Let $V_t \sim $Beta$(S_1(t)+1,\Nt{1} - S_1(t) + 1)$ be the agent's posterior on the expected reward from the optimal arm (notice that $y_{1,\Nt{1}} = (S_1(t)+1,\Nt{1} - S_1(t) + 1)$ in this case). Using the simplified equation for the continuation value when $K =1$, namely $V^1_{\gamma_t}$ (see Equation~\eqref{eqn:ogi_k1}), 
	\[
		(1-\gamma_t)V^1_{\gamma_t}\left((S_1(t)+1, \Nt{1} - S_1(t) + 1), \eta\right) = \Ee{V_t} + \gamma_t\Ee{(\eta - V_t)^+},
	\] 
	we find that
	\begin{align}
	\left\{(1-\gamma_t)V^1_{\gamma_t}(y_{1,N_{1}(t)}, \eta) < \eta + \frac{\zeta}{t}\right\} & = \left\{ \Ee{V_t } + \gamma_t\Ee{(\eta - V_t)^+} < \eta + \frac{\zeta}{t}\right\} \nonumber\\
	& =  \left\{ (1-1/t)\Ee{(\eta - V_t)^+} < \Ee{\eta - V_t} +\frac{\zeta}{t} \right\} \label{eq:def_gamma_t} \\
	& =  \left\{ \Ee{(\eta - V_t)^+} - \Ee{\eta - V_t} <  \frac{1}{t}\Ee{(\eta - V_t)^+} +\frac{\zeta}{t}\right\} \nonumber\\
	& =  \left\{ \Ee{(V_t - \eta)^+}<  \frac{1}{t}\Ee{(\eta - V_t)^+} +\frac{\zeta}{t}\right\} \nonumber\\
	& \subseteq \left\{ \Ee{ (V_t - \eta)^+}< \frac{\zeta + 1}{t}  \right\} \label{eq:intermediate_event}
	\end{align}
	where \eqref{eq:def_gamma_t} follows from the definition of $\gamma_t$ and \eqref{eq:intermediate_event} is due to $V_t, \eta$ both lying in the interval $[0,1]$. We approximate the conditional expectation in \eqref{eq:intermediate_event} with the following bound:
	\begin{align}
	\Ee{(V_t - \eta)^+ } & = \Ee{(V_t - \eta) \ind{V_t \ge \eta} }\nonumber \\
	& = \Ee{(V_t - \eta) \ind{\eta + \delta > V_t \ge \eta} }  \nonumber \\
	& \qquad + \Ee{(V_t - \eta) \ind{ V_t \ge \eta + \delta} } \nonumber \\
	& > \Ee{(V_t - \eta) \ind{ V_t \ge \eta + \delta} } \nonumber \\
	& \ge \delta\P{ V_t \ge \eta' } \nonumber \\
	& = \delta (1 - F^\beta_{S_1(t)+1,\Nt{1}-S_1(t)+1}(\eta'))  \nonumber\\ 
	& = \delta F^B_{\Nt{1}+1,\eta'}(S_1(t)) \label{ineq:lower_bound_on_ppart_term}
	\end{align}
	where the final equality is due to Fact~\ref{fact:equation_for_beta_binomial_cdfs}. The claim then follows from the above bound and \eqref{eq:intermediate_event}. We proceed with the second part of the base case's proof:
	\subsubsection*{Claim 2: $\mathbb{P}\left(F^B_{\Nt{1}+1, \eta'}(S_1(t)) < \frac{\zeta + 1}{\delta t}\right) = O\left(\frac{1}{t^{1 + h_\eta}}\right)$ for some $h_\eta > 0$}
	Let us fix the sequence $f_t \defeq -\frac{\log (\delta t/(\zeta+1)) }{\log (1-\eta')}-1 = O(\log t)$. We then have by a straightforward decomposition that
	\begin{align}
	\P{F^B_{\Nt{1}+1, \eta'}(S_1(t)) < \frac{\zeta + 1}{\delta t}} & = \P{F^B_{\Nt{1}+1, \eta'}(S_1(t)) < \frac{\zeta + 1}{\delta t}, \; \Nt{1} > f_t}  \nonumber \\
	& \qquad + \P{F^B_{\Nt{1}+1, \eta'}(S_1(t)) < \frac{\zeta + 1}{\delta t}, \; \Nt{1} \le f_t} \label{eq:decomp2}.
	\end{align}
	Then notice that for the second term in the RHS of \eqref{eq:decomp2} we have the following bound,
	\begin{align}
	\P{F^B_{\Nt{1}+1, \eta'}(S_1(t)) < \frac{\zeta + 1}{\delta t}, \; \Nt{1} \le f_t}  &  \le \P{F^B_{\Nt{1}+1,\eta'}(0) < \frac{\zeta + 1}{\delta  t}, \; \Nt{1} \le f_t} \nonumber \\
	& = \P{(1-\eta')^{\Nt{1}+1} <  \frac{\zeta + 1}{\delta  t}, \; \Nt{1} \le f_t} \nonumber \\
	& \le \P{(1-\eta')^{f_t+1} <  \frac{\zeta + 1}{\delta  t}} \nonumber \\
	& = 0. \label{bound:bdd_by_zero}
	\end{align}
	Now we use the following fact to correspondingly bound the left term on the RHS of \eqref{eq:decomp2}. Define the function
	\[
	F^{-B}_{n,p}(u) := \inf\{x : F^B_{n,p}(x) \ge u\}
	\]
	which is the inverse CDF. Then it is known that if $U \sim \text{Uniform}(0,1)$, then $F^{-B}_{n,p}(U) \sim \text{Binomial}(n,p)$. Furthermore, the event $F^B_{n,p}(F^{-B}_{n,p}(U)) \ge U$ occurs with probability 1 due to the definition of the inverse CDF.
	
	Now let us only consider large $t$, in particular $t > M = M(\theta_1, \eta')$ where:
	\begin{enumerate}
		\item $M$ is such that $e^{d(\eta', \theta_1)f_{M}/2} > (f_M + 1)^4$ (we need this condition when we use Lemma~\ref{lemma:ratio_of_cdfs})
		\item $M > \frac{4(\zeta + 1)}{(1-\eta')\delta }$
		\item $\ceil{f_M} > 0$ and $F^B_{t',\eta'}(t' \eta') > 1/4$ for all $t' > \ceil{f_M}$. Note that there is a large enough integer for this because $F^B_{\ceil{f_t},\eta'}(f_t \eta') \to \frac{1}{2}$ as $t \to \infty$.
	\end{enumerate} 
	Suppose that $t > M$. It then follows that the event \[\left\{F^B_{\Nt{1}, \eta'}(S_1(t)) < \frac{\zeta + 1}{(1-\eta')\delta t},\; S_1(t) \ge \Nt{1} \eta', \; \Nt{1} > f_t\right\}\] has measure zero because of the assumptions made on $M$. Therefore if $t > M$, we have
	\begin{align}
	\mathbb{P}\bigg(F^B_{\Nt{1}+1,  \eta'}(S_1(t)) &< \frac{\zeta + 1}{\delta t }  , \; \Nt{1} > f_t \bigg) \nonumber \\
	& \le \P{F^B_{\Nt{1},  \eta'}(S_1(t)) < \frac{\zeta + 1}{(1-\eta')\delta t}, \; \Nt{1} > f_t} \label{eqn:part1_decomp_the_cdf_of_y} \\ 
	& = \P{F^B_{\Nt{1},  \eta'}(S_1(t)) < \frac{\zeta + 1}{(1-\eta')\delta t}, \; S_1(t) < \Nt{1} \eta', \; \Nt{1} > f_t} \nonumber \\ 
	& =  \P{F^B_{\Nt{1},\theta_1}(S_1(t)) \frac{F^B_{\Nt{1},\eta'}(S_1(t))}{F^B_{\Nt{1},\theta_1}(S_1(t))} < \frac{\zeta + 1}{(1-\eta')\delta  t}, \;S_1(t) < \Nt{1} \eta', \; \Nt{1} > f_t} \nonumber \\
	& \le  \P{F_{\Nt{1},\theta_1}^B(S_1(t))  e^{\Nt{1} D} < \frac{\zeta + 1}{(1-\eta')\delta  t} , \; \Nt{1} > f_t} \label{eqn:part1_app_of_lemma2} \\
	& \le  \P{F_{\Nt{1},\theta_1}^B(S_1(t)) e^{f_t D} < \frac{\zeta + 1}{(1-\eta')\delta  t}} \nonumber \\
	& =  \P{F_{\Nt{1},\theta_1}^B(F^{-B}_{\Nt{1},\theta_1}(U)) < \frac{e^{-f_t D}(\zeta + 1)}{(1-\eta')\delta  t} } \label{eqn:part1_propert_of_inverse_sampling}\\
	& \le  \P{U < \frac{e^{-f_t D}(\zeta + 1)}{(1-\eta')\delta  t} } \nonumber \\  
	& =  \frac{e^{-f_t D}(\zeta + 1)}{(1-\eta')\delta  t} \nonumber  \nonumber\\
	& = \mathcal O_{\eta, \zeta}\left( \frac{1}{t^{1+Dc_{\eta'}}} \right)  \label{bound:one_over_t_plus_eps} 
	\end{align}
	where $D = d(\eta',\theta_1) > 0$ and $c_{\eta'} = -\log^{-1}(1-\eta') > 0$ are constant. The bound \eqref{eqn:part1_decomp_the_cdf_of_y} holds due to Fact~\eqref{fact:relationship_with_binom_cdfs}. Bound \eqref{eqn:part1_app_of_lemma2} follows from an application of Lemma~\ref{lemma:ratio_of_cdfs} and the fact that $t > M$. Equation \eqref{eqn:part1_propert_of_inverse_sampling} follows from $S_1(t) \sim \text{Binomial}(\Nt{1}, \theta_1)$ and the inverse sampling technique. By combining bounds \eqref{bound:one_over_t_plus_eps}, \eqref{bound:bdd_by_zero} and \eqref{eq:decomp2}, we finally obtain the result for the base case by taking $h_\eta = Dc_{\eta'}$.

	\subsubsection*{Proof of the inductive step}
	 Now, suppose that for $K-1 \ge 1$ and any $\zeta \ge 0$,  the following induction hypothesis holds
	\[
	\P{(1-\gamma_t)V^{K-1}_{\gamma_t}(y_{1,\Nt{1}}, \eta) < \eta + \frac{\zeta}{t}} = O_{\eta,\zeta}\left(\frac{1}{t^{1 + h_\eta}}\right)
	\]
	for some $h_\eta > 0$. We prove the same result for the next integer $K$. Observe that when $t$ is large enough, using the Bellman equation for $V^K_\gamma$, we have
	\begin{align}
	&\P{(1-\gamma_t)V^K_{\gamma_t}(y_{1,\Nt{1}}, \eta) < \eta + \frac{\zeta}{t}} \nonumber  \\
	&\qquad = \mathbb{P}\left((1-\gamma_t)\Ee{X_{1,t} \given y_{1,\Nt{1}}}  \right. \nonumber\\
	&\hspace{6em} + \gamma_t \Ee{\max(\eta, (1-\gamma_t)V^{K-1}_{\gamma_t}(y_{1,\Nt{1}+1}, \eta)) \given y_{1,\Nt{1}}} < \eta + \frac{\zeta}{t}\bigg) \label{eq:appl_of_lemma_9} \\
	& \qquad \le \P{ \left(1-\frac{1}{t}\right) \Ee{(1-\gamma_t)V^{K-1}_{\gamma_{t}}(y_{1,\Nt{1}+1}, \eta) \given y_{1,\Nt{1}} }< \eta + \frac{\zeta}{t}} \nonumber \\
	& \qquad \le \P{\left(1-\frac{1}{t}\right)(1-\gamma_t) V^{K-1}_{\gamma_{t}}(y_{1,\Nt{1}}, \eta)< \eta  + \frac{\zeta}{t}} \label{ineq:missing_step} \\
	& \qquad \le \P{ (1-\gamma_t)V^{K-1}_{\gamma_{t}}(y_{1,\Nt{1}}, \eta)< \frac{t}{t-1}\left(\eta +  \frac{\zeta}{t}}\right) \nonumber \\
	& \qquad \le \P{ (1-\gamma_t)V^{K-1}_{\gamma_{t}}(y_{1,\Nt{1}}, \eta)< \eta + \frac{\eta}{t-1} +  \frac{\zeta}{t-1}} \nonumber \\
	& \qquad \le \P{ (1-\gamma_t)V^{K-1}_{\gamma_{t}}(y_{1,\Nt{1}}, \eta)< \eta +   \frac{\zeta + 1}{t}} \nonumber \\
	&\qquad =  O_{\eta,\zeta}\left(\frac{1}{t^{1 + h_\eta}}\right) \label{eqn:ind_hyp}
	\end{align}
	where the final inequality holds when $t$ is large enough because $\eta < 1$, equation \eqref{eq:appl_of_lemma_9} results from an expansion of Bellman's equation and bound \eqref{ineq:missing_step} follows from Lemma~\ref{lemma:vk_bound}. %\eqref{eqn:another_appl_of_lemma_9} follows from both \eqref{bnd:conc_result} and Lemma~\ref{cor:equivalent_event}. 
	Finally, equation \eqref{eqn:ind_hyp} follows from the induction hypothesis.
	%We finish the proof by using the following asymptotic argument. Take $M$ to be a large enough integer, then we have, using the result of the preceding induction proof, that
	%\begin{align*}
	%\sum_{t=1}^\infty \P{v^K_{1,t} < \eta} & \le M +  \sum_{t=M+1}^\infty \frac{C_1}{t^{1 + h(\eta)}} \le M +  C_2
	%\end{align*}
	%where $C_2 = C_2(\eta)$ is the limit of the series and $C_1$ is a constant used in the definition of the big-Oh.
\end{myproof}

\subsection{Proof of Lemma~\ref{lemma:overestimation}} \label{proof:overestimation_proof}

\begin{proof}[Proof.]
	See the main proof of Theorem~\ref{thm:frequentist_optimal_bound} to recall the definition of constants $\eta_1$, $\eta_3$ and their relationship with $\theta_2$ and $\theta_1$. As an abbreviation we let $L = L(T)$. Moreover, because for any arm $i$ $v^K_{i,t} \le v^{K-1}_{i,t} \le \ldots \le v^1_{i,t}$ (Lemma~\ref{lemma:approx_bound}), it will be sufficient to consider this proof only for $v^1_{2,t}$, which we also will abbreviate as $v_{2,t} \defeq v^1_{2,t}$. Similarly, we will abbreviate the notation for the OGI policy as $\pi^{OG}$ and suppress the parameter $K$.
	
	Firstly, by the law of total probability and the definition of $P_i(t)$ in Section~\ref{sec:appendix_properties_of_ogi}, we find that
	\begin{align} 
	\sum_{t=1}^T \mathbb{P}(v_{2,t} & \ge \eta_3 ,\; N_{2}(t-1) \ge L,\; \pi^{\rm OG}_t = 2) \nonumber \\
	& = \sum_{t=1}^T \P{v_{2,t} \ge \eta_3 ,\; \Nt{2} \ge L, \; S_2(t) < \floor{\Nt{2} \eta_1}, \; \pi^{\rm OG}_t = 2} \nonumber \\
	& \qquad + \sum_{t=1}^T \P{v_{2,t} \ge \eta_3 ,\; \Nt{2} \ge L, \; S_2(t) \ge \floor{\Nt{2} \eta_1},\; \pi^{\rm OG}_t = 2} \nonumber \\
	& \le \sum_{t=1}^T \P{v_{2,t} \ge \eta_3 ,\; \Nt{2} \ge L, \; S_2(t) < \floor{\Nt{2} \eta_1}} + \sum_{t=1}^T \P{\pi^{\rm OG}_t = 2,\; S_2(t) \ge \floor{\Nt{2} \eta_1}} \label{eqn:splitting_not_underestimate},
	\end{align}
	where $S_2(t)$ is also defined in Section~\ref{sec:appendix_properties_of_ogi} as the total reward from the second arm observed up to time $t-1$. Let $V_t \sim \text{Beta}(S_2(t) + 1, \Nt{2}- S_2(t) + 1)$ denote the agent's posterior on the second arm at time $t$, then
	\begin{align}
	\sum_{t=1}^T \mathbb{P}(v_{2,t} \ge \eta_3 ,\; & \; \Nt{2} \ge L,\; S_2(t) < \floor{\Nt{2} \eta_1})  \nonumber\\
	& = \sum_{t=1}^T \P{\Ee{V_t} + \gamma_t \Ee{(\eta_3 - V_t)^+} \ge \eta_3, \; \Nt{2} \ge L,\; S_2(t) < \floor{\Nt{2} \eta_1}} \nonumber \\
	& = \sum_{t=1}^T \P{\frac{\Ee{(\eta_3-V_t)^+ }}{  \Ee{(V_t - \eta_3)^+ }} \le t , \; \Nt{2} \ge L,\; S_2(t) < \floor{\Nt{2} \eta_1}} \label{eq:complicated_rv_in_part2}
	\end{align}
	where the first equality follows from Lemma~\ref{cor:equivalent_event} and the simplified form of the continuation value (defined in Section~\ref{sec:appendix_properties_of_ogi}) when $K = 1$. The following result lets us bound \eqref{eq:complicated_rv_in_part2},
	\begin{lemma} \label{lem:lb_rv2}
		Let $0 < x < y < 1$. For any non-negative integers $s$ and $k$ with $s < \floor{kx}$, it holds that
		\begin{equation*}
		\frac{\Ee{(y-V)^+ }}{  \Ee{(V - y)^+ } } \ge \frac{(y-x) \exp(k d(x,y))}{2}
		\end{equation*}
		where $V \sim \text{Beta}(s+1,k-s+1)$.
	\end{lemma}
	\begin{myproof}[Proof.]
		See Appendix~\ref{prf:proof_of_lb_rv2}.
	\end{myproof}
	Therefore, from equation \eqref{eq:complicated_rv_in_part2} and Lemma~\ref{lem:lb_rv2}, we find that whenever $T > \left(\frac{2}{\eta_3-\eta_1}\right)^{1/\eps} =: T^*(\eps, \theta)$,
	\begin{align}
	\sum_{t=1}^T \mathbb{P}(v_{2,t} \ge \eta_3 ,\; & \; \Nt{2} \ge L,\; S_2(t) < \floor{\Nt{2} \eta_1}) \nonumber \\
	& \le  \sum_{t=1}^T\P{  (\eta_3-\eta_1) \exp\{\Nt{2} d(\eta_1,\eta_3) \} \le 2t,\; \Nt{2} \ge L} \nonumber \\
	& \le  \sum_{t=1}^T\P{  (\eta_3-\eta_1) \exp\{L d(\eta_1,\eta_3) \} \le 2t} \nonumber \\
	& =   \sum_{t=1}^T\P{  (\eta_3-\eta_1) T^{1+\eps} \le 2t} \nonumber \\
	& = 0. \label{bound:equal_to_zero}
	\end{align}
	All that is left is to bound the second term in \eqref{eqn:splitting_not_underestimate}, and to do so we apply the following Lemma whose proof is in Appendix~\ref{prf:proof_of_acc_sub_means}
	\begin{lemma} \label{lem:accurate_suboptimal_mean}
		There exist positive constants $C = C(\theta_2,\eta_1)$ and $x' > \theta_2$ such that
		\begin{equation*}
		\sum_{t=1}^T \P{S_2(t) \ge \floor{\Nt{2} \eta_1}, \; \pi^{\rm OG}_t = 2} \le  K + \frac{1}{1 - e^{-d(x',\theta_2)}} 
		\end{equation*}
	\end{lemma}
	Combining Lemma~\ref{lem:accurate_suboptimal_mean}, \eqref{bound:equal_to_zero}, \eqref{eqn:splitting_not_underestimate} and \eqref{eq:complicated_rv_in_part2} shows the claim.
\end{proof}

\subsubsection{Proof of Lemma~\ref{lem:lb_rv2}.} \label{prf:proof_of_lb_rv2}
\begin{myproof}[Proof.]
	We upper bound the denominator as follows. Given that $s < \floor{k x}$, we have $s \le kx - 1$. Let $B(a,b)$ denote the Beta function for parameters $a, b > 0$, that is
	\[
	B(a, b) \defeq \int_0^1 t^{a-1}(1-t)^{b-1} \;dt,
	\]
	which is used in the definition of the Beta CDF. We can derive an upper bound on the denominator in the following way. Namely, we have
	\begin{align}
	\Ee{(V - y)^+ } & = \frac{1}{B(s+1,k-s+1)}\int_{y}^1 (t-y) t^s (1-t)^{k-s} \; dt \nonumber \\
	& = \frac{1}{B(s+1,k-s+1)}\int_{y}^1 t^{s+1} (1-t)^{k-s} \; dt - y \P{V \ge y} \nonumber \\
	& = \frac{B(s+2,k-s+1)}{B(s+1,j-s+1)}\left( \frac{1}{B(s+2,k-s+1)} \right)\int_{y}^1 t^{s+1} (1-t)^{k-s}\; dt - y \P{V \ge y} \nonumber \\
	& = \frac{s+1}{k+2} F^B_{k+2,y}(s+1)  - y \P{V \ge y} \label{eq:part2_use_of_equiv_between_beta_and_binom} \\
	& \le \frac{s+1}{k+2} F^B_{k+2,y}(s+1) \nonumber \\
	& \le  F^B_{k,y}(k x)  \nonumber \\
	& \le \exp\left\{- k d(x,y) \label{ineq:chernoff_app} \right\}
	\end{align}
	where we use Fact~\ref{fact:equation_for_beta_binomial_cdfs} and the definition of the Beta CDF to establish equation \eqref{eq:part2_use_of_equiv_between_beta_and_binom}. The final bound in \eqref{ineq:chernoff_app} is the result of the Chernoff-Hoeffding theorem and Fact~\ref{fact:relationship_with_binom_cdfs}. Let $\delta:=y-x$, and note that $s < kx \Longrightarrow s \le \floor{(k+1)x}$ due to $s$ being integer, whence
	\begin{align}
	\Ee{(y - V)^+ } & =  \Ee{(y - V) \ind{V \le y}} \nonumber \\
	& = \Ee{(y - V) \ind{y - \delta \le V \le y} } +  \Ee{(y - V) \ind{V < y - \delta} } \nonumber\\
	& > \Ee{(y - V) \ind{V < y - \delta} }\nonumber \\
	& \ge \delta\Ee{\ind{V < y-\delta} }\\
	& = \delta \P{V < x } \nonumber \\
	& = \delta\left(1 - F^B_{k+1,x}(s) \right) \label{eq:use_of_bin_beta_identity}  \\
	& \ge \delta/2  \label{eq:use_of_median_prop}
	\end{align}
	where equation \eqref{eq:use_of_bin_beta_identity} relies on Fact~\ref{fact:equation_for_beta_binomial_cdfs}. The bound \eqref{eq:use_of_median_prop} is justified from Fact~\ref{fact:median_of_binomial_dist} and $s \le \floor{(k+1) x}$. Thus using the inequalities for both the numerator and denominator, we obtain the desired bound.
\end{myproof}
\subsubsection{Proof of Lemma~\ref{lem:accurate_suboptimal_mean}.} \label{prf:proof_of_acc_sub_means}
\begin{proof}[Proof.]
	The steps in this proof follow a similar one in \cite{agrawal2013further} but we show them for completeness. We bound the number of times the sub-optimal arm's mean is overestimated. Let $\tau_\ell$ be the time step in which the  sub-optimal arm is sampled for the $\ell$\textsuperscript{th} time. Because for any $x$, $\lim_{n\to\infty}\frac{\floor{nx}}{nx} = 1$, we can let $N$ be a large enough integer so that if $\ell \ge N$, then $\eta_1 \frac{\floor{\ell \eta_1}}{\ell \eta_1} > x' := (\theta_2 + \eta_1)/2 > \theta_2$. In that case,
	\begin{align}
	\sum_{t=1}^T\P{S_2(t) \ge \floor{\Nt{2} \eta_1}, \; \pi^{\rm OG}_t = 2} & \le \Ee{\sum_{\ell=1}^T \sum_{t=\tau_\ell}^{\tau_{\ell+1}-1}\ind{S_2(\ell) \ge \floor{\Ntg{2}{\ell} \eta_1}} \ind{\pi^{\rm OG}_t = 2}} \nonumber \\
	& = \Ee{\sum_{\ell=1}^T \ind{S_2(\tau_{\ell}) \ge \floor{(\ell-1) \eta_1}} \sum_{t=\tau_\ell}^{\tau_{\ell+1}-1} \ind{\pi^{\rm OG}_t = 2}} \nonumber\\
	& = \Ee{\sum_{\ell=0}^{T-1} \ind{S_2(\tau_{\ell+1}) \ge \floor{\ell \eta_1}}} \nonumber\\
	& \le  N + \sum_{\ell=N+1}^{T-1} \P{ S_2(\tau_{\ell+1}) \ge \ell \eta_1 \frac{\floor{\ell \eta_1}}{\ell \eta_1}} \nonumber \\
	& \le N + \sum_{\ell=N+1}^{T-1} \P{ S_2(\tau_{\ell+1}) \ge \ell x'} \nonumber \\
	& \le  N + \sum_{\ell=1}^{\infty} \exp(-\ell d(x', \theta_2)) \label{bound:cf_thm} \\
	& = N + \frac{1}{1 - e^{-d(x',\theta_2)}} \nonumber
	\end{align}
	where \eqref{bound:cf_thm} follows from the Chernoff-Hoeffding theorem and the fact that $S_2(\tau_{\ell+1})$ is drawn from a $\text{Binomial}(\Ntg{2}{\ell+1}, \theta_2) \equiv \text{Binomial}(\ell, \theta_2)$ distribution.
\end{proof}

\section{Further experiment results} \label{sec:further_exp}
\subsection{Bayes UCB experiment} \label{exp:bayes_ucb}
This experiment is motivated by \cite{kaufmann2012bayesian} and in it we simulate the Bernoulli bandit problem with a $T = 500$ and two arms. Since we are interested in measuring expected regret over the prior, we draw the arms' mean rewards at random from the uniform distribution. There are 5,000 independent trials and we show the results in Figures~\ref{fig:kaufmann_regret}. OGI offers notable performance improvements over both Thompson Sampling and IDS for this modest horizon.
\begin{figure}[h!]
	\centering
	%% Creator: Matplotlib, PGF backend
%%
%% To include the figure in your LaTeX document, write
%%   \input{<filename>.pgf}
%%
%% Make sure the required packages are loaded in your preamble
%%   \usepackage{pgf}
%%
%% Figures using additional raster images can only be included by \input if
%% they are in the same directory as the main LaTeX file. For loading figures
%% from other directories you can use the `import` package
%%   \usepackage{import}
%% and then include the figures with
%%   \import{<path to file>}{<filename>.pgf}
%%
%% Matplotlib used the following preamble
%%   \usepackage[utf8x]{inputenc}
%%   \usepackage[T1]{fontenc}
%%
\begingroup%
\makeatletter%
\begin{pgfpicture}%
\pgfpathrectangle{\pgfpointorigin}{\pgfqpoint{6.099066in}{3.769430in}}%
\pgfusepath{use as bounding box, clip}%
\begin{pgfscope}%
\pgfsetbuttcap%
\pgfsetmiterjoin%
\definecolor{currentfill}{rgb}{1.000000,1.000000,1.000000}%
\pgfsetfillcolor{currentfill}%
\pgfsetlinewidth{0.000000pt}%
\definecolor{currentstroke}{rgb}{1.000000,1.000000,1.000000}%
\pgfsetstrokecolor{currentstroke}%
\pgfsetdash{}{0pt}%
\pgfpathmoveto{\pgfqpoint{0.000000in}{0.000000in}}%
\pgfpathlineto{\pgfqpoint{6.099066in}{0.000000in}}%
\pgfpathlineto{\pgfqpoint{6.099066in}{3.769430in}}%
\pgfpathlineto{\pgfqpoint{0.000000in}{3.769430in}}%
\pgfpathclose%
\pgfusepath{fill}%
\end{pgfscope}%
\begin{pgfscope}%
\pgfsetbuttcap%
\pgfsetmiterjoin%
\definecolor{currentfill}{rgb}{1.000000,1.000000,1.000000}%
\pgfsetfillcolor{currentfill}%
\pgfsetlinewidth{0.000000pt}%
\definecolor{currentstroke}{rgb}{0.000000,0.000000,0.000000}%
\pgfsetstrokecolor{currentstroke}%
\pgfsetstrokeopacity{0.000000}%
\pgfsetdash{}{0pt}%
\pgfpathmoveto{\pgfqpoint{0.762383in}{0.471179in}}%
\pgfpathlineto{\pgfqpoint{5.489159in}{0.471179in}}%
\pgfpathlineto{\pgfqpoint{5.489159in}{3.317098in}}%
\pgfpathlineto{\pgfqpoint{0.762383in}{3.317098in}}%
\pgfpathclose%
\pgfusepath{fill}%
\end{pgfscope}%
\begin{pgfscope}%
\pgfsetbuttcap%
\pgfsetroundjoin%
\definecolor{currentfill}{rgb}{0.000000,0.000000,0.000000}%
\pgfsetfillcolor{currentfill}%
\pgfsetlinewidth{0.803000pt}%
\definecolor{currentstroke}{rgb}{0.000000,0.000000,0.000000}%
\pgfsetstrokecolor{currentstroke}%
\pgfsetdash{}{0pt}%
\pgfsys@defobject{currentmarker}{\pgfqpoint{0.000000in}{-0.048611in}}{\pgfqpoint{0.000000in}{0.000000in}}{%
\pgfpathmoveto{\pgfqpoint{0.000000in}{0.000000in}}%
\pgfpathlineto{\pgfqpoint{0.000000in}{-0.048611in}}%
\pgfusepath{stroke,fill}%
}%
\begin{pgfscope}%
\pgfsys@transformshift{0.762383in}{0.471179in}%
\pgfsys@useobject{currentmarker}{}%
\end{pgfscope}%
\end{pgfscope}%
\begin{pgfscope}%
\pgftext[x=0.762383in,y=0.373957in,,top]{\rmfamily\fontsize{8.000000}{9.600000}\selectfont \(\displaystyle 0\)}%
\end{pgfscope}%
\begin{pgfscope}%
\pgfsetbuttcap%
\pgfsetroundjoin%
\definecolor{currentfill}{rgb}{0.000000,0.000000,0.000000}%
\pgfsetfillcolor{currentfill}%
\pgfsetlinewidth{0.803000pt}%
\definecolor{currentstroke}{rgb}{0.000000,0.000000,0.000000}%
\pgfsetstrokecolor{currentstroke}%
\pgfsetdash{}{0pt}%
\pgfsys@defobject{currentmarker}{\pgfqpoint{0.000000in}{-0.048611in}}{\pgfqpoint{0.000000in}{0.000000in}}{%
\pgfpathmoveto{\pgfqpoint{0.000000in}{0.000000in}}%
\pgfpathlineto{\pgfqpoint{0.000000in}{-0.048611in}}%
\pgfusepath{stroke,fill}%
}%
\begin{pgfscope}%
\pgfsys@transformshift{1.709633in}{0.471179in}%
\pgfsys@useobject{currentmarker}{}%
\end{pgfscope}%
\end{pgfscope}%
\begin{pgfscope}%
\pgftext[x=1.709633in,y=0.373957in,,top]{\rmfamily\fontsize{8.000000}{9.600000}\selectfont \(\displaystyle 100\)}%
\end{pgfscope}%
\begin{pgfscope}%
\pgfsetbuttcap%
\pgfsetroundjoin%
\definecolor{currentfill}{rgb}{0.000000,0.000000,0.000000}%
\pgfsetfillcolor{currentfill}%
\pgfsetlinewidth{0.803000pt}%
\definecolor{currentstroke}{rgb}{0.000000,0.000000,0.000000}%
\pgfsetstrokecolor{currentstroke}%
\pgfsetdash{}{0pt}%
\pgfsys@defobject{currentmarker}{\pgfqpoint{0.000000in}{-0.048611in}}{\pgfqpoint{0.000000in}{0.000000in}}{%
\pgfpathmoveto{\pgfqpoint{0.000000in}{0.000000in}}%
\pgfpathlineto{\pgfqpoint{0.000000in}{-0.048611in}}%
\pgfusepath{stroke,fill}%
}%
\begin{pgfscope}%
\pgfsys@transformshift{2.656883in}{0.471179in}%
\pgfsys@useobject{currentmarker}{}%
\end{pgfscope}%
\end{pgfscope}%
\begin{pgfscope}%
\pgftext[x=2.656883in,y=0.373957in,,top]{\rmfamily\fontsize{8.000000}{9.600000}\selectfont \(\displaystyle 200\)}%
\end{pgfscope}%
\begin{pgfscope}%
\pgfsetbuttcap%
\pgfsetroundjoin%
\definecolor{currentfill}{rgb}{0.000000,0.000000,0.000000}%
\pgfsetfillcolor{currentfill}%
\pgfsetlinewidth{0.803000pt}%
\definecolor{currentstroke}{rgb}{0.000000,0.000000,0.000000}%
\pgfsetstrokecolor{currentstroke}%
\pgfsetdash{}{0pt}%
\pgfsys@defobject{currentmarker}{\pgfqpoint{0.000000in}{-0.048611in}}{\pgfqpoint{0.000000in}{0.000000in}}{%
\pgfpathmoveto{\pgfqpoint{0.000000in}{0.000000in}}%
\pgfpathlineto{\pgfqpoint{0.000000in}{-0.048611in}}%
\pgfusepath{stroke,fill}%
}%
\begin{pgfscope}%
\pgfsys@transformshift{3.604132in}{0.471179in}%
\pgfsys@useobject{currentmarker}{}%
\end{pgfscope}%
\end{pgfscope}%
\begin{pgfscope}%
\pgftext[x=3.604132in,y=0.373957in,,top]{\rmfamily\fontsize{8.000000}{9.600000}\selectfont \(\displaystyle 300\)}%
\end{pgfscope}%
\begin{pgfscope}%
\pgfsetbuttcap%
\pgfsetroundjoin%
\definecolor{currentfill}{rgb}{0.000000,0.000000,0.000000}%
\pgfsetfillcolor{currentfill}%
\pgfsetlinewidth{0.803000pt}%
\definecolor{currentstroke}{rgb}{0.000000,0.000000,0.000000}%
\pgfsetstrokecolor{currentstroke}%
\pgfsetdash{}{0pt}%
\pgfsys@defobject{currentmarker}{\pgfqpoint{0.000000in}{-0.048611in}}{\pgfqpoint{0.000000in}{0.000000in}}{%
\pgfpathmoveto{\pgfqpoint{0.000000in}{0.000000in}}%
\pgfpathlineto{\pgfqpoint{0.000000in}{-0.048611in}}%
\pgfusepath{stroke,fill}%
}%
\begin{pgfscope}%
\pgfsys@transformshift{4.551382in}{0.471179in}%
\pgfsys@useobject{currentmarker}{}%
\end{pgfscope}%
\end{pgfscope}%
\begin{pgfscope}%
\pgftext[x=4.551382in,y=0.373957in,,top]{\rmfamily\fontsize{8.000000}{9.600000}\selectfont \(\displaystyle 400\)}%
\end{pgfscope}%
\begin{pgfscope}%
\pgfsetbuttcap%
\pgfsetroundjoin%
\definecolor{currentfill}{rgb}{0.000000,0.000000,0.000000}%
\pgfsetfillcolor{currentfill}%
\pgfsetlinewidth{0.803000pt}%
\definecolor{currentstroke}{rgb}{0.000000,0.000000,0.000000}%
\pgfsetstrokecolor{currentstroke}%
\pgfsetdash{}{0pt}%
\pgfsys@defobject{currentmarker}{\pgfqpoint{-0.048611in}{0.000000in}}{\pgfqpoint{0.000000in}{0.000000in}}{%
\pgfpathmoveto{\pgfqpoint{0.000000in}{0.000000in}}%
\pgfpathlineto{\pgfqpoint{-0.048611in}{0.000000in}}%
\pgfusepath{stroke,fill}%
}%
\begin{pgfscope}%
\pgfsys@transformshift{0.762383in}{0.515648in}%
\pgfsys@useobject{currentmarker}{}%
\end{pgfscope}%
\end{pgfscope}%
\begin{pgfscope}%
\pgftext[x=0.606132in,y=0.477386in,left,base]{\rmfamily\fontsize{8.000000}{9.600000}\selectfont \(\displaystyle 0\)}%
\end{pgfscope}%
\begin{pgfscope}%
\pgfsetbuttcap%
\pgfsetroundjoin%
\definecolor{currentfill}{rgb}{0.000000,0.000000,0.000000}%
\pgfsetfillcolor{currentfill}%
\pgfsetlinewidth{0.803000pt}%
\definecolor{currentstroke}{rgb}{0.000000,0.000000,0.000000}%
\pgfsetstrokecolor{currentstroke}%
\pgfsetdash{}{0pt}%
\pgfsys@defobject{currentmarker}{\pgfqpoint{-0.048611in}{0.000000in}}{\pgfqpoint{0.000000in}{0.000000in}}{%
\pgfpathmoveto{\pgfqpoint{0.000000in}{0.000000in}}%
\pgfpathlineto{\pgfqpoint{-0.048611in}{0.000000in}}%
\pgfusepath{stroke,fill}%
}%
\begin{pgfscope}%
\pgfsys@transformshift{0.762383in}{1.005062in}%
\pgfsys@useobject{currentmarker}{}%
\end{pgfscope}%
\end{pgfscope}%
\begin{pgfscope}%
\pgftext[x=0.606132in,y=0.966800in,left,base]{\rmfamily\fontsize{8.000000}{9.600000}\selectfont \(\displaystyle 1\)}%
\end{pgfscope}%
\begin{pgfscope}%
\pgfsetbuttcap%
\pgfsetroundjoin%
\definecolor{currentfill}{rgb}{0.000000,0.000000,0.000000}%
\pgfsetfillcolor{currentfill}%
\pgfsetlinewidth{0.803000pt}%
\definecolor{currentstroke}{rgb}{0.000000,0.000000,0.000000}%
\pgfsetstrokecolor{currentstroke}%
\pgfsetdash{}{0pt}%
\pgfsys@defobject{currentmarker}{\pgfqpoint{-0.048611in}{0.000000in}}{\pgfqpoint{0.000000in}{0.000000in}}{%
\pgfpathmoveto{\pgfqpoint{0.000000in}{0.000000in}}%
\pgfpathlineto{\pgfqpoint{-0.048611in}{0.000000in}}%
\pgfusepath{stroke,fill}%
}%
\begin{pgfscope}%
\pgfsys@transformshift{0.762383in}{1.494477in}%
\pgfsys@useobject{currentmarker}{}%
\end{pgfscope}%
\end{pgfscope}%
\begin{pgfscope}%
\pgftext[x=0.606132in,y=1.456215in,left,base]{\rmfamily\fontsize{8.000000}{9.600000}\selectfont \(\displaystyle 2\)}%
\end{pgfscope}%
\begin{pgfscope}%
\pgfsetbuttcap%
\pgfsetroundjoin%
\definecolor{currentfill}{rgb}{0.000000,0.000000,0.000000}%
\pgfsetfillcolor{currentfill}%
\pgfsetlinewidth{0.803000pt}%
\definecolor{currentstroke}{rgb}{0.000000,0.000000,0.000000}%
\pgfsetstrokecolor{currentstroke}%
\pgfsetdash{}{0pt}%
\pgfsys@defobject{currentmarker}{\pgfqpoint{-0.048611in}{0.000000in}}{\pgfqpoint{0.000000in}{0.000000in}}{%
\pgfpathmoveto{\pgfqpoint{0.000000in}{0.000000in}}%
\pgfpathlineto{\pgfqpoint{-0.048611in}{0.000000in}}%
\pgfusepath{stroke,fill}%
}%
\begin{pgfscope}%
\pgfsys@transformshift{0.762383in}{1.983891in}%
\pgfsys@useobject{currentmarker}{}%
\end{pgfscope}%
\end{pgfscope}%
\begin{pgfscope}%
\pgftext[x=0.606132in,y=1.945629in,left,base]{\rmfamily\fontsize{8.000000}{9.600000}\selectfont \(\displaystyle 3\)}%
\end{pgfscope}%
\begin{pgfscope}%
\pgfsetbuttcap%
\pgfsetroundjoin%
\definecolor{currentfill}{rgb}{0.000000,0.000000,0.000000}%
\pgfsetfillcolor{currentfill}%
\pgfsetlinewidth{0.803000pt}%
\definecolor{currentstroke}{rgb}{0.000000,0.000000,0.000000}%
\pgfsetstrokecolor{currentstroke}%
\pgfsetdash{}{0pt}%
\pgfsys@defobject{currentmarker}{\pgfqpoint{-0.048611in}{0.000000in}}{\pgfqpoint{0.000000in}{0.000000in}}{%
\pgfpathmoveto{\pgfqpoint{0.000000in}{0.000000in}}%
\pgfpathlineto{\pgfqpoint{-0.048611in}{0.000000in}}%
\pgfusepath{stroke,fill}%
}%
\begin{pgfscope}%
\pgfsys@transformshift{0.762383in}{2.473306in}%
\pgfsys@useobject{currentmarker}{}%
\end{pgfscope}%
\end{pgfscope}%
\begin{pgfscope}%
\pgftext[x=0.606132in,y=2.435044in,left,base]{\rmfamily\fontsize{8.000000}{9.600000}\selectfont \(\displaystyle 4\)}%
\end{pgfscope}%
\begin{pgfscope}%
\pgfsetbuttcap%
\pgfsetroundjoin%
\definecolor{currentfill}{rgb}{0.000000,0.000000,0.000000}%
\pgfsetfillcolor{currentfill}%
\pgfsetlinewidth{0.803000pt}%
\definecolor{currentstroke}{rgb}{0.000000,0.000000,0.000000}%
\pgfsetstrokecolor{currentstroke}%
\pgfsetdash{}{0pt}%
\pgfsys@defobject{currentmarker}{\pgfqpoint{-0.048611in}{0.000000in}}{\pgfqpoint{0.000000in}{0.000000in}}{%
\pgfpathmoveto{\pgfqpoint{0.000000in}{0.000000in}}%
\pgfpathlineto{\pgfqpoint{-0.048611in}{0.000000in}}%
\pgfusepath{stroke,fill}%
}%
\begin{pgfscope}%
\pgfsys@transformshift{0.762383in}{2.962720in}%
\pgfsys@useobject{currentmarker}{}%
\end{pgfscope}%
\end{pgfscope}%
\begin{pgfscope}%
\pgftext[x=0.606132in,y=2.924458in,left,base]{\rmfamily\fontsize{8.000000}{9.600000}\selectfont \(\displaystyle 5\)}%
\end{pgfscope}%
\begin{pgfscope}%
\pgfpathrectangle{\pgfqpoint{0.762383in}{0.471179in}}{\pgfqpoint{4.726776in}{2.845920in}} %
\pgfusepath{clip}%
\pgfsetrectcap%
\pgfsetroundjoin%
\pgfsetlinewidth{1.505625pt}%
\definecolor{currentstroke}{rgb}{0.121569,0.466667,0.705882}%
\pgfsetstrokecolor{currentstroke}%
\pgfsetdash{}{0pt}%
\pgfpathmoveto{\pgfqpoint{0.762383in}{0.600539in}}%
\pgfpathlineto{\pgfqpoint{0.781328in}{0.681933in}}%
\pgfpathlineto{\pgfqpoint{0.800273in}{0.745512in}}%
\pgfpathlineto{\pgfqpoint{0.838163in}{0.820597in}}%
\pgfpathlineto{\pgfqpoint{0.847636in}{0.841864in}}%
\pgfpathlineto{\pgfqpoint{0.857108in}{0.854517in}}%
\pgfpathlineto{\pgfqpoint{0.885526in}{0.907454in}}%
\pgfpathlineto{\pgfqpoint{0.904471in}{0.934131in}}%
\pgfpathlineto{\pgfqpoint{0.923416in}{0.954348in}}%
\pgfpathlineto{\pgfqpoint{0.942361in}{0.970748in}}%
\pgfpathlineto{\pgfqpoint{0.951833in}{0.985751in}}%
\pgfpathlineto{\pgfqpoint{0.961306in}{0.993412in}}%
\pgfpathlineto{\pgfqpoint{0.970778in}{1.007142in}}%
\pgfpathlineto{\pgfqpoint{0.980251in}{1.015095in}}%
\pgfpathlineto{\pgfqpoint{0.989723in}{1.028241in}}%
\pgfpathlineto{\pgfqpoint{0.999196in}{1.034236in}}%
\pgfpathlineto{\pgfqpoint{1.008668in}{1.042879in}}%
\pgfpathlineto{\pgfqpoint{1.018141in}{1.046819in}}%
\pgfpathlineto{\pgfqpoint{1.027613in}{1.054875in}}%
\pgfpathlineto{\pgfqpoint{1.046558in}{1.064810in}}%
\pgfpathlineto{\pgfqpoint{1.056031in}{1.071104in}}%
\pgfpathlineto{\pgfqpoint{1.065503in}{1.084245in}}%
\pgfpathlineto{\pgfqpoint{1.074976in}{1.089266in}}%
\pgfpathlineto{\pgfqpoint{1.084448in}{1.100841in}}%
\pgfpathlineto{\pgfqpoint{1.112866in}{1.119810in}}%
\pgfpathlineto{\pgfqpoint{1.122338in}{1.124636in}}%
\pgfpathlineto{\pgfqpoint{1.131811in}{1.127401in}}%
\pgfpathlineto{\pgfqpoint{1.141283in}{1.136827in}}%
\pgfpathlineto{\pgfqpoint{1.160228in}{1.145103in}}%
\pgfpathlineto{\pgfqpoint{1.169701in}{1.154231in}}%
\pgfpathlineto{\pgfqpoint{1.207591in}{1.171958in}}%
\pgfpathlineto{\pgfqpoint{1.217063in}{1.175310in}}%
\pgfpathlineto{\pgfqpoint{1.236008in}{1.192885in}}%
\pgfpathlineto{\pgfqpoint{1.245481in}{1.195557in}}%
\pgfpathlineto{\pgfqpoint{1.254953in}{1.201944in}}%
\pgfpathlineto{\pgfqpoint{1.264426in}{1.205982in}}%
\pgfpathlineto{\pgfqpoint{1.302316in}{1.229679in}}%
\pgfpathlineto{\pgfqpoint{1.321261in}{1.234236in}}%
\pgfpathlineto{\pgfqpoint{1.340206in}{1.244567in}}%
\pgfpathlineto{\pgfqpoint{1.349678in}{1.251933in}}%
\pgfpathlineto{\pgfqpoint{1.387568in}{1.268485in}}%
\pgfpathlineto{\pgfqpoint{1.397041in}{1.275067in}}%
\pgfpathlineto{\pgfqpoint{1.415986in}{1.279037in}}%
\pgfpathlineto{\pgfqpoint{1.425458in}{1.282688in}}%
\pgfpathlineto{\pgfqpoint{1.444403in}{1.294390in}}%
\pgfpathlineto{\pgfqpoint{1.463348in}{1.297869in}}%
\pgfpathlineto{\pgfqpoint{1.472821in}{1.301907in}}%
\pgfpathlineto{\pgfqpoint{1.482293in}{1.303693in}}%
\pgfpathlineto{\pgfqpoint{1.491766in}{1.310379in}}%
\pgfpathlineto{\pgfqpoint{1.501238in}{1.321268in}}%
\pgfpathlineto{\pgfqpoint{1.510711in}{1.323842in}}%
\pgfpathlineto{\pgfqpoint{1.520183in}{1.331502in}}%
\pgfpathlineto{\pgfqpoint{1.529656in}{1.333195in}}%
\pgfpathlineto{\pgfqpoint{1.539128in}{1.341540in}}%
\pgfpathlineto{\pgfqpoint{1.558073in}{1.346194in}}%
\pgfpathlineto{\pgfqpoint{1.577018in}{1.358581in}}%
\pgfpathlineto{\pgfqpoint{1.586491in}{1.360172in}}%
\pgfpathlineto{\pgfqpoint{1.595963in}{1.366857in}}%
\pgfpathlineto{\pgfqpoint{1.605436in}{1.369231in}}%
\pgfpathlineto{\pgfqpoint{1.614908in}{1.376596in}}%
\pgfpathlineto{\pgfqpoint{1.624381in}{1.376528in}}%
\pgfpathlineto{\pgfqpoint{1.662270in}{1.392101in}}%
\pgfpathlineto{\pgfqpoint{1.671743in}{1.392321in}}%
\pgfpathlineto{\pgfqpoint{1.681215in}{1.395380in}}%
\pgfpathlineto{\pgfqpoint{1.690688in}{1.402359in}}%
\pgfpathlineto{\pgfqpoint{1.700160in}{1.405614in}}%
\pgfpathlineto{\pgfqpoint{1.719105in}{1.407038in}}%
\pgfpathlineto{\pgfqpoint{1.728578in}{1.411864in}}%
\pgfpathlineto{\pgfqpoint{1.738050in}{1.412280in}}%
\pgfpathlineto{\pgfqpoint{1.756995in}{1.416347in}}%
\pgfpathlineto{\pgfqpoint{1.766468in}{1.423712in}}%
\pgfpathlineto{\pgfqpoint{1.775940in}{1.425308in}}%
\pgfpathlineto{\pgfqpoint{1.785413in}{1.430618in}}%
\pgfpathlineto{\pgfqpoint{1.804358in}{1.432434in}}%
\pgfpathlineto{\pgfqpoint{1.823303in}{1.441591in}}%
\pgfpathlineto{\pgfqpoint{1.832775in}{1.437798in}}%
\pgfpathlineto{\pgfqpoint{1.842248in}{1.447028in}}%
\pgfpathlineto{\pgfqpoint{1.851720in}{1.449402in}}%
\pgfpathlineto{\pgfqpoint{1.861193in}{1.455402in}}%
\pgfpathlineto{\pgfqpoint{1.870665in}{1.455720in}}%
\pgfpathlineto{\pgfqpoint{1.889610in}{1.462626in}}%
\pgfpathlineto{\pgfqpoint{1.908555in}{1.466203in}}%
\pgfpathlineto{\pgfqpoint{1.918028in}{1.467598in}}%
\pgfpathlineto{\pgfqpoint{1.927500in}{1.467236in}}%
\pgfpathlineto{\pgfqpoint{1.936973in}{1.473036in}}%
\pgfpathlineto{\pgfqpoint{1.946445in}{1.474533in}}%
\pgfpathlineto{\pgfqpoint{1.955918in}{1.485129in}}%
\pgfpathlineto{\pgfqpoint{1.965390in}{1.488188in}}%
\pgfpathlineto{\pgfqpoint{1.974863in}{1.493405in}}%
\pgfpathlineto{\pgfqpoint{1.984335in}{1.494800in}}%
\pgfpathlineto{\pgfqpoint{1.993808in}{1.493263in}}%
\pgfpathlineto{\pgfqpoint{2.003280in}{1.496616in}}%
\pgfpathlineto{\pgfqpoint{2.012753in}{1.502322in}}%
\pgfpathlineto{\pgfqpoint{2.031698in}{1.505900in}}%
\pgfpathlineto{\pgfqpoint{2.050643in}{1.508689in}}%
\pgfpathlineto{\pgfqpoint{2.069588in}{1.511876in}}%
\pgfpathlineto{\pgfqpoint{2.107478in}{1.513158in}}%
\pgfpathlineto{\pgfqpoint{2.126423in}{1.520063in}}%
\pgfpathlineto{\pgfqpoint{2.135895in}{1.528408in}}%
\pgfpathlineto{\pgfqpoint{2.145368in}{1.531570in}}%
\pgfpathlineto{\pgfqpoint{2.154840in}{1.532573in}}%
\pgfpathlineto{\pgfqpoint{2.173785in}{1.542904in}}%
\pgfpathlineto{\pgfqpoint{2.183258in}{1.543320in}}%
\pgfpathlineto{\pgfqpoint{2.192730in}{1.551376in}}%
\pgfpathlineto{\pgfqpoint{2.211675in}{1.560337in}}%
\pgfpathlineto{\pgfqpoint{2.221148in}{1.559481in}}%
\pgfpathlineto{\pgfqpoint{2.240093in}{1.567757in}}%
\pgfpathlineto{\pgfqpoint{2.249565in}{1.574242in}}%
\pgfpathlineto{\pgfqpoint{2.287455in}{1.579243in}}%
\pgfpathlineto{\pgfqpoint{2.296928in}{1.579958in}}%
\pgfpathlineto{\pgfqpoint{2.315873in}{1.576880in}}%
\pgfpathlineto{\pgfqpoint{2.334818in}{1.584074in}}%
\pgfpathlineto{\pgfqpoint{2.344290in}{1.589781in}}%
\pgfpathlineto{\pgfqpoint{2.353763in}{1.590882in}}%
\pgfpathlineto{\pgfqpoint{2.363235in}{1.589639in}}%
\pgfpathlineto{\pgfqpoint{2.382180in}{1.593999in}}%
\pgfpathlineto{\pgfqpoint{2.391653in}{1.597841in}}%
\pgfpathlineto{\pgfqpoint{2.401125in}{1.599530in}}%
\pgfpathlineto{\pgfqpoint{2.410598in}{1.603279in}}%
\pgfpathlineto{\pgfqpoint{2.420070in}{1.601443in}}%
\pgfpathlineto{\pgfqpoint{2.429543in}{1.606269in}}%
\pgfpathlineto{\pgfqpoint{2.439015in}{1.608447in}}%
\pgfpathlineto{\pgfqpoint{2.457960in}{1.608892in}}%
\pgfpathlineto{\pgfqpoint{2.486378in}{1.616997in}}%
\pgfpathlineto{\pgfqpoint{2.514795in}{1.622659in}}%
\pgfpathlineto{\pgfqpoint{2.524268in}{1.629829in}}%
\pgfpathlineto{\pgfqpoint{2.533740in}{1.627700in}}%
\pgfpathlineto{\pgfqpoint{2.543213in}{1.628513in}}%
\pgfpathlineto{\pgfqpoint{2.562158in}{1.637572in}}%
\pgfpathlineto{\pgfqpoint{2.571630in}{1.641414in}}%
\pgfpathlineto{\pgfqpoint{2.600048in}{1.641203in}}%
\pgfpathlineto{\pgfqpoint{2.618993in}{1.638609in}}%
\pgfpathlineto{\pgfqpoint{2.637938in}{1.635531in}}%
\pgfpathlineto{\pgfqpoint{2.647410in}{1.637224in}}%
\pgfpathlineto{\pgfqpoint{2.656883in}{1.641164in}}%
\pgfpathlineto{\pgfqpoint{2.685300in}{1.647018in}}%
\pgfpathlineto{\pgfqpoint{2.694773in}{1.646166in}}%
\pgfpathlineto{\pgfqpoint{2.704245in}{1.654315in}}%
\pgfpathlineto{\pgfqpoint{2.713718in}{1.659042in}}%
\pgfpathlineto{\pgfqpoint{2.723190in}{1.655641in}}%
\pgfpathlineto{\pgfqpoint{2.732663in}{1.664088in}}%
\pgfpathlineto{\pgfqpoint{2.751608in}{1.666780in}}%
\pgfpathlineto{\pgfqpoint{2.789498in}{1.680689in}}%
\pgfpathlineto{\pgfqpoint{2.798970in}{1.680719in}}%
\pgfpathlineto{\pgfqpoint{2.817915in}{1.684884in}}%
\pgfpathlineto{\pgfqpoint{2.827388in}{1.682069in}}%
\pgfpathlineto{\pgfqpoint{2.836860in}{1.681017in}}%
\pgfpathlineto{\pgfqpoint{2.855805in}{1.689195in}}%
\pgfpathlineto{\pgfqpoint{2.865278in}{1.694119in}}%
\pgfpathlineto{\pgfqpoint{2.893695in}{1.696351in}}%
\pgfpathlineto{\pgfqpoint{2.903168in}{1.699899in}}%
\pgfpathlineto{\pgfqpoint{2.912640in}{1.701299in}}%
\pgfpathlineto{\pgfqpoint{2.922113in}{1.700246in}}%
\pgfpathlineto{\pgfqpoint{2.931585in}{1.703408in}}%
\pgfpathlineto{\pgfqpoint{2.941058in}{1.710088in}}%
\pgfpathlineto{\pgfqpoint{2.950530in}{1.711978in}}%
\pgfpathlineto{\pgfqpoint{2.960003in}{1.720420in}}%
\pgfpathlineto{\pgfqpoint{2.978948in}{1.727717in}}%
\pgfpathlineto{\pgfqpoint{3.016838in}{1.730468in}}%
\pgfpathlineto{\pgfqpoint{3.026310in}{1.734701in}}%
\pgfpathlineto{\pgfqpoint{3.035783in}{1.735704in}}%
\pgfpathlineto{\pgfqpoint{3.045255in}{1.735049in}}%
\pgfpathlineto{\pgfqpoint{3.073673in}{1.740510in}}%
\pgfpathlineto{\pgfqpoint{3.083145in}{1.746511in}}%
\pgfpathlineto{\pgfqpoint{3.102090in}{1.754004in}}%
\pgfpathlineto{\pgfqpoint{3.121035in}{1.755814in}}%
\pgfpathlineto{\pgfqpoint{3.130508in}{1.760934in}}%
\pgfpathlineto{\pgfqpoint{3.139980in}{1.760860in}}%
\pgfpathlineto{\pgfqpoint{3.149453in}{1.767154in}}%
\pgfpathlineto{\pgfqpoint{3.168398in}{1.772592in}}%
\pgfpathlineto{\pgfqpoint{3.177870in}{1.770267in}}%
\pgfpathlineto{\pgfqpoint{3.196815in}{1.777075in}}%
\pgfpathlineto{\pgfqpoint{3.206288in}{1.780525in}}%
\pgfpathlineto{\pgfqpoint{3.215760in}{1.780946in}}%
\pgfpathlineto{\pgfqpoint{3.225233in}{1.784005in}}%
\pgfpathlineto{\pgfqpoint{3.253650in}{1.781440in}}%
\pgfpathlineto{\pgfqpoint{3.263123in}{1.783623in}}%
\pgfpathlineto{\pgfqpoint{3.272595in}{1.787759in}}%
\pgfpathlineto{\pgfqpoint{3.282068in}{1.788669in}}%
\pgfpathlineto{\pgfqpoint{3.291540in}{1.787910in}}%
\pgfpathlineto{\pgfqpoint{3.301013in}{1.792246in}}%
\pgfpathlineto{\pgfqpoint{3.310485in}{1.790411in}}%
\pgfpathlineto{\pgfqpoint{3.319958in}{1.790729in}}%
\pgfpathlineto{\pgfqpoint{3.329430in}{1.792814in}}%
\pgfpathlineto{\pgfqpoint{3.338903in}{1.799690in}}%
\pgfpathlineto{\pgfqpoint{3.367320in}{1.802514in}}%
\pgfpathlineto{\pgfqpoint{3.376793in}{1.802049in}}%
\pgfpathlineto{\pgfqpoint{3.395737in}{1.809347in}}%
\pgfpathlineto{\pgfqpoint{3.405210in}{1.811622in}}%
\pgfpathlineto{\pgfqpoint{3.414682in}{1.810379in}}%
\pgfpathlineto{\pgfqpoint{3.433627in}{1.812293in}}%
\pgfpathlineto{\pgfqpoint{3.443100in}{1.810947in}}%
\pgfpathlineto{\pgfqpoint{3.452572in}{1.812542in}}%
\pgfpathlineto{\pgfqpoint{3.490462in}{1.825174in}}%
\pgfpathlineto{\pgfqpoint{3.499935in}{1.823833in}}%
\pgfpathlineto{\pgfqpoint{3.509407in}{1.824837in}}%
\pgfpathlineto{\pgfqpoint{3.518880in}{1.823985in}}%
\pgfpathlineto{\pgfqpoint{3.528352in}{1.827435in}}%
\pgfpathlineto{\pgfqpoint{3.537825in}{1.827656in}}%
\pgfpathlineto{\pgfqpoint{3.547297in}{1.830034in}}%
\pgfpathlineto{\pgfqpoint{3.556770in}{1.829961in}}%
\pgfpathlineto{\pgfqpoint{3.566242in}{1.831654in}}%
\pgfpathlineto{\pgfqpoint{3.575715in}{1.830602in}}%
\pgfpathlineto{\pgfqpoint{3.623077in}{1.836998in}}%
\pgfpathlineto{\pgfqpoint{3.642022in}{1.843415in}}%
\pgfpathlineto{\pgfqpoint{3.651495in}{1.843836in}}%
\pgfpathlineto{\pgfqpoint{3.660967in}{1.849733in}}%
\pgfpathlineto{\pgfqpoint{3.679912in}{1.848319in}}%
\pgfpathlineto{\pgfqpoint{3.689385in}{1.853139in}}%
\pgfpathlineto{\pgfqpoint{3.698857in}{1.855616in}}%
\pgfpathlineto{\pgfqpoint{3.708330in}{1.853095in}}%
\pgfpathlineto{\pgfqpoint{3.746220in}{1.862796in}}%
\pgfpathlineto{\pgfqpoint{3.755692in}{1.862037in}}%
\pgfpathlineto{\pgfqpoint{3.765165in}{1.864122in}}%
\pgfpathlineto{\pgfqpoint{3.774637in}{1.864538in}}%
\pgfpathlineto{\pgfqpoint{3.784110in}{1.862316in}}%
\pgfpathlineto{\pgfqpoint{3.793582in}{1.861851in}}%
\pgfpathlineto{\pgfqpoint{3.803055in}{1.866187in}}%
\pgfpathlineto{\pgfqpoint{3.812527in}{1.873553in}}%
\pgfpathlineto{\pgfqpoint{3.822000in}{1.876710in}}%
\pgfpathlineto{\pgfqpoint{3.850417in}{1.876793in}}%
\pgfpathlineto{\pgfqpoint{3.859890in}{1.878775in}}%
\pgfpathlineto{\pgfqpoint{3.869362in}{1.875672in}}%
\pgfpathlineto{\pgfqpoint{3.878835in}{1.879514in}}%
\pgfpathlineto{\pgfqpoint{3.888307in}{1.876896in}}%
\pgfpathlineto{\pgfqpoint{3.897780in}{1.876827in}}%
\pgfpathlineto{\pgfqpoint{3.916725in}{1.873063in}}%
\pgfpathlineto{\pgfqpoint{3.926197in}{1.874752in}}%
\pgfpathlineto{\pgfqpoint{3.935670in}{1.872726in}}%
\pgfpathlineto{\pgfqpoint{3.945142in}{1.868639in}}%
\pgfpathlineto{\pgfqpoint{3.954615in}{1.871703in}}%
\pgfpathlineto{\pgfqpoint{3.964087in}{1.872119in}}%
\pgfpathlineto{\pgfqpoint{3.973560in}{1.877527in}}%
\pgfpathlineto{\pgfqpoint{3.983032in}{1.875794in}}%
\pgfpathlineto{\pgfqpoint{3.992505in}{1.879343in}}%
\pgfpathlineto{\pgfqpoint{4.020922in}{1.881971in}}%
\pgfpathlineto{\pgfqpoint{4.049340in}{1.881560in}}%
\pgfpathlineto{\pgfqpoint{4.058812in}{1.884716in}}%
\pgfpathlineto{\pgfqpoint{4.068285in}{1.891695in}}%
\pgfpathlineto{\pgfqpoint{4.077757in}{1.896516in}}%
\pgfpathlineto{\pgfqpoint{4.096702in}{1.896864in}}%
\pgfpathlineto{\pgfqpoint{4.134592in}{1.905781in}}%
\pgfpathlineto{\pgfqpoint{4.172482in}{1.911071in}}%
\pgfpathlineto{\pgfqpoint{4.181955in}{1.909045in}}%
\pgfpathlineto{\pgfqpoint{4.191427in}{1.910440in}}%
\pgfpathlineto{\pgfqpoint{4.200900in}{1.908218in}}%
\pgfpathlineto{\pgfqpoint{4.219845in}{1.907391in}}%
\pgfpathlineto{\pgfqpoint{4.229317in}{1.906926in}}%
\pgfpathlineto{\pgfqpoint{4.238790in}{1.903236in}}%
\pgfpathlineto{\pgfqpoint{4.248262in}{1.908252in}}%
\pgfpathlineto{\pgfqpoint{4.257735in}{1.908179in}}%
\pgfpathlineto{\pgfqpoint{4.267207in}{1.903999in}}%
\pgfpathlineto{\pgfqpoint{4.276680in}{1.901870in}}%
\pgfpathlineto{\pgfqpoint{4.286152in}{1.902683in}}%
\pgfpathlineto{\pgfqpoint{4.295625in}{1.907993in}}%
\pgfpathlineto{\pgfqpoint{4.305097in}{1.903226in}}%
\pgfpathlineto{\pgfqpoint{4.314570in}{1.903544in}}%
\pgfpathlineto{\pgfqpoint{4.324042in}{1.908267in}}%
\pgfpathlineto{\pgfqpoint{4.333515in}{1.911135in}}%
\pgfpathlineto{\pgfqpoint{4.342987in}{1.910670in}}%
\pgfpathlineto{\pgfqpoint{4.352460in}{1.915006in}}%
\pgfpathlineto{\pgfqpoint{4.361932in}{1.916401in}}%
\pgfpathlineto{\pgfqpoint{4.380877in}{1.914987in}}%
\pgfpathlineto{\pgfqpoint{4.390350in}{1.915696in}}%
\pgfpathlineto{\pgfqpoint{4.409295in}{1.920253in}}%
\pgfpathlineto{\pgfqpoint{4.418767in}{1.925861in}}%
\pgfpathlineto{\pgfqpoint{4.447185in}{1.929561in}}%
\pgfpathlineto{\pgfqpoint{4.456657in}{1.927335in}}%
\pgfpathlineto{\pgfqpoint{4.466130in}{1.927462in}}%
\pgfpathlineto{\pgfqpoint{4.513492in}{1.917610in}}%
\pgfpathlineto{\pgfqpoint{4.522965in}{1.916758in}}%
\pgfpathlineto{\pgfqpoint{4.532437in}{1.917664in}}%
\pgfpathlineto{\pgfqpoint{4.551382in}{1.925059in}}%
\pgfpathlineto{\pgfqpoint{4.560855in}{1.923811in}}%
\pgfpathlineto{\pgfqpoint{4.579800in}{1.927193in}}%
\pgfpathlineto{\pgfqpoint{4.589272in}{1.928103in}}%
\pgfpathlineto{\pgfqpoint{4.617690in}{1.923483in}}%
\pgfpathlineto{\pgfqpoint{4.636635in}{1.924711in}}%
\pgfpathlineto{\pgfqpoint{4.646107in}{1.926596in}}%
\pgfpathlineto{\pgfqpoint{4.674525in}{1.940084in}}%
\pgfpathlineto{\pgfqpoint{4.693470in}{1.940823in}}%
\pgfpathlineto{\pgfqpoint{4.750305in}{1.955080in}}%
\pgfpathlineto{\pgfqpoint{4.759777in}{1.953636in}}%
\pgfpathlineto{\pgfqpoint{4.778722in}{1.966512in}}%
\pgfpathlineto{\pgfqpoint{4.788195in}{1.968108in}}%
\pgfpathlineto{\pgfqpoint{4.797667in}{1.960791in}}%
\pgfpathlineto{\pgfqpoint{4.807140in}{1.967183in}}%
\pgfpathlineto{\pgfqpoint{4.826085in}{1.965861in}}%
\pgfpathlineto{\pgfqpoint{4.835557in}{1.963150in}}%
\pgfpathlineto{\pgfqpoint{4.845030in}{1.966209in}}%
\pgfpathlineto{\pgfqpoint{4.854502in}{1.965553in}}%
\pgfpathlineto{\pgfqpoint{4.882920in}{1.976301in}}%
\pgfpathlineto{\pgfqpoint{4.892392in}{1.980436in}}%
\pgfpathlineto{\pgfqpoint{4.901865in}{1.978312in}}%
\pgfpathlineto{\pgfqpoint{4.920810in}{1.977583in}}%
\pgfpathlineto{\pgfqpoint{4.939755in}{1.983803in}}%
\pgfpathlineto{\pgfqpoint{4.949227in}{1.980206in}}%
\pgfpathlineto{\pgfqpoint{4.958700in}{1.985223in}}%
\pgfpathlineto{\pgfqpoint{4.977645in}{1.985179in}}%
\pgfpathlineto{\pgfqpoint{4.987117in}{1.988340in}}%
\pgfpathlineto{\pgfqpoint{4.996590in}{1.986505in}}%
\pgfpathlineto{\pgfqpoint{5.015535in}{1.998403in}}%
\pgfpathlineto{\pgfqpoint{5.034480in}{1.998554in}}%
\pgfpathlineto{\pgfqpoint{5.062897in}{2.006757in}}%
\pgfpathlineto{\pgfqpoint{5.072370in}{2.006493in}}%
\pgfpathlineto{\pgfqpoint{5.081842in}{2.009943in}}%
\pgfpathlineto{\pgfqpoint{5.091315in}{2.010755in}}%
\pgfpathlineto{\pgfqpoint{5.100787in}{2.009703in}}%
\pgfpathlineto{\pgfqpoint{5.110260in}{2.013349in}}%
\pgfpathlineto{\pgfqpoint{5.119732in}{2.014553in}}%
\pgfpathlineto{\pgfqpoint{5.148149in}{2.006311in}}%
\pgfpathlineto{\pgfqpoint{5.157622in}{2.008494in}}%
\pgfpathlineto{\pgfqpoint{5.167094in}{2.007834in}}%
\pgfpathlineto{\pgfqpoint{5.176567in}{2.010403in}}%
\pgfpathlineto{\pgfqpoint{5.195512in}{2.007814in}}%
\pgfpathlineto{\pgfqpoint{5.204984in}{2.007843in}}%
\pgfpathlineto{\pgfqpoint{5.214457in}{2.012468in}}%
\pgfpathlineto{\pgfqpoint{5.233402in}{2.017123in}}%
\pgfpathlineto{\pgfqpoint{5.242874in}{2.017832in}}%
\pgfpathlineto{\pgfqpoint{5.252347in}{2.016393in}}%
\pgfpathlineto{\pgfqpoint{5.280764in}{2.020974in}}%
\pgfpathlineto{\pgfqpoint{5.290237in}{2.025996in}}%
\pgfpathlineto{\pgfqpoint{5.299709in}{2.020637in}}%
\pgfpathlineto{\pgfqpoint{5.318654in}{2.020984in}}%
\pgfpathlineto{\pgfqpoint{5.328127in}{2.025315in}}%
\pgfpathlineto{\pgfqpoint{5.356544in}{2.020994in}}%
\pgfpathlineto{\pgfqpoint{5.375489in}{2.024963in}}%
\pgfpathlineto{\pgfqpoint{5.384962in}{2.024204in}}%
\pgfpathlineto{\pgfqpoint{5.403907in}{2.027586in}}%
\pgfpathlineto{\pgfqpoint{5.413379in}{2.025066in}}%
\pgfpathlineto{\pgfqpoint{5.432324in}{2.022379in}}%
\pgfpathlineto{\pgfqpoint{5.441797in}{2.021527in}}%
\pgfpathlineto{\pgfqpoint{5.451269in}{2.024782in}}%
\pgfpathlineto{\pgfqpoint{5.460742in}{2.033126in}}%
\pgfpathlineto{\pgfqpoint{5.470214in}{2.029632in}}%
\pgfpathlineto{\pgfqpoint{5.489159in}{2.026162in}}%
\pgfpathlineto{\pgfqpoint{5.489159in}{2.026162in}}%
\pgfusepath{stroke}%
\end{pgfscope}%
\begin{pgfscope}%
\pgfpathrectangle{\pgfqpoint{0.762383in}{0.471179in}}{\pgfqpoint{4.726776in}{2.845920in}} %
\pgfusepath{clip}%
\pgfsetbuttcap%
\pgfsetroundjoin%
\pgfsetlinewidth{1.505625pt}%
\definecolor{currentstroke}{rgb}{1.000000,0.498039,0.054902}%
\pgfsetstrokecolor{currentstroke}%
\pgfsetdash{{5.550000pt}{2.400000pt}}{0.000000pt}%
\pgfpathmoveto{\pgfqpoint{0.762383in}{0.603279in}}%
\pgfpathlineto{\pgfqpoint{0.781328in}{0.687414in}}%
\pgfpathlineto{\pgfqpoint{0.800273in}{0.743554in}}%
\pgfpathlineto{\pgfqpoint{0.819218in}{0.788927in}}%
\pgfpathlineto{\pgfqpoint{0.828691in}{0.806377in}}%
\pgfpathlineto{\pgfqpoint{0.838163in}{0.828134in}}%
\pgfpathlineto{\pgfqpoint{0.857108in}{0.855986in}}%
\pgfpathlineto{\pgfqpoint{0.894998in}{0.922261in}}%
\pgfpathlineto{\pgfqpoint{0.904471in}{0.934523in}}%
\pgfpathlineto{\pgfqpoint{0.913943in}{0.942086in}}%
\pgfpathlineto{\pgfqpoint{0.923416in}{0.956208in}}%
\pgfpathlineto{\pgfqpoint{0.942361in}{0.973685in}}%
\pgfpathlineto{\pgfqpoint{0.970778in}{1.008708in}}%
\pgfpathlineto{\pgfqpoint{0.980251in}{1.013529in}}%
\pgfpathlineto{\pgfqpoint{0.989723in}{1.023053in}}%
\pgfpathlineto{\pgfqpoint{1.027613in}{1.052624in}}%
\pgfpathlineto{\pgfqpoint{1.037086in}{1.056172in}}%
\pgfpathlineto{\pgfqpoint{1.065503in}{1.085811in}}%
\pgfpathlineto{\pgfqpoint{1.103393in}{1.119982in}}%
\pgfpathlineto{\pgfqpoint{1.122338in}{1.127475in}}%
\pgfpathlineto{\pgfqpoint{1.141283in}{1.141722in}}%
\pgfpathlineto{\pgfqpoint{1.150756in}{1.141942in}}%
\pgfpathlineto{\pgfqpoint{1.160228in}{1.149410in}}%
\pgfpathlineto{\pgfqpoint{1.179173in}{1.171879in}}%
\pgfpathlineto{\pgfqpoint{1.188646in}{1.176406in}}%
\pgfpathlineto{\pgfqpoint{1.198118in}{1.178780in}}%
\pgfpathlineto{\pgfqpoint{1.207591in}{1.183410in}}%
\pgfpathlineto{\pgfqpoint{1.217063in}{1.184022in}}%
\pgfpathlineto{\pgfqpoint{1.245481in}{1.201039in}}%
\pgfpathlineto{\pgfqpoint{1.264426in}{1.217042in}}%
\pgfpathlineto{\pgfqpoint{1.273898in}{1.223826in}}%
\pgfpathlineto{\pgfqpoint{1.292843in}{1.241205in}}%
\pgfpathlineto{\pgfqpoint{1.302316in}{1.245732in}}%
\pgfpathlineto{\pgfqpoint{1.311788in}{1.245370in}}%
\pgfpathlineto{\pgfqpoint{1.321261in}{1.247156in}}%
\pgfpathlineto{\pgfqpoint{1.330733in}{1.250998in}}%
\pgfpathlineto{\pgfqpoint{1.340206in}{1.261599in}}%
\pgfpathlineto{\pgfqpoint{1.359151in}{1.274965in}}%
\pgfpathlineto{\pgfqpoint{1.387568in}{1.289530in}}%
\pgfpathlineto{\pgfqpoint{1.463348in}{1.320872in}}%
\pgfpathlineto{\pgfqpoint{1.482293in}{1.331786in}}%
\pgfpathlineto{\pgfqpoint{1.491766in}{1.338765in}}%
\pgfpathlineto{\pgfqpoint{1.501238in}{1.342411in}}%
\pgfpathlineto{\pgfqpoint{1.520183in}{1.360671in}}%
\pgfpathlineto{\pgfqpoint{1.529656in}{1.362462in}}%
\pgfpathlineto{\pgfqpoint{1.539128in}{1.369436in}}%
\pgfpathlineto{\pgfqpoint{1.548601in}{1.370244in}}%
\pgfpathlineto{\pgfqpoint{1.577018in}{1.381877in}}%
\pgfpathlineto{\pgfqpoint{1.586491in}{1.382978in}}%
\pgfpathlineto{\pgfqpoint{1.595963in}{1.393286in}}%
\pgfpathlineto{\pgfqpoint{1.614908in}{1.403710in}}%
\pgfpathlineto{\pgfqpoint{1.624381in}{1.403642in}}%
\pgfpathlineto{\pgfqpoint{1.643325in}{1.416224in}}%
\pgfpathlineto{\pgfqpoint{1.652798in}{1.418598in}}%
\pgfpathlineto{\pgfqpoint{1.690688in}{1.434465in}}%
\pgfpathlineto{\pgfqpoint{1.700160in}{1.437524in}}%
\pgfpathlineto{\pgfqpoint{1.728578in}{1.450821in}}%
\pgfpathlineto{\pgfqpoint{1.766468in}{1.459538in}}%
\pgfpathlineto{\pgfqpoint{1.775940in}{1.459469in}}%
\pgfpathlineto{\pgfqpoint{1.785413in}{1.467520in}}%
\pgfpathlineto{\pgfqpoint{1.804358in}{1.469923in}}%
\pgfpathlineto{\pgfqpoint{1.813830in}{1.478860in}}%
\pgfpathlineto{\pgfqpoint{1.832775in}{1.477930in}}%
\pgfpathlineto{\pgfqpoint{1.842248in}{1.484224in}}%
\pgfpathlineto{\pgfqpoint{1.851720in}{1.487674in}}%
\pgfpathlineto{\pgfqpoint{1.861193in}{1.495828in}}%
\pgfpathlineto{\pgfqpoint{1.899083in}{1.498475in}}%
\pgfpathlineto{\pgfqpoint{1.918028in}{1.505479in}}%
\pgfpathlineto{\pgfqpoint{1.927500in}{1.509326in}}%
\pgfpathlineto{\pgfqpoint{1.955918in}{1.531428in}}%
\pgfpathlineto{\pgfqpoint{1.993808in}{1.542009in}}%
\pgfpathlineto{\pgfqpoint{2.003280in}{1.539586in}}%
\pgfpathlineto{\pgfqpoint{2.012753in}{1.546369in}}%
\pgfpathlineto{\pgfqpoint{2.022225in}{1.544926in}}%
\pgfpathlineto{\pgfqpoint{2.031698in}{1.547304in}}%
\pgfpathlineto{\pgfqpoint{2.041170in}{1.547524in}}%
\pgfpathlineto{\pgfqpoint{2.050643in}{1.554694in}}%
\pgfpathlineto{\pgfqpoint{2.060115in}{1.555703in}}%
\pgfpathlineto{\pgfqpoint{2.069588in}{1.555140in}}%
\pgfpathlineto{\pgfqpoint{2.107478in}{1.563372in}}%
\pgfpathlineto{\pgfqpoint{2.126423in}{1.569103in}}%
\pgfpathlineto{\pgfqpoint{2.135895in}{1.575881in}}%
\pgfpathlineto{\pgfqpoint{2.154840in}{1.585625in}}%
\pgfpathlineto{\pgfqpoint{2.164313in}{1.587319in}}%
\pgfpathlineto{\pgfqpoint{2.202203in}{1.605236in}}%
\pgfpathlineto{\pgfqpoint{2.211675in}{1.610943in}}%
\pgfpathlineto{\pgfqpoint{2.221148in}{1.609891in}}%
\pgfpathlineto{\pgfqpoint{2.230620in}{1.613248in}}%
\pgfpathlineto{\pgfqpoint{2.249565in}{1.624162in}}%
\pgfpathlineto{\pgfqpoint{2.268510in}{1.624412in}}%
\pgfpathlineto{\pgfqpoint{2.277983in}{1.628454in}}%
\pgfpathlineto{\pgfqpoint{2.287455in}{1.625542in}}%
\pgfpathlineto{\pgfqpoint{2.296928in}{1.628116in}}%
\pgfpathlineto{\pgfqpoint{2.306400in}{1.626477in}}%
\pgfpathlineto{\pgfqpoint{2.315873in}{1.627583in}}%
\pgfpathlineto{\pgfqpoint{2.344290in}{1.645867in}}%
\pgfpathlineto{\pgfqpoint{2.353763in}{1.646479in}}%
\pgfpathlineto{\pgfqpoint{2.363235in}{1.642691in}}%
\pgfpathlineto{\pgfqpoint{2.372708in}{1.646044in}}%
\pgfpathlineto{\pgfqpoint{2.382180in}{1.652729in}}%
\pgfpathlineto{\pgfqpoint{2.391653in}{1.652166in}}%
\pgfpathlineto{\pgfqpoint{2.401125in}{1.656889in}}%
\pgfpathlineto{\pgfqpoint{2.420070in}{1.659977in}}%
\pgfpathlineto{\pgfqpoint{2.429543in}{1.671459in}}%
\pgfpathlineto{\pgfqpoint{2.448488in}{1.668381in}}%
\pgfpathlineto{\pgfqpoint{2.457960in}{1.673397in}}%
\pgfpathlineto{\pgfqpoint{2.467433in}{1.674792in}}%
\pgfpathlineto{\pgfqpoint{2.486378in}{1.680523in}}%
\pgfpathlineto{\pgfqpoint{2.495850in}{1.678790in}}%
\pgfpathlineto{\pgfqpoint{2.524268in}{1.694041in}}%
\pgfpathlineto{\pgfqpoint{2.533740in}{1.693869in}}%
\pgfpathlineto{\pgfqpoint{2.543213in}{1.697227in}}%
\pgfpathlineto{\pgfqpoint{2.562158in}{1.708243in}}%
\pgfpathlineto{\pgfqpoint{2.590575in}{1.716054in}}%
\pgfpathlineto{\pgfqpoint{2.600048in}{1.717063in}}%
\pgfpathlineto{\pgfqpoint{2.628465in}{1.714400in}}%
\pgfpathlineto{\pgfqpoint{2.656883in}{1.718883in}}%
\pgfpathlineto{\pgfqpoint{2.666355in}{1.716172in}}%
\pgfpathlineto{\pgfqpoint{2.675828in}{1.719818in}}%
\pgfpathlineto{\pgfqpoint{2.694773in}{1.723885in}}%
\pgfpathlineto{\pgfqpoint{2.704245in}{1.731838in}}%
\pgfpathlineto{\pgfqpoint{2.713718in}{1.735195in}}%
\pgfpathlineto{\pgfqpoint{2.723190in}{1.736394in}}%
\pgfpathlineto{\pgfqpoint{2.751608in}{1.744010in}}%
\pgfpathlineto{\pgfqpoint{2.761080in}{1.744822in}}%
\pgfpathlineto{\pgfqpoint{2.789498in}{1.755863in}}%
\pgfpathlineto{\pgfqpoint{2.817915in}{1.758589in}}%
\pgfpathlineto{\pgfqpoint{2.827388in}{1.759299in}}%
\pgfpathlineto{\pgfqpoint{2.836860in}{1.762847in}}%
\pgfpathlineto{\pgfqpoint{2.846333in}{1.761017in}}%
\pgfpathlineto{\pgfqpoint{2.855805in}{1.761139in}}%
\pgfpathlineto{\pgfqpoint{2.931585in}{1.776820in}}%
\pgfpathlineto{\pgfqpoint{2.941058in}{1.783011in}}%
\pgfpathlineto{\pgfqpoint{2.960003in}{1.787568in}}%
\pgfpathlineto{\pgfqpoint{2.978948in}{1.797116in}}%
\pgfpathlineto{\pgfqpoint{2.988420in}{1.797924in}}%
\pgfpathlineto{\pgfqpoint{2.997893in}{1.795408in}}%
\pgfpathlineto{\pgfqpoint{3.007365in}{1.800229in}}%
\pgfpathlineto{\pgfqpoint{3.016838in}{1.800454in}}%
\pgfpathlineto{\pgfqpoint{3.035783in}{1.809704in}}%
\pgfpathlineto{\pgfqpoint{3.045255in}{1.810614in}}%
\pgfpathlineto{\pgfqpoint{3.064200in}{1.817422in}}%
\pgfpathlineto{\pgfqpoint{3.073673in}{1.815782in}}%
\pgfpathlineto{\pgfqpoint{3.083145in}{1.820706in}}%
\pgfpathlineto{\pgfqpoint{3.092618in}{1.822884in}}%
\pgfpathlineto{\pgfqpoint{3.111563in}{1.829398in}}%
\pgfpathlineto{\pgfqpoint{3.121035in}{1.836568in}}%
\pgfpathlineto{\pgfqpoint{3.130508in}{1.837968in}}%
\pgfpathlineto{\pgfqpoint{3.139980in}{1.843376in}}%
\pgfpathlineto{\pgfqpoint{3.149453in}{1.843307in}}%
\pgfpathlineto{\pgfqpoint{3.158925in}{1.851162in}}%
\pgfpathlineto{\pgfqpoint{3.168398in}{1.848842in}}%
\pgfpathlineto{\pgfqpoint{3.177870in}{1.849258in}}%
\pgfpathlineto{\pgfqpoint{3.187343in}{1.852807in}}%
\pgfpathlineto{\pgfqpoint{3.196815in}{1.851759in}}%
\pgfpathlineto{\pgfqpoint{3.206288in}{1.852665in}}%
\pgfpathlineto{\pgfqpoint{3.215760in}{1.851911in}}%
\pgfpathlineto{\pgfqpoint{3.244178in}{1.855905in}}%
\pgfpathlineto{\pgfqpoint{3.253650in}{1.853874in}}%
\pgfpathlineto{\pgfqpoint{3.263123in}{1.858503in}}%
\pgfpathlineto{\pgfqpoint{3.272595in}{1.860192in}}%
\pgfpathlineto{\pgfqpoint{3.301013in}{1.872217in}}%
\pgfpathlineto{\pgfqpoint{3.310485in}{1.871850in}}%
\pgfpathlineto{\pgfqpoint{3.319958in}{1.874125in}}%
\pgfpathlineto{\pgfqpoint{3.329430in}{1.872001in}}%
\pgfpathlineto{\pgfqpoint{3.338903in}{1.879465in}}%
\pgfpathlineto{\pgfqpoint{3.357848in}{1.886469in}}%
\pgfpathlineto{\pgfqpoint{3.367320in}{1.885421in}}%
\pgfpathlineto{\pgfqpoint{3.376793in}{1.889557in}}%
\pgfpathlineto{\pgfqpoint{3.386265in}{1.889684in}}%
\pgfpathlineto{\pgfqpoint{3.395737in}{1.894407in}}%
\pgfpathlineto{\pgfqpoint{3.405210in}{1.896193in}}%
\pgfpathlineto{\pgfqpoint{3.414682in}{1.900725in}}%
\pgfpathlineto{\pgfqpoint{3.424155in}{1.902805in}}%
\pgfpathlineto{\pgfqpoint{3.433627in}{1.901464in}}%
\pgfpathlineto{\pgfqpoint{3.452572in}{1.906706in}}%
\pgfpathlineto{\pgfqpoint{3.462045in}{1.909863in}}%
\pgfpathlineto{\pgfqpoint{3.471517in}{1.911257in}}%
\pgfpathlineto{\pgfqpoint{3.480990in}{1.910602in}}%
\pgfpathlineto{\pgfqpoint{3.490462in}{1.911996in}}%
\pgfpathlineto{\pgfqpoint{3.499935in}{1.910460in}}%
\pgfpathlineto{\pgfqpoint{3.509407in}{1.914106in}}%
\pgfpathlineto{\pgfqpoint{3.518880in}{1.915603in}}%
\pgfpathlineto{\pgfqpoint{3.528352in}{1.915530in}}%
\pgfpathlineto{\pgfqpoint{3.537825in}{1.920057in}}%
\pgfpathlineto{\pgfqpoint{3.556770in}{1.918153in}}%
\pgfpathlineto{\pgfqpoint{3.566242in}{1.919553in}}%
\pgfpathlineto{\pgfqpoint{3.594660in}{1.929517in}}%
\pgfpathlineto{\pgfqpoint{3.604132in}{1.929346in}}%
\pgfpathlineto{\pgfqpoint{3.632550in}{1.933736in}}%
\pgfpathlineto{\pgfqpoint{3.651495in}{1.927917in}}%
\pgfpathlineto{\pgfqpoint{3.660967in}{1.930584in}}%
\pgfpathlineto{\pgfqpoint{3.679912in}{1.933085in}}%
\pgfpathlineto{\pgfqpoint{3.727275in}{1.940069in}}%
\pgfpathlineto{\pgfqpoint{3.736747in}{1.945384in}}%
\pgfpathlineto{\pgfqpoint{3.746220in}{1.945017in}}%
\pgfpathlineto{\pgfqpoint{3.755692in}{1.942301in}}%
\pgfpathlineto{\pgfqpoint{3.765165in}{1.951238in}}%
\pgfpathlineto{\pgfqpoint{3.774637in}{1.954101in}}%
\pgfpathlineto{\pgfqpoint{3.793582in}{1.950827in}}%
\pgfpathlineto{\pgfqpoint{3.803055in}{1.945864in}}%
\pgfpathlineto{\pgfqpoint{3.812527in}{1.953719in}}%
\pgfpathlineto{\pgfqpoint{3.831472in}{1.963365in}}%
\pgfpathlineto{\pgfqpoint{3.840945in}{1.965250in}}%
\pgfpathlineto{\pgfqpoint{3.850417in}{1.964692in}}%
\pgfpathlineto{\pgfqpoint{3.859890in}{1.967848in}}%
\pgfpathlineto{\pgfqpoint{3.869362in}{1.963179in}}%
\pgfpathlineto{\pgfqpoint{3.878835in}{1.966923in}}%
\pgfpathlineto{\pgfqpoint{3.888307in}{1.966067in}}%
\pgfpathlineto{\pgfqpoint{3.897780in}{1.968739in}}%
\pgfpathlineto{\pgfqpoint{3.907252in}{1.969253in}}%
\pgfpathlineto{\pgfqpoint{3.916725in}{1.963703in}}%
\pgfpathlineto{\pgfqpoint{3.926197in}{1.966077in}}%
\pgfpathlineto{\pgfqpoint{3.935670in}{1.966400in}}%
\pgfpathlineto{\pgfqpoint{3.945142in}{1.964271in}}%
\pgfpathlineto{\pgfqpoint{3.954615in}{1.969488in}}%
\pgfpathlineto{\pgfqpoint{3.964087in}{1.968925in}}%
\pgfpathlineto{\pgfqpoint{3.973560in}{1.974529in}}%
\pgfpathlineto{\pgfqpoint{3.983032in}{1.974362in}}%
\pgfpathlineto{\pgfqpoint{3.992505in}{1.978009in}}%
\pgfpathlineto{\pgfqpoint{4.001977in}{1.974906in}}%
\pgfpathlineto{\pgfqpoint{4.020922in}{1.981028in}}%
\pgfpathlineto{\pgfqpoint{4.030395in}{1.978214in}}%
\pgfpathlineto{\pgfqpoint{4.039867in}{1.981567in}}%
\pgfpathlineto{\pgfqpoint{4.058812in}{1.981914in}}%
\pgfpathlineto{\pgfqpoint{4.068285in}{1.988012in}}%
\pgfpathlineto{\pgfqpoint{4.077757in}{1.991658in}}%
\pgfpathlineto{\pgfqpoint{4.096702in}{1.990244in}}%
\pgfpathlineto{\pgfqpoint{4.115647in}{1.991766in}}%
\pgfpathlineto{\pgfqpoint{4.125120in}{1.998544in}}%
\pgfpathlineto{\pgfqpoint{4.144065in}{1.999186in}}%
\pgfpathlineto{\pgfqpoint{4.163010in}{2.005993in}}%
\pgfpathlineto{\pgfqpoint{4.181955in}{2.001740in}}%
\pgfpathlineto{\pgfqpoint{4.200900in}{2.007569in}}%
\pgfpathlineto{\pgfqpoint{4.210372in}{2.007398in}}%
\pgfpathlineto{\pgfqpoint{4.219845in}{2.011147in}}%
\pgfpathlineto{\pgfqpoint{4.229317in}{2.006081in}}%
\pgfpathlineto{\pgfqpoint{4.238790in}{2.004740in}}%
\pgfpathlineto{\pgfqpoint{4.257735in}{2.009292in}}%
\pgfpathlineto{\pgfqpoint{4.267207in}{2.008734in}}%
\pgfpathlineto{\pgfqpoint{4.286152in}{2.001838in}}%
\pgfpathlineto{\pgfqpoint{4.305097in}{2.007080in}}%
\pgfpathlineto{\pgfqpoint{4.314570in}{2.005147in}}%
\pgfpathlineto{\pgfqpoint{4.342987in}{2.012762in}}%
\pgfpathlineto{\pgfqpoint{4.352460in}{2.011812in}}%
\pgfpathlineto{\pgfqpoint{4.361932in}{2.013403in}}%
\pgfpathlineto{\pgfqpoint{4.371405in}{2.010985in}}%
\pgfpathlineto{\pgfqpoint{4.380877in}{2.012087in}}%
\pgfpathlineto{\pgfqpoint{4.399822in}{2.018013in}}%
\pgfpathlineto{\pgfqpoint{4.409295in}{2.019310in}}%
\pgfpathlineto{\pgfqpoint{4.418767in}{2.025213in}}%
\pgfpathlineto{\pgfqpoint{4.428240in}{2.028663in}}%
\pgfpathlineto{\pgfqpoint{4.437712in}{2.029378in}}%
\pgfpathlineto{\pgfqpoint{4.466130in}{2.022213in}}%
\pgfpathlineto{\pgfqpoint{4.494547in}{2.024248in}}%
\pgfpathlineto{\pgfqpoint{4.513492in}{2.020093in}}%
\pgfpathlineto{\pgfqpoint{4.522965in}{2.024430in}}%
\pgfpathlineto{\pgfqpoint{4.532437in}{2.025531in}}%
\pgfpathlineto{\pgfqpoint{4.541910in}{2.028981in}}%
\pgfpathlineto{\pgfqpoint{4.551382in}{2.028717in}}%
\pgfpathlineto{\pgfqpoint{4.570327in}{2.036993in}}%
\pgfpathlineto{\pgfqpoint{4.579800in}{2.040541in}}%
\pgfpathlineto{\pgfqpoint{4.589272in}{2.046150in}}%
\pgfpathlineto{\pgfqpoint{4.598745in}{2.044119in}}%
\pgfpathlineto{\pgfqpoint{4.608217in}{2.037781in}}%
\pgfpathlineto{\pgfqpoint{4.617690in}{2.034776in}}%
\pgfpathlineto{\pgfqpoint{4.627162in}{2.037149in}}%
\pgfpathlineto{\pgfqpoint{4.636635in}{2.044129in}}%
\pgfpathlineto{\pgfqpoint{4.646107in}{2.044349in}}%
\pgfpathlineto{\pgfqpoint{4.655580in}{2.046140in}}%
\pgfpathlineto{\pgfqpoint{4.665052in}{2.046262in}}%
\pgfpathlineto{\pgfqpoint{4.693470in}{2.050452in}}%
\pgfpathlineto{\pgfqpoint{4.712415in}{2.050897in}}%
\pgfpathlineto{\pgfqpoint{4.721887in}{2.053569in}}%
\pgfpathlineto{\pgfqpoint{4.740832in}{2.051372in}}%
\pgfpathlineto{\pgfqpoint{4.750305in}{2.053746in}}%
\pgfpathlineto{\pgfqpoint{4.759777in}{2.065222in}}%
\pgfpathlineto{\pgfqpoint{4.769250in}{2.063490in}}%
\pgfpathlineto{\pgfqpoint{4.778722in}{2.067821in}}%
\pgfpathlineto{\pgfqpoint{4.816612in}{2.075074in}}%
\pgfpathlineto{\pgfqpoint{4.826085in}{2.077937in}}%
\pgfpathlineto{\pgfqpoint{4.845030in}{2.074565in}}%
\pgfpathlineto{\pgfqpoint{4.854502in}{2.075182in}}%
\pgfpathlineto{\pgfqpoint{4.863975in}{2.073934in}}%
\pgfpathlineto{\pgfqpoint{4.873447in}{2.075236in}}%
\pgfpathlineto{\pgfqpoint{4.882920in}{2.080839in}}%
\pgfpathlineto{\pgfqpoint{4.892392in}{2.088597in}}%
\pgfpathlineto{\pgfqpoint{4.911337in}{2.089140in}}%
\pgfpathlineto{\pgfqpoint{4.920810in}{2.090540in}}%
\pgfpathlineto{\pgfqpoint{4.930282in}{2.094773in}}%
\pgfpathlineto{\pgfqpoint{4.939755in}{2.097152in}}%
\pgfpathlineto{\pgfqpoint{4.949227in}{2.095316in}}%
\pgfpathlineto{\pgfqpoint{4.968172in}{2.103984in}}%
\pgfpathlineto{\pgfqpoint{4.977645in}{2.103715in}}%
\pgfpathlineto{\pgfqpoint{4.987117in}{2.101689in}}%
\pgfpathlineto{\pgfqpoint{4.996590in}{2.107488in}}%
\pgfpathlineto{\pgfqpoint{5.006062in}{2.106637in}}%
\pgfpathlineto{\pgfqpoint{5.015535in}{2.111261in}}%
\pgfpathlineto{\pgfqpoint{5.025007in}{2.110606in}}%
\pgfpathlineto{\pgfqpoint{5.043952in}{2.112612in}}%
\pgfpathlineto{\pgfqpoint{5.053425in}{2.119493in}}%
\pgfpathlineto{\pgfqpoint{5.062897in}{2.120595in}}%
\pgfpathlineto{\pgfqpoint{5.072370in}{2.123658in}}%
\pgfpathlineto{\pgfqpoint{5.081842in}{2.124662in}}%
\pgfpathlineto{\pgfqpoint{5.091315in}{2.123223in}}%
\pgfpathlineto{\pgfqpoint{5.100787in}{2.125205in}}%
\pgfpathlineto{\pgfqpoint{5.110260in}{2.122782in}}%
\pgfpathlineto{\pgfqpoint{5.119732in}{2.124671in}}%
\pgfpathlineto{\pgfqpoint{5.129205in}{2.124207in}}%
\pgfpathlineto{\pgfqpoint{5.138677in}{2.118852in}}%
\pgfpathlineto{\pgfqpoint{5.148149in}{2.117409in}}%
\pgfpathlineto{\pgfqpoint{5.157622in}{2.121647in}}%
\pgfpathlineto{\pgfqpoint{5.167094in}{2.118148in}}%
\pgfpathlineto{\pgfqpoint{5.176567in}{2.121011in}}%
\pgfpathlineto{\pgfqpoint{5.204984in}{2.124618in}}%
\pgfpathlineto{\pgfqpoint{5.223929in}{2.121148in}}%
\pgfpathlineto{\pgfqpoint{5.233402in}{2.126066in}}%
\pgfpathlineto{\pgfqpoint{5.242874in}{2.126678in}}%
\pgfpathlineto{\pgfqpoint{5.252347in}{2.125141in}}%
\pgfpathlineto{\pgfqpoint{5.280764in}{2.127569in}}%
\pgfpathlineto{\pgfqpoint{5.290237in}{2.125347in}}%
\pgfpathlineto{\pgfqpoint{5.299709in}{2.128308in}}%
\pgfpathlineto{\pgfqpoint{5.309182in}{2.126184in}}%
\pgfpathlineto{\pgfqpoint{5.328127in}{2.129169in}}%
\pgfpathlineto{\pgfqpoint{5.337599in}{2.125283in}}%
\pgfpathlineto{\pgfqpoint{5.347072in}{2.128440in}}%
\pgfpathlineto{\pgfqpoint{5.356544in}{2.127099in}}%
\pgfpathlineto{\pgfqpoint{5.366017in}{2.134563in}}%
\pgfpathlineto{\pgfqpoint{5.375489in}{2.134788in}}%
\pgfpathlineto{\pgfqpoint{5.384962in}{2.138630in}}%
\pgfpathlineto{\pgfqpoint{5.394434in}{2.135718in}}%
\pgfpathlineto{\pgfqpoint{5.403907in}{2.140054in}}%
\pgfpathlineto{\pgfqpoint{5.422852in}{2.138737in}}%
\pgfpathlineto{\pgfqpoint{5.432324in}{2.135629in}}%
\pgfpathlineto{\pgfqpoint{5.441797in}{2.130177in}}%
\pgfpathlineto{\pgfqpoint{5.451269in}{2.136564in}}%
\pgfpathlineto{\pgfqpoint{5.460742in}{2.135904in}}%
\pgfpathlineto{\pgfqpoint{5.470214in}{2.131332in}}%
\pgfpathlineto{\pgfqpoint{5.489159in}{2.131093in}}%
\pgfpathlineto{\pgfqpoint{5.489159in}{2.131093in}}%
\pgfusepath{stroke}%
\end{pgfscope}%
\begin{pgfscope}%
\pgfpathrectangle{\pgfqpoint{0.762383in}{0.471179in}}{\pgfqpoint{4.726776in}{2.845920in}} %
\pgfusepath{clip}%
\pgfsetbuttcap%
\pgfsetroundjoin%
\pgfsetlinewidth{1.505625pt}%
\definecolor{currentstroke}{rgb}{0.172549,0.627451,0.172549}%
\pgfsetstrokecolor{currentstroke}%
\pgfsetdash{{9.600000pt}{2.400000pt}{1.500000pt}{2.400000pt}}{0.000000pt}%
\pgfpathmoveto{\pgfqpoint{0.762383in}{0.600539in}}%
\pgfpathlineto{\pgfqpoint{0.800273in}{0.763718in}}%
\pgfpathlineto{\pgfqpoint{0.828691in}{0.860017in}}%
\pgfpathlineto{\pgfqpoint{0.838163in}{0.886178in}}%
\pgfpathlineto{\pgfqpoint{0.866581in}{0.974940in}}%
\pgfpathlineto{\pgfqpoint{0.885526in}{1.021585in}}%
\pgfpathlineto{\pgfqpoint{0.904471in}{1.060400in}}%
\pgfpathlineto{\pgfqpoint{0.913943in}{1.084406in}}%
\pgfpathlineto{\pgfqpoint{0.932888in}{1.116468in}}%
\pgfpathlineto{\pgfqpoint{0.951833in}{1.155675in}}%
\pgfpathlineto{\pgfqpoint{0.989723in}{1.218525in}}%
\pgfpathlineto{\pgfqpoint{0.999196in}{1.233917in}}%
\pgfpathlineto{\pgfqpoint{1.065503in}{1.312312in}}%
\pgfpathlineto{\pgfqpoint{1.074976in}{1.326339in}}%
\pgfpathlineto{\pgfqpoint{1.084448in}{1.344080in}}%
\pgfpathlineto{\pgfqpoint{1.093921in}{1.354974in}}%
\pgfpathlineto{\pgfqpoint{1.103393in}{1.369681in}}%
\pgfpathlineto{\pgfqpoint{1.122338in}{1.391269in}}%
\pgfpathlineto{\pgfqpoint{1.131811in}{1.397069in}}%
\pgfpathlineto{\pgfqpoint{1.179173in}{1.440568in}}%
\pgfpathlineto{\pgfqpoint{1.188646in}{1.454002in}}%
\pgfpathlineto{\pgfqpoint{1.198118in}{1.464598in}}%
\pgfpathlineto{\pgfqpoint{1.207591in}{1.471871in}}%
\pgfpathlineto{\pgfqpoint{1.217063in}{1.482565in}}%
\pgfpathlineto{\pgfqpoint{1.226536in}{1.489152in}}%
\pgfpathlineto{\pgfqpoint{1.254953in}{1.516246in}}%
\pgfpathlineto{\pgfqpoint{1.283371in}{1.546276in}}%
\pgfpathlineto{\pgfqpoint{1.311788in}{1.572397in}}%
\pgfpathlineto{\pgfqpoint{1.330733in}{1.582234in}}%
\pgfpathlineto{\pgfqpoint{1.368623in}{1.614056in}}%
\pgfpathlineto{\pgfqpoint{1.378096in}{1.625048in}}%
\pgfpathlineto{\pgfqpoint{1.387568in}{1.628694in}}%
\pgfpathlineto{\pgfqpoint{1.397041in}{1.630480in}}%
\pgfpathlineto{\pgfqpoint{1.406513in}{1.636578in}}%
\pgfpathlineto{\pgfqpoint{1.415986in}{1.645021in}}%
\pgfpathlineto{\pgfqpoint{1.425458in}{1.649357in}}%
\pgfpathlineto{\pgfqpoint{1.434931in}{1.657408in}}%
\pgfpathlineto{\pgfqpoint{1.463348in}{1.673446in}}%
\pgfpathlineto{\pgfqpoint{1.482293in}{1.690820in}}%
\pgfpathlineto{\pgfqpoint{1.510711in}{1.709991in}}%
\pgfpathlineto{\pgfqpoint{1.548601in}{1.745233in}}%
\pgfpathlineto{\pgfqpoint{1.558073in}{1.746046in}}%
\pgfpathlineto{\pgfqpoint{1.567546in}{1.749007in}}%
\pgfpathlineto{\pgfqpoint{1.614908in}{1.774490in}}%
\pgfpathlineto{\pgfqpoint{1.624381in}{1.781274in}}%
\pgfpathlineto{\pgfqpoint{1.633853in}{1.785116in}}%
\pgfpathlineto{\pgfqpoint{1.643325in}{1.794150in}}%
\pgfpathlineto{\pgfqpoint{1.652798in}{1.800439in}}%
\pgfpathlineto{\pgfqpoint{1.671743in}{1.807541in}}%
\pgfpathlineto{\pgfqpoint{1.709633in}{1.829574in}}%
\pgfpathlineto{\pgfqpoint{1.719105in}{1.834395in}}%
\pgfpathlineto{\pgfqpoint{1.747523in}{1.852978in}}%
\pgfpathlineto{\pgfqpoint{1.756995in}{1.852121in}}%
\pgfpathlineto{\pgfqpoint{1.775940in}{1.863530in}}%
\pgfpathlineto{\pgfqpoint{1.785413in}{1.863163in}}%
\pgfpathlineto{\pgfqpoint{1.804358in}{1.873396in}}%
\pgfpathlineto{\pgfqpoint{1.813830in}{1.875090in}}%
\pgfpathlineto{\pgfqpoint{1.823303in}{1.881379in}}%
\pgfpathlineto{\pgfqpoint{1.842248in}{1.882901in}}%
\pgfpathlineto{\pgfqpoint{1.851720in}{1.888211in}}%
\pgfpathlineto{\pgfqpoint{1.861193in}{1.888240in}}%
\pgfpathlineto{\pgfqpoint{1.870665in}{1.894138in}}%
\pgfpathlineto{\pgfqpoint{1.880138in}{1.905130in}}%
\pgfpathlineto{\pgfqpoint{1.899083in}{1.911150in}}%
\pgfpathlineto{\pgfqpoint{1.908555in}{1.913039in}}%
\pgfpathlineto{\pgfqpoint{1.918028in}{1.919426in}}%
\pgfpathlineto{\pgfqpoint{1.936973in}{1.922710in}}%
\pgfpathlineto{\pgfqpoint{1.955918in}{1.933726in}}%
\pgfpathlineto{\pgfqpoint{1.965390in}{1.936492in}}%
\pgfpathlineto{\pgfqpoint{1.984335in}{1.945355in}}%
\pgfpathlineto{\pgfqpoint{2.012753in}{1.956695in}}%
\pgfpathlineto{\pgfqpoint{2.022225in}{1.959264in}}%
\pgfpathlineto{\pgfqpoint{2.031698in}{1.965558in}}%
\pgfpathlineto{\pgfqpoint{2.041170in}{1.969987in}}%
\pgfpathlineto{\pgfqpoint{2.069588in}{1.978973in}}%
\pgfpathlineto{\pgfqpoint{2.107478in}{1.999636in}}%
\pgfpathlineto{\pgfqpoint{2.116950in}{1.999171in}}%
\pgfpathlineto{\pgfqpoint{2.126423in}{2.000277in}}%
\pgfpathlineto{\pgfqpoint{2.135895in}{2.005391in}}%
\pgfpathlineto{\pgfqpoint{2.154840in}{2.007794in}}%
\pgfpathlineto{\pgfqpoint{2.164313in}{2.015459in}}%
\pgfpathlineto{\pgfqpoint{2.173785in}{2.015385in}}%
\pgfpathlineto{\pgfqpoint{2.183258in}{2.020597in}}%
\pgfpathlineto{\pgfqpoint{2.202203in}{2.038074in}}%
\pgfpathlineto{\pgfqpoint{2.230620in}{2.051568in}}%
\pgfpathlineto{\pgfqpoint{2.240093in}{2.052571in}}%
\pgfpathlineto{\pgfqpoint{2.249565in}{2.058370in}}%
\pgfpathlineto{\pgfqpoint{2.259038in}{2.066328in}}%
\pgfpathlineto{\pgfqpoint{2.268510in}{2.064493in}}%
\pgfpathlineto{\pgfqpoint{2.277983in}{2.064327in}}%
\pgfpathlineto{\pgfqpoint{2.296928in}{2.070645in}}%
\pgfpathlineto{\pgfqpoint{2.306400in}{2.077423in}}%
\pgfpathlineto{\pgfqpoint{2.315873in}{2.076082in}}%
\pgfpathlineto{\pgfqpoint{2.325345in}{2.081295in}}%
\pgfpathlineto{\pgfqpoint{2.334818in}{2.083962in}}%
\pgfpathlineto{\pgfqpoint{2.344290in}{2.082523in}}%
\pgfpathlineto{\pgfqpoint{2.353763in}{2.083233in}}%
\pgfpathlineto{\pgfqpoint{2.363235in}{2.082185in}}%
\pgfpathlineto{\pgfqpoint{2.372708in}{2.087104in}}%
\pgfpathlineto{\pgfqpoint{2.382180in}{2.085763in}}%
\pgfpathlineto{\pgfqpoint{2.391653in}{2.086962in}}%
\pgfpathlineto{\pgfqpoint{2.401125in}{2.094719in}}%
\pgfpathlineto{\pgfqpoint{2.410598in}{2.098566in}}%
\pgfpathlineto{\pgfqpoint{2.420070in}{2.098884in}}%
\pgfpathlineto{\pgfqpoint{2.429543in}{2.105374in}}%
\pgfpathlineto{\pgfqpoint{2.439015in}{2.108531in}}%
\pgfpathlineto{\pgfqpoint{2.457960in}{2.107410in}}%
\pgfpathlineto{\pgfqpoint{2.467433in}{2.112328in}}%
\pgfpathlineto{\pgfqpoint{2.495850in}{2.119263in}}%
\pgfpathlineto{\pgfqpoint{2.514795in}{2.133510in}}%
\pgfpathlineto{\pgfqpoint{2.524268in}{2.132164in}}%
\pgfpathlineto{\pgfqpoint{2.543213in}{2.137504in}}%
\pgfpathlineto{\pgfqpoint{2.562158in}{2.140494in}}%
\pgfpathlineto{\pgfqpoint{2.571630in}{2.145413in}}%
\pgfpathlineto{\pgfqpoint{2.581103in}{2.146029in}}%
\pgfpathlineto{\pgfqpoint{2.600048in}{2.154893in}}%
\pgfpathlineto{\pgfqpoint{2.609520in}{2.162944in}}%
\pgfpathlineto{\pgfqpoint{2.618993in}{2.165905in}}%
\pgfpathlineto{\pgfqpoint{2.647410in}{2.161877in}}%
\pgfpathlineto{\pgfqpoint{2.656883in}{2.168361in}}%
\pgfpathlineto{\pgfqpoint{2.675828in}{2.169492in}}%
\pgfpathlineto{\pgfqpoint{2.685300in}{2.176075in}}%
\pgfpathlineto{\pgfqpoint{2.694773in}{2.184620in}}%
\pgfpathlineto{\pgfqpoint{2.713718in}{2.189470in}}%
\pgfpathlineto{\pgfqpoint{2.742135in}{2.190233in}}%
\pgfpathlineto{\pgfqpoint{2.751608in}{2.198676in}}%
\pgfpathlineto{\pgfqpoint{2.770553in}{2.205777in}}%
\pgfpathlineto{\pgfqpoint{2.780025in}{2.206198in}}%
\pgfpathlineto{\pgfqpoint{2.798970in}{2.216921in}}%
\pgfpathlineto{\pgfqpoint{2.817915in}{2.221184in}}%
\pgfpathlineto{\pgfqpoint{2.827388in}{2.221404in}}%
\pgfpathlineto{\pgfqpoint{2.836860in}{2.226616in}}%
\pgfpathlineto{\pgfqpoint{2.865278in}{2.228070in}}%
\pgfpathlineto{\pgfqpoint{2.874750in}{2.230933in}}%
\pgfpathlineto{\pgfqpoint{2.884223in}{2.229298in}}%
\pgfpathlineto{\pgfqpoint{2.912640in}{2.233096in}}%
\pgfpathlineto{\pgfqpoint{2.922113in}{2.239190in}}%
\pgfpathlineto{\pgfqpoint{2.950530in}{2.249942in}}%
\pgfpathlineto{\pgfqpoint{2.960003in}{2.256720in}}%
\pgfpathlineto{\pgfqpoint{2.969475in}{2.260171in}}%
\pgfpathlineto{\pgfqpoint{2.978948in}{2.260885in}}%
\pgfpathlineto{\pgfqpoint{2.988420in}{2.266979in}}%
\pgfpathlineto{\pgfqpoint{3.007365in}{2.272905in}}%
\pgfpathlineto{\pgfqpoint{3.016838in}{2.279199in}}%
\pgfpathlineto{\pgfqpoint{3.026310in}{2.276091in}}%
\pgfpathlineto{\pgfqpoint{3.035783in}{2.276214in}}%
\pgfpathlineto{\pgfqpoint{3.054728in}{2.281749in}}%
\pgfpathlineto{\pgfqpoint{3.064200in}{2.280212in}}%
\pgfpathlineto{\pgfqpoint{3.083145in}{2.280560in}}%
\pgfpathlineto{\pgfqpoint{3.102090in}{2.285801in}}%
\pgfpathlineto{\pgfqpoint{3.111563in}{2.294537in}}%
\pgfpathlineto{\pgfqpoint{3.130508in}{2.303303in}}%
\pgfpathlineto{\pgfqpoint{3.139980in}{2.303034in}}%
\pgfpathlineto{\pgfqpoint{3.149453in}{2.305216in}}%
\pgfpathlineto{\pgfqpoint{3.158925in}{2.303577in}}%
\pgfpathlineto{\pgfqpoint{3.177870in}{2.307252in}}%
\pgfpathlineto{\pgfqpoint{3.187343in}{2.305711in}}%
\pgfpathlineto{\pgfqpoint{3.196815in}{2.313864in}}%
\pgfpathlineto{\pgfqpoint{3.206288in}{2.317413in}}%
\pgfpathlineto{\pgfqpoint{3.215760in}{2.318323in}}%
\pgfpathlineto{\pgfqpoint{3.225233in}{2.321088in}}%
\pgfpathlineto{\pgfqpoint{3.234705in}{2.321313in}}%
\pgfpathlineto{\pgfqpoint{3.244178in}{2.319772in}}%
\pgfpathlineto{\pgfqpoint{3.263123in}{2.324524in}}%
\pgfpathlineto{\pgfqpoint{3.282068in}{2.332702in}}%
\pgfpathlineto{\pgfqpoint{3.291540in}{2.336740in}}%
\pgfpathlineto{\pgfqpoint{3.310485in}{2.341883in}}%
\pgfpathlineto{\pgfqpoint{3.319958in}{2.345823in}}%
\pgfpathlineto{\pgfqpoint{3.329430in}{2.345755in}}%
\pgfpathlineto{\pgfqpoint{3.338903in}{2.348618in}}%
\pgfpathlineto{\pgfqpoint{3.348375in}{2.354520in}}%
\pgfpathlineto{\pgfqpoint{3.357848in}{2.355621in}}%
\pgfpathlineto{\pgfqpoint{3.367320in}{2.358685in}}%
\pgfpathlineto{\pgfqpoint{3.376793in}{2.357241in}}%
\pgfpathlineto{\pgfqpoint{3.395737in}{2.360427in}}%
\pgfpathlineto{\pgfqpoint{3.405210in}{2.363193in}}%
\pgfpathlineto{\pgfqpoint{3.414682in}{2.363222in}}%
\pgfpathlineto{\pgfqpoint{3.424155in}{2.368434in}}%
\pgfpathlineto{\pgfqpoint{3.443100in}{2.365943in}}%
\pgfpathlineto{\pgfqpoint{3.452572in}{2.372922in}}%
\pgfpathlineto{\pgfqpoint{3.462045in}{2.369912in}}%
\pgfpathlineto{\pgfqpoint{3.499935in}{2.386073in}}%
\pgfpathlineto{\pgfqpoint{3.509407in}{2.381399in}}%
\pgfpathlineto{\pgfqpoint{3.518880in}{2.381526in}}%
\pgfpathlineto{\pgfqpoint{3.556770in}{2.399248in}}%
\pgfpathlineto{\pgfqpoint{3.585187in}{2.400114in}}%
\pgfpathlineto{\pgfqpoint{3.594660in}{2.408850in}}%
\pgfpathlineto{\pgfqpoint{3.604132in}{2.411909in}}%
\pgfpathlineto{\pgfqpoint{3.642022in}{2.411527in}}%
\pgfpathlineto{\pgfqpoint{3.651495in}{2.409697in}}%
\pgfpathlineto{\pgfqpoint{3.660967in}{2.413930in}}%
\pgfpathlineto{\pgfqpoint{3.679912in}{2.417997in}}%
\pgfpathlineto{\pgfqpoint{3.708330in}{2.431975in}}%
\pgfpathlineto{\pgfqpoint{3.727275in}{2.434378in}}%
\pgfpathlineto{\pgfqpoint{3.746220in}{2.430516in}}%
\pgfpathlineto{\pgfqpoint{3.755692in}{2.431911in}}%
\pgfpathlineto{\pgfqpoint{3.774637in}{2.438719in}}%
\pgfpathlineto{\pgfqpoint{3.784110in}{2.437280in}}%
\pgfpathlineto{\pgfqpoint{3.793582in}{2.441807in}}%
\pgfpathlineto{\pgfqpoint{3.812527in}{2.436575in}}%
\pgfpathlineto{\pgfqpoint{3.831472in}{2.444362in}}%
\pgfpathlineto{\pgfqpoint{3.850417in}{2.447744in}}%
\pgfpathlineto{\pgfqpoint{3.878835in}{2.450367in}}%
\pgfpathlineto{\pgfqpoint{3.888307in}{2.454698in}}%
\pgfpathlineto{\pgfqpoint{3.897780in}{2.452868in}}%
\pgfpathlineto{\pgfqpoint{3.907252in}{2.456220in}}%
\pgfpathlineto{\pgfqpoint{3.916725in}{2.461242in}}%
\pgfpathlineto{\pgfqpoint{3.926197in}{2.458134in}}%
\pgfpathlineto{\pgfqpoint{3.935670in}{2.457478in}}%
\pgfpathlineto{\pgfqpoint{3.945142in}{2.454077in}}%
\pgfpathlineto{\pgfqpoint{3.954615in}{2.454987in}}%
\pgfpathlineto{\pgfqpoint{3.964087in}{2.453348in}}%
\pgfpathlineto{\pgfqpoint{3.973560in}{2.454057in}}%
\pgfpathlineto{\pgfqpoint{3.983032in}{2.456534in}}%
\pgfpathlineto{\pgfqpoint{3.992505in}{2.461061in}}%
\pgfpathlineto{\pgfqpoint{4.011450in}{2.466302in}}%
\pgfpathlineto{\pgfqpoint{4.030395in}{2.467335in}}%
\pgfpathlineto{\pgfqpoint{4.049340in}{2.472577in}}%
\pgfpathlineto{\pgfqpoint{4.058812in}{2.471622in}}%
\pgfpathlineto{\pgfqpoint{4.077757in}{2.477941in}}%
\pgfpathlineto{\pgfqpoint{4.087230in}{2.483451in}}%
\pgfpathlineto{\pgfqpoint{4.096702in}{2.484455in}}%
\pgfpathlineto{\pgfqpoint{4.106175in}{2.487905in}}%
\pgfpathlineto{\pgfqpoint{4.125120in}{2.497649in}}%
\pgfpathlineto{\pgfqpoint{4.144065in}{2.502108in}}%
\pgfpathlineto{\pgfqpoint{4.153537in}{2.496558in}}%
\pgfpathlineto{\pgfqpoint{4.163010in}{2.499323in}}%
\pgfpathlineto{\pgfqpoint{4.172482in}{2.497194in}}%
\pgfpathlineto{\pgfqpoint{4.181955in}{2.499181in}}%
\pgfpathlineto{\pgfqpoint{4.200900in}{2.497571in}}%
\pgfpathlineto{\pgfqpoint{4.219845in}{2.500953in}}%
\pgfpathlineto{\pgfqpoint{4.248262in}{2.504457in}}%
\pgfpathlineto{\pgfqpoint{4.267207in}{2.500596in}}%
\pgfpathlineto{\pgfqpoint{4.276680in}{2.501501in}}%
\pgfpathlineto{\pgfqpoint{4.286152in}{2.499279in}}%
\pgfpathlineto{\pgfqpoint{4.295625in}{2.502240in}}%
\pgfpathlineto{\pgfqpoint{4.305097in}{2.499431in}}%
\pgfpathlineto{\pgfqpoint{4.314570in}{2.503566in}}%
\pgfpathlineto{\pgfqpoint{4.333515in}{2.507927in}}%
\pgfpathlineto{\pgfqpoint{4.342987in}{2.511573in}}%
\pgfpathlineto{\pgfqpoint{4.352460in}{2.508960in}}%
\pgfpathlineto{\pgfqpoint{4.361932in}{2.503796in}}%
\pgfpathlineto{\pgfqpoint{4.371405in}{2.510188in}}%
\pgfpathlineto{\pgfqpoint{4.418767in}{2.527939in}}%
\pgfpathlineto{\pgfqpoint{4.428240in}{2.523657in}}%
\pgfpathlineto{\pgfqpoint{4.447185in}{2.531835in}}%
\pgfpathlineto{\pgfqpoint{4.456657in}{2.534013in}}%
\pgfpathlineto{\pgfqpoint{4.466130in}{2.539622in}}%
\pgfpathlineto{\pgfqpoint{4.494547in}{2.542049in}}%
\pgfpathlineto{\pgfqpoint{4.504020in}{2.545994in}}%
\pgfpathlineto{\pgfqpoint{4.541910in}{2.532687in}}%
\pgfpathlineto{\pgfqpoint{4.560855in}{2.540963in}}%
\pgfpathlineto{\pgfqpoint{4.570327in}{2.536391in}}%
\pgfpathlineto{\pgfqpoint{4.579800in}{2.536514in}}%
\pgfpathlineto{\pgfqpoint{4.589272in}{2.539871in}}%
\pgfpathlineto{\pgfqpoint{4.598745in}{2.539406in}}%
\pgfpathlineto{\pgfqpoint{4.617690in}{2.546605in}}%
\pgfpathlineto{\pgfqpoint{4.627162in}{2.545553in}}%
\pgfpathlineto{\pgfqpoint{4.636635in}{2.546757in}}%
\pgfpathlineto{\pgfqpoint{4.646107in}{2.545901in}}%
\pgfpathlineto{\pgfqpoint{4.655580in}{2.548964in}}%
\pgfpathlineto{\pgfqpoint{4.665052in}{2.548010in}}%
\pgfpathlineto{\pgfqpoint{4.674525in}{2.551754in}}%
\pgfpathlineto{\pgfqpoint{4.683997in}{2.547477in}}%
\pgfpathlineto{\pgfqpoint{4.702942in}{2.556046in}}%
\pgfpathlineto{\pgfqpoint{4.759777in}{2.566481in}}%
\pgfpathlineto{\pgfqpoint{4.769250in}{2.573068in}}%
\pgfpathlineto{\pgfqpoint{4.778722in}{2.572407in}}%
\pgfpathlineto{\pgfqpoint{4.797667in}{2.581173in}}%
\pgfpathlineto{\pgfqpoint{4.816612in}{2.582303in}}%
\pgfpathlineto{\pgfqpoint{4.835557in}{2.584021in}}%
\pgfpathlineto{\pgfqpoint{4.845030in}{2.584535in}}%
\pgfpathlineto{\pgfqpoint{4.854502in}{2.587403in}}%
\pgfpathlineto{\pgfqpoint{4.863975in}{2.586938in}}%
\pgfpathlineto{\pgfqpoint{4.873447in}{2.588240in}}%
\pgfpathlineto{\pgfqpoint{4.882920in}{2.592473in}}%
\pgfpathlineto{\pgfqpoint{4.892392in}{2.591323in}}%
\pgfpathlineto{\pgfqpoint{4.901865in}{2.594681in}}%
\pgfpathlineto{\pgfqpoint{4.911337in}{2.594411in}}%
\pgfpathlineto{\pgfqpoint{4.920810in}{2.589547in}}%
\pgfpathlineto{\pgfqpoint{4.930282in}{2.587907in}}%
\pgfpathlineto{\pgfqpoint{4.977645in}{2.602722in}}%
\pgfpathlineto{\pgfqpoint{4.987117in}{2.596584in}}%
\pgfpathlineto{\pgfqpoint{4.996590in}{2.600916in}}%
\pgfpathlineto{\pgfqpoint{5.006062in}{2.607014in}}%
\pgfpathlineto{\pgfqpoint{5.015535in}{2.611247in}}%
\pgfpathlineto{\pgfqpoint{5.025007in}{2.609613in}}%
\pgfpathlineto{\pgfqpoint{5.034480in}{2.605526in}}%
\pgfpathlineto{\pgfqpoint{5.043952in}{2.613185in}}%
\pgfpathlineto{\pgfqpoint{5.053425in}{2.611649in}}%
\pgfpathlineto{\pgfqpoint{5.072370in}{2.618065in}}%
\pgfpathlineto{\pgfqpoint{5.100787in}{2.618045in}}%
\pgfpathlineto{\pgfqpoint{5.110260in}{2.625998in}}%
\pgfpathlineto{\pgfqpoint{5.119732in}{2.628377in}}%
\pgfpathlineto{\pgfqpoint{5.129205in}{2.633295in}}%
\pgfpathlineto{\pgfqpoint{5.157622in}{2.632302in}}%
\pgfpathlineto{\pgfqpoint{5.167094in}{2.630369in}}%
\pgfpathlineto{\pgfqpoint{5.176567in}{2.630785in}}%
\pgfpathlineto{\pgfqpoint{5.186039in}{2.628465in}}%
\pgfpathlineto{\pgfqpoint{5.195512in}{2.621638in}}%
\pgfpathlineto{\pgfqpoint{5.204984in}{2.628715in}}%
\pgfpathlineto{\pgfqpoint{5.223929in}{2.632586in}}%
\pgfpathlineto{\pgfqpoint{5.233402in}{2.632121in}}%
\pgfpathlineto{\pgfqpoint{5.252347in}{2.639222in}}%
\pgfpathlineto{\pgfqpoint{5.261819in}{2.639247in}}%
\pgfpathlineto{\pgfqpoint{5.299709in}{2.646108in}}%
\pgfpathlineto{\pgfqpoint{5.309182in}{2.649466in}}%
\pgfpathlineto{\pgfqpoint{5.318654in}{2.645673in}}%
\pgfpathlineto{\pgfqpoint{5.328127in}{2.648927in}}%
\pgfpathlineto{\pgfqpoint{5.337599in}{2.648761in}}%
\pgfpathlineto{\pgfqpoint{5.347072in}{2.646436in}}%
\pgfpathlineto{\pgfqpoint{5.366017in}{2.648644in}}%
\pgfpathlineto{\pgfqpoint{5.375489in}{2.647694in}}%
\pgfpathlineto{\pgfqpoint{5.384962in}{2.649676in}}%
\pgfpathlineto{\pgfqpoint{5.394434in}{2.645100in}}%
\pgfpathlineto{\pgfqpoint{5.403907in}{2.646891in}}%
\pgfpathlineto{\pgfqpoint{5.413379in}{2.653376in}}%
\pgfpathlineto{\pgfqpoint{5.432324in}{2.652060in}}%
\pgfpathlineto{\pgfqpoint{5.441797in}{2.654634in}}%
\pgfpathlineto{\pgfqpoint{5.451269in}{2.649569in}}%
\pgfpathlineto{\pgfqpoint{5.470214in}{2.658530in}}%
\pgfpathlineto{\pgfqpoint{5.489159in}{2.660835in}}%
\pgfpathlineto{\pgfqpoint{5.489159in}{2.660835in}}%
\pgfusepath{stroke}%
\end{pgfscope}%
\begin{pgfscope}%
\pgfpathrectangle{\pgfqpoint{0.762383in}{0.471179in}}{\pgfqpoint{4.726776in}{2.845920in}} %
\pgfusepath{clip}%
\pgfsetbuttcap%
\pgfsetroundjoin%
\pgfsetlinewidth{1.505625pt}%
\definecolor{currentstroke}{rgb}{0.839216,0.152941,0.156863}%
\pgfsetstrokecolor{currentstroke}%
\pgfsetdash{{1.500000pt}{2.475000pt}}{0.000000pt}%
\pgfpathmoveto{\pgfqpoint{0.762383in}{0.600539in}}%
\pgfpathlineto{\pgfqpoint{0.771856in}{0.688562in}}%
\pgfpathlineto{\pgfqpoint{0.781328in}{0.724022in}}%
\pgfpathlineto{\pgfqpoint{0.800273in}{0.776541in}}%
\pgfpathlineto{\pgfqpoint{0.819218in}{0.817607in}}%
\pgfpathlineto{\pgfqpoint{0.828691in}{0.832610in}}%
\pgfpathlineto{\pgfqpoint{0.838163in}{0.855345in}}%
\pgfpathlineto{\pgfqpoint{0.847636in}{0.865943in}}%
\pgfpathlineto{\pgfqpoint{0.857108in}{0.883882in}}%
\pgfpathlineto{\pgfqpoint{0.866581in}{0.897221in}}%
\pgfpathlineto{\pgfqpoint{0.885526in}{0.935644in}}%
\pgfpathlineto{\pgfqpoint{0.904471in}{0.962517in}}%
\pgfpathlineto{\pgfqpoint{0.913943in}{0.970179in}}%
\pgfpathlineto{\pgfqpoint{0.923416in}{0.980581in}}%
\pgfpathlineto{\pgfqpoint{0.942361in}{0.998155in}}%
\pgfpathlineto{\pgfqpoint{0.970778in}{1.036116in}}%
\pgfpathlineto{\pgfqpoint{0.980251in}{1.043579in}}%
\pgfpathlineto{\pgfqpoint{0.989723in}{1.057116in}}%
\pgfpathlineto{\pgfqpoint{1.027613in}{1.087764in}}%
\pgfpathlineto{\pgfqpoint{1.037086in}{1.091801in}}%
\pgfpathlineto{\pgfqpoint{1.056031in}{1.105852in}}%
\pgfpathlineto{\pgfqpoint{1.065503in}{1.118112in}}%
\pgfpathlineto{\pgfqpoint{1.074976in}{1.126560in}}%
\pgfpathlineto{\pgfqpoint{1.084448in}{1.138624in}}%
\pgfpathlineto{\pgfqpoint{1.093921in}{1.147560in}}%
\pgfpathlineto{\pgfqpoint{1.112866in}{1.159649in}}%
\pgfpathlineto{\pgfqpoint{1.122338in}{1.165062in}}%
\pgfpathlineto{\pgfqpoint{1.131811in}{1.167827in}}%
\pgfpathlineto{\pgfqpoint{1.141283in}{1.175589in}}%
\pgfpathlineto{\pgfqpoint{1.150756in}{1.186087in}}%
\pgfpathlineto{\pgfqpoint{1.160228in}{1.191206in}}%
\pgfpathlineto{\pgfqpoint{1.198118in}{1.220282in}}%
\pgfpathlineto{\pgfqpoint{1.207591in}{1.223444in}}%
\pgfpathlineto{\pgfqpoint{1.217063in}{1.228558in}}%
\pgfpathlineto{\pgfqpoint{1.236008in}{1.245546in}}%
\pgfpathlineto{\pgfqpoint{1.254953in}{1.253822in}}%
\pgfpathlineto{\pgfqpoint{1.264426in}{1.259132in}}%
\pgfpathlineto{\pgfqpoint{1.311788in}{1.298422in}}%
\pgfpathlineto{\pgfqpoint{1.330733in}{1.307966in}}%
\pgfpathlineto{\pgfqpoint{1.349678in}{1.322506in}}%
\pgfpathlineto{\pgfqpoint{1.368623in}{1.331272in}}%
\pgfpathlineto{\pgfqpoint{1.378096in}{1.339523in}}%
\pgfpathlineto{\pgfqpoint{1.387568in}{1.343659in}}%
\pgfpathlineto{\pgfqpoint{1.406513in}{1.356927in}}%
\pgfpathlineto{\pgfqpoint{1.425458in}{1.363343in}}%
\pgfpathlineto{\pgfqpoint{1.434931in}{1.372471in}}%
\pgfpathlineto{\pgfqpoint{1.444403in}{1.374458in}}%
\pgfpathlineto{\pgfqpoint{1.482293in}{1.395410in}}%
\pgfpathlineto{\pgfqpoint{1.491766in}{1.397592in}}%
\pgfpathlineto{\pgfqpoint{1.501238in}{1.407405in}}%
\pgfpathlineto{\pgfqpoint{1.510711in}{1.411252in}}%
\pgfpathlineto{\pgfqpoint{1.529656in}{1.424031in}}%
\pgfpathlineto{\pgfqpoint{1.539128in}{1.434528in}}%
\pgfpathlineto{\pgfqpoint{1.548601in}{1.435825in}}%
\pgfpathlineto{\pgfqpoint{1.558073in}{1.438889in}}%
\pgfpathlineto{\pgfqpoint{1.567546in}{1.447234in}}%
\pgfpathlineto{\pgfqpoint{1.577018in}{1.452255in}}%
\pgfpathlineto{\pgfqpoint{1.586491in}{1.455412in}}%
\pgfpathlineto{\pgfqpoint{1.595963in}{1.460629in}}%
\pgfpathlineto{\pgfqpoint{1.605436in}{1.468876in}}%
\pgfpathlineto{\pgfqpoint{1.614908in}{1.474088in}}%
\pgfpathlineto{\pgfqpoint{1.633853in}{1.476295in}}%
\pgfpathlineto{\pgfqpoint{1.652798in}{1.488291in}}%
\pgfpathlineto{\pgfqpoint{1.662270in}{1.493508in}}%
\pgfpathlineto{\pgfqpoint{1.681215in}{1.501094in}}%
\pgfpathlineto{\pgfqpoint{1.700160in}{1.512013in}}%
\pgfpathlineto{\pgfqpoint{1.709633in}{1.513119in}}%
\pgfpathlineto{\pgfqpoint{1.738050in}{1.526803in}}%
\pgfpathlineto{\pgfqpoint{1.747523in}{1.526734in}}%
\pgfpathlineto{\pgfqpoint{1.766468in}{1.539997in}}%
\pgfpathlineto{\pgfqpoint{1.775940in}{1.543550in}}%
\pgfpathlineto{\pgfqpoint{1.794885in}{1.547128in}}%
\pgfpathlineto{\pgfqpoint{1.804358in}{1.552928in}}%
\pgfpathlineto{\pgfqpoint{1.813830in}{1.552076in}}%
\pgfpathlineto{\pgfqpoint{1.823303in}{1.556603in}}%
\pgfpathlineto{\pgfqpoint{1.832775in}{1.557802in}}%
\pgfpathlineto{\pgfqpoint{1.842248in}{1.568109in}}%
\pgfpathlineto{\pgfqpoint{1.851720in}{1.572930in}}%
\pgfpathlineto{\pgfqpoint{1.861193in}{1.572372in}}%
\pgfpathlineto{\pgfqpoint{1.870665in}{1.578759in}}%
\pgfpathlineto{\pgfqpoint{1.880138in}{1.581725in}}%
\pgfpathlineto{\pgfqpoint{1.889610in}{1.588210in}}%
\pgfpathlineto{\pgfqpoint{1.908555in}{1.591298in}}%
\pgfpathlineto{\pgfqpoint{1.918028in}{1.597293in}}%
\pgfpathlineto{\pgfqpoint{1.927500in}{1.598986in}}%
\pgfpathlineto{\pgfqpoint{1.936973in}{1.597347in}}%
\pgfpathlineto{\pgfqpoint{1.955918in}{1.612377in}}%
\pgfpathlineto{\pgfqpoint{1.965390in}{1.623071in}}%
\pgfpathlineto{\pgfqpoint{1.974863in}{1.626134in}}%
\pgfpathlineto{\pgfqpoint{1.984335in}{1.627431in}}%
\pgfpathlineto{\pgfqpoint{1.993808in}{1.632648in}}%
\pgfpathlineto{\pgfqpoint{2.003280in}{1.629834in}}%
\pgfpathlineto{\pgfqpoint{2.012753in}{1.634366in}}%
\pgfpathlineto{\pgfqpoint{2.022225in}{1.642515in}}%
\pgfpathlineto{\pgfqpoint{2.031698in}{1.643034in}}%
\pgfpathlineto{\pgfqpoint{2.041170in}{1.646680in}}%
\pgfpathlineto{\pgfqpoint{2.050643in}{1.646313in}}%
\pgfpathlineto{\pgfqpoint{2.060115in}{1.647615in}}%
\pgfpathlineto{\pgfqpoint{2.069588in}{1.654882in}}%
\pgfpathlineto{\pgfqpoint{2.079060in}{1.652367in}}%
\pgfpathlineto{\pgfqpoint{2.088533in}{1.660614in}}%
\pgfpathlineto{\pgfqpoint{2.098005in}{1.658098in}}%
\pgfpathlineto{\pgfqpoint{2.107478in}{1.661157in}}%
\pgfpathlineto{\pgfqpoint{2.116950in}{1.660594in}}%
\pgfpathlineto{\pgfqpoint{2.135895in}{1.665542in}}%
\pgfpathlineto{\pgfqpoint{2.145368in}{1.673696in}}%
\pgfpathlineto{\pgfqpoint{2.183258in}{1.695822in}}%
\pgfpathlineto{\pgfqpoint{2.192730in}{1.702997in}}%
\pgfpathlineto{\pgfqpoint{2.202203in}{1.706545in}}%
\pgfpathlineto{\pgfqpoint{2.211675in}{1.712839in}}%
\pgfpathlineto{\pgfqpoint{2.221148in}{1.714527in}}%
\pgfpathlineto{\pgfqpoint{2.230620in}{1.714361in}}%
\pgfpathlineto{\pgfqpoint{2.240093in}{1.724761in}}%
\pgfpathlineto{\pgfqpoint{2.259038in}{1.736169in}}%
\pgfpathlineto{\pgfqpoint{2.277983in}{1.737985in}}%
\pgfpathlineto{\pgfqpoint{2.306400in}{1.733071in}}%
\pgfpathlineto{\pgfqpoint{2.315873in}{1.735646in}}%
\pgfpathlineto{\pgfqpoint{2.325345in}{1.734593in}}%
\pgfpathlineto{\pgfqpoint{2.344290in}{1.741793in}}%
\pgfpathlineto{\pgfqpoint{2.353763in}{1.747788in}}%
\pgfpathlineto{\pgfqpoint{2.363235in}{1.745860in}}%
\pgfpathlineto{\pgfqpoint{2.391653in}{1.755237in}}%
\pgfpathlineto{\pgfqpoint{2.420070in}{1.766768in}}%
\pgfpathlineto{\pgfqpoint{2.429543in}{1.767384in}}%
\pgfpathlineto{\pgfqpoint{2.448488in}{1.773800in}}%
\pgfpathlineto{\pgfqpoint{2.467433in}{1.774339in}}%
\pgfpathlineto{\pgfqpoint{2.486378in}{1.783006in}}%
\pgfpathlineto{\pgfqpoint{2.505323in}{1.792163in}}%
\pgfpathlineto{\pgfqpoint{2.514795in}{1.796499in}}%
\pgfpathlineto{\pgfqpoint{2.524268in}{1.795154in}}%
\pgfpathlineto{\pgfqpoint{2.533740in}{1.796059in}}%
\pgfpathlineto{\pgfqpoint{2.581103in}{1.810878in}}%
\pgfpathlineto{\pgfqpoint{2.609520in}{1.821430in}}%
\pgfpathlineto{\pgfqpoint{2.618993in}{1.814897in}}%
\pgfpathlineto{\pgfqpoint{2.628465in}{1.815317in}}%
\pgfpathlineto{\pgfqpoint{2.637938in}{1.813972in}}%
\pgfpathlineto{\pgfqpoint{2.647410in}{1.815665in}}%
\pgfpathlineto{\pgfqpoint{2.656883in}{1.819703in}}%
\pgfpathlineto{\pgfqpoint{2.666355in}{1.820417in}}%
\pgfpathlineto{\pgfqpoint{2.685300in}{1.829080in}}%
\pgfpathlineto{\pgfqpoint{2.694773in}{1.833612in}}%
\pgfpathlineto{\pgfqpoint{2.704245in}{1.834517in}}%
\pgfpathlineto{\pgfqpoint{2.713718in}{1.841203in}}%
\pgfpathlineto{\pgfqpoint{2.723190in}{1.843283in}}%
\pgfpathlineto{\pgfqpoint{2.732663in}{1.840767in}}%
\pgfpathlineto{\pgfqpoint{2.742135in}{1.846958in}}%
\pgfpathlineto{\pgfqpoint{2.751608in}{1.849723in}}%
\pgfpathlineto{\pgfqpoint{2.761080in}{1.855038in}}%
\pgfpathlineto{\pgfqpoint{2.770553in}{1.854671in}}%
\pgfpathlineto{\pgfqpoint{2.780025in}{1.859105in}}%
\pgfpathlineto{\pgfqpoint{2.798970in}{1.864641in}}%
\pgfpathlineto{\pgfqpoint{2.808443in}{1.869266in}}%
\pgfpathlineto{\pgfqpoint{2.817915in}{1.871155in}}%
\pgfpathlineto{\pgfqpoint{2.836860in}{1.877077in}}%
\pgfpathlineto{\pgfqpoint{2.846333in}{1.877106in}}%
\pgfpathlineto{\pgfqpoint{2.855805in}{1.879675in}}%
\pgfpathlineto{\pgfqpoint{2.865278in}{1.884305in}}%
\pgfpathlineto{\pgfqpoint{2.884223in}{1.882793in}}%
\pgfpathlineto{\pgfqpoint{2.893695in}{1.886341in}}%
\pgfpathlineto{\pgfqpoint{2.903168in}{1.886953in}}%
\pgfpathlineto{\pgfqpoint{2.912640in}{1.892464in}}%
\pgfpathlineto{\pgfqpoint{2.922113in}{1.894250in}}%
\pgfpathlineto{\pgfqpoint{2.931585in}{1.898586in}}%
\pgfpathlineto{\pgfqpoint{2.941058in}{1.901156in}}%
\pgfpathlineto{\pgfqpoint{2.960003in}{1.911977in}}%
\pgfpathlineto{\pgfqpoint{2.969475in}{1.923062in}}%
\pgfpathlineto{\pgfqpoint{2.978948in}{1.927496in}}%
\pgfpathlineto{\pgfqpoint{2.988420in}{1.929478in}}%
\pgfpathlineto{\pgfqpoint{3.007365in}{1.926987in}}%
\pgfpathlineto{\pgfqpoint{3.026310in}{1.932131in}}%
\pgfpathlineto{\pgfqpoint{3.035783in}{1.935777in}}%
\pgfpathlineto{\pgfqpoint{3.045255in}{1.936100in}}%
\pgfpathlineto{\pgfqpoint{3.054728in}{1.934167in}}%
\pgfpathlineto{\pgfqpoint{3.064200in}{1.936350in}}%
\pgfpathlineto{\pgfqpoint{3.073673in}{1.940289in}}%
\pgfpathlineto{\pgfqpoint{3.092618in}{1.942888in}}%
\pgfpathlineto{\pgfqpoint{3.102090in}{1.950944in}}%
\pgfpathlineto{\pgfqpoint{3.121035in}{1.956572in}}%
\pgfpathlineto{\pgfqpoint{3.139980in}{1.957213in}}%
\pgfpathlineto{\pgfqpoint{3.149453in}{1.959396in}}%
\pgfpathlineto{\pgfqpoint{3.158925in}{1.964902in}}%
\pgfpathlineto{\pgfqpoint{3.168398in}{1.968162in}}%
\pgfpathlineto{\pgfqpoint{3.177870in}{1.968480in}}%
\pgfpathlineto{\pgfqpoint{3.225233in}{1.979183in}}%
\pgfpathlineto{\pgfqpoint{3.234705in}{1.983911in}}%
\pgfpathlineto{\pgfqpoint{3.244178in}{1.983544in}}%
\pgfpathlineto{\pgfqpoint{3.253650in}{1.989148in}}%
\pgfpathlineto{\pgfqpoint{3.272595in}{1.989593in}}%
\pgfpathlineto{\pgfqpoint{3.291540in}{1.998358in}}%
\pgfpathlineto{\pgfqpoint{3.301013in}{1.998779in}}%
\pgfpathlineto{\pgfqpoint{3.310485in}{1.997140in}}%
\pgfpathlineto{\pgfqpoint{3.329430in}{2.000326in}}%
\pgfpathlineto{\pgfqpoint{3.338903in}{2.005734in}}%
\pgfpathlineto{\pgfqpoint{3.367320in}{2.014724in}}%
\pgfpathlineto{\pgfqpoint{3.376793in}{2.015043in}}%
\pgfpathlineto{\pgfqpoint{3.386265in}{2.012233in}}%
\pgfpathlineto{\pgfqpoint{3.395737in}{2.013237in}}%
\pgfpathlineto{\pgfqpoint{3.405210in}{2.021190in}}%
\pgfpathlineto{\pgfqpoint{3.414682in}{2.027092in}}%
\pgfpathlineto{\pgfqpoint{3.424155in}{2.026823in}}%
\pgfpathlineto{\pgfqpoint{3.433627in}{2.029691in}}%
\pgfpathlineto{\pgfqpoint{3.443100in}{2.027464in}}%
\pgfpathlineto{\pgfqpoint{3.452572in}{2.033073in}}%
\pgfpathlineto{\pgfqpoint{3.462045in}{2.034370in}}%
\pgfpathlineto{\pgfqpoint{3.471517in}{2.037624in}}%
\pgfpathlineto{\pgfqpoint{3.480990in}{2.045191in}}%
\pgfpathlineto{\pgfqpoint{3.490462in}{2.046977in}}%
\pgfpathlineto{\pgfqpoint{3.499935in}{2.045832in}}%
\pgfpathlineto{\pgfqpoint{3.509407in}{2.048695in}}%
\pgfpathlineto{\pgfqpoint{3.518880in}{2.049703in}}%
\pgfpathlineto{\pgfqpoint{3.537825in}{2.047501in}}%
\pgfpathlineto{\pgfqpoint{3.556770in}{2.055385in}}%
\pgfpathlineto{\pgfqpoint{3.585187in}{2.060362in}}%
\pgfpathlineto{\pgfqpoint{3.594660in}{2.064302in}}%
\pgfpathlineto{\pgfqpoint{3.604132in}{2.064816in}}%
\pgfpathlineto{\pgfqpoint{3.613605in}{2.069348in}}%
\pgfpathlineto{\pgfqpoint{3.623077in}{2.066730in}}%
\pgfpathlineto{\pgfqpoint{3.632550in}{2.071555in}}%
\pgfpathlineto{\pgfqpoint{3.651495in}{2.072490in}}%
\pgfpathlineto{\pgfqpoint{3.679912in}{2.086468in}}%
\pgfpathlineto{\pgfqpoint{3.689385in}{2.088058in}}%
\pgfpathlineto{\pgfqpoint{3.717802in}{2.097440in}}%
\pgfpathlineto{\pgfqpoint{3.727275in}{2.095116in}}%
\pgfpathlineto{\pgfqpoint{3.736747in}{2.097103in}}%
\pgfpathlineto{\pgfqpoint{3.746220in}{2.101434in}}%
\pgfpathlineto{\pgfqpoint{3.755692in}{2.100284in}}%
\pgfpathlineto{\pgfqpoint{3.765165in}{2.101586in}}%
\pgfpathlineto{\pgfqpoint{3.774637in}{2.106015in}}%
\pgfpathlineto{\pgfqpoint{3.784110in}{2.108589in}}%
\pgfpathlineto{\pgfqpoint{3.803055in}{2.107860in}}%
\pgfpathlineto{\pgfqpoint{3.822000in}{2.113684in}}%
\pgfpathlineto{\pgfqpoint{3.831472in}{2.119097in}}%
\pgfpathlineto{\pgfqpoint{3.840945in}{2.120590in}}%
\pgfpathlineto{\pgfqpoint{3.850417in}{2.124437in}}%
\pgfpathlineto{\pgfqpoint{3.859890in}{2.124950in}}%
\pgfpathlineto{\pgfqpoint{3.869362in}{2.123218in}}%
\pgfpathlineto{\pgfqpoint{3.888307in}{2.127182in}}%
\pgfpathlineto{\pgfqpoint{3.897780in}{2.124960in}}%
\pgfpathlineto{\pgfqpoint{3.907252in}{2.126355in}}%
\pgfpathlineto{\pgfqpoint{3.916725in}{2.126091in}}%
\pgfpathlineto{\pgfqpoint{3.926197in}{2.129541in}}%
\pgfpathlineto{\pgfqpoint{3.935670in}{2.125753in}}%
\pgfpathlineto{\pgfqpoint{3.954615in}{2.124632in}}%
\pgfpathlineto{\pgfqpoint{3.964087in}{2.127495in}}%
\pgfpathlineto{\pgfqpoint{3.973560in}{2.122821in}}%
\pgfpathlineto{\pgfqpoint{3.983032in}{2.128920in}}%
\pgfpathlineto{\pgfqpoint{3.992505in}{2.125910in}}%
\pgfpathlineto{\pgfqpoint{4.011450in}{2.128508in}}%
\pgfpathlineto{\pgfqpoint{4.020922in}{2.132649in}}%
\pgfpathlineto{\pgfqpoint{4.039867in}{2.132111in}}%
\pgfpathlineto{\pgfqpoint{4.049340in}{2.133412in}}%
\pgfpathlineto{\pgfqpoint{4.058812in}{2.137646in}}%
\pgfpathlineto{\pgfqpoint{4.068285in}{2.139046in}}%
\pgfpathlineto{\pgfqpoint{4.087230in}{2.153195in}}%
\pgfpathlineto{\pgfqpoint{4.096702in}{2.151555in}}%
\pgfpathlineto{\pgfqpoint{4.106175in}{2.157550in}}%
\pgfpathlineto{\pgfqpoint{4.144065in}{2.164706in}}%
\pgfpathlineto{\pgfqpoint{4.153537in}{2.164148in}}%
\pgfpathlineto{\pgfqpoint{4.172482in}{2.170461in}}%
\pgfpathlineto{\pgfqpoint{4.181955in}{2.167945in}}%
\pgfpathlineto{\pgfqpoint{4.191427in}{2.171690in}}%
\pgfpathlineto{\pgfqpoint{4.200900in}{2.171523in}}%
\pgfpathlineto{\pgfqpoint{4.210372in}{2.175267in}}%
\pgfpathlineto{\pgfqpoint{4.229317in}{2.178845in}}%
\pgfpathlineto{\pgfqpoint{4.257735in}{2.171484in}}%
\pgfpathlineto{\pgfqpoint{4.276680in}{2.173300in}}%
\pgfpathlineto{\pgfqpoint{4.286152in}{2.170295in}}%
\pgfpathlineto{\pgfqpoint{4.295625in}{2.173549in}}%
\pgfpathlineto{\pgfqpoint{4.305097in}{2.179256in}}%
\pgfpathlineto{\pgfqpoint{4.314570in}{2.182021in}}%
\pgfpathlineto{\pgfqpoint{4.324042in}{2.179990in}}%
\pgfpathlineto{\pgfqpoint{4.352460in}{2.187708in}}%
\pgfpathlineto{\pgfqpoint{4.361932in}{2.187145in}}%
\pgfpathlineto{\pgfqpoint{4.380877in}{2.193268in}}%
\pgfpathlineto{\pgfqpoint{4.390350in}{2.197012in}}%
\pgfpathlineto{\pgfqpoint{4.399822in}{2.193615in}}%
\pgfpathlineto{\pgfqpoint{4.418767in}{2.204338in}}%
\pgfpathlineto{\pgfqpoint{4.428240in}{2.203286in}}%
\pgfpathlineto{\pgfqpoint{4.447185in}{2.212247in}}%
\pgfpathlineto{\pgfqpoint{4.456657in}{2.220983in}}%
\pgfpathlineto{\pgfqpoint{4.466130in}{2.219153in}}%
\pgfpathlineto{\pgfqpoint{4.475602in}{2.215458in}}%
\pgfpathlineto{\pgfqpoint{4.485075in}{2.217249in}}%
\pgfpathlineto{\pgfqpoint{4.494547in}{2.214435in}}%
\pgfpathlineto{\pgfqpoint{4.504020in}{2.214464in}}%
\pgfpathlineto{\pgfqpoint{4.532437in}{2.207593in}}%
\pgfpathlineto{\pgfqpoint{4.541910in}{2.208890in}}%
\pgfpathlineto{\pgfqpoint{4.551382in}{2.208528in}}%
\pgfpathlineto{\pgfqpoint{4.560855in}{2.211782in}}%
\pgfpathlineto{\pgfqpoint{4.570327in}{2.208777in}}%
\pgfpathlineto{\pgfqpoint{4.579800in}{2.212130in}}%
\pgfpathlineto{\pgfqpoint{4.589272in}{2.210691in}}%
\pgfpathlineto{\pgfqpoint{4.598745in}{2.211890in}}%
\pgfpathlineto{\pgfqpoint{4.608217in}{2.214753in}}%
\pgfpathlineto{\pgfqpoint{4.617690in}{2.220753in}}%
\pgfpathlineto{\pgfqpoint{4.627162in}{2.219701in}}%
\pgfpathlineto{\pgfqpoint{4.636635in}{2.215228in}}%
\pgfpathlineto{\pgfqpoint{4.646107in}{2.212609in}}%
\pgfpathlineto{\pgfqpoint{4.674525in}{2.213569in}}%
\pgfpathlineto{\pgfqpoint{4.683997in}{2.219960in}}%
\pgfpathlineto{\pgfqpoint{4.693470in}{2.222236in}}%
\pgfpathlineto{\pgfqpoint{4.702942in}{2.218546in}}%
\pgfpathlineto{\pgfqpoint{4.721887in}{2.223592in}}%
\pgfpathlineto{\pgfqpoint{4.740832in}{2.231868in}}%
\pgfpathlineto{\pgfqpoint{4.750305in}{2.232773in}}%
\pgfpathlineto{\pgfqpoint{4.759777in}{2.236713in}}%
\pgfpathlineto{\pgfqpoint{4.769250in}{2.238015in}}%
\pgfpathlineto{\pgfqpoint{4.778722in}{2.242542in}}%
\pgfpathlineto{\pgfqpoint{4.788195in}{2.243354in}}%
\pgfpathlineto{\pgfqpoint{4.797667in}{2.248077in}}%
\pgfpathlineto{\pgfqpoint{4.807140in}{2.248205in}}%
\pgfpathlineto{\pgfqpoint{4.816612in}{2.251166in}}%
\pgfpathlineto{\pgfqpoint{4.826085in}{2.251092in}}%
\pgfpathlineto{\pgfqpoint{4.835557in}{2.257973in}}%
\pgfpathlineto{\pgfqpoint{4.845030in}{2.254963in}}%
\pgfpathlineto{\pgfqpoint{4.854502in}{2.254308in}}%
\pgfpathlineto{\pgfqpoint{4.863975in}{2.262163in}}%
\pgfpathlineto{\pgfqpoint{4.873447in}{2.263073in}}%
\pgfpathlineto{\pgfqpoint{4.892392in}{2.271246in}}%
\pgfpathlineto{\pgfqpoint{4.901865in}{2.273037in}}%
\pgfpathlineto{\pgfqpoint{4.911337in}{2.269049in}}%
\pgfpathlineto{\pgfqpoint{4.939755in}{2.274418in}}%
\pgfpathlineto{\pgfqpoint{4.949227in}{2.280021in}}%
\pgfpathlineto{\pgfqpoint{4.968172in}{2.282131in}}%
\pgfpathlineto{\pgfqpoint{4.977645in}{2.285288in}}%
\pgfpathlineto{\pgfqpoint{4.996590in}{2.288571in}}%
\pgfpathlineto{\pgfqpoint{5.015535in}{2.285199in}}%
\pgfpathlineto{\pgfqpoint{5.025007in}{2.289242in}}%
\pgfpathlineto{\pgfqpoint{5.043952in}{2.299569in}}%
\pgfpathlineto{\pgfqpoint{5.053425in}{2.296466in}}%
\pgfpathlineto{\pgfqpoint{5.081842in}{2.302809in}}%
\pgfpathlineto{\pgfqpoint{5.091315in}{2.302055in}}%
\pgfpathlineto{\pgfqpoint{5.100787in}{2.306484in}}%
\pgfpathlineto{\pgfqpoint{5.119732in}{2.317697in}}%
\pgfpathlineto{\pgfqpoint{5.148149in}{2.319341in}}%
\pgfpathlineto{\pgfqpoint{5.157622in}{2.318881in}}%
\pgfpathlineto{\pgfqpoint{5.167094in}{2.320178in}}%
\pgfpathlineto{\pgfqpoint{5.176567in}{2.319909in}}%
\pgfpathlineto{\pgfqpoint{5.186039in}{2.314750in}}%
\pgfpathlineto{\pgfqpoint{5.195512in}{2.319963in}}%
\pgfpathlineto{\pgfqpoint{5.204984in}{2.318622in}}%
\pgfpathlineto{\pgfqpoint{5.223929in}{2.314075in}}%
\pgfpathlineto{\pgfqpoint{5.233402in}{2.313121in}}%
\pgfpathlineto{\pgfqpoint{5.242874in}{2.319507in}}%
\pgfpathlineto{\pgfqpoint{5.252347in}{2.321397in}}%
\pgfpathlineto{\pgfqpoint{5.271292in}{2.321548in}}%
\pgfpathlineto{\pgfqpoint{5.280764in}{2.323726in}}%
\pgfpathlineto{\pgfqpoint{5.290237in}{2.323853in}}%
\pgfpathlineto{\pgfqpoint{5.299709in}{2.330925in}}%
\pgfpathlineto{\pgfqpoint{5.309182in}{2.332423in}}%
\pgfpathlineto{\pgfqpoint{5.318654in}{2.339006in}}%
\pgfpathlineto{\pgfqpoint{5.328127in}{2.338639in}}%
\pgfpathlineto{\pgfqpoint{5.337599in}{2.335829in}}%
\pgfpathlineto{\pgfqpoint{5.356544in}{2.337058in}}%
\pgfpathlineto{\pgfqpoint{5.366017in}{2.339431in}}%
\pgfpathlineto{\pgfqpoint{5.375489in}{2.337503in}}%
\pgfpathlineto{\pgfqpoint{5.384962in}{2.332731in}}%
\pgfpathlineto{\pgfqpoint{5.394434in}{2.336182in}}%
\pgfpathlineto{\pgfqpoint{5.403907in}{2.335330in}}%
\pgfpathlineto{\pgfqpoint{5.413379in}{2.336627in}}%
\pgfpathlineto{\pgfqpoint{5.432324in}{2.340890in}}%
\pgfpathlineto{\pgfqpoint{5.441797in}{2.340919in}}%
\pgfpathlineto{\pgfqpoint{5.479687in}{2.349636in}}%
\pgfpathlineto{\pgfqpoint{5.489159in}{2.351231in}}%
\pgfpathlineto{\pgfqpoint{5.489159in}{2.351231in}}%
\pgfusepath{stroke}%
\end{pgfscope}%
\begin{pgfscope}%
\pgfpathrectangle{\pgfqpoint{0.762383in}{0.471179in}}{\pgfqpoint{4.726776in}{2.845920in}} %
\pgfusepath{clip}%
\pgfsetrectcap%
\pgfsetroundjoin%
\pgfsetlinewidth{1.505625pt}%
\definecolor{currentstroke}{rgb}{0.580392,0.403922,0.741176}%
\pgfsetstrokecolor{currentstroke}%
\pgfsetdash{}{0pt}%
\pgfpathmoveto{\pgfqpoint{0.762383in}{0.600539in}}%
\pgfpathlineto{\pgfqpoint{0.771856in}{0.688562in}}%
\pgfpathlineto{\pgfqpoint{0.781328in}{0.724022in}}%
\pgfpathlineto{\pgfqpoint{0.800273in}{0.776541in}}%
\pgfpathlineto{\pgfqpoint{0.828691in}{0.840538in}}%
\pgfpathlineto{\pgfqpoint{0.838163in}{0.865623in}}%
\pgfpathlineto{\pgfqpoint{0.866581in}{0.916112in}}%
\pgfpathlineto{\pgfqpoint{0.894998in}{0.976292in}}%
\pgfpathlineto{\pgfqpoint{0.923416in}{1.015232in}}%
\pgfpathlineto{\pgfqpoint{0.942361in}{1.042008in}}%
\pgfpathlineto{\pgfqpoint{0.951833in}{1.057988in}}%
\pgfpathlineto{\pgfqpoint{0.961306in}{1.069469in}}%
\pgfpathlineto{\pgfqpoint{0.989723in}{1.112322in}}%
\pgfpathlineto{\pgfqpoint{1.008668in}{1.132442in}}%
\pgfpathlineto{\pgfqpoint{1.018141in}{1.148128in}}%
\pgfpathlineto{\pgfqpoint{1.046558in}{1.167783in}}%
\pgfpathlineto{\pgfqpoint{1.074976in}{1.205943in}}%
\pgfpathlineto{\pgfqpoint{1.103393in}{1.240182in}}%
\pgfpathlineto{\pgfqpoint{1.112866in}{1.248331in}}%
\pgfpathlineto{\pgfqpoint{1.131811in}{1.260033in}}%
\pgfpathlineto{\pgfqpoint{1.188646in}{1.319707in}}%
\pgfpathlineto{\pgfqpoint{1.198118in}{1.324919in}}%
\pgfpathlineto{\pgfqpoint{1.245481in}{1.380849in}}%
\pgfpathlineto{\pgfqpoint{1.254953in}{1.385964in}}%
\pgfpathlineto{\pgfqpoint{1.292843in}{1.427770in}}%
\pgfpathlineto{\pgfqpoint{1.302316in}{1.435331in}}%
\pgfpathlineto{\pgfqpoint{1.311788in}{1.446128in}}%
\pgfpathlineto{\pgfqpoint{1.330733in}{1.455769in}}%
\pgfpathlineto{\pgfqpoint{1.349678in}{1.472855in}}%
\pgfpathlineto{\pgfqpoint{1.387568in}{1.501740in}}%
\pgfpathlineto{\pgfqpoint{1.397041in}{1.507931in}}%
\pgfpathlineto{\pgfqpoint{1.415986in}{1.525310in}}%
\pgfpathlineto{\pgfqpoint{1.444403in}{1.540565in}}%
\pgfpathlineto{\pgfqpoint{1.453876in}{1.543330in}}%
\pgfpathlineto{\pgfqpoint{1.463348in}{1.549918in}}%
\pgfpathlineto{\pgfqpoint{1.472821in}{1.561395in}}%
\pgfpathlineto{\pgfqpoint{1.482293in}{1.568173in}}%
\pgfpathlineto{\pgfqpoint{1.510711in}{1.600557in}}%
\pgfpathlineto{\pgfqpoint{1.520183in}{1.605280in}}%
\pgfpathlineto{\pgfqpoint{1.539128in}{1.624030in}}%
\pgfpathlineto{\pgfqpoint{1.548601in}{1.634038in}}%
\pgfpathlineto{\pgfqpoint{1.595963in}{1.670000in}}%
\pgfpathlineto{\pgfqpoint{1.605436in}{1.670123in}}%
\pgfpathlineto{\pgfqpoint{1.624381in}{1.683489in}}%
\pgfpathlineto{\pgfqpoint{1.643325in}{1.697246in}}%
\pgfpathlineto{\pgfqpoint{1.652798in}{1.707059in}}%
\pgfpathlineto{\pgfqpoint{1.662270in}{1.709633in}}%
\pgfpathlineto{\pgfqpoint{1.671743in}{1.717586in}}%
\pgfpathlineto{\pgfqpoint{1.681215in}{1.722505in}}%
\pgfpathlineto{\pgfqpoint{1.700160in}{1.738611in}}%
\pgfpathlineto{\pgfqpoint{1.719105in}{1.746692in}}%
\pgfpathlineto{\pgfqpoint{1.738050in}{1.762211in}}%
\pgfpathlineto{\pgfqpoint{1.747523in}{1.771148in}}%
\pgfpathlineto{\pgfqpoint{1.766468in}{1.783041in}}%
\pgfpathlineto{\pgfqpoint{1.775940in}{1.789922in}}%
\pgfpathlineto{\pgfqpoint{1.794885in}{1.794968in}}%
\pgfpathlineto{\pgfqpoint{1.804358in}{1.803116in}}%
\pgfpathlineto{\pgfqpoint{1.813830in}{1.807746in}}%
\pgfpathlineto{\pgfqpoint{1.823303in}{1.814133in}}%
\pgfpathlineto{\pgfqpoint{1.842248in}{1.824465in}}%
\pgfpathlineto{\pgfqpoint{1.851720in}{1.834082in}}%
\pgfpathlineto{\pgfqpoint{1.861193in}{1.834111in}}%
\pgfpathlineto{\pgfqpoint{1.880138in}{1.845421in}}%
\pgfpathlineto{\pgfqpoint{1.889610in}{1.852298in}}%
\pgfpathlineto{\pgfqpoint{1.899083in}{1.854280in}}%
\pgfpathlineto{\pgfqpoint{1.918028in}{1.867646in}}%
\pgfpathlineto{\pgfqpoint{1.936973in}{1.870538in}}%
\pgfpathlineto{\pgfqpoint{1.955918in}{1.881946in}}%
\pgfpathlineto{\pgfqpoint{1.965390in}{1.893717in}}%
\pgfpathlineto{\pgfqpoint{1.993808in}{1.907014in}}%
\pgfpathlineto{\pgfqpoint{2.003280in}{1.910954in}}%
\pgfpathlineto{\pgfqpoint{2.012753in}{1.917737in}}%
\pgfpathlineto{\pgfqpoint{2.022225in}{1.926669in}}%
\pgfpathlineto{\pgfqpoint{2.031698in}{1.932865in}}%
\pgfpathlineto{\pgfqpoint{2.050643in}{1.949456in}}%
\pgfpathlineto{\pgfqpoint{2.069588in}{1.954796in}}%
\pgfpathlineto{\pgfqpoint{2.079060in}{1.962460in}}%
\pgfpathlineto{\pgfqpoint{2.107478in}{1.975067in}}%
\pgfpathlineto{\pgfqpoint{2.116950in}{1.973819in}}%
\pgfpathlineto{\pgfqpoint{2.135895in}{1.982976in}}%
\pgfpathlineto{\pgfqpoint{2.145368in}{1.987704in}}%
\pgfpathlineto{\pgfqpoint{2.154840in}{1.995657in}}%
\pgfpathlineto{\pgfqpoint{2.173785in}{2.003150in}}%
\pgfpathlineto{\pgfqpoint{2.183258in}{2.006894in}}%
\pgfpathlineto{\pgfqpoint{2.192730in}{2.018082in}}%
\pgfpathlineto{\pgfqpoint{2.221148in}{2.040869in}}%
\pgfpathlineto{\pgfqpoint{2.230620in}{2.040507in}}%
\pgfpathlineto{\pgfqpoint{2.240093in}{2.044838in}}%
\pgfpathlineto{\pgfqpoint{2.249565in}{2.054455in}}%
\pgfpathlineto{\pgfqpoint{2.287455in}{2.082655in}}%
\pgfpathlineto{\pgfqpoint{2.306400in}{2.084471in}}%
\pgfpathlineto{\pgfqpoint{2.315873in}{2.083228in}}%
\pgfpathlineto{\pgfqpoint{2.325345in}{2.087951in}}%
\pgfpathlineto{\pgfqpoint{2.334818in}{2.090912in}}%
\pgfpathlineto{\pgfqpoint{2.344290in}{2.096422in}}%
\pgfpathlineto{\pgfqpoint{2.363235in}{2.112040in}}%
\pgfpathlineto{\pgfqpoint{2.382180in}{2.113855in}}%
\pgfpathlineto{\pgfqpoint{2.391653in}{2.119068in}}%
\pgfpathlineto{\pgfqpoint{2.401125in}{2.127412in}}%
\pgfpathlineto{\pgfqpoint{2.420070in}{2.133828in}}%
\pgfpathlineto{\pgfqpoint{2.429543in}{2.142471in}}%
\pgfpathlineto{\pgfqpoint{2.448488in}{2.145755in}}%
\pgfpathlineto{\pgfqpoint{2.467433in}{2.151775in}}%
\pgfpathlineto{\pgfqpoint{2.476905in}{2.152490in}}%
\pgfpathlineto{\pgfqpoint{2.486378in}{2.156234in}}%
\pgfpathlineto{\pgfqpoint{2.495850in}{2.156948in}}%
\pgfpathlineto{\pgfqpoint{2.505323in}{2.160594in}}%
\pgfpathlineto{\pgfqpoint{2.524268in}{2.170339in}}%
\pgfpathlineto{\pgfqpoint{2.533740in}{2.173789in}}%
\pgfpathlineto{\pgfqpoint{2.543213in}{2.179202in}}%
\pgfpathlineto{\pgfqpoint{2.552685in}{2.178835in}}%
\pgfpathlineto{\pgfqpoint{2.581103in}{2.194775in}}%
\pgfpathlineto{\pgfqpoint{2.590575in}{2.205371in}}%
\pgfpathlineto{\pgfqpoint{2.600048in}{2.206477in}}%
\pgfpathlineto{\pgfqpoint{2.647410in}{2.225403in}}%
\pgfpathlineto{\pgfqpoint{2.656883in}{2.225623in}}%
\pgfpathlineto{\pgfqpoint{2.666355in}{2.227414in}}%
\pgfpathlineto{\pgfqpoint{2.675828in}{2.225579in}}%
\pgfpathlineto{\pgfqpoint{2.694773in}{2.242273in}}%
\pgfpathlineto{\pgfqpoint{2.704245in}{2.247877in}}%
\pgfpathlineto{\pgfqpoint{2.713718in}{2.246340in}}%
\pgfpathlineto{\pgfqpoint{2.723190in}{2.252727in}}%
\pgfpathlineto{\pgfqpoint{2.751608in}{2.264551in}}%
\pgfpathlineto{\pgfqpoint{2.761080in}{2.272509in}}%
\pgfpathlineto{\pgfqpoint{2.770553in}{2.274393in}}%
\pgfpathlineto{\pgfqpoint{2.780025in}{2.280883in}}%
\pgfpathlineto{\pgfqpoint{2.789498in}{2.298233in}}%
\pgfpathlineto{\pgfqpoint{2.817915in}{2.315837in}}%
\pgfpathlineto{\pgfqpoint{2.846333in}{2.322865in}}%
\pgfpathlineto{\pgfqpoint{2.855805in}{2.321715in}}%
\pgfpathlineto{\pgfqpoint{2.865278in}{2.328008in}}%
\pgfpathlineto{\pgfqpoint{2.874750in}{2.331361in}}%
\pgfpathlineto{\pgfqpoint{2.884223in}{2.331782in}}%
\pgfpathlineto{\pgfqpoint{2.903168in}{2.335746in}}%
\pgfpathlineto{\pgfqpoint{2.912640in}{2.333720in}}%
\pgfpathlineto{\pgfqpoint{2.922113in}{2.338051in}}%
\pgfpathlineto{\pgfqpoint{2.941058in}{2.354647in}}%
\pgfpathlineto{\pgfqpoint{2.960003in}{2.362336in}}%
\pgfpathlineto{\pgfqpoint{2.988420in}{2.386102in}}%
\pgfpathlineto{\pgfqpoint{3.016838in}{2.396071in}}%
\pgfpathlineto{\pgfqpoint{3.026310in}{2.393942in}}%
\pgfpathlineto{\pgfqpoint{3.045255in}{2.406329in}}%
\pgfpathlineto{\pgfqpoint{3.054728in}{2.408410in}}%
\pgfpathlineto{\pgfqpoint{3.073673in}{2.406408in}}%
\pgfpathlineto{\pgfqpoint{3.083145in}{2.416519in}}%
\pgfpathlineto{\pgfqpoint{3.102090in}{2.419412in}}%
\pgfpathlineto{\pgfqpoint{3.111563in}{2.425211in}}%
\pgfpathlineto{\pgfqpoint{3.121035in}{2.428074in}}%
\pgfpathlineto{\pgfqpoint{3.139980in}{2.440657in}}%
\pgfpathlineto{\pgfqpoint{3.149453in}{2.440295in}}%
\pgfpathlineto{\pgfqpoint{3.158925in}{2.444920in}}%
\pgfpathlineto{\pgfqpoint{3.168398in}{2.455227in}}%
\pgfpathlineto{\pgfqpoint{3.177870in}{2.457992in}}%
\pgfpathlineto{\pgfqpoint{3.196815in}{2.469498in}}%
\pgfpathlineto{\pgfqpoint{3.225233in}{2.477016in}}%
\pgfpathlineto{\pgfqpoint{3.244178in}{2.487054in}}%
\pgfpathlineto{\pgfqpoint{3.263123in}{2.492002in}}%
\pgfpathlineto{\pgfqpoint{3.272595in}{2.491830in}}%
\pgfpathlineto{\pgfqpoint{3.282068in}{2.493328in}}%
\pgfpathlineto{\pgfqpoint{3.291540in}{2.498638in}}%
\pgfpathlineto{\pgfqpoint{3.301013in}{2.500429in}}%
\pgfpathlineto{\pgfqpoint{3.310485in}{2.508774in}}%
\pgfpathlineto{\pgfqpoint{3.319958in}{2.509288in}}%
\pgfpathlineto{\pgfqpoint{3.329430in}{2.513722in}}%
\pgfpathlineto{\pgfqpoint{3.338903in}{2.515704in}}%
\pgfpathlineto{\pgfqpoint{3.376793in}{2.534311in}}%
\pgfpathlineto{\pgfqpoint{3.386265in}{2.530915in}}%
\pgfpathlineto{\pgfqpoint{3.405210in}{2.534096in}}%
\pgfpathlineto{\pgfqpoint{3.414682in}{2.542641in}}%
\pgfpathlineto{\pgfqpoint{3.433627in}{2.549253in}}%
\pgfpathlineto{\pgfqpoint{3.443100in}{2.548103in}}%
\pgfpathlineto{\pgfqpoint{3.452572in}{2.550090in}}%
\pgfpathlineto{\pgfqpoint{3.462045in}{2.549429in}}%
\pgfpathlineto{\pgfqpoint{3.480990in}{2.561523in}}%
\pgfpathlineto{\pgfqpoint{3.490462in}{2.567225in}}%
\pgfpathlineto{\pgfqpoint{3.499935in}{2.562751in}}%
\pgfpathlineto{\pgfqpoint{3.547297in}{2.577566in}}%
\pgfpathlineto{\pgfqpoint{3.556770in}{2.582876in}}%
\pgfpathlineto{\pgfqpoint{3.566242in}{2.579382in}}%
\pgfpathlineto{\pgfqpoint{3.585187in}{2.586581in}}%
\pgfpathlineto{\pgfqpoint{3.604132in}{2.593188in}}%
\pgfpathlineto{\pgfqpoint{3.613605in}{2.593315in}}%
\pgfpathlineto{\pgfqpoint{3.623077in}{2.588935in}}%
\pgfpathlineto{\pgfqpoint{3.632550in}{2.590824in}}%
\pgfpathlineto{\pgfqpoint{3.642022in}{2.598386in}}%
\pgfpathlineto{\pgfqpoint{3.651495in}{2.602526in}}%
\pgfpathlineto{\pgfqpoint{3.660967in}{2.602746in}}%
\pgfpathlineto{\pgfqpoint{3.670440in}{2.606887in}}%
\pgfpathlineto{\pgfqpoint{3.679912in}{2.616112in}}%
\pgfpathlineto{\pgfqpoint{3.689385in}{2.618584in}}%
\pgfpathlineto{\pgfqpoint{3.698857in}{2.626542in}}%
\pgfpathlineto{\pgfqpoint{3.708330in}{2.626958in}}%
\pgfpathlineto{\pgfqpoint{3.717802in}{2.634818in}}%
\pgfpathlineto{\pgfqpoint{3.727275in}{2.638464in}}%
\pgfpathlineto{\pgfqpoint{3.736747in}{2.639276in}}%
\pgfpathlineto{\pgfqpoint{3.755692in}{2.648820in}}%
\pgfpathlineto{\pgfqpoint{3.774637in}{2.659249in}}%
\pgfpathlineto{\pgfqpoint{3.784110in}{2.665445in}}%
\pgfpathlineto{\pgfqpoint{3.803055in}{2.673819in}}%
\pgfpathlineto{\pgfqpoint{3.812527in}{2.674627in}}%
\pgfpathlineto{\pgfqpoint{3.831472in}{2.687209in}}%
\pgfpathlineto{\pgfqpoint{3.840945in}{2.688800in}}%
\pgfpathlineto{\pgfqpoint{3.859890in}{2.697565in}}%
\pgfpathlineto{\pgfqpoint{3.878835in}{2.698696in}}%
\pgfpathlineto{\pgfqpoint{3.888307in}{2.703615in}}%
\pgfpathlineto{\pgfqpoint{3.897780in}{2.711475in}}%
\pgfpathlineto{\pgfqpoint{3.907252in}{2.712282in}}%
\pgfpathlineto{\pgfqpoint{3.935670in}{2.719511in}}%
\pgfpathlineto{\pgfqpoint{3.945142in}{2.716403in}}%
\pgfpathlineto{\pgfqpoint{3.964087in}{2.721742in}}%
\pgfpathlineto{\pgfqpoint{4.001977in}{2.733498in}}%
\pgfpathlineto{\pgfqpoint{4.011450in}{2.730586in}}%
\pgfpathlineto{\pgfqpoint{4.020922in}{2.735314in}}%
\pgfpathlineto{\pgfqpoint{4.030395in}{2.736905in}}%
\pgfpathlineto{\pgfqpoint{4.039867in}{2.733503in}}%
\pgfpathlineto{\pgfqpoint{4.049340in}{2.734609in}}%
\pgfpathlineto{\pgfqpoint{4.068285in}{2.745430in}}%
\pgfpathlineto{\pgfqpoint{4.077757in}{2.745846in}}%
\pgfpathlineto{\pgfqpoint{4.096702in}{2.757744in}}%
\pgfpathlineto{\pgfqpoint{4.106175in}{2.760411in}}%
\pgfpathlineto{\pgfqpoint{4.115647in}{2.768663in}}%
\pgfpathlineto{\pgfqpoint{4.134592in}{2.778798in}}%
\pgfpathlineto{\pgfqpoint{4.144065in}{2.782738in}}%
\pgfpathlineto{\pgfqpoint{4.153537in}{2.780125in}}%
\pgfpathlineto{\pgfqpoint{4.163010in}{2.780345in}}%
\pgfpathlineto{\pgfqpoint{4.172482in}{2.787417in}}%
\pgfpathlineto{\pgfqpoint{4.191427in}{2.791093in}}%
\pgfpathlineto{\pgfqpoint{4.200900in}{2.796603in}}%
\pgfpathlineto{\pgfqpoint{4.210372in}{2.795943in}}%
\pgfpathlineto{\pgfqpoint{4.219845in}{2.800279in}}%
\pgfpathlineto{\pgfqpoint{4.229317in}{2.801478in}}%
\pgfpathlineto{\pgfqpoint{4.238790in}{2.806206in}}%
\pgfpathlineto{\pgfqpoint{4.248262in}{2.808188in}}%
\pgfpathlineto{\pgfqpoint{4.267207in}{2.807850in}}%
\pgfpathlineto{\pgfqpoint{4.276680in}{2.810322in}}%
\pgfpathlineto{\pgfqpoint{4.286152in}{2.818182in}}%
\pgfpathlineto{\pgfqpoint{4.295625in}{2.812333in}}%
\pgfpathlineto{\pgfqpoint{4.305097in}{2.811677in}}%
\pgfpathlineto{\pgfqpoint{4.333515in}{2.819097in}}%
\pgfpathlineto{\pgfqpoint{4.342987in}{2.821177in}}%
\pgfpathlineto{\pgfqpoint{4.361932in}{2.834151in}}%
\pgfpathlineto{\pgfqpoint{4.371405in}{2.835551in}}%
\pgfpathlineto{\pgfqpoint{4.380877in}{2.839491in}}%
\pgfpathlineto{\pgfqpoint{4.390350in}{2.839907in}}%
\pgfpathlineto{\pgfqpoint{4.399822in}{2.842481in}}%
\pgfpathlineto{\pgfqpoint{4.428240in}{2.859493in}}%
\pgfpathlineto{\pgfqpoint{4.437712in}{2.866472in}}%
\pgfpathlineto{\pgfqpoint{4.447185in}{2.870118in}}%
\pgfpathlineto{\pgfqpoint{4.456657in}{2.876701in}}%
\pgfpathlineto{\pgfqpoint{4.466130in}{2.879079in}}%
\pgfpathlineto{\pgfqpoint{4.475602in}{2.887130in}}%
\pgfpathlineto{\pgfqpoint{4.485075in}{2.889215in}}%
\pgfpathlineto{\pgfqpoint{4.504020in}{2.896415in}}%
\pgfpathlineto{\pgfqpoint{4.513492in}{2.895656in}}%
\pgfpathlineto{\pgfqpoint{4.541910in}{2.906697in}}%
\pgfpathlineto{\pgfqpoint{4.560855in}{2.906555in}}%
\pgfpathlineto{\pgfqpoint{4.579800in}{2.915223in}}%
\pgfpathlineto{\pgfqpoint{4.589272in}{2.913490in}}%
\pgfpathlineto{\pgfqpoint{4.598745in}{2.918507in}}%
\pgfpathlineto{\pgfqpoint{4.608217in}{2.917944in}}%
\pgfpathlineto{\pgfqpoint{4.636635in}{2.931633in}}%
\pgfpathlineto{\pgfqpoint{4.655580in}{2.933644in}}%
\pgfpathlineto{\pgfqpoint{4.665052in}{2.932200in}}%
\pgfpathlineto{\pgfqpoint{4.674525in}{2.928995in}}%
\pgfpathlineto{\pgfqpoint{4.712415in}{2.948288in}}%
\pgfpathlineto{\pgfqpoint{4.731360in}{2.953138in}}%
\pgfpathlineto{\pgfqpoint{4.740832in}{2.958355in}}%
\pgfpathlineto{\pgfqpoint{4.750305in}{2.961512in}}%
\pgfpathlineto{\pgfqpoint{4.759777in}{2.969073in}}%
\pgfpathlineto{\pgfqpoint{4.769250in}{2.972332in}}%
\pgfpathlineto{\pgfqpoint{4.788195in}{2.974638in}}%
\pgfpathlineto{\pgfqpoint{4.797667in}{2.982003in}}%
\pgfpathlineto{\pgfqpoint{4.807140in}{2.987123in}}%
\pgfpathlineto{\pgfqpoint{4.816612in}{2.985777in}}%
\pgfpathlineto{\pgfqpoint{4.845030in}{3.004257in}}%
\pgfpathlineto{\pgfqpoint{4.854502in}{3.003210in}}%
\pgfpathlineto{\pgfqpoint{4.901865in}{3.017535in}}%
\pgfpathlineto{\pgfqpoint{4.911337in}{3.023334in}}%
\pgfpathlineto{\pgfqpoint{4.939755in}{3.025473in}}%
\pgfpathlineto{\pgfqpoint{4.958700in}{3.031199in}}%
\pgfpathlineto{\pgfqpoint{4.977645in}{3.033407in}}%
\pgfpathlineto{\pgfqpoint{4.987117in}{3.041169in}}%
\pgfpathlineto{\pgfqpoint{5.006062in}{3.043767in}}%
\pgfpathlineto{\pgfqpoint{5.053425in}{3.061127in}}%
\pgfpathlineto{\pgfqpoint{5.081842in}{3.071483in}}%
\pgfpathlineto{\pgfqpoint{5.091315in}{3.077092in}}%
\pgfpathlineto{\pgfqpoint{5.110260in}{3.082720in}}%
\pgfpathlineto{\pgfqpoint{5.119732in}{3.087350in}}%
\pgfpathlineto{\pgfqpoint{5.138677in}{3.102184in}}%
\pgfpathlineto{\pgfqpoint{5.157622in}{3.107621in}}%
\pgfpathlineto{\pgfqpoint{5.167094in}{3.108429in}}%
\pgfpathlineto{\pgfqpoint{5.176567in}{3.105321in}}%
\pgfpathlineto{\pgfqpoint{5.186039in}{3.110636in}}%
\pgfpathlineto{\pgfqpoint{5.195512in}{3.111150in}}%
\pgfpathlineto{\pgfqpoint{5.214457in}{3.116979in}}%
\pgfpathlineto{\pgfqpoint{5.223929in}{3.114561in}}%
\pgfpathlineto{\pgfqpoint{5.233402in}{3.115075in}}%
\pgfpathlineto{\pgfqpoint{5.242874in}{3.110401in}}%
\pgfpathlineto{\pgfqpoint{5.252347in}{3.111312in}}%
\pgfpathlineto{\pgfqpoint{5.261819in}{3.117013in}}%
\pgfpathlineto{\pgfqpoint{5.271292in}{3.119196in}}%
\pgfpathlineto{\pgfqpoint{5.280764in}{3.124212in}}%
\pgfpathlineto{\pgfqpoint{5.290237in}{3.130996in}}%
\pgfpathlineto{\pgfqpoint{5.309182in}{3.135650in}}%
\pgfpathlineto{\pgfqpoint{5.328127in}{3.141474in}}%
\pgfpathlineto{\pgfqpoint{5.337599in}{3.143168in}}%
\pgfpathlineto{\pgfqpoint{5.347072in}{3.148184in}}%
\pgfpathlineto{\pgfqpoint{5.366017in}{3.161256in}}%
\pgfpathlineto{\pgfqpoint{5.375489in}{3.164712in}}%
\pgfpathlineto{\pgfqpoint{5.384962in}{3.170903in}}%
\pgfpathlineto{\pgfqpoint{5.394434in}{3.171514in}}%
\pgfpathlineto{\pgfqpoint{5.403907in}{3.169390in}}%
\pgfpathlineto{\pgfqpoint{5.413379in}{3.173428in}}%
\pgfpathlineto{\pgfqpoint{5.432324in}{3.175048in}}%
\pgfpathlineto{\pgfqpoint{5.441797in}{3.174196in}}%
\pgfpathlineto{\pgfqpoint{5.451269in}{3.175493in}}%
\pgfpathlineto{\pgfqpoint{5.460742in}{3.174539in}}%
\pgfpathlineto{\pgfqpoint{5.470214in}{3.177113in}}%
\pgfpathlineto{\pgfqpoint{5.479687in}{3.184577in}}%
\pgfpathlineto{\pgfqpoint{5.489159in}{3.187738in}}%
\pgfpathlineto{\pgfqpoint{5.489159in}{3.187738in}}%
\pgfusepath{stroke}%
\end{pgfscope}%
\begin{pgfscope}%
\pgfpathrectangle{\pgfqpoint{0.762383in}{0.471179in}}{\pgfqpoint{4.726776in}{2.845920in}} %
\pgfusepath{clip}%
\pgfsetbuttcap%
\pgfsetroundjoin%
\definecolor{currentfill}{rgb}{0.580392,0.403922,0.741176}%
\pgfsetfillcolor{currentfill}%
\pgfsetlinewidth{1.003750pt}%
\definecolor{currentstroke}{rgb}{0.580392,0.403922,0.741176}%
\pgfsetstrokecolor{currentstroke}%
\pgfsetdash{}{0pt}%
\pgfsys@defobject{currentmarker}{\pgfqpoint{-0.020833in}{-0.020833in}}{\pgfqpoint{0.020833in}{0.020833in}}{%
\pgfpathmoveto{\pgfqpoint{0.000000in}{-0.020833in}}%
\pgfpathcurveto{\pgfqpoint{0.005525in}{-0.020833in}}{\pgfqpoint{0.010825in}{-0.018638in}}{\pgfqpoint{0.014731in}{-0.014731in}}%
\pgfpathcurveto{\pgfqpoint{0.018638in}{-0.010825in}}{\pgfqpoint{0.020833in}{-0.005525in}}{\pgfqpoint{0.020833in}{0.000000in}}%
\pgfpathcurveto{\pgfqpoint{0.020833in}{0.005525in}}{\pgfqpoint{0.018638in}{0.010825in}}{\pgfqpoint{0.014731in}{0.014731in}}%
\pgfpathcurveto{\pgfqpoint{0.010825in}{0.018638in}}{\pgfqpoint{0.005525in}{0.020833in}}{\pgfqpoint{0.000000in}{0.020833in}}%
\pgfpathcurveto{\pgfqpoint{-0.005525in}{0.020833in}}{\pgfqpoint{-0.010825in}{0.018638in}}{\pgfqpoint{-0.014731in}{0.014731in}}%
\pgfpathcurveto{\pgfqpoint{-0.018638in}{0.010825in}}{\pgfqpoint{-0.020833in}{0.005525in}}{\pgfqpoint{-0.020833in}{0.000000in}}%
\pgfpathcurveto{\pgfqpoint{-0.020833in}{-0.005525in}}{\pgfqpoint{-0.018638in}{-0.010825in}}{\pgfqpoint{-0.014731in}{-0.014731in}}%
\pgfpathcurveto{\pgfqpoint{-0.010825in}{-0.018638in}}{\pgfqpoint{-0.005525in}{-0.020833in}}{\pgfqpoint{0.000000in}{-0.020833in}}%
\pgfpathclose%
\pgfusepath{stroke,fill}%
}%
\begin{pgfscope}%
\pgfsys@transformshift{0.762383in}{0.600539in}%
\pgfsys@useobject{currentmarker}{}%
\end{pgfscope}%
\begin{pgfscope}%
\pgfsys@transformshift{0.771856in}{0.688562in}%
\pgfsys@useobject{currentmarker}{}%
\end{pgfscope}%
\begin{pgfscope}%
\pgfsys@transformshift{0.781328in}{0.724022in}%
\pgfsys@useobject{currentmarker}{}%
\end{pgfscope}%
\begin{pgfscope}%
\pgfsys@transformshift{0.790801in}{0.751945in}%
\pgfsys@useobject{currentmarker}{}%
\end{pgfscope}%
\begin{pgfscope}%
\pgfsys@transformshift{0.800273in}{0.776541in}%
\pgfsys@useobject{currentmarker}{}%
\end{pgfscope}%
\begin{pgfscope}%
\pgfsys@transformshift{0.809746in}{0.797025in}%
\pgfsys@useobject{currentmarker}{}%
\end{pgfscope}%
\begin{pgfscope}%
\pgfsys@transformshift{0.819218in}{0.819956in}%
\pgfsys@useobject{currentmarker}{}%
\end{pgfscope}%
\begin{pgfscope}%
\pgfsys@transformshift{0.828691in}{0.840538in}%
\pgfsys@useobject{currentmarker}{}%
\end{pgfscope}%
\begin{pgfscope}%
\pgfsys@transformshift{0.838163in}{0.865623in}%
\pgfsys@useobject{currentmarker}{}%
\end{pgfscope}%
\begin{pgfscope}%
\pgfsys@transformshift{0.847636in}{0.881506in}%
\pgfsys@useobject{currentmarker}{}%
\end{pgfscope}%
\begin{pgfscope}%
\pgfsys@transformshift{0.857108in}{0.897782in}%
\pgfsys@useobject{currentmarker}{}%
\end{pgfscope}%
\begin{pgfscope}%
\pgfsys@transformshift{0.866581in}{0.916112in}%
\pgfsys@useobject{currentmarker}{}%
\end{pgfscope}%
\begin{pgfscope}%
\pgfsys@transformshift{0.876053in}{0.936890in}%
\pgfsys@useobject{currentmarker}{}%
\end{pgfscope}%
\begin{pgfscope}%
\pgfsys@transformshift{0.885526in}{0.960898in}%
\pgfsys@useobject{currentmarker}{}%
\end{pgfscope}%
\begin{pgfscope}%
\pgfsys@transformshift{0.894998in}{0.976292in}%
\pgfsys@useobject{currentmarker}{}%
\end{pgfscope}%
\begin{pgfscope}%
\pgfsys@transformshift{0.904471in}{0.990022in}%
\pgfsys@useobject{currentmarker}{}%
\end{pgfscope}%
\begin{pgfscope}%
\pgfsys@transformshift{0.913943in}{1.002480in}%
\pgfsys@useobject{currentmarker}{}%
\end{pgfscope}%
\begin{pgfscope}%
\pgfsys@transformshift{0.923416in}{1.015232in}%
\pgfsys@useobject{currentmarker}{}%
\end{pgfscope}%
\begin{pgfscope}%
\pgfsys@transformshift{0.932888in}{1.027688in}%
\pgfsys@useobject{currentmarker}{}%
\end{pgfscope}%
\begin{pgfscope}%
\pgfsys@transformshift{0.942361in}{1.042008in}%
\pgfsys@useobject{currentmarker}{}%
\end{pgfscope}%
\begin{pgfscope}%
\pgfsys@transformshift{0.951833in}{1.057988in}%
\pgfsys@useobject{currentmarker}{}%
\end{pgfscope}%
\begin{pgfscope}%
\pgfsys@transformshift{0.961306in}{1.069469in}%
\pgfsys@useobject{currentmarker}{}%
\end{pgfscope}%
\begin{pgfscope}%
\pgfsys@transformshift{0.970778in}{1.083687in}%
\pgfsys@useobject{currentmarker}{}%
\end{pgfscope}%
\begin{pgfscope}%
\pgfsys@transformshift{0.980251in}{1.098394in}%
\pgfsys@useobject{currentmarker}{}%
\end{pgfscope}%
\begin{pgfscope}%
\pgfsys@transformshift{0.989723in}{1.112322in}%
\pgfsys@useobject{currentmarker}{}%
\end{pgfscope}%
\begin{pgfscope}%
\pgfsys@transformshift{0.999196in}{1.122233in}%
\pgfsys@useobject{currentmarker}{}%
\end{pgfscope}%
\begin{pgfscope}%
\pgfsys@transformshift{1.008668in}{1.132442in}%
\pgfsys@useobject{currentmarker}{}%
\end{pgfscope}%
\begin{pgfscope}%
\pgfsys@transformshift{1.018141in}{1.148128in}%
\pgfsys@useobject{currentmarker}{}%
\end{pgfscope}%
\begin{pgfscope}%
\pgfsys@transformshift{1.027613in}{1.154618in}%
\pgfsys@useobject{currentmarker}{}%
\end{pgfscope}%
\begin{pgfscope}%
\pgfsys@transformshift{1.037086in}{1.161004in}%
\pgfsys@useobject{currentmarker}{}%
\end{pgfscope}%
\begin{pgfscope}%
\pgfsys@transformshift{1.046558in}{1.167783in}%
\pgfsys@useobject{currentmarker}{}%
\end{pgfscope}%
\begin{pgfscope}%
\pgfsys@transformshift{1.056031in}{1.181222in}%
\pgfsys@useobject{currentmarker}{}%
\end{pgfscope}%
\begin{pgfscope}%
\pgfsys@transformshift{1.065503in}{1.195244in}%
\pgfsys@useobject{currentmarker}{}%
\end{pgfscope}%
\begin{pgfscope}%
\pgfsys@transformshift{1.074976in}{1.205943in}%
\pgfsys@useobject{currentmarker}{}%
\end{pgfscope}%
\begin{pgfscope}%
\pgfsys@transformshift{1.084448in}{1.218007in}%
\pgfsys@useobject{currentmarker}{}%
\end{pgfscope}%
\begin{pgfscope}%
\pgfsys@transformshift{1.093921in}{1.231446in}%
\pgfsys@useobject{currentmarker}{}%
\end{pgfscope}%
\begin{pgfscope}%
\pgfsys@transformshift{1.103393in}{1.240182in}%
\pgfsys@useobject{currentmarker}{}%
\end{pgfscope}%
\begin{pgfscope}%
\pgfsys@transformshift{1.112866in}{1.248331in}%
\pgfsys@useobject{currentmarker}{}%
\end{pgfscope}%
\begin{pgfscope}%
\pgfsys@transformshift{1.122338in}{1.254429in}%
\pgfsys@useobject{currentmarker}{}%
\end{pgfscope}%
\begin{pgfscope}%
\pgfsys@transformshift{1.131811in}{1.260033in}%
\pgfsys@useobject{currentmarker}{}%
\end{pgfscope}%
\begin{pgfscope}%
\pgfsys@transformshift{1.141283in}{1.270340in}%
\pgfsys@useobject{currentmarker}{}%
\end{pgfscope}%
\begin{pgfscope}%
\pgfsys@transformshift{1.150756in}{1.278684in}%
\pgfsys@useobject{currentmarker}{}%
\end{pgfscope}%
\begin{pgfscope}%
\pgfsys@transformshift{1.160228in}{1.292711in}%
\pgfsys@useobject{currentmarker}{}%
\end{pgfscope}%
\begin{pgfscope}%
\pgfsys@transformshift{1.169701in}{1.301545in}%
\pgfsys@useobject{currentmarker}{}%
\end{pgfscope}%
\begin{pgfscope}%
\pgfsys@transformshift{1.179173in}{1.310090in}%
\pgfsys@useobject{currentmarker}{}%
\end{pgfscope}%
\begin{pgfscope}%
\pgfsys@transformshift{1.188646in}{1.319707in}%
\pgfsys@useobject{currentmarker}{}%
\end{pgfscope}%
\begin{pgfscope}%
\pgfsys@transformshift{1.198118in}{1.324919in}%
\pgfsys@useobject{currentmarker}{}%
\end{pgfscope}%
\begin{pgfscope}%
\pgfsys@transformshift{1.207591in}{1.336401in}%
\pgfsys@useobject{currentmarker}{}%
\end{pgfscope}%
\begin{pgfscope}%
\pgfsys@transformshift{1.217063in}{1.345822in}%
\pgfsys@useobject{currentmarker}{}%
\end{pgfscope}%
\begin{pgfscope}%
\pgfsys@transformshift{1.226536in}{1.357793in}%
\pgfsys@useobject{currentmarker}{}%
\end{pgfscope}%
\begin{pgfscope}%
\pgfsys@transformshift{1.236008in}{1.370640in}%
\pgfsys@useobject{currentmarker}{}%
\end{pgfscope}%
\begin{pgfscope}%
\pgfsys@transformshift{1.245481in}{1.380849in}%
\pgfsys@useobject{currentmarker}{}%
\end{pgfscope}%
\begin{pgfscope}%
\pgfsys@transformshift{1.254953in}{1.385964in}%
\pgfsys@useobject{currentmarker}{}%
\end{pgfscope}%
\begin{pgfscope}%
\pgfsys@transformshift{1.264426in}{1.396853in}%
\pgfsys@useobject{currentmarker}{}%
\end{pgfscope}%
\begin{pgfscope}%
\pgfsys@transformshift{1.273898in}{1.409020in}%
\pgfsys@useobject{currentmarker}{}%
\end{pgfscope}%
\begin{pgfscope}%
\pgfsys@transformshift{1.283371in}{1.419224in}%
\pgfsys@useobject{currentmarker}{}%
\end{pgfscope}%
\begin{pgfscope}%
\pgfsys@transformshift{1.292843in}{1.427770in}%
\pgfsys@useobject{currentmarker}{}%
\end{pgfscope}%
\begin{pgfscope}%
\pgfsys@transformshift{1.302316in}{1.435331in}%
\pgfsys@useobject{currentmarker}{}%
\end{pgfscope}%
\begin{pgfscope}%
\pgfsys@transformshift{1.311788in}{1.446128in}%
\pgfsys@useobject{currentmarker}{}%
\end{pgfscope}%
\begin{pgfscope}%
\pgfsys@transformshift{1.321261in}{1.451438in}%
\pgfsys@useobject{currentmarker}{}%
\end{pgfscope}%
\begin{pgfscope}%
\pgfsys@transformshift{1.330733in}{1.455769in}%
\pgfsys@useobject{currentmarker}{}%
\end{pgfscope}%
\begin{pgfscope}%
\pgfsys@transformshift{1.340206in}{1.465195in}%
\pgfsys@useobject{currentmarker}{}%
\end{pgfscope}%
\begin{pgfscope}%
\pgfsys@transformshift{1.349678in}{1.472855in}%
\pgfsys@useobject{currentmarker}{}%
\end{pgfscope}%
\begin{pgfscope}%
\pgfsys@transformshift{1.359151in}{1.480323in}%
\pgfsys@useobject{currentmarker}{}%
\end{pgfscope}%
\begin{pgfscope}%
\pgfsys@transformshift{1.368623in}{1.485829in}%
\pgfsys@useobject{currentmarker}{}%
\end{pgfscope}%
\begin{pgfscope}%
\pgfsys@transformshift{1.378096in}{1.493885in}%
\pgfsys@useobject{currentmarker}{}%
\end{pgfscope}%
\begin{pgfscope}%
\pgfsys@transformshift{1.387568in}{1.501740in}%
\pgfsys@useobject{currentmarker}{}%
\end{pgfscope}%
\begin{pgfscope}%
\pgfsys@transformshift{1.397041in}{1.507931in}%
\pgfsys@useobject{currentmarker}{}%
\end{pgfscope}%
\begin{pgfscope}%
\pgfsys@transformshift{1.406513in}{1.517161in}%
\pgfsys@useobject{currentmarker}{}%
\end{pgfscope}%
\begin{pgfscope}%
\pgfsys@transformshift{1.415986in}{1.525310in}%
\pgfsys@useobject{currentmarker}{}%
\end{pgfscope}%
\begin{pgfscope}%
\pgfsys@transformshift{1.425458in}{1.530331in}%
\pgfsys@useobject{currentmarker}{}%
\end{pgfscope}%
\begin{pgfscope}%
\pgfsys@transformshift{1.434931in}{1.536327in}%
\pgfsys@useobject{currentmarker}{}%
\end{pgfscope}%
\begin{pgfscope}%
\pgfsys@transformshift{1.444403in}{1.540565in}%
\pgfsys@useobject{currentmarker}{}%
\end{pgfscope}%
\begin{pgfscope}%
\pgfsys@transformshift{1.453876in}{1.543330in}%
\pgfsys@useobject{currentmarker}{}%
\end{pgfscope}%
\begin{pgfscope}%
\pgfsys@transformshift{1.463348in}{1.549918in}%
\pgfsys@useobject{currentmarker}{}%
\end{pgfscope}%
\begin{pgfscope}%
\pgfsys@transformshift{1.472821in}{1.561395in}%
\pgfsys@useobject{currentmarker}{}%
\end{pgfscope}%
\begin{pgfscope}%
\pgfsys@transformshift{1.482293in}{1.568173in}%
\pgfsys@useobject{currentmarker}{}%
\end{pgfscope}%
\begin{pgfscope}%
\pgfsys@transformshift{1.491766in}{1.578284in}%
\pgfsys@useobject{currentmarker}{}%
\end{pgfscope}%
\begin{pgfscope}%
\pgfsys@transformshift{1.501238in}{1.588684in}%
\pgfsys@useobject{currentmarker}{}%
\end{pgfscope}%
\begin{pgfscope}%
\pgfsys@transformshift{1.510711in}{1.600557in}%
\pgfsys@useobject{currentmarker}{}%
\end{pgfscope}%
\begin{pgfscope}%
\pgfsys@transformshift{1.520183in}{1.605280in}%
\pgfsys@useobject{currentmarker}{}%
\end{pgfscope}%
\begin{pgfscope}%
\pgfsys@transformshift{1.529656in}{1.613825in}%
\pgfsys@useobject{currentmarker}{}%
\end{pgfscope}%
\begin{pgfscope}%
\pgfsys@transformshift{1.539128in}{1.624030in}%
\pgfsys@useobject{currentmarker}{}%
\end{pgfscope}%
\begin{pgfscope}%
\pgfsys@transformshift{1.548601in}{1.634038in}%
\pgfsys@useobject{currentmarker}{}%
\end{pgfscope}%
\begin{pgfscope}%
\pgfsys@transformshift{1.558073in}{1.641409in}%
\pgfsys@useobject{currentmarker}{}%
\end{pgfscope}%
\begin{pgfscope}%
\pgfsys@transformshift{1.567546in}{1.648872in}%
\pgfsys@useobject{currentmarker}{}%
\end{pgfscope}%
\begin{pgfscope}%
\pgfsys@transformshift{1.577018in}{1.656439in}%
\pgfsys@useobject{currentmarker}{}%
\end{pgfscope}%
\begin{pgfscope}%
\pgfsys@transformshift{1.586491in}{1.663119in}%
\pgfsys@useobject{currentmarker}{}%
\end{pgfscope}%
\begin{pgfscope}%
\pgfsys@transformshift{1.595963in}{1.670000in}%
\pgfsys@useobject{currentmarker}{}%
\end{pgfscope}%
\begin{pgfscope}%
\pgfsys@transformshift{1.605436in}{1.670123in}%
\pgfsys@useobject{currentmarker}{}%
\end{pgfscope}%
\begin{pgfscope}%
\pgfsys@transformshift{1.614908in}{1.677782in}%
\pgfsys@useobject{currentmarker}{}%
\end{pgfscope}%
\begin{pgfscope}%
\pgfsys@transformshift{1.624381in}{1.683489in}%
\pgfsys@useobject{currentmarker}{}%
\end{pgfscope}%
\begin{pgfscope}%
\pgfsys@transformshift{1.633853in}{1.690169in}%
\pgfsys@useobject{currentmarker}{}%
\end{pgfscope}%
\begin{pgfscope}%
\pgfsys@transformshift{1.643325in}{1.697246in}%
\pgfsys@useobject{currentmarker}{}%
\end{pgfscope}%
\begin{pgfscope}%
\pgfsys@transformshift{1.652798in}{1.707059in}%
\pgfsys@useobject{currentmarker}{}%
\end{pgfscope}%
\begin{pgfscope}%
\pgfsys@transformshift{1.662270in}{1.709633in}%
\pgfsys@useobject{currentmarker}{}%
\end{pgfscope}%
\begin{pgfscope}%
\pgfsys@transformshift{1.671743in}{1.717586in}%
\pgfsys@useobject{currentmarker}{}%
\end{pgfscope}%
\begin{pgfscope}%
\pgfsys@transformshift{1.681215in}{1.722505in}%
\pgfsys@useobject{currentmarker}{}%
\end{pgfscope}%
\begin{pgfscope}%
\pgfsys@transformshift{1.690688in}{1.731442in}%
\pgfsys@useobject{currentmarker}{}%
\end{pgfscope}%
\begin{pgfscope}%
\pgfsys@transformshift{1.700160in}{1.738611in}%
\pgfsys@useobject{currentmarker}{}%
\end{pgfscope}%
\begin{pgfscope}%
\pgfsys@transformshift{1.709633in}{1.742556in}%
\pgfsys@useobject{currentmarker}{}%
\end{pgfscope}%
\begin{pgfscope}%
\pgfsys@transformshift{1.719105in}{1.746692in}%
\pgfsys@useobject{currentmarker}{}%
\end{pgfscope}%
\begin{pgfscope}%
\pgfsys@transformshift{1.728578in}{1.753964in}%
\pgfsys@useobject{currentmarker}{}%
\end{pgfscope}%
\begin{pgfscope}%
\pgfsys@transformshift{1.738050in}{1.762211in}%
\pgfsys@useobject{currentmarker}{}%
\end{pgfscope}%
\begin{pgfscope}%
\pgfsys@transformshift{1.747523in}{1.771148in}%
\pgfsys@useobject{currentmarker}{}%
\end{pgfscope}%
\begin{pgfscope}%
\pgfsys@transformshift{1.756995in}{1.776458in}%
\pgfsys@useobject{currentmarker}{}%
\end{pgfscope}%
\begin{pgfscope}%
\pgfsys@transformshift{1.766468in}{1.783041in}%
\pgfsys@useobject{currentmarker}{}%
\end{pgfscope}%
\begin{pgfscope}%
\pgfsys@transformshift{1.775940in}{1.789922in}%
\pgfsys@useobject{currentmarker}{}%
\end{pgfscope}%
\begin{pgfscope}%
\pgfsys@transformshift{1.785413in}{1.792100in}%
\pgfsys@useobject{currentmarker}{}%
\end{pgfscope}%
\begin{pgfscope}%
\pgfsys@transformshift{1.794885in}{1.794968in}%
\pgfsys@useobject{currentmarker}{}%
\end{pgfscope}%
\begin{pgfscope}%
\pgfsys@transformshift{1.804358in}{1.803116in}%
\pgfsys@useobject{currentmarker}{}%
\end{pgfscope}%
\begin{pgfscope}%
\pgfsys@transformshift{1.813830in}{1.807746in}%
\pgfsys@useobject{currentmarker}{}%
\end{pgfscope}%
\begin{pgfscope}%
\pgfsys@transformshift{1.823303in}{1.814133in}%
\pgfsys@useobject{currentmarker}{}%
\end{pgfscope}%
\begin{pgfscope}%
\pgfsys@transformshift{1.832775in}{1.818562in}%
\pgfsys@useobject{currentmarker}{}%
\end{pgfscope}%
\begin{pgfscope}%
\pgfsys@transformshift{1.842248in}{1.824465in}%
\pgfsys@useobject{currentmarker}{}%
\end{pgfscope}%
\begin{pgfscope}%
\pgfsys@transformshift{1.851720in}{1.834082in}%
\pgfsys@useobject{currentmarker}{}%
\end{pgfscope}%
\begin{pgfscope}%
\pgfsys@transformshift{1.861193in}{1.834111in}%
\pgfsys@useobject{currentmarker}{}%
\end{pgfscope}%
\begin{pgfscope}%
\pgfsys@transformshift{1.870665in}{1.839519in}%
\pgfsys@useobject{currentmarker}{}%
\end{pgfscope}%
\begin{pgfscope}%
\pgfsys@transformshift{1.880138in}{1.845421in}%
\pgfsys@useobject{currentmarker}{}%
\end{pgfscope}%
\begin{pgfscope}%
\pgfsys@transformshift{1.889610in}{1.852298in}%
\pgfsys@useobject{currentmarker}{}%
\end{pgfscope}%
\begin{pgfscope}%
\pgfsys@transformshift{1.899083in}{1.854280in}%
\pgfsys@useobject{currentmarker}{}%
\end{pgfscope}%
\begin{pgfscope}%
\pgfsys@transformshift{1.908555in}{1.861063in}%
\pgfsys@useobject{currentmarker}{}%
\end{pgfscope}%
\begin{pgfscope}%
\pgfsys@transformshift{1.918028in}{1.867646in}%
\pgfsys@useobject{currentmarker}{}%
\end{pgfscope}%
\begin{pgfscope}%
\pgfsys@transformshift{1.927500in}{1.868556in}%
\pgfsys@useobject{currentmarker}{}%
\end{pgfscope}%
\begin{pgfscope}%
\pgfsys@transformshift{1.936973in}{1.870538in}%
\pgfsys@useobject{currentmarker}{}%
\end{pgfscope}%
\begin{pgfscope}%
\pgfsys@transformshift{1.946445in}{1.876636in}%
\pgfsys@useobject{currentmarker}{}%
\end{pgfscope}%
\begin{pgfscope}%
\pgfsys@transformshift{1.955918in}{1.881946in}%
\pgfsys@useobject{currentmarker}{}%
\end{pgfscope}%
\begin{pgfscope}%
\pgfsys@transformshift{1.965390in}{1.893717in}%
\pgfsys@useobject{currentmarker}{}%
\end{pgfscope}%
\begin{pgfscope}%
\pgfsys@transformshift{1.974863in}{1.898640in}%
\pgfsys@useobject{currentmarker}{}%
\end{pgfscope}%
\begin{pgfscope}%
\pgfsys@transformshift{1.984335in}{1.901895in}%
\pgfsys@useobject{currentmarker}{}%
\end{pgfscope}%
\begin{pgfscope}%
\pgfsys@transformshift{1.993808in}{1.907014in}%
\pgfsys@useobject{currentmarker}{}%
\end{pgfscope}%
\begin{pgfscope}%
\pgfsys@transformshift{2.003280in}{1.910954in}%
\pgfsys@useobject{currentmarker}{}%
\end{pgfscope}%
\begin{pgfscope}%
\pgfsys@transformshift{2.012753in}{1.917737in}%
\pgfsys@useobject{currentmarker}{}%
\end{pgfscope}%
\begin{pgfscope}%
\pgfsys@transformshift{2.022225in}{1.926669in}%
\pgfsys@useobject{currentmarker}{}%
\end{pgfscope}%
\begin{pgfscope}%
\pgfsys@transformshift{2.031698in}{1.932865in}%
\pgfsys@useobject{currentmarker}{}%
\end{pgfscope}%
\begin{pgfscope}%
\pgfsys@transformshift{2.041170in}{1.940916in}%
\pgfsys@useobject{currentmarker}{}%
\end{pgfscope}%
\begin{pgfscope}%
\pgfsys@transformshift{2.050643in}{1.949456in}%
\pgfsys@useobject{currentmarker}{}%
\end{pgfscope}%
\begin{pgfscope}%
\pgfsys@transformshift{2.060115in}{1.952520in}%
\pgfsys@useobject{currentmarker}{}%
\end{pgfscope}%
\begin{pgfscope}%
\pgfsys@transformshift{2.069588in}{1.954796in}%
\pgfsys@useobject{currentmarker}{}%
\end{pgfscope}%
\begin{pgfscope}%
\pgfsys@transformshift{2.079060in}{1.962460in}%
\pgfsys@useobject{currentmarker}{}%
\end{pgfscope}%
\begin{pgfscope}%
\pgfsys@transformshift{2.088533in}{1.966791in}%
\pgfsys@useobject{currentmarker}{}%
\end{pgfscope}%
\begin{pgfscope}%
\pgfsys@transformshift{2.098005in}{1.970442in}%
\pgfsys@useobject{currentmarker}{}%
\end{pgfscope}%
\begin{pgfscope}%
\pgfsys@transformshift{2.107478in}{1.975067in}%
\pgfsys@useobject{currentmarker}{}%
\end{pgfscope}%
\begin{pgfscope}%
\pgfsys@transformshift{2.116950in}{1.973819in}%
\pgfsys@useobject{currentmarker}{}%
\end{pgfscope}%
\begin{pgfscope}%
\pgfsys@transformshift{2.126423in}{1.979232in}%
\pgfsys@useobject{currentmarker}{}%
\end{pgfscope}%
\begin{pgfscope}%
\pgfsys@transformshift{2.135895in}{1.982976in}%
\pgfsys@useobject{currentmarker}{}%
\end{pgfscope}%
\begin{pgfscope}%
\pgfsys@transformshift{2.145368in}{1.987704in}%
\pgfsys@useobject{currentmarker}{}%
\end{pgfscope}%
\begin{pgfscope}%
\pgfsys@transformshift{2.154840in}{1.995657in}%
\pgfsys@useobject{currentmarker}{}%
\end{pgfscope}%
\begin{pgfscope}%
\pgfsys@transformshift{2.164313in}{2.000091in}%
\pgfsys@useobject{currentmarker}{}%
\end{pgfscope}%
\begin{pgfscope}%
\pgfsys@transformshift{2.173785in}{2.003150in}%
\pgfsys@useobject{currentmarker}{}%
\end{pgfscope}%
\begin{pgfscope}%
\pgfsys@transformshift{2.183258in}{2.006894in}%
\pgfsys@useobject{currentmarker}{}%
\end{pgfscope}%
\begin{pgfscope}%
\pgfsys@transformshift{2.192730in}{2.018082in}%
\pgfsys@useobject{currentmarker}{}%
\end{pgfscope}%
\begin{pgfscope}%
\pgfsys@transformshift{2.202203in}{2.026328in}%
\pgfsys@useobject{currentmarker}{}%
\end{pgfscope}%
\begin{pgfscope}%
\pgfsys@transformshift{2.211675in}{2.033210in}%
\pgfsys@useobject{currentmarker}{}%
\end{pgfscope}%
\begin{pgfscope}%
\pgfsys@transformshift{2.221148in}{2.040869in}%
\pgfsys@useobject{currentmarker}{}%
\end{pgfscope}%
\begin{pgfscope}%
\pgfsys@transformshift{2.230620in}{2.040507in}%
\pgfsys@useobject{currentmarker}{}%
\end{pgfscope}%
\begin{pgfscope}%
\pgfsys@transformshift{2.240093in}{2.044838in}%
\pgfsys@useobject{currentmarker}{}%
\end{pgfscope}%
\begin{pgfscope}%
\pgfsys@transformshift{2.249565in}{2.054455in}%
\pgfsys@useobject{currentmarker}{}%
\end{pgfscope}%
\begin{pgfscope}%
\pgfsys@transformshift{2.259038in}{2.061532in}%
\pgfsys@useobject{currentmarker}{}%
\end{pgfscope}%
\begin{pgfscope}%
\pgfsys@transformshift{2.268510in}{2.068408in}%
\pgfsys@useobject{currentmarker}{}%
\end{pgfscope}%
\begin{pgfscope}%
\pgfsys@transformshift{2.277983in}{2.076856in}%
\pgfsys@useobject{currentmarker}{}%
\end{pgfscope}%
\begin{pgfscope}%
\pgfsys@transformshift{2.287455in}{2.082655in}%
\pgfsys@useobject{currentmarker}{}%
\end{pgfscope}%
\begin{pgfscope}%
\pgfsys@transformshift{2.296928in}{2.084153in}%
\pgfsys@useobject{currentmarker}{}%
\end{pgfscope}%
\begin{pgfscope}%
\pgfsys@transformshift{2.306400in}{2.084471in}%
\pgfsys@useobject{currentmarker}{}%
\end{pgfscope}%
\begin{pgfscope}%
\pgfsys@transformshift{2.315873in}{2.083228in}%
\pgfsys@useobject{currentmarker}{}%
\end{pgfscope}%
\begin{pgfscope}%
\pgfsys@transformshift{2.325345in}{2.087951in}%
\pgfsys@useobject{currentmarker}{}%
\end{pgfscope}%
\begin{pgfscope}%
\pgfsys@transformshift{2.334818in}{2.090912in}%
\pgfsys@useobject{currentmarker}{}%
\end{pgfscope}%
\begin{pgfscope}%
\pgfsys@transformshift{2.344290in}{2.096422in}%
\pgfsys@useobject{currentmarker}{}%
\end{pgfscope}%
\begin{pgfscope}%
\pgfsys@transformshift{2.353763in}{2.104865in}%
\pgfsys@useobject{currentmarker}{}%
\end{pgfscope}%
\begin{pgfscope}%
\pgfsys@transformshift{2.363235in}{2.112040in}%
\pgfsys@useobject{currentmarker}{}%
\end{pgfscope}%
\begin{pgfscope}%
\pgfsys@transformshift{2.372708in}{2.112456in}%
\pgfsys@useobject{currentmarker}{}%
\end{pgfscope}%
\begin{pgfscope}%
\pgfsys@transformshift{2.382180in}{2.113855in}%
\pgfsys@useobject{currentmarker}{}%
\end{pgfscope}%
\begin{pgfscope}%
\pgfsys@transformshift{2.391653in}{2.119068in}%
\pgfsys@useobject{currentmarker}{}%
\end{pgfscope}%
\begin{pgfscope}%
\pgfsys@transformshift{2.401125in}{2.127412in}%
\pgfsys@useobject{currentmarker}{}%
\end{pgfscope}%
\begin{pgfscope}%
\pgfsys@transformshift{2.410598in}{2.130770in}%
\pgfsys@useobject{currentmarker}{}%
\end{pgfscope}%
\begin{pgfscope}%
\pgfsys@transformshift{2.420070in}{2.133828in}%
\pgfsys@useobject{currentmarker}{}%
\end{pgfscope}%
\begin{pgfscope}%
\pgfsys@transformshift{2.429543in}{2.142471in}%
\pgfsys@useobject{currentmarker}{}%
\end{pgfscope}%
\begin{pgfscope}%
\pgfsys@transformshift{2.439015in}{2.144258in}%
\pgfsys@useobject{currentmarker}{}%
\end{pgfscope}%
\begin{pgfscope}%
\pgfsys@transformshift{2.448488in}{2.145755in}%
\pgfsys@useobject{currentmarker}{}%
\end{pgfscope}%
\begin{pgfscope}%
\pgfsys@transformshift{2.457960in}{2.149597in}%
\pgfsys@useobject{currentmarker}{}%
\end{pgfscope}%
\begin{pgfscope}%
\pgfsys@transformshift{2.467433in}{2.151775in}%
\pgfsys@useobject{currentmarker}{}%
\end{pgfscope}%
\begin{pgfscope}%
\pgfsys@transformshift{2.476905in}{2.152490in}%
\pgfsys@useobject{currentmarker}{}%
\end{pgfscope}%
\begin{pgfscope}%
\pgfsys@transformshift{2.486378in}{2.156234in}%
\pgfsys@useobject{currentmarker}{}%
\end{pgfscope}%
\begin{pgfscope}%
\pgfsys@transformshift{2.495850in}{2.156948in}%
\pgfsys@useobject{currentmarker}{}%
\end{pgfscope}%
\begin{pgfscope}%
\pgfsys@transformshift{2.505323in}{2.160594in}%
\pgfsys@useobject{currentmarker}{}%
\end{pgfscope}%
\begin{pgfscope}%
\pgfsys@transformshift{2.514795in}{2.165910in}%
\pgfsys@useobject{currentmarker}{}%
\end{pgfscope}%
\begin{pgfscope}%
\pgfsys@transformshift{2.524268in}{2.170339in}%
\pgfsys@useobject{currentmarker}{}%
\end{pgfscope}%
\begin{pgfscope}%
\pgfsys@transformshift{2.533740in}{2.173789in}%
\pgfsys@useobject{currentmarker}{}%
\end{pgfscope}%
\begin{pgfscope}%
\pgfsys@transformshift{2.543213in}{2.179202in}%
\pgfsys@useobject{currentmarker}{}%
\end{pgfscope}%
\begin{pgfscope}%
\pgfsys@transformshift{2.552685in}{2.178835in}%
\pgfsys@useobject{currentmarker}{}%
\end{pgfscope}%
\begin{pgfscope}%
\pgfsys@transformshift{2.562158in}{2.183954in}%
\pgfsys@useobject{currentmarker}{}%
\end{pgfscope}%
\begin{pgfscope}%
\pgfsys@transformshift{2.571630in}{2.190439in}%
\pgfsys@useobject{currentmarker}{}%
\end{pgfscope}%
\begin{pgfscope}%
\pgfsys@transformshift{2.581103in}{2.194775in}%
\pgfsys@useobject{currentmarker}{}%
\end{pgfscope}%
\begin{pgfscope}%
\pgfsys@transformshift{2.590575in}{2.205371in}%
\pgfsys@useobject{currentmarker}{}%
\end{pgfscope}%
\begin{pgfscope}%
\pgfsys@transformshift{2.600048in}{2.206477in}%
\pgfsys@useobject{currentmarker}{}%
\end{pgfscope}%
\begin{pgfscope}%
\pgfsys@transformshift{2.609520in}{2.210025in}%
\pgfsys@useobject{currentmarker}{}%
\end{pgfscope}%
\begin{pgfscope}%
\pgfsys@transformshift{2.618993in}{2.214259in}%
\pgfsys@useobject{currentmarker}{}%
\end{pgfscope}%
\begin{pgfscope}%
\pgfsys@transformshift{2.628465in}{2.217420in}%
\pgfsys@useobject{currentmarker}{}%
\end{pgfscope}%
\begin{pgfscope}%
\pgfsys@transformshift{2.637938in}{2.219990in}%
\pgfsys@useobject{currentmarker}{}%
\end{pgfscope}%
\begin{pgfscope}%
\pgfsys@transformshift{2.647410in}{2.225403in}%
\pgfsys@useobject{currentmarker}{}%
\end{pgfscope}%
\begin{pgfscope}%
\pgfsys@transformshift{2.656883in}{2.225623in}%
\pgfsys@useobject{currentmarker}{}%
\end{pgfscope}%
\begin{pgfscope}%
\pgfsys@transformshift{2.666355in}{2.227414in}%
\pgfsys@useobject{currentmarker}{}%
\end{pgfscope}%
\begin{pgfscope}%
\pgfsys@transformshift{2.675828in}{2.225579in}%
\pgfsys@useobject{currentmarker}{}%
\end{pgfscope}%
\begin{pgfscope}%
\pgfsys@transformshift{2.685300in}{2.234413in}%
\pgfsys@useobject{currentmarker}{}%
\end{pgfscope}%
\begin{pgfscope}%
\pgfsys@transformshift{2.694773in}{2.242273in}%
\pgfsys@useobject{currentmarker}{}%
\end{pgfscope}%
\begin{pgfscope}%
\pgfsys@transformshift{2.704245in}{2.247877in}%
\pgfsys@useobject{currentmarker}{}%
\end{pgfscope}%
\begin{pgfscope}%
\pgfsys@transformshift{2.713718in}{2.246340in}%
\pgfsys@useobject{currentmarker}{}%
\end{pgfscope}%
\begin{pgfscope}%
\pgfsys@transformshift{2.723190in}{2.252727in}%
\pgfsys@useobject{currentmarker}{}%
\end{pgfscope}%
\begin{pgfscope}%
\pgfsys@transformshift{2.732663in}{2.256476in}%
\pgfsys@useobject{currentmarker}{}%
\end{pgfscope}%
\begin{pgfscope}%
\pgfsys@transformshift{2.742135in}{2.260905in}%
\pgfsys@useobject{currentmarker}{}%
\end{pgfscope}%
\begin{pgfscope}%
\pgfsys@transformshift{2.751608in}{2.264551in}%
\pgfsys@useobject{currentmarker}{}%
\end{pgfscope}%
\begin{pgfscope}%
\pgfsys@transformshift{2.761080in}{2.272509in}%
\pgfsys@useobject{currentmarker}{}%
\end{pgfscope}%
\begin{pgfscope}%
\pgfsys@transformshift{2.770553in}{2.274393in}%
\pgfsys@useobject{currentmarker}{}%
\end{pgfscope}%
\begin{pgfscope}%
\pgfsys@transformshift{2.780025in}{2.280883in}%
\pgfsys@useobject{currentmarker}{}%
\end{pgfscope}%
\begin{pgfscope}%
\pgfsys@transformshift{2.789498in}{2.298233in}%
\pgfsys@useobject{currentmarker}{}%
\end{pgfscope}%
\begin{pgfscope}%
\pgfsys@transformshift{2.798970in}{2.304429in}%
\pgfsys@useobject{currentmarker}{}%
\end{pgfscope}%
\begin{pgfscope}%
\pgfsys@transformshift{2.808443in}{2.308858in}%
\pgfsys@useobject{currentmarker}{}%
\end{pgfscope}%
\begin{pgfscope}%
\pgfsys@transformshift{2.817915in}{2.315837in}%
\pgfsys@useobject{currentmarker}{}%
\end{pgfscope}%
\begin{pgfscope}%
\pgfsys@transformshift{2.827388in}{2.318700in}%
\pgfsys@useobject{currentmarker}{}%
\end{pgfscope}%
\begin{pgfscope}%
\pgfsys@transformshift{2.836860in}{2.320584in}%
\pgfsys@useobject{currentmarker}{}%
\end{pgfscope}%
\begin{pgfscope}%
\pgfsys@transformshift{2.846333in}{2.322865in}%
\pgfsys@useobject{currentmarker}{}%
\end{pgfscope}%
\begin{pgfscope}%
\pgfsys@transformshift{2.855805in}{2.321715in}%
\pgfsys@useobject{currentmarker}{}%
\end{pgfscope}%
\begin{pgfscope}%
\pgfsys@transformshift{2.865278in}{2.328008in}%
\pgfsys@useobject{currentmarker}{}%
\end{pgfscope}%
\begin{pgfscope}%
\pgfsys@transformshift{2.874750in}{2.331361in}%
\pgfsys@useobject{currentmarker}{}%
\end{pgfscope}%
\begin{pgfscope}%
\pgfsys@transformshift{2.884223in}{2.331782in}%
\pgfsys@useobject{currentmarker}{}%
\end{pgfscope}%
\begin{pgfscope}%
\pgfsys@transformshift{2.893695in}{2.333862in}%
\pgfsys@useobject{currentmarker}{}%
\end{pgfscope}%
\begin{pgfscope}%
\pgfsys@transformshift{2.903168in}{2.335746in}%
\pgfsys@useobject{currentmarker}{}%
\end{pgfscope}%
\begin{pgfscope}%
\pgfsys@transformshift{2.912640in}{2.333720in}%
\pgfsys@useobject{currentmarker}{}%
\end{pgfscope}%
\begin{pgfscope}%
\pgfsys@transformshift{2.922113in}{2.338051in}%
\pgfsys@useobject{currentmarker}{}%
\end{pgfscope}%
\begin{pgfscope}%
\pgfsys@transformshift{2.931585in}{2.345911in}%
\pgfsys@useobject{currentmarker}{}%
\end{pgfscope}%
\begin{pgfscope}%
\pgfsys@transformshift{2.941058in}{2.354647in}%
\pgfsys@useobject{currentmarker}{}%
\end{pgfscope}%
\begin{pgfscope}%
\pgfsys@transformshift{2.950530in}{2.358396in}%
\pgfsys@useobject{currentmarker}{}%
\end{pgfscope}%
\begin{pgfscope}%
\pgfsys@transformshift{2.960003in}{2.362336in}%
\pgfsys@useobject{currentmarker}{}%
\end{pgfscope}%
\begin{pgfscope}%
\pgfsys@transformshift{2.969475in}{2.369800in}%
\pgfsys@useobject{currentmarker}{}%
\end{pgfscope}%
\begin{pgfscope}%
\pgfsys@transformshift{2.978948in}{2.377660in}%
\pgfsys@useobject{currentmarker}{}%
\end{pgfscope}%
\begin{pgfscope}%
\pgfsys@transformshift{2.988420in}{2.386102in}%
\pgfsys@useobject{currentmarker}{}%
\end{pgfscope}%
\begin{pgfscope}%
\pgfsys@transformshift{2.997893in}{2.388970in}%
\pgfsys@useobject{currentmarker}{}%
\end{pgfscope}%
\begin{pgfscope}%
\pgfsys@transformshift{3.007365in}{2.391735in}%
\pgfsys@useobject{currentmarker}{}%
\end{pgfscope}%
\begin{pgfscope}%
\pgfsys@transformshift{3.016838in}{2.396071in}%
\pgfsys@useobject{currentmarker}{}%
\end{pgfscope}%
\begin{pgfscope}%
\pgfsys@transformshift{3.026310in}{2.393942in}%
\pgfsys@useobject{currentmarker}{}%
\end{pgfscope}%
\begin{pgfscope}%
\pgfsys@transformshift{3.035783in}{2.399742in}%
\pgfsys@useobject{currentmarker}{}%
\end{pgfscope}%
\begin{pgfscope}%
\pgfsys@transformshift{3.045255in}{2.406329in}%
\pgfsys@useobject{currentmarker}{}%
\end{pgfscope}%
\begin{pgfscope}%
\pgfsys@transformshift{3.054728in}{2.408410in}%
\pgfsys@useobject{currentmarker}{}%
\end{pgfscope}%
\begin{pgfscope}%
\pgfsys@transformshift{3.064200in}{2.407852in}%
\pgfsys@useobject{currentmarker}{}%
\end{pgfscope}%
\begin{pgfscope}%
\pgfsys@transformshift{3.073673in}{2.406408in}%
\pgfsys@useobject{currentmarker}{}%
\end{pgfscope}%
\begin{pgfscope}%
\pgfsys@transformshift{3.083145in}{2.416519in}%
\pgfsys@useobject{currentmarker}{}%
\end{pgfscope}%
\begin{pgfscope}%
\pgfsys@transformshift{3.092618in}{2.418305in}%
\pgfsys@useobject{currentmarker}{}%
\end{pgfscope}%
\begin{pgfscope}%
\pgfsys@transformshift{3.102090in}{2.419412in}%
\pgfsys@useobject{currentmarker}{}%
\end{pgfscope}%
\begin{pgfscope}%
\pgfsys@transformshift{3.111563in}{2.425211in}%
\pgfsys@useobject{currentmarker}{}%
\end{pgfscope}%
\begin{pgfscope}%
\pgfsys@transformshift{3.121035in}{2.428074in}%
\pgfsys@useobject{currentmarker}{}%
\end{pgfscope}%
\begin{pgfscope}%
\pgfsys@transformshift{3.130508in}{2.433781in}%
\pgfsys@useobject{currentmarker}{}%
\end{pgfscope}%
\begin{pgfscope}%
\pgfsys@transformshift{3.139980in}{2.440657in}%
\pgfsys@useobject{currentmarker}{}%
\end{pgfscope}%
\begin{pgfscope}%
\pgfsys@transformshift{3.149453in}{2.440295in}%
\pgfsys@useobject{currentmarker}{}%
\end{pgfscope}%
\begin{pgfscope}%
\pgfsys@transformshift{3.158925in}{2.444920in}%
\pgfsys@useobject{currentmarker}{}%
\end{pgfscope}%
\begin{pgfscope}%
\pgfsys@transformshift{3.168398in}{2.455227in}%
\pgfsys@useobject{currentmarker}{}%
\end{pgfscope}%
\begin{pgfscope}%
\pgfsys@transformshift{3.177870in}{2.457992in}%
\pgfsys@useobject{currentmarker}{}%
\end{pgfscope}%
\begin{pgfscope}%
\pgfsys@transformshift{3.187343in}{2.463204in}%
\pgfsys@useobject{currentmarker}{}%
\end{pgfscope}%
\begin{pgfscope}%
\pgfsys@transformshift{3.196815in}{2.469498in}%
\pgfsys@useobject{currentmarker}{}%
\end{pgfscope}%
\begin{pgfscope}%
\pgfsys@transformshift{3.206288in}{2.471676in}%
\pgfsys@useobject{currentmarker}{}%
\end{pgfscope}%
\begin{pgfscope}%
\pgfsys@transformshift{3.215760in}{2.474153in}%
\pgfsys@useobject{currentmarker}{}%
\end{pgfscope}%
\begin{pgfscope}%
\pgfsys@transformshift{3.225233in}{2.477016in}%
\pgfsys@useobject{currentmarker}{}%
\end{pgfscope}%
\begin{pgfscope}%
\pgfsys@transformshift{3.234705in}{2.482918in}%
\pgfsys@useobject{currentmarker}{}%
\end{pgfscope}%
\begin{pgfscope}%
\pgfsys@transformshift{3.244178in}{2.487054in}%
\pgfsys@useobject{currentmarker}{}%
\end{pgfscope}%
\begin{pgfscope}%
\pgfsys@transformshift{3.253650in}{2.489623in}%
\pgfsys@useobject{currentmarker}{}%
\end{pgfscope}%
\begin{pgfscope}%
\pgfsys@transformshift{3.263123in}{2.492002in}%
\pgfsys@useobject{currentmarker}{}%
\end{pgfscope}%
\begin{pgfscope}%
\pgfsys@transformshift{3.272595in}{2.491830in}%
\pgfsys@useobject{currentmarker}{}%
\end{pgfscope}%
\begin{pgfscope}%
\pgfsys@transformshift{3.282068in}{2.493328in}%
\pgfsys@useobject{currentmarker}{}%
\end{pgfscope}%
\begin{pgfscope}%
\pgfsys@transformshift{3.291540in}{2.498638in}%
\pgfsys@useobject{currentmarker}{}%
\end{pgfscope}%
\begin{pgfscope}%
\pgfsys@transformshift{3.301013in}{2.500429in}%
\pgfsys@useobject{currentmarker}{}%
\end{pgfscope}%
\begin{pgfscope}%
\pgfsys@transformshift{3.310485in}{2.508774in}%
\pgfsys@useobject{currentmarker}{}%
\end{pgfscope}%
\begin{pgfscope}%
\pgfsys@transformshift{3.319958in}{2.509288in}%
\pgfsys@useobject{currentmarker}{}%
\end{pgfscope}%
\begin{pgfscope}%
\pgfsys@transformshift{3.329430in}{2.513722in}%
\pgfsys@useobject{currentmarker}{}%
\end{pgfscope}%
\begin{pgfscope}%
\pgfsys@transformshift{3.338903in}{2.515704in}%
\pgfsys@useobject{currentmarker}{}%
\end{pgfscope}%
\begin{pgfscope}%
\pgfsys@transformshift{3.348375in}{2.519942in}%
\pgfsys@useobject{currentmarker}{}%
\end{pgfscope}%
\begin{pgfscope}%
\pgfsys@transformshift{3.357848in}{2.525154in}%
\pgfsys@useobject{currentmarker}{}%
\end{pgfscope}%
\begin{pgfscope}%
\pgfsys@transformshift{3.367320in}{2.527925in}%
\pgfsys@useobject{currentmarker}{}%
\end{pgfscope}%
\begin{pgfscope}%
\pgfsys@transformshift{3.376793in}{2.534311in}%
\pgfsys@useobject{currentmarker}{}%
\end{pgfscope}%
\begin{pgfscope}%
\pgfsys@transformshift{3.386265in}{2.530915in}%
\pgfsys@useobject{currentmarker}{}%
\end{pgfscope}%
\begin{pgfscope}%
\pgfsys@transformshift{3.395737in}{2.533093in}%
\pgfsys@useobject{currentmarker}{}%
\end{pgfscope}%
\begin{pgfscope}%
\pgfsys@transformshift{3.405210in}{2.534096in}%
\pgfsys@useobject{currentmarker}{}%
\end{pgfscope}%
\begin{pgfscope}%
\pgfsys@transformshift{3.414682in}{2.542641in}%
\pgfsys@useobject{currentmarker}{}%
\end{pgfscope}%
\begin{pgfscope}%
\pgfsys@transformshift{3.424155in}{2.545309in}%
\pgfsys@useobject{currentmarker}{}%
\end{pgfscope}%
\begin{pgfscope}%
\pgfsys@transformshift{3.433627in}{2.549253in}%
\pgfsys@useobject{currentmarker}{}%
\end{pgfscope}%
\begin{pgfscope}%
\pgfsys@transformshift{3.443100in}{2.548103in}%
\pgfsys@useobject{currentmarker}{}%
\end{pgfscope}%
\begin{pgfscope}%
\pgfsys@transformshift{3.452572in}{2.550090in}%
\pgfsys@useobject{currentmarker}{}%
\end{pgfscope}%
\begin{pgfscope}%
\pgfsys@transformshift{3.462045in}{2.549429in}%
\pgfsys@useobject{currentmarker}{}%
\end{pgfscope}%
\begin{pgfscope}%
\pgfsys@transformshift{3.471517in}{2.556208in}%
\pgfsys@useobject{currentmarker}{}%
\end{pgfscope}%
\begin{pgfscope}%
\pgfsys@transformshift{3.480990in}{2.561523in}%
\pgfsys@useobject{currentmarker}{}%
\end{pgfscope}%
\begin{pgfscope}%
\pgfsys@transformshift{3.490462in}{2.567225in}%
\pgfsys@useobject{currentmarker}{}%
\end{pgfscope}%
\begin{pgfscope}%
\pgfsys@transformshift{3.499935in}{2.562751in}%
\pgfsys@useobject{currentmarker}{}%
\end{pgfscope}%
\begin{pgfscope}%
\pgfsys@transformshift{3.509407in}{2.565712in}%
\pgfsys@useobject{currentmarker}{}%
\end{pgfscope}%
\begin{pgfscope}%
\pgfsys@transformshift{3.518880in}{2.567112in}%
\pgfsys@useobject{currentmarker}{}%
\end{pgfscope}%
\begin{pgfscope}%
\pgfsys@transformshift{3.528352in}{2.571345in}%
\pgfsys@useobject{currentmarker}{}%
\end{pgfscope}%
\begin{pgfscope}%
\pgfsys@transformshift{3.537825in}{2.573328in}%
\pgfsys@useobject{currentmarker}{}%
\end{pgfscope}%
\begin{pgfscope}%
\pgfsys@transformshift{3.547297in}{2.577566in}%
\pgfsys@useobject{currentmarker}{}%
\end{pgfscope}%
\begin{pgfscope}%
\pgfsys@transformshift{3.556770in}{2.582876in}%
\pgfsys@useobject{currentmarker}{}%
\end{pgfscope}%
\begin{pgfscope}%
\pgfsys@transformshift{3.566242in}{2.579382in}%
\pgfsys@useobject{currentmarker}{}%
\end{pgfscope}%
\begin{pgfscope}%
\pgfsys@transformshift{3.575715in}{2.583713in}%
\pgfsys@useobject{currentmarker}{}%
\end{pgfscope}%
\begin{pgfscope}%
\pgfsys@transformshift{3.585187in}{2.586581in}%
\pgfsys@useobject{currentmarker}{}%
\end{pgfscope}%
\begin{pgfscope}%
\pgfsys@transformshift{3.594660in}{2.589738in}%
\pgfsys@useobject{currentmarker}{}%
\end{pgfscope}%
\begin{pgfscope}%
\pgfsys@transformshift{3.604132in}{2.593188in}%
\pgfsys@useobject{currentmarker}{}%
\end{pgfscope}%
\begin{pgfscope}%
\pgfsys@transformshift{3.613605in}{2.593315in}%
\pgfsys@useobject{currentmarker}{}%
\end{pgfscope}%
\begin{pgfscope}%
\pgfsys@transformshift{3.623077in}{2.588935in}%
\pgfsys@useobject{currentmarker}{}%
\end{pgfscope}%
\begin{pgfscope}%
\pgfsys@transformshift{3.632550in}{2.590824in}%
\pgfsys@useobject{currentmarker}{}%
\end{pgfscope}%
\begin{pgfscope}%
\pgfsys@transformshift{3.642022in}{2.598386in}%
\pgfsys@useobject{currentmarker}{}%
\end{pgfscope}%
\begin{pgfscope}%
\pgfsys@transformshift{3.651495in}{2.602526in}%
\pgfsys@useobject{currentmarker}{}%
\end{pgfscope}%
\begin{pgfscope}%
\pgfsys@transformshift{3.660967in}{2.602746in}%
\pgfsys@useobject{currentmarker}{}%
\end{pgfscope}%
\begin{pgfscope}%
\pgfsys@transformshift{3.670440in}{2.606887in}%
\pgfsys@useobject{currentmarker}{}%
\end{pgfscope}%
\begin{pgfscope}%
\pgfsys@transformshift{3.679912in}{2.616112in}%
\pgfsys@useobject{currentmarker}{}%
\end{pgfscope}%
\begin{pgfscope}%
\pgfsys@transformshift{3.689385in}{2.618584in}%
\pgfsys@useobject{currentmarker}{}%
\end{pgfscope}%
\begin{pgfscope}%
\pgfsys@transformshift{3.698857in}{2.626542in}%
\pgfsys@useobject{currentmarker}{}%
\end{pgfscope}%
\begin{pgfscope}%
\pgfsys@transformshift{3.708330in}{2.626958in}%
\pgfsys@useobject{currentmarker}{}%
\end{pgfscope}%
\begin{pgfscope}%
\pgfsys@transformshift{3.717802in}{2.634818in}%
\pgfsys@useobject{currentmarker}{}%
\end{pgfscope}%
\begin{pgfscope}%
\pgfsys@transformshift{3.727275in}{2.638464in}%
\pgfsys@useobject{currentmarker}{}%
\end{pgfscope}%
\begin{pgfscope}%
\pgfsys@transformshift{3.736747in}{2.639276in}%
\pgfsys@useobject{currentmarker}{}%
\end{pgfscope}%
\begin{pgfscope}%
\pgfsys@transformshift{3.746220in}{2.644782in}%
\pgfsys@useobject{currentmarker}{}%
\end{pgfscope}%
\begin{pgfscope}%
\pgfsys@transformshift{3.755692in}{2.648820in}%
\pgfsys@useobject{currentmarker}{}%
\end{pgfscope}%
\begin{pgfscope}%
\pgfsys@transformshift{3.765165in}{2.653254in}%
\pgfsys@useobject{currentmarker}{}%
\end{pgfscope}%
\begin{pgfscope}%
\pgfsys@transformshift{3.774637in}{2.659249in}%
\pgfsys@useobject{currentmarker}{}%
\end{pgfscope}%
\begin{pgfscope}%
\pgfsys@transformshift{3.784110in}{2.665445in}%
\pgfsys@useobject{currentmarker}{}%
\end{pgfscope}%
\begin{pgfscope}%
\pgfsys@transformshift{3.793582in}{2.669483in}%
\pgfsys@useobject{currentmarker}{}%
\end{pgfscope}%
\begin{pgfscope}%
\pgfsys@transformshift{3.803055in}{2.673819in}%
\pgfsys@useobject{currentmarker}{}%
\end{pgfscope}%
\begin{pgfscope}%
\pgfsys@transformshift{3.812527in}{2.674627in}%
\pgfsys@useobject{currentmarker}{}%
\end{pgfscope}%
\begin{pgfscope}%
\pgfsys@transformshift{3.822000in}{2.680622in}%
\pgfsys@useobject{currentmarker}{}%
\end{pgfscope}%
\begin{pgfscope}%
\pgfsys@transformshift{3.831472in}{2.687209in}%
\pgfsys@useobject{currentmarker}{}%
\end{pgfscope}%
\begin{pgfscope}%
\pgfsys@transformshift{3.840945in}{2.688800in}%
\pgfsys@useobject{currentmarker}{}%
\end{pgfscope}%
\begin{pgfscope}%
\pgfsys@transformshift{3.850417in}{2.693332in}%
\pgfsys@useobject{currentmarker}{}%
\end{pgfscope}%
\begin{pgfscope}%
\pgfsys@transformshift{3.859890in}{2.697565in}%
\pgfsys@useobject{currentmarker}{}%
\end{pgfscope}%
\begin{pgfscope}%
\pgfsys@transformshift{3.869362in}{2.697399in}%
\pgfsys@useobject{currentmarker}{}%
\end{pgfscope}%
\begin{pgfscope}%
\pgfsys@transformshift{3.878835in}{2.698696in}%
\pgfsys@useobject{currentmarker}{}%
\end{pgfscope}%
\begin{pgfscope}%
\pgfsys@transformshift{3.888307in}{2.703615in}%
\pgfsys@useobject{currentmarker}{}%
\end{pgfscope}%
\begin{pgfscope}%
\pgfsys@transformshift{3.897780in}{2.711475in}%
\pgfsys@useobject{currentmarker}{}%
\end{pgfscope}%
\begin{pgfscope}%
\pgfsys@transformshift{3.907252in}{2.712282in}%
\pgfsys@useobject{currentmarker}{}%
\end{pgfscope}%
\begin{pgfscope}%
\pgfsys@transformshift{3.916725in}{2.714954in}%
\pgfsys@useobject{currentmarker}{}%
\end{pgfscope}%
\begin{pgfscope}%
\pgfsys@transformshift{3.926197in}{2.717132in}%
\pgfsys@useobject{currentmarker}{}%
\end{pgfscope}%
\begin{pgfscope}%
\pgfsys@transformshift{3.935670in}{2.719511in}%
\pgfsys@useobject{currentmarker}{}%
\end{pgfscope}%
\begin{pgfscope}%
\pgfsys@transformshift{3.945142in}{2.716403in}%
\pgfsys@useobject{currentmarker}{}%
\end{pgfscope}%
\begin{pgfscope}%
\pgfsys@transformshift{3.954615in}{2.718684in}%
\pgfsys@useobject{currentmarker}{}%
\end{pgfscope}%
\begin{pgfscope}%
\pgfsys@transformshift{3.964087in}{2.721742in}%
\pgfsys@useobject{currentmarker}{}%
\end{pgfscope}%
\begin{pgfscope}%
\pgfsys@transformshift{3.973560in}{2.724997in}%
\pgfsys@useobject{currentmarker}{}%
\end{pgfscope}%
\begin{pgfscope}%
\pgfsys@transformshift{3.983032in}{2.728844in}%
\pgfsys@useobject{currentmarker}{}%
\end{pgfscope}%
\begin{pgfscope}%
\pgfsys@transformshift{3.992505in}{2.730337in}%
\pgfsys@useobject{currentmarker}{}%
\end{pgfscope}%
\begin{pgfscope}%
\pgfsys@transformshift{4.001977in}{2.733498in}%
\pgfsys@useobject{currentmarker}{}%
\end{pgfscope}%
\begin{pgfscope}%
\pgfsys@transformshift{4.011450in}{2.730586in}%
\pgfsys@useobject{currentmarker}{}%
\end{pgfscope}%
\begin{pgfscope}%
\pgfsys@transformshift{4.020922in}{2.735314in}%
\pgfsys@useobject{currentmarker}{}%
\end{pgfscope}%
\begin{pgfscope}%
\pgfsys@transformshift{4.030395in}{2.736905in}%
\pgfsys@useobject{currentmarker}{}%
\end{pgfscope}%
\begin{pgfscope}%
\pgfsys@transformshift{4.039867in}{2.733503in}%
\pgfsys@useobject{currentmarker}{}%
\end{pgfscope}%
\begin{pgfscope}%
\pgfsys@transformshift{4.049340in}{2.734609in}%
\pgfsys@useobject{currentmarker}{}%
\end{pgfscope}%
\begin{pgfscope}%
\pgfsys@transformshift{4.058812in}{2.739234in}%
\pgfsys@useobject{currentmarker}{}%
\end{pgfscope}%
\begin{pgfscope}%
\pgfsys@transformshift{4.068285in}{2.745430in}%
\pgfsys@useobject{currentmarker}{}%
\end{pgfscope}%
\begin{pgfscope}%
\pgfsys@transformshift{4.077757in}{2.745846in}%
\pgfsys@useobject{currentmarker}{}%
\end{pgfscope}%
\begin{pgfscope}%
\pgfsys@transformshift{4.087230in}{2.751651in}%
\pgfsys@useobject{currentmarker}{}%
\end{pgfscope}%
\begin{pgfscope}%
\pgfsys@transformshift{4.096702in}{2.757744in}%
\pgfsys@useobject{currentmarker}{}%
\end{pgfscope}%
\begin{pgfscope}%
\pgfsys@transformshift{4.106175in}{2.760411in}%
\pgfsys@useobject{currentmarker}{}%
\end{pgfscope}%
\begin{pgfscope}%
\pgfsys@transformshift{4.115647in}{2.768663in}%
\pgfsys@useobject{currentmarker}{}%
\end{pgfscope}%
\begin{pgfscope}%
\pgfsys@transformshift{4.125120in}{2.774266in}%
\pgfsys@useobject{currentmarker}{}%
\end{pgfscope}%
\begin{pgfscope}%
\pgfsys@transformshift{4.134592in}{2.778798in}%
\pgfsys@useobject{currentmarker}{}%
\end{pgfscope}%
\begin{pgfscope}%
\pgfsys@transformshift{4.144065in}{2.782738in}%
\pgfsys@useobject{currentmarker}{}%
\end{pgfscope}%
\begin{pgfscope}%
\pgfsys@transformshift{4.153537in}{2.780125in}%
\pgfsys@useobject{currentmarker}{}%
\end{pgfscope}%
\begin{pgfscope}%
\pgfsys@transformshift{4.163010in}{2.780345in}%
\pgfsys@useobject{currentmarker}{}%
\end{pgfscope}%
\begin{pgfscope}%
\pgfsys@transformshift{4.172482in}{2.787417in}%
\pgfsys@useobject{currentmarker}{}%
\end{pgfscope}%
\begin{pgfscope}%
\pgfsys@transformshift{4.181955in}{2.789110in}%
\pgfsys@useobject{currentmarker}{}%
\end{pgfscope}%
\begin{pgfscope}%
\pgfsys@transformshift{4.191427in}{2.791093in}%
\pgfsys@useobject{currentmarker}{}%
\end{pgfscope}%
\begin{pgfscope}%
\pgfsys@transformshift{4.200900in}{2.796603in}%
\pgfsys@useobject{currentmarker}{}%
\end{pgfscope}%
\begin{pgfscope}%
\pgfsys@transformshift{4.210372in}{2.795943in}%
\pgfsys@useobject{currentmarker}{}%
\end{pgfscope}%
\begin{pgfscope}%
\pgfsys@transformshift{4.219845in}{2.800279in}%
\pgfsys@useobject{currentmarker}{}%
\end{pgfscope}%
\begin{pgfscope}%
\pgfsys@transformshift{4.229317in}{2.801478in}%
\pgfsys@useobject{currentmarker}{}%
\end{pgfscope}%
\begin{pgfscope}%
\pgfsys@transformshift{4.238790in}{2.806206in}%
\pgfsys@useobject{currentmarker}{}%
\end{pgfscope}%
\begin{pgfscope}%
\pgfsys@transformshift{4.248262in}{2.808188in}%
\pgfsys@useobject{currentmarker}{}%
\end{pgfscope}%
\begin{pgfscope}%
\pgfsys@transformshift{4.257735in}{2.807723in}%
\pgfsys@useobject{currentmarker}{}%
\end{pgfscope}%
\begin{pgfscope}%
\pgfsys@transformshift{4.267207in}{2.807850in}%
\pgfsys@useobject{currentmarker}{}%
\end{pgfscope}%
\begin{pgfscope}%
\pgfsys@transformshift{4.276680in}{2.810322in}%
\pgfsys@useobject{currentmarker}{}%
\end{pgfscope}%
\begin{pgfscope}%
\pgfsys@transformshift{4.286152in}{2.818182in}%
\pgfsys@useobject{currentmarker}{}%
\end{pgfscope}%
\begin{pgfscope}%
\pgfsys@transformshift{4.295625in}{2.812333in}%
\pgfsys@useobject{currentmarker}{}%
\end{pgfscope}%
\begin{pgfscope}%
\pgfsys@transformshift{4.305097in}{2.811677in}%
\pgfsys@useobject{currentmarker}{}%
\end{pgfscope}%
\begin{pgfscope}%
\pgfsys@transformshift{4.314570in}{2.814540in}%
\pgfsys@useobject{currentmarker}{}%
\end{pgfscope}%
\begin{pgfscope}%
\pgfsys@transformshift{4.324042in}{2.817991in}%
\pgfsys@useobject{currentmarker}{}%
\end{pgfscope}%
\begin{pgfscope}%
\pgfsys@transformshift{4.333515in}{2.819097in}%
\pgfsys@useobject{currentmarker}{}%
\end{pgfscope}%
\begin{pgfscope}%
\pgfsys@transformshift{4.342987in}{2.821177in}%
\pgfsys@useobject{currentmarker}{}%
\end{pgfscope}%
\begin{pgfscope}%
\pgfsys@transformshift{4.352460in}{2.826981in}%
\pgfsys@useobject{currentmarker}{}%
\end{pgfscope}%
\begin{pgfscope}%
\pgfsys@transformshift{4.361932in}{2.834151in}%
\pgfsys@useobject{currentmarker}{}%
\end{pgfscope}%
\begin{pgfscope}%
\pgfsys@transformshift{4.371405in}{2.835551in}%
\pgfsys@useobject{currentmarker}{}%
\end{pgfscope}%
\begin{pgfscope}%
\pgfsys@transformshift{4.380877in}{2.839491in}%
\pgfsys@useobject{currentmarker}{}%
\end{pgfscope}%
\begin{pgfscope}%
\pgfsys@transformshift{4.390350in}{2.839907in}%
\pgfsys@useobject{currentmarker}{}%
\end{pgfscope}%
\begin{pgfscope}%
\pgfsys@transformshift{4.399822in}{2.842481in}%
\pgfsys@useobject{currentmarker}{}%
\end{pgfscope}%
\begin{pgfscope}%
\pgfsys@transformshift{4.409295in}{2.847987in}%
\pgfsys@useobject{currentmarker}{}%
\end{pgfscope}%
\begin{pgfscope}%
\pgfsys@transformshift{4.418767in}{2.852617in}%
\pgfsys@useobject{currentmarker}{}%
\end{pgfscope}%
\begin{pgfscope}%
\pgfsys@transformshift{4.428240in}{2.859493in}%
\pgfsys@useobject{currentmarker}{}%
\end{pgfscope}%
\begin{pgfscope}%
\pgfsys@transformshift{4.437712in}{2.866472in}%
\pgfsys@useobject{currentmarker}{}%
\end{pgfscope}%
\begin{pgfscope}%
\pgfsys@transformshift{4.447185in}{2.870118in}%
\pgfsys@useobject{currentmarker}{}%
\end{pgfscope}%
\begin{pgfscope}%
\pgfsys@transformshift{4.456657in}{2.876701in}%
\pgfsys@useobject{currentmarker}{}%
\end{pgfscope}%
\begin{pgfscope}%
\pgfsys@transformshift{4.466130in}{2.879079in}%
\pgfsys@useobject{currentmarker}{}%
\end{pgfscope}%
\begin{pgfscope}%
\pgfsys@transformshift{4.475602in}{2.887130in}%
\pgfsys@useobject{currentmarker}{}%
\end{pgfscope}%
\begin{pgfscope}%
\pgfsys@transformshift{4.485075in}{2.889215in}%
\pgfsys@useobject{currentmarker}{}%
\end{pgfscope}%
\begin{pgfscope}%
\pgfsys@transformshift{4.494547in}{2.893547in}%
\pgfsys@useobject{currentmarker}{}%
\end{pgfscope}%
\begin{pgfscope}%
\pgfsys@transformshift{4.504020in}{2.896415in}%
\pgfsys@useobject{currentmarker}{}%
\end{pgfscope}%
\begin{pgfscope}%
\pgfsys@transformshift{4.513492in}{2.895656in}%
\pgfsys@useobject{currentmarker}{}%
\end{pgfscope}%
\begin{pgfscope}%
\pgfsys@transformshift{4.522965in}{2.899405in}%
\pgfsys@useobject{currentmarker}{}%
\end{pgfscope}%
\begin{pgfscope}%
\pgfsys@transformshift{4.532437in}{2.903149in}%
\pgfsys@useobject{currentmarker}{}%
\end{pgfscope}%
\begin{pgfscope}%
\pgfsys@transformshift{4.541910in}{2.906697in}%
\pgfsys@useobject{currentmarker}{}%
\end{pgfscope}%
\begin{pgfscope}%
\pgfsys@transformshift{4.551382in}{2.906531in}%
\pgfsys@useobject{currentmarker}{}%
\end{pgfscope}%
\begin{pgfscope}%
\pgfsys@transformshift{4.560855in}{2.906555in}%
\pgfsys@useobject{currentmarker}{}%
\end{pgfscope}%
\begin{pgfscope}%
\pgfsys@transformshift{4.570327in}{2.911577in}%
\pgfsys@useobject{currentmarker}{}%
\end{pgfscope}%
\begin{pgfscope}%
\pgfsys@transformshift{4.579800in}{2.915223in}%
\pgfsys@useobject{currentmarker}{}%
\end{pgfscope}%
\begin{pgfscope}%
\pgfsys@transformshift{4.589272in}{2.913490in}%
\pgfsys@useobject{currentmarker}{}%
\end{pgfscope}%
\begin{pgfscope}%
\pgfsys@transformshift{4.598745in}{2.918507in}%
\pgfsys@useobject{currentmarker}{}%
\end{pgfscope}%
\begin{pgfscope}%
\pgfsys@transformshift{4.608217in}{2.917944in}%
\pgfsys@useobject{currentmarker}{}%
\end{pgfscope}%
\begin{pgfscope}%
\pgfsys@transformshift{4.617690in}{2.922574in}%
\pgfsys@useobject{currentmarker}{}%
\end{pgfscope}%
\begin{pgfscope}%
\pgfsys@transformshift{4.627162in}{2.926318in}%
\pgfsys@useobject{currentmarker}{}%
\end{pgfscope}%
\begin{pgfscope}%
\pgfsys@transformshift{4.636635in}{2.931633in}%
\pgfsys@useobject{currentmarker}{}%
\end{pgfscope}%
\begin{pgfscope}%
\pgfsys@transformshift{4.646107in}{2.933125in}%
\pgfsys@useobject{currentmarker}{}%
\end{pgfscope}%
\begin{pgfscope}%
\pgfsys@transformshift{4.655580in}{2.933644in}%
\pgfsys@useobject{currentmarker}{}%
\end{pgfscope}%
\begin{pgfscope}%
\pgfsys@transformshift{4.665052in}{2.932200in}%
\pgfsys@useobject{currentmarker}{}%
\end{pgfscope}%
\begin{pgfscope}%
\pgfsys@transformshift{4.674525in}{2.928995in}%
\pgfsys@useobject{currentmarker}{}%
\end{pgfscope}%
\begin{pgfscope}%
\pgfsys@transformshift{4.683997in}{2.933723in}%
\pgfsys@useobject{currentmarker}{}%
\end{pgfscope}%
\begin{pgfscope}%
\pgfsys@transformshift{4.693470in}{2.940012in}%
\pgfsys@useobject{currentmarker}{}%
\end{pgfscope}%
\begin{pgfscope}%
\pgfsys@transformshift{4.702942in}{2.943075in}%
\pgfsys@useobject{currentmarker}{}%
\end{pgfscope}%
\begin{pgfscope}%
\pgfsys@transformshift{4.712415in}{2.948288in}%
\pgfsys@useobject{currentmarker}{}%
\end{pgfscope}%
\begin{pgfscope}%
\pgfsys@transformshift{4.721887in}{2.950666in}%
\pgfsys@useobject{currentmarker}{}%
\end{pgfscope}%
\begin{pgfscope}%
\pgfsys@transformshift{4.731360in}{2.953138in}%
\pgfsys@useobject{currentmarker}{}%
\end{pgfscope}%
\begin{pgfscope}%
\pgfsys@transformshift{4.740832in}{2.958355in}%
\pgfsys@useobject{currentmarker}{}%
\end{pgfscope}%
\begin{pgfscope}%
\pgfsys@transformshift{4.750305in}{2.961512in}%
\pgfsys@useobject{currentmarker}{}%
\end{pgfscope}%
\begin{pgfscope}%
\pgfsys@transformshift{4.759777in}{2.969073in}%
\pgfsys@useobject{currentmarker}{}%
\end{pgfscope}%
\begin{pgfscope}%
\pgfsys@transformshift{4.769250in}{2.972332in}%
\pgfsys@useobject{currentmarker}{}%
\end{pgfscope}%
\begin{pgfscope}%
\pgfsys@transformshift{4.778722in}{2.973434in}%
\pgfsys@useobject{currentmarker}{}%
\end{pgfscope}%
\begin{pgfscope}%
\pgfsys@transformshift{4.788195in}{2.974638in}%
\pgfsys@useobject{currentmarker}{}%
\end{pgfscope}%
\begin{pgfscope}%
\pgfsys@transformshift{4.797667in}{2.982003in}%
\pgfsys@useobject{currentmarker}{}%
\end{pgfscope}%
\begin{pgfscope}%
\pgfsys@transformshift{4.807140in}{2.987123in}%
\pgfsys@useobject{currentmarker}{}%
\end{pgfscope}%
\begin{pgfscope}%
\pgfsys@transformshift{4.816612in}{2.985777in}%
\pgfsys@useobject{currentmarker}{}%
\end{pgfscope}%
\begin{pgfscope}%
\pgfsys@transformshift{4.826085in}{2.991674in}%
\pgfsys@useobject{currentmarker}{}%
\end{pgfscope}%
\begin{pgfscope}%
\pgfsys@transformshift{4.835557in}{2.997968in}%
\pgfsys@useobject{currentmarker}{}%
\end{pgfscope}%
\begin{pgfscope}%
\pgfsys@transformshift{4.845030in}{3.004257in}%
\pgfsys@useobject{currentmarker}{}%
\end{pgfscope}%
\begin{pgfscope}%
\pgfsys@transformshift{4.854502in}{3.003210in}%
\pgfsys@useobject{currentmarker}{}%
\end{pgfscope}%
\begin{pgfscope}%
\pgfsys@transformshift{4.863975in}{3.006171in}%
\pgfsys@useobject{currentmarker}{}%
\end{pgfscope}%
\begin{pgfscope}%
\pgfsys@transformshift{4.873447in}{3.009724in}%
\pgfsys@useobject{currentmarker}{}%
\end{pgfscope}%
\begin{pgfscope}%
\pgfsys@transformshift{4.882920in}{3.010727in}%
\pgfsys@useobject{currentmarker}{}%
\end{pgfscope}%
\begin{pgfscope}%
\pgfsys@transformshift{4.892392in}{3.015156in}%
\pgfsys@useobject{currentmarker}{}%
\end{pgfscope}%
\begin{pgfscope}%
\pgfsys@transformshift{4.901865in}{3.017535in}%
\pgfsys@useobject{currentmarker}{}%
\end{pgfscope}%
\begin{pgfscope}%
\pgfsys@transformshift{4.911337in}{3.023334in}%
\pgfsys@useobject{currentmarker}{}%
\end{pgfscope}%
\begin{pgfscope}%
\pgfsys@transformshift{4.920810in}{3.023657in}%
\pgfsys@useobject{currentmarker}{}%
\end{pgfscope}%
\begin{pgfscope}%
\pgfsys@transformshift{4.930282in}{3.024465in}%
\pgfsys@useobject{currentmarker}{}%
\end{pgfscope}%
\begin{pgfscope}%
\pgfsys@transformshift{4.939755in}{3.025473in}%
\pgfsys@useobject{currentmarker}{}%
\end{pgfscope}%
\begin{pgfscope}%
\pgfsys@transformshift{4.949227in}{3.029119in}%
\pgfsys@useobject{currentmarker}{}%
\end{pgfscope}%
\begin{pgfscope}%
\pgfsys@transformshift{4.958700in}{3.031199in}%
\pgfsys@useobject{currentmarker}{}%
\end{pgfscope}%
\begin{pgfscope}%
\pgfsys@transformshift{4.968172in}{3.032501in}%
\pgfsys@useobject{currentmarker}{}%
\end{pgfscope}%
\begin{pgfscope}%
\pgfsys@transformshift{4.977645in}{3.033407in}%
\pgfsys@useobject{currentmarker}{}%
\end{pgfscope}%
\begin{pgfscope}%
\pgfsys@transformshift{4.987117in}{3.041169in}%
\pgfsys@useobject{currentmarker}{}%
\end{pgfscope}%
\begin{pgfscope}%
\pgfsys@transformshift{4.996590in}{3.042172in}%
\pgfsys@useobject{currentmarker}{}%
\end{pgfscope}%
\begin{pgfscope}%
\pgfsys@transformshift{5.006062in}{3.043767in}%
\pgfsys@useobject{currentmarker}{}%
\end{pgfscope}%
\begin{pgfscope}%
\pgfsys@transformshift{5.015535in}{3.047511in}%
\pgfsys@useobject{currentmarker}{}%
\end{pgfscope}%
\begin{pgfscope}%
\pgfsys@transformshift{5.025007in}{3.050184in}%
\pgfsys@useobject{currentmarker}{}%
\end{pgfscope}%
\begin{pgfscope}%
\pgfsys@transformshift{5.034480in}{3.056571in}%
\pgfsys@useobject{currentmarker}{}%
\end{pgfscope}%
\begin{pgfscope}%
\pgfsys@transformshift{5.043952in}{3.059434in}%
\pgfsys@useobject{currentmarker}{}%
\end{pgfscope}%
\begin{pgfscope}%
\pgfsys@transformshift{5.053425in}{3.061127in}%
\pgfsys@useobject{currentmarker}{}%
\end{pgfscope}%
\begin{pgfscope}%
\pgfsys@transformshift{5.062897in}{3.064479in}%
\pgfsys@useobject{currentmarker}{}%
\end{pgfscope}%
\begin{pgfscope}%
\pgfsys@transformshift{5.072370in}{3.067250in}%
\pgfsys@useobject{currentmarker}{}%
\end{pgfscope}%
\begin{pgfscope}%
\pgfsys@transformshift{5.081842in}{3.071483in}%
\pgfsys@useobject{currentmarker}{}%
\end{pgfscope}%
\begin{pgfscope}%
\pgfsys@transformshift{5.091315in}{3.077092in}%
\pgfsys@useobject{currentmarker}{}%
\end{pgfscope}%
\begin{pgfscope}%
\pgfsys@transformshift{5.100787in}{3.079955in}%
\pgfsys@useobject{currentmarker}{}%
\end{pgfscope}%
\begin{pgfscope}%
\pgfsys@transformshift{5.110260in}{3.082720in}%
\pgfsys@useobject{currentmarker}{}%
\end{pgfscope}%
\begin{pgfscope}%
\pgfsys@transformshift{5.119732in}{3.087350in}%
\pgfsys@useobject{currentmarker}{}%
\end{pgfscope}%
\begin{pgfscope}%
\pgfsys@transformshift{5.129205in}{3.095303in}%
\pgfsys@useobject{currentmarker}{}%
\end{pgfscope}%
\begin{pgfscope}%
\pgfsys@transformshift{5.138677in}{3.102184in}%
\pgfsys@useobject{currentmarker}{}%
\end{pgfscope}%
\begin{pgfscope}%
\pgfsys@transformshift{5.148149in}{3.105439in}%
\pgfsys@useobject{currentmarker}{}%
\end{pgfscope}%
\begin{pgfscope}%
\pgfsys@transformshift{5.157622in}{3.107621in}%
\pgfsys@useobject{currentmarker}{}%
\end{pgfscope}%
\begin{pgfscope}%
\pgfsys@transformshift{5.167094in}{3.108429in}%
\pgfsys@useobject{currentmarker}{}%
\end{pgfscope}%
\begin{pgfscope}%
\pgfsys@transformshift{5.176567in}{3.105321in}%
\pgfsys@useobject{currentmarker}{}%
\end{pgfscope}%
\begin{pgfscope}%
\pgfsys@transformshift{5.186039in}{3.110636in}%
\pgfsys@useobject{currentmarker}{}%
\end{pgfscope}%
\begin{pgfscope}%
\pgfsys@transformshift{5.195512in}{3.111150in}%
\pgfsys@useobject{currentmarker}{}%
\end{pgfscope}%
\begin{pgfscope}%
\pgfsys@transformshift{5.204984in}{3.113431in}%
\pgfsys@useobject{currentmarker}{}%
\end{pgfscope}%
\begin{pgfscope}%
\pgfsys@transformshift{5.214457in}{3.116979in}%
\pgfsys@useobject{currentmarker}{}%
\end{pgfscope}%
\begin{pgfscope}%
\pgfsys@transformshift{5.223929in}{3.114561in}%
\pgfsys@useobject{currentmarker}{}%
\end{pgfscope}%
\begin{pgfscope}%
\pgfsys@transformshift{5.233402in}{3.115075in}%
\pgfsys@useobject{currentmarker}{}%
\end{pgfscope}%
\begin{pgfscope}%
\pgfsys@transformshift{5.242874in}{3.110401in}%
\pgfsys@useobject{currentmarker}{}%
\end{pgfscope}%
\begin{pgfscope}%
\pgfsys@transformshift{5.252347in}{3.111312in}%
\pgfsys@useobject{currentmarker}{}%
\end{pgfscope}%
\begin{pgfscope}%
\pgfsys@transformshift{5.261819in}{3.117013in}%
\pgfsys@useobject{currentmarker}{}%
\end{pgfscope}%
\begin{pgfscope}%
\pgfsys@transformshift{5.271292in}{3.119196in}%
\pgfsys@useobject{currentmarker}{}%
\end{pgfscope}%
\begin{pgfscope}%
\pgfsys@transformshift{5.280764in}{3.124212in}%
\pgfsys@useobject{currentmarker}{}%
\end{pgfscope}%
\begin{pgfscope}%
\pgfsys@transformshift{5.290237in}{3.130996in}%
\pgfsys@useobject{currentmarker}{}%
\end{pgfscope}%
\begin{pgfscope}%
\pgfsys@transformshift{5.299709in}{3.132978in}%
\pgfsys@useobject{currentmarker}{}%
\end{pgfscope}%
\begin{pgfscope}%
\pgfsys@transformshift{5.309182in}{3.135650in}%
\pgfsys@useobject{currentmarker}{}%
\end{pgfscope}%
\begin{pgfscope}%
\pgfsys@transformshift{5.318654in}{3.139003in}%
\pgfsys@useobject{currentmarker}{}%
\end{pgfscope}%
\begin{pgfscope}%
\pgfsys@transformshift{5.328127in}{3.141474in}%
\pgfsys@useobject{currentmarker}{}%
\end{pgfscope}%
\begin{pgfscope}%
\pgfsys@transformshift{5.337599in}{3.143168in}%
\pgfsys@useobject{currentmarker}{}%
\end{pgfscope}%
\begin{pgfscope}%
\pgfsys@transformshift{5.347072in}{3.148184in}%
\pgfsys@useobject{currentmarker}{}%
\end{pgfscope}%
\begin{pgfscope}%
\pgfsys@transformshift{5.356544in}{3.155555in}%
\pgfsys@useobject{currentmarker}{}%
\end{pgfscope}%
\begin{pgfscope}%
\pgfsys@transformshift{5.366017in}{3.161256in}%
\pgfsys@useobject{currentmarker}{}%
\end{pgfscope}%
\begin{pgfscope}%
\pgfsys@transformshift{5.375489in}{3.164712in}%
\pgfsys@useobject{currentmarker}{}%
\end{pgfscope}%
\begin{pgfscope}%
\pgfsys@transformshift{5.384962in}{3.170903in}%
\pgfsys@useobject{currentmarker}{}%
\end{pgfscope}%
\begin{pgfscope}%
\pgfsys@transformshift{5.394434in}{3.171514in}%
\pgfsys@useobject{currentmarker}{}%
\end{pgfscope}%
\begin{pgfscope}%
\pgfsys@transformshift{5.403907in}{3.169390in}%
\pgfsys@useobject{currentmarker}{}%
\end{pgfscope}%
\begin{pgfscope}%
\pgfsys@transformshift{5.413379in}{3.173428in}%
\pgfsys@useobject{currentmarker}{}%
\end{pgfscope}%
\begin{pgfscope}%
\pgfsys@transformshift{5.422852in}{3.174632in}%
\pgfsys@useobject{currentmarker}{}%
\end{pgfscope}%
\begin{pgfscope}%
\pgfsys@transformshift{5.432324in}{3.175048in}%
\pgfsys@useobject{currentmarker}{}%
\end{pgfscope}%
\begin{pgfscope}%
\pgfsys@transformshift{5.441797in}{3.174196in}%
\pgfsys@useobject{currentmarker}{}%
\end{pgfscope}%
\begin{pgfscope}%
\pgfsys@transformshift{5.451269in}{3.175493in}%
\pgfsys@useobject{currentmarker}{}%
\end{pgfscope}%
\begin{pgfscope}%
\pgfsys@transformshift{5.460742in}{3.174539in}%
\pgfsys@useobject{currentmarker}{}%
\end{pgfscope}%
\begin{pgfscope}%
\pgfsys@transformshift{5.470214in}{3.177113in}%
\pgfsys@useobject{currentmarker}{}%
\end{pgfscope}%
\begin{pgfscope}%
\pgfsys@transformshift{5.479687in}{3.184577in}%
\pgfsys@useobject{currentmarker}{}%
\end{pgfscope}%
\begin{pgfscope}%
\pgfsys@transformshift{5.489159in}{3.187738in}%
\pgfsys@useobject{currentmarker}{}%
\end{pgfscope}%
\end{pgfscope}%
\begin{pgfscope}%
\pgfsetrectcap%
\pgfsetmiterjoin%
\pgfsetlinewidth{0.803000pt}%
\definecolor{currentstroke}{rgb}{0.000000,0.000000,0.000000}%
\pgfsetstrokecolor{currentstroke}%
\pgfsetdash{}{0pt}%
\pgfpathmoveto{\pgfqpoint{0.762383in}{0.471179in}}%
\pgfpathlineto{\pgfqpoint{0.762383in}{3.317098in}}%
\pgfusepath{stroke}%
\end{pgfscope}%
\begin{pgfscope}%
\pgfsetrectcap%
\pgfsetmiterjoin%
\pgfsetlinewidth{0.803000pt}%
\definecolor{currentstroke}{rgb}{0.000000,0.000000,0.000000}%
\pgfsetstrokecolor{currentstroke}%
\pgfsetdash{}{0pt}%
\pgfpathmoveto{\pgfqpoint{5.489159in}{0.471179in}}%
\pgfpathlineto{\pgfqpoint{5.489159in}{3.317098in}}%
\pgfusepath{stroke}%
\end{pgfscope}%
\begin{pgfscope}%
\pgfsetrectcap%
\pgfsetmiterjoin%
\pgfsetlinewidth{0.803000pt}%
\definecolor{currentstroke}{rgb}{0.000000,0.000000,0.000000}%
\pgfsetstrokecolor{currentstroke}%
\pgfsetdash{}{0pt}%
\pgfpathmoveto{\pgfqpoint{0.762383in}{0.471179in}}%
\pgfpathlineto{\pgfqpoint{5.489159in}{0.471179in}}%
\pgfusepath{stroke}%
\end{pgfscope}%
\begin{pgfscope}%
\pgfsetrectcap%
\pgfsetmiterjoin%
\pgfsetlinewidth{0.803000pt}%
\definecolor{currentstroke}{rgb}{0.000000,0.000000,0.000000}%
\pgfsetstrokecolor{currentstroke}%
\pgfsetdash{}{0pt}%
\pgfpathmoveto{\pgfqpoint{0.762383in}{3.317098in}}%
\pgfpathlineto{\pgfqpoint{5.489159in}{3.317098in}}%
\pgfusepath{stroke}%
\end{pgfscope}%
\begin{pgfscope}%
\pgfsetbuttcap%
\pgfsetmiterjoin%
\definecolor{currentfill}{rgb}{1.000000,1.000000,1.000000}%
\pgfsetfillcolor{currentfill}%
\pgfsetfillopacity{0.800000}%
\pgfsetlinewidth{1.003750pt}%
\definecolor{currentstroke}{rgb}{0.800000,0.800000,0.800000}%
\pgfsetstrokecolor{currentstroke}%
\pgfsetstrokeopacity{0.800000}%
\pgfsetdash{}{0pt}%
\pgfpathmoveto{\pgfqpoint{0.840161in}{2.453544in}}%
\pgfpathlineto{\pgfqpoint{1.789553in}{2.453544in}}%
\pgfpathquadraticcurveto{\pgfqpoint{1.811775in}{2.453544in}}{\pgfqpoint{1.811775in}{2.475767in}}%
\pgfpathlineto{\pgfqpoint{1.811775in}{3.239321in}}%
\pgfpathquadraticcurveto{\pgfqpoint{1.811775in}{3.261543in}}{\pgfqpoint{1.789553in}{3.261543in}}%
\pgfpathlineto{\pgfqpoint{0.840161in}{3.261543in}}%
\pgfpathquadraticcurveto{\pgfqpoint{0.817939in}{3.261543in}}{\pgfqpoint{0.817939in}{3.239321in}}%
\pgfpathlineto{\pgfqpoint{0.817939in}{2.475767in}}%
\pgfpathquadraticcurveto{\pgfqpoint{0.817939in}{2.453544in}}{\pgfqpoint{0.840161in}{2.453544in}}%
\pgfpathclose%
\pgfusepath{stroke,fill}%
\end{pgfscope}%
\begin{pgfscope}%
\pgfsetrectcap%
\pgfsetroundjoin%
\pgfsetlinewidth{1.505625pt}%
\definecolor{currentstroke}{rgb}{0.121569,0.466667,0.705882}%
\pgfsetstrokecolor{currentstroke}%
\pgfsetdash{}{0pt}%
\pgfpathmoveto{\pgfqpoint{0.862383in}{3.178210in}}%
\pgfpathlineto{\pgfqpoint{1.084605in}{3.178210in}}%
\pgfusepath{stroke}%
\end{pgfscope}%
\begin{pgfscope}%
\pgftext[x=1.173494in,y=3.139321in,left,base]{\rmfamily\fontsize{8.000000}{9.600000}\selectfont OGI}%
\end{pgfscope}%
\begin{pgfscope}%
\pgfsetbuttcap%
\pgfsetroundjoin%
\pgfsetlinewidth{1.505625pt}%
\definecolor{currentstroke}{rgb}{1.000000,0.498039,0.054902}%
\pgfsetstrokecolor{currentstroke}%
\pgfsetdash{{5.550000pt}{2.400000pt}}{0.000000pt}%
\pgfpathmoveto{\pgfqpoint{0.862383in}{3.023277in}}%
\pgfpathlineto{\pgfqpoint{1.084605in}{3.023277in}}%
\pgfusepath{stroke}%
\end{pgfscope}%
\begin{pgfscope}%
\pgftext[x=1.173494in,y=2.984388in,left,base]{\rmfamily\fontsize{8.000000}{9.600000}\selectfont IDS}%
\end{pgfscope}%
\begin{pgfscope}%
\pgfsetbuttcap%
\pgfsetroundjoin%
\pgfsetlinewidth{1.505625pt}%
\definecolor{currentstroke}{rgb}{0.172549,0.627451,0.172549}%
\pgfsetstrokecolor{currentstroke}%
\pgfsetdash{{9.600000pt}{2.400000pt}{1.500000pt}{2.400000pt}}{0.000000pt}%
\pgfpathmoveto{\pgfqpoint{0.862383in}{2.868343in}}%
\pgfpathlineto{\pgfqpoint{1.084605in}{2.868343in}}%
\pgfusepath{stroke}%
\end{pgfscope}%
\begin{pgfscope}%
\pgftext[x=1.173494in,y=2.829455in,left,base]{\rmfamily\fontsize{8.000000}{9.600000}\selectfont Thompson}%
\end{pgfscope}%
\begin{pgfscope}%
\pgfsetbuttcap%
\pgfsetroundjoin%
\pgfsetlinewidth{1.505625pt}%
\definecolor{currentstroke}{rgb}{0.839216,0.152941,0.156863}%
\pgfsetstrokecolor{currentstroke}%
\pgfsetdash{{1.500000pt}{2.475000pt}}{0.000000pt}%
\pgfpathmoveto{\pgfqpoint{0.862383in}{2.713410in}}%
\pgfpathlineto{\pgfqpoint{1.084605in}{2.713410in}}%
\pgfusepath{stroke}%
\end{pgfscope}%
\begin{pgfscope}%
\pgftext[x=1.173494in,y=2.674522in,left,base]{\rmfamily\fontsize{8.000000}{9.600000}\selectfont Bayes UCB}%
\end{pgfscope}%
\begin{pgfscope}%
\pgfsetrectcap%
\pgfsetroundjoin%
\pgfsetlinewidth{1.505625pt}%
\definecolor{currentstroke}{rgb}{0.580392,0.403922,0.741176}%
\pgfsetstrokecolor{currentstroke}%
\pgfsetdash{}{0pt}%
\pgfpathmoveto{\pgfqpoint{0.862383in}{2.558477in}}%
\pgfpathlineto{\pgfqpoint{1.084605in}{2.558477in}}%
\pgfusepath{stroke}%
\end{pgfscope}%
\begin{pgfscope}%
\pgfsetbuttcap%
\pgfsetroundjoin%
\definecolor{currentfill}{rgb}{0.580392,0.403922,0.741176}%
\pgfsetfillcolor{currentfill}%
\pgfsetlinewidth{1.003750pt}%
\definecolor{currentstroke}{rgb}{0.580392,0.403922,0.741176}%
\pgfsetstrokecolor{currentstroke}%
\pgfsetdash{}{0pt}%
\pgfsys@defobject{currentmarker}{\pgfqpoint{-0.020833in}{-0.020833in}}{\pgfqpoint{0.020833in}{0.020833in}}{%
\pgfpathmoveto{\pgfqpoint{0.000000in}{-0.020833in}}%
\pgfpathcurveto{\pgfqpoint{0.005525in}{-0.020833in}}{\pgfqpoint{0.010825in}{-0.018638in}}{\pgfqpoint{0.014731in}{-0.014731in}}%
\pgfpathcurveto{\pgfqpoint{0.018638in}{-0.010825in}}{\pgfqpoint{0.020833in}{-0.005525in}}{\pgfqpoint{0.020833in}{0.000000in}}%
\pgfpathcurveto{\pgfqpoint{0.020833in}{0.005525in}}{\pgfqpoint{0.018638in}{0.010825in}}{\pgfqpoint{0.014731in}{0.014731in}}%
\pgfpathcurveto{\pgfqpoint{0.010825in}{0.018638in}}{\pgfqpoint{0.005525in}{0.020833in}}{\pgfqpoint{0.000000in}{0.020833in}}%
\pgfpathcurveto{\pgfqpoint{-0.005525in}{0.020833in}}{\pgfqpoint{-0.010825in}{0.018638in}}{\pgfqpoint{-0.014731in}{0.014731in}}%
\pgfpathcurveto{\pgfqpoint{-0.018638in}{0.010825in}}{\pgfqpoint{-0.020833in}{0.005525in}}{\pgfqpoint{-0.020833in}{0.000000in}}%
\pgfpathcurveto{\pgfqpoint{-0.020833in}{-0.005525in}}{\pgfqpoint{-0.018638in}{-0.010825in}}{\pgfqpoint{-0.014731in}{-0.014731in}}%
\pgfpathcurveto{\pgfqpoint{-0.010825in}{-0.018638in}}{\pgfqpoint{-0.005525in}{-0.020833in}}{\pgfqpoint{0.000000in}{-0.020833in}}%
\pgfpathclose%
\pgfusepath{stroke,fill}%
}%
\begin{pgfscope}%
\pgfsys@transformshift{0.973494in}{2.558477in}%
\pgfsys@useobject{currentmarker}{}%
\end{pgfscope}%
\end{pgfscope}%
\begin{pgfscope}%
\pgftext[x=1.173494in,y=2.519588in,left,base]{\rmfamily\fontsize{8.000000}{9.600000}\selectfont KL-UCB}%
\end{pgfscope}%
\end{pgfpicture}%
\makeatother%
\endgroup%

	\caption{Frequentist regret. The OGI policy is configured with $K=1$ and $\alpha=100$.}
	\label{fig:kaufmann_regret}
\end{figure}
{
\subsection{Additional benchmark algorithms}
In this section, we simulate a few additional algorithms to understand the importance of the varying discount factor, and to try out a different approximation of the Gittins index.
We also simulate the greedy policy to see the inherent value of exploration in our benchmark problems.
Specifically, the algorithms we experiment with are:
\begin{itemize}
	\item OGI with a one-step lookahead and a fixed discount factor of $\gamma$, which we will refer to throughout as ``FOGI($1/(1-\gamma)$)". The quantity $1/(1-\gamma)$ can be interpreted as a rough horizon over which this policy is optimal.
	\item OGI in which the Gittins index approximation equals the closed-form expression given in \cite{brezzi2002optimal}. We will refer to this policy as ``BL-OGI".
	\item The greedy policy, which plays the arm in $\argmax_i \E_{y_{i, N_i(t-1)}}[X_{i,t}]$. Effectively it is equivalent to FOGI($1$), and completely disregards the value of future exploration. We will simply call this policy ``Greedy" in our tables and plots.
\end{itemize}
Recall that the Gittins index policy is optimal for a geometrically distributed horizon with mean $T$. Since FOGI$(T)$ is precisely an approximation for that policy, we would expect it to perform well in our experiments when the horizon is $T$ (even though really it ought to be geometrically distributed).

We reuse the two main experimental setups from Section~\ref{sec:experiments}: the Gaussian bandit with 10 independent arms, and the Bernoulli equivalent. Notice from Table~\ref{table:gaussian_experiment2}, in the Gaussian setup, that there is value in knowing the true horizon $T$ because FOGI($T$) is the dominant policy\footnote{Knowing the horizon $T$, in the context of this paper, should be viwed as a form of ``cheating" since we are interested in anytime policies.}.
We also see that either over or under-estimating the horizon leads to worse performance as demonstrated by the regret from FOGI($T/10$), FOGI($10T$) and Greedy. Interestingly, we also see that the BL-OGI shows larger regret than OGI (see Table~\ref{table:gaussian_experiment1}), suggesting that there is perhaps value in using our optimistic approximation for this particular problem. The comparison against FOGI, BL-OGI and Greedy in the Bernoulli case, presented in Table~\ref{table:bernoulli_experiment2}, tells a similar story as in Table~\ref{table:gaussian_experiment2}.

\begin{table}[h!]
	\centering
	{
	\begin{tabular}{lrrrrrr}
		\toprule
		{} &  \textbf{BL-OGI} &  \textbf{Greedy} &  \textbf{OGI(1)} &  \textbf{FOGI($T$)} &  \textbf{FOGI($T/10$)} &  \textbf{FOGI($10T$)} \\
		\midrule
		Mean  &   58.54 &  167.16 &  49.19 &    49.61 &         60.72 &        59.09 \\
		Standard error   &    2.14 &  9.74 & 1.61  &       1.90 &          4.25 &         1.47 \\
		25\%   &   45.83 &  102.75 &  17.49 &   39.28 &         34.85 &        50.94 \\
		50\%   &   56.87 &  156.63 &  41.72 &    47.29 &         52.19 &        57.84 \\
		75\%   &   67.67 &  216.77 & 73.24 &     60.04 &         87.63 &        67.59 \\
		\bottomrule
	\end{tabular}
	\caption[Table caption text]{Comparison against some of OGI's simpler variants in the  Gaussian setup.}
	\label{table:gaussian_experiment2}
	}
\end{table}
}
	
\begin{table}[h!]
	\centering
	{
		\begin{tabular}{lrrrrrr}
			\toprule
			\textbf{Algorithm} &  \textbf{BL-OGI} &  \textbf{Greedy} & \textbf{OGI(1)} & \textbf{FOGI($T$)} &  \textbf{FOGI($T/10$)} &  \textbf{FOGI($10T$)} \\
			\midrule
			Mean  &   23.50 &   56.32 & 18.12 &    16.52 &         18.69 &        20.14 \\
			Standard error  &    0.63  &  2.36 &  0.65 &       0.62 &          0.82 &         0.58 \\
			25\%   &   18.74 &   37.61 & 6.26 &    12.36 &         12.70 &        16.36 \\
			50\%   &   22.50 &   55.16 &  15.08 &    15.55 &         17.14 &        19.43 \\
			75\%   &   28.18 &   74.64 &  27.63  &    19.72 &         23.29 &        23.62 \\
			\bottomrule
		\end{tabular}
		\caption[Table caption text]{Comparison against some of OGI's simpler variants in the  Bernoulli setup.}
		\label{table:bernoulli_experiment2}
	}
\end{table}

\subsection{Additional tables for Section~\ref{sec:experiments}}
\begin{table}
	\centering
	\begin{tabular}{rrrrrr} 
		\toprule
		{}    $\alpha$ &   $\beta$ &  OGI(1) &  OGI(3) &  OGI(5) &  Gittins \\
		\midrule
		   1 & 1 &   0.760 &   0.721 &   0.712 &    0.703 \\
		   1 & 2 &   0.571 &   0.522 &   0.511 &    0.500 \\
		   1 & 3 &   0.452 &   0.401 &   0.389 &    0.380 \\
		   1 & 4 &   0.374 &   0.321 &   0.312 &    0.302 \\
		  2 & 1 &   0.853 &   0.818 &   0.809 &    0.800 \\
		  2 & 2 &   0.702 &   0.657 &   0.646 &    0.635 \\
		  2 & 3 &   0.591 &   0.543 &   0.530 &    0.516 \\
		  2 & 4 &   0.508 &   0.458 &   0.445 &    0.434 \\
		  3 & 1 &   0.893 &   0.864 &   0.855 &    0.845 \\
		  3 & 2 &   0.771 &   0.729 &   0.719 &    0.707 \\
		  3 & 3 &   0.671 &   0.626 &   0.613 &    0.601 \\
		  3 & 4 &   0.592 &   0.545 &   0.532 &    0.518 \\
		 4 & 1 &   0.916 &   0.890 &   0.882 &    0.872 \\
		  4 & 2 &   0.813 &   0.776 &   0.765 &    0.754 \\
		  4 & 3 &   0.724 &   0.682 &   0.670 &    0.658 \\
		  4 & 4 &   0.651 &   0.607 &   0.593 &    0.581 \\
		\bottomrule
	\end{tabular}
	\caption{Optimistic and exact Gittins Indices when $\gamma = 0.9$ for different Beta-Bernoulli parameters}
	\label{table:ogi_table_for_gamma_9}
\end{table}

\begin{table}
	\centering
	\begin{tabular}{rrrrrr}
		\toprule
		{}    $\alpha$ &   $\beta$ &  OGI(1) &  OGI(3) &  OGI(5) &  Gittins \\
		\midrule
		 1.0 & 1.0 &   0.817 &   0.784 &   0.774 &    0.761 \\
		 1.0 & 2.0 &   0.637 &   0.590 &   0.577 &    0.560 \\
		 1.0 & 3.0 &   0.514 &   0.463 &   0.449 &    0.433 \\
		 1.0 & 4.0 &   0.430 &   0.376 &   0.364 &    0.348 \\
		 2.0 & 1.0 &   0.890 &   0.860 &   0.851 &    0.838 \\
		 2.0 & 2.0 &   0.752 &   0.710 &   0.698 &    0.681 \\
		 2.0 & 3.0 &   0.643 &   0.596 &   0.581 &    0.562 \\
		2.0 & 4.0 &   0.558 &   0.509 &   0.494 &    0.475 \\
		 3.0 & 1.0 &   0.921 &   0.896 &   0.887 &    0.874 \\
		3.0 & 2.0 &   0.811 &   0.773 &   0.762 &    0.744 \\
		 3.0 & 3.0 &   0.715 &   0.672 &   0.658 &    0.639 \\
		 3.0 & 4.0 &   0.637 &   0.591 &   0.575 &    0.556 \\
		4.0 & 1.0 &   0.938 &   0.916 &   0.908 &    0.895 \\
		4.0 & 2.0 &   0.847 &   0.812 &   0.801 &    0.784 \\
		4.0 & 3.0 &   0.763 &   0.722 &   0.709 &    0.690 \\
		4.0 & 4.0 &   0.691 &   0.648 &   0.633 &    0.613 \\
		\bottomrule
	\end{tabular}
	\caption{Optimistic and exact Gittins Indices when $\gamma = 0.95$ for different Beta-Bernoulli parameters}
	\label{table:ogi_table_for_gamma_95}
\end{table}


% CASE 2: BiBTeX used to generate mypaper.bbl (to be further fine tuned)
%\input{mypaper.bbl} % outcomment this line in Case 2

%If you don't use BiBTex, you can manually itemize references as shown below.




%%%%%%%%%%%%%%%%%
\end{document}
%%%%%%%%%%%%%%%%%
